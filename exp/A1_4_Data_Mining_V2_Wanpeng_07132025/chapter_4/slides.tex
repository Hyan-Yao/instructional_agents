\documentclass[aspectratio=169]{beamer}

% Theme and Color Setup
\usetheme{Madrid}
\usecolortheme{whale}
\useinnertheme{rectangles}
\useoutertheme{miniframes}

% Additional Packages
\usepackage[utf8]{inputenc}
\usepackage[T1]{fontenc}
\usepackage{graphicx}
\usepackage{booktabs}
\usepackage{listings}
\usepackage{amsmath}
\usepackage{amssymb}
\usepackage{xcolor}
\usepackage{tikz}
\usepackage{pgfplots}
\pgfplotsset{compat=1.18}
\usetikzlibrary{positioning}
\usepackage{hyperref}

% Custom Colors
\definecolor{myblue}{RGB}{31, 73, 125}
\definecolor{mygray}{RGB}{100, 100, 100}
\definecolor{mygreen}{RGB}{0, 128, 0}
\definecolor{myorange}{RGB}{230, 126, 34}
\definecolor{mycodebackground}{RGB}{245, 245, 245}

% Set Theme Colors
\setbeamercolor{structure}{fg=myblue}
\setbeamercolor{frametitle}{fg=white, bg=myblue}
\setbeamercolor{title}{fg=myblue}
\setbeamercolor{section in toc}{fg=myblue}
\setbeamercolor{item projected}{fg=white, bg=myblue}
\setbeamercolor{block title}{bg=myblue!20, fg=myblue}
\setbeamercolor{block body}{bg=myblue!10}
\setbeamercolor{alerted text}{fg=myorange}

% Set Fonts
\setbeamerfont{title}{size=\Large, series=\bfseries}
\setbeamerfont{frametitle}{size=\large, series=\bfseries}
\setbeamerfont{caption}{size=\small}
\setbeamerfont{footnote}{size=\tiny}

% Document Start
\begin{document}

\frame{\titlepage}

\begin{frame}[fragile]
  \title{Introduction to Supervised Learning}
  \author{John Smith, Ph.D.}
  \date{\today}
  \titlepage
\end{frame}

\begin{frame}[fragile]
  \frametitle{What is Supervised Learning?}
  \begin{block}{Definition}
    Supervised learning is a fundamental approach in data mining and machine learning where a model is trained using a labeled dataset. Each instance in the training set consists of input features and the corresponding output label, which the model learns to map.
  \end{block}
  \begin{block}{Goal}
    The goal is to learn a function that can accurately predict unknown labels from unseen data.
  \end{block}
\end{frame}

\begin{frame}[fragile]
  \frametitle{Significance of Supervised Learning}
  \begin{enumerate}
    \item \textbf{Decision-Making and Predictions:}
      \begin{itemize}
        \item Used for predicting outcomes such as customer churn and loan defaults.
      \end{itemize}
    
    \item \textbf{Data-Driven Insights:}
      \begin{itemize}
        \item Helps organizations identify patterns and trends from data.
      \end{itemize}

    \item \textbf{Automation of Processes:}
      \begin{itemize}
        \item Automates repetitive tasks like image recognition and spam detection.
      \end{itemize}

    \item \textbf{Adaptability and Efficiency:}
      \begin{itemize}
        \item Models can quickly adapt to new data, improving prediction accuracy.
      \end{itemize}
  \end{enumerate}
\end{frame}

\begin{frame}[fragile]
  \frametitle{Recent Applications in AI}
  \begin{itemize}
    \item \textbf{Natural Language Processing (NLP):}
      \begin{itemize}
        \item Tools like ChatGPT utilize supervised learning algorithms to understand and generate human-like text.
      \end{itemize}

    \item \textbf{Healthcare:}
      \begin{itemize}
        \item Assists in diagnosing diseases by analyzing patient data.
      \end{itemize}

    \item \textbf{Finance:}
      \begin{itemize}
        \item Used in credit scoring systems to assess loan applicants' risk.
      \end{itemize}
  \end{itemize}
\end{frame}

\begin{frame}[fragile]
  \frametitle{Key Points and Summary}
  \begin{itemize}
    \item Supervised Learning involves training on labeled data to predict outcomes.
    \item Essential for creating automated and data-driven decision-making systems.
    \item Foundational to many AI applications, including NLP and finance.
  \end{itemize}

  \begin{block}{Summary}
    Supervised learning is a vital aspect of data mining, empowering businesses and technologies to function efficiently across various sectors, promoting improved decision-making and operational excellence.
  \end{block}
\end{frame}

\begin{frame}[fragile]
    \frametitle{What is Supervised Learning?}
    \begin{block}{Definition and Explanation}
        Supervised learning is a branch of machine learning where the model is trained on a labeled dataset, meaning each training example is paired with an output label. 
    \end{block}
    \begin{itemize}
        \item **Key Components**:
        \begin{itemize}
            \item **Input Features**: Attributes or independent variables (e.g., age, income)
            \item **Output Labels**: Target labels or dependent variables (e.g., purchase decisions)
        \end{itemize}
    \end{itemize}
\end{frame}

\begin{frame}[fragile]
    \frametitle{Distinction from Unsupervised Learning}
    \begin{block}{Supervised Learning}
        \begin{itemize}
            \item Involves labeled datasets
            \item Examples include classification and regression tasks
            \item Objective: Learn to predict outcomes from known inputs
        \end{itemize}
    \end{block}
    \begin{block}{Unsupervised Learning}
        \begin{itemize}
            \item Involves unlabeled datasets
            \item Examples include clustering and association tasks
            \item Objective: Discover patterns or groupings in data
        \end{itemize}
    \end{block}
\end{frame}

\begin{frame}[fragile]
    \frametitle{Applications and Importance of Data Mining}
    \begin{block}{Real-World Applications}
        \begin{itemize}
            \item Email Classification: Spam vs. Non-spam
            \item Credit Scoring: Predicting loan default
            \item Image Recognition: Identifying objects or faces
        \end{itemize}
    \end{block}
    \begin{block}{Why Do We Need Data Mining?}
        Data mining allows organizations to extract meaningful insights from data, enhancing decision-making and improving business performance.
    \end{block}
    \begin{itemize}
        \item Companies analyze customer behavior to refine marketing strategies.
        \item Recent AI applications, such as ChatGPT, benefit significantly from supervised learning techniques.
    \end{itemize}
\end{frame}

\begin{frame}[fragile]
    \frametitle{Types of Supervised Learning - Overview}
    \begin{block}{Supervised Learning}
        Supervised learning is a type of machine learning where models are trained using labeled data. Each instance in the training dataset is associated with a known output, facilitating the model’s learning process. 
    \end{block}
    
    \begin{itemize}
        \item Goal: Make accurate predictions on new, unseen data based on this training.
        \item Key Types: 
            \begin{itemize}
                \item Classification
                \item Regression
            \end{itemize}
    \end{itemize}
\end{frame}

\begin{frame}[fragile]
    \frametitle{Types of Supervised Learning - Classification}
    \begin{block}{Classification}
        \begin{itemize}
            \item Definition: Predicting a categorical label for a given input. The output variable is a discrete label.
            \item Examples:
                \begin{itemize}
                    \item Email Spam Detection (spam or not spam)
                    \item Image Recognition (labels like cat, dog, or car)
                \end{itemize}
        \end{itemize}        
    \end{block}
    
    \begin{itemize}
        \item Key Points:
            \begin{itemize}
                \item Output Types: Binary or multiclass
                \item Common Algorithms: Logistic Regression, Decision Trees, SVMs, Neural Networks
            \end{itemize}
        \item Evaluation Metrics:
            \begin{itemize}
                \item Accuracy: Proportion of true results among total cases
                \item Precision/Recall: Relevance in classification tasks
            \end{itemize}
    \end{itemize}
\end{frame}

\begin{frame}[fragile]
    \frametitle{Types of Supervised Learning - Regression}
    \begin{block}{Regression}
        \begin{itemize}
            \item Definition: Predicting a continuous output value based on input features. The output variable is numerical.
            \item Examples:
                \begin{itemize}
                    \item Housing Price Prediction (based on size, location, etc.)
                    \item Stock Price Forecasting (using historical data)
                \end{itemize}
        \end{itemize}        
    \end{block}
    
    \begin{itemize}
        \item Key Points:
            \begin{itemize}
                \item Output Types: Continuous values
                \item Common Algorithms: Linear Regression, Polynomial Regression, Decision Trees, Neural Networks
            \end{itemize}
        \item Evaluation Metrics:
            \begin{itemize}
                \item Mean Absolute Error (MAE): Average of absolute errors
                \item Root Mean Squared Error (RMSE): Emphasizes larger errors
            \end{itemize}
    \end{itemize}
\end{frame}

\begin{frame}[fragile]
    \frametitle{Conclusion: Supervised Learning Types}
    \begin{block}{Summary}
        \begin{itemize}
            \item Supervised learning is crucial for predicting outcomes based on labeled data.
            \item Classification: Used for discrete outputs.
            \item Regression: Used for continuous outputs.
            \item Understanding these categories helps in selecting the right algorithms and metrics.
        \end{itemize}
    \end{block}
    
    \begin{block}{Conclusion}
        Identifying whether a problem is classification or regression is foundational in supervised learning. This choice influences both data preprocessing and model selection, ultimately determining prediction effectiveness.
    \end{block}
\end{frame}

\begin{frame}[fragile]
    \frametitle{Introduction to Logistic Regression}
    \begin{block}{What is Logistic Regression?}
        Logistic Regression is a statistical method used for binary classification problems.
    \end{block}
    \begin{itemize}
        \item Predicts the outcome of a dependent variable with two possible values (0 and 1).
        \item Unlike linear regression, it is designed for categorical outcomes.
        \item Applications include spam detection, disease classification, and customer behavior prediction.
    \end{itemize}
\end{frame}

\begin{frame}[fragile]
    \frametitle{Mathematical Foundation}
    Logistic regression uses the logistic function to model probabilities. The key equation is:
    
    \begin{equation}
        P(Y=1|X) = \frac{1}{1 + e^{-(\beta_0 + \beta_1X_1 + \beta_2X_2 + ... + \beta_nX_n)}}
    \end{equation}
    
    Where:
    \begin{itemize}
        \item \(P(Y=1|X)\): probability that outcome \(Y\) is 1 given predictors \(X\).
        \item \(\beta_0\): intercept, \(\beta_1, \beta_2, \ldots, \beta_n\): coefficients of independent variables.
        \item \(e\): base of natural logarithm.
    \end{itemize}
    
    \begin{block}{Interpretation}
        Output ranges between 0 and 1, with a threshold to determine binary outcome.
    \end{block}
\end{frame}

\begin{frame}[fragile]
    \frametitle{When to Use Logistic Regression?}
    Use logistic regression when:
    \begin{itemize}
        \item The dependent variable is binary.
        \item You seek probabilities of an event occurring (e.g., customer churn).
        \item Predictors can be continuous or categorical (e.g., demographics).
    \end{itemize}

    \begin{block}{Examples of Applications}
        \begin{itemize}
            \item Classifying medical patients as 'disease' or 'no disease'.
            \item Predicting customer purchasing behavior.
            \item Assessing risk of loan defaults based on financial data.
        \end{itemize}
    \end{block}
\end{frame}

\begin{frame}[fragile]
    \frametitle{Key Points to Remember}
    \begin{enumerate}
        \item Suitable for predicting one of two classes.
        \item Converts linear predictions into probabilities using the logistic function.
        \item A positive coefficient for a predictor increases the likelihood of the outcome.
        \item Logistic regression is widely used in various fields like healthcare and finance.
    \end{enumerate}
\end{frame}

\begin{frame}[fragile]
    \frametitle{Logistic Regression Example - Introduction}
    \begin{block}{Introduction}
        Logistic Regression is a powerful statistical method used for classification problems where the outcome is binary (e.g., yes/no, success/failure). 
        This example illustrates its application using a real-world dataset—the famous Titanic passenger data, which aims to predict survival based on various features.
    \end{block}
\end{frame}

\begin{frame}[fragile]
    \frametitle{Logistic Regression Example - The Dataset}
    \begin{block}{The Dataset}
        \begin{itemize}
            \item \textbf{Source}: The Titanic dataset is available through Kaggle and contains information on passengers aboard the Titanic who survived or did not survive the tragic sinking in 1912.
            \item \textbf{Features}:
            \begin{enumerate}
                \item Pclass: Passenger class (1st, 2nd, or 3rd)
                \item Sex: Gender of the passenger
                \item Age: Age of the passenger
                \item SibSp: Number of siblings/spouses aboard
                \item Parch: Number of parents/children aboard
                \item Fare: Ticket fare
                \item Survived: 0 = No, 1 = Yes (Target variable)
            \end{enumerate}
        \end{itemize}
    \end{block}
\end{frame}

\begin{frame}[fragile]
    \frametitle{Logistic Regression Example - Objective and Process}
    \begin{block}{Objective}
        Predict whether a passenger survived (1) or did not survive (0) based on the above features using Logistic Regression.
    \end{block}

    \begin{block}{Process}
        \begin{enumerate}
            \item \textbf{Data Preparation}:
            \begin{itemize}
                \item Handling Missing Values: Fill missing ages with the median age.
                \item Encoding Categorical Variables: Convert 'Sex' into binary (0 for female, 1 for male) and use one-hot encoding for 'Pclass'.
            \end{itemize}

            \item \textbf{Model Development}:
            \begin{itemize}
                \item Split the dataset into a training set (80\%) and a test set (20\%).
                \item Implement Logistic Regression using a programming language like Python.
                \begin{lstlisting}[language=Python]
from sklearn.model_selection import train_test_split
from sklearn.linear_model import LogisticRegression
from sklearn.preprocessing import StandardScaler
import pandas as pd

# Load dataset
df = pd.read_csv('titanic.csv')  # Assuming the Titanic dataset is in CSV format
# Data preprocessing goes here...

X = df[['Pclass', 'Sex', 'Age', 'SibSp', 'Parch', 'Fare']]  # Feature set
y = df['Survived']  # Target variable

# Convert 'Sex' to binary
X['Sex'] = X['Sex'].map({'female': 0, 'male': 1})

# Split the dataset
X_train, X_test, y_train, y_test = train_test_split(X, y, test_size=0.2, random_state=42)

# Standardize the features
scaler = StandardScaler()
X_train = scaler.fit_transform(X_train)
X_test = scaler.transform(X_test)

# Train the Logistic Regression model
model = LogisticRegression()
model.fit(X_train, y_train)

# Predict on the test set
predictions = model.predict(X_test)
                \end{lstlisting}
            \end{itemize}

            \item \textbf{Model Evaluation}:
            \begin{itemize}
                \item Evaluate the trained Logistic Regression model using metrics such as accuracy, confusion matrix, and ROC curve.
                \item More details on evaluation metrics will be covered in the next slide.
            \end{itemize}
        \end{enumerate}
    \end{block}
\end{frame}

\begin{frame}[fragile]
    \frametitle{Logistic Regression Example - Key Points and Conclusion}
    \begin{block}{Key Points to Emphasize}
        \begin{itemize}
            \item \textbf{Interpretability}: The coefficients of the logistic regression model can be interpreted in terms of odds ratios, providing insights into how changes in features affect the odds of survival.
            \item \textbf{Use Cases}: Beyond the Titanic, logistic regression is widely used in various fields such as finance (credit scoring), healthcare (disease prediction), and marketing (conversion prediction).
            \item \textbf{Limitations}: Logistic regression assumes a linear relationship between the features and the log-odds of the outcome. It doesn't perform well with non-linear data unless combined with polynomial terms or interactions.
        \end{itemize}
    \end{block}

    \begin{block}{Conclusion}
        Logistic Regression is a crucial tool in the data scientist's toolkit for binary classification problems. Its application to real-world datasets such as the Titanic exemplifies its effectiveness in drawing meaningful conclusions from data.
    \end{block}

    \begin{block}{Next Steps}
        For further exploration, see the next slide on evaluation metrics for comprehensive model assessment!
    \end{block}
\end{frame}

\begin{frame}[fragile]
    \frametitle{Evaluation Metrics for Logistic Regression}
    Understanding the Importance of Evaluation Metrics:
    \begin{itemize}
        \item In supervised learning, especially logistic regression, evaluating model performance is crucial.
        \item Logistic regression is commonly used for binary classification (0s and 1s).
        \item Various metrics help assess model quality.
    \end{itemize}
\end{frame}

\begin{frame}[fragile]
    \frametitle{Key Evaluation Metrics - Part 1}
    \begin{enumerate}
        \item \textbf{Accuracy}
        \begin{itemize}
            \item \textbf{Definition}: Proportion of true results among total cases.
            \item \textbf{Formula}:
            \begin{equation}
            \text{Accuracy} = \frac{TP + TN}{TP + TN + FP + FN}
            \end{equation}
            \item \textbf{Example}: 80 correct predictions out of 100 gives an accuracy of 80\%.
        \end{itemize}
        
        \item \textbf{Precision}
        \begin{itemize}
            \item \textbf{Definition}: Proportion of true positives out of predicted positives.
            \item \textbf{Formula}:
            \begin{equation}
            \text{Precision} = \frac{TP}{TP + FP}
            \end{equation}
            \item \textbf{Example}: 15 true positives out of 20 predicted positives gives a precision of 75\%.
        \end{itemize}        
    \end{enumerate}
\end{frame}

\begin{frame}[fragile]
    \frametitle{Key Evaluation Metrics - Part 2}
    \begin{enumerate}
        \setcounter{enumi}{2} % Continue the numbering
        \item \textbf{Recall (Sensitivity)}
        \begin{itemize}
            \item \textbf{Definition}: Proportion of true positives out of actual positives.
            \item \textbf{Formula}:
            \begin{equation}
            \text{Recall} = \frac{TP}{TP + FN}
            \end{equation}
            \item \textbf{Example}: 20 out of 30 actual positives gives a recall of approximately 67\%.
        \end{itemize}
        
        \item \textbf{F1-Score}
        \begin{itemize}
            \item \textbf{Definition}: Harmonic mean of precision and recall, balancing both metrics.
            \item \textbf{Formula}:
            \begin{equation}
            \text{F1-Score} = 2 \times \frac{\text{Precision} \times \text{Recall}}{\text{Precision} + \text{Recall}}
            \end{equation}
            \item \textbf{Example}: Precision of 0.75 and recall of 0.67 gives an F1-score of approximately 0.71.
        \end{itemize}
    \end{enumerate}
\end{frame}

\begin{frame}[fragile]
    \frametitle{Key Points & Conclusion}
    \begin{itemize}
        \item \textbf{Model Assessment}: Different metrics provide varying insights; no single metric suffices.
        \item \textbf{Trade-offs}: A model with high recall may compromise on precision.
        \item \textbf{Choice of Metric}: Depends on application; in fraud detection, for instance, recall may be prioritized.
    \end{itemize}
    
    \vspace{1em}
    \textbf{Conclusion}:
    Understanding evaluation metrics is crucial for optimizing logistic regression model performance, allowing data scientists to make informed adjustments to meet specific needs.
\end{frame}

\begin{frame}[fragile]
    \frametitle{Introduction to Decision Trees}
    \begin{block}{What are Decision Trees?}
        Decision Trees are a popular supervised learning algorithm used for classification and regression tasks. 
        They model decisions and their possible consequences in a tree-like structure, making them easy to understand and visualize.
    \end{block}
\end{frame}

\begin{frame}[fragile]
    \frametitle{Structure of a Decision Tree}
    \begin{enumerate}
        \item \textbf{Root Node:} Represents the entire dataset and is the starting point of the decision tree. It splits into branches based on a decision criterion.
        \item \textbf{Internal Nodes:} Represent the features of the data. Each node tests a feature and decides how to split the data into subsets.
        \item \textbf{Branches:} The outcome of a decision at each node. They lead to other nodes or final outcomes.
        \item \textbf{Leaf Nodes:} The end points of the tree; they represent the final output (class label for classification or a continuous value for regression).
    \end{enumerate}
\end{frame}

\begin{frame}[fragile]
    \frametitle{How Do Decision Trees Work?}
    \begin{enumerate}
        \item \textbf{Data Splitting:} The algorithm selects the feature that best splits the data into homogeneous subsets, typically measured by Gini impurity, entropy (for classification), or mean squared error (for regression).
        \item \textbf{Recursive Partitioning:} The process of splitting continues recursively until a stopping criterion is met (e.g., maximum depth, minimum samples in a node, or maximum leaf nodes reached).
        \item \textbf{Making Predictions:} To classify a new sample, it starts at the root and follows the branches according to feature values until reaching a leaf node that provides the predicted class or value.
    \end{enumerate}
\end{frame}

\begin{frame}[fragile]
    \frametitle{Example of a Decision Tree}
    \begin{block}{Simple Dataset Example}
        Consider a dataset about whether individuals like to play a sport based on two features: Age (young, adult, senior) and Income (low, medium, high). The decision tree might look like this:
        \begin{itemize}
            \item \textbf{Root:} Age?
            \begin{itemize}
                \item Young: Leaf Node - 'No Sports'
                \item Adult: Income?
                \begin{itemize}
                    \item Low: Leaf Node - 'No Sports'
                    \item High: Leaf Node - 'Sports'
                \end{itemize}
                \item Senior: Leaf Node - 'No Sports'
            \end{itemize}
        \end{itemize}
    \end{block}
\end{frame}

\begin{frame}[fragile]
    \frametitle{Key Points and Applications}
    \begin{block}{Key Points}
        \begin{itemize}
            \item \textbf{Interpretability:} Easy to understand and interpret.
            \item \textbf{Versatility:} Suitable for both classification and regression.
            \item \textbf{Overfitting:} Can easily overfit training data; techniques like pruning are used to mitigate this.
        \end{itemize}
    \end{block}
    \begin{block}{Applications in AI and Data Mining}
        Used in various domains including:
        \begin{itemize}
            \item Healthcare (disease diagnosis)
            \item Finance (credit scoring)
            \item Marketing (customer segmentation)
        \end{itemize}
        Recent advancements in AI, such as tools like ChatGPT, utilize decision trees to enhance predictions based on user input.
    \end{block}
\end{frame}

\begin{frame}[fragile]
    \frametitle{Summary Outline}
    \begin{itemize}
        \item Definition and importance of Decision Trees in supervised learning.
        \item Structure: Root Node, Internal Nodes, Branches, Leaf Nodes.
        \item Working Mechanism: Data Splitting, Recursive Partitioning, Making Predictions.
        \item Example visualization of a Decision Tree.
        \item Key points: Interpretability, Versatility, Overfitting.
        \item AI Applications: Real-world usage and importance in modern data mining.
    \end{itemize}
\end{frame}

\begin{frame}[fragile]
    \frametitle{Building Decision Trees - Introduction}
    \begin{block}{Introduction to Decision Trees}
        Decision trees are a powerful model in supervised learning, used for both classification and regression tasks.
        Understanding the construction process of decision trees is essential for practitioners of data mining and machine learning.
    \end{block}
\end{frame}

\begin{frame}[fragile]
    \frametitle{Building Decision Trees - Step-by-Step Process}
    \begin{enumerate}
        \item \textbf{Select the Best Feature to Split the Data}
            \begin{itemize}
                \item \textbf{Definition:} The feature should maximize information gain or minimize impurity in resultant nodes.
                \item \textbf{Criteria:} 
                    \begin{itemize}
                        \item \textbf{Gini Impurity:}
                        \[
                        Gini(D) = 1 - \sum_{i=1}^{C} p_i^2
                        \]
                        \item \textbf{Information Gain:}
                        \[
                        IG = H(D) - H(D|feature)
                        \]
                    \end{itemize}
            \end{itemize}
            
        \item \textbf{Create Branches for Each Feature Value}
        \item \textbf{Split the Dataset into Subsets}
        \item \textbf{Repeat Until Stopping Criteria are Met}
            \begin{itemize}
                \item Stopping criteria: maximum depth, minimum samples, or no further information gain.
            \end{itemize}
        \item \textbf{Label the Leaf Nodes}
    \end{enumerate}
\end{frame}

\begin{frame}[fragile]
    \frametitle{Building Decision Trees - Example and Applications}
    \begin{block}{Example of Building a Decision Tree}
        Consider a dataset on weather and whether to play tennis:
        
        \begin{tabular}{|c|c|c|c|c|}
            \hline
            Weather & Temperature & Humidity & Windy & Play \\
            \hline
            Sunny   & Hot         & High     & False & No   \\
            Sunny   & Hot         & High     & True  & No   \\
            Overcast& Hot         & High     & False & Yes  \\
            Rainy   & Mild        & High     & False & Yes  \\
            \hline
        \end{tabular}
        
        Steps to build the tree:
        \begin{enumerate}
            \item Calculate Gini impurity or Information Gain for features.
            \item Create branches for the feature with the highest gain.
            \item Split data and repeat until stopping criteria are met.
        \end{enumerate}
    \end{block}
    
    \begin{block}{Applications in AI}
        Decision trees are foundational in AI, aiding in:
        \begin{itemize}
            \item Classifying user inputs
            \item Making recommendations based on behavior
        \end{itemize}
    \end{block}
\end{frame}

\begin{frame}[fragile]
    \frametitle{Pros and Cons of Decision Trees}
    \begin{block}{Overview of Decision Trees}
        Decision trees are a popular supervised learning algorithm used for classification and regression tasks. They model decisions and their possible consequences in a tree-like graph, making them easy to interpret and visualize.
    \end{block}
\end{frame}

\begin{frame}[fragile]
    \frametitle{Advantages of Decision Trees}
    \begin{enumerate}
        \item \textbf{Simplicity and Interpretability}:
        \begin{itemize}
            \item Decision trees are intuitive and easy to visualize.
            \item \textit{Example}: A tree for classifying emails as spam or not can show criteria like keywords.
        \end{itemize}

        \item \textbf{No Need for Data Normalization}:
        \begin{itemize}
            \item Minimal preprocessing, such as normalization or scaling, is needed.
            \item \textit{Example}: Raw categorical and continuous data can be used directly.
        \end{itemize}

        \item \textbf{Handles Numerical and Categorical Data}:
        \begin{itemize}
            \item Decision trees can process diverse data types without complex transformations.
            \item \textit{Example}: A tree can classify numerical attributes (income) with categorical attributes (city).
        \end{itemize}

        \item \textbf{Non-Parametric Nature}:
        \begin{itemize}
            \item No assumption of a specific data distribution enhances versatility.
            \item \textit{Example}: They perform well with non-normally distributed data.
        \end{itemize}

        \item \textbf{Feature Importance}:
        \begin{itemize}
            \item Insights into feature relevance aid in selection and data understanding, especially in high-dimensional datasets.
        \end{itemize}
    \end{enumerate}
\end{frame}

\begin{frame}[fragile]
    \frametitle{Disadvantages of Decision Trees}
    \begin{enumerate}
        \item \textbf{Overfitting}:
        \begin{itemize}
            \item Prone to overfitting, especially with deep trees.
            \item \textit{Example}: A tree that classifies training data perfectly may not generalize well.
        \end{itemize}

        \item \textbf{Instability}:
        \begin{itemize}
            \item Small data changes can lead to different tree structures, reducing reliability.
        \end{itemize}
        
        \item \textbf{Bias Towards Predominant Classes}:
        \begin{itemize}
            \item Tendency to favor larger classes in imbalanced datasets, resulting in misleading accuracy.
        \end{itemize}
        
        \item \textbf{Limited Expressiveness}:
        \begin{itemize}
            \item Struggles with linear relationships and interaction effects, potentially leading to underfitting.
        \end{itemize}
        
        \item \textbf{Costly to Evaluate}:
        \begin{itemize}
            \item Evaluating large datasets can be computationally expensive due to recursive splits.
            \item \textit{Example}: Training time increases significantly with more features and records.
        \end{itemize}
    \end{enumerate}
\end{frame}

\begin{frame}[fragile]
    \frametitle{Conclusion and Key Takeaways}
    \begin{block}{Conclusion}
        Weighing the strengths and weaknesses of decision trees is essential. Interpretability and ease of use are significant advantages, but potential overfitting and instability must be considered.
    \end{block}

    \begin{block}{Key Points to Remember}
        \begin{itemize}
            \item Decision trees are favored for their simplicity and adaptability to various data types.
            \item They can become complex and overfit; strategies like pruning are useful.
            \item Understanding pros and cons aids effective model selection in supervised learning tasks.
        \end{itemize}
    \end{block}
\end{frame}

\begin{frame}[fragile]
    \frametitle{Introduction to Random Forests}
    \begin{block}{Overview}
        Random Forests is an ensemble learning algorithm that enhances the capabilities of decision trees by combining multiple trees to improve predictions.
    \end{block}
    \begin{itemize}
        \item Reduces overfitting by promoting diversity among trees.
        \item Improves accuracy and provides insights into feature importance.
    \end{itemize}
\end{frame}

\begin{frame}[fragile]
    \frametitle{Why Use Random Forests?}
    \begin{itemize}
        \item \textbf{Improved Accuracy:} Averages multiple models for enhanced performance.
        \item \textbf{Reduced Overfitting:} Generalized models from diverse tree structures.
        \item \textbf{Feature Importance:} Insights into influential features for predictions.
    \end{itemize}
    \begin{block}{Outline}
        \begin{enumerate}
            \item What Are Random Forests?
            \item Why Use Random Forests?
        \end{enumerate}
    \end{block}
\end{frame}

\begin{frame}[fragile]
    \frametitle{How Random Forests Work}
    \begin{enumerate}
        \item \textbf{Bootstrap Sampling:} Create subsets of the dataset using sampling with replacement.
        \item \textbf{Building Decision Trees:} Each tree uses a random subset of features at each split, ensuring variety.
        \item \textbf{Aggregation:}
            \begin{itemize}
                \item For regression: Average of predictions.
                \item For classification: Mode of predictions.
            \end{itemize}
    \end{enumerate}
\end{frame}

\begin{frame}[fragile]
    \frametitle{Key Concepts and Example Application}
    \begin{block}{Key Concepts}
        \begin{itemize}
            \item \textbf{Ensemble Learning:} Combining models for improved performance.
            \item \textbf{Diversity of Models:} Each tree learns uniquely, resulting in robust predictions.
        \end{itemize}
    \end{block}
    \begin{block}{Example Application}
        In a medical diagnosis scenario, a random forest could predict disease presence by analyzing various patient features.
    \end{block}
\end{frame}

\begin{frame}[fragile]
    \frametitle{Formula for Random Forest Predictions}
    For a classification problem, denoting:
    \begin{itemize}
        \item $T_1, T_2, ..., T_N$ as the $N$ trees,
        \item $x$ as the input feature vector.
    \end{itemize}
    The final prediction $P(x)$ can be represented as:
    \begin{equation}
        P(x) = \text{mode}(T_1(x), T_2(x), \ldots, T_N(x))
    \end{equation}
\end{frame}

\begin{frame}[fragile]
    \frametitle{Conclusion}
    Random forests leverage multiple decision trees for more accurate and reliable predictions. Their effectiveness in handling large datasets and uncovering feature importance reinforces their role in supervised learning.
    \begin{block}{Outline}
        \begin{enumerate}
            \item Key Concepts
            \item Example Application
            \item Prediction Formula
            \item Conclusion
        \end{enumerate}
    \end{block}
\end{frame}

\begin{frame}[fragile]
    \frametitle{Working of Random Forests - Introduction}
    \begin{block}{Overview}
        Random Forests is an ensemble learning technique primarily used for classification and regression tasks.
        By combining the predictions of multiple decision trees, it enhances model accuracy and robustness.
    \end{block}
    \begin{block}{Key Concepts}
        Random Forests leverage two critical concepts:
        \begin{itemize}
            \item \textbf{Bagging} - Bootstrap aggregating technique.
            \item \textbf{Feature Randomness} - Random selection of features for splitting nodes.
        \end{itemize}
    \end{block}
\end{frame}

\begin{frame}[fragile]
    \frametitle{Working of Random Forests - Bagging}
    \begin{block}{Bagging (Bootstrap Aggregating)}
        \begin{itemize}
            \item \textbf{Definition}: Creating multiple subsets of training data by sampling with replacement to build individual decision trees.
            \item \textbf{Process}:
            \begin{itemize}
                \item Create 'n' random samples (subsets) through bootstrapping from the original dataset.
                \item Each subset is of the same size as the original dataset; duplicates and exclusions may occur.
            \end{itemize}
            \item \textbf{Purpose}: Reduces overfitting and improves generalization by averaging predictions from numerous trees.
            \item \textbf{Example}: For a dataset of 1000 records, multiple subsets (1000 records each with possible repetitions) are created.
        \end{itemize}
    \end{block}
\end{frame}

\begin{frame}[fragile]
    \frametitle{Working of Random Forests - Feature Randomness}
    \begin{block}{Feature Randomness}
        \begin{itemize}
            \item \textbf{Definition}: Randomly selecting a subset of features at each node for the best split.
            \item \textbf{Process}:
            \begin{itemize}
                \item Instead of all features, select 'm' features where \( m < M \) (total features).
            \end{itemize}
            \item \textbf{Purpose}: Promotes tree diversity and reduces correlations among individual trees.
            \item \textbf{Example}: In a dataset with 10 features, a decision tree may use 3 random features for each split.
        \end{itemize}
    \end{block}

    \begin{block}{Final Prediction}
        \begin{itemize}
            \item \textbf{For Classification}: Majority vote among trees for class labels.
            \item \textbf{For Regression}: Average of all tree predictions for numeric values.
        \end{itemize}
    \end{block}
\end{frame}

\begin{frame}[fragile]
    \frametitle{Working of Random Forests - Key Points & Conclusion}
    \begin{block}{Key Points}
        \begin{itemize}
            \item \textbf{Ensemble Learning}: Combines multiple models for robust prediction.
            \item \textbf{Reduces Overfitting}: Minimizes risk through bagging and randomness.
            \item \textbf{Versatility}: Effective for both continuous and categorical outcomes.
        \end{itemize}
    \end{block}

    \begin{block}{Conclusion}
        Understanding how bagging and feature randomness work helps appreciate Random Forests' effectiveness in predictive performance, achieving balance between bias and variance for reliable predictions.
    \end{block}
\end{frame}

\begin{frame}[fragile]
    \frametitle{Applications of Random Forests - Introduction}
    \begin{block}{Overview}
        Random forests are a versatile and powerful supervised learning technique utilized across various domains. 
        Their strengths include:
    \end{block}
    \begin{itemize}
        \item Handling large datasets with high dimensionality
        \item Robustness against overfitting
        \item Versatile applicability in real-world scenarios
    \end{itemize}
\end{frame}

\begin{frame}[fragile]
    \frametitle{Applications of Random Forests - Key Applications}
    \begin{enumerate}
        \item \textbf{Healthcare}
        \begin{itemize}
            \item Disease Prediction: Predicts diseases like diabetes and cancer using patient data.
            \item Genomic Classification: Classifies gene expression data to assess cancer risk.
        \end{itemize}
        
        \item \textbf{Finance}
        \begin{itemize}
            \item Credit Scoring: Assesses applicants' creditworthiness using financial history.
            \item Fraud Detection: Identifies transaction anomalies suggesting fraud.
        \end{itemize}
    \end{enumerate}
\end{frame}

\begin{frame}[fragile]
    \frametitle{Applications of Random Forests - Continued}
    \begin{enumerate}
        \setcounter{enumi}{2} % Continue numbering from the last frame
        \item \textbf{Marketing}
        \begin{itemize}
            \item Customer Segmentation: Segments customers for personalized marketing.
            \item Churn Prediction: Analyzes data to predict customer retention.
        \end{itemize}

        \item \textbf{Environmental Science}
        \begin{itemize}
            \item Ecological Modeling: Predicts species distribution based on environmental factors.
            \item Remote Sensing: Classifies land cover from satellite imagery.
        \end{itemize}

        \item \textbf{Natural Language Processing (NLP)}
        \begin{itemize}
            \item Text Classification: Classifies text documents and feedback to gauge sentiment.
        \end{itemize}
    \end{enumerate}
\end{frame}

\begin{frame}[fragile]
    \frametitle{Advantages of Random Forests}
    \begin{itemize}
        \item \textbf{Robustness:} Less sensitive to noise and overfitting compared to decision trees.
        \item \textbf{Versatility:} Applicable to both classification and regression tasks.
        \item \textbf{Feature Importance:} Provides insights into the importance of features in predictions.
    \end{itemize}
\end{frame}

\begin{frame}[fragile]
    \frametitle{Conclusion and Key Points}
    \begin{block}{Key Points}
        \begin{itemize}
            \item Random forests use ensemble learning, improving predictive accuracy.
            \item Applications span diverse fields, making them a preferred choice among machine learning practitioners.
            \item Understanding their strengths and contexts is crucial for success.
        \end{itemize}
    \end{block}
\end{frame}

\begin{frame}[fragile]
    \frametitle{Next Up: Comparison of Learning Algorithms}
    \begin{block}{Overview}
        Prepare to explore how random forests compare with other algorithms 
        such as logistic regression and traditional decision trees, focusing on their unique advantages and use cases.
    \end{block}
\end{frame}

\begin{frame}[fragile]
    \frametitle{Comparison of Learning Algorithms - Introduction}
    \begin{block}{Overview}
        In this presentation, we will compare three popular supervised learning algorithms: 
        \textbf{Logistic Regression}, \textbf{Decision Trees}, and \textbf{Random Forests}. 
        Understanding the strengths and weaknesses of each algorithm allows us to choose the right tool for our data analysis tasks.
    \end{block}
    \begin{itemize}
        \item Key algorithms in supervised learning
        \item Importance of understanding their characteristics
    \end{itemize}
\end{frame}

\begin{frame}[fragile]
    \frametitle{Comparison of Learning Algorithms - Logistic Regression}
    \begin{block}{Explanation}
        \begin{itemize}
            \item \textbf{Logistic Regression} is a probabilistic model used for binary classification. 
            It estimates the probability of an event occurring based on one or more independent variables.
            \item The logistic function transforms outputs to a probability range of 0 to 1:
            \begin{equation}
            P(Y=1|X) = \frac{1}{1 + e^{-(\beta_0 + \beta_1X_1 + \beta_2X_2 + ... + \beta_kX_k)}}
            \end{equation}
        \end{itemize}
    \end{block}
    \begin{block}{Key Points}
        \begin{itemize}
            \item \textbf{Use Cases:} Credit scoring, customer churn prediction 
            \item \textbf{Pros:} Simple, interpretable coefficients, efficient for large datasets 
            \item \textbf{Cons:} Assumes linear relationship, not suitable for non-linear boundaries
        \end{itemize}
    \end{block}
\end{frame}

\begin{frame}[fragile]
    \frametitle{Comparison of Learning Algorithms - Decision Trees and Random Forests}
    \begin{block}{Decision Trees}
        \begin{itemize}
            \item \textbf{Decision Trees} are flowchart-like structures representing tests on features.
            \item The algorithm creates splits based on maximizing information gain or minimizing impurity (e.g., Gini impurity or entropy).
        \end{itemize}
        \begin{itemize}
            \item \textbf{Use Cases:} Customer segmentation, fraud detection
            \item \textbf{Pros:} Intuitive, easy to visualize, handles both categorical and numerical data
            \item \textbf{Cons:} Prone to overfitting, sensitive to noisy data
        \end{itemize}
    \end{block}

    \begin{block}{Random Forests}
        \begin{itemize}
            \item \textbf{Random Forests} constructs multiple decision trees during training and outputs the mode of the classes (classification) or mean prediction (regression).
            \item It improves accuracy and controls overfitting by combining predictions of many trees. 
        \end{itemize}
        \begin{itemize}
            \item \textbf{Use Cases:} Medical diagnosis, stock market predictions
            \item \textbf{Pros:} Highly accurate, robust against overfitting, handles large datasets
            \item \textbf{Cons:} Less interpretable than a single decision tree, more computationally intensive
        \end{itemize}
    \end{block}
\end{frame}

\begin{frame}[fragile]
    \frametitle{Comparison of Learning Algorithms - Summary and Conclusion}
    \begin{block}{Comparison Summary}
        \begin{center}
        \begin{tabular}{|c|c|c|c|}
            \hline
            \textbf{Aspect} & \textbf{Logistic Regression} & \textbf{Decision Trees} & \textbf{Random Forests} \\
            \hline
            Model Type & Linear model & Non-linear model & Ensemble of trees \\
            \hline
            Interpretability & High & Moderate & Low \\
            \hline
            Overfitting Risk & Low & High & Lower due to ensemble nature \\
            \hline
            Data Handling & Best for linear relationships & Good for mixed data types & Excellent for complex datasets \\
            \hline
        \end{tabular}
        \end{center}
    \end{block}
    \begin{block}{Conclusion}
        Choosing the right learning algorithm depends on the problem, dataset characteristics, and desired interpretability. 
        Logistic Regression is best for linear problems, Decision Trees offer intuitive interpretations, 
        and Random Forests are preferable for high accuracy on complex tasks.
    \end{block}
\end{frame}

\begin{frame}[fragile]
    \frametitle{Common Use Cases of Supervised Learning - Overview}
    \begin{itemize}
        \item \textbf{Definition}: Supervised learning is a subset of machine learning.
        \item \textbf{Data}: Involves labeled datasets with input-output pairs.
        \item \textbf{Purpose}: Models are trained to make predictions or classifications on unseen data.
    \end{itemize}
\end{frame}

\begin{frame}[fragile]
    \frametitle{Common Use Cases of Supervised Learning - Motivations}
    \begin{block}{Motivations for Supervised Learning}
        Organizations utilize supervised learning to:
        \begin{itemize}
            \item Gain insights from historical data.
            \item Automate processes.
            \item Make informed decisions.
        \end{itemize}
    \end{block}
    \begin{itemize}
        \item Examples: 
            \begin{itemize}
                \item Predicting customer behavior.
                \item Diagnosing diseases.
            \end{itemize}
    \end{itemize}
\end{frame}

\begin{frame}[fragile]
    \frametitle{Common Use Cases of Supervised Learning - Applications}
    \begin{enumerate}
        \item \textbf{Healthcare}
            \begin{itemize}
                \item \textbf{Application}: Disease Diagnosis
                \item \textbf{Example}: Logistic regression predicts disease likelihood (e.g., diabetes).
            \end{itemize}
        \item \textbf{Finance}
            \begin{itemize}
                \item \textbf{Application}: Credit Scoring
                \item \textbf{Example}: Decision trees classify loan applicants.
            \end{itemize}
        \item \textbf{Retail}
            \begin{itemize}
                \item \textbf{Application}: Customer Churn Prediction
                \item \textbf{Example}: Random forests analyze purchase history.
            \end{itemize}
        \item \textbf{Marketing}
            \begin{itemize}
                \item \textbf{Application}: Email Campaign Targeting
                \item \textbf{Example}: Logistic regression predicts responses to marketing emails.
            \end{itemize}
        \item \textbf{Manufacturing}
            \begin{itemize}
                \item \textbf{Application}: Quality Control
                \item \textbf{Example}: Models predict product defects based on production data.
            \end{itemize}
        \item \textbf{Natural Language Processing}
            \begin{itemize}
                \item \textbf{Application}: Sentiment Analysis
                \item \textbf{Example}: Text classification predicts sentiment from feedback.
            \end{itemize}
    \end{enumerate}
\end{frame}

\begin{frame}[fragile]
    \frametitle{Common Use Cases of Supervised Learning - Modern Examples}
    \begin{block}{Modern Applications}
        \begin{itemize}
            \item \textbf{Artificial Intelligence}: Applications like ChatGPT use supervised learning extensively.
            \item Trained on large datasets of human-generated text for coherent responses.
        \end{itemize}
    \end{block}
    
    \begin{block}{Conclusion}
        Supervised learning transforms raw data into actionable intelligence across various industries.
    \end{block}
\end{frame}

\begin{frame}[fragile]
    \frametitle{Common Use Cases of Supervised Learning - Key Points}
    \begin{itemize}
        \item Uses labeled data for model training.
        \item Applications span healthcare, finance, retail, and more.
        \item Real-world examples include disease diagnosis, credit scoring, and customer churn prediction.
    \end{itemize}
    \begin{block}{Outline}
        \begin{itemize}
            \item Overview of supervised learning
            \item Motivations for its use
            \item Industry-specific applications
            \item Examples of modern AI applications
        \end{itemize}
    \end{block}
\end{frame}

\begin{frame}[fragile]
    \frametitle{Future Trends in Supervised Learning - Overview}
    Supervised learning has seen transformative growth in recent years. The future promises dynamic advancements and evolving applications within artificial intelligence (AI) and machine learning (ML) technologies.

    \begin{block}{Key Developments to Watch}
        - Integration of Deep Learning Techniques
        - AutoML (Automated Machine Learning)
        - Federated Learning
        - Explainable AI (XAI)
        - Sustainability and Efficiency
    \end{block}
\end{frame}

\begin{frame}[fragile]
    \frametitle{Future Trends in Supervised Learning - Key Developments}
    \begin{enumerate}
        \item \textbf{Integration of Deep Learning Techniques}
            \begin{itemize}
                \item **Explanation**: Utilizes neural networks with multiple layers, enhancing performance in tasks such as image and speech recognition.
                \item **Example**: Technologies like ChatGPT leverage deep learning to improve conversational agents and personalized content.
            \end{itemize}
        
        \item \textbf{AutoML (Automated Machine Learning)}
            \begin{itemize}
                \item **Explanation**: Automates tasks like feature selection, model building, and hyperparameter tuning.
                \item **Benefits**: Lowers barriers for non-experts and increases efficiency for seasoned practitioners.
                \item **Example**: Google Cloud AutoML allows users to train models without extensive coding knowledge.
            \end{itemize}
    \end{enumerate}
\end{frame}

\begin{frame}[fragile]
    \frametitle{Future Trends in Supervised Learning - Continued Key Developments}
    \begin{enumerate}
        \setcounter{enumi}{2}
        \item \textbf{Federated Learning}
            \begin{itemize}
                \item **Explanation**: Enables model training across decentralized data sources while keeping data local. This approach addresses privacy concerns.
                \item **Example**: In healthcare, patient data remains on-site while models learn from a diverse set of data.
            \end{itemize}
        
        \item \textbf{Explainable AI (XAI)}
            \begin{itemize}
                \item **Explanation**: Makes complex models transparent and interpretable, vital for stakeholder trust.
                \item **Example**: Methods like SHAP help elucidate feature impacts on predictions, fostering trust in AI systems.
            \end{itemize}
        
        \item \textbf{Sustainability and Efficiency}
            \begin{itemize}
                \item **Explanation**: Focuses on environmentally-friendly AI practices that reduce computational costs and energy usage.
                \item **Example**: Development of models that require fewer resources while achieving high accuracy.
            \end{itemize}
    \end{enumerate}
\end{frame}

\begin{frame}[fragile]
    \frametitle{Future Trends in Supervised Learning - Conclusion and Key Points}
    The landscape of supervised learning is evolving rapidly. Innovations in deep learning, automation, data privacy, interpretability with XAI, and sustainability practices will enhance capabilities and address ethical concerns.

    \begin{block}{Key Points to Remember}
        \begin{itemize}
            \item Innovations drive better accuracy and efficiency in supervised learning.
            \item AI applications, such as ChatGPT, leverage advancements in ML techniques.
            \item Ethical considerations and interpretability are becoming essential for AI adoption.
        \end{itemize}
    \end{block}
\end{frame}

\begin{frame}[fragile]
    \frametitle{Q\&A / Discussion on Supervised Learning}
    
    \begin{block}{Introduction to Supervised Learning}
        \begin{itemize}
            \item \textbf{Definition}: Supervised learning is a type of machine learning where an algorithm is trained on labeled data. Each training example is paired with an output label.
        \end{itemize}
    \end{block}
\end{frame}

\begin{frame}[fragile]
    \frametitle{Importance of Supervised Learning}
    
    \begin{block}{Why Supervised Learning is Important}
        \begin{itemize}
            \item \textbf{Real-World Applications}:
              \begin{itemize}
                  \item \textbf{Spam Detection}: Email providers use supervised learning to classify emails as 'spam' or 'not spam'.
                  \item \textbf{Medical Diagnosis}: Algorithms trained on patient data predict diseases, aiding doctors in diagnosis with higher accuracy.
              \end{itemize}
        \end{itemize}
    \end{block}
\end{frame}

\begin{frame}[fragile]
    \frametitle{Supervised Learning Techniques}
    
    \begin{block}{Key Concepts}
        \begin{enumerate}
            \item \textbf{Types of Techniques}:
                \begin{itemize}
                    \item \textbf{Classification}: Assigns categories (e.g., images of dogs vs. cats).
                    \item \textbf{Regression}: Predicts continuous values (e.g., predicting house prices).
                \end{itemize}
            
            \item \textbf{Common Algorithms}:
                \begin{itemize}
                    \item \textbf{Linear Regression}: Useful for predicting numerical values based on linear relationships. 
                      \begin{equation}
                          y = mx + b
                      \end{equation}
                    \item \textbf{Decision Trees}: Splits data into branches to make predictions.
                    \item \textbf{Support Vector Machines (SVM)}: Effective for high-dimensional spaces, used in classification tasks.
                \end{itemize}
            
            \item \textbf{Evaluation Metrics}:
                \begin{itemize}
                    \item \textbf{Accuracy}: Proportion of true results.
                    \item \textbf{Precision and Recall}: Important for imbalanced datasets.
                    \item \textbf{F1 Score}: Harmonic mean of precision and recall.
                \end{itemize}
        \end{enumerate}
    \end{block}
\end{frame}

\begin{frame}[fragile]
    \frametitle{Recent Applications and Discussion Points}
    
    \begin{block}{Recent Applications}
        \begin{itemize}
            \item \textbf{AI Tools like ChatGPT}: Models like ChatGPT utilize vast amounts of labeled data to learn language patterns, creating human-like text.
        \end{itemize}
    \end{block}
    
    \begin{block}{Discussion Points}
        \begin{itemize}
            \item What challenges do you face when implementing supervised learning?
            \item How can we handle overfitting when using complex models?
            \item Are there any real-world datasets you are curious about applying supervised learning techniques on?
        \end{itemize}
    \end{block}
\end{frame}

\begin{frame}[fragile]
    \frametitle{Closing Thoughts}
    
    \begin{block}{Engagement}
        \begin{itemize}
            \item The landscapes of AI and supervised learning are rapidly evolving.
            \item Engaging with the latest advancements and practical applications will enhance your understanding and skills in this domain.
        \end{itemize}
    \end{block}
    
    \begin{block}{Note to Students}
        \begin{itemize}
            \item Feel free to ask any questions or express topics you wish to explore further.
            \item Your active participation enriches our discussion and solidifies your learning of supervised learning techniques.
        \end{itemize}
    \end{block}
\end{frame}


\end{document}