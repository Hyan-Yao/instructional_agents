\documentclass[aspectratio=169]{beamer}

% Theme and Color Setup
\usetheme{Madrid}
\usecolortheme{whale}
\useinnertheme{rectangles}
\useoutertheme{miniframes}

% Additional Packages
\usepackage[utf8]{inputenc}
\usepackage[T1]{fontenc}
\usepackage{graphicx}
\usepackage{booktabs}
\usepackage{listings}
\usepackage{amsmath}
\usepackage{amssymb}
\usepackage{xcolor}
\usepackage{tikz}
\usepackage{pgfplots}
\pgfplotsset{compat=1.18}
\usetikzlibrary{positioning}
\usepackage{hyperref}

% Custom Colors
\definecolor{myblue}{RGB}{31, 73, 125}
\definecolor{mygray}{RGB}{100, 100, 100}
\definecolor{mygreen}{RGB}{0, 128, 0}
\definecolor{myorange}{RGB}{230, 126, 34}
\definecolor{mycodebackground}{RGB}{245, 245, 245}

% Set Theme Colors
\setbeamercolor{structure}{fg=myblue}
\setbeamercolor{frametitle}{fg=white, bg=myblue}
\setbeamercolor{title}{fg=myblue}
\setbeamercolor{section in toc}{fg=myblue}
\setbeamercolor{item projected}{fg=white, bg=myblue}
\setbeamercolor{block title}{bg=myblue!20, fg=myblue}
\setbeamercolor{block body}{bg=myblue!10}
\setbeamercolor{alerted text}{fg=myorange}

% Set Fonts
\setbeamerfont{title}{size=\Large, series=\bfseries}
\setbeamerfont{frametitle}{size=\large, series=\bfseries}
\setbeamerfont{caption}{size=\small}
\setbeamerfont{footnote}{size=\tiny}

% Document Start
\begin{document}

\frame{\titlepage}

\begin{frame}[fragile]
    \frametitle{Introduction to Propositional Logic}
    \begin{block}{Overview}
        Propositional logic, also known as sentential logic, is a branch of logic that deals with propositions. A proposition is a declarative statement that can either be true or false. This logic forms the foundational framework for reasoning in formal systems, including artificial intelligence (AI).
    \end{block}
\end{frame}

\begin{frame}[fragile]
    \frametitle{Key Concepts}
    \begin{enumerate}
        \item \textbf{Propositions}
            \begin{itemize}
                \item A statement that can be classified as either true (T) or false (F).
                \item Example: "It is raining." (This statement can be true or false based on the actual weather conditions.)
            \end{itemize}
        \item \textbf{Logical Connectives}
            \begin{itemize}
                \item \textbf{AND ($\land$)}: True if both propositions are true.
                \item \textbf{OR ($\lor$)}: True if at least one of the propositions is true.
                \item \textbf{NOT ($\neg$)}: Negates the truth value of a proposition.
                \item \textbf{IMPLIES ($\rightarrow$)}: Represents logical implication.
                \item \textbf{IFF ($\leftrightarrow$)}: True when both propositions have the same truth value.
            \end{itemize}
    \end{enumerate}
\end{frame}

\begin{frame}[fragile]
    \frametitle{Significance in AI Reasoning}
    \begin{itemize}
        \item \textbf{Foundation of Logical Reasoning}: Propositional logic provides a structure for more complex logical systems in AI, enabling machines to simulate human reasoning.
        \item \textbf{Reasoning About Knowledge}: Critical for AI applications to make inferences based on available facts.
        \item \textbf{Decision Making}: Supports decision-making processes by providing clear logical relationships.
    \end{itemize}
\end{frame}

\begin{frame}[fragile]
    \frametitle{Example of Application}
    Consider a simple AI system deciding whether to carry an umbrella:
    \begin{itemize}
        \item Let P: "It is raining."
        \item Let Q: "I will carry an umbrella."
    \end{itemize}
    
    The relationship can be expressed as:
    \begin{itemize}
        \item Implication: $P \rightarrow Q$ (If it is raining, then I will carry an umbrella.)
    \end{itemize}
    
    \begin{table}[ht]
        \centering
        \begin{tabular}{|c|c|c|}
            \hline
            P (It is raining) & Q (I carry an umbrella) & $P \rightarrow Q$  \\
            \hline
            T & T & T \\
            T & F & F \\
            F & T & T \\
            F & F & T \\
            \hline
        \end{tabular}
        \caption{Truth Table for $P \rightarrow Q$}
    \end{table}
\end{frame}

\begin{frame}[fragile]
    \frametitle{Summary}
    \begin{itemize}
        \item Propositional logic is essential for formal reasoning in AI.
        \item It involves propositions and logical connectives forming the basis for complex reasoning.
        \item Understanding propositional logic empowers us to analyze and evaluate logical arguments effectively.
    \end{itemize}
\end{frame}

\begin{frame}[fragile]{What is Propositional Logic?}
    \begin{block}{Definition}
        Propositional Logic is a fundamental area of logic that deals with propositions, which are statements that can either be true or false. 
        It serves as the basis for formal reasoning in mathematics, computer science, and artificial intelligence. 
        Propositional Logic allows us to analyze and construct logical arguments through the use of various connectives.
    \end{block}
\end{frame}

\begin{frame}[fragile]{Components of Propositional Logic}
    \begin{enumerate}
        \item \textbf{Propositions:}
        \begin{itemize}
            \item A proposition is a declarative statement that is either true (T) or false (F), but not both.
            \item \textbf{Examples:}
            \begin{itemize}
                \item "It is raining." (Can be true or false)
                \item "5 is greater than 3." (True)
                \item "The cat is a mammal." (True)
            \end{itemize}
        \end{itemize}
        
        \item \textbf{Logical Connectives:}
        \begin{itemize}
            \item \textbf{AND (Conjunction)} $\land$: True only if both propositions are true.
            \item \textbf{OR (Disjunction)} $\lor$: True if at least one proposition is true.
            \item \textbf{NOT (Negation)} $\neg$: Reverses the truth value of a proposition.
            \item \textbf{IMPLIES (Conditional)} $\rightarrow$: True unless a true proposition leads to a false one.
            \item \textbf{BICONDITIONAL (If and Only If)} $\leftrightarrow$: True if both propositions have the same truth value.
        \end{itemize}
    \end{enumerate}
\end{frame}

\begin{frame}[fragile]{Key Points and Conclusion}
    \begin{block}{Key Points to Emphasize}
        \begin{itemize}
            \item Propositional logic forms the foundation for more complex logical systems.
            \item Understanding how to construct and analyze propositions is crucial for logical reasoning.
            \item Logical connectives are essential tools for combining and modifying propositions.
        \end{itemize}
    \end{block}
    
    \begin{block}{Conclusion}
        In summary, propositional logic provides a structured way to analyze logical statements and their relationships. 
        Mastering the components of propositional logic equips you with valuable reasoning skills applicable across diverse fields such as mathematics, computer science, and artificial intelligence.
    \end{block}

    \begin{block}{Formula Recap}
        \begin{itemize}
            \item Conjunction: $P \land Q$
            \item Disjunction: $P \lor Q$
            \item Negation: $\neg P$
            \item Implication: $P \rightarrow Q$
            \item Biconditional: $P \leftrightarrow Q$
        \end{itemize}
    \end{block}
\end{frame}

\begin{frame}[fragile]
    \frametitle{Logical Connectives}
    \begin{block}{Introduction to Logical Connectives}
        Logical connectives are symbols used to connect propositions (statements that can be either true or false) to form more complex logical statements. Understanding these connectives is essential for mastering propositional logic.
    \end{block}
\end{frame}

\begin{frame}[fragile]
    \frametitle{Key Logical Connectives}
    \begin{enumerate}
        \item \textbf{AND (Conjunction)} $\land$
        \item \textbf{OR (Disjunction)} $\lor$
        \item \textbf{NOT (Negation)} $\neg$
        \item \textbf{IMPLIES (Implication)} $\rightarrow$
        \item \textbf{BICONDITIONAL (Equivalence)} $\leftrightarrow$
    \end{enumerate}
\end{frame}

\begin{frame}[fragile]
    \frametitle{AND (Conjunction) - Definition and Example}
    \begin{block}{Definition}
        The conjunction of two propositions is true if and only if both propositions are true. 
    \end{block}
    \begin{block}{Example}
        \begin{itemize}
            \item Let \( P \): "It is raining."
            \item Let \( Q \): "I have an umbrella."
            \item The statement \( P \land Q \) translates to "It is raining AND I have an umbrella." This statement is only true if both \( P \) and \( Q \) are true.
        \end{itemize}
    \end{block}
    \begin{block}{Truth Table}
        \begin{tabular}{|c|c|c|}
            \hline
            P & Q & P $\land$ Q \\
            \hline
            T & T & T \\
            T & F & F \\
            F & T & F \\
            F & F & F \\
            \hline
        \end{tabular}
    \end{block}
\end{frame}

\begin{frame}[fragile]
    \frametitle{OR (Disjunction) - Definition and Example}
    \begin{block}{Definition}
        The disjunction of two propositions is true if at least one of the propositions is true.
    \end{block}
    \begin{block}{Example}
        \begin{itemize}
            \item Using the same propositions, \( P \lor Q \) translates to "It is raining OR I have an umbrella." This statement is true if either \( P \) or \( Q \) (or both) are true.
        \end{itemize}
    \end{block}
    \begin{block}{Truth Table}
        \begin{tabular}{|c|c|c|}
            \hline
            P & Q & P $\lor$ Q \\
            \hline
            T & T & T \\
            T & F & T \\
            F & T & T \\
            F & F & F \\
            \hline
        \end{tabular}
    \end{block}
\end{frame}

\begin{frame}[fragile]
    \frametitle{NOT (Negation) - Definition and Example}
    \begin{block}{Definition}
        The negation of a proposition is true if the original proposition is false.
    \end{block}
    \begin{block}{Example}
        For \( P \), \( \neg P \) means "It is NOT raining." If \( P \) is true (it is raining), then \( \neg P \) is false.
    \end{block}
    \begin{block}{Truth Table}
        \begin{tabular}{|c|c|}
            \hline
            P & $\neg$P \\
            \hline
            T & F \\
            F & T \\
            \hline
        \end{tabular}
    \end{block}
\end{frame}

\begin{frame}[fragile]
    \frametitle{IMPLIES (Implication) - Definition and Example}
    \begin{block}{Definition}
        An implication \( P \rightarrow Q \) means "If \( P \) is true, then \( Q \) is true." It is false only when \( P \) is true and \( Q \) is false.
    \end{block}
    \begin{block}{Example}
        \( P \rightarrow Q \) translates to "If it is raining, then I will carry an umbrella." The only scenario where this is false is if it is raining and I do not carry an umbrella.
    \end{block}
    \begin{block}{Truth Table}
        \begin{tabular}{|c|c|c|}
            \hline
            P & Q & P $\rightarrow$ Q \\
            \hline
            T & T & T \\
            T & F & F \\
            F & T & T \\
            F & F & T \\
            \hline
        \end{tabular}
    \end{block}
\end{frame}

\begin{frame}[fragile]
    \frametitle{BICONDITIONAL (Equivalence) - Definition and Example}
    \begin{block}{Definition}
        The biconditional \( P \leftrightarrow Q \) means "\( P \) is true if and only if \( Q \) is true." It is true when both \( P \) and \( Q \) have the same truth value.
    \end{block}
    \begin{block}{Example}
        \( P \leftrightarrow Q \) translates to "It is raining if and only if I have an umbrella." This statement is true when both are true or both are false.
    \end{block}
    \begin{block}{Truth Table}
        \begin{tabular}{|c|c|c|}
            \hline
            P & Q & P $\leftrightarrow$ Q \\
            \hline
            T & T & T \\
            T & F & F \\
            F & T & F \\
            F & F & T \\
            \hline
        \end{tabular}
    \end{block}
\end{frame}

\begin{frame}[fragile]
    \frametitle{Key Points and Conclusion}
    \begin{block}{Key Points to Remember}
        \begin{itemize}
            \item Logical connectives combine simple propositions into complex logical statements.
            \item Each connective has its own truth table that illustrates how it operates based on the truth values of its components.
            \item Understanding how these connectives work helps in building and analyzing logical arguments effectively.
        \end{itemize}
    \end{block}
    \begin{block}{Conclusion}
        Logical connectives are foundational elements in propositional logic, allowing for the construction of intricate logical expressions. Mastery of these concepts is crucial for reasoning and problem-solving in both mathematics and computer science.
    \end{block}
\end{frame}

\begin{frame}[fragile]
    \frametitle{Truth Tables - Introduction}
    \begin{block}{Introduction to Truth Tables}
        Truth tables are fundamental tools in propositional logic used to evaluate 
        the validity of logical expressions. They systematically outline all possible 
        truth values of propositions, allowing us to visualize how the truth of a 
        compound statement is determined by the truth values of its components.
    \end{block}
\end{frame}

\begin{frame}[fragile]
    \frametitle{Truth Tables - Structure}
    \begin{block}{Structure of a Truth Table}
        A truth table typically consists of the following columns:
        \begin{enumerate}
            \item Individual propositions (e.g., P, Q) that represent basic statements.
            \item Columns for each logical connective (e.g., AND, OR, NOT, IMPLIES).
            \item A final column representing the overall expression’s truth value based on the combinations of inputs.
        \end{enumerate}
    \end{block}
\end{frame}

\begin{frame}[fragile]
    \frametitle{Truth Tables - Examples}
    \begin{block}{Example 1: Basic Truth Table for AND (P $\land$ Q)}
        \begin{tabular}{|c|c|c|}
            \hline
            P & Q & P $\land$ Q \\
            \hline
            T & T & T \\
            T & F & F \\
            F & T & F \\
            F & F & F \\
            \hline
        \end{tabular}
        \begin{itemize}
            \item \textbf{Interpretation:} The result of `P AND Q` is true only when both P and Q are true.
        \end{itemize}
    \end{block}
    
    \begin{block}{Example 2: Truth Table for OR (P $\lor$ Q)}
        \begin{tabular}{|c|c|c|}
            \hline
            P & Q & P $\lor$ Q \\
            \hline
            T & T & T \\
            T & F & T \\
            F & T & T \\
            F & F & F \\
            \hline
        \end{tabular}
        \begin{itemize}
            \item \textbf{Interpretation:} The result of `P OR Q` is true if at least one of P or Q is true.
        \end{itemize}
    \end{block}
\end{frame}

\begin{frame}[fragile]
    \frametitle{Truth Tables - Importance and Conclusion}
    \begin{block}{Importance of Truth Tables}
        \begin{itemize}
            \item \textbf{Validity Assessment:} Help in determining whether a logical expression is valid (always true),
            unsatisfiable (always false), or contingent (true under some interpretations and false under others).
            \item \textbf{Logical Equivalence:} By comparing truth values of different logical expressions, we can establish their equivalence.
        \end{itemize}
    \end{block}

    \begin{block}{Conclusion}
        Truth tables serve as a foundational element in propositional logic, 
        aiding in understanding how logical connectives interact and allowing 
        for the evaluation of logical validity with clarity and precision. 
        By the end of this slide, students should feel confident in their ability to 
        construct and interpret truth tables.
    \end{block}
\end{frame}

\begin{frame}[fragile]
    \frametitle{Logical Equivalence - Introduction}
    \begin{block}{What is Logical Equivalence?}
        Logical equivalence refers to the relationship between two statements or propositions that have the same truth value in all possible scenarios. 
        In simpler terms, two statements are logically equivalent if they lead to the same conclusion or outcome regardless of the truth values of their components.
    \end{block}
    
    \begin{block}{Why is it Important?}
        Understanding logical equivalence is crucial in propositional logic because it allows us to:
        \begin{itemize}
            \item Simplify logical expressions
            \item Understand their relationships better
            \item Prove the truth of one statement based on another
        \end{itemize}
    \end{block}
\end{frame}

\begin{frame}[fragile]
    \frametitle{Establishing Logical Equivalence with Truth Tables}
    To determine if two propositions \( P \) and \( Q \) are logically equivalent, we construct a \textbf{truth table}. 
    A truth table lists all possible truth values for the propositions and shows the results of their logical operations. 
    If the columns for \( P \) and \( Q \) match for every possible input, then \( P \) and \( Q \) are logically equivalent.

    \begin{block}{Example}
        Consider the propositions:
        \begin{itemize}
            \item \( P: \) "It is raining."
            \item \( Q: \) "If it is not raining, then the ground is not wet."
        \end{itemize}
        Mathematically, this can be noted as:
        \[ P \equiv \neg P \rightarrow \neg Q. \]

        Here is the truth table for these expressions:
    \end{block}
\end{frame}

\begin{frame}[fragile]
    \frametitle{Truth Table for \( P \) and \( Q \)}
    \begin{tabular}{|c|c|c|c|c|}
        \hline
        \( P \) & \( \neg P \) & \( Q \) & \( \neg Q \) & \( \neg P \rightarrow \neg Q \) \\
        \hline
        T & F & T & F & T \\
        T & F & F & T & T \\
        F & T & T & F & F \\
        F & T & F & T & T \\
        \hline
    \end{tabular}

    From this table, we observe that whenever \( P \) is true, \( \neg P \rightarrow \neg Q \) also holds true, confirming that \( P \) and \( Q \) are logically equivalent. 
\end{frame}

\begin{frame}[fragile]
  \frametitle{Inference Rules Overview}
  Inference rules are fundamental components of propositional logic that allow us to derive new propositions from existing ones. 
  Understanding these rules is essential for reasoning effectively within logical systems.
  
  \begin{block}{Key Points}
    \begin{itemize}
      \item These rules form the basis for logical deductions in mathematics, computer science, and philosophy.
      \item They are utilized in algorithm design, automated theorem proving, and knowledge representation in AI systems.
    \end{itemize}
  \end{block}
\end{frame}

\begin{frame}[fragile]
  \frametitle{Key Inference Rules - Part 1}
  \begin{enumerate}
    \item \textbf{Modus Ponens (MP)}  
      \begin{itemize}
        \item \textbf{Form}: If \( P \rightarrow Q \) (If P then Q) and \( P \) is true, then \( Q \) is true.
        \item \textbf{Example}:
        \begin{itemize}
          \item Premises:  
            \begin{itemize}
              \item If it rains, the ground will be wet. ( \( P \rightarrow Q \) )  
              \item It is raining. ( \( P \) )
            \end{itemize}
          \item Conclusion: The ground is wet. ( \( Q \) )
        \end{itemize}
      \end{itemize}
    
    \item \textbf{Modus Tollens (MT)}  
      \begin{itemize}
        \item \textbf{Form}: If \( P \rightarrow Q \) and \( \neg Q \) (not Q) is true, then \( \neg P \) is true.
        \item \textbf{Example}:
        \begin{itemize}
          \item Premises:  
            \begin{itemize}
              \item If it rains, the ground will be wet. ( \( P \rightarrow Q \) )  
              \item The ground is not wet. ( \( \neg Q \) )
            \end{itemize}
          \item Conclusion: It is not raining. ( \( \neg P \) )
        \end{itemize}
      \end{itemize}
  \end{enumerate}
\end{frame}

\begin{frame}[fragile]
  \frametitle{Key Inference Rules - Part 2}
  \begin{enumerate}
    \setcounter{enumi}{2}
    \item \textbf{Disjunctive Syllogism (DS)}  
      \begin{itemize}
        \item \textbf{Form}: If \( P \lor Q \) (P or Q) and \( \neg P \) is true, then \( Q \) is true.
        \item \textbf{Example}:
        \begin{itemize}
          \item Premises:  
            \begin{itemize}
              \item I will eat pizza or pasta. ( \( P \lor Q \) )  
              \item I will not eat pizza. ( \( \neg P \) )
            \end{itemize}
          \item Conclusion: I will eat pasta. ( \( Q \) )
        \end{itemize}
      \end{itemize}
    
    \item \textbf{Hypothetical Syllogism (HS)}  
      \begin{itemize}
        \item \textbf{Form}: If \( P \rightarrow Q \) and \( Q \rightarrow R \), then \( P \rightarrow R \).
        \item \textbf{Example}:
        \begin{itemize}
          \item Premises:  
            \begin{itemize}
              \item If I study, I will pass the exam. ( \( P \rightarrow Q \) )  
              \item If I pass the exam, I will graduate. ( \( Q \rightarrow R \) )
            \end{itemize}
          \item Conclusion: If I study, I will graduate. ( \( P \rightarrow R \) )
        \end{itemize}
      \end{itemize}
    
    \item \textbf{Constructive Dilemma (CD)}  
      \begin{itemize}
        \item \textbf{Form}: If \( P \rightarrow Q \) and \( R \rightarrow S \), then \( P \lor R \) implies \( Q \lor S \).
        \item \textbf{Example}:
        \begin{itemize}
          \item Premises:  
            \begin{itemize}
              \item If I go for a walk, I will get some exercise. ( \( P \rightarrow Q \) )  
              \item If I go swimming, I will feel refreshed. ( \( R \rightarrow S \) )  
              \item I will go for a walk or swim. ( \( P \lor R \) )
            \end{itemize}
          \item Conclusion: I will either get some exercise or feel refreshed. ( \( Q \lor S \) )
        \end{itemize}
      \end{itemize}
  \end{enumerate}
\end{frame}

\begin{frame}[fragile]
    \frametitle{Applications of Propositional Logic in AI - Introduction}
    Propositional logic serves as a foundational framework in AI for representing and processing knowledge. It utilizes simple statements, called propositions, which can be either true or false. The basic elements of propositional logic include logical operators such as:

    \begin{itemize}
        \item AND ($\land$)
        \item OR ($\lor$)
        \item NOT ($\lnot$)
        \item Implications ($\rightarrow$)
    \end{itemize}
\end{frame}

\begin{frame}[fragile]
    \frametitle{Applications of Propositional Logic in AI - Key Applications}
    \begin{enumerate}
        \item \textbf{Knowledge Representation}
            \begin{itemize}
                \item Propositional logic enables AI systems to formally represent facts about the world.
                \item \textbf{Example:} A knowledge base may include:
                    \begin{itemize}
                        \item $P$: "It is raining."
                        \item $Q$: "The ground is wet."
                    \end{itemize}
                    \begin{equation}
                        P \rightarrow Q \quad \text{(If It is raining then The ground is wet)}
                    \end{equation}
            \end{itemize}
        
        \item \textbf{Reasoning Systems}
            \begin{itemize}
                \item Uses inference rules to deduce implications from known facts.
                \item \textbf{Example:} Employing Modus Ponens:
                    \begin{itemize}
                        \item Given: $P \rightarrow Q$ (If it is raining, the ground is wet) and $P$ (It is raining).
                        \item Conclusion: $Q$ (The ground is wet) can be deduced.
                    \end{itemize}
            \end{itemize}
    \end{enumerate}
\end{frame}

\begin{frame}[fragile]
    \frametitle{Applications of Propositional Logic in AI - Decision Making}
    \begin{itemize}
        \item \textbf{Decision Making}
            \begin{itemize}
                \item AI systems use propositional logic to evaluate scenarios for decision-making.
                \item \textbf{Example:} Using logical propositions in decision trees:
                    \begin{equation}
                    \text{If } P \text{ (the user is logged in) and } R \text{ (the user has admin rights) then allow access.}
                    \end{equation}
                \end{itemize}
        \item Propositional logic simplifies complex reasoning into manageable logical statements.
        \item Supports structured decision-making processes in AI.
    \end{itemize}
\end{frame}

\begin{frame}[fragile]
    \frametitle{Summary of Logical Operators}
    \begin{tabular}{|c|c|l|}
        \hline
        \textbf{Operator} & \textbf{Symbol} & \textbf{Description} \\
        \hline
        AND & $\land$ & True if both propositions are true \\
        \hline
        OR & $\lor$ & True if at least one proposition is true \\
        \hline
        NOT & $\lnot$ & True if the proposition is false \\
        \hline
        IMPLIES & $\rightarrow$ & True unless a true proposition implies a false one \\
        \hline
    \end{tabular}
\end{frame}

\begin{frame}[fragile]
    \frametitle{Applications of Propositional Logic in AI - Conclusion}
    Propositional logic forms a critical segment of artificial intelligence. It enables systems to:

    \begin{itemize}
        \item Represent knowledge clearly.
        \item Reason effectively.
        \item Make informed decisions.
    \end{itemize}
    
    Understanding these applications allows students to appreciate the practicality of logic in AI, fostering a deeper interest in both fields.
\end{frame}

\begin{frame}[fragile]
    \frametitle{Propositional Logic in Action}
    \begin{block}{Understanding Propositional Logic}
        Propositional Logic is a formal system where propositions can be assigned a truth value of either True (T) or False (F). It aids in reasoning through logical relationships and decision-making.
    \end{block}
\end{frame}

\begin{frame}[fragile]
    \frametitle{Scenario Overview: Home Security System}
    \begin{block}{Context}
        Imagine an AI application in smart home automation. This system uses propositional logic to make decisions about home security.
    \end{block}

    \begin{itemize}
        \item Let \( P \): "The front door is locked."
        \item Let \( Q \): "The garage door is locked."
        \item Let \( R \): "No motion detected inside the house."
    \end{itemize}

    \begin{block}{Logical Structure}
        The home is secure if \( S \equiv P \land Q \land R \).
    \end{block}
\end{frame}

\begin{frame}[fragile]
    \frametitle{Decision-Making and Evaluations}
    \begin{block}{Decision-Making Process}
        The AI evaluates the truth values of \( P \), \( Q \), and \( R \):
        \begin{itemize}
            \item If \( P \) = T, \( Q \) = T, and \( R \) = T, then \( S \) is True (secure).
            \item If any proposition is False, \( S \) becomes False (not secure).
        \end{itemize}
    \end{block}

    \begin{block}{Example Evaluations}
        \begin{enumerate}
            \item Case 1: \( P \) = T, \( Q \) = T, \( R \) = T → \( S \) = T (Secure)
            \item Case 2: \( P \) = T, \( Q \) = F, \( R \) = T → \( S \) = F (Not Secure)
            \item Case 3: \( P \) = F, \( Q \) = T, \( R \) = T → \( S \) = F (Not Secure)
        \end{enumerate}
    \end{block}
\end{frame}

\begin{frame}[fragile]
    \frametitle{Key Points and Conclusion}
    \begin{block}{Key Points}
        \begin{itemize}
            \item Propositional logic simplifies conditions into manageable statements.
            \item The system automates decisions based on a clear set of rules, enhancing efficiency and security.
            \item This logical framework underlies many AI applications, demonstrating versatility.
        \end{itemize}
    \end{block}

    \begin{block}{Conclusion}
        Propositional logic is fundamental for decision-making in AI systems, allowing them to intelligently navigate complex scenarios through a systematic structure of information.
    \end{block}
\end{frame}

\begin{frame}[fragile]
    \frametitle{Formalizing Reasoning Processes}
    \begin{block}{Understanding Propositional Logic in AI}
        Propositional logic serves as a foundational framework in artificial intelligence (AI) for formalizing reasoning processes. 
        It allows us to represent knowledge about the world in a structured manner and make deductions based on that knowledge.
    \end{block}
\end{frame}

\begin{frame}[fragile]
    \frametitle{What is Propositional Logic?}
    Propositional logic involves statements that can either be true or false but not both. These statements are called propositions.
    
    \begin{itemize}
        \item \textbf{Propositions}: Simple statements (e.g., "It is raining").
        \item \textbf{Logical Connectives}: Operators that combine propositions:
            \begin{itemize}
                \item \textbf{AND ($\land$)}: True if both propositions are true.
                \item \textbf{OR ($\lor$)}: True if at least one proposition is true.
                \item \textbf{NOT ($\neg$)}: Inverts the truth value of a proposition.
                \item \textbf{IMPLIES ($\rightarrow$)}: Indicates a conditional relationship (if-then).
            \end{itemize}
    \end{itemize}
\end{frame}

\begin{frame}[fragile]
    \frametitle{Importance of Formalizing Reasoning}
    \begin{itemize}
        \item \textbf{Clarity}: Provides a clear structure for arguments and helps identify relationships between different pieces of information.
        \item \textbf{Automation}: Enables computers to process and reason about knowledge systematically.
        \item \textbf{Verification}: Allows for the verification of inferences, ensuring logical consistency.
    \end{itemize}
\end{frame}

\begin{frame}[fragile]
    \frametitle{Example Scenario: A Smart Home System}
    Consider the following propositions:
    \begin{itemize}
        \item $P$: "The door is locked."
        \item $Q$: "The security alarm is on."
    \end{itemize}
    
    Using propositional logic, we can formalize reasoning in the smart home:
    
    \begin{equation}
        P \rightarrow Q
    \end{equation}

    If the door is locked ($P$), then the security alarm is on ($Q$). Therefore, if we know that the door is indeed locked, we can deduce that the alarm is activated.
\end{frame}

\begin{frame}[fragile]
    \frametitle{Applications in AI Reasoning}
    \begin{itemize}
        \item \textbf{Expert Systems}: Helps in formulating rules for decision-making (e.g., medical diagnostics).
        \item \textbf{Natural Language Processing}: Aids in understanding relationships expressed in language.
        \item \textbf{Robotics}: Enables robots to navigate environments based on sensor input.
    \end{itemize}
\end{frame}

\begin{frame}[fragile]
    \frametitle{Key Points to Emphasize}
    \begin{itemize}
        \item Propositional logic is fundamental for creating automated reasoning systems in AI.
        \item It offers a structured approach to represent and manipulate knowledge.
        \item The ability to infer new information from existing knowledge is crucial for intelligent systems.
    \end{itemize}
\end{frame}

\begin{frame}[fragile]
    \frametitle{Conclusion}
    By formalizing reasoning processes through propositional logic, AI systems can operate more efficiently and effectively in decision-making scenarios, leading to advancements across various applications.
    
    This basic understanding of propositional logic lays the groundwork for more complex reasoning systems, which will be further explored in the next slide, addressing the limitations of propositional logic in AI reasoning contexts.
\end{frame}

\begin{frame}[fragile]
    \frametitle{Limitations of Propositional Logic}
    \begin{block}{Introduction to Propositional Logic}
        Propositional logic, also known as propositional calculus, is a branch of mathematical logic that deals with propositions that can be either true or false. It provides a formal framework for reasoning but has notable limitations when applied to complex reasoning scenarios.
    \end{block}
\end{frame}

\begin{frame}[fragile]
    \frametitle{Key Limitations - Part 1}
    \begin{enumerate}
        \item \textbf{Lack of Expressiveness:}
        \begin{itemize}
            \item Propositional logic can only represent simple statements and their combinations.
            \item It cannot express statements involving individuals, properties, or relations.
            \item \textit{Example:} The statement ``All humans are mortal'' cannot be expressed.
        \end{itemize}
        
        \item \textbf{Inability to Handle Quantifiers:}
        \begin{itemize}
            \item Propositional logic does not incorporate quantifiers (like ``for all'' and ``exists'').
            \item \textit{Illustration:} The proposition ``There exists at least one student who passed the exam'' cannot be formulated.
        \end{itemize}
    \end{enumerate}
\end{frame}

\begin{frame}[fragile]
    \frametitle{Key Limitations - Part 2}
    \begin{enumerate}
        \setcounter{enumi}{2}
        \item \textbf{Binary Outcomes:}
        \begin{itemize}
            \item Each proposition has only two truth values: true or false.
            \item This binary nature can be overly simplistic for many scenarios.
            \item \textit{Real-world Context:} Situations like uncertainty or partial truth cannot be directly modeled.
        \end{itemize}

        \item \textbf{Complexity with Larger Statements:}
        \begin{itemize}
            \item Propositional logic becomes unwieldy with larger statements or many variables.
            \item \textit{Example:} Asserting multiple conditions becomes exponentially challenging.
        \end{itemize}

        \item \textbf{No Conditional Relationships:}
        \begin{itemize}
            \item It offers limited ability to express implications and the structure of relationships.
            \item \textit{Illustration:} While you can write ``A $\rightarrow$ B'', it does not explore dependence intricately.
        \end{itemize}
    \end{enumerate}
\end{frame}

\begin{frame}[fragile]
    \frametitle{Conclusion and Key Takeaway}
    \begin{block}{Conclusion}
        While propositional logic serves as a fundamental tool in logic and reasoning, its limitations necessitate the use of more advanced logical systems (like first-order logic) for complex reasoning tasks. Understanding these limitations helps in choosing the appropriate logical framework.
    \end{block}
    
    \begin{block}{Key Takeaway}
        Knowing when propositional logic falls short enables identification of when a more powerful reasoning mechanism is required.
    \end{block}
\end{frame}

\begin{frame}[fragile]
    \frametitle{References for Further Learning}
    \begin{enumerate}
        \item ``Introduction to Logic'' by Patrick Suppes.
        \item ``Logic: A Very Short Introduction'' by Graham Priest.
    \end{enumerate}
\end{frame}

\begin{frame}[fragile]
  \frametitle{Comparing Propositional Logic and First-Order Logic}
  \begin{block}{Overview}
    Propositional Logic and First-Order Logic (FOL) are fundamental frameworks for reasoning in philosophy, mathematics, and computer science. 
    This slide offers a comparison between the two systems, highlighting their distinct features, advantages, and limitations.
  \end{block}
\end{frame}

\begin{frame}[fragile]
  \frametitle{Definition}
  \begin{itemize}
    \item \textbf{Propositional Logic}: 
      \begin{itemize}
        \item Deals with propositions, which can be either true or false.
        \item Uses logical connectives (AND, OR, NOT, IMPLIES) to form complex propositions.
      \end{itemize}
    
    \item \textbf{First-Order Logic}: 
      \begin{itemize}
        \item Extends propositional logic by incorporating quantifiers and predicates.
        \item Allows for the expression of statements about objects and their relationships.
      \end{itemize}
  \end{itemize}
\end{frame}

\begin{frame}[fragile]
  \frametitle{Key Differences}
  \begin{enumerate}
    \item \textbf{Complexity of Expressions}
      \begin{itemize}
        \item Propositional Logic: Limited to simple statements (e.g., P, Q, R).
          \begin{itemize}
            \item Example: "It is raining" (represented as P).
          \end{itemize}
        \item First-Order Logic: Able to express relationships and properties.
          \begin{itemize}
            \item Example: $\forall x (Human(x) \rightarrow Mortal(x))$.
            \item "All humans are mortal" where "Human" and "Mortal" are predicates.
          \end{itemize}
      \end{itemize}
    
    \item \textbf{Quantifiers}
      \begin{itemize}
        \item Propositional Logic: No quantifiers.
        \item First-Order Logic: Utilizes existential $(\exists)$ and universal $(\forall)$ quantifiers to generalize statements about entities.
          \begin{itemize}
            \item Example: "There exists a person who is a teacher" $(\exists x Teacher(x))$.
          \end{itemize}
      \end{itemize}
    
    \item \textbf{Domains of Discourse}
      \begin{itemize}
        \item Propositional Logic: Does not specify or require a domain.
        \item First-Order Logic: Requires a domain over which variables range, allowing for more detailed assertions.
      \end{itemize}
  \end{enumerate}
\end{frame}

\begin{frame}[fragile]
  \frametitle{Advantages and Applications}
  \begin{block}{Advantages}
    \begin{itemize}
      \item \textbf{Propositional Logic}:
        \begin{itemize}
          \item Simplicity: Easier to learn and apply to basic reasoning problems.
          \item Efficiency: Many algorithms (like truth tables) effectively solve propositional logic problems.
        \end{itemize}
      \item \textbf{First-Order Logic}:
        \begin{itemize}
          \item Expressiveness: Able to capture more complex statements and relations.
          \item Richer semantics: Suitable for reasoning about objects, properties, and relationships.
        \end{itemize}
    \end{itemize}
  \end{block}
  
  \begin{block}{Applications}
    \begin{itemize}
      \item \textbf{Propositional Logic}: Used in circuit design, simple decision-making processes, and AI problems where outcomes are binary.
      \item \textbf{First-Order Logic}: Foundational in automated theorem proving, knowledge representation in AI, and natural language processing.
    \end{itemize}
  \end{block}
\end{frame}

\begin{frame}[fragile]
    \frametitle{Practical Exercises}
    Interactive exercises for applying propositional logic to solve various problems.
\end{frame}

\begin{frame}[fragile]
    \frametitle{Introduction to Practical Exercises in Propositional Logic}
    \begin{itemize}
        \item Propositional logic deals with propositions, which are statements that can either be true or false.
        \item Engaging in practical exercises reinforces understanding and enhances problem-solving skills.
        \item This series of exercises is designed to apply and solidify your knowledge of propositional logic.
    \end{itemize}
\end{frame}

\begin{frame}[fragile]
    \frametitle{Key Concepts to Remember}
    \begin{enumerate}
        \item \textbf{Propositions}: Declarative sentences that are either true (T) or false (F).
        \item \textbf{Logical Connectives}:
            \begin{itemize}
                \item \textbf{AND ( $\land$ )}: True if both propositions are true.
                \item \textbf{OR ( $\lor$ )}: True if at least one proposition is true.
                \item \textbf{NOT ( $\neg$ )}: True if the proposition is false.
                \item \textbf{IMPLICATION ( $\rightarrow$ )}: True unless a true proposition leads to a false one.
                \item \textbf{BICONDITIONAL ( $\leftrightarrow$ )}: True if both propositions are either true or false.
            \end{itemize}
    \end{enumerate}
\end{frame}

\begin{frame}[fragile]
    \frametitle{Exercise 1: Evaluate Compound Statements}
    \begin{block}{Instructions}
        Determine the truth value of the compound statement.
    \end{block}
    
    \textbf{Statement}:  
    “If it rains (P), then the ground is wet (Q). If it does not rain, then the ground is not wet.”  
    Translate into propositional form:  
    \[
    (P \rightarrow Q) \land (\neg P \rightarrow \neg Q)
    \]

    \textbf{Solution Steps}:
    \begin{enumerate}
        \item Identify individual propositions:  
            \begin{itemize}
                \item P: It rains (True/False)  
                \item Q: The ground is wet (True/False)  
            \end{itemize}
        \item Use truth tables to evaluate the statement based on possible values of P and Q.
    \end{enumerate}
\end{frame}

\begin{frame}[fragile]
    \frametitle{Exercise 2: Truth Table Construction}
    \begin{block}{Instructions}
        Construct a truth table for the expression \( P \lor (Q \land R) \).
    \end{block}
    
    \begin{table}[]
        \centering
        \begin{tabular}{|c|c|c|c|c|}
            \hline
            P & Q & R & Q $\land$ R & P $\lor$ (Q $\land$ R) \\ \hline
            T & T & T & T & T \\ \hline
            T & T & F & F & T \\ \hline
            T & F & T & F & T \\ \hline
            T & F & F & F & T \\ \hline
            F & T & T & T & T \\ \hline
            F & T & F & F & F \\ \hline
            F & F & T & F & F \\ \hline
            F & F & F & F & F \\ \hline
        \end{tabular}
        \caption{Truth Table for \( P \lor (Q \land R) \)}
    \end{table}
\end{frame}

\begin{frame}[fragile]
    \frametitle{Exercise 3: Real-World Application}
    \begin{block}{Scenario}
        Given the propositions:
        \begin{itemize}
            \item P: “You study.”
            \item Q: “You will pass the exam.”
        \end{itemize}
        Construct logical arguments using the connectives and identify true/false conclusions.
    \end{block}

    \textbf{Example Argument}:
    \begin{enumerate}
        \item If you study, then you will pass the exam (\(P \rightarrow Q\)).
        \item You did not pass the exam (\(\neg Q\)).
        \item Therefore, you did not study (\(\neg P\)) – Evaluate using the rules of inference.
    \end{enumerate}
\end{frame}

\begin{frame}[fragile]
    \frametitle{Recap and Key Points}
    \begin{itemize}
        \item \textbf{Practice Makes Perfect}: Engage regularly with propositional logic exercises to enhance understanding.
        \item \textbf{Seek Diverse Problems}: Apply logical reasoning to real-world scenarios for better contextual grasp.
        \item \textbf{Consistency is Key}: Regular review of truth tables and logical arguments strengthens reasoning skills.
    \end{itemize}
\end{frame}

\begin{frame}[fragile]
    \frametitle{Next Step}
    \begin{block}{Prepare for the Summary Slide}
        Reflect on these exercises as we transition to the summary of key points in propositional logic.
    \end{block}
\end{frame}

\begin{frame}[fragile]
    \frametitle{Conclusion}
    Engaging in practical exercises not only solidifies your theoretical understanding but also equips you with critical thinking skills applicable in various fields, including computer science and artificial intelligence.
\end{frame}

\begin{frame}[fragile]
  \frametitle{Summary of Key Points - Part 1}
  \begin{enumerate}
      \item \textbf{Introduction to Propositional Logic}
      \begin{itemize}
          \item Propositional logic deals with propositions that can be either true or false.
          \item Uses logical connectives to form complex propositions.
      \end{itemize}
  
      \item \textbf{Basic Components}
      \begin{itemize}
          \item \textbf{Propositions}: Simple statements with a truth value.
          \begin{itemize}
              \item Example: "It is raining." (True or False)
          \end{itemize}
          \item \textbf{Logical Connectives}:
          \begin{itemize}
              \item AND (\(\land\)), OR (\(\lor\)), NOT (\(\lnot\)), IMPLIES (\(\rightarrow\)), IFF (\(\leftrightarrow\)).
          \end{itemize}
      \end{itemize}
  \end{enumerate}
\end{frame}

\begin{frame}[fragile]
  \frametitle{Summary of Key Points - Part 2}
  \begin{itemize}
      \item \textbf{Truth Tables}
      \begin{itemize}
          \item Purpose: To explore truth values of logical statements systematically.
          \item Structure: Lists all combinations of truth values for propositions.
          \item Example for propositions \(P\) and \(Q\):

          \begin{center}
          \begin{tabular}{|c|c|c|c|c|c|}
          \hline
          \(P\) & \(Q\) & \(P \land Q\) & \(P \lor Q\) & \(P \rightarrow Q\) & \(P \leftrightarrow Q\) \\
          \hline
          True & True & True & True & True & True \\
          True & False & False & True & False & False \\
          False & True & False & True & True & False \\
          False & False & False & False & True & True \\
          \hline
          \end{tabular}
          \end{center}
      \end{itemize}
  \end{itemize}
\end{frame}

\begin{frame}[fragile]
  \frametitle{Summary of Key Points - Part 3}
  \begin{itemize}
      \item \textbf{Applications in AI}
      \begin{itemize}
          \item \textbf{Reasoning Systems:} Used for formal verification and automated reasoning.
          \item \textbf{Expert Systems:} Simulate human reasoning with rule-based logic.
          \item \textbf{Knowledge Representation:} Models information for computer understanding.
      \end{itemize}
  
      \item \textbf{Key Points to Emphasize}
      \begin{itemize}
          \item Foundation of computer science and AI.
          \item Importance of understanding logical connectives.
          \item Truth tables as tools to evaluate propositions.
      \end{itemize}
  
      \item \textbf{Conclusion}
      \begin{itemize}
          \item Mastering propositional logic is crucial for AI and logical reasoning.
      \end{itemize}
  \end{itemize}
\end{frame}

\begin{frame}[fragile]
    \frametitle{Discussion Questions}
    \begin{block}{Overview}
        Propositional logic is a foundational component in the field of artificial intelligence (AI). It serves as the basis for reasoning, decision-making, and problem-solving in various AI applications. This slide presents discussion questions designed to deepen understanding, stimulate critical thinking, and encourage collaboration among students.
    \end{block}
\end{frame}

\begin{frame}[fragile]
    \frametitle{Discussion Questions - Part 1}
    \begin{enumerate}
        \item \textbf{Understanding Propositional Statements}  
              What distinguishes a propositional statement from a non-propositional one?  
              \begin{itemize}
                  \item Example: "It is raining" is a propositional statement; while "Close the door" is not, as it does not offer a truth value.
              \end{itemize}
        
        \item \textbf{Truth Tables}  
              How do truth tables help us evaluate the validity of compound propositions?  
              \begin{itemize}
                  \item Example: Construct a truth table for \( p \land q \) and explain how it shows when the compound statement is true.
              \end{itemize}
        
        \item \textbf{Applications in AI}  
              In what ways does propositional logic underpin decision-making algorithms in AI?  
              \begin{itemize}
                  \item Key Point: AI systems often represent knowledge and rules using logical propositions to infer new information.
              \end{itemize}
    \end{enumerate}
\end{frame}

\begin{frame}[fragile]
    \frametitle{Discussion Questions - Part 2}
    \begin{enumerate}
        \setcounter{enumi}{3} % to continue the enumeration
        \item \textbf{Limitations of Propositional Logic}  
              What are the limitations of using propositional logic in representing complex scenarios?  
              \begin{itemize}
                  \item Example: Propositional logic can struggle with uncertainty and vagueness, which are common in real-world applications.
              \end{itemize}
        
        \item \textbf{Impact of Logical Operators}  
              How do different logical operators (AND, OR, NOT) affect the outcomes in propositional logic?  
              \begin{itemize}
                  \item Key Point: Understanding these operators is crucial for constructing accurate logical expressions that model real-world situations.
              \end{itemize}

        \item \textbf{Real-Life Scenarios}  
              Can you provide a real-world scenario where propositional logic might be applied?  
              \begin{itemize}
                  \item Example: Smart home systems using logical conditions to automate tasks, such as turning on lights when motion is detected.
              \end{itemize}
    \end{enumerate}
\end{frame}

\begin{frame}[fragile]
    \frametitle{Key Points and Next Steps}
    \begin{block}{Key Points to Emphasize}
        \begin{itemize}
            \item Propositional logic serves as the backbone of logical reasoning in AI.
            \item Truth tables and logical operators are essential tools for evaluating propositions.
            \item Understanding the limitations of propositional logic can guide the exploration of more complex logical frameworks (such as predicate logic or fuzzy logic).
        \end{itemize}
    \end{block}
    
    \begin{block}{Next Steps}
        Reflect on these questions during group discussions to explore how propositional logic can play a vital role in the development and understanding of AI systems.
    \end{block}
\end{frame}

\begin{frame}[fragile]
  \frametitle{Resources for Further Learning}
  \begin{block}{Introduction to Propositional Logic}
    Propositional logic is a branch of logic that deals with propositions that can either be true or false. Understanding this concept is fundamental for various applications, including computer science, philosophy, and artificial intelligence.
    To deepen your understanding of propositional logic, here is a collection of resources that can complement your learning experience.
  \end{block}
\end{frame}

\begin{frame}[fragile]
  \frametitle{Recommended Readings}
  \begin{enumerate}
    \item \textbf{"Logic: A Very Short Introduction" by Graham Priest}
      \begin{itemize}
        \item Offers an overview of logic's key ideas with accessible explanations for beginners.
      \end{itemize}

    \item \textbf{"Propositional Logic" from "Mathematical Logic" by Elliot Mendelsohn}
      \begin{itemize}
        \item Lays the groundwork for propositional logic and its applications in mathematical reasoning.
      \end{itemize}

    \item \textbf{"How to Prove It: A Structured Approach" by Daniel J. Velleman}
      \begin{itemize}
        \item Teaches techniques for constructing proofs, enhancing logical reasoning skills.
      \end{itemize}
  \end{enumerate}
\end{frame}

\begin{frame}[fragile]
  \frametitle{Video and Online Resources}
  \begin{block}{Video Resources}
    \begin{enumerate}
      \item \textbf{Khan Academy: Logic and Propositional Equivalence}
        \begin{itemize}
          \item Beginner-friendly videos on the basics of logic and equivalence in propositional logic. 
          \item \texttt{Watch Here: \textless https://www.khanacademy.org/math/statistics-probability/probability-statistics\textgreater}
        \end{itemize}

      \item \textbf{MIT OpenCourseWare: Introduction to Logic}
        \begin{itemize}
          \item In-depth lecture series on logic with notes and problem sets. 
          \item \texttt{Explore the Course: \textless https://ocw.mit.edu/courses/philosophy/24-241-introduction-to-logic-fall-2005/\textgreater}
        \end{itemize}
    \end{enumerate}
  \end{block}

  \begin{block}{Online Resources}
    \begin{enumerate}
      \item \textbf{Stanford Encyclopedia of Philosophy - Propositional Logic}
        \begin{itemize}
          \item Comprehensive overview of propositional logic and its significance in philosophy. 
          \item \texttt{Read Here: \textless https://plato.stanford.edu/entries/logic-propositional/\textgreater}
        \end{itemize}

      \item \textbf{Coursera: Introduction to Logic}
        \begin{itemize}
          \item Free online course introducing basic logical concepts through interactive exercises. 
          \item \texttt{Join the Course: \textless https://www.coursera.org/learn/logic-introduction\textgreater}
        \end{itemize}
    \end{enumerate}
  \end{block}
\end{frame}

\begin{frame}[fragile]
  \frametitle{Q\&A Session - Overview}
  \begin{block}{Introduction to Propositional Logic}
    Propositional Logic is a fundamental area of logic that deals with propositions, which can be either true or false. 
    Understanding its principles is essential for developing critical thinking and analytical skills.
  \end{block}

  \begin{block}{Encouraging Engagement}
    \begin{itemize}
      \item How can understanding propositional logic improve decision-making?
      \item Can you think of a real-life example where logical reasoning is crucial?
    \end{itemize}
  \end{block}
\end{frame}

\begin{frame}[fragile]
  \frametitle{Q\&A Session - Key Concepts to Review}
  \begin{enumerate}
    \item \textbf{Propositions:} Statements that declare something definite.
    \begin{itemize}
      \item Example: "The sky is blue." (True) 
      \item Example: "2 + 2 = 5." (False)
    \end{itemize}
  
    \item \textbf{Logical Connectives:}
    \begin{itemize}
      \item \textbf{And ( $\land$ ):} Both propositions must be true.
      \item \textbf{Or ( $\lor$ ):} At least one proposition must be true.
      \item \textbf{Not ( $\neg$ ):} Negates the truth of a proposition.
      \item \textbf{If...Then ( $\rightarrow$ ):} A conditional statement.
      \item \textbf{If and Only If ( $\leftrightarrow$ ):} Both propositions are equivalent.
    \end{itemize}
  
    \item \textbf{Truth Tables:} A systematic way to compute truth values based on logical connectives.
  \end{enumerate}
\end{frame}

\begin{frame}[fragile]
  \frametitle{Q\&A Session - Truth Table Example}
  \begin{center}
    \begin{tabular}{|c|c|c|c|c|c|}
      \hline
      \( P \) & \( Q \) & \( P \land Q \) & \( P \lor Q \) & \( P \rightarrow Q \) & \( P \leftrightarrow Q \) \\
      \hline
      T & T & T & T & T & T \\
      T & F & F & T & F & F \\
      F & T & F & T & T & F \\
      F & F & F & F & T & T \\
      \hline
    \end{tabular}
  \end{center}

  \begin{block}{Applications of Propositional Logic}
    \begin{itemize}
      \item Used in computer science for programming logic, algorithms, and artificial intelligence.
      \item Fundamental for constructing valid arguments in philosophy, mathematics, and law.
    \end{itemize}
  \end{block}
\end{frame}


\end{document}