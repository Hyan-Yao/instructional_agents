\documentclass{beamer}

% Theme choice
\usetheme{Madrid} % You can change to e.g., Warsaw, Berlin, CambridgeUS, etc.

% Encoding and font
\usepackage[utf8]{inputenc}
\usepackage[T1]{fontenc}

% Graphics and tables
\usepackage{graphicx}
\usepackage{booktabs}

% Code listings
\usepackage{listings}
\lstset{
    basicstyle=\ttfamily\small,
    keywordstyle=\color{blue},
    commentstyle=\color{gray},
    stringstyle=\color{red},
    breaklines=true,
    frame=single
}

% Math packages
\usepackage{amsmath}
\usepackage{amssymb}

% Colors
\usepackage{xcolor}

% TikZ and PGFPlots
\usepackage{tikz}
\usepackage{pgfplots}
\pgfplotsset{compat=1.18}
\usetikzlibrary{positioning}

% Hyperlinks
\usepackage{hyperref}

% Title information
\title{Chapter 5: Introduction to Linear Regression}
\author{Your Name}
\institute{Your Institution}
\date{\today}

\begin{document}

\frame{\titlepage}

\begin{frame}[fragile]
    \frametitle{Introduction to Linear Regression - Overview}
    \begin{block}{Overview of Linear Regression}
        Linear regression is a foundational statistical method used primarily for predictive analytics. It helps us understand the relationship between a dependent variable and one or more independent variables by fitting the best possible linear equation to the observed data.
    \end{block}
\end{frame}

\begin{frame}[fragile]
    \frametitle{Introduction to Linear Regression - Key Concepts}
    \begin{itemize}
        \item \textbf{Dependent Variable (Y)}: The outcome we are trying to predict or explain (e.g., sales, temperature).
        \item \textbf{Independent Variables (X)}: The predictors that we believe influence the dependent variable (e.g., advertising spending, time of year).
        \item \textbf{Linear Relationship}: Assumes a straight-line relationship between the dependent and independent variables.
    \end{itemize}
\end{frame}

\begin{frame}[fragile]
    \frametitle{The Linear Regression Model}
    The simple linear regression model can be expressed by the equation:
    \begin{equation}
        Y = \beta_0 + \beta_1 X + \epsilon
    \end{equation}
    \begin{itemize}
        \item \( Y \): The predicted value of the dependent variable.
        \item \( \beta_0 \): The Y-intercept of the regression line.
        \item \( \beta_1 \): The slope of the regression line (change in Y for a one-unit change in X).
        \item \( X \): The independent variable.
        \item \( \epsilon \): The error term, accounting for variability in Y that cannot be explained by X.
    \end{itemize}
\end{frame}

\begin{frame}[fragile]
    \frametitle{Example of Linear Regression}
    \begin{block}{Example}
        Imagine a business wants to predict sales based on advertising expenditures. Suppose they gather data over several months and find that:
        \begin{itemize}
            \item For every \$1,000 spent on advertising, sales increase by \$5,000 (\(\beta_1 = 5\)).
            \item When no money is spent on advertising, baseline sales are \$20,000 (\(\beta_0 = 20000\)).
        \end{itemize}
        The model is:
        \begin{equation}
            \text{Sales} = 20000 + 5000 \times (\text{Advertising Spend})
        \end{equation}
    \end{block}
\end{frame}

\begin{frame}[fragile]
    \frametitle{Key Points to Emphasize}
    \begin{itemize}
        \item \textbf{Predictive Power}: Linear regression is powerful for making informed predictions based on lean, interpretable models.
        \item \textbf{Assumptions}:
        \begin{itemize}
            \item The relationship between variables is linear.
            \item Residuals (errors) are normally distributed.
            \item Homoscedasticity (constant variance of errors).
        \end{itemize}
        \item \textbf{Application}: Widely used in finance, economics, real estate, and social sciences for modeling and forecasting trends.
    \end{itemize}
\end{frame}

\begin{frame}[fragile]
    \frametitle{Benefits of Using Linear Regression}
    \begin{itemize}
        \item \textbf{Simplicity}: Easy to interpret and implement.
        \item \textbf{Efficiency}: Requires fewer computational resources compared to more complex models.
        \item \textbf{Foundation for Advanced Techniques}: Serves as the basis for more advanced regression techniques, including multiple linear regression and polynomial regression.
    \end{itemize}
\end{frame}

\begin{frame}[fragile]
    \frametitle{Conclusion}
    Understanding linear regression is crucial for anyone involved in data analysis, as it is one of the most widely used statistical methods to uncover relationships, make predictions, and inform decision-making processes.
\end{frame}

\begin{frame}[fragile]
    \frametitle{Understanding Linear Regression - Definition}
    \begin{block}{Definition of Linear Regression}
        Linear Regression is a statistical method that models the relationship between a dependent variable (outcome) and one or more independent variables (predictors). 
        The primary goal is to find the best-fitting linear equation that can predict the value of the dependent variable based on the values of the independent variables.
    \end{block}
\end{frame}

\begin{frame}[fragile]
    \frametitle{Understanding Linear Regression - Fundamental Principles}
    \begin{enumerate}
        \item \textbf{Linear Relationship}:
        Linear regression assumes a linear relationship between the independent and dependent variables, meaning the change in the dependent variable is proportional to the change in the independent variable(s).
        
        \item \textbf{Equation of the Model}:
        The linear regression model can be expressed with the equation:
        \begin{equation}
            Y = \beta_0 + \beta_1 X_1 + \beta_2 X_2 + ... + \beta_n X_n + \epsilon
        \end{equation}
        Where:
        \begin{itemize}
            \item \( Y \): Dependent variable
            \item \( \beta_0 \): Y-intercept (constant term)
            \item \( \beta_1, \beta_2, \ldots, \beta_n \): Coefficients for each independent variable \( X_1, X_2, \ldots, X_n \)
            \item \( \epsilon \): Error term representing the difference between observed and predicted values
        \end{itemize}
        
        \item \textbf{Finding the Best Fit}:
        Coefficients (\( \beta \) values) are determined using Ordinary Least Squares (OLS), which minimizes the sum of the squares of the residuals (differences between observed and predicted values).
    \end{enumerate}
\end{frame}

\begin{frame}[fragile]
    \frametitle{Understanding Linear Regression - Example and Key Points}
    \begin{block}{Example}
        Consider a simple linear regression model predicting a student's score based on hours studied:
        \begin{equation}
            Y = 50 + 10X + \epsilon
        \end{equation}
        Here, 50 is the expected score with zero hours studied, and each additional hour studied increases the score by 10 points.
    \end{block}
    
    \begin{itemize}
        \item \textbf{Dependent vs. Independent Variables}: The dependent variable is what you are trying to predict, while independent variables are the predictors.
        \item \textbf{Linearity Assumption}: It's crucial that the relationship is linear; if not, results may be misleading.
        \item \textbf{Residuals}: Understanding the prediction error helps in diagnosing model fit; the goal is to minimize these residuals.
    \end{itemize}

    \begin{block}{Conclusion}
        Linear regression is a foundational tool in statistics and machine learning, enabling effective prediction and insights into relationships between variables.
    \end{block}
\end{frame}

\begin{frame}[fragile]
    \frametitle{Key Terminology - Introduction}
    Understanding the basic terminology of linear regression is crucial for grasping how this statistical method works. 
    In this slide, we will define four foundational concepts:
    \begin{itemize}
        \item Dependent Variable
        \item Independent Variable
        \item Coefficients
        \item Residual Errors
    \end{itemize}
\end{frame}

\begin{frame}[fragile]
    \frametitle{Key Terminology - Dependent and Independent Variables}
    \begin{enumerate}
        \item \textbf{Dependent Variable (Y)}:
            \begin{itemize}
                \item \textbf{Definition}: The outcome or response variable that you are trying to predict or explain.
                \item \textbf{Example}: In a study predicting house prices, the dependent variable would be the house price itself.
                \item \textbf{Key Point}: The dependent variable is affected by changes in the independent variable(s).
            \end{itemize}
        \item \textbf{Independent Variable (X)}:
            \begin{itemize}
                \item \textbf{Definition}: The predictor or explanatory variable that is believed to influence the dependent variable.
                \item \textbf{Example}: In our house price example, potential independent variables might include size of the house, location, and number of bedrooms.
                \item \textbf{Key Point}: There can be multiple independent variables in a regression model.
            \end{itemize}
    \end{enumerate}
\end{frame}

\begin{frame}[fragile]
    \frametitle{Key Terminology - Coefficients and Residual Errors}
    \begin{enumerate}
        \setcounter{enumi}{2}
        \item \textbf{Coefficients ($\beta$)}:
            \begin{itemize}
                \item \textbf{Definition}: Values that represent the relationship between each independent variable and the dependent variable. Coefficients indicate how much Y changes for a one-unit change in X.
                \item \textbf{Example}: If the coefficient for the size of the house (in square feet) is 200, this means that with each additional square foot, the house price is expected to increase by \$200.
                \item \textbf{Key Point}: Each independent variable has its own coefficient, reflecting its unique contribution to the model.
            \end{itemize}
        \item \textbf{Residual Errors ($\epsilon$)}:
            \begin{itemize}
                \item \textbf{Definition}: The difference between the observed value of the dependent variable and the value predicted by the model. Residuals indicate how well the model fits the data.
                \item \textbf{Formula}: Residual = Observed value (Y) - Predicted value
                \item \textbf{Example}: If the actual price of a house is \$300,000, and the model predicts it to be \$290,000, the residual error is \$10,000.
                \item \textbf{Key Point}: Analyzing residuals helps assess the accuracy of predictions and identify possible model improvements.
            \end{itemize}
    \end{enumerate}
\end{frame}

\begin{frame}[fragile]
    \frametitle{Key Terminology - Conclusion and Next Steps}
    \begin{block}{Conclusion}
        Understanding these key terms lays the groundwork for more advanced discussions about linear regression. 
        As we move forward, we will apply these concepts to the linear regression equation, connecting these definitions to a practical context.
    \end{block}
    
    \vspace{1em}
    
    \textbf{Next Step:}
    Transition to the next slide, which will explain the standard form of the linear regression equation, reinforcing the concepts introduced here.
\end{frame}

\begin{frame}[fragile]
    \frametitle{The Linear Regression Equation}
    \begin{block}{Overview}
        The linear regression equation is used to model relationships between independent variables and a dependent variable. It is represented as:
        \begin{equation}
        Y = \beta_0 + \beta_1 X_1 + \beta_2 X_2 + \ldots + \epsilon
        \end{equation}
    \end{block}
\end{frame}

\begin{frame}[fragile]
    \frametitle{Components of the Linear Regression Equation}
    \begin{itemize}
        \item \textbf{Y}: Dependent Variable (the outcome being predicted)
        \item \textbf{β₀ (Beta Zero)}: Intercept (value of Y when all X's are zero)
        \item \textbf{β₁, β₂, ... (Beta Coefficients)}: Changes in Y for unit changes in X
        \item \textbf{X₁, X₂, ...}: Independent Variables (the predictors)
        \item \textbf{ε (Epsilon)}: Error Term (the variability not explained by the model)
    \end{itemize}
\end{frame}

\begin{frame}[fragile]
    \frametitle{Example of Linear Regression}
    \begin{block}{Predicted Car Price}
        Let's consider the regression equation for predicting the price of a car based on its age and mileage:
        \begin{equation}
        \text{Price} = 20000 - 1000 \times \text{Age} - 0.05 \times \text{Mileage} + \epsilon
        \end{equation}
    \end{block}
    \begin{itemize}
        \item Intercept of 20000: Price when age and mileage are 0.
        \item Coefficient -1000: For each year older, the price decreases by $1000.
        \item Coefficient -0.05: For each additional mile, the price decreases by $0.05.
    \end{itemize}
\end{frame}

\begin{frame}[fragile]
    \frametitle{Assumptions of Linear Regression - Overview}
    Linear regression analysis rests on several foundational assumptions that ensure the validity of the model. 
    Understanding and verifying these assumptions is crucial for accurate model interpretation and prediction.
    
    The four main assumptions are:
    \begin{enumerate}
        \item Linearity
        \item Independence
        \item Homoscedasticity
        \item Normality of Residuals
    \end{enumerate}
\end{frame}

\begin{frame}[fragile]
    \frametitle{Assumptions of Linear Regression - Linearity}
    \begin{block}{Linearity}
        \textbf{Explanation:} The relationship between the independent variable(s) (X) and the dependent variable (Y) should be linear.
        This means changes in Y should be proportional to changes in X.
    \end{block}
    
    \textbf{Example:} If we predict a car’s price based on its age, we expect a linear decrease in value as the car ages.
    
    \textbf{Key Point:} If this assumption is violated, the model may not capture the true relationship, leading to biased estimates.
\end{frame}

\begin{frame}[fragile]
    \frametitle{Assumptions of Linear Regression - Independence and Homoscedasticity}
    \begin{block}{Independence}
        \textbf{Explanation:} The residuals (errors) from the model should be independent of each other.
        This means that the error for one observation should not predict the error for another.
        
        \textbf{Example:} In a study tracking individual students' exam scores, the score of one student should not affect the score of another.
        
        \textbf{Key Point:} Violation of independence can occur in time series data, leading to autocorrelation.
    \end{block}
    
    \begin{block}{Homoscedasticity}
        \textbf{Explanation:} The variance of the errors should remain constant across all levels of the independent variable(s).
        
        \textbf{Example:} A dataset where the spread of Y values increases as X increases indicates heteroscedasticity.
        
        \textbf{Key Point:} Homoscedasticity is essential for reliable hypothesis testing. If violated, it may lead to inefficient estimates.
    \end{block}
\end{frame}

\begin{frame}[fragile]
    \frametitle{Assumptions of Linear Regression - Normality of Residuals}
    \begin{block}{Normality of Residuals}
        \textbf{Explanation:} The residuals should be approximately normally distributed for the purposes of hypothesis testing.
        
        \textbf{Example:} Plotting the residuals after fitting a model should resemble a bell-shaped curve.
        
        \textbf{Key Point:} Non-normally distributed residuals can affect the validity of confidence intervals and significance tests.
    \end{block}
    
    \begin{block}{Summary}
        To ensure robust and credible results from your linear regression analysis, always check these assumptions:
        \begin{itemize}
            \item \textbf{Linearity:} Visualize your data plots.
            \item \textbf{Independence:} Consider the data collection method.
            \item \textbf{Homoscedasticity:} Use statistical tests like Breusch-Pagan or graphical analysis.
            \item \textbf{Normality:} Employ histograms or Q-Q plots to assess residual distribution.
        \end{itemize}
    \end{block}
\end{frame}

\begin{frame}[fragile]
    \frametitle{Fitting a Linear Model - Introduction}
    \begin{block}{Introduction to Fitting a Linear Model}
        Fitting a linear model involves estimating the coefficients (or parameters) of the linear regression equation. This process is critical because it allows us to make predictions based on our data. The most common method for fitting these models is called Ordinary Least Squares (OLS).
    \end{block}
\end{frame}

\begin{frame}[fragile]
    \frametitle{Fitting a Linear Model - Ordinary Least Squares}
    \begin{block}{What is Ordinary Least Squares (OLS)?}
        \begin{itemize}
            \item \textbf{Definition}: OLS is a method used to estimate the parameters in a linear regression model by minimizing the sum of the squared differences between the observed and predicted values.
            \item \textbf{Mathematical Representation}: For a simple linear regression, the model can be expressed as:
            \begin{equation}
                Y = \beta_0 + \beta_1X + \epsilon
            \end{equation}
            Where:
            \begin{itemize}
                \item \(Y\) = dependent variable
                \item \(X\) = independent variable
                \item \(\beta_0\) = intercept
                \item \(\beta_1\) = slope coefficient
                \item \(\epsilon\) = error term
            \end{itemize}
            \item \textbf{Goal}: The goal of OLS is to find \(\beta_0\) and \(\beta_1\) that minimize:
            \begin{equation}
                \text{Minimize } \sum (Y_i - (\beta_0 + \beta_1X_i))^2
            \end{equation}
        \end{itemize}
    \end{block}
\end{frame}

\begin{frame}[fragile]
    \frametitle{Fitting a Linear Model - Steps and Example}
    \begin{block}{Steps in Fitting a Linear Model Using OLS}
        \begin{enumerate}
            \item \textbf{Collect Data}: Gather data points for the dependent and independent variables.
            \item \textbf{Set Up the Model}: Define the linear equation and identify the independent and dependent variables.
            \item \textbf{Estimate Coefficients}: Use OLS to calculate \(\beta_0\) and \(\beta_1\).
            \item \textbf{Analyze Results}: Check the estimated coefficients and perform diagnostic checks.
        \end{enumerate}
    \end{block}

    \begin{block}{Example}
        Imagine we want to predict a student’s final grade based on the number of hours studied:
        \begin{center}
        \begin{tabular}{|c|c|}
            \hline
            Hours Studied (X) & Final Grade (Y) \\
            \hline
            1 & 60 \\
            2 & 70 \\
            3 & 80 \\
            4 & 90 \\
            \hline
        \end{tabular}
        \end{center}
        The OLS regression equation might be:
        \begin{equation}
            \text{Final Grade} = 50 + 10 \cdot (\text{Hours Studied})
        \end{equation}
        This suggests that for each additional hour of study, the final grade is expected to increase by 10 points.
    \end{block}
\end{frame}

\begin{frame}[fragile]
    \frametitle{Evaluating Model Performance - Introduction}
    \begin{block}{Introduction}
        Evaluating the performance of linear regression models is essential to ensure they make accurate predictions and effectively capture the relationship between the dependent and independent variables. This slide discusses four critical metrics used to assess linear regression model performance: R², Adjusted R², RMSE, and MAE.
    \end{block}
\end{frame}

\begin{frame}[fragile]
    \frametitle{Evaluating Model Performance - Key Metrics}
    \begin{enumerate}
        \item \textbf{R-Squared (R²)}
            \begin{itemize}
                \item \textbf{Definition:} Measures the proportion of variance in the dependent variable explained by the independent variables.
                \item \textbf{Formula:}
                  \begin{equation}
                    R^2 = 1 - \frac{SS_{\text{res}}}{SS_{\text{tot}}}
                  \end{equation}
                \item \textbf{Interpretation:} Values range from 0 to 1; e.g., an R² of 0.8 means 80% variance explained.
                \item \textbf{Example:} An R² of 0.85 in a house price model explains 85% of price variability.
            \end{itemize}

        \item \textbf{Adjusted R-Squared}
            \begin{itemize}
                \item \textbf{Definition:} Modifies R² to account for the number of predictors in the model.
                \item \textbf{Formula:}
                  \begin{equation}
                    \text{Adjusted } R^2 = 1 - \left( \frac{(1 - R^2)(n - 1)}{n - p - 1} \right)
                  \end{equation}
                \item \textbf{Usage:} Compares models with different numbers of predictors.
                \item \textbf{Example:} Helps determine the better model between R² values of 0.75 with 3 predictors and 0.76 with 5 predictors.
            \end{itemize}
    \end{enumerate}
\end{frame}

\begin{frame}[fragile]
    \frametitle{Evaluating Model Performance - Error Metrics}
    \begin{enumerate}[resume]
        \item \textbf{Root Mean Squared Error (RMSE)}
            \begin{itemize}
                \item \textbf{Definition:} Measures the average magnitude of prediction errors.
                \item \textbf{Formula:}
                  \begin{equation}
                    RMSE = \sqrt{\frac{1}{n} \sum_{i=1}^{n} (y_i - \hat{y}_i)^2}
                  \end{equation}
                \item \textbf{Interpretation:} Lower RMSE values indicate better performance; sensitive to outliers.
                \item \textbf{Example:} An RMSE of \$30,000 in house prices indicates prediction error magnitude.
            \end{itemize}

        \item \textbf{Mean Absolute Error (MAE)}
            \begin{itemize}
                \item \textbf{Definition:} Quantifies the average absolute difference between actual and predicted values.
                \item \textbf{Formula:}
                  \begin{equation}
                    MAE = \frac{1}{n} \sum_{i=1}^{n} |y_i - \hat{y}_i|
                  \end{equation}
                \item \textbf{Interpretation:} Lower MAE indicates better fit; less sensitive to outliers than RMSE.
                \item \textbf{Example:} An MAE of \$25,000 suggests that predictions differ by this amount on average.
            \end{itemize}
    \end{enumerate}
\end{frame}

\begin{frame}[fragile]
    \frametitle{Evaluating Model Performance - Conclusion}
    \begin{block}{Key Points to Emphasize}
        \begin{itemize}
            \item R² measures goodness-of-fit but may be misleading with multiple predictors; Adjusted R² offers a correction.
            \item RMSE and MAE provide insights into prediction errors; RMSE is more sensitive to outliers.
            \item Selecting the right metric depends on the context of the problem and nature of data.
        \end{itemize}
    \end{block}
    \begin{block}{Final Note}
        By understanding and applying these metrics, you can effectively evaluate and improve your linear regression models.
    \end{block}
\end{frame}

\begin{frame}[fragile]
    \frametitle{Applications of Linear Regression - Overview}
    Linear regression is a powerful statistical technique used for modeling the relationship between:
    \begin{itemize}
        \item A dependent variable
        \item One or more independent variables
    \end{itemize} 
    Its applications span various industries, providing insights and predictions that guide decision-making.
\end{frame}

\begin{frame}[fragile]
    \frametitle{Applications of Linear Regression - Key Applications}
    \begin{enumerate}
        \item \textbf{Finance}
            \begin{itemize}
                \item \textbf{Risk Assessment}: Evaluates credit risk by predicting the likelihood of default.
                \item \textbf{Stock Price Prediction}: Forecasts stock prices using historical data.
            \end{itemize}

        \item \textbf{Healthcare}
            \begin{itemize}
                \item \textbf{Predicting Health Outcomes}: Forecasts patient outcomes based on various factors.
                \item \textbf{Resource Allocation}: Analyzes trends to optimize staffing and resources.
            \end{itemize}

        \item \textbf{Marketing}
            \begin{itemize}
                \item \textbf{Sales Forecasting}: Predicts future sales based on advertising spend.
                \item \textbf{Customer Segmentation}: Analyzes behavior data to tailor marketing strategies.
            \end{itemize}
    \end{enumerate}
\end{frame}

\begin{frame}[fragile]
    \frametitle{Applications of Linear Regression - Examples and Key Points}
    
    \textbf{Examples of Models:}
    \begin{itemize}
        \item Finance: 
            \[
            \text{Price} = \beta_0 + \beta_1 \times \text{Previous Price} + \epsilon
            \]
        \item Healthcare:
            \[
            \text{Length of Stay} = \beta_0 + \beta_1 \times \text{Age} + \beta_2 \times \text{Health Condition} + \epsilon
            \]
        \item Marketing:
            \[
            \text{Sales} = \beta_0 + \beta_1 \times \text{Advertising Budget} + \beta_2 \times \text{Economic Index} + \epsilon
            \]
    \end{itemize}

    \textbf{Key Points to Emphasize:}
    \begin{itemize}
        \item \textbf{Versatility}: Used across diverse fields.
        \item \textbf{Simplicity of Interpretation}: Output coefficients are easy to understand.
        \item \textbf{Foundation for Advanced Models}: Essential for more complex predictive models.
    \end{itemize}
\end{frame}

\begin{frame}[fragile]
    \frametitle{Applications of Linear Regression - Conclusion and Further Reading}
    \textbf{Conclusion:} Linear regression is a vital analytical tool aiding decision-making across sectors. 

    \textbf{Further Reading:}
    \begin{itemize}
        \item Explore case studies highlighting specific uses of linear regression.
        \item Understand regression assumptions: linearity, normality, and homoscedasticity.
    \end{itemize}
\end{frame}

\begin{frame}[fragile]
    \frametitle{Limitations of Linear Regression - Overview}
    While linear regression is a powerful tool for modeling relationships between variables, it has some inherent limitations that users must be aware of to ensure appropriate application and interpretation of results.
\end{frame}

\begin{frame}[fragile]
    \frametitle{Limitations of Linear Regression - Assumption of Linearity}
    \begin{block}{Concept}
        Linear regression assumes a linear relationship between the independent variable(s) and the dependent variable.
    \end{block}
    \begin{itemize}
        \item If the true relationship is not linear (e.g., quadratic, exponential), the model will underfit.
        \item This leads to poor predictions and misleading conclusions.
    \end{itemize}
    \begin{block}{Example}
        Consider a scenario where the relationship between advertising spend and sales revenue is quadratic. A linear model would inadequately capture this behavior.
    \end{block}
\end{frame}

\begin{frame}[fragile]
    \frametitle{Limitations of Linear Regression - Outlier Sensitivity}
    \begin{block}{Concept}
        Linear regression is sensitive to outliers, which are observations that differ significantly from other observations.
    \end{block}
    \begin{itemize}
        \item Even a single outlier can skew results, leading to an inaccurate model.
        \item This adversely affects the generalization of any conclusions.
    \end{itemize}
    \begin{block}{Example}
        A dataset with extreme sales values from a one-time promotion could distort the overall trend, prompting incorrect predictions for regular sales.
    \end{block}
\end{frame}

\begin{frame}[fragile]
    \frametitle{Limitations of Linear Regression - Homoscedasticity Requirement}
    \begin{block}{Concept}
        Linear regression assumes that the residuals of the model are distributed evenly across all levels of the independent variable(s).
    \end{block}
    \begin{itemize}
        \item If residuals show different variances (heteroscedasticity), it can lead to biased coefficients.
    \end{itemize}
    \begin{block}{Example}
        In predicting housing prices, if errors for lower-priced homes are much larger than for higher-priced homes, the model would need revision.
    \end{block}
\end{frame}

\begin{frame}[fragile]
    \frametitle{Limitations of Linear Regression - Multicollinearity}
    \begin{block}{Concept}
        Multicollinearity occurs when two or more independent variables are highly correlated with one another.
    \end{block}
    \begin{itemize}
        \item This makes it difficult to determine the individual effect of each variable on the dependent variable.
        \item It can lead to inflated standard errors and unreliable coefficient estimates.
    \end{itemize}
    \begin{block}{Example}
        In a model predicting weight, including both height and body mass index (BMI) as predictors may complicate interpretation due to overlapping information.
    \end{block}
\end{frame}

\begin{frame}[fragile]
    \frametitle{Key Points to Emphasize and Conclusion}
    \begin{itemize}
        \item \textbf{Non-linearity}: Be cautious of the underlying relationship in your data.
        \item \textbf{Outliers}: Always check for outliers and understand their impact on your model.
        \item \textbf{Homoscedasticity}: Assess the distribution of residuals to ensure they meet assumptions.
        \item \textbf{Multicollinearity}: Investigate correlations among predictors to reduce redundancy.
    \end{itemize}
    
    Understanding these limitations is crucial for effective use of linear regression. In cases of violated assumptions, consider alternatives like polynomial regression or machine learning methods.
\end{frame}

\begin{frame}[fragile]
    \frametitle{Relevant Formula}
    The linear regression equation is given by:
    \begin{equation}
        \hat{y} = \beta_0 + \beta_1 x_1 + \beta_2 x_2 + \ldots + \beta_n x_n + \epsilon 
    \end{equation}
    Where:  
    \begin{itemize}
        \item $\hat{y}$ = predicted value  
        \item $\beta_0$ = y-intercept  
        \item $\beta_n$ = coefficients of the independent variables  
        \item $x_n$ = independent variables  
        \item $\epsilon$ = error term  
    \end{itemize}

    Understanding these limitations will empower students to apply linear regression more effectively.
\end{frame}

\begin{frame}[fragile]
    \frametitle{Conclusion and Key Takeaways - Part 1}
    \begin{block}{Understanding Linear Regression}
        Linear regression is a foundational predictive modeling technique used to analyze the relationship between independent variables (predictors) and a dependent variable (outcome).
        The simple linear regression formula is given by:  
        \begin{equation}
            Y = \beta_0 + \beta_1X + \epsilon
        \end{equation}
        Where:
        \begin{itemize}
            \item $Y$ is the predicted outcome.
            \item $\beta_0$ is the y-intercept.
            \item $\beta_1$ is the slope of the line (the change in $Y$ for a unit change in $X$).
            \item $X$ is the independent variable.
            \item $\epsilon$ represents the error term.
        \end{itemize}
    \end{block}
    
    \begin{block}{Strengths of Linear Regression}
        \begin{itemize}
            \item \textbf{Simplicity}: Easy to understand and implement.
            \item \textbf{Interpretability}: Provides clear insights into the relationships between variables.
            \item \textbf{Efficiency}: Requires less computational power compared to many advanced techniques.
        \end{itemize}
    \end{block}
\end{frame}

\begin{frame}[fragile]
    \frametitle{Conclusion and Key Takeaways - Part 2}
    \begin{block}{Limitations of Linear Regression}
        As discussed, linear regression can struggle with non-linear relationships and is sensitive to outliers. 
        These limitations underscore the need for more advanced techniques in certain scenarios.
    \end{block}

    \begin{block}{Future Directions in Predictive Analytics}
        \begin{itemize}
            \item \textbf{Polynomial Regression}: Fits non-linear trends by adding polynomial terms to the regression equation.
            \item \textbf{Regularization Techniques}:
                \begin{itemize}
                    \item \textbf{Ridge and Lasso Regression}: Address overfitting by adding penalty terms. Lasso enables variable selection by shrinking coefficients to zero.
                \end{itemize}
            \item \textbf{Machine Learning Models}:
                \begin{itemize}
                    \item \textbf{Decision Trees}: Non-linear models that split data based on feature values.
                    \item \textbf{Support Vector Regression (SVR)}: Uses kernel functions for fitting non-linear data.
                \end{itemize}
            \item \textbf{Ensemble Methods}:
                \begin{itemize}
                    \item \textbf{Random Forests}: Improve accuracy by combining multiple decision trees.
                    \item \textbf{Gradient Boosting}: Sequentially builds models to optimize errors.
                \end{itemize}
        \end{itemize}
    \end{block}
\end{frame}

\begin{frame}[fragile]
    \frametitle{Conclusion and Key Takeaways - Summary}
    \begin{block}{Key Points to Emphasize}
        \begin{itemize}
            \item Linear regression serves as a springboard for understanding more complex models.
            \item Familiarity with evolving techniques enhances predictive analytics capabilities.
            \item Integrating linear regression with advanced methods illustrates continuous improvements in modeling accuracy and data interpretation.
        \end{itemize}
    \end{block}
    
    Emphasizing these concepts will aid in preparing students for real-world applications of regression analysis while appreciating the evolution of predictive modeling techniques.
\end{frame}


\end{document}