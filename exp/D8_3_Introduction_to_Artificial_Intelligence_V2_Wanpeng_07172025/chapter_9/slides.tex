\documentclass[aspectratio=169]{beamer}

% Theme and Color Setup
\usetheme{Madrid}
\usecolortheme{whale}
\useinnertheme{rectangles}
\useoutertheme{miniframes}

% Additional Packages
\usepackage[utf8]{inputenc}
\usepackage[T1]{fontenc}
\usepackage{graphicx}
\usepackage{booktabs}
\usepackage{listings}
\usepackage{amsmath}
\usepackage{amssymb}
\usepackage{xcolor}
\usepackage{tikz}
\usepackage{pgfplots}
\pgfplotsset{compat=1.18}
\usetikzlibrary{positioning}
\usepackage{hyperref}

% Custom Colors
\definecolor{myblue}{RGB}{31, 73, 125}
\definecolor{mygray}{RGB}{100, 100, 100}
\definecolor{mygreen}{RGB}{0, 128, 0}
\definecolor{myorange}{RGB}{230, 126, 34}
\definecolor{mycodebackground}{RGB}{245, 245, 245}

% Set Theme Colors
\setbeamercolor{structure}{fg=myblue}
\setbeamercolor{frametitle}{fg=white, bg=myblue}
\setbeamercolor{title}{fg=myblue}
\setbeamercolor{section in toc}{fg=myblue}
\setbeamercolor{item projected}{fg=white, bg=myblue}
\setbeamercolor{block title}{bg=myblue!20, fg=myblue}
\setbeamercolor{block body}{bg=myblue!10}
\setbeamercolor{alerted text}{fg=myorange}

% Set Fonts
\setbeamerfont{title}{size=\Large, series=\bfseries}
\setbeamerfont{frametitle}{size=\large, series=\bfseries}
\setbeamerfont{caption}{size=\small}
\setbeamerfont{footnote}{size=\tiny}

% Document Start
\begin{document}

\frame{\titlepage}

\begin{frame}[fragile]
    \frametitle{Introduction to Ethical Dilemmas in AI - Overview}
    \begin{block}{Understanding Ethical Dilemmas}
        Ethical dilemmas in Artificial Intelligence (AI) refer to challenging situations where stakeholders must make decisions that could lead to conflicting values or ethical principles. As AI systems become increasingly integrated into societal structures, professionals in the field must navigate a complex landscape of ethical considerations.
    \end{block}
\end{frame}

\begin{frame}[fragile]
    \frametitle{Introduction to Ethical Dilemmas in AI - Importance}
    \begin{itemize}
        \item \textbf{Societal Impact:} 
        AI significantly affects our daily lives, from healthcare to privacy. Understanding these dilemmas informs responsible AI deployment.
        
        \item \textbf{Trust and Acceptance:} 
        Ethical concerns are vital for public trust. Transparency in AI enhances user acceptance and compliance with regulations.
        
        \item \textbf{Legal and Compliance:} 
        Addressing ethical dilemmas is crucial for adhering to emerging laws and standards regarding AI use, necessitating proactive ethical integration into practices.
    \end{itemize}
\end{frame}

\begin{frame}[fragile]
    \frametitle{Introduction to Ethical Dilemmas in AI - Key Concepts}
    \begin{itemize}
        \item \textbf{Bias in AI:} 
        AI systems can perpetuate existing biases in data, leading to discriminatory outcomes in various applications.
        
        \item \textbf{Autonomy and Decision-Making:} 
        AI’s ability to make autonomous decisions introduces complexities regarding accountability for harmful outcomes.

        \item \textbf{Privacy Concerns:} 
        The reliance on large datasets raises ethical questions about user consent and data ownership.
    \end{itemize}
\end{frame}

\begin{frame}[fragile]
    \frametitle{Introduction to Ethical Dilemmas in AI - Illustrative Example}
    \begin{block}{Case Study: Predictive Policing}
        Predictive policing algorithms analyze historical crime data to forecast future crime locations. While this can improve resource allocation, it raises ethical concerns regarding racial profiling and civil liberties. Biased training data can reinforce stereotypes and lead to disproportionate policing in specific communities.
    \end{block}
\end{frame}

\begin{frame}[fragile]
    \frametitle{Introduction to Ethical Dilemmas in AI - Conclusion}
    \begin{itemize}
        \item \textbf{Multidimensional Nature:} Ethical dilemmas in AI require nuanced insights from various stakeholders. 
        \item \textbf{Need for Ethical Frameworks:} Engaging with established frameworks can guide responsible navigation of these dilemmas.
        \item \textbf{Proactive Engagement:} Stakeholders should actively engage with ethical issues, anticipating dilemmas rather than reacting post-factum.
    \end{itemize}
    \begin{block}{Next Steps}
        In our next slide, we will explore various ethical frameworks that can help structure our discussions about AI ethics.
    \end{block}
\end{frame}

\begin{frame}[fragile]{Ethical Frameworks in AI}
    \begin{block}{Understanding Ethical Frameworks}
        Ethical frameworks provide a structured approach to examining and addressing moral questions. In AI, these frameworks guide developers, policymakers, and stakeholders in navigating ethical dilemmas.
    \end{block}
\end{frame}

\begin{frame}[fragile]{Key Ethical Frameworks for AI}
    \begin{enumerate}
        \item \textbf{IEEE Global Initiative for Ethical Considerations in AI and Autonomous Systems}
            \begin{itemize}
                \item \textbf{Objective:} Ensure that AI technologies prioritize ethical considerations and human-centric values.
                \item \textbf{Principles:}
                    \begin{itemize}
                        \item Transparency: AI systems should be understandable.
                        \item Accountability: Entities must ensure systems are accountable.
                        \item Fairness: Promote equality and avoid biases.
                        \item Privacy: Safeguard personal data.
                    \end{itemize}
            \end{itemize}
        
        \item \textbf{Asilomar AI Principles}
            \begin{itemize}
                \item \textbf{Focus:} Promote safe and beneficial AI.
                \item \textbf{Key Principles:}
                    \begin{itemize}
                        \item Research: Prioritize safety research and long-term considerations.
                        \item Value Alignment: Align AI systems with human values.
                    \end{itemize}
            \end{itemize}

        \item \textbf{EU Guidelines on Ethical AI}
            \begin{itemize}
                \item \textbf{Aim:} Regulate and guide the ethical development of AI.
                \item \textbf{Core Principles:}
                    \begin{itemize}
                        \item Human agency: Empower users and ensure oversight.
                        \item Technical robustness: AI must be reliable and secure.
                    \end{itemize}
            \end{itemize}
    \end{enumerate}
\end{frame}

\begin{frame}[fragile]{The Relevance of Ethical Frameworks}
    \begin{itemize}
        \item \textbf{Decision Making:} Guides organizations in informed decisions about AI deployment.
        \item \textbf{Public Trust:} Ethical adherence fosters public confidence in AI technologies.
        \item \textbf{Regulatory Compliance:} Assists organizations in meeting emerging regulations.
    \end{itemize}
\end{frame}

\begin{frame}[fragile]{Example in Practice: AI Hiring Algorithms}
    \begin{block}{Case Study}
        AI systems in recruitment can perpetuate bias if trained on skewed historical data. For instance, favoring one demographic over others may lead to discriminatory outcomes in hiring processes.
    \end{block}
    
    \begin{block}{Framework Application}
        By applying ethical frameworks, organizations can ensure the training data is representative, promoting fairness and transparency in AI systems.
    \end{block}
\end{frame}

\begin{frame}[fragile]{Key Points to Emphasize}
    \begin{itemize}
        \item Ethical frameworks are crucial for responsible AI development.
        \item Adhering to these frameworks mitigates risks of bias and discrimination.
        \item Institutions should stay informed about global guidelines and local regulations.
    \end{itemize}
\end{frame}

\begin{frame}[fragile]{Conclusion}
    Understanding and implementing ethical frameworks in AI fosters societal advancement while respecting human rights and values.
\end{frame}

\begin{frame}[fragile]
    \frametitle{Algorithmic Bias}
    % Overview of what algorithmic bias is and its significance in AI systems.
    \begin{block}{What is Algorithmic Bias?}
        \textbf{Definition}: Algorithmic bias refers to systematic and unfair discrimination embedded within AI systems, often resulting from the data used to train these models. When AI algorithms make decisions or predictions, they may reflect prejudices that exist in society, leading to unequal treatment of different groups.
    \end{block}
\end{frame}

\begin{frame}[fragile]
    \frametitle{How Does Algorithmic Bias Manifest?}
    % Explanation of manifesting forms of algorithmic bias.
    \begin{enumerate}
        \item \textbf{Data Bias}:
            \begin{itemize}
                \item Biases present in training data can lead to skewed results.
                \item \textit{Example}: If a facial recognition system is primarily trained on images of light-skinned individuals, it may struggle to accurately identify dark-skinned faces.
            \end{itemize}
    
        \item \textbf{Model Bias}:
            \begin{itemize}
                \item Design choices and assumptions embedded within algorithm development can introduce bias.
                \item \textit{Example}: An algorithm designed to predict recidivism rates may overemphasize certain demographic factors, leading to disproportionate predictions for minority groups.
            \end{itemize}
    
        \item \textbf{Feedback Loops}:
            \begin{itemize}
                \item Algorithms can perpetuate existing bias over time through reinforcement.
                \item \textit{Example}: If an AI model continuously finds that certain groups are less likely to receive loan approvals, it may reduce opportunities for these groups, compounding the bias.
            \end{itemize}
    \end{enumerate}
\end{frame}

\begin{frame}[fragile]
    \frametitle{Ethical Implications of Algorithmic Bias}
    % Discussion on ethical implications concerning algorithmic bias.
    \begin{itemize}
        \item \textbf{Discrimination}: Biased algorithms can lead to unfair treatment of individuals based on race, gender, or socioeconomic status, undermining the ethical principle of equality.
        
        \item \textbf{Accountability}: Determining who is responsible for biased outcomes is often complex. Is it the data collector, algorithm designer, or organization deploying the AI?
        
        \item \textbf{Trust in AI}: Consumers and stakeholders may lose trust in AI technologies if they perceive them as biased, undermining their adoption and potential benefits.
    \end{itemize}
    
    \begin{block}{Key Points to Emphasize}
        \begin{itemize}
            \item \textbf{Awareness}: Understanding that algorithmic bias is a real issue in AI systems is the first step in addressing it.
            \item \textbf{Importance of Diversity}: Ensuring diverse datasets and inclusive design teams can help mitigate bias in AI development.
            \item \textbf{Regulation and Oversight}: There is a growing need for guidelines and policies to govern the ethical use of AI technologies.
        \end{itemize}
    \end{block}
\end{frame}

\begin{frame}[fragile]
    \frametitle{Case Study: Facial Recognition Technology}
    \begin{block}{Overview of Facial Recognition Technology}
        Facial recognition technology (FRT) uses AI algorithms to identify or verify a person's identity by analyzing facial features from images or video feeds. Its applications include:
        \begin{itemize}
            \item Security and surveillance
            \item User authentication (e.g., unlocking smartphones)
        \end{itemize}
    \end{block}
\end{frame}

\begin{frame}[fragile]
    \frametitle{Ethical Concerns and Biases}
    \begin{enumerate}
        \item \textbf{Algorithmic Bias}
        \begin{itemize}
            \item \textbf{Definition:} Bias occurs when an algorithm produces systematically prejudiced results due to erroneous assumptions in the machine learning process.
            \item \textbf{Manifestation:} FRT tends to struggle with accurately identifying individuals from marginalized groups, particularly women and people of color.
            \item \textbf{Example:} A study by MIT Media Lab (2018) showed 34\% error rates for darker-skinned women compared to less than 1\% for lighter-skinned men.
        \end{itemize}
        \item \textbf{Privacy Violations}
        \begin{itemize}
            \item \textbf{Concerns:} Continuous surveillance raises significant privacy issues; many users may not consent to their images being analyzed.
            \item \textbf{Example:} In 2019, San Francisco banned city use of FRT due to concerns about monitoring without consent.
        \end{itemize}
    \end{enumerate}
\end{frame}

\begin{frame}[fragile]
    \frametitle{Ethical Concerns and Biases (Cont'd)}
    \begin{enumerate}
        \setcounter{enumi}{2}
        \item \textbf{Misuse by Authorities}
        \begin{itemize}
            \item \textbf{Risk:} Law enforcement may misuse FRT for discriminatory profiling or surveillance of specific community groups (e.g., racial minorities).
            \item \textbf{Illustration:} FRT in protests can lead to tracking and identifying individuals, raising issues of freedom of expression and assembly.
        \end{itemize}
    \end{enumerate}
\end{frame}

\begin{frame}[fragile]
    \frametitle{Key Takeaways}
    \begin{itemize}
        \item \textbf{Impact on Marginalized Groups:} FRT disproportionately affects minorities, emphasizing the need for inclusive AI development practices.
        \item \textbf{Legislation and Oversight:} Regulation is crucial to ensure ethical use; lack of oversight could lead to misuse.
        \item \textbf{Technological Transparency:} Developers must ensure transparency in algorithms, methodologies, and data sources to build trust and accountability.
    \end{itemize}
\end{frame}

\begin{frame}[fragile]
    \frametitle{Conclusion}
    As facial recognition technology evolves, addressing ethical concerns and biases is critical. Stakeholders must collaborate to develop guidelines that promote:
    \begin{itemize}
        \item Fairness
        \item Accountability
        \item Respect for human rights
    \end{itemize}
    This content equips students to critically evaluate the implications of facial recognition technology in contemporary society.
\end{frame}

\begin{frame}[fragile]
    \frametitle{Case Study: Autonomous Vehicles}
    Examination of ethical dilemmas in the context of self-driving cars, including decision-making in accident scenarios.
\end{frame}

\begin{frame}[fragile]
    \frametitle{Overview}
    Autonomous vehicles (AVs) represent a significant advancement in transportation technology, promising increased safety and efficiency. However, they also introduce complex ethical dilemmas, particularly concerning:
    \begin{itemize}
        \item Decision-making during unavoidable accident scenarios.
        \item Implications on ethical decision-making in AI.
    \end{itemize}
\end{frame}

\begin{frame}[fragile]
    \frametitle{1. Ethical Dilemmas in Autonomous Vehicles}
    \textbf{Definition of Ethical Dilemma}: A situation involving conflicting moral principles, making it challenging to choose ethically aligned actions.

    \begin{block}{Key Dilemmas}
        \begin{itemize}
            \item \textbf{Trolley Problem Analogy}: An AV must decide between two harmful outcomes (e.g., hitting a pedestrian or swerving into a barrier).
            \item \textbf{Value of Lives}: How should an AV weigh lives? Should it prioritize passengers, pedestrians, or specific demographics?
        \end{itemize}
    \end{block}
\end{frame}

\begin{frame}[fragile]
    \frametitle{2. Decision-Making Scenarios}
    \textbf{Examples of Decision-Making}:
    \begin{enumerate}
        \item \textbf{Collision Avoidance}:
            \begin{itemize}
                \item \textit{Situation}: An AV must choose between crashing into a wall (risking the passenger's life) or swerving into pedestrians.
                \item \textit{Ethical Question}: Which choice is more defensible? Is it permissible to sacrifice the few to save the many?
            \end{itemize}
        \item \textbf{Emergency Response}:
            \begin{itemize}
                \item \textit{Situation}: Should an AV communicate with emergency services first or prioritize saving its occupants?
                \item \textit{Ethical Question}: What does the responsibility of an autonomous system entail in emergencies?
            \end{itemize}
    \end{enumerate}
\end{frame}

\begin{frame}[fragile]
    \frametitle{3. Ethical Frameworks for Analysis}
    Various ethical frameworks can be applied to navigate dilemmas:
    \begin{itemize}
        \item \textbf{Utilitarianism}: Aim for the greatest good for the greatest number; actions should minimize overall harm.
        \item \textbf{Deontological Ethics}: Focus on the morality of actions, adhering to strict moral rules, like never intentionally harming a human.
        \item \textbf{Virtue Ethics}: Centers on character; an AV should embody virtues like compassion, though challenging to program.
    \end{itemize}
\end{frame}

\begin{frame}[fragile]
    \frametitle{4. Key Points to Emphasize}
    \begin{itemize}
        \item \textbf{Transparency}: AVs must explain decision-making processes to build public trust.
        \item \textbf{Accountability}: Determine who is responsible in an accident involving an AV (manufacturer, programmer, or owner).
        \item \textbf{Public Policy}: Regulations should guide the ethical deployment of AVs and align ethical decisions with societal values.
    \end{itemize}
\end{frame}

\begin{frame}[fragile]
    \frametitle{Conclusion}
    The design and deployment of autonomous vehicles require careful consideration of ethical dilemmas. As technology progresses, continuous dialogue over ethical frameworks is essential to ensure safety and societal alignment.
    
    \alert{This foundational understanding will enable further exploration and application of ethical frameworks in the following slide titled `Framework Application in Ethical Analysis'.}
\end{frame}

\begin{frame}[fragile]{Framework Application in Ethical Analysis - Part 1}
    \frametitle{Introduction to Ethical Frameworks}
    \begin{itemize}
        \item Ethical frameworks serve as structured approaches to evaluate dilemmas.
        \item They guide decisions that balance diverse values and principles.
        \item In the context of AI, they help navigate complex moral scenarios, notably in autonomous vehicles.
    \end{itemize}
\end{frame}

\begin{frame}[fragile]{Framework Application in Ethical Analysis - Part 2}
    \frametitle{Common Ethical Frameworks}
    \begin{enumerate}
        \item \textbf{Utilitarianism}
            \begin{itemize}
                \item \textbf{Definition}: Focuses on outcomes; the best action maximizes overall happiness.
                \item \textbf{Application}: Prioritizes minimizing accidents in autonomous vehicles.
            \end{itemize}
        
        \item \textbf{Deontological Ethics (Duty-Based)}
            \begin{itemize}
                \item \textbf{Definition}: Emphasizes rules and duties over consequences.
                \item \textbf{Application}: Asserts that harming any individual is inherently wrong in self-driving cars.
            \end{itemize}
        
        \item \textbf{Virtue Ethics}
            \begin{itemize}
                \item \textbf{Definition}: Focuses on the character of the moral agent.
                \item \textbf{Application}: Encourages virtues like honesty in AI technology design and deployment.
            \end{itemize}
    \end{enumerate}
\end{frame}

\begin{frame}[fragile]{Framework Application in Ethical Analysis - Part 3}
    \frametitle{Case Study: Autonomous Vehicles}
    \begin{itemize}
        \item \textbf{Scenario}: An autonomous vehicle must decide in an unavoidable accident situation.
    \end{itemize}
    
    \begin{block}{Ethical Analysis Using Frameworks}
        \begin{itemize}
            \item \textbf{Utilitarian Perspective}: 
                \begin{itemize}
                    \item Programmed to minimize total harm, possibly sacrificing one passenger’s life to save multiple pedestrians.
                \end{itemize}
            \item \textbf{Deontological Perspective}: 
                \begin{itemize}
                    \item Must prioritize human life equally, potentially halting operation to avoid any harm.
                \end{itemize}
            \item \textbf{Virtue Ethics Perspective}: 
                \begin{itemize}
                    \item Reflects virtues like care and responsibility in decision-making processes.
                \end{itemize}
        \end{itemize}
    \end{block}
    
    \begin{itemize}
        \item \textbf{Key Points to Emphasize}:
            \begin{itemize}
                \item Complexity of ethical analysis in AI.
                \item Importance of transparency in ethical decision-making.
                \item Need for continuous reflection as ethics in AI evolves.
            \end{itemize}
    \end{itemize}
\end{frame}

\begin{frame}[fragile]
    \frametitle{Group Discussions and Reflections}
    \begin{block}{Understanding AI Ethics}
        AI ethics encompasses the moral implications and responsibilities of designing, deploying, and using artificial intelligence systems. As technology evolves, ethical dilemmas arise, prompting essential discussions surrounding fairness, accountability, and transparency.
    \end{block}
\end{frame}

\begin{frame}[fragile]
    \frametitle{Objectives of Group Discussions}
    \begin{itemize}
        \item \textbf{Personal Reflection}: Encourage students to contemplate their own experiences and projects involving AI. What ethical dilemmas have you faced?
        \item \textbf{Collaborative Learning}: Foster an environment where students share insights, challenges, and resolutions regarding ethical issues in AI.
        \item \textbf{Applied Ethics}: Reinforce concepts learned in previous sessions, including ethical frameworks and decision-making practices.
    \end{itemize}
\end{frame}

\begin{frame}[fragile]
    \frametitle{Key Questions for Discussion}
    \begin{enumerate}
        \item \textbf{Identifying Ethical Dilemmas}
        \begin{itemize}
            \item What ethical challenges did you encounter in your AI projects?
            \item Were there unforeseen consequences when developing or implementing your AI solution?
        \end{itemize}
        
        \item \textbf{Personal Experience}
        \begin{itemize}
            \item Reflect on a time when ethical considerations impacted your decision-making.
            \item How did your personal values align with the ethical implications of AI usage in that scenario?
        \end{itemize}
        
        \item \textbf{Framework Application}
        \begin{itemize}
            \item How can ethical frameworks discussed in the previous session (e.g., Utilitarianism, Deontological Ethics) assist in guiding your project's ethics?
            \item Can you apply any framework to evaluate a dilemma from your project?
        \end{itemize}
    \end{enumerate}
\end{frame}

\begin{frame}[fragile]
    \frametitle{Strategies for Effective Discussion}
    \begin{itemize}
        \item \textbf{Active Listening}: Encourage members to listen actively and give constructive feedback.
        \item \textbf{Respect Diverse Perspectives}: Acknowledge that each student may have different ethical viewpoints shaped by personal experiences.
        \item \textbf{Document Insights}: Take notes on key points during discussions for future reference or project adjustments.
    \end{itemize}
\end{frame}

\begin{frame}[fragile]
    \frametitle{Example Scenario: Facial Recognition Technology}
    \begin{block}{Dilemma}
        A project involving facial recognition software raises concerns about privacy and potential biases.
    \end{block}
    \begin{itemize}
        \item \textbf{Discussion Points}:
        \begin{itemize}
            \item How do we balance public security and personal privacy?
            \item What steps can be taken to mitigate bias in AI algorithms?
        \end{itemize}
    \end{itemize}
\end{frame}

\begin{frame}[fragile]
    \frametitle{Conclusion}
    Group discussions on AI ethics provide a platform for critical examination and learning. By sharing experiences and insights, we can better navigate the complexities of ethical dilemmas in AI, ultimately guiding us towards responsible innovation and application in our projects.

    \smallskip
    \textit{Feel free to adapt your discussion strategies based on the specific context of your projects and experiences, fostering a rich exchange of ideas while upholding ethical standards in the field of AI.}
\end{frame}

\begin{frame}[fragile]
    \frametitle{Conclusion and Next Steps - Summary of Key Points}
    
    \begin{enumerate}
        \item \textbf{Understanding Ethical Dilemmas in AI}:
        \begin{itemize}
            \item Ethical dilemmas arise when AI systems make decisions impacting individuals or society, often containing conflicting moral principles.
            \item Key dilemmas include bias in algorithms, transparency of AI models, and developer responsibilities.
        \end{itemize}

        \item \textbf{Core Ethical Principles}:
        \begin{itemize}
            \item \textbf{Transparency}: Understanding how AI systems make decisions.
            \item \textbf{Fairness}: Ensuring AI does not perpetuate bias or discrimination.
            \item \textbf{Accountability}: Identifying responsibility when AI causes harm.
            \item \textbf{Privacy}: Safeguarding user data and maintaining confidentiality.
        \end{itemize}
    \end{enumerate}
\end{frame}

\begin{frame}[fragile]
    \frametitle{Conclusion and Next Steps - Case Studies and Future Directions}
    
    \begin{enumerate}
        \setcounter{enumi}{3}
        \item \textbf{Case Studies and Real-World Examples}:
        \begin{itemize}
            \item Examined ethical pitfalls in facial recognition technology and automated hiring algorithms.
            \item Considered mitigation strategies through improved design and policies.
        \end{itemize}

        \item \textbf{Future Directions}:
        \begin{itemize}
            \item Need for interdisciplinary collaboration among ethicists, technologists, and policymakers for effective AI regulations.
            \item Importance of ongoing education and engagement with current research in ethical AI practices.
        \end{itemize}
    \end{enumerate}
\end{frame}

\begin{frame}[fragile]
    \frametitle{Conclusion and Next Steps - Directions for Further Learning}

    \begin{enumerate}
        \item \textbf{Engage in Continued Dialogue}:
        \begin{itemize}
            \item Join forums and group discussions on AI ethics.
        \end{itemize}

        \item \textbf{Explore Relevant Resources}:
        \begin{itemize}
            \item Suggested readings: 
            \begin{itemize}
                \item "Weapons of Math Destruction" by Cathy O'Neil.
                \item Research papers discussing ethical implications of AI technologies.
            \end{itemize}
        \end{itemize}

        \item \textbf{Stay Updated on Legislative Developments}:
        \begin{itemize}
            \item Follow updates on AI regulations like the EU's AI Act.
        \end{itemize}

        \item \textbf{Practical Application}:
        \begin{itemize}
            \item Implement ethical principles in projects and conduct ethical audits.
            \item Use frameworks like the Fairness-Accuracy Tradeoff to balance AI metrics with ethics.
        \end{itemize}
    \end{enumerate}
\end{frame}


\end{document}