\documentclass[aspectratio=169]{beamer}

% Theme and Color Setup
\usetheme{Madrid}
\usecolortheme{whale}
\useinnertheme{rectangles}
\useoutertheme{miniframes}

% Additional Packages
\usepackage[utf8]{inputenc}
\usepackage[T1]{fontenc}
\usepackage{graphicx}
\usepackage{booktabs}
\usepackage{listings}
\usepackage{amsmath}
\usepackage{amssymb}
\usepackage{xcolor}
\usepackage{tikz}
\usepackage{pgfplots}
\pgfplotsset{compat=1.18}
\usetikzlibrary{positioning}
\usepackage{hyperref}

% Custom Colors
\definecolor{myblue}{RGB}{31, 73, 125}
\definecolor{mygray}{RGB}{100, 100, 100}
\definecolor{mygreen}{RGB}{0, 128, 0}
\definecolor{myorange}{RGB}{230, 126, 34}
\definecolor{mycodebackground}{RGB}{245, 245, 245}

% Set Theme Colors
\setbeamercolor{structure}{fg=myblue}
\setbeamercolor{frametitle}{fg=white, bg=myblue}
\setbeamercolor{title}{fg=myblue}
\setbeamercolor{section in toc}{fg=myblue}
\setbeamercolor{item projected}{fg=white, bg=myblue}
\setbeamercolor{block title}{bg=myblue!20, fg=myblue}
\setbeamercolor{block body}{bg=myblue!10}
\setbeamercolor{alerted text}{fg=myorange}

% Set Fonts
\setbeamerfont{title}{size=\Large, series=\bfseries}
\setbeamerfont{frametitle}{size=\large, series=\bfseries}
\setbeamerfont{caption}{size=\small}
\setbeamerfont{footnote}{size=\tiny}

% Custom Commands
\newcommand{\hilight}[1]{\colorbox{myorange!30}{#1}}
\newcommand{\source}[1]{\vspace{0.2cm}\hfill{\tiny\textcolor{mygray}{Source: #1}}}
\newcommand{\concept}[1]{\textcolor{myblue}{\textbf{#1}}}
\newcommand{\separator}{\begin{center}\rule{0.5\linewidth}{0.5pt}\end{center}}

\title[Reinforcement Learning]{Chapter 13: Advanced Topic - Reinforcement Learning}
\author[Your Name]{Your Name}
\date{\today}

% Document Start
\begin{document}

\frame{\titlepage}

\begin{frame}[fragile]
    \frametitle{Introduction to Reinforcement Learning}
    \begin{block}{What is Reinforcement Learning?}
        Reinforcement Learning (RL) is a subset of machine learning in which an agent learns to make decisions by taking actions in an environment to maximize cumulative rewards.
    \end{block}
    \begin{block}{Key Differences:}
        - RL differs from supervised learning as it learns from feedback given by the environment rather than from labeled input-output pairs.
    \end{block}
\end{frame}

\begin{frame}[fragile]
    \frametitle{Importance in Artificial Intelligence}
    \begin{itemize}
        \item \textbf{Real-World Applications:}
        \begin{itemize}
            \item Critical for robotics, game playing (e.g., AlphaGo), autonomous vehicles, and personalized recommendations.
        \end{itemize}
        
        \item \textbf{Decision-Making:}
        \begin{itemize}
            \item Mirrors human learning processes, useful for AI that requires sequential decision-making, such as navigation in self-driving cars.
        \end{itemize}
        
        \item \textbf{Exploration vs. Exploitation:}
        \begin{itemize}
            \item Balancing exploration of new actions and exploiting known high-reward actions is crucial for effective learning.
        \end{itemize}
    \end{itemize}
\end{frame}

\begin{frame}[fragile]
    \frametitle{Key Concepts in Reinforcement Learning}
    \begin{itemize}
        \item \textbf{Agent:} The learner or decision-maker (e.g., a robot or software).
        \item \textbf{Environment:} The setting with which the agent interacts (e.g., a game or the physical world).
        \item \textbf{Actions:} Choices made by the agent that affect the state of the environment.
        \item \textbf{Rewards:} Feedback received after actions, which can be positive or negative.
        \item \textbf{States:} Different situations the agent can encounter in the environment.
    \end{itemize}
\end{frame}

\begin{frame}[fragile]
    \frametitle{Example: Robot Navigating a Maze}
    \begin{itemize}
        \item \textbf{Scenario:} Robot can move forward (action A), turn left (action B), or turn right (action C).
        \item \textbf{Goal:} Reach the exit of the maze for a high reward.
        \item \textbf{Learning Process:}
        \begin{itemize}
            \item Starts with random actions (exploration).
            \item Learns which actions lead to quicker exits (exploitation) based on rewards.
        \end{itemize}
    \end{itemize}
\end{frame}

\begin{frame}[fragile]
    \frametitle{Key Takeaway}
    Reinforcement Learning is a powerful AI approach that enables systems to learn optimal behaviors through interaction with their environment. This capability is essential for complex decision-making processes across various fields.
    
    \textbf{Next up:} We'll define reinforcement learning more formally and explore its foundational principles.
\end{frame}

\begin{frame}[fragile]
    \frametitle{What is Reinforcement Learning? - Part 1}
    \begin{block}{Definition}
        Reinforcement Learning (RL) is a subset of machine learning where an agent learns to make decisions by taking actions in an environment to maximize cumulative rewards over time. It is based on the idea of an agent interacting with its environment and receiving feedback in the form of rewards or penalties.
    \end{block}
\end{frame}

\begin{frame}[fragile]
    \frametitle{What is Reinforcement Learning? - Part 2}
    \begin{block}{Foundational Principles}
        \begin{enumerate}
            \item \textbf{Agent}: The learner or decision-maker that performs actions within the environment.
            \item \textbf{Environment}: The setting within which the agent operates, providing feedback based on actions.
            \item \textbf{Actions}: Choices available to the agent that affect the state of the environment.
            \item \textbf{States}: Different situations or configurations of the environment the agent can encounter.
            \item \textbf{Rewards}: A scalar feedback signal received after performing actions; indicates the success or failure of an action relative to the agent's goal.
        \end{enumerate}
    \end{block}
\end{frame}

\begin{frame}[fragile]
    \frametitle{What is Reinforcement Learning? - Part 3}
    \begin{block}{Key Features}
        \begin{itemize}
            \item \textbf{Exploration vs. Exploitation}: Trade-off between exploring new actions and leveraging known high-reward actions.
            \item \textbf{Delayed Rewards}: Agents connect actions to long-term rewards, acknowledging consequences may not be immediate.
            \item \textbf{Policy}: Strategy employed by the agent defining the action to take in each state.
        \end{itemize}
    \end{block}
    \begin{block}{Example}
        Imagine training a dog to fetch a ball. Each time the dog retrieves the ball (action) and brings it back (state), you give it a treat (reward), reinforcing the behavior.
    \end{block}
\end{frame}

\begin{frame}[fragile]
    \frametitle{What is Reinforcement Learning? - Part 4}
    \begin{block}{Formula}
        The goal of RL can be mathematically represented as maximizing the expected cumulative rewards \( R \):
        \begin{equation}
            R = \sum_{t=0}^{T} \gamma^t r_t
        \end{equation}
        where \( r_t \) is the reward at time \( t \) and \( \gamma \) (0 ≤ \( \gamma \) < 1) is the discount factor that balances immediate and future rewards.
    \end{block}
    \begin{block}{Conclusion}
        Reinforcement Learning is a powerful framework that empowers agents to learn from their interactions, making it a fundamental pillar in the field of artificial intelligence.
    \end{block}
\end{frame}

\begin{frame}[fragile]
    \frametitle{Components of Reinforcement Learning - Overview}
    Reinforcement Learning (RL) involves an agent learning to make decisions by interacting with its environment. This slide explains the five fundamental components of RL:
    
    \begin{itemize}
        \item \textbf{Agent}
        \item \textbf{Environment}
        \item \textbf{Actions}
        \item \textbf{Rewards}
        \item \textbf{States}
    \end{itemize}
\end{frame}

\begin{frame}[fragile]
    \frametitle{Components of Reinforcement Learning - Detailed Breakdown}
    \begin{enumerate}
        \item \textbf{Agent}
        \begin{itemize}
            \item \textbf{Definition}: The learner or decision-maker in a reinforcement learning scenario.
            \item \textbf{Example}: In a game of chess, the player is the agent, making decisions based on the current game state and their strategy.
        \end{itemize}

        \item \textbf{Environment}
        \begin{itemize}
            \item \textbf{Definition}: Everything that the agent interacts with; represents the context or scenario in which the agent operates.
            \item \textbf{Example}: In a driving simulator, the environment includes the road, other vehicles, pedestrians, and traffic signals.
        \end{itemize}

        \item \textbf{Actions}
        \begin{itemize}
            \item \textbf{Definition}: The set of all possible moves or decisions the agent can make at any given time.
            \item \textbf{Example}: In a video game, actions may include moving left, right, jumping, or shooting.
            \item \textbf{Notation}: Actions are often denoted as \textbf{A} in mathematical models.
        \end{itemize}
    \end{enumerate}
\end{frame}

\begin{frame}[fragile]
    \frametitle{Components of Reinforcement Learning - Rewards and States}
    \begin{enumerate}
        \setcounter{enumi}{3}
        \item \textbf{Rewards}
        \begin{itemize}
            \item \textbf{Definition}: A scalar feedback signal that tells the agent how well or poorly it is performing in achieving its goal.
            \item \textbf{Example}: In a reinforcement learning task involving robot navigation, an agent may receive +10 for reaching a destination and -5 for colliding with an obstacle.
            \item \textbf{Formula}: The reward at time step \textbf{t} is often denoted as \( r(t) \).
        \end{itemize}

        \item \textbf{States}
        \begin{itemize}
            \item \textbf{Definition}: The current situation or configuration of the environment, which the agent observes before making actions.
            \item \textbf{Example}: In a maze, a state could represent the agent's current position within the maze.
            \item \textbf{Notation}: States are typically represented as \textbf{S} in RL notation.
        \end{itemize}
    \end{enumerate}
\end{frame}

\begin{frame}[fragile]
    \frametitle{Key Points and Diagram Suggestion}
    \textbf{Key Points to Emphasize:}
    \begin{itemize}
        \item The interaction between these components is cyclical: the agent observes the state, takes an action, receives a reward, and then updates its knowledge based on this experience.
        \item The goal of the agent is to maximize the cumulative reward over time, which involves exploring different actions to learn what yields the best outcomes.
        \item Understanding these components is crucial for grasping the more complex algorithms and strategies used in RL.
    \end{itemize}
    
    \textbf{Diagram Suggestion:}
    \begin{itemize}
        \item \textbf{State (S)} → Agent observes state
        \item \textbf{Action (A)} → Agent takes action
        \item \textbf{Environment reacts} → New State (S') + Reward (r)
        \item \textbf{Feedback loop} → Continue until a goal is reached.
    \end{itemize}
\end{frame}

\begin{frame}[fragile]
    \frametitle{Types of Reinforcement Learning: Model-Based vs. Model-Free}
    
    \begin{block}{Introduction to Reinforcement Learning Types}
        Reinforcement Learning (RL) can be broadly categorized into two main types: 
        \begin{itemize}
            \item Model-Based Learning
            \item Model-Free Learning
        \end{itemize}
        Understanding the differences between these two approaches is crucial for designing efficient RL systems.
    \end{block}
\end{frame}

\begin{frame}[fragile]
    \frametitle{1. Model-Based Reinforcement Learning}
    
    \begin{block}{Definition}
        Model-based RL refers to methods where an agent learns a model of the environment to predict future states and rewards.
    \end{block}
    
    \begin{itemize}
        \item \textbf{Planning:} The agent simulates different action sequences to evaluate their outcomes.
        \item \textbf{Sample Efficiency:} Requires fewer interactions with the environment by leveraging the learned model.
    \end{itemize}
    
    \begin{block}{Example}
        Imagine teaching a robot to navigate a maze. 
        The robot builds a model of the maze and simulates paths to find the best route to the exit without physically trying every path.
    \end{block}
\end{frame}

\begin{frame}[fragile]
    \frametitle{2. Model-Free Reinforcement Learning}
    
    \begin{block}{Definition}
        Model-free RL does not learn a model of the environment but focuses on learning an optimal policy through past experiences.
    \end{block}
    
    \begin{itemize}
        \item \textbf{Simplicity:} Easier to implement as it does not require knowledge of the environment's dynamics.
        \item \textbf{Less Sample Efficient:} Typically requires more samples to learn the optimal policy since it does not simulate outcomes.
    \end{itemize}
    
    \begin{block}{Example}
        The robot in the maze tries various paths without prior knowledge of the layout, improving its strategy based solely on experiences.
    \end{block}
\end{frame}

\begin{frame}[fragile]
    \frametitle{Comparison Summary}
    
    \begin{center}
        \begin{tabular}{|c|c|c|}
            \hline
            \textbf{Feature} & \textbf{Model-Based RL} & \textbf{Model-Free RL} \\
            \hline
            Knowledge of Environment & Yes, builds a model & No, directly learns policies \\
            \hline
            Sample Efficiency & High & Lower \\
            \hline
            Complexity & More complex (model learning) & Simpler (policy learning) \\
            \hline
        \end{tabular}
    \end{center}
\end{frame}

\begin{frame}[fragile]
    \frametitle{Conclusion and Key Points}
    
    \begin{block}{Conclusion}
        Both model-based and model-free RL approaches have advantages and setbacks. The choice depends on:
        \begin{itemize}
            \item Specific problem
            \item Complexity of the environment
            \item Computational resources
        \end{itemize}
        Understanding these differences is essential for advancing reinforcement learning techniques.
    \end{block}
    
    \begin{itemize}
        \item \textbf{Key Points to Remember:}
        \begin{itemize}
            \item Model-Based: Learns and uses a model for predictions, often more sample efficient.
            \item Model-Free: Focuses on learning the best actions from experiences without a model.
            \item Choice depends on task requirements, environment complexity, and efficiency considerations.
        \end{itemize}
    \end{itemize}
    
    \begin{block}{Next Up}
        We will explore popular algorithms in reinforcement learning, including Q-learning, Deep Q-Networks (DQN), and Policy Gradients!
    \end{block}
\end{frame}

\begin{frame}[fragile]
    \frametitle{Key Algorithms - Introduction}
    \begin{block}{Introduction to Popular Reinforcement Learning Algorithms}
        Reinforcement Learning (RL) employs various algorithms that allow agents to learn optimal behaviors through trial and error. This slide focuses on three key algorithms: Q-learning, Deep Q-Networks (DQN), and Policy Gradients.
    \end{block}
\end{frame}

\begin{frame}[fragile]
    \frametitle{Key Algorithms - Q-Learning}
    \begin{block}{1. Q-Learning}
        \begin{itemize}
            \item \textbf{Concept:} A model-free algorithm that learns the value of an action in a given state by updating action-value estimates.
            \item \textbf{Formula:}
            \begin{equation}
                Q(s, a) \leftarrow Q(s, a) + \alpha \left( r + \gamma \max_{a'} Q(s', a') - Q(s, a) \right)
            \end{equation}
            \item \textbf{Example:} Training a dog; it learns which actions yield the highest rewards (e.g., sitting on command leads to treats).
        \end{itemize}
    \end{block}
\end{frame}

\begin{frame}[fragile]
    \frametitle{Key Algorithms - DQN and Policy Gradients}
    \begin{block}{2. Deep Q-Networks (DQN)}
        \begin{itemize}
            \item \textbf{Concept:} An extension of Q-learning using neural networks to approximate the Q-value function for high-dimensional state spaces.
            \item \textbf{Key Features:}
            \begin{itemize}
                \item \textbf{Experience Replay:} Stores past experiences to improve learning stability.
                \item \textbf{Target Network:} Stabilizes training by providing consistent Q-value targets.
            \end{itemize}
            \item \textbf{Example:} Playing video games by processing visual inputs to predict optimal actions for scoring.
        \end{itemize}
    \end{block}
    
    \begin{block}{3. Policy Gradients}
        \begin{itemize}
            \item \textbf{Concept:} Directly parameterizes the policy and optimizes it based on the gradient of expected rewards.
            \item \textbf{Formula:}
            \begin{equation}
                J(\theta) = \mathbb{E}_{\tau \sim \pi_\theta} \left[ \sum_{t=0}^{T} r_t \right]
            \end{equation}
            \item \textbf{Example:} Learning to ride a bicycle involves fine-tuning balance and steering directly rather than estimating the "best move".
        \end{itemize}
    \end{block}
\end{frame}

\begin{frame}[fragile]
    \frametitle{Exploration vs. Exploitation}
    \begin{block}{Overview}
        In reinforcement learning (RL), the trade-off between exploration and exploitation is crucial for optimizing learning strategies. 
        Understanding when to explore new actions versus when to exploit known ones affects the performance of an RL agent.
    \end{block}
\end{frame}

\begin{frame}[fragile]
    \frametitle{Key Concepts}
    \begin{itemize}
        \item \textbf{Exploration}:
        \begin{itemize}
            \item Taking actions that have not been tried before to discover potential rewards.
            \item Purpose: To gather more information about the environment for better future decisions.
            \item \textit{Example}: A robot exploring a new section of a maze to find the fastest route to the exit.
        \end{itemize}
        
        \item \textbf{Exploitation}:
        \begin{itemize}
            \item Choosing the action that has the highest reward based on past experiences.
            \item Purpose: To maximize immediate rewards based on existing knowledge.
            \item \textit{Example}: A player in a game consistently using the strategy that has worked best before.
        \end{itemize}
    \end{itemize}
\end{frame}

\begin{frame}[fragile]
    \frametitle{The Trade-off and Strategies}
    \begin{block}{The Trade-off}
        The balance between exploration and exploitation is known as the \textbf{exploration-exploitation dilemma}.
        
        \begin{itemize}
            \item Too much exploration can lead to wasted resources and time.
            \item Too much exploitation can cause suboptimal performance due to missed opportunities.
        \end{itemize}
    \end{block}

    \begin{block}{Strategies to Balance}
        \begin{enumerate}
            \item \textbf{Epsilon-Greedy Strategy}:
            \begin{itemize}
                \item Selects the best-known action with probability $1 - \epsilon$ and a random action with probability $\epsilon$.
                \item Example of use: Start with a high $\epsilon$ (e.g., 0.1) and decay it over time.
                \item Formula:
                \[
                a = 
                \begin{cases} 
                \text{random action} & \text{with probability } \epsilon \\ 
                \text{argmax}(Q(s, a)) & \text{with probability } 1 - \epsilon
                \end{cases}
                \]
            \end{itemize}
            \item \textbf{Softmax Action Selection}:
            \begin{itemize}
                \item Assign probabilities to each action based on their expected rewards, promoting exploration.
            \end{itemize}
        \end{enumerate}
    \end{block}
\end{frame}

\begin{frame}[fragile]
    \frametitle{Reward Structures}
    % Examining how rewards influence learning and decision-making in reinforcement learning.

    \begin{block}{Description}
        Examining how rewards influence learning and decision-making in reinforcement learning.
    \end{block}
\end{frame}

\begin{frame}[fragile]
    \frametitle{Understanding Reward Structures}
    
    In reinforcement learning (RL), rewards are critical signals that guide an agent's learning. They provide feedback on the effectiveness of an action taken in a particular state. A good reward system helps the agent learn optimal strategies over time.

    \begin{itemize}
        \item \textbf{Reward Signal}: Represents the immediate return from taking an action. Can be positive (reward) or negative (penalty).
        \item \textbf{Cumulative Reward}: Total reward obtained over time, guiding the agent’s learning process and influencing long-term decision-making.
    \end{itemize}
\end{frame}

\begin{frame}[fragile]
    \frametitle{Types of Reward Structures}

    \begin{enumerate}
        \item \textbf{Sparse Rewards}:
            \begin{itemize}
                \item Rewards are infrequent; feedback given only after long sequences of actions. 
                \item \textit{Example}: A game like chess where the only reward is winning.
            \end{itemize}
        
        \item \textbf{Dense Rewards}:
            \begin{itemize}
                \item Feedback is more frequent, allowing for faster learning with immediate feedback.
                \item \textit{Example}: Video games where points are awarded for every achievement.
            \end{itemize}
    \end{enumerate}
\end{frame}

\begin{frame}[fragile]
    \frametitle{Positive vs. Negative Rewards}

    \begin{itemize}
        \item \textbf{Positive Rewards}: Encourage behavior.
            \begin{itemize}
                \item \textit{Example}: In a maze, rewarding the agent for getting closer to the exit, e.g., +1 for each step closer.
            \end{itemize}
        
        \item \textbf{Negative Rewards (Penalties)}: Discourage undesirable actions.
            \begin{itemize}
                \item \textit{Example}: In the maze, receiving -1 for hitting a wall.
            \end{itemize}
    \end{itemize}
\end{frame}

\begin{frame}[fragile]
    \frametitle{Formulating Rewards}

    The reward at time \( t \) can be represented as \( R_t \). The overall goal of the agent is to maximize the expected cumulative reward, often defined as:

    \begin{equation}
        G_t = R_t + \gamma R_{t+1} + \gamma^2 R_{t+2} + \ldots
    \end{equation}

    Where:
    \begin{itemize}
        \item \( G_t \) = total expected return from time \( t \)
        \item \( \gamma \) = discount factor (0 ≤ \( \gamma \) < 1) prioritizing immediate rewards over distant ones.
    \end{itemize}
\end{frame}

\begin{frame}[fragile]
    \frametitle{Key Points to Emphasize}

    \begin{itemize}
        \item The design of the reward structure greatly influences the efficiency and effectiveness of RL agents.
        \item Balancing between sparsity and density of rewards can help optimize learning in complex environments.
        \item Careful consideration of positive and negative rewards shapes desired behaviors effectively.
    \end{itemize}
\end{frame}

\begin{frame}[fragile]
    \frametitle{Practical Application: Case Study}

    In a practical application, consider a self-driving car:
    
    \begin{itemize}
        \item Receives positive rewards (+10) for maintaining a safe distance from other vehicles.
        \item Receives negative rewards (-5) for speeding or crossing lanes incorrectly.
    \end{itemize}

    This reward structure helps the car learn to prioritize safety and compliance with traffic laws.
\end{frame}

\begin{frame}[fragile]
    \frametitle{Conclusion}

    Reward structures form the bedrock of reinforcement learning. By understanding and effectively designing these structures, we can enhance the learning algorithms, enabling agents to make better and more informed decisions in a variety of applications.
\end{frame}

\begin{frame}[fragile]
    \frametitle{Real World Applications of Reinforcement Learning}
    \begin{block}{Introduction to Reinforcement Learning (RL)}
        Reinforcement Learning is a branch of machine learning where an agent learns to make decisions by taking actions in an environment to maximize cumulative rewards. The agent learns through trial-and-error interactions.
    \end{block}
\end{frame}

\begin{frame}[fragile]
    \frametitle{Key Areas of Application - Part 1}
    \begin{enumerate}
        \item \textbf{Robotics}
            \begin{itemize}
                \item \textbf{Autonomous Navigation:} Robots use RL to navigate environments. 
                \begin{itemize}
                    \item \textit{Example:} A robotic vacuum learns optimal cleaning routes.
                \end{itemize}
                \item \textbf{Manipulation Tasks:} Robots learn to perform complex tasks.
                \begin{itemize}
                    \item \textit{Illustration:} A robot trained in item manipulation through RL.
                \end{itemize}
            \end{itemize}
    \end{enumerate}
\end{frame}

\begin{frame}[fragile]
    \frametitle{Key Areas of Application - Part 2}
    \begin{enumerate}
        \setcounter{enumi}{1} % Continue numbering from the previous frame
        \item \textbf{Gaming}
            \begin{itemize}
                \item \textbf{Game Strategy Development:} RL enables agents to learn game strategies.
                \begin{itemize}
                    \item \textit{Example:} AlphaGo's mastery of Go using deep RL.
                \end{itemize}
                \item \textbf{Dynamic Difficulty Adjustment:} Games adapt difficulty based on player performance.
            \end{itemize}
        \item \textbf{Healthcare}
            \begin{itemize}
                \item \textbf{Personalized Treatment Planning:} RL models and optimizes treatment strategies.
                \begin{itemize}
                    \item \textit{Example:} RL suggests medication dosages in chronic disease management.
                \end{itemize}
                \item \textbf{Robotic Surgery:} Training robotic systems for surgery using RL.
            \end{itemize}
    \end{enumerate}
\end{frame}

\begin{frame}[fragile]
    \frametitle{Key Points and Summary}
    \begin{block}{Key Points to Emphasize}
        \begin{itemize}
            \item \textbf{Adaptability:} RL systems adapt continuously based on feedback.
            \item \textbf{Generalization:} Successful applications generalize learned behaviors.
            \item \textbf{Safety Considerations:} Ensuring ethical learning in critical applications.
        \end{itemize}
    \end{block}
    \begin{block}{Summary}
        Reinforcement learning is transforming various fields by enhancing decision-making and optimizing through autonomous learning. 
    \end{block}
    \begin{block}{Further Exploration}
        Encourage students to explore real-world case studies of RL applications.
    \end{block}
\end{frame}

\begin{frame}[fragile]
    \frametitle{Applications in Robotics - Overview}
    % Overview of Reinforcement Learning in Robotics
    \begin{block}{Reinforcement Learning (RL)}
        \begin{itemize}
            \item A type of machine learning where an agent learns to make decisions by acting in an environment.
            \item Aims to maximize cumulative rewards.
            \item Mimics trial and error learning, similar to human and animal behavior.
        \end{itemize}
    \end{block}
\end{frame}

\begin{frame}[fragile]
    \frametitle{Applications in Robotics - Mechanism}
    % How Reinforcement Learning Works
    \begin{block}{Mechanism of Learning}
        \begin{enumerate}
            \item \textbf{Agent and Environment Interaction:}
                \begin{itemize}
                    \item The robot (agent) receives states from the environment.
                    \item Chooses actions based on a policy.
                    \item Receives rewards or penalties based on actions taken.
                \end{itemize}
                
            \item \textbf{Learning Process:}
                \begin{itemize}
                    \item Updates its policy based on feedback for future improvements.
                    \item Continues until achieving consistent task performance.
                \end{itemize}
        \end{enumerate}
    \end{block}
\end{frame}

\begin{frame}[fragile]
    \frametitle{Applications in Robotics - Key Components and Applications}
    % Key Components and Practical Applications
    \begin{block}{Key Components}
        \begin{itemize}
            \item \textbf{State (s)}: Current situation of the robot.
            \item \textbf{Action (a)}: Possible movements the robot can make.
            \item \textbf{Reward (r)}: Numerical feedback indicating success or failure.
            \item \textbf{Policy (\(\pi\))}: Strategy for selecting actions.
            \item \textbf{Value Function (V)}: Predicts expected total reward from any state.
        \end{itemize}
    \end{block}

    \begin{block}{Example Applications}
        \begin{itemize}
            \item Autonomous Vehicles: Training to navigate traffic and obstacles.
            \item Warehouse Robotics: Learning to optimize delivery routes.
            \item Humanoid Robots: Improving capabilities like walking and dexterity.
        \end{itemize}
    \end{block}
\end{frame}

\begin{frame}[fragile]
    \frametitle{Applications in Game Playing}
    Examining how reinforcement learning has revolutionized game AI, including AlphaGo and DOTA 2 AI.
\end{frame}

\begin{frame}[fragile]
    \frametitle{Introduction to Reinforcement Learning in Games}
    \begin{block}{Definition}
        Reinforcement Learning (RL) has transformed game AI, enabling machines to learn and adapt through experiences.
    \end{block}
    \begin{itemize}
        \item Unlike traditional programming, RL allows agents to learn optimal strategies.
        \item Agents receive feedback from the environment based on their actions: winning, losing, or neutral outcomes.
    \end{itemize}
\end{frame}

\begin{frame}[fragile]
    \frametitle{Key Concepts of Reinforcement Learning}
    \begin{itemize}
        \item \textbf{Agent}: The learner or decision-maker (e.g., the AI playing a game).
        \item \textbf{Environment}: The setting in which the agent operates (e.g., the game board).
        \item \textbf{States}: All possible situations the agent can be in (e.g., configurations of the game).
        \item \textbf{Actions}: Choices available to the agent.
        \item \textbf{Reward}: A score received after taking an action, reinforcing certain behaviors.
    \end{itemize}
\end{frame}

\begin{frame}[fragile]
    \frametitle{AlphaGo: A Milestone in Game AI}
    \begin{itemize}
        \item \textbf{Overview}:
        \begin{itemize}
            \item Developed by DeepMind, AlphaGo defeated a human professional Go player in 2015.
        \end{itemize}
        \item \textbf{How it Works}:
        \begin{itemize}
            \item Combines deep learning with RL using \textbf{Monte Carlo Tree Search (MCTS)}.
            \item Implements \textbf{policy networks} for move prediction and a \textbf{value network} for board assessment.
        \end{itemize}
        \item \textbf{Impact}: Outperformed human strategies by learning from large datasets and self-play.
    \end{itemize}
\end{frame}

\begin{frame}[fragile]
    \frametitle{DOTA 2 AI: Competing with the Best}
    \begin{itemize}
        \item \textbf{Overview}:
        \begin{itemize}
            \item OpenAI's “OpenAI Five” competed against professional teams in DOTA 2.
        \end{itemize}
        \item \textbf{Key Features}:
        \begin{itemize}
            \item Large-scale reinforcement learning through multi-agent training.
            \item Agents share information, learning from collective experiences.
            \item Self-play training, refining strategies through millions of matches.
        \end{itemize}
        \item \textbf{Outcome}: Demonstrated the potential of RL in real-time, multi-agent cooperation.
    \end{itemize}
\end{frame}

\begin{frame}[fragile]
    \frametitle{Key Points to Emphasize}
    \begin{itemize}
        \item \textbf{Trial and Error Learning}: RL learns through experimentation.
        \item \textbf{Complex Decision Making}: Handles high-dimensional and continuous action spaces.
        \item \textbf{Emergence of New Strategies}: RL AIs discover previously unknown strategies.
    \end{itemize}
\end{frame}

\begin{frame}[fragile]
    \frametitle{Conclusion}
    Reinforcement learning highlights AI's potential in gaming and opens avenues for exploration in various fields. 
    The achievements of AlphaGo and DOTA 2 AI are monumental advancements, paving the way for the future of AI.
\end{frame}

\begin{frame}[fragile]
    \frametitle{Basic RL Algorithm Pseudocode}
    \begin{block}{Pseudocode}
    \begin{lstlisting}
    For each episode:
        Initialize state
        For each step in episode:
            Choose action based on policy
            Take action, observe reward, and new state
            Update policy based on reward received
    \end{lstlisting}
    \end{block}
    This pseudocode represents the fundamental loop of an RL agent's interaction with its environment.
\end{frame}

\begin{frame}[fragile]
  \frametitle{Applications in Healthcare}
  % Overview of reinforcement learning's impact on personalized medicine and treatment strategies.
  Reinforcement Learning (RL) is revolutionizing healthcare by enabling personalized medicine and improving treatment strategies. 
\end{frame}

\begin{frame}[fragile]
  \frametitle{Overview of Reinforcement Learning (RL) in Healthcare}
  \begin{block}{Definition}
    Reinforcement Learning (RL) is a subset of machine learning where an agent learns to make decisions by taking actions in an environment to maximize cumulative rewards.
  \end{block}
  In healthcare, RL utilizes this framework to tailor treatments to individual patients, enhancing personalized medicine and improving overall treatment strategies.
\end{frame}

\begin{frame}[fragile]
  \frametitle{Key Concepts}
  \begin{itemize}
    \item \textbf{Personalized Medicine}: Customizing healthcare decisions and treatments tailored to individual patients based on genetic, environmental, and lifestyle factors.
    \item \textbf{Dynamic Treatment Regimes}: RL algorithms can adjust treatment pathways in real-time based on patient responses to interventions.
  \end{itemize}
\end{frame}

\begin{frame}[fragile]
  \frametitle{Applications of RL in Healthcare}
  \begin{itemize}
    \item \textbf{Treatment Optimization}
      \begin{itemize}
        \item \textit{Example}: In oncology, RL determines optimal dosage and timing of chemotherapy based on patient responses.
        \item \textit{Mechanism}: The RL agent receives feedback (rewards) based on patient outcomes and adjusts treatment strategies accordingly.
      \end{itemize}
    
    \item \textbf{Chronic Disease Management}
      \begin{itemize}
        \item \textit{Example}: Managing diabetes through RL to recommend insulin dosage based on various patient factors.
        \item \textit{Outcome}: Improved blood sugar control and reduced complications.
      \end{itemize}
    
    \item \textbf{Resource Allocation}
      \begin{itemize}
        \item \textit{Example}: Optimizing allocation of limited hospital resources (ventilators, ICU beds) based on predicted patient needs.
      \end{itemize}
  \end{itemize}
\end{frame}

\begin{frame}[fragile]
  \frametitle{Reinforcement Learning Cycle}
  \begin{enumerate}
    \item \textbf{State}: Current health metrics of the patient (e.g., blood pressure, glucose levels).
    \item \textbf{Action}: Administer medication (e.g., increase or decrease dosage).
    \item \textbf{Reward}: Patient's health improvement (e.g., reduction in symptoms, improved lab results).
    \item \textbf{Policy Update}: Using feedback to improve future decisions for better outcomes.
  \end{enumerate}
\end{frame}

\begin{frame}[fragile]
  \frametitle{Conclusion}
  Reinforcement learning holds significant promise in revolutionizing healthcare by making it more personalized, efficient, and adaptive. 
  The integration of RL in treatment protocols can lead to better patient management, ultimately improving health outcomes and operational efficiencies in medical facilities.
\end{frame}

\begin{frame}[fragile]
    \frametitle{Challenges in Reinforcement Learning - Introduction}
    \begin{itemize}
        \item Reinforcement Learning (RL) has significant applications in various domains:
        \begin{itemize}
            \item Robotics
            \item Gaming
            \item Healthcare
        \end{itemize}
        \item Despite its potential, RL faces several critical challenges:
        \begin{itemize}
            \item Sample Efficiency
            \item Convergence Issues
            \item Computational Costs
        \end{itemize}
        \item Addressing these is essential for the development of effective RL algorithms.
    \end{itemize}
\end{frame}

\begin{frame}[fragile]
    \frametitle{Challenges in Reinforcement Learning - Sample Efficiency}
    \begin{block}{1. Sample Efficiency}
        \begin{itemize}
            \item \textbf{Definition}: The ability of an RL algorithm to learn effectively from limited interactions.
            \item \textbf{Challenges}:
            \begin{itemize}
                \item Many algorithms require extensive episodes to gain sufficient experience.
                \item This can be impractical in costly/limited environments (e.g., healthcare).
            \end{itemize}
            \item \textbf{Example}: 
            \begin{itemize}
                \item In healthcare, training an RL agent for treatment optimization may need thousands of interactions.
            \end{itemize}
        \end{itemize}
        \textbf{Key Point}: Improving sample efficiency can drastically reduce training time and resources.
    \end{block}
\end{frame}

\begin{frame}[fragile]
    \frametitle{Challenges in Reinforcement Learning - Convergence Issues and Computational Costs}
    \begin{block}{2. Convergence Issues}
        \begin{itemize}
            \item \textbf{Definition}: Process by which an RL algorithm approaches a stable policy/value function.
            \item \textbf{Challenges}:
            \begin{itemize}
                \item RL algorithms may oscillate or fail to converge in complex environments.
                \item Non-stationary environments can render learned policies outdated.
            \end{itemize}
            \item \textbf{Example}: 
            \begin{itemize}
                \item An RL chess agent may struggle to find optimal strategies against evolving opponents.
            \end{itemize}
        \end{itemize}
        \textbf{Key Point}: Reliable convergence is vital for deploying RL in dynamic scenarios.
    \end{block}
    
    \begin{block}{3. Computational Costs}
        \begin{itemize}
            \item \textbf{Definition}: Time and resources required for training RL models.
            \item \textbf{Challenges}:
            \begin{itemize}
                \item Many RL algorithms need substantial computational resources and time for training.
            \end{itemize}
            \item \textbf{Example}: 
            \begin{itemize}
                \item Training AlphaGo can take days on high-performance hardware.
            \end{itemize}
        \end{itemize}
        \textbf{Key Point}: Reducing computational costs enhances accessibility and scalability of RL techniques.
    \end{block}
\end{frame}

\begin{frame}[fragile]
    \frametitle{Challenges in Reinforcement Learning - Summary}
    \begin{itemize}
        \item Key challenges in reinforcement learning include:
        \begin{itemize}
            \item Sample Efficiency
            \item Convergence Issues
            \item Computational Costs
        \end{itemize}
        \item Understanding and addressing these challenges is crucial for:
        \begin{itemize}
            \item Developing more effective algorithms
            \item Enhancing practical applications
        \end{itemize}
        \item Future advancements in RL will depend on overcoming these obstacles.
    \end{itemize}
\end{frame}

\begin{frame}
    \frametitle{Future Directions in Reinforcement Learning}
    \begin{block}{Overview}
        Reinforcement Learning (RL) has made significant strides in recent years, yet it is still a rapidly evolving field. Exploring future directions allows researchers and practitioners to pinpoint where breakthroughs can occur and how RL can be applied even more effectively across various domains.
    \end{block}
\end{frame}

\begin{frame}
    \frametitle{Key Research Areas}
    \begin{enumerate}
        \item Sample Efficiency
        \item Generalization Across Tasks
        \item Exploration Strategies
        \item Real-world Applications
        \item Shaping Reward Functions
        \item Addressing Safety and Robustness
        \item Interpretability of RL Algorithms
    \end{enumerate}
\end{frame}

\begin{frame}
    \frametitle{Sample Efficiency}
    \begin{block}{Concept}
        Sample efficiency refers to the ability of a learning algorithm to achieve comparable performance with fewer samples from the environment.
    \end{block}
    \begin{itemize}
        \item \textbf{Future Direction:} Develop algorithms that can learn effectively with limited interaction.
        \item \textbf{Example:} In robotics, reducing the number of physical trials for training can save time and resources.
    \end{itemize}
\end{frame}

\begin{frame}
    \frametitle{Generalization Across Tasks}
    \begin{block}{Concept}
        The capacity of an RL agent to generalize knowledge from one task to another.
    \end{block}
    \begin{itemize}
        \item \textbf{Future Direction:} Investigate methods to enhance an agent’s ability to solve multiple tasks without starting from scratch each time.
        \item \textbf{Example:} A robot trained to stack blocks could generalize to manage a more complex task like sorting objects.
    \end{itemize}
\end{frame}

\begin{frame}
    \frametitle{Exploration Strategies}
    \begin{block}{Concept}
        Exploration involves the agent trying new actions to discover their effects, while exploitation involves using known information to maximize rewards.
    \end{block}
    \begin{itemize}
        \item \textbf{Future Direction:} Innovate on exploration-exploitation trade-offs for improved learning performance.
        \item \textbf{Example:} Using curiosity-driven exploration where the agent is internally motivated to explore states that appear novel.
    \end{itemize}
\end{frame}

\begin{frame}
    \frametitle{Real-world Applications}
    \begin{block}{Future Direction}
        Expand RL applications in fields like healthcare, finance, and autonomous systems.
    \end{block}
    \begin{itemize}
        \item \textbf{Example:} Personalized treatment plans in healthcare optimized using RL by considering patient histories and treatment responses.
    \end{itemize}
\end{frame}

\begin{frame}
    \frametitle{Shaping Reward Functions}
    \begin{block}{Concept}
        Designing appropriate reward functions is crucial for guiding agents towards desired behaviors.
    \end{block}
    \begin{itemize}
        \item \textbf{Future Direction:} Automated methods for reward shaping that derive effective rewards based on desired outcomes.
        \item \textbf{Example:} In game design, creating rewards that encourage specific play styles or strategies.
    \end{itemize}
\end{frame}

\begin{frame}
    \frametitle{Safety and Robustness}
    \begin{block}{Concept}
        Ensuring the reliability and safety of RL agents in unpredictable environments.
    \end{block}
    \begin{itemize}
        \item \textbf{Future Direction:} Research on developing safety mechanisms that prevent agents from taking harmful actions.
        \item \textbf{Example:} Self-driving cars that must adhere to traffic rules and avoid unsafe maneuvers.
    \end{itemize}
\end{frame}

\begin{frame}
    \frametitle{Interpretability of RL Algorithms}
    \begin{block}{Concept}
        Understanding why an RL agent makes certain decisions.
    \end{block}
    \begin{itemize}
        \item \textbf{Future Direction:} Create methods to make RL outcomes interpretable and explainable to foster trust and transparency.
        \item \textbf{Example:} In finance, explaining why a trading algorithm made a specific trade to evaluate its decisions effectively.
    \end{itemize}
\end{frame}

\begin{frame}[fragile]
    \frametitle{Code Snippet Example}
    \begin{lstlisting}[language=Python]
import numpy as np

class RLAgent:
    def __init__(self, actions):
        self.actions = actions
        self.q_values = np.zeros(len(actions))
    
    def update_q_values(self, action, reward):
        self.q_values[action] += reward
    
    def choose_action(self):
        return np.random.choice(self.actions)  # Exploration

# Future exploration methods will replace random choice with advanced strategies.
    \end{lstlisting}
\end{frame}

\begin{frame}
    \frametitle{Key Points to Emphasize}
    \begin{itemize}
        \item RL is poised for significant advancements across various domains.
        \item Enhancing sample efficiency, generalization, and safety are priority research areas.
        \item Real-world applications can greatly benefit from novel algorithms and techniques.
    \end{itemize}
\end{frame}

\begin{frame}[fragile]
    \frametitle{Ethical Considerations in Reinforcement Learning - Introduction}
    \begin{itemize}
        \item Reinforcement Learning (RL) enables agents to learn through feedback.
        \item As RL gains traction, ethical implications must be critically examined.
    \end{itemize}
\end{frame}

\begin{frame}[fragile]
    \frametitle{Ethical Considerations in Reinforcement Learning - Key Topics}
    \begin{enumerate}
        \item Bias and Fairness
        \item Exploration vs. Exploitation
        \item Accountability and Transparency
        \item Environmental Impact
        \item Human-AI Interaction
    \end{enumerate}
\end{frame}

\begin{frame}[fragile]
    \frametitle{Ethical Considerations in Reinforcement Learning - Detailed Points}
    \begin{block}{1. Bias and Fairness}
        RL systems can inherit biases from training data, leading to unfair outcomes.
        \begin{itemize}
            \item Example: A job recruitment algorithm trained on biased data may favor specific demographics.
        \end{itemize}
    \end{block}
    
    \begin{block}{2. Exploration vs. Exploitation}
        The balance between exploring new strategies and exploiting known ones comes with ethical implications.
        \begin{itemize}
            \item Example: An RL agent in healthcare may overlook untested treatments due to prioritizing known methods.
        \end{itemize}
    \end{block}
\end{frame}

\begin{frame}[fragile]
    \frametitle{Ethical Considerations in Reinforcement Learning - Continued}
    \begin{block}{3. Accountability and Transparency}
        The opaque decision-making process of RL agents complicates accountability.
        \begin{itemize}
            \item Example: Liability in accidents caused by RL-driven self-driving cars necessitates understanding the decision logic.
        \end{itemize}
    \end{block}

    \begin{block}{4. Environmental Impact}
        RL applications can lead to significant environmental consequences if not designed with sustainability in mind.
        \begin{itemize}
            \item Example: Traffic management algorithms could increase carbon emissions without sustainability goals.
        \end{itemize}
    \end{block}
\end{frame}

\begin{frame}[fragile]
    \frametitle{Ethical Considerations in Reinforcement Learning - Final Points}
    \begin{block}{5. Human-AI Interaction}
        Ethical design in human-AI interfaces is crucial to maintain trust.
        \begin{itemize}
            \item Example: RL-based chatbots in customer service must handle data ethically.
        \end{itemize}
    \end{block}
    
    \begin{block}{Key Points to Emphasize}
        \begin{itemize}
            \item Proactive Bias Mitigation
            \item Transparency in decision-making
            \item Incorporation of sustainability goals
            \item Responsible design for human-AI collaboration
        \end{itemize}
    \end{block}
\end{frame}

\begin{frame}[fragile]
    \frametitle{Ethical Considerations in Reinforcement Learning - Conclusion}
    Addressing ethical considerations in RL is essential for its responsible advancement. Engaging stakeholders like ethicists and policymakers ensures positive societal contributions of RL technologies.
\end{frame}

\begin{frame}[fragile]
    \frametitle{Conclusion - Reinforcement Learning}
    \begin{block}{Key Takeaways}
        \begin{enumerate}
            \item \textbf{Definition and Framework:}
            \begin{itemize}
                \item RL involves learning to make decisions to maximize cumulative rewards.
                \item Modeled using Markov Decision Process (MDP) with states (S), actions (A), rewards (R), and policies ($\pi$).
            \end{itemize}
            \item \textbf{Significance in AI:}
            \begin{itemize}
                \item Solves complex problems in gaming, robotics, and autonomous systems.
                \item Facilitates advancements in various fields like healthcare and finance.
            \end{itemize}
        \end{enumerate}
    \end{block}
\end{frame}

\begin{frame}[fragile]
    \frametitle{Conclusion - Reinforcement Learning (cont.)}
    \begin{block}{Formula}
        The goal is to maximize the expected return (cumulative reward):
        \begin{equation}
            G_t = R_{t+1} + \gamma R_{t+2} + \gamma^2 R_{t+3} + \ldots
        \end{equation}
        Where $\gamma$ (0 ≤ $\gamma$ < 1) is the discount factor for balancing immediate and future rewards.
    \end{block}

    \begin{block}{Advancements and Challenges}
        \begin{itemize}
            \item Breakthroughs in Deep Reinforcement Learning for impressive results in image and speech recognition.
            \item Challenges include sample inefficiency, data requirements, and ethical considerations.
        \end{itemize}
    \end{block}
\end{frame}

\begin{frame}[fragile]
    \frametitle{Conclusion - Reinforcement Learning (final thoughts)}
    \begin{block}{Future Implications}
        \begin{itemize}
            \item RL is expected to play a critical role in the evolution of AI, leading to adaptive systems that learn autonomously.
            \item Integration of RL into AI promises to revolutionize industries and enhance decision-making processes.
        \end{itemize}
    \end{block}

    \begin{block}{Illustrative Example}
        Consider an AI agent learning to play chess: Each board configuration is a state. The agent explores moves (actions), receives rewards for winning moves, and learns optimal strategies over time.
    \end{block}

    \begin{block}{Conclusion}
        Reinforcement Learning stands at the forefront of AI research and applications, carrying a responsibility to address ethical considerations and ensure safe implementations.
    \end{block}
\end{frame}

\begin{frame}[fragile]
    \frametitle{Q\&A on Reinforcement Learning}
    \begin{block}{Understanding Reinforcement Learning (RL)}
        Reinforcement Learning is a subset of machine learning where agents learn to make decisions by interacting with an environment. They receive rewards or penalties based on their actions and use these signals to improve their future performance.
    \end{block}
    \begin{itemize}
        \item \textbf{Agent}: The learner or decision-maker.
        \item \textbf{Environment}: The agent's surroundings, which provide feedback.
        \item \textbf{Actions}: Choices made by the agent.
        \item \textbf{State}: Current situation of the agent in the environment.
        \item \textbf{Reward}: Feedback from the environment (positive or negative).
    \end{itemize}
\end{frame}

\begin{frame}[fragile]
    \frametitle{Key Concepts to Discuss}
    \begin{itemize}
        \item \textbf{Exploration vs. Exploitation}:
        \begin{itemize}
            \item \textbf{Exploration}: Trying new actions to discover effects.
            \item \textbf{Exploitation}: Selecting best-known actions based on past experiences.
        \end{itemize}
        \item \textbf{Markov Decision Process (MDP)}:
        \begin{itemize}
            \item A framework for modeling decision-making, defined by states \( S \), actions \( A \), transition probabilities \( P \), and rewards \( R \).
            \item \textbf{Expected Reward Formula}:
            \[
            R(s, a) = \sum_{s'} P(s'|s, a) \cdot R(s, a, s')
            \]
        \end{itemize}
        \item \textbf{Value Functions}:
        \begin{itemize}
            \item \textbf{State Value Function \( V(s) \)}: Expected return from state \( s \).
            \item \textbf{Action Value Function \( Q(s, a) \)}: Expected return of action \( a \) in state \( s \).
            \item \textbf{Formula}:
            \[
            Q(s, a) = R(s, a) + \gamma \sum_{s'} P(s'|s, a) V(s')
            \]
        \end{itemize}
    \end{itemize}
\end{frame}

\begin{frame}[fragile]
    \frametitle{Applications and Open Discussion}
    \begin{itemize}
        \item \textbf{Applications}:
        \begin{itemize}
            \item Game Playing: RL in agents like AlphaGo and Atari games.
            \item Robotics: Teaching robots through trial and error.
            \item Self-driving Cars: Navigating complex environments based on feedback.
        \end{itemize}
        \item \textbf{Key Points to Emphasize}:
        \begin{itemize}
            \item Balance between exploration and exploitation is crucial.
            \item The learning process is iterative.
            \item Real-world applications showcase RL's capabilities.
        \end{itemize}
    \end{itemize}
    \textbf{Open Floor for Questions:} Please feel free to ask about specific RL algorithms, challenges, ethical considerations, or future directions in this exciting field.
\end{frame}


\end{document}