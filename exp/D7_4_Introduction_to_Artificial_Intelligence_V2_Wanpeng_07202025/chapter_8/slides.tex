\documentclass[aspectratio=169]{beamer}

% Theme and Color Setup
\usetheme{Madrid}
\usecolortheme{whale}
\useinnertheme{rectangles}
\useoutertheme{miniframes}

% Additional Packages
\usepackage[utf8]{inputenc}
\usepackage[T1]{fontenc}
\usepackage{graphicx}
\usepackage{booktabs}
\usepackage{listings}
\usepackage{amsmath}
\usepackage{amssymb}
\usepackage{xcolor}
\usepackage{tikz}
\usepackage{pgfplots}
\pgfplotsset{compat=1.18}
\usetikzlibrary{positioning}
\usepackage{hyperref}

% Custom Colors
\definecolor{myblue}{RGB}{31, 73, 125}
\definecolor{mygray}{RGB}{100, 100, 100}
\definecolor{mygreen}{RGB}{0, 128, 0}
\definecolor{myorange}{RGB}{230, 126, 34}
\definecolor{mycodebackground}{RGB}{245, 245, 245}

% Set Theme Colors
\setbeamercolor{structure}{fg=myblue}
\setbeamercolor{frametitle}{fg=white, bg=myblue}
\setbeamercolor{title}{fg=myblue}
\setbeamercolor{section in toc}{fg=myblue}
\setbeamercolor{item projected}{fg=white, bg=myblue}
\setbeamercolor{block title}{bg=myblue!20, fg=myblue}
\setbeamercolor{block body}{bg=myblue!10}
\setbeamercolor{alerted text}{fg=myorange}

% Set Fonts
\setbeamerfont{title}{size=\Large, series=\bfseries}
\setbeamerfont{frametitle}{size=\large, series=\bfseries}
\setbeamerfont{caption}{size=\small}
\setbeamerfont{footnote}{size=\tiny}

% Code Listing Style
\lstdefinestyle{customcode}{
  backgroundcolor=\color{mycodebackground},
  basicstyle=\footnotesize\ttfamily,
  breakatwhitespace=false,
  breaklines=true,
  commentstyle=\color{mygreen}\itshape,
  keywordstyle=\color{blue}\bfseries,
  stringstyle=\color{myorange},
  numbers=left,
  numbersep=8pt,
  numberstyle=\tiny\color{mygray},
  frame=single,
  framesep=5pt,
  rulecolor=\color{mygray},
  showspaces=false,
  showstringspaces=false,
  showtabs=false,
  tabsize=2,
  captionpos=b
}
\lstset{style=customcode}

% Custom Commands
\newcommand{\hilight}[1]{\colorbox{myorange!30}{#1}}
\newcommand{\source}[1]{\vspace{0.2cm}\hfill{\tiny\textcolor{mygray}{Source: #1}}}
\newcommand{\concept}[1]{\textcolor{myblue}{\textbf{#1}}}
\newcommand{\separator}{\begin{center}\rule{0.5\linewidth}{0.5pt}\end{center}}

% Footer and Navigation Setup
\setbeamertemplate{footline}{
  \leavevmode%
  \hbox{%
  \begin{beamercolorbox}[wd=.3\paperwidth,ht=2.25ex,dp=1ex,center]{author in head/foot}%
    \usebeamerfont{author in head/foot}\insertshortauthor
  \end{beamercolorbox}%
  \begin{beamercolorbox}[wd=.5\paperwidth,ht=2.25ex,dp=1ex,center]{title in head/foot}%
    \usebeamerfont{title in head/foot}\insertshorttitle
  \end{beamercolorbox}%
  \begin{beamercolorbox}[wd=.2\paperwidth,ht=2.25ex,dp=1ex,center]{date in head/foot}%
    \usebeamerfont{date in head/foot}
    \insertframenumber{} / \inserttotalframenumber
  \end{beamercolorbox}}%
  \vskip0pt%
}

% Turn off navigation symbols
\setbeamertemplate{navigation symbols}{}

% Title Page Information
\title[Propositional Logic]{Week 8: Propositional Logic}
\author[J. Smith]{John Smith, Ph.D.}
\institute[University Name]{
  Department of Computer Science\\
  University Name\\
  \vspace{0.3cm}
  Email: email@university.edu\\
  Website: www.university.edu
}
\date{\today}

% Document Start
\begin{document}

\frame{\titlepage}

\begin{frame}[fragile]
    \frametitle{Introduction to Propositional Logic}
    \begin{block}{Overview}
        Propositional logic is a branch of logic dealing with propositions that can be true or false. It forms a foundational aspect of reasoning in mathematics, computer science, and artificial intelligence (AI).
    \end{block}
\end{frame}

\begin{frame}[fragile]
    \frametitle{What is a Proposition?}
    \begin{itemize}
        \item A proposition is a declarative statement with a definite truth value: \textbf{True (T)} or \textbf{False (F)}.
        \item \textbf{Examples}:
        \begin{itemize}
            \item "The sky is blue." (Can be true or false)
            \item "2 + 2 = 4." (This is always true)
        \end{itemize}
    \end{itemize}
\end{frame}

\begin{frame}[fragile]
    \frametitle{Key Components of Propositional Logic}
    \begin{enumerate}
        \item Propositions: The basic units of propositional logic.
        \item Logical Connectives: Operators combining propositions into complex statements.
            \begin{itemize}
                \item AND (\(\land\))
                \item OR (\(\lor\))
                \item NOT (\(\neg\))
                \item IMPLIES (\(\rightarrow\))
                \item BICONDITIONAL (\(\leftrightarrow\))
            \end{itemize}
    \end{enumerate}
\end{frame}

\begin{frame}[fragile]
    \frametitle{Examples of Logical Connectives}
    \begin{block}{Conjunction (AND)}
        \begin{itemize}
            \item Statement: "It is raining AND it is cold."
            \item Symbol: \( P \land Q \)
        \end{itemize}
        \begin{center}
            \begin{tabular}{|c|c|c|}
                \hline
                P & Q & P ∧ Q \\
                \hline
                T & T & T \\
                T & F & F \\
                F & T & F \\
                F & F & F \\
                \hline
            \end{tabular}
        \end{center}
    \end{block}

    \begin{block}{Disjunction (OR)}
        \begin{itemize}
            \item Statement: "It is raining OR it is cold."
            \item Symbol: \( P \lor Q \)
        \end{itemize}
        \begin{center}
            \begin{tabular}{|c|c|c|}
                \hline
                P & Q & P ∨ Q \\
                \hline
                T & T & T \\
                T & F & T \\
                F & T & T \\
                F & F & F \\
                \hline
            \end{tabular}
        \end{center}
    \end{block}
\end{frame}

\begin{frame}[fragile]
    \frametitle{Significance of Propositional Logic in AI}
    \begin{itemize}
        \item \textbf{Decision Making}: It provides a framework for systems to make decisions based on input combinations.
        \item \textbf{Knowledge Representation}: Enables AI systems to formalize knowledge and reasoning.
        \item \textbf{Algorithm Development}: Many AI algorithms, such as search and rule-based systems, rely on logical reasoning.
    \end{itemize}
\end{frame}

\begin{frame}[fragile]
    \frametitle{Key Points and Conclusion}
    \begin{itemize}
        \item Propositional logic is foundational in AI and computer science.
        \item Understanding logical connectives and propositions is crucial for complex logical constructs.
        \item It is essential for reasoning about the truth of statements and their implications.
    \end{itemize}

    \begin{block}{Conclusion}
        Propositional logic underlies AI reasoning and decision-making. Mastering its principles allows for better understanding of how AI analyzes and interprets information.
    \end{block}
\end{frame}

\begin{frame}[fragile]
    \frametitle{Historical Context - Overview}
    \begin{itemize}
        \item Logic: The study of reasoning and argumentation.
        \item Serves as the backbone of:
        \begin{itemize}
            \item Mathematics
            \item Philosophy
            \item Computer Science
        \end{itemize}
    \end{itemize}
\end{frame}

\begin{frame}[fragile]
    \frametitle{Historical Context - Development of Logic}
    \begin{enumerate}
        \item \textbf{Ancient Beginnings}
        \begin{itemize}
            \item Aristotle (384–322 BC): Father of logic; developed syllogistic logic.
            \item Stoics: Introduced conditional statements.
        \end{itemize}
        
        \item \textbf{Medieval Enhancements}
        \begin{itemize}
            \item Scholasticism (12th-17th Century): Used logic for rigorous argument analysis.
        \end{itemize}
        
        \item \textbf{19th Century Formalization}
        \begin{itemize}
            \item George Boole: Introduced Boolean algebra.
            \item Gottlob Frege: Modern logic transformation in "Begriffsschrift".
        \end{itemize}
        
        \item \textbf{20th Century Advances}
        \begin{itemize}
            \item Whitehead and Russell: "Principia Mathematica" expanded propositional logic.
            \item Role in Computer Science: Essential for AI development.
        \end{itemize}
    \end{enumerate}
\end{frame}

\begin{frame}[fragile]
    \frametitle{Historical Context - Key Concepts and Examples}
    \begin{block}{Key Concepts}
        \begin{itemize}
            \item Syllogism: Argument structure with two premises leading to a conclusion.
            \item Boolean Algebra: Mathematical structure for operating on propositions.
            \item Propositional Logic: Logic branch combining propositions using operators.
        \end{itemize}
    \end{block}
    
    \begin{block}{Examples}
        \begin{itemize}
            \item \textbf{Propositions}: Statements that are either true or false.
            \begin{itemize}
                \item Example: "The sky is blue." (True/False)
            \end{itemize}
            \item \textbf{Logical Operators}:
            \begin{itemize}
                \item AND (\land): True if both propositions are true.\\
                      Example: "Today is Monday AND it is raining."
                \item OR (\lor): True if at least one proposition is true.\\
                      Example: "It is sunny OR it is raining."
                \item NOT (\neg): True if the proposition is false.\\
                      Example: "It is NOT raining."
            \end{itemize}
        \end{itemize}
    \end{block}
\end{frame}

\begin{frame}[fragile]
    \frametitle{Fundamentals of Logic - Introduction}
    \begin{block}{Introduction to Logic}
        Logic is the systematic study of valid reasoning, a foundational element in mathematics, philosophy, and computational theory. 
        It allows us to differentiate between valid and invalid arguments, forming the backbone of formal reasoning.
    \end{block}
\end{frame}

\begin{frame}[fragile]
    \frametitle{Fundamentals of Logic - Key Definitions}
    \begin{enumerate}
        \item \textbf{Proposition:}
            \begin{itemize}
                \item A declarative statement that is either true or false, but not both.
                \item \textit{Example:} "The sky is blue." (True or False)
            \end{itemize}

        \item \textbf{Logical Connectives:}
            \begin{itemize}
                \item AND (\(\land\)): True if both propositions are true.
                \item OR (\(\lor\)): True if at least one proposition is true.
                \item NOT (\(\neg\)): Inverts the truth value of a proposition.
                \item IMPLICATION (\(\rightarrow\)): If the first proposition is true, then the second must be true.
                \item BICONDITIONAL (\(\leftrightarrow\)): True if both propositions are either true or false.
            \end{itemize}

        \item \textbf{Truth Values:}
            \begin{itemize}
                \item Assigning truth values (True/False) to propositions helps in evaluating logical expressions.
            \end{itemize}
    \end{enumerate}
\end{frame}

\begin{frame}[fragile]
    \frametitle{Fundamentals of Logic - Importance and Examples}
    \begin{block}{Importance of Logic in Computational Theory}
        \begin{itemize}
            \item \textbf{Reasoning \& Inference:} Logic supports the development of algorithms and formal verification methods, ensuring the reliability of software and systems.
            \item \textbf{Decision Making:} Aids in making decisions based on conditions, fundamental in programming.
            \item \textbf{Artificial Intelligence:} Forms the basis for knowledge representation, enabling machines to process and deduce information.
        \end{itemize}
    \end{block}
    
    \begin{block}{Examples of Logical Statements}
        \begin{enumerate}
            \item \textbf{Conjunction:} 
                \begin{itemize}
                    \item Let \( P \): "It is raining."
                    \item Let \( Q \): "I will take an umbrella."
                    \item Statement: \( P \land Q \) (True if both P and Q are true)
                \end{itemize}

            \item \textbf{Disjunction:}
                \begin{itemize}
                    \item Statement: \( P \lor Q \) (True if at least one is true)
                \end{itemize}

            \item \textbf{Negation:}
                \begin{itemize}
                    \item Statement: \( \neg P \) (True if P is false)
                \end{itemize}
        \end{enumerate}
    \end{block}
\end{frame}

\begin{frame}[fragile]
    \frametitle{Syntax of Propositional Logic - Overview}
    \begin{block}{Learning Objectives}
        \begin{itemize}
            \item Understand the fundamental syntax rules governing propositional statements.
            \item Identify and utilize the basic components of propositional logic.
            \item Construct valid propositional expressions using logical operators.
        \end{itemize}
    \end{block}
    
    \begin{block}{What is Propositional Logic?}
        Propositional logic, also known as sentential logic, deals with propositions, which are statements that can be either true or false.
    \end{block}
\end{frame}

\begin{frame}[fragile]
    \frametitle{Key Components of Propositional Logic Syntax}
    \begin{enumerate}
        \item \textbf{Propositions}
        \begin{itemize}
            \item A declarative sentence that is either true or false.
            \item Examples:
                \begin{itemize}
                    \item "The sky is blue." (True)
                    \item "5 is greater than 10." (False)
                \end{itemize}
        \end{itemize}
        
        \item \textbf{Logical Variables}
        \begin{itemize}
            \item Representations such as \( P, Q, R, \) etc.
            \item Examples:
                \begin{itemize}
                    \item \( P \): "It is raining."
                    \item \( Q \): "I will take an umbrella."
                \end{itemize}
        \end{itemize}
        
        \item \textbf{Logical Connectives}
        \begin{itemize}
            \item AND (\(\land\)): True if both propositions are true.
            \item OR (\(\lor\)): True if at least one proposition is true.
            \item NOT (\(\neg\)): Inverts the truth value.
            \item IMPLIES (\(\rightarrow\)): Conditional relationship.
            \item IFF (\(\leftrightarrow\)): True if both propositions are the same.
        \end{itemize}
    \end{enumerate}
\end{frame}

\begin{frame}[fragile]
    \frametitle{Syntax Rules for Formulating Propositional Expressions}
    \begin{enumerate}
        \item \textbf{Formation}
        \begin{itemize}
            \item Any logical variable (e.g., \( P, Q, R \)) is valid.
            \item If \( A \) and \( B \) are valid, then \( A \land B, A \lor B, \neg A, A \rightarrow B, A \leftrightarrow B \) are also valid.
        \end{itemize}
        
        \item \textbf{Parentheses}
        \begin{itemize}
            \item Use to indicate precedence and clarify structure.
            \item Example: \( (P \lor Q) \land R \) is different from \( P \lor (Q \land R) \).
        \end{itemize}
        
        \item \textbf{Expression Rules}
        \begin{itemize}
            \item Order of operations: NOT, AND, OR, IMPLIES, IFF.
        \end{itemize}
    \end{enumerate}
    
    \begin{block}{Illustrative Example}
        Consider \( P \): "It is sunny." and \( Q \): "I will go for a walk."
        The expression \( (P \land Q) \lor \neg P \) translates to:
        "It is sunny and I will go for a walk, or it is not sunny."
    \end{block}
\end{frame}

\begin{frame}[fragile]
    \frametitle{Key Points to Remember}
    \begin{itemize}
        \item Propositional logic focuses on the syntax and structure of logical statements.
        \item Understanding components and syntax rules aids in forming and interpreting expressions.
        \item Logical operations combine simple propositions to create complex statements.
    \end{itemize}
    
    \begin{block}{Conclusion}
        Grasping the syntax of propositional logic lays a solid foundation for advanced logical reasoning.
    \end{block}
\end{frame}

\begin{frame}[fragile]
  \frametitle{Semantics of Propositional Logic}
  \begin{block}{Learning Objectives}
    \begin{itemize}
      \item Understand the meaning of propositions.
      \item Explore truth values associated with propositions.
      \item Analyze the implications of truth values in logical reasoning.
    \end{itemize}
  \end{block}
\end{frame}

\begin{frame}[fragile]
  \frametitle{Understanding Propositions}
  \begin{itemize}
    \item A \textbf{proposition} is a declarative statement that is either \textbf{true} or \textbf{false}, but not both.
    \item Examples:
      \begin{itemize}
        \item “The sky is blue.” (Context-dependent)
        \item “2 + 2 = 4.” (Always true)
      \end{itemize}
  \end{itemize}
\end{frame}

\begin{frame}[fragile]
  \frametitle{Truth Values and Logical Operations}
  \begin{itemize}
    \item The \textbf{truth value} of a proposition indicates its truthfulness.
      \begin{itemize}
        \item \textbf{True (T)}: Accurately describes reality.
        \item \textbf{False (F)}: Inaccurately describes reality.
      \end{itemize}
  
    \item Logical operations and their semantics:
      \begin{itemize}
        \item \textbf{Negation (¬)}: Inverts the truth value.
        \item \textbf{Conjunction (∧)}: True if both propositions are true.
        \item \textbf{Disjunction (∨)}: True if at least one proposition is true.
      \end{itemize}
  \end{itemize}
\end{frame}

\begin{frame}[fragile]
  \frametitle{Truth Tables}
  Truth tables visually represent the semantics of propositional logic. They list each possible combination of truth values for the propositions and the result of the logical operations.

  \begin{block}{Example: Truth Table for Conjunction ($P \land Q$)}
  \begin{tabular}{|c|c|c|}
    \hline
    $P$ & $Q$ & $P \land Q$ \\
    \hline
    T   & T   & T    \\
    T   & F   & F    \\
    F   & T   & F    \\
    F   & F   & F    \\
    \hline
  \end{tabular}
  \end{block}
\end{frame}

\begin{frame}[fragile]
  \frametitle{Conclusion}
  \begin{itemize}
    \item Understanding the semantics of propositional logic is crucial as it lays the foundation for constructing logical arguments.
    \item It enables the utilization of truth values and analysis of logical structure in various applications.
  \end{itemize}

  \textbf{Next up:} Truth Tables – Starting to evaluate propositions meaningfully.
\end{frame}

\begin{frame}[fragile]
    \frametitle{Truth Tables - Introduction}
    \begin{block}{Introduction to Truth Tables}
        Truth tables are a fundamental tool in propositional logic used to evaluate the truth values of logical expressions based on the truth values of their individual propositions. They provide a systematic method for analyzing complex statements by listing all possible combinations of truth values and determining the truth value of the expression for each combination.
    \end{block}
\end{frame}

\begin{frame}[fragile]
    \frametitle{Truth Tables - Key Concepts}
    \begin{block}{Key Concepts}
        \begin{itemize}
            \item \textbf{Propositions:} Statements that are either true (T) or false (F).
            \item \textbf{Logical Operators:} Connect propositions to form compound statements. Common operators include:
            \begin{itemize}
                \item AND ($\land$)
                \item OR ($\lor$)
                \item NOT ($\neg$)
                \item Implication ($\to$)
                \item Biconditional ($\leftrightarrow$)
            \end{itemize}
        \end{itemize}
    \end{block}
\end{frame}

\begin{frame}[fragile]
    \frametitle{Truth Tables - Structure and Example}
    \begin{block}{Structure of a Truth Table}
        A truth table includes:
        \begin{itemize}
            \item Rows representing all possible combinations of truth values for the propositions involved.
            \item Columns for each individual proposition and each possible evaluation of the expression.
        \end{itemize}
    \end{block}
    
    \begin{block}{Example: Truth Table for \( P \land Q \)}
        \textbf{Statement:} $P \land Q$ (P AND Q)

        \begin{center}
        \begin{tabular}{|c|c|c|}
            \hline
            P & Q & P $\land$ Q \\
            \hline
            T & T & T \\
            T & F & F \\
            F & T & F \\
            F & F & F \\
            \hline
        \end{tabular}
        \end{center}
    \end{block}
\end{frame}

\begin{frame}[fragile]
    \frametitle{Truth Tables - Analysis and Logical Connectives}
    \begin{block}{Analysis}
        - **True (T)** only when both \( P \) and \( Q \) are true.
        - Understanding the combination of truth values aids in logical reasoning.
    \end{block}

    \begin{block}{Formulae for Logical Connectives}
        \begin{enumerate}
            \item \textbf{AND:} $P \land Q$ = T only if both $P$ = T and $Q$ = T.
            \item \textbf{OR:} $P \lor Q$ = T if at least one of $P$ or $Q$ is T.
            \item \textbf{NOT:} $\neg P$ = T if $P$ is F, and vice versa.
            \item \textbf{Implication:} $P \to Q$ = F only if $P$ is T and $Q$ is F.
            \item \textbf{Biconditional:} $P \leftrightarrow Q$ = T if both 
            $P$ and $Q$ have the same truth values.
        \end{enumerate}
    \end{block}
\end{frame}

\begin{frame}[fragile]
    \frametitle{Truth Tables - Key Points and Conclusion}
    \begin{block}{Key Points to Emphasize}
        \begin{itemize}
            \item Truth tables are invaluable for verifying the validity of logical arguments.
            \item They illustrate complex logical relationships and help determine logical equivalence.
            \item Digital circuits and programming heavily utilize truth tables for decision-making processes.
        \end{itemize}
    \end{block}
    
    \begin{block}{Conclusion}
        Understanding truth tables lays the groundwork for exploring propositional logic, such as analyzing complex statements using logical connectives. As we progress, we will delve deeper into these logical connectives to enhance our understanding of propositional expressions.
    \end{block}
\end{frame}

\begin{frame}[fragile]
    \frametitle{Logical Connectives}
    \begin{block}{Learning Objectives}
        \begin{itemize}
            \item Understand the definition and function of key logical connectives.
            \item Learn how to evaluate expressions using truth tables.
            \item Apply logical connectives in constructing and analyzing propositions.
        \end{itemize}
    \end{block}
\end{frame}

\begin{frame}[fragile]
    \frametitle{What are Logical Connectives?}
    Logical connectives are operators that combine one or more propositions, enabling us to construct complex logical statements. They form the foundation of propositional logic.
\end{frame}

\begin{frame}[fragile]
    \frametitle{AND (Conjunction)}
    \begin{block}{Definition}
        \begin{itemize}
            \item \textbf{Symbol:} $\land$
            \item \textbf{Meaning:} The conjunction of two propositions is true only if both propositions are true.
        \end{itemize}
    \end{block}
    \begin{block}{Truth Table}
        \begin{center}
            \begin{tabular}{|c|c|c|}
                \hline
                P & Q & P $\land$ Q \\
                \hline
                T & T & T \\
                T & F & F \\
                F & T & F \\
                F & F & F \\
                \hline
            \end{tabular}
        \end{center}
    \end{block}
    \begin{block}{Example}
        P: "It is raining." \\
        Q: "I will take an umbrella." \\
        The statement "It is raining AND I will take an umbrella" (P $\land$ Q) is only true if both conditions are true.
    \end{block}
\end{frame}

\begin{frame}[fragile]
    \frametitle{OR (Disjunction)}
    \begin{block}{Definition}
        \begin{itemize}
            \item \textbf{Symbol:} $\lor$
            \item \textbf{Meaning:} The disjunction of two propositions is true if at least one of the propositions is true.
        \end{itemize}
    \end{block}
    \begin{block}{Truth Table}
        \begin{center}
            \begin{tabular}{|c|c|c|}
                \hline
                P & Q & P $\lor$ Q \\
                \hline
                T & T & T \\
                T & F & T \\
                F & T & T \\
                F & F & F \\
                \hline
            \end{tabular}
        \end{center}
    \end{block}
    \begin{block}{Example}
        P: "I will eat pizza." \\
        Q: "I will eat salad." \\
        The statement "I will eat pizza OR I will eat salad" (P $\lor$ Q) is true if I eat either (or both).
    \end{block}
\end{frame}

\begin{frame}[fragile]
    \frametitle{NOT (Negation)}
    \begin{block}{Definition}
        \begin{itemize}
            \item \textbf{Symbol:} $\neg$
            \item \textbf{Meaning:} The negation of a proposition is true if the proposition is false, and vice versa.
        \end{itemize}
    \end{block}
    \begin{block}{Truth Table}
        \begin{center}
            \begin{tabular}{|c|c|}
                \hline
                P & $\neg$P \\
                \hline
                T & F \\
                F & T \\
                \hline
            \end{tabular}
        \end{center}
    \end{block}
    \begin{block}{Example}
        P: "Today is Monday." \\
        The statement "It is NOT true that today is Monday" ($\neg$P) is true if today is any day that is not Monday.
    \end{block}
\end{frame}

\begin{frame}[fragile]
    \frametitle{Implication (Conditional)}
    \begin{block}{Definition}
        \begin{itemize}
            \item \textbf{Symbol:} $\rightarrow$
            \item \textbf{Meaning:} An implication states that if proposition P is true, then proposition Q is also true. It is false only when P is true and Q is false.
        \end{itemize}
    \end{block}
    \begin{block}{Truth Table}
        \begin{center}
            \begin{tabular}{|c|c|c|}
                \hline
                P & Q & P $\rightarrow$ Q \\
                \hline
                T & T & T \\
                T & F & F \\
                F & T & T \\
                F & F & T \\
                \hline
            \end{tabular}
        \end{center}
    \end{block}
    \begin{block}{Example}
        P: "It is 5 PM." \\
        Q: "I will go home." \\
        The statement "If it is 5 PM, then I will go home" (P $\rightarrow$ Q) is false only if it is indeed 5 PM but I do not go home.
    \end{block}
\end{frame}

\begin{frame}[fragile]
    \frametitle{Biconditional (If and Only If)}
    \begin{block}{Definition}
        \begin{itemize}
            \item \textbf{Symbol:} $\leftrightarrow$
            \item \textbf{Meaning:} A biconditional statement is true when both propositions are either true or both are false.
        \end{itemize}
    \end{block}
    \begin{block}{Truth Table}
        \begin{center}
            \begin{tabular}{|c|c|c|}
                \hline
                P & Q & P $\leftrightarrow$ Q \\
                \hline
                T & T & T \\
                T & F & F \\
                F & T & F \\
                F & F & T \\
                \hline
            \end{tabular}
        \end{center}
    \end{block}
    \begin{block}{Example}
        P: "You win." \\
        Q: "You receive a prize." \\
        The statement "You win if and only if you receive a prize" (P $\leftrightarrow$ Q) is true when both events occur or neither occurs.
    \end{block}
\end{frame}

\begin{frame}[fragile]
    \frametitle{Key Points to Remember}
    \begin{itemize}
        \item \textbf{Conjunction (AND)} requires both propositions to be true.
        \item \textbf{Disjunction (OR)} requires at least one proposition to be true.
        \item \textbf{Negation (NOT)} reverses the truth value.
        \item \textbf{Implication ($\rightarrow$)} establishes a cause-effect relationship between two propositions.
        \item \textbf{Biconditional ($\leftrightarrow$)} indicates that both propositions are equivalently true or false.
    \end{itemize}
    Understanding these logical connectives is essential for constructing and analyzing complex propositions in propositional logic.
\end{frame}

\begin{frame}[fragile]
    \frametitle{Constructing Propositions}
    \begin{block}{Learning Objectives}
        \begin{itemize}
            \item Understand how to combine simple propositions using logical connectives.
            \item Construct complex propositions and analyze their meanings.
            \item Utilize truth tables to evaluate the validity of complex propositions.
        \end{itemize}
    \end{block}
\end{frame}

\begin{frame}[fragile]
    \frametitle{Introduction to Propositions}
    \begin{block}{Definition}
        A proposition is a declarative statement that is either \textbf{true} or \textbf{false}, but not both. In propositional logic, we use \textbf{logical connectives} to form complex propositions from simpler ones.
    \end{block}
\end{frame}

\begin{frame}[fragile]
    \frametitle{Logical Connectives}
    From the previous slide, we learned about the following logical connectives:
    \begin{itemize}
        \item \textbf{AND ( $\land$ )}: True if both propositions are true.
        \item \textbf{OR ( $\lor$ )}: True if at least one proposition is true.
        \item \textbf{NOT ( $\neg$ )}: Inverts the truth value of a proposition.
        \item \textbf{Implication ( $\to$ )}: True unless a true proposition implies a false one.
        \item \textbf{Biconditional ( $\leftrightarrow$ )}: True if both propositions have the same truth value.
    \end{itemize}
\end{frame}

\begin{frame}[fragile]
    \frametitle{Constructing Complex Propositions}
    To construct a complex proposition, follow these steps:
    
    \begin{enumerate}
        \item \textbf{Identify Simple Propositions:}
        \begin{itemize}
            \item Let \( p \): "It is raining."
            \item Let \( q \): "I will take an umbrella."
        \end{itemize}
        
        \item \textbf{Choose Logical Connectives to Combine:}
        \begin{itemize}
            \item Using \textbf{AND}: \( p \land q \) means "It is raining AND I will take an umbrella."
            \item Using \textbf{OR}: \( p \lor q \) means "It is raining OR I will take an umbrella."
            \item Using \textbf{NOT}: \( \neg p \) means "It is NOT raining."
            \item Using \textbf{Implication}: \( p \to q \) means "If it is raining, then I will take an umbrella."
            \item Using \textbf{Biconditional}: \( p \leftrightarrow q \) means "It is raining if and only if I will take an umbrella."
        \end{itemize}
    \end{enumerate}
\end{frame}

\begin{frame}[fragile]
    \frametitle{Examples of Complex Propositions}
    \begin{enumerate}
        \item \textbf{Example 1}:
        \begin{itemize}
            \item Statement: "If it is raining, I will take an umbrella and it is cold."
            \item Representation: \( p \to (q \land r) \) where \( r \) is "It is cold."
        \end{itemize}

        \item \textbf{Example 2}:
        \begin{itemize}
            \item Statement: "Either it is not raining or I have a raincoat."
            \item Representation: \( \neg p \lor s \) where \( s \) is "I have a raincoat."
        \end{itemize}
    \end{enumerate}
\end{frame}

\begin{frame}[fragile]
    \frametitle{Truth Tables}
    To evaluate these propositions, we can create \textbf{truth tables} illustrating the combinations of truth values for \( p, q, \) and \( r \).

    For instance, consider \( p \land q \):

    \begin{center}
    \begin{tabular}{|c|c|c|}
        \hline
        \( p \) & \( q \) & \( p \land q \) \\
        \hline
        T       & T       & T                \\
        T       & F       & F                \\
        F       & T       & F                \\
        F       & F       & F                \\
        \hline
    \end{tabular}
    \end{center}
\end{frame}

\begin{frame}[fragile]
    \frametitle{Key Points to Remember}
    \begin{itemize}
        \item Complex propositions can be formed using logical connectives to convey more intricate statements.
        \item Correctly identifying and applying connectives is essential to constructing meaningful propositions.
        \item Evaluating the truth of complex propositions often requires the use of truth tables to analyze combinations of truth values.
    \end{itemize}
\end{frame}

\begin{frame}[fragile]
    \frametitle{Equivalence of Propositions}
    \begin{block}{Learning Objectives}
        \begin{itemize}
            \item \textbf{Understand} the concept of equivalent propositions.
            \item \textbf{Utilize} truth tables to determine equivalence.
            \item \textbf{Apply} laws of logic to demonstrate propositional equivalence.
        \end{itemize}
    \end{block}
\end{frame}

\begin{frame}[fragile]
    \frametitle{Introduction}
    In propositional logic, two propositions are considered \textbf{equivalent} if they yield the same truth value in all possible scenarios. Recognizing equivalent propositions is crucial for simplifying logical expressions and ensuring consistent reasoning.
\end{frame}

\begin{frame}[fragile]
    \frametitle{Key Concepts: Equivalent Propositions}
    \begin{block}{Equivalent Propositions}
        Two propositions \( P \) and \( Q \) are equivalent (denoted as \( P \equiv Q \)):
        \begin{itemize}
            \item \( P \) is true whenever \( Q \) is true, and vice versa.
            \item Example: \( P: \) "It is raining" and \( Q: \) "The ground is wet" could be considered logically equivalent under certain conditions.
        \end{itemize}
    \end{block}
\end{frame}

\begin{frame}[fragile]
    \frametitle{Truth Tables}
    \begin{block}{Truth Tables}
        A truth table allows us to systematically explore all possible truth values of propositions. 
        To determine if \( P \) and \( Q \) are equivalent, we create and compare truth tables.
    \end{block}
    \begin{center}
        % Example Truth Table
        \begin{tabular}{|c|c|c|c|c|c|}
            \hline
            \( P \) & \( Q \) & \( P \land Q \) & \( P \lor Q \) & \( \neg P \) & \( \neg Q \) \\ 
            \hline
            T       & T       & T                & T               & F            & F            \\ 
            T       & F       & F                & T               & F            & T            \\ 
            F       & T       & F                & T               & T            & F            \\ 
            F       & F       & F                & F               & T            & T            \\ 
            \hline
        \end{tabular}
    \end{center}
    \begin{block}{Analysis}
        If \( P \land Q \equiv Q \land P \) holds for all rows, then the propositions are equivalent.
    \end{block}
\end{frame}

\begin{frame}[fragile]
    \frametitle{Laws of Logic}
    \begin{block}{Laws of Logic}
        Certain logical identities allow for quick determination of equivalence:
        \begin{itemize}
            \item \textbf{De Morgan's Laws}:
            \begin{itemize}
                \item \( \neg (P \land Q) \equiv \neg P \lor \neg Q \)
                \item \( \neg (P \lor Q) \equiv \neg P \land \neg Q \)
            \end{itemize}
            \item \textbf{Implication}: \( P \implies Q \equiv \neg P \lor Q \)
        \end{itemize}
    \end{block}
\end{frame}

\begin{frame}[fragile]
    \frametitle{Practical Example: Implication}
    Show that \( P \implies Q \equiv \neg P \lor Q \) using a truth table.

    \begin{center}
        % Example Truth Table for Implication
        \begin{tabular}{|c|c|c|c|c|}
            \hline
            \( P \) & \( Q \) & \( P \implies Q \) & \( \neg P \) & \( \neg P \lor Q \) \\ 
            \hline
            T       & T       & T                   & F             & T                    \\ 
            T       & F       & F                   & F             & F                    \\ 
            F       & T       & T                   & T             & T                    \\ 
            F       & F       & T                   & T             & T                    \\ 
            \hline
        \end{tabular}
    \end{center}
    \begin{block}{Conclusion}
        The columns for \( P \implies Q \) and \( \neg P \lor Q \) match, confirming their equivalence.
    \end{block}
\end{frame}

\begin{frame}[fragile]
    \frametitle{Key Points to Emphasize}
    \begin{itemize}
        \item Always use truth tables for clear visual confirmation of equivalence.
        \item Understand and apply logical laws for simplified equivalences.
        \item Equivalence in logic is foundational for valid argumentation and reasoning.
    \end{itemize}
\end{frame}

\begin{frame}[fragile]
    \frametitle{Logical Inference}
    \begin{block}{Introduction to Inference Rules}
        Logical inference is a vital aspect of propositional logic, allowing us to derive conclusions from premises using established inference rules. By understanding these rules, we can construct valid arguments and prove the truth of propositions based on given assumptions.
    \end{block}
\end{frame}

\begin{frame}[fragile]
    \frametitle{Learning Objectives}
    \begin{itemize}
        \item Understand what inference rules are in propositional logic.
        \item Recognize common inference rules and their applications.
        \item Apply rules of inference to derive new propositions.
    \end{itemize}
\end{frame}

\begin{frame}[fragile]
    \frametitle{Key Inference Rules}
    \begin{enumerate}
        \item \textbf{Modus Ponens (MP)}:
              \begin{itemize}
                  \item If \( P \) implies \( Q \) (i.e., \( P \rightarrow Q \)) and \( P \) is true, then \( Q \) must be true.
                  \item Example:
                  \begin{itemize}
                      \item Premises: If it rains, then the ground is wet. \( (P \rightarrow Q) \). It is raining. \( (P) \).
                      \item Conclusion: The ground is wet. \( (Q) \).
                  \end{itemize}
              \end{itemize}

        \item \textbf{Modus Tollens (MT)}:
              \begin{itemize}
                  \item If \( P \) implies \( Q \) (i.e., \( P \rightarrow Q \)) and \( Q \) is false, then \( P \) is also false.
                  \item Example:
                  \begin{itemize}
                      \item Premises: If it is sunny, then the park is busy. \( (P \rightarrow Q) \). The park is not busy. \( (\neg Q) \).
                      \item Conclusion: It is not sunny. \( (\neg P) \).
                  \end{itemize}
              \end{itemize}
    \end{enumerate}
\end{frame}

\begin{frame}[fragile]
    \frametitle{Key Inference Rules (Continued)}
    \begin{enumerate}
        \setcounter{enumi}{2}
        \item \textbf{Disjunctive Syllogism (DS)}:
              \begin{itemize}
                  \item If \( P \) or \( Q \) is true (i.e., \( P \lor Q \)) and \( P \) is false, then \( Q \) must be true.
                  \item Example:
                  \begin{itemize}
                      \item Premises: It is either day or night. \( (P \lor Q) \). It is not day. \( (\neg P) \).
                      \item Conclusion: It is night. \( (Q) \).
                  \end{itemize}
              \end{itemize}

        \item \textbf{Hypothetical Syllogism (HS)}:
              \begin{itemize}
                  \item If \( P \) implies \( Q \) (i.e., \( P \rightarrow Q \)) and \( Q \) implies \( R \) (i.e., \( Q \rightarrow R \)), then \( P \) implies \( R \) (i.e., \( P \rightarrow R \)).
                  \item Example:
                  \begin{itemize}
                      \item Premises: If it snows, school is canceled. \( (P \rightarrow Q) \). If school is canceled, I will stay home. \( (Q \rightarrow R) \).
                      \item Conclusion: If it snows, I will stay home. \( (P \rightarrow R) \).
                  \end{itemize}
              \end{itemize}
    \end{enumerate}
\end{frame}

\begin{frame}[fragile]
    \frametitle{Applications of Logical Inference}
    \begin{itemize}
        \item \textbf{Mathematical Proofs}: Used to prove theorems and lemmas by establishing a chain of logical deductions.
        \item \textbf{Computer Science}: Logic programming utilizes inference rules to automate reasoning and derive conclusions from a set of knowledge.
        \item \textbf{Artificial Intelligence}: Knowledge representation and reasoning systems leverage inference rules for decision-making processes.
    \end{itemize}
\end{frame}

\begin{frame}[fragile]
    \frametitle{Summary and Key Points}
    \begin{block}{Summary}
        Understanding and applying inference rules is crucial for logical reasoning in propositional logic. By mastering these rules, you can construct valid arguments, evaluate logical statements, and effectively apply logic in various fields.
    \end{block}
    \begin{itemize}
        \item Modus Ponens and Modus Tollens are foundational rules for direct inference.
        \item Disjunctive Syllogism and Hypothetical Syllogism help build complex arguments.
        \item Inference rules enable logical reasoning in mathematical and real-world applications.
    \end{itemize}
\end{frame}

\begin{frame}[fragile]
    \frametitle{Applications of Propositional Logic}
    \begin{block}{Learning Objectives}
        \begin{itemize}
            \item Understand the role of propositional logic in artificial intelligence (AI).
            \item Explore real-world applications enhancing decision-making and problem-solving.
            \item Analyze how propositional logic underpins various AI systems.
        \end{itemize}
    \end{block}
\end{frame}

\begin{frame}[fragile]
    \frametitle{Propositional Logic in AI: Introduction}
    \begin{block}{Overview}
        Propositional logic serves as a foundational framework in AI, enabling machines to reason about information by representing statements as propositions (true or false). This modeling helps in drawing crucial inferences necessary for decision-making.
    \end{block}
\end{frame}

\begin{frame}[fragile]
    \frametitle{Key Applications of Propositional Logic}
    \begin{enumerate}
        \item \textbf{Expert Systems}
            \begin{itemize}
                \item Emulate human expert decision-making using logical rules.
                \item Example: Medical diagnosis if (cough $\land$ fever) $\rightarrow$ (likely flu).
            \end{itemize}
        \item \textbf{Automated Reasoning}
            \begin{itemize}
                \item Enabling automated theorem proving.
                \item Example: If (all inputs valid) $\land$ (input received) $\rightarrow$ (process input).
            \end{itemize}
        \item \textbf{Natural Language Processing (NLP)}
            \begin{itemize}
                \item Aids in structuring human language.
                \item Example: Sentiment analysis for positive/negative assertions.
            \end{itemize}
        \item \textbf{Game Theory and Decision-Making}
            \begin{itemize}
                \item Analyze strategies in competitive situations.
                \item Example: If (player chooses strategy A) $\rightarrow$ (outcome X with probability P).
            \end{itemize}
        \item \textbf{Robotics}
            \begin{itemize}
                \item Decision-making in uncertain environments.
                \item Example: If (object detected) $\land$ (path clear) $\rightarrow$ (move towards object).
            \end{itemize}
    \end{enumerate}
\end{frame}

\begin{frame}[fragile]
    \frametitle{Conclusion and Key Points}
    \begin{block}{Conclusion}
        By employing propositional logic, AI systems can effectively model knowledge and automate reasoning, enhancing decision-making based on logical relationships among data.
    \end{block}
    \begin{block}{Key Points to Remember}
        \begin{itemize}
            \item Propositional logic is essential for modeling and reasoning in AI.
            \item Applications range from expert systems to robotics and NLP.
            \item Structured decisions using logical propositions enhance efficiency.
        \end{itemize}
    \end{block}
\end{frame}

\begin{frame}[fragile]
    \frametitle{Example of Logical Inference}
    \begin{block}{Logical Example}
        \begin{lstlisting}
        P: "It is raining."
        Q: "The ground is wet."
        Logical Relation: P → Q (If it is raining, then the ground is wet.)
        \end{lstlisting}
    \end{block}
    \begin{block}{Future Exploration}
        By mastering these concepts, you will appreciate how propositional logic underpins critical decision-making in AI, paving the way for our next slide on the limitations of propositional logic.
    \end{block}
\end{frame}

\begin{frame}[fragile]
    \frametitle{Limitations of Propositional Logic - Overview}
    \begin{itemize}
        \item Propositional logic deals with propositions that are either true or false.
        \item It is a foundational tool for reasoning with several applications.
        \item Despite its strengths, there are notable limitations:
        \begin{itemize}
            \item Inability to express quantities.
            \item Lack of expressive power regarding relationships.
            \item Ambiguity in interpreting compound propositions.
            \item Restrictions to binary true/false values.
            \item Absence of internal structural analysis.
        \end{itemize}
    \end{itemize}
\end{frame}

\begin{frame}[fragile]
    \frametitle{Limitations of Propositional Logic - Key Points}
    \begin{enumerate}
        \item \textbf{Inability to Express Quantities:}
            \begin{itemize}
                \item Can't represent qualitative statements like "All cats are mammals."
            \end{itemize}
        
        \item \textbf{Lack of Expressive Power:}
            \begin{itemize}
                \item Example: "If it rains, the ground gets wet." 
                \item Represented as \( P \rightarrow Q \), lacks nuance.
            \end{itemize}
        
        \item \textbf{Ambiguity in Compound Propositions:}
            \begin{itemize}
                \item Context can alter the truth of propositions (e.g., \( P \land Q \)).
            \end{itemize}
        
        \item \textbf{Limited to True/False Values:}
            \begin{itemize}
                \item Doesn't accommodate uncertain states (e.g., fuzzy logic).
            \end{itemize}
        
        \item \textbf{No Internal Structure:}
            \begin{itemize}
                \item Atomic propositions restrict deeper analysis (e.g., "John loves Mary").
            \end{itemize}
    \end{enumerate}
\end{frame}

\begin{frame}[fragile]
    \frametitle{Example of Limitations}
    \begin{block}{Scenario}
        \begin{itemize}
            \item \( P: \text{"The sky is blue."} \)
            \item \( Q: \text{"It is daytime."} \)
            \item Represented as \( P \rightarrow Q \).
        \end{itemize}
    \end{block}
    
    \begin{block}{Limitation}
        \begin{itemize}
            \item Does not account for cloud cover or dusk, leading to incomplete reasoning about conditions affecting \( Q \).
        \end{itemize}
    \end{block}
\end{frame}

\begin{frame}[fragile]
    \frametitle{Conclusion}
    \begin{itemize}
        \item Recognizing the limitations of propositional logic is essential.
        \item It lays the groundwork for more sophisticated logical systems, such as first-order logic.
        \item These systems address the nuanced relationships and quantification that propositional logic cannot.
    \end{itemize}
\end{frame}

\begin{frame}[fragile]
    \frametitle{Transition to First-Order Logic - Overview}
    \begin{itemize}
        \item Explore the foundational transition from propositional logic to first-order logic (FOL).
        \item Understanding this transition is key to grasping more complex logical structures and reasoning.
    \end{itemize}
\end{frame}

\begin{frame}[fragile]
    \frametitle{Transition to First-Order Logic - Propositional Logic Recap}
    \begin{enumerate}
        \item \textbf{Definition}: Propositional logic consists of propositions that can either be true or false.
        \item \textbf{Limitations}:
        \begin{itemize}
            \item Unable to express relationships between objects or quantify over them.
            \item Example: "All humans are mortal" cannot be represented.
        \end{itemize}
    \end{enumerate}
\end{frame}

\begin{frame}[fragile]
    \frametitle{Transition to First-Order Logic - Introduction to First-Order Logic}
    \begin{enumerate}
        \item \textbf{Definition}: First-Order Logic (FOL) extends propositional logic by incorporating quantifiers and predicates.
        
        \item \textbf{Components of FOL}:
        \begin{itemize}
            \item \textbf{Predicates}: Functions that return true or false depending on the input (e.g., $ \text{Human}(x) $).
            \item \textbf{Quantifiers}:
            \begin{itemize}
                \item \textit{Universal Quantifier} $ \forall $: \( \forall x (\text{Human}(x) \rightarrow \text{Mortal}(x)) \)
                \item \textit{Existential Quantifier} $ \exists $: \( \exists x (\text{Human}(x) \land \text{Happy}(x)) \)
            \end{itemize}
        \end{itemize}
    \end{enumerate}
\end{frame}

\begin{frame}[fragile]
    \frametitle{Transition to First-Order Logic - From Propositional to FOL}
    \begin{enumerate}
        \item \textbf{Building Blocks}: Propositional symbols (e.g., $ P $, $ Q $) are the simplest units.
        
        \item \textbf{Example of Transition}:
        \begin{itemize}
            \item Propositional Logic Example: $ P $ - "Socrates is a human."
            \item First-Order Logic Representation: \( \text{Human}(\text{Socrates}) \).
            \item Allows expressions like \( \forall x (\text{Human}(x) \rightarrow \text{Mortal}(x)) \).
        \end{itemize}
    \end{enumerate}
\end{frame}

\begin{frame}[fragile]
    \frametitle{Transition to First-Order Logic - Real-World Application}
    \begin{itemize}
        \item \textbf{Use Cases}:
        \begin{itemize}
            \item Propositional logic for simple, binary decisions.
            \item FOL is crucial in mathematics, computer science, and artificial intelligence.
        \end{itemize}
        
        \item \textbf{Example}:
        \begin{itemize}
            \item In AI: FOL can express "Every person who loves programming also enjoys logic" using predicates and quantifiers.
        \end{itemize}
    \end{itemize}
\end{frame}

\begin{frame}[fragile]
    \frametitle{Transition to First-Order Logic - Conclusion}
    \begin{itemize}
        \item Transitioning from propositional logic to first-order logic enables a broader understanding of relationships.
        \item This foundation is essential for deeper logical reasoning in mathematical and computational contexts.
        \item Future sessions will delve deeper into first-order logic.
    \end{itemize}
\end{frame}

\begin{frame}[fragile]
    \frametitle{Summary of Key Points - Overview}
    \begin{block}{Overview of Propositional Logic}
        Propositional logic, also known as propositional calculus, is a branch of logic that deals with propositions—statements that can be either true or false. Understanding the foundations of propositional logic is crucial as it paves the way for more complex logical systems such as first-order logic.
    \end{block}
\end{frame}

\begin{frame}[fragile]
    \frametitle{Summary of Key Points - Key Concepts}
    \begin{enumerate}
        \item \textbf{Propositions}:
            \begin{itemize}
                \item \textbf{Definition}: A proposition is a declarative statement that can be assigned a truth value (true or false).
                \item \textbf{Examples}:
                    \begin{itemize}
                        \item "The sky is blue." (True)
                        \item "2 + 2 = 5." (False)
                    \end{itemize}
            \end{itemize}

        \item \textbf{Logical Connectives}:
            \begin{itemize}
                \item \textbf{AND (Conjunction, $\land$)}: True only if both propositions are true.
                \item \textbf{OR (Disjunction, $\lor$)}: True if at least one proposition is true.
                \item \textbf{NOT (Negation, $\neg$)}: Reverses the truth value of a proposition.
                \item \textbf{IMPLICATION ($\rightarrow$)}: $p \rightarrow q$ is true unless $p$ is true and $q$ is false.
                \item \textbf{BICONDITIONAL ($\leftrightarrow$)}: True if both propositions have the same truth value.
            \end{itemize}
    \end{enumerate}
\end{frame}

\begin{frame}[fragile]
    \frametitle{Summary of Key Points - Advanced Concepts}
    \begin{enumerate}
        \setcounter{enumi}{2}
        
        \item \textbf{Truth Tables}:
            \begin{itemize}
                \item A method to represent the possible truth values of propositions and their combinations.
                \item \textbf{Example for $p \land q$}:
                \begin{center}
                \begin{tabular}{|c|c|c|}
                    \hline
                    $p$ & $q$ & $p \land q$ \\
                    \hline
                    True & True & True \\
                    True & False & False \\
                    False & True & False \\
                    False & False & False \\
                    \hline
                \end{tabular}
                \end{center}
            \end{itemize}

        \item \textbf{Additional Concepts}:
            \begin{itemize}
                \item \textbf{Logical Equivalence}: Two propositions are equivalent if they hold the same truth value in all scenarios.
                \item \textbf{Tautologies and Contradictions}:
                    \begin{itemize}
                        \item \textbf{Tautology}: Always true (e.g., $p \lor \neg p$).
                        \item \textbf{Contradiction}: Always false (e.g., $p \land \neg p$).
                    \end{itemize}
                \item \textbf{Applications}: Used in computer science, digital circuit design, artificial intelligence, etc.
            \end{itemize}
    \end{enumerate}
\end{frame}

\begin{frame}[fragile]
    \frametitle{Summary of Key Points - Conclusion}
    \begin{block}{Important Points to Emphasize}
        \begin{itemize}
            \item Propositional logic serves as a foundational element for first-order logic.
            \item Understanding how to construct truth tables is essential for analyzing logical arguments.
            \item Mastery of logical connectives is crucial for theoretical applications and practical problem-solving.
        \end{itemize}
    \end{block}
    
    \begin{block}{Conclusion}
        Propositional logic provides the toolkit for forming compound statements and reasoning about truth values.
        Mastery of its principles lays the groundwork for more advanced topics in logic and computational theory, paving the way for deeper exploration into first-order logic.
    \end{block}
\end{frame}

\begin{frame}[fragile]
    \frametitle{Practice Problems - Overview}
    \begin{block}{Overview of Propositional Logic}
        Propositional logic is a branch of logic that deals with propositions, which are statements that can either be true or false. Understanding propositional logic is essential for various fields including mathematics, computer science, and philosophy. It allows us to form logical arguments, analyze statements, and derive conclusions based on given premises.
    \end{block}
\end{frame}

\begin{frame}[fragile]
    \frametitle{Practice Problems - Learning Objectives}
    \begin{block}{Learning Objectives}
        By the end of this section, you should be able to:
        \begin{enumerate}
            \item Identify basic logical connectives.
            \item Solve practice problems using truth tables.
            \item Apply logical reasoning to deduce new information.
        \end{enumerate}
    \end{block}
\end{frame}

\begin{frame}[fragile]
    \frametitle{Practice Problems - Key Concepts}
    \begin{block}{Key Concepts}
        \begin{itemize}
            \item \textbf{Proposition}: A declarative statement that can be either true (T) or false (F).
            \item \textbf{Logical Connectives}:
            \begin{itemize}
                \item \textbf{Conjunction (AND)}: $p \land q$
                \item \textbf{Disjunction (OR)}: $p \lor q$
                \item \textbf{Negation (NOT)}: $\neg p$
                \item \textbf{Implication (IF...THEN)}: $p \rightarrow q$
                \item \textbf{Biconditional (IF AND ONLY IF)}: $p \leftrightarrow q$
            \end{itemize}
        \end{itemize}
    \end{block}
\end{frame}

\begin{frame}[fragile]
    \frametitle{Practice Problems - Example Problems}
    \begin{block}{Practice Problems}
        \begin{enumerate}
            \item \textbf{Problem 1: Truth Table Creation}
                Construct the truth table for the expression $p \land (q \lor \neg r)$.
            \item \textbf{Problem 2: Logical Equivalence}
                Determine whether $p \rightarrow q$ and $\neg q \rightarrow \neg p$ are logically equivalent.
            \item \textbf{Problem 3: Application of Logic}
                Rewrite and analyze implications involving:
                \begin{itemize}
                    \item $p$: "It is raining."
                    \item $q$: "The ground is wet."
                    \item $r$: "I will take an umbrella."
                \end{itemize}
        \end{enumerate}
    \end{block}
\end{frame}

\begin{frame}[fragile]
    \frametitle{Practice Problems - Key Points}
    \begin{block}{Key Points to Emphasize}
        \begin{itemize}
            \item Understanding truth tables is crucial for visualizing logical expressions and their outcomes.
            \item Logical equivalence is essential in simplifying propositions.
            \item Applying propositional logic to real-world scenarios enhances comprehension and retention.
        \end{itemize}
    \end{block}
\end{frame}

\begin{frame}[fragile]
  \frametitle{Conclusion and Next Steps - Wrap-Up of Propositional Logic}
  
  \begin{itemize}
    \item Explored foundational concepts of propositional logic.
    \item Key elements include:
      \begin{enumerate}
        \item Basic Definitions
        \item Truth Tables
        \item Tautologies and Contradictions
        \item Logical Equivalence
      \end{enumerate}
  \end{itemize}
  
  \begin{block}{Key Points to Emphasize}
    \begin{itemize}
      \item Critical for logical reasoning, programming, and algorithms.
      \item Valuable skill in constructing and interpreting truth tables.
    \end{itemize}
  \end{block}
\end{frame}

\begin{frame}[fragile]
  \frametitle{Conclusion and Next Steps - Illustrative Example}
  
  \begin{block}{Example}
    Consider the propositions:
    \begin{itemize}
      \item P: "It is raining."
      \item Q: "I will take an umbrella."
    \end{itemize}
    
    The statement "It is not raining OR I will take an umbrella" can be represented as:
    \begin{equation}
      \neg P \lor Q
    \end{equation}
    
    A truth table can be constructed to evaluate this statement under various scenarios of P and Q.
  \end{block}
\end{frame}

\begin{frame}[fragile]
  \frametitle{Conclusion and Next Steps - Upcoming Topics}
  
  \begin{itemize}
    \item Next sessions will cover **Predicate Logic**.
    \item Predicate logic introduces:
      \begin{itemize}
        \item Quantifiers (e.g., "for all," "there exists")
        \item More complex statements involving relationships.
      \end{itemize}
  \end{itemize}
  
  \begin{block}{Preparation}
    \begin{itemize}
      \item Review examples and practice problems.
      \item Read ahead on Predicate Logic concepts.
      \item Reflect on applications in computer science.
    \end{itemize}
  \end{block}
  
  \begin{block}{Homework}
    \begin{itemize}
      \item Complete practice problems from the previous slide.
    \end{itemize}
  \end{block}
\end{frame}


\end{document}