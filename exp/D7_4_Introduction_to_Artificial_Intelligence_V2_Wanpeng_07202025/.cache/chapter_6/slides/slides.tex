\documentclass[aspectratio=169]{beamer}

% Theme and Color Setup
\usetheme{Madrid}
\usecolortheme{whale}
\useinnertheme{rectangles}
\useoutertheme{miniframes}

% Additional Packages
\usepackage[utf8]{inputenc}
\usepackage[T1]{fontenc}
\usepackage{graphicx}
\usepackage{booktabs}
\usepackage{listings}
\usepackage{amsmath}
\usepackage{amssymb}
\usepackage{xcolor}
\usepackage{tikz}
\usepackage{pgfplots}
\pgfplotsset{compat=1.18}
\usetikzlibrary{positioning}
\usepackage{hyperref}

% Custom Colors
\definecolor{myblue}{RGB}{31, 73, 125}
\definecolor{mygray}{RGB}{100, 100, 100}
\definecolor{mygreen}{RGB}{0, 128, 0}
\definecolor{myorange}{RGB}{230, 126, 34}
\definecolor{mycodebackground}{RGB}{245, 245, 245}

% Set Theme Colors
\setbeamercolor{structure}{fg=myblue}
\setbeamercolor{frametitle}{fg=white, bg=myblue}
\setbeamercolor{title}{fg=myblue}
\setbeamercolor{section in toc}{fg=myblue}
\setbeamercolor{item projected}{fg=white, bg=myblue}
\setbeamercolor{block title}{bg=myblue!20, fg=myblue}
\setbeamercolor{block body}{bg=myblue!10}
\setbeamercolor{alerted text}{fg=myorange}

% Set Fonts
\setbeamerfont{title}{size=\Large, series=\bfseries}
\setbeamerfont{frametitle}{size=\large, series=\bfseries}
\setbeamerfont{caption}{size=\small}
\setbeamerfont{footnote}{size=\tiny}

\title[Week 6: Multi-Agent Systems]{Week 6: Multi-Agent Systems}
\author[J. Smith]{John Smith, Ph.D.}
\institute[University Name]{
  Department of Computer Science\\
  University Name\\
  Email: email@university.edu\\
  Website: www.university.edu
}
\date{\today}

% Document Start
\begin{document}

\frame{\titlepage}

\begin{frame}[fragile]
    \frametitle{Introduction to Multi-Agent Systems}
    \begin{block}{What are Multi-Agent Systems (MAS)?}
        Multi-Agent Systems refer to a system composed of multiple interacting agents that can be either autonomous entities or collaborative systems designed to solve complex problems or achieve specific goals. 
        These agents can be software programs, robots, or even humans.
    \end{block}
\end{frame}

\begin{frame}[fragile]
    \frametitle{Significance in Artificial Intelligence}
    \begin{enumerate}
        \item \textbf{Distributed Problem Solving}: 
        MAS allows for the distribution of tasks among agents that can work simultaneously towards common objectives, enhancing efficiency and effectiveness.
        \begin{itemize}
            \item \textit{Example}: In a traffic management scenario, multiple agents can communicate to control traffic lights, optimizing traffic flow.
        \end{itemize}
        
        \item \textbf{Adaptability and Scalability}: 
        Systems can easily adapt to changes in the environment or tasks by adding or modifying agents without restarting the entire system.
        \begin{itemize}
            \item \textit{Illustration}: Consider e-commerce platforms that use MAS to manage inventory and delivery logistics, allowing for quick adjustments based on demand.
        \end{itemize}
        
        \item \textbf{Complex Systems Handling}: 
        MAS is crucial for addressing problems that are too complex for a single agent to solve independently. They can model and manage intricate interactions and dependencies.
        \begin{itemize}
            \item \textit{Example}: Climate modeling, where various elements (atmosphere, ocean currents, etc.) interact, requires multiple agents to predict and simulate changes.
        \end{itemize}
    \end{enumerate}
\end{frame}

\begin{frame}[fragile]
    \frametitle{Key Points to Emphasize}
    \begin{itemize}
        \item \textbf{Autonomy}: Each agent operates independently but cooperates or competes with other agents.
        \item \textbf{Communication}: Agents often communicate and negotiate with each other to share knowledge or resources.
        \item \textbf{Goal-Oriented}: MAS enhances the decision-making process in reaching specific objectives, either collaboratively or competitively.
    \end{itemize}

    \begin{block}{Examples of Applications}
        \begin{itemize}
            \item \textbf{Robotics}: Swarms of drones can coordinate their movements to complete large-scale applications like search and rescue.
            \item \textbf{Game AI}: Video games often utilize MAS for realistic behavior of non-player characters (NPCs).
        \end{itemize}
    \end{block}
\end{frame}

\begin{frame}[fragile]
    \frametitle{Conclusion}
    Multi-Agent Systems are pivotal in the advancement of artificial intelligence, enabling systems that can learn, adapt, and operate effectively in complex environments.
    Understanding MAS is essential for developers and researchers aiming to implement intelligent solutions in real-world applications.
\end{frame}

\begin{frame}[fragile]
    \frametitle{Next Slide Preview}
    We'll outline the learning objectives to guide our exploration of Multi-Agent Systems.
\end{frame}

\begin{frame}[fragile]
    \frametitle{Learning Objectives - Overview}
    \begin{block}{Learning Objectives}
        By the end of this chapter, you should be able to:
    \end{block}
    \begin{enumerate}
        \item Define Multi-Agent Systems (MAS)
        \item Identify Types of Agents
        \item Understand Agent Communication Protocols
        \item Explore Learning and Adaptation in Agents
        \item Recognize Applications of MAS
        \item Analyze Coordination Strategies
        \item Evaluate Challenges in MAS
    \end{enumerate}
\end{frame}

\begin{frame}[fragile]
    \frametitle{Learning Objectives - Definitions and Types}
    \begin{enumerate}[resume]
        \item \textbf{Define Multi-Agent Systems (MAS)}  
        - Explain what a multi-agent system is, including the basic elements such as agents, environment, and interactions.  
        - \textbf{Key Point}: A multi-agent system consists of multiple interacting intelligent agents that can be cooperative, competitive, or a mixture of both.

        \item \textbf{Identify Types of Agents}  
        - Distinguish among different types of agents: autonomous agents, reactive agents, and deliberative agents.  
        - \textbf{Example}: An autonomous agent, like a self-driving car, makes decisions based on its environment without human intervention.
    \end{enumerate}
\end{frame}

\begin{frame}[fragile]
    \frametitle{Learning Objectives - Communication and Applications}
    \begin{enumerate}[resume]
        \item \textbf{Understand Agent Communication Protocols}  
        - Describe how agents communicate with each other through predefined protocols.  
        - \textbf{Illustration}: Imagine agents in a shipping logistics system sharing information about their locations and cargo to optimize delivery routes.

        \item \textbf{Recognize Applications of MAS}  
        - Examine real-world applications of multi-agent systems in various fields such as robotics, telecommunications, and traffic management.  
        - \textbf{Example}: Smart traffic lights that adjust based on the flow of cars, employing a MAS to minimize congestion.
    \end{enumerate}
\end{frame}

\begin{frame}[fragile]
    \frametitle{Learning Objectives - Coordination and Challenges}
    \begin{enumerate}[resume]
        \item \textbf{Explore Learning and Adaptation in Agents}  
        - Discuss methods by which agents learn from their environment, including reinforcement learning and supervised learning.  
        - \textbf{Formula Example}: In reinforcement learning, agent performance can be evaluated using a reward function \( R(s, a) \) to assess the quality of actions taken.

        \item \textbf{Analyze Coordination Strategies}  
        - Investigate coordination mechanisms, including negotiation, auctions, and centralized vs decentralized control.  
        - \textbf{Key Point}: Effective coordination can significantly enhance the performance of multi-agent systems in collaborative tasks.

        \item \textbf{Evaluate Challenges in MAS}  
        - Identify challenges such as scalability, robustness, and conflict resolution that are inherent in multi-agent systems.  
        - \textbf{Example}: In a large-scale emergency response scenario, agents must quickly adapt to new situations and coordinate actions to effectively manage resources.
    \end{enumerate}
\end{frame}

\begin{frame}[fragile]
    \frametitle{Learning Objectives - Conclusion}
    \begin{block}{Conclusion}
        By explicitly focusing on these objectives, you will gain a comprehensive understanding of multi-agent systems, their functionalities, and their significance in the field of artificial intelligence. Through examples, definitions, and applications, this chapter will clarify the essential components and operations involved in multi-agent systems.
    \end{block}
\end{frame}

\begin{frame}[fragile]
    \frametitle{Definition of Agents}
    \begin{block}{What Constitutes an Agent in Artificial Intelligence?}
        In AI, an agent is an entity that perceives its environment through sensors and acts upon it using actuators, making decisions based on perceptions to achieve specific goals.
    \end{block}
\end{frame}

\begin{frame}[fragile]
    \frametitle{Key Components of an Agent}
    \begin{itemize}
        \item \textbf{Perception:}
        \begin{itemize}
            \item Uses sensors to gather information about the environment.
            \item \textit{Example:} A self-driving car uses cameras and lidar to perceive surroundings.
        \end{itemize}
        
        \item \textbf{Action:}
        \begin{itemize}
            \item Takes actions with actuators like motors and screens.
            \item \textit{Example:} The car navigates using its steering and braking systems.
        \end{itemize}
        
        \item \textbf{Autonomy:}
        \begin{itemize}
            \item Operates without direct human intervention and adapts based on past experiences.
            \item \textit{Example:} An online shopping system adjusts suggestions based on user behavior.
        \end{itemize}
        
        \item \textbf{Goal-oriented:}
        \begin{itemize}
            \item Designed to achieve specific objectives, which can evolve over time.
            \item \textit{Example:} A vacuum robot aims to clean its area efficiently.
        \end{itemize}
    \end{itemize}
\end{frame}

\begin{frame}[fragile]
    \frametitle{Types of Agents}
    \begin{itemize}
        \item \textbf{Simple Reflex Agents:} 
        \begin{itemize}
            \item Respond to current perceptions only.
            \item \textit{Example:} A thermostat regulating temperature.
        \end{itemize}

        \item \textbf{Model-Based Reflex Agents:} 
        \begin{itemize}
            \item Maintain an internal state to make informed decisions.
            \item \textit{Example:} Chess programs anticipate moves based on game state.
        \end{itemize}

        \item \textbf{Goal-Based Agents:}
        \begin{itemize}
            \item Make decisions based on predefined goals.
            \item \textit{Example:} A video game AI navigating to collect items.
        \end{itemize}

        \item \textbf{Learning Agents:}
        \begin{itemize}
            \item Improve performance over time through experience.
            \item \textit{Example:} AI chatbots learning to enhance responses.
        \end{itemize}
    \end{itemize}
\end{frame}

\begin{frame}[fragile]
    \frametitle{Types of Agents - Overview}
    \begin{block}{Learning Objectives}
        \begin{itemize}
            \item Understand different classifications of agents in multi-agent systems.
            \item Identify the characteristics of reactive, deliberative, and hybrid agents.
            \item Recognize the applications and limitations of each agent type.
        \end{itemize}
    \end{block}
\end{frame}

\begin{frame}[fragile]
    \frametitle{Types of Agents - Reactive Agents}
    \begin{block}{Definition}
        Reactive agents operate based on the current state of the environment without a model of the world. They respond directly to stimuli from the environment.
    \end{block}
    
    \begin{block}{Characteristics}
        \begin{itemize}
            \item Simple Responses: Act based on specific conditions or inputs (if-then rules).
            \item Limited Memory and Planning: No consideration of past actions or future consequences.
            \item Fast and Efficient: Quick response times make them ideal for dynamic environments.
        \end{itemize}
    \end{block}
    
    \begin{block}{Example}
        \begin{itemize}
            \item \textbf{Robotic Vacuum Cleaner}: Detects obstacles (e.g., walls, furniture) and changes direction immediately without complex planning.
        \end{itemize}
    \end{block}

    \begin{block}{Key Points}
        \begin{itemize}
            \item Good for environments where immediate reaction is crucial.
            \item Limited adaptability to new situations or complex tasks.
        \end{itemize}
    \end{block}
\end{frame}

\begin{frame}[fragile]
    \frametitle{Types of Agents - Deliberative and Hybrid Agents}
    \begin{block}{Deliberative Agents}
        \begin{itemize}
            \item \textbf{Definition}: Have a model of the world and use reasoning to make decisions, incorporating planning strategies based on goals and past actions.
            \item \textbf{Characteristics}:
                \begin{itemize}
                    \item Goal-Oriented
                    \item Advanced Planning
                    \item Knowledge Representation
                \end{itemize}
            \item \textbf{Example}: 
                \begin{itemize}
                    \item \textbf{Game-Playing AI (e.g., Chess)}: Evaluates all possible moves and devises strategies to win the game.
                \end{itemize}
            \item \textbf{Key Points}:
                \begin{itemize}
                    \item Trade-off between action speed and decision-making depth.
                    \item Highly adaptable but may require significant computational resources.
                \end{itemize}
        \end{itemize}
    \end{block}

    \begin{block}{Hybrid Agents}
        \begin{itemize}
            \item \textbf{Definition}: Combine the strengths of reactive and deliberative approaches, enabling quick response while planning ahead.
            \item \textbf{Characteristics}:
                \begin{itemize}
                    \item Integration of Techniques
                    \item Flexibility
                \end{itemize}
            \item \textbf{Example}: 
                \begin{itemize}
                    \item \textbf{Autonomous Vehicles}: Respond to immediate obstacles while navigating based on traffic patterns and fuel efficiency.
                \end{itemize}
            \item \textbf{Key Points}:
                \begin{itemize}
                    \item Balance between speed and strategy.
                    \item Suitable for complex environments with unpredictable elements.
                \end{itemize}
        \end{itemize}
    \end{block}
\end{frame}

\begin{frame}[fragile]
    \frametitle{Agent Architecture}
    \begin{block}{Learning Objectives}
        \begin{itemize}
            \item Understand the different types of agent architectures in multi-agent systems.
            \item Analyze the strengths and weaknesses of each architecture.
            \item Relate each architecture to practical applications.
        \end{itemize}
    \end{block}
\end{frame}

\begin{frame}[fragile]
    \frametitle{Introduction to Agent Architectures}
    Agent architectures define how an agent perceives its environment, processes information, and acts upon it. Here, we will explore three fundamental types of agent architectures:
    \begin{itemize}
        \item Simple Reflex Agents
        \item Goal-Based Agents
        \item Utility-Based Agents
    \end{itemize}
\end{frame}

\begin{frame}[fragile]
    \frametitle{Simple Reflex Agents}
    \begin{block}{Definition}
        These agents function based on predefined rules or conditions, reacting to specific inputs from their environment without storing the history of previous states.
    \end{block}
    \begin{block}{How it Works}
        A simple reflex agent employs condition-action rules (production rules). For example, if it senses a certain stimulus, it will immediately respond with an action.
    \end{block}
    \begin{block}{Example}
        A thermostat that switches on the heating when the temperature drops below a certain threshold.
    \end{block}
    \begin{itemize}
        \item \textbf{Strengths}: Simple to implement; fast response time.
        \item \textbf{Weaknesses}: No learning capability; limited to predefined situations.
    \end{itemize}
\end{frame}

\begin{frame}[fragile]
    \frametitle{Goal-Based Agents}
    \begin{block}{Definition}
        Agents that operate with specific goals in mind. They evaluate possible actions based on their potential to achieve these goals.
    \end{block}
    \begin{block}{How it Works}
        These agents not only react to current stimuli but also consider future consequences of their actions to reach desired outcomes.
    \end{block}
    \begin{block}{Example}
        A navigation app determines the best route to a user's destination based on current traffic conditions and user-defined preferences.
    \end{block}
    \begin{itemize}
        \item \textbf{Strengths}: More flexible than reflex agents; can operate in dynamic environments.
        \item \textbf{Weaknesses}: Requires more computational resources; may get distracted by conflicting goals.
    \end{itemize}
\end{frame}

\begin{frame}[fragile]
    \frametitle{Utility-Based Agents}
    \begin{block}{Definition}
        These agents assess and compare the utility (or satisfaction) of different states and choose actions that maximize their expected utility.
    \end{block}
    \begin{block}{How it Works}
        They utilize a utility function that quantifies preferences over different states, allowing them to choose between competing actions based on their expected outcomes.
    \end{block}
    \begin{block}{Example}
        A shopping assistant agent that recommends products based on user preferences, price, and quality ratings to maximize user satisfaction.
    \end{block}
    \begin{itemize}
        \item \textbf{Strengths}: Capable of handling complex decision-making; can adapt to user preferences dynamically.
        \item \textbf{Weaknesses}: Implementation complexity; requires accurate utility measurements which might be hard to define.
    \end{itemize}
\end{frame}

\begin{frame}[fragile]
    \frametitle{Summary of Agent Architectures}
    \begin{itemize}
        \item \textbf{Simple Reflex Agents}: Reactive and rule-based; best for straightforward tasks.
        \item \textbf{Goal-Based Agents}: Goal-oriented with more flexibility; useful in dynamic situations.
        \item \textbf{Utility-Based Agents}: Optimize actions based on utility; ideal for environments with varied preferences.
    \end{itemize}
\end{frame}

\begin{frame}[fragile]
    \frametitle{Learning Objectives}
    \begin{itemize}
        \item Understand different types of interactions among agents in multi-agent systems.
        \item Analyze the impact of interactions on system performance and outcomes.
        \item Explore the various communication and coordination strategies among agents.
    \end{itemize}
\end{frame}

\begin{frame}[fragile]
    \frametitle{Introduction to Agent Interactions}
    In multi-agent systems (MAS), agents interact to achieve their individual and collective goals. 
    These interactions can be classified into several categories:
    \begin{itemize}
        \item \textbf{Direct Communication}: Explicit messaging between agents.
        \item \textbf{Indirect Communication}: Information conveyed through shared environments, e.g., trails or symbols.
        \item \textbf{Observational Interactions}: Agents monitor others' behaviors to inform their decisions.
    \end{itemize}
\end{frame}

\begin{frame}[fragile]
    \frametitle{Interaction Protocols}
    Agents follow specific protocols to manage interactions, which ensure effective communication:
    \begin{itemize}
        \item \textbf{Messaging Protocols}: Define message structure and exchange (e.g., ACL - Agent Communication Language).
        \item \textbf{Negotiation Protocols}: Techniques for reaching agreements (e.g., bargaining strategies).
        \item \textbf{Coordination Protocols}: Rules for scheduling joint actions (e.g., task allocation).
    \end{itemize}
\end{frame}

\begin{frame}[fragile]
    \frametitle{Influence of Interactions on System Performance}
    The nature of interactions can significantly affect a multi-agent system's functionality:
    \begin{itemize}
        \item \textbf{Trust and Reputation}: Past interactions influence current decisions.
        \item \textbf{Conflict Resolution}: Essential mechanisms must be established to resolve disputes.
    \end{itemize}
    
    Key Points to Consider:
    \begin{itemize}
        \item Successful interactions foster cooperative behaviors, enhancing performance.
        \item Poorly managed interactions can lead to conflicts or inefficiencies.
    \end{itemize}
\end{frame}

\begin{frame}[fragile]
    \frametitle{Example Scenario}
    Consider a team of autonomous delivery drones:
    \begin{itemize}
        \item \textbf{Direct Communication}: Drones share real-time traffic updates.
        \item \textbf{Indirect Communication}: Drones leave markers for others to follow optimal routes.
        \item \textbf{Observational Learning}: Drones adjust flight paths based on each other's delivery experiences.
    \end{itemize}
\end{frame}

\begin{frame}[fragile]
    \frametitle{Conclusion}
    Understanding agent interactions in multi-agent systems is crucial for designing effective AI systems. 
    These interactions enhance individual and overall system capabilities, fostering collaboration and achieving optimal outcomes.
\end{frame}

\begin{frame}[fragile]
    \frametitle{Cooperative vs. Competitive Behavior}
    \begin{block}{Learning Objectives}
        \begin{itemize}
            \item Understand the concepts of cooperative and competitive behavior in multi-agent systems.
            \item Differentiate between the two behaviors with real-world examples.
            \item Recognize how these behaviors influence the design and function of multi-agent systems.
        \end{itemize}
    \end{block}
\end{frame}

\begin{frame}[fragile]
    \frametitle{Definitions and Examples}
    \begin{block}{Definitions}
        \begin{itemize}
            \item \textbf{Cooperative Behavior:} Involves agents working together towards a common goal, sharing resources, knowledge, or tasks.
            \item \textbf{Competitive Behavior:} Occurs when agents act independently or against each other to maximize their own gains.
        \end{itemize}
    \end{block}
    
    \begin{block}{Examples}
        \begin{itemize}
            \item \textbf{Cooperative Example:} Robots in a warehouse coordinate to transport items, optimizing paths.
            \item \textbf{Competitive Example:} Players in real-time strategy games compete for resources and strategic advantages.
        \end{itemize}
    \end{block}
\end{frame}

\begin{frame}[fragile]
    \frametitle{Key Points and Conclusion}
    \begin{block}{Key Points to Emphasize}
        \begin{enumerate}
            \item \textbf{Incentives:} Cooperative agents enhance efficiency; competitive agents may lower overall efficiency.
            \item \textbf{Conflict Resolution:} Cooperative systems support negotiation; competitive environments require conflict management strategies.
            \item \textbf{Hybrid Approaches:} Many systems utilize both behaviors for shared objectives while pursuing individual metrics.
        \end{enumerate}
    \end{block}

    \begin{block}{Conclusion}
        Understanding cooperative and competitive behaviors is crucial for effective algorithm design in multi-agent systems.
    \end{block}
\end{frame}

\begin{frame}[fragile]
    \frametitle{Game Theory Framework}
    Consider the following payoff matrix representing agent interactions in game theory:

    \begin{equation}
        \begin{array}{c|c|c}
            \text{Agent 2 / Agent 1} & \text{Cooperate (C)} & \text{Compete (D)} \\
            \hline
            \text{Cooperate (C)} & (R, R) & (S, T) \\
            \text{Compete (D)} & (T, S) & (P, P) \\
        \end{array}
    \end{equation}

    Where:
    \begin{itemize}
        \item $R$: Reward for mutual cooperation
        \item $T$: Temptation to defect
        \item $S$: Sucker's payoff for cooperating while the other defects
        \item $P$: Punishment for mutual defection
    \end{itemize}
\end{frame}

\begin{frame}[fragile]
    \frametitle{Multi-Agent Search Algorithms}
    \begin{block}{Learning Objectives}
        \begin{itemize}
            \item Understand what multi-agent search algorithms are.
            \item Compare different types of search algorithms in multi-agent environments.
            \item Recognize the importance of coordination and communication in multi-agent searches.
        \end{itemize}
    \end{block}
\end{frame}

\begin{frame}[fragile]
    \frametitle{Introduction to Multi-Agent Search Algorithms}
    Multi-Agent Search Algorithms are specialized strategies used by multiple agents to explore, retrieve, or optimize information collectively in shared environments. They address complexities like interactions, competition, and cooperation among agents.

    \begin{block}{Key Concepts}
        \begin{itemize}
            \item **Agent Autonomy**: Each agent operates independently, affecting others' actions.
            \item **Environment Representation**: The dynamic search space requires adaptation to changes.
            \item **Coordination**: Agents need strategies to avoid conflicts and optimize results.
        \end{itemize}
    \end{block}
\end{frame}

\begin{frame}[fragile]
    \frametitle{Types of Multi-Agent Search Algorithms}
    \begin{enumerate}
        \item \textbf{Cooperative Search Algorithms}
            \begin{itemize}
                \item Description: All agents work together towards a common goal.
                \item Example: Modified A* search algorithm where agents share explored paths.
            \end{itemize}
        \item \textbf{Competitive Search Algorithms}
            \begin{itemize}
                \item Description: Agents compete for limited resources or objectives.
                \item Example: Minimax algorithm in games like multi-agent capture the flag.
            \end{itemize}
        \item \textbf{Hybrid Search Algorithms}
            \begin{itemize}
                \item Description: A blend of cooperative and competitive approaches.
                \item Example: Coalitional Game Theory for resource allocation negotiations.
            \end{itemize}
    \end{enumerate}
\end{frame}

\begin{frame}[fragile]
    \frametitle{Importance of Coordination and Communication}
    \begin{itemize}
        \item **Efficiency**: Effective communication can reduce overlapping efforts, speeding up the search process.
        \item **Conflict Resolution**: Agents can negotiate agreements to maximize success even in competitive scenarios.
    \end{itemize}

    \begin{block}{Illustrative Example}
        Imagine a group of drones (agents) searching for survivors after a disaster, needing to share findings to maximize efficiency.
    \end{block}
\end{frame}

\begin{frame}[fragile]
    \frametitle{Summary and Further Study}
    \begin{itemize}
        \item Multi-Agent Search Algorithms enhance navigational efficiency.
        \item Key algorithm types are cooperative, competitive, and hybrid.
        \item Coordination and communication are critical for agent performance.
    \end{itemize}

    \begin{block}{Further Study}
        \begin{itemize}
            \item Explore implementations of algorithms like A* in multi-agent contexts with Python or NetLogo.
            \item Review case studies on real-world applications in disaster response, logistics, and robotics.
        \end{itemize}
    \end{block}
\end{frame}

\begin{frame}[fragile]
    \frametitle{Learning Objectives}
    \begin{itemize}
        \item Understand the different modes of communication in multi-agent systems (MAS).
        \item Explore practical examples for each communication mode.
        \item Recognize the significance of coordination and information sharing among agents.
    \end{itemize}
\end{frame}

\begin{frame}[fragile]
    \frametitle{Modes of Communication in Multi-Agent Systems}
    Communication between agents is crucial for collaboration and achieving common goals. The main modes include:
    \begin{enumerate}
        \item Direct Communication
        \item Indirect Communication
        \item Coordinated Communication
    \end{enumerate}
\end{frame}

\begin{frame}[fragile]
    \frametitle{Direct Communication}
    \begin{itemize}
        \item \textbf{Definition}: Agents exchange information through explicit messages.
        \item \textbf{Mechanism}: Can use protocols such as message passing or shared data structures.
        \item \textbf{Example}: A robot sending a status message to another robot, like "I found the target location".
    \end{itemize}
\end{frame}

\begin{frame}[fragile]
    \frametitle{Indirect Communication}
    \begin{itemize}
        \item \textbf{Definition}: Agents communicate indirectly through shared environments or artifacts.
        \item \textbf{Mechanism}: Known as stigmergy, relying on environmental cues for information.
        \item \textbf{Example}: Ants depositing pheromones to indicate food sources, which other ants follow.
    \end{itemize}
\end{frame}

\begin{frame}[fragile]
    \frametitle{Coordinated Communication}
    \begin{itemize}
        \item \textbf{Definition}: Agents work together through pre-defined agreements or plans.
        \item \textbf{Mechanism}: Involves synchronization and planning.
        \item \textbf{Example}: Autonomous drones coordinating to cover a large area while avoiding overlap.
    \end{itemize}
\end{frame}

\begin{frame}[fragile]
    \frametitle{Importance of Communication}
    \begin{itemize}
        \item \textbf{Efficiency}: Streamlines decision-making and reduces redundancy.
        \item \textbf{Flexibility}: Increases adaptability to changing environments and tasks.
        \item \textbf{Collaboration}: Enhances coordination towards a shared objective.
    \end{itemize}
\end{frame}

\begin{frame}[fragile]
    \frametitle{Key Points to Remember}
    \begin{itemize}
        \item Communication enhances collaboration and efficiency in MAS.
        \item Different modes of communication serve different purposes based on context.
        \item Strategies should be evaluated based on task requirements.
    \end{itemize}
\end{frame}

\begin{frame}[fragile]
    \frametitle{Conclusion}
    Understanding communication modes in multi-agent systems is vital for designing agents that collaborate efficiently. The chosen strategies directly impact the success of the system's operations.
\end{frame}

\begin{frame}[fragile]
    \frametitle{Applications of Multi-Agent Systems}
    \begin{block}{Introduction}
        Multi-agent systems (MAS) consist of multiple interacting agents that can operate cooperatively or competitively to solve tasks in various real-world applications. Below are some key applications illustrating the versatility and effectiveness of MAS.
    \end{block}
\end{frame}

\begin{frame}[fragile]
    \frametitle{Key Applications - Part 1}
    \begin{enumerate}
        \item \textbf{Robotics}
            \begin{itemize}
                \item \textbf{Autonomous Vehicles:} Agents like cars and drones navigate while avoiding obstacles.
                \item \textbf{Swarm Robotics:} Simple robots collaborate for complex tasks, e.g., swarm drones for environmental monitoring.
            \end{itemize}
        
        \item \textbf{Simulations}
            \begin{itemize}
                \item \textbf{Traffic Simulations:} Models traffic flow and interactions to assist city planners.
                \item \textbf{Epidemic Simulations:} Simulates disease spread among individuals to evaluate interventions.
            \end{itemize}
    \end{enumerate}
\end{frame}

\begin{frame}[fragile]
    \frametitle{Key Applications - Part 2}
    \begin{enumerate}
        \setcounter{enumi}{3}
        \item \textbf{Game Theory and Economics}
            \begin{itemize}
                \item \textbf{Market Simulation:} Studies market behaviors through agents representing buyers and sellers.
                \item \textbf{Negotiation Protocols:} Agents simulate negotiations for mutually beneficial agreements.
            \end{itemize}

        \item \textbf{Distributed Problem Solving}
            \begin{itemize}
                \item \textbf{Resource Allocation:} Agents collaborate, e.g., in smart grids for dynamic electricity distribution.
            \end{itemize}

        \item \textbf{Healthcare Systems}
            \begin{itemize}
                \item \textbf{Patient Management:} Agents optimize resource allocation based on patient profiles and urgency.
            \end{itemize}
    \end{enumerate}
\end{frame}

\begin{frame}[fragile]
    \frametitle{Key Concepts and Conclusion}
    \begin{itemize}
        \item \textbf{Inter-agent Communication:} Vital for coordination and task resolution.
        \item \textbf{Scalability and Flexibility:} MAS can easily adapt to new agents and changing environments.
        \item \textbf{Emergence:} Complex behaviors arise from simple interactions; e.g., flocking behaviors in autonomous drones.
    \end{itemize}
    
    \begin{block}{Conclusion}
        Multi-agent systems encompass diverse applications, demonstrating their ability to solve complex tasks efficiently. Understanding these systems prepares students for exploration and implementation in various fields.
    \end{block}
\end{frame}

\begin{frame}[fragile]
    \frametitle{Challenges in Multi-Agent Systems}
    \begin{block}{Learning Objectives}
        \begin{itemize}
            \item Understand key challenges in designing and implementing multi-agent systems (MAS).
            \item Identify real-world implications of these challenges through examples.
        \end{itemize}
    \end{block}
\end{frame}

\begin{frame}[fragile]
    \frametitle{Key Challenges - Communication and Coordination}
    \begin{enumerate}
        \item \textbf{Communication and Coordination}
            \begin{itemize}
                \item \textbf{Explanation}: Agents must communicate effectively to achieve their goals through information sharing and collective decisions.
                \item \textbf{Example}: In a robotic swarm, if one robot detects an obstacle, it must communicate this to others to avoid collisions.
            \end{itemize}
    \end{enumerate}
\end{frame}

\begin{frame}[fragile]
    \frametitle{Key Challenges - Scalability and Autonomy}
    \begin{enumerate}
        \setcounter{enumi}{1}
        \item \textbf{Scalability}
            \begin{itemize}
                \item \textbf{Explanation}: As the number of agents increases, complexity of interactions grows. Solutions must efficiently accommodate varying agent populations.
                \item \textbf{Example}: Managing traffic in smart cities with thousands of autonomous vehicles requires scalable algorithms.
            \end{itemize}
        
        \item \textbf{Agent Autonomy and Interaction}
            \begin{itemize}
                \item \textbf{Explanation}: Each agent acts independently, leading to potential conflicts or cooperation challenges.
                \item \textbf{Example}: In disaster response scenarios, agents must collaborate without central control, risking conflicting decisions.
            \end{itemize}
    \end{enumerate}
\end{frame}

\begin{frame}[fragile]
    \frametitle{Key Challenges - Learning and Fault Tolerance}
    \begin{enumerate}
        \setcounter{enumi}{3}
        \item \textbf{Learning and Adaptability}
            \begin{itemize}
                \item \textbf{Explanation}: Agents should learn and adapt to changes in their environment, adding complexity to their implementation.
                \item \textbf{Example}: In financial systems, agents use reinforcement learning for market adaptation.
            \end{itemize}

        \item \textbf{Reliability and Fault Tolerance}
            \begin{itemize}
                \item \textbf{Explanation}: Systems must be robust against failures to prevent collapse.
                \item \textbf{Example}: In search-and-rescue operations, if one drone fails, others must be capable of completing the mission effectively.
            \end{itemize}
    \end{enumerate}
\end{frame}

\begin{frame}[fragile]
    \frametitle{Key Challenges - Resource Management and Conclusion}
    \begin{enumerate}
        \setcounter{enumi}{5}
        \item \textbf{Resource Management}
            \begin{itemize}
                \item \textbf{Explanation}: Efficiently allocating limited resources among agents is essential.
                \item \textbf{Example}: Charging station distribution for electric cars requires careful planning to avoid congestion.
            \end{itemize}
    \end{enumerate}
    \begin{block}{Conclusion}
        Understanding these challenges is crucial for developing robust and adaptive multi-agent systems in various fields.
    \end{block}
\end{frame}

\begin{frame}[fragile]
    \frametitle{Case Studies}
    \begin{block}{Introduction to Multi-Agent Systems (MAS)}
        Multi-Agent Systems (MAS) consist of multiple interacting agents that operate in a shared environment. These agents can be autonomous entities like robots, software programs, or even people. By collaborating and competing, agents solve complex problems that are challenging for individual agents to manage alone.
    \end{block}
\end{frame}

\begin{frame}[fragile]
    \frametitle{Case Study Highlights - Part 1}
    \begin{enumerate}
        \item \textbf{Smart Grids}
        \begin{itemize}
            \item \textbf{Description:} Agents representing energy producers, consumers, and storage units optimize energy distribution.
            \item \textbf{Implementation Example:} CAISO utilizes a multi-agent system for efficient electricity flow management.
        \end{itemize}

        \item \textbf{Autonomous Vehicles}
        \begin{itemize}
            \item \textbf{Description:} A fleet of autonomous vehicles coordinates movements to avoid collisions.
            \item \textbf{Implementation Example:} Waymo's system shares information among vehicles to enhance safety and efficiency.
        \end{itemize}
    \end{enumerate}
\end{frame}

\begin{frame}[fragile]
    \frametitle{Case Study Highlights - Part 2}
    \begin{enumerate}
        \setcounter{enumi}{2} % continue the enumeration
        \item \textbf{Healthcare Systems}
        \begin{itemize}
            \item \textbf{Description:} Enables agents to collaborate in patient diagnosis and treatment.
            \item \textbf{Implementation Example:} The "Smart Health" project uses agents to share health data for timely interventions.
        \end{itemize}

        \item \textbf{E-Commerce and Marketplaces}
        \begin{itemize}
            \item \textbf{Description:} Agents act as buyers or sellers to negotiate prices based on information.
            \item \textbf{Implementation Example:} eBay employs bots to facilitate transactions and manage bidding.
        \end{itemize}
    \end{enumerate}
\end{frame}

\begin{frame}[fragile]
    \frametitle{Key Points and Conclusion}
    \begin{block}{Key Points to Emphasize}
        \begin{itemize}
            \item Multi-agent systems enhance collaboration and problem-solving by distributing tasks.
            \item Successful implementations arise in domains requiring real-time data exchange.
            \item The ability of agents to learn and adapt is crucial for system effectiveness.
        \end{itemize}
    \end{block}

    \begin{block}{Conclusion}
        The successful case studies illustrate the versatility of MAS in increasing efficiency, safety, and responsiveness. The potential for MAS continues to expand, paving the way for innovative applications.
    \end{block}
\end{frame}

\begin{frame}[fragile]
    \frametitle{Learning Objectives}
    \begin{itemize}
        \item Recognize real-world applications of multi-agent systems.
        \item Analyze the benefits and challenges of implementing these systems.
        \item Inspire future explorations in multi-agent system development.
    \end{itemize}
\end{frame}

\begin{frame}[fragile]
    \frametitle{Future Trends in Multi-Agent Systems}
    \begin{itemize}
        \item Discuss emerging trends and future directions in multi-agent systems research
        \item Focus on improved human-agent collaboration, integration with machine learning, scalable systems, security, and ethics
    \end{itemize}
\end{frame}

\begin{frame}[fragile]
    \frametitle{Emerging Trends}
    \begin{enumerate}
        \item \textbf{Improved Human-Agent Collaboration}
            \begin{itemize}
                \item Agents will understand human emotions, intentions, and context
                \item Example: Collaborative robots (cobots) in manufacturing adapt to human actions
            \end{itemize}
        
        \item \textbf{Integration with Machine Learning}
            \begin{itemize}
                \item Combines deep learning techniques and MAS for better performance
                \item Example: Reinforcement learning optimizes communication among agents in trading
            \end{itemize}
    \end{enumerate}
\end{frame}

\begin{frame}[fragile]
    \frametitle{Scalability and Security}
    \begin{enumerate}
        \setcounter{enumi}{2}
        \item \textbf{Scalable Systems with Edge Computing}
            \begin{itemize}
                \item Vital for operating at the edge with IoT devices
                \item Example: Smart cities using local data for dynamic traffic management
            \end{itemize}

        \item \textbf{Enhanced Security Protocols}
            \begin{itemize}
                \item Critical for protecting MAS against breaches and attacks
                \item Example: Blockchain technology securing multi-agent transactions in logistics
            \end{itemize}

        \item \textbf{Ethical and Societal Considerations}
            \begin{itemize}
                \item Focus on autonomy, decision-making transparency, and accountability
                \item Example: Guidelines for deploying autonomous agents in public spaces
            \end{itemize}
    \end{enumerate}
\end{frame}

\begin{frame}[fragile]
    \frametitle{Key Points to Remember}
    \begin{itemize}
        \item Future MAS systems will prioritize user experience and cooperation
        \item Integration of innovative machine learning methods enhances agent capabilities
        \item MAS will be designed to operate efficiently at scale with local processing
        \item Strong security measures are essential as MAS become prevalent in critical areas
        \item Societal impact and ethical implications must be considered in future research
    \end{itemize}
\end{frame}

\begin{frame}[fragile]
    \frametitle{Summary of Key Takeaways - Part 1}
    \begin{block}{Multi-Agent Systems: Key Concepts}
        \begin{enumerate}
            \item \textbf{Definition of Multi-Agent Systems (MAS):}
                \begin{itemize}
                    \item Consist of multiple autonomous entities (agents).
                    \item Agents interact with each other and their environment to achieve specific goals.
                \end{itemize}
            \item \textbf{Characteristics of Agents:}
                \begin{itemize}
                    \item \textbf{Autonomy:} Agents operate independently.
                    \item \textbf{Social Ability:} Interaction and communication among agents.
                    \item \textbf{Reactivity:} Real-time response to environmental changes.
                    \item \textbf{Proactivity:} Initiative in achieving objectives.
                \end{itemize}
        \end{enumerate}
    \end{block}
\end{frame}

\begin{frame}[fragile]
    \frametitle{Summary of Key Takeaways - Part 2}
    \begin{block}{Types and Architectures of Multi-Agent Systems}
        \begin{enumerate}
            \item \textbf{Types of Multi-Agent Systems:}
                \begin{itemize}
                    \item \textbf{Cooperative MAS:} Agents work together (e.g., robot teams).
                    \item \textbf{Competitive MAS:} Agents act in opposition (e.g., trading agents).
                    \item \textbf{Heterogeneous vs. Homogeneous:} Similar or diverse agents.
                \end{itemize}
            \item \textbf{Architectures:}
                \begin{itemize}
                    \item \textbf{Reactive Systems:} Simple responses to stimuli.
                    \item \textbf{Deliberative Systems:} Use reasoning and internal models.
                    \item \textbf{Hybrid Systems:} Combines both reactive and deliberative characteristics.
                \end{itemize}
        \end{enumerate}
    \end{block}
\end{frame}

\begin{frame}[fragile]
    \frametitle{Summary of Key Takeaways - Part 3}
    \begin{block}{Applications and Challenges}
        \begin{enumerate}
            \item \textbf{Applications:}
                \begin{itemize}
                    \item \textbf{Smart Grids:} Manage energy resources.
                    \item \textbf{Traffic Management:} Adjust signals to minimize congestion.
                    \item \textbf{Game Development:} Use MAS for realistic NPC behaviors.
                \end{itemize}
            \item \textbf{Challenges:}
                \begin{itemize}
                    \item \textbf{Coordination:} Complex interactions in dynamic environments.
                    \item \textbf{Scalability:} Effective management of numerous agents.
                    \item \textbf{Communication:} Protocols for information exchange among agents.
                \end{itemize}
        \end{enumerate}
    \end{block}
\end{frame}

\begin{frame}[fragile]
    \frametitle{Key Points to Emphasize}
    \begin{itemize}
        \item Agents operate independently but can collaborate, enhancing problem-solving capabilities.
        \item Understanding diverse types and architectures is crucial for effective MAS design.
        \item Real-world applications underscore the significance of MAS in technology and research.
    \end{itemize}
\end{frame}

\begin{frame}[fragile]
    \frametitle{Examples and Diagrams}
    \begin{block}{Diagram of Cooperative vs. Competitive Agents}
        \begin{verbatim}
        Cooperative Agents: [Agent A] -- [Agent B]
                 \          /
                  \        /
                   \      /
                       [Goal]

        Competitive Agents: [Agent C]  ↔  [Agent D]
        \end{verbatim}
    \end{block}
    
    \begin{block}{Example Communication Protocol Snippet}
        \begin{lstlisting}[language=Python]
class Agent:
    def communicate(self, message):
        # send message to other agents
        return f"Sending message: {message}"
        \end{lstlisting}
    \end{block}
\end{frame}

\begin{frame}[fragile]
    \frametitle{Discussion Questions - Objectives}
    \begin{itemize}
        \item Foster critical thinking about the role and applications of multi-agent systems (MAS).
        \item Encourage students to explore real-world implications and theoretical foundations of MAS. 
    \end{itemize}
\end{frame}

\begin{frame}[fragile]
    \frametitle{Discussion Questions - 1}
    \begin{enumerate}
        \item \textbf{What defines a Multi-Agent System?}
            \begin{itemize}
                \item Definition of agents: autonomy, interaction, etc.
                \item Differences between single and multi-agent systems.
                \item Examples of MAS in real-life situations: robotic swarms, online gaming, market mechanisms.
            \end{itemize}
    \end{enumerate}
\end{frame}

\begin{frame}[fragile]
    \frametitle{Discussion Questions - 2}
    \begin{enumerate}
        \setcounter{enumi}{1}
        \item \textbf{How do communication protocols affect the efficiency of a Multi-Agent System?}
            \begin{itemize}
                \item Importance of communication in agent collaboration.
                \item Examples of protocols: FIPA standards, ACL.
                \item Impacts on response time and decision-making processes.
            \end{itemize}
        \item \textbf{Can you identify a situation where a Multi-Agent System might outperform a centralized approach?}
            \begin{itemize}
                \item Scenarios with dynamic environments (like transportation systems).
                \item Examples: traffic management systems using autonomous vehicles.
                \item Pros and cons in terms of scalability and flexibility.
            \end{itemize}
    \end{enumerate}
\end{frame}

\begin{frame}[fragile]
    \frametitle{Discussion Questions - 3}
    \begin{enumerate}
        \setcounter{enumi}{3}
        \item \textbf{Discuss the ethical implications of deploying Multi-Agent Systems.}
            \begin{itemize}
                \item Privacy concerns: surveillance systems.
                \item Decision-making in critical systems: healthcare, military applications.
                \item Frameworks for accountability and transparency.
            \end{itemize}
        \item \textbf{How can we measure the performance of Multi-Agent Systems?}
            \begin{itemize}
                \item Possible metrics: efficiency, robustness, adaptability.
                \item Benchmarking against established standards.
                \item Real-world case studies to illustrate performance evaluations.
            \end{itemize}
    \end{enumerate}
\end{frame}

\begin{frame}[fragile]
    \frametitle{Engaging Activity and Conclusion}
    \begin{block}{Engaging Activity}
        \textbf{Group Discussion:}
        \begin{itemize}
            \item Break into small groups and present a scenario where a multi-agent system can be implemented.
            \item Discuss the possible agent interactions and outcomes, focusing on communication, collaboration, and ethical concerns.
        \end{itemize}
    \end{block}

    \begin{block}{Conclusion}
        These questions aim to deepen your understanding of Multi-Agent Systems and their implications in both theoretical and practical domains. Think critically about the answers and be prepared to share your insights.
    \end{block}
\end{frame}

\begin{frame}[fragile]
    \frametitle{Final Thoughts - Overview}
    \begin{block}{Understanding Multi-Agent Systems in AI}
        Multi-agent systems (MAS) are composed of multiple autonomous agents that interact with each other and their environment. This interaction allows agents to achieve both individual and collective goals, resulting in dynamic and complex systems.
    \end{block}
\end{frame}

\begin{frame}[fragile]
    \frametitle{Final Thoughts - Importance in AI}
    \begin{itemize}
        \item \textbf{Collaboration and Coordination:} 
        \begin{itemize}
            \item Tasks may be too complex for a single agent; MAS enables efficient collaboration. For instance, multiple drones can collectively survey an area.
        \end{itemize}
        
        \item \textbf{Distributed Problem Solving:}
        \begin{itemize}
            \item MAS allows large problems to be divided into manageable tasks suitable for different agents, as seen in traffic control systems.
        \end{itemize}

        \item \textbf{Adaptability and Learning:}
        \begin{itemize}
            \item Agents learn from interactions, crucial for applications like financial markets where conditions change rapidly.
        \end{itemize}

        \item \textbf{Scalability:}
        \begin{itemize}
            \item Easy to scale up or down, allowing flexibility in problem-solving, particularly in smart city management.
        \end{itemize}
    \end{itemize}
\end{frame}

\begin{frame}[fragile]
    \frametitle{Final Thoughts - Key Points and Applications}
    \begin{block}{Key Points to Emphasize}
        \begin{itemize}
            \item \textbf{Inter-Agent Communication:} Effective protocols like FIPA ACL facilitate sharing of information and coordination between agents.
            \item \textbf{Trust and Security:} Essential in MAS, especially in environments with conflicting interests among agents.
        \end{itemize}
    \end{block}
    
    \begin{block}{Real-World Applications}
        \begin{itemize}
            \item \textbf{Autonomous Vehicles:} Collaboration among vehicles enhances traffic flow and safety.
            \item \textbf{Distributed Energy Systems:} Agents optimize energy distribution for maximum efficiency.
            \item \textbf{Gaming:} NPCs use MAS principles for realistic, unpredictable behavior.
        \end{itemize}
    \end{block}
\end{frame}


\end{document}