\documentclass[aspectratio=169]{beamer}

% Theme and Color Setup
\usetheme{Madrid}
\usecolortheme{whale}
\useinnertheme{rectangles}
\useoutertheme{miniframes}

% Additional Packages
\usepackage[utf8]{inputenc}
\usepackage[T1]{fontenc}
\usepackage{graphicx}
\usepackage{booktabs}
\usepackage{listings}
\usepackage{amsmath}
\usepackage{amssymb}
\usepackage{xcolor}
\usepackage{tikz}
\usepackage{pgfplots}
\pgfplotsset{compat=1.18}
\usetikzlibrary{positioning}
\usepackage{hyperref}

% Custom Colors
\definecolor{myblue}{RGB}{31, 73, 125}
\definecolor{mygray}{RGB}{100, 100, 100}
\definecolor{mygreen}{RGB}{0, 128, 0}
\definecolor{myorange}{RGB}{230, 126, 34}
\definecolor{mycodebackground}{RGB}{245, 245, 245}

% Set Theme Colors
\setbeamercolor{structure}{fg=myblue}
\setbeamercolor{frametitle}{fg=white, bg=myblue}
\setbeamercolor{title}{fg=myblue}
\setbeamercolor{section in toc}{fg=myblue}
\setbeamercolor{item projected}{fg=white, bg=myblue}
\setbeamercolor{block title}{bg=myblue!20, fg=myblue}
\setbeamercolor{block body}{bg=myblue!10}
\setbeamercolor{alerted text}{fg=myorange}

% Set Fonts
\setbeamerfont{title}{size=\Large, series=\bfseries}
\setbeamerfont{frametitle}{size=\large, series=\bfseries}
\setbeamerfont{caption}{size=\small}
\setbeamerfont{footnote}{size=\tiny}

% Code Listing Style
\lstdefinestyle{customcode}{
  backgroundcolor=\color{mycodebackground},
  basicstyle=\footnotesize\ttfamily,
  breakatwhitespace=false,
  breaklines=true,
  commentstyle=\color{mygreen}\itshape,
  keywordstyle=\color{blue}\bfseries,
  stringstyle=\color{myorange},
  numbers=left,
  numbersep=8pt,
  numberstyle=\tiny\color{mygray},
  frame=single,
  framesep=5pt,
  rulecolor=\color{mygray},
  showspaces=false,
  showstringspaces=false,
  showtabs=false,
  tabsize=2,
  captionpos=b
}
\lstset{style=customcode}

% Custom Commands
\newcommand{\hilight}[1]{\colorbox{myorange!30}{#1}}
\newcommand{\source}[1]{\vspace{0.2cm}\hfill{\tiny\textcolor{mygray}{Source: #1}}}
\newcommand{\concept}[1]{\textcolor{myblue}{\textbf{#1}}}
\newcommand{\separator}{\begin{center}\rule{0.5\linewidth}{0.5pt}\end{center}}

% Footer and Navigation Setup
\setbeamertemplate{footline}{
  \leavevmode%
  \hbox{%
  \begin{beamercolorbox}[wd=.3\paperwidth,ht=2.25ex,dp=1ex,center]{author in head/foot}%
    \usebeamerfont{author in head/foot}\insertshortauthor
  \end{beamercolorbox}%
  \begin{beamercolorbox}[wd=.5\paperwidth,ht=2.25ex,dp=1ex,center]{title in head/foot}%
    \usebeamerfont{title in head/foot}\insertshorttitle
  \end{beamercolorbox}%
  \begin{beamercolorbox}[wd=.2\paperwidth,ht=2.25ex,dp=1ex,center]{date in head/foot}%
    \usebeamerfont{date in head/foot}
    \insertframenumber{} / \inserttotalframenumber
  \end{beamercolorbox}}%
  \vskip0pt%
}

% Turn off navigation symbols
\setbeamertemplate{navigation symbols}{}

% Title Page Information
\title[Text Mining]{Week 13: Advanced Topic - Text Mining}
\author[J. Smith]{John Smith, Ph.D.}
\institute[University Name]{
  Department of Computer Science\\
  University Name\\
  \vspace{0.3cm}
  Email: email@university.edu\\
  Website: www.university.edu
}
\date{\today}

% Document Start
\begin{document}

\frame{\titlepage}

\begin{frame}[fragile]
    \frametitle{Introduction to Text Mining - Overview}
    \begin{block}{What is Text Mining?}
        Text Mining is the process of deriving high-quality information from text. 
        It involves applying Natural Language Processing (NLP) to convert unstructured text data into structured data for easier analysis.
    \end{block}
    
    \begin{block}{Importance in NLP}
        \begin{itemize}
            \item Data Abundance: Leveraging text mining to extract insights from growing text data.
            \item Insight Generation: Identifying trends and relationships allowing businesses to understand customer sentiment.
            \item Automation of Information Retrieval: Enhancing efficiency through automated searching and extracting information.
        \end{itemize}
    \end{block}
\end{frame}

\begin{frame}[fragile]
    \frametitle{Key Applications of Text Mining}
    \begin{itemize}
        \item \textbf{Sentiment Analysis:} Evaluating opinions in text data (e.g., reviews on products).
        \item \textbf{Topic Modeling:} Automatically identifying topics present in a collection of documents.
        \item \textbf{Spam Detection:} Classifying emails as spam or legitimate.
    \end{itemize}
\end{frame}

\begin{frame}[fragile]
    \frametitle{Example Scenario}
    Consider a company analyzing customer reviews on a product to improve its features:
    \begin{itemize}
        \item \textbf{Original Reviews:} "I love the design but it lacks durability."
        \item \textbf{Text Mining Process:}
            \begin{itemize}
                \item \textbf{Sentiment Detection:} "love" (positive), "lacks" (negative).
                \item \textbf{Feature Extraction:} Design and durability are key features mentioned.
            \end{itemize}
        \item \textbf{Outcome:} Focus on enhancing durability in future versions while maintaining design appeal.
    \end{itemize}
\end{frame}

\begin{frame}[fragile]
    \frametitle{Key Points to Remember}
    \begin{itemize}
        \item Text mining transforms unstructured text into actionable insights.
        \item It utilizes NLP techniques to analyze and interpret large datasets.
        \item Applications range from business intelligence to academic research.
    \end{itemize}
    
    \begin{block}{Wrap-up}
        Text mining is integral to modern data analytics, unlocking the potential hidden in textual information and driving informed decision-making across industries.
    \end{block}
\end{frame}

\begin{frame}[fragile]
    \frametitle{What is Text Mining? - Definition}
    \begin{block}{Definition}
        \textbf{Text Mining} is the process of deriving high-quality information from text. It involves converting unstructured text data into structured data for analysis and interpretation. This transformation utilizes various techniques from Natural Language Processing (NLP), statistics, and machine learning to uncover trends, patterns, and insights.
    \end{block}
\end{frame}

\begin{frame}[fragile]
    \frametitle{What is Text Mining? - Importance}
    \begin{block}{Importance}
        Text mining plays a vital role in today’s digital world, where vast amounts of text data are generated daily. By effectively processing this data, organizations can:
        \begin{itemize}
            \item Gain critical insights that inform decision-making
            \item Enhance customer experiences
            \item Foster innovation
        \end{itemize}
    \end{block}
\end{frame}

\begin{frame}[fragile]
    \frametitle{Key Concepts in Text Mining}
    \begin{enumerate}
        \item \textbf{Unstructured Data}: Text data without a predefined format (e.g., emails, social media posts, articles, reviews).
        \item \textbf{Information Extraction}: Identifying specific pieces of information from text (e.g., extracting names or dates).
        \item \textbf{Sentiment Analysis}: Assessing emotions conveyed in text for gauging public opinion.
        \item \textbf{Topic Modeling}: Uncovering hidden thematic structures to categorize large textual datasets.
    \end{enumerate}
\end{frame}

\begin{frame}[fragile]
    \frametitle{Examples of Text Mining Applications}
    \begin{itemize}
        \item \textbf{Healthcare}: Extracting patient information from clinical notes to enhance patient care by identifying health trends.
        \item \textbf{Marketing}: Analyzing customer reviews to gauge satisfaction levels and improve products based on feedback trends.
    \end{itemize}
\end{frame}

\begin{frame}[fragile]
    \frametitle{Text Mining Process Diagram}
    \begin{block}{Diagram Idea}
        A flowchart illustrating the text mining process:
        \begin{enumerate}
            \item Data Collection
            \item Text Preprocessing (Cleaning, Tokenization)
            \item Feature Extraction (TF-IDF, Word Embeddings)
            \item Model Application (Classification, Clustering)
            \item Insights
        \end{enumerate}
    \end{block}
\end{frame}

\begin{frame}[fragile]
    \frametitle{Key Points to Emphasize}
    \begin{itemize}
        \item Text mining is essential for understanding large datasets and converting raw text into actionable insights.
        \item Applications of text mining span various industries, including sentiment analysis in social media and automatic content categorization in publishing.
    \end{itemize}
\end{frame}

\begin{frame}[fragile]
    \frametitle{Conclusion}
    \begin{block}{Conclusion}
        Text mining not only enhances data management but also supports decision-making across different domains. Its ability to unlock hidden value in text data makes it integral to modern data analysis strategies.
    \end{block}
    \begin{block}{Final Thought}
        By incorporating text mining into your analytical toolkit, you can extract powerful insights that drive strategic improvements and innovation.
    \end{block}
\end{frame}

\begin{frame}[fragile]{Applications of Text Mining - Overview}
    Text mining plays a crucial role in analyzing and extracting insights from vast amounts of unstructured text data. In this slide, we discuss three key applications:
    \begin{itemize}
        \item \textbf{Sentiment Analysis}
        \item \textbf{Content Categorization}
        \item \textbf{Information Retrieval}
    \end{itemize}
\end{frame}

\begin{frame}[fragile]{Applications of Text Mining - Sentiment Analysis}
    \textbf{1. Sentiment Analysis}
    \begin{itemize}
        \item \textbf{Definition}: Determining the emotional tone behind a series of words.
        \item \textbf{Purpose}: Commonly applied in social media monitoring, customer feedback analysis, and market research.
        \item \textbf{Example}: Using sentiment analysis on Twitter data about products to gauge customer reactions.
        \item \textbf{Key Techniques}:
        \begin{itemize}
            \item \textbf{Lexicon-based methods}: Use predefined lists of words with associated sentiment scores.
            \item \textbf{Machine learning}: Models trained on labeled datasets (e.g., Naive Bayes, SVM).
        \end{itemize}
        \item \textbf{Illustration}:
        \begin{itemize}
            \item Positive: “I love this product!”
            \item Negative: “This is the worst service ever.”
        \end{itemize}
    \end{itemize}
\end{frame}

\begin{frame}[fragile]{Applications of Text Mining - Content Categorization}
    \textbf{2. Content Categorization}
    \begin{itemize}
        \item \textbf{Definition}: Automatically classifying text into predefined categories.
        \item \textbf{Purpose}: Efficient organization of information for easier retrieval.
        \item \textbf{Example}: News articles sorted into categories like Sports, Politics, Technology.
        \item \textbf{Key Techniques}:
        \begin{itemize}
            \item \textbf{Supervised Learning}: Training classifiers with known labels (e.g., Decision Trees, Random Forests).
            \item \textbf{Clustering Techniques}: Grouping texts without pre-labeled categories (e.g., K-Means).
        \end{itemize}
        \item \textbf{Illustration}:
        \begin{itemize}
            \item Articles about climate change under “Environment,” those about gadgets under “Technology.”
        \end{itemize}
    \end{itemize}
\end{frame}

\begin{frame}[fragile]{Applications of Text Mining - Information Retrieval}
    \textbf{3. Information Retrieval}
    \begin{itemize}
        \item \textbf{Definition}: Finding and retrieving information based on user queries from large databases.
        \item \textbf{Purpose}: Deliver relevant results efficiently, enhancing search engines and databases.
        \item \textbf{Example}: Google Search uses text mining techniques to index and retrieve web pages.
        \item \textbf{Key Techniques}:
        \begin{itemize}
            \item \textbf{TF-IDF}: Determines word importance in a document relative to a corpus.
            \item \textbf{Vector Space Model}: Multi-dimensional representation of documents for similarity searches.
        \end{itemize}
        \item \textbf{Formula}:
        \begin{equation}
            \text{TF-IDF}(t, d) = \text{TF}(t, d) \times \log\left(\frac{N}{\text{DF}(t)}\right)
        \end{equation}
        Where:
        \begin{itemize}
            \item \( \text{TF}(t, d) \) = Term frequency of term \( t \) in document \( d \)
            \item \( N \) = Total number of documents
            \item \( \text{DF}(t) \) = Number of documents containing term \( t \)
        \end{itemize}
    \end{itemize}
\end{frame}

\begin{frame}[fragile]{Key Points to Emphasize}
    \begin{itemize}
        \item Text mining enables organizations to derive meaningful insights from unstructured data.
        \item Diverse applications enhance decision-making processes across industries.
        \item Understanding different methodologies and algorithms is crucial for effective text mining.
    \end{itemize}
    This slide provides a foundational understanding of text mining applications crucial for various sectors, enabling students to explore its real-world impact!
\end{frame}

\begin{frame}[fragile]
    \frametitle{NLP Basics - Introduction}
    \begin{block}{Introduction to Natural Language Processing (NLP)}
        Natural Language Processing (NLP) is a field at the intersection of computer science, artificial intelligence, and linguistics. 
        It focuses on the interaction between computers and humans through natural language.
        The ultimate goal of NLP is to enable computers to understand, interpret, and generate human language in a valuable way.
    \end{block}
\end{frame}

\begin{frame}[fragile]
    \frametitle{NLP Basics - Key Concepts}
    \begin{enumerate}
        \item \textbf{Tokenization}
        \begin{itemize}
            \item Definition: Breaking down text into smaller units called tokens (e.g., words, phrases).
            \item Example: "NLP is fascinating!" tokenized into \texttt{["NLP", "is", "fascinating", "!"]}
        \end{itemize}

        \item \textbf{Morphological Analysis}
        \begin{itemize}
            \item Definition: Study of the structure of words and their meaning components (morphemes).
            \item Example: "unhappiness" can be broken down into "un-" (prefix), "happy" (root), and "-ness" (suffix).
        \end{itemize}

        \item \textbf{Part-of-Speech (POS) Tagging}
        \begin{itemize}
            \item Definition: Assigning parts of speech to each token based on context.
            \item Example: In "The cat sits," "The" is a determiner, "cat" is a noun, and "sits" is a verb.
        \end{itemize}
    \end{enumerate}
\end{frame}

\begin{frame}[fragile]
    \frametitle{NLP Basics - More Key Concepts}
    \begin{enumerate}    
        \item \textbf{Named Entity Recognition (NER)}
        \begin{itemize}
            \item Definition: Identifying and classifying key entities in text into predefined categories.
            \item Example: In "Apple Inc. is based in Cupertino," "Apple Inc." is a company name and "Cupertino" is a location.
        \end{itemize}

        \item \textbf{Sentiment Analysis}
        \begin{itemize}
            \item Definition: Determine the sentiment expressed in text (positive, negative, or neutral).
            \item Example: "I love this movie!" indicates a positive sentiment.
        \end{itemize}

        \item \textbf{Importance of NLP in Text Mining}
        \begin{itemize}
            \item NLP decodes vast amounts of unstructured data, enabling insights and decision-making.
            \item Automates tasks like classification and summarization, improving efficiency.
        \end{itemize}
    \end{enumerate}
\end{frame}

\begin{frame}[fragile]{NLP Basics - Practical Application}
    \begin{block}{Practical Application Example}
        \begin{lstlisting}[language=Python]
# Sample Python Code for Tokenization using NLTK
import nltk
from nltk.tokenize import word_tokenize

text = "Natural Language Processing is an exciting field!"
tokens = word_tokenize(text)
print(tokens)  # Output: ['Natural', 'Language', 'Processing', 'is', 'an', 'exciting', 'field', '!']
        \end{lstlisting}
    \end{block}
\end{frame}

\begin{frame}[fragile]{NLP Basics - Conclusion and Key Points}
    \begin{block}{Key Points to Emphasize}
        \begin{itemize}
            \item NLP is essential for analyzing and deriving insights from text, applicable in various domains such as customer sentiment analysis, information retrieval, and automated summarization.
            \item Understanding the basics of NLP is crucial before diving into advanced text mining techniques.
        \end{itemize}
    \end{block}
    
    \begin{block}{Conclusion}
        NLP serves as the foundational layer for text mining. 
        Understanding how to process and analyze linguistic data is imperative for successful text-based data analytics. 
        The next slide will delve into essential text preprocessing techniques.
    \end{block}
\end{frame}

\begin{frame}[fragile]
    \frametitle{Text Preprocessing Techniques}
    \begin{block}{Overview}
        Text preprocessing is a crucial step in natural language processing (NLP) that prepares raw text for analysis and modeling. 
        Essential techniques include:
        \begin{itemize}
            \item Tokenization
            \item Stopword Removal
            \item Stemming
            \item Lemmatization
        \end{itemize}
    \end{block}
\end{frame}

\begin{frame}[fragile]
    \frametitle{Tokenization}
    \begin{block}{Definition}
        Tokenization is the process of splitting a stream of text into individual units, called tokens. 
    \end{block}
    \begin{block}{Example}
        Input: "Text mining is fascinating!" \\
        Output: \{ "Text", "mining", "is", "fascinating", "!" \}
    \end{block}
    \begin{block}{Key Points}
        \begin{itemize}
            \item It involves dealing with punctuation and case variations.
            \item Commonly performed using libraries, such as NLTK in Python.
        \end{itemize}
    \end{block}
    \begin{lstlisting}[language=Python]
import nltk
from nltk.tokenize import word_tokenize

text = "Text mining is fascinating!"
tokens = word_tokenize(text)
print(tokens)  # Output: ['Text', 'mining', 'is', 'fascinating', '!']
    \end{lstlisting}
\end{frame}

\begin{frame}[fragile]
    \frametitle{Stopword Removal}
    \begin{block}{Definition}
        Stopwords are common words that carry little useful information for text analysis (e.g., "and", "the", "is").
    \end{block}
    \begin{block}{Example}
        Input: \{ "Text", "mining", "is", "fascinating" \} \\
        Output: \{ "Text", "mining", "fascinating" \}
    \end{block}
    \begin{block}{Key Points}
        \begin{itemize}
            \item Helps focus on significant words, improving model performance by reducing noise.
            \item Many NLP libraries provide predefined stopwords lists.
        \end{itemize}
    \end{block}
    \begin{lstlisting}[language=Python]
from nltk.corpus import stopwords

nltk.download('stopwords')
stop_words = set(stopwords.words('english'))
filtered_tokens = [word for word in tokens if word.lower() not in stop_words]
print(filtered_tokens)  # Output: ['Text', 'mining', 'fascinating']
    \end{lstlisting}
\end{frame}

\begin{frame}[fragile]
    \frametitle{Stemming and Lemmatization}
    \begin{block}{Stemming}
        \begin{itemize}
            \item Reduces words to their root form (e.g., "running", "runner" -> "run").
            \item May produce non-words (e.g., "fascin").
        \end{itemize}
        \begin{lstlisting}[language=Python]
from nltk.stem import PorterStemmer

stemmer = PorterStemmer()
stemmed_words = [stemmer.stem(word) for word in filtered_tokens]
print(stemmed_words)  # Output: ['text', 'mine', 'fascin']
        \end{lstlisting}
    \end{block}

    \begin{block}{Lemmatization}
        \begin{itemize}
            \item Reduces words to their base or dictionary form with context consideration.
            \item More semantically correct compared to stemming.
        \end{itemize}
        \begin{lstlisting}[language=Python]
from nltk.stem import WordNetLemmatizer

lemmatizer = WordNetLemmatizer()
lemmatized_words = [lemmatizer.lemmatize(word, pos='a') for word in filtered_tokens]
print(lemmatized_words)  # Output: ['Text', 'mining', 'fascinating']
        \end{lstlisting}
    \end{block}
\end{frame}

\begin{frame}[fragile]
    \frametitle{Conclusion and References}
    \begin{block}{Conclusion}
        Text preprocessing techniques such as:
        \begin{itemize}
            \item Tokenization
            \item Stopword removal
            \item Stemming
            \item Lemmatization
        \end{itemize}
        are essential for transforming raw text data into structured formats for further analysis.
    \end{block}
    \begin{block}{References}
        \begin{itemize}
            \item NLTK (Natural Language Toolkit)
            \item "Speech and Language Processing" by Jurafsky and Martin
        \end{itemize}
    \end{block}
\end{frame}

\begin{frame}[fragile]{Feature Extraction in Text Mining - Introduction}
  \begin{block}{Introduction}
    Feature extraction is a critical step in text mining that transforms raw text data into numerical representations, allowing algorithms to understand and process textual information. 
  \end{block}
  
  \begin{itemize}
    \item Techniques for transformation:
      \begin{itemize}
        \item Bag of Words (BoW)
        \item Term Frequency-Inverse Document Frequency (TF-IDF)
      \end{itemize}
  \end{itemize}
\end{frame}

\begin{frame}[fragile]{Feature Extraction in Text Mining - Bag of Words (BoW)}
  \begin{block}{Bag of Words (BoW)}
    \begin{itemize}
      \item **Concept**: Treats text as a collection of words, ignoring grammar and context.
      \item **How It Works**:
      \begin{itemize}
        \item Tokenization: Split text into words.
        \item Vocabulary Creation: List of unique words.
        \item Feature Vector: Represents each document as a vector of word counts.
      \end{itemize}
    \end{itemize}
  \end{block}

  \begin{block}{Example}
    Documents:
    \begin{enumerate}
      \item "I love programming."
      \item "Programming is fun."
    \end{enumerate}
    
    Vocabulary: ["I", "love", "programming", "is", "fun"]

    \begin{itemize}
      \item Document 1 Vector: [1, 1, 1, 0, 0]
      \item Document 2 Vector: [0, 0, 1, 1, 1]
    \end{itemize}
  \end{block}
\end{frame}

\begin{frame}[fragile]{Feature Extraction in Text Mining - TF-IDF}
  \begin{block}{Term Frequency-Inverse Document Frequency (TF-IDF)}
    \begin{itemize}
      \item **Concept**: Enhances BoW by reducing the weight of common words, emphasizing unique terms.
      \item **How It Works**:
      \begin{itemize}
        \item **Term Frequency (TF)**:
          \[
          TF(t, d) = \frac{\text{number of times term } t \text{ appears in document } d}{\text{total number of terms in } d}
          \]
        \item **Inverse Document Frequency (IDF)**:
          \[
          IDF(t, D) = \log\left(\frac{\text{total number of documents in } D}{\text{number of documents containing term } t}\right)
          \]
        \item **Final TF-IDF Score**:
          \[
          TF\text{-}IDF(t, d, D) = TF(t, d) \times IDF(t, D)
          \]
      \end{itemize}
    \end{itemize}
  \end{block}

  \begin{block}{Example}
    For the word "programming":
    \begin{itemize}
      \item TF for both documents = \(1/3\)
      \item IDF = \(\log(3/2)\)
    \end{itemize}
    
    \begin{itemize}
      \item Document 1 TF-IDF("programming") = \((1/3) \times \log(3/2)\)
      \item Document 2 TF-IDF("programming") = \((1/3) \times \log(3/2)\)
    \end{itemize}
  \end{block}
\end{frame}

\begin{frame}[fragile]{Feature Extraction in Text Mining - Key Points and Closing}
  \begin{block}{Key Points to Emphasize}
    \begin{itemize}
      \item BoW is simple and effective but loses semantic meaning and context.
      \item TF-IDF captures the significance of words based on rarity and relevance.
      \item Both techniques convert unstructured text into structured data suitable for machine learning.
    \end{itemize}
  \end{block}

  \begin{block}{Closing}
    Feature extraction is fundamental in text mining, providing the groundwork for text analysis and building predictive models. It is crucial for deriving insights from textual data.
  \end{block}
\end{frame}

\begin{frame}[fragile]
    \frametitle{Text Representation Models - Overview}
    \begin{block}{Overview of Text Representation in Text Mining}
        Text representation models transform textual data into numerical formats for machine learning algorithms. 
        We will focus on embedding methods such as \textbf{Word2Vec} and \textbf{GloVe} (Global Vectors for Word Representation) that capture contextual meanings effectively.
    \end{block}
\end{frame}

\begin{frame}[fragile]
    \frametitle{Text Representation Models - Key Concepts}
    \begin{enumerate}
        \item \textbf{Text Representation}  
            \begin{itemize}
                \item \textbf{Definition:} Converting text data into a numerical format for processing by machine learning algorithms.
                \item \textbf{Importance:} Influences performance in classification, sentiment analysis, and clustering tasks.
            \end{itemize}

        \item \textbf{Word Embeddings}  
            \begin{itemize}
                \item Captures semantic relationships by placing similar words close together in a high-dimensional space, unlike traditional methods (e.g., Bag of Words) that treat words independently.
            \end{itemize}
    \end{enumerate}
\end{frame}

\begin{frame}[fragile]
    \frametitle{Text Representation Models - Prominent Models}
    \begin{enumerate}
        \item \textbf{Word2Vec}
            \begin{itemize}
                \item Developed by Google using neural networks.
                \item \textbf{Methods:}
                    \begin{itemize}
                        \item \textbf{Continuous Bag of Words (CBOW):} Predicts a target word from its context.
                        \item \textbf{Skip-gram:} Predicts context words from a target word.
                    \end{itemize}
                \item \textbf{Example:} 
                    \[
                    \text{king} - \text{man} + \text{woman} \approx \text{queen}
                    \]
            \end{itemize}
        
        \item \textbf{GloVe}
            \begin{itemize}
                \item Developed by Stanford, it factors the word co-occurrence matrix.
                \item Aims to derive embeddings such that the dot product captures log probabilities of co-occurrences:
                \[
                J = \sum_{i,j=1}^{V} f(x_{ij})(w_i^T w_j + b_i + b_j - \log(x_{ij}))^2
                \]
            \end{itemize}
    \end{enumerate}
\end{frame}

\begin{frame}[fragile]
    \frametitle{Key Points and Applications}
    \begin{block}{Key Points}
        \begin{itemize}
            \item Word embeddings offer nuanced representations capturing rich linguistic features.
            \item Contextual information is crucial for understanding word meanings based on usage.
            \item Effective embeddings enhance performance in various downstream text mining tasks.
        \end{itemize}
    \end{block}

    \begin{block}{Applications}
        \begin{itemize}
            \item Sentiment analysis
            \item Document clustering
            \item Machine Translation
        \end{itemize}
    \end{block}
\end{frame}

\begin{frame}[fragile]
    \frametitle{Next Steps}
    With these advanced text representation models, you are now equipped to move onto supervised learning techniques in text mining, where we will apply these embeddings to real-world classification challenges.
\end{frame}

\begin{frame}[fragile]{Supervised Learning in Text Mining - Overview}
    \begin{block}{Understanding Supervised Learning}
        Supervised Learning is a machine learning approach where models are trained on labeled data. 
        In text mining, this means training with datasets where documents are associated with specific categories (e.g., spam vs. not spam).
    \end{block}
    
    \begin{block}{Goals of Text Classification}
        \begin{itemize}
            \item Identify Document Categories: Classify text into predefined categories.
            \item Predict Labels for Unseen Data: Generalize knowledge to predict labels on new, unseen documents.
        \end{itemize}
    \end{block}
\end{frame}

\begin{frame}[fragile]{Supervised Learning Models}
    \begin{block}{Common Models}
        \begin{enumerate}
            \item \textbf{Logistic Regression}
                \begin{itemize}
                    \item A fundamental model used for binary classification.
                    \item Estimates the probability of a document belonging to a class.
                    \item \begin{equation}
                    P(Y=1|X) = \frac{1}{1 + e^{-(\beta_0 + \beta_1X_1 + ... + \beta_nX_n)}}
                    \end{equation}
                \end{itemize}
            \item \textbf{Support Vector Machines (SVM)}
                \begin{itemize}
                    \item Finds the hyperplane that best separates different classes.
                    \item Can handle linear and non-linear boundaries using the kernel trick.
                    \item \begin{equation}
                    f(X) = W \cdot X + b
                    \end{equation}
                \end{itemize}
        \end{enumerate}
    \end{block}
\end{frame}

\begin{frame}[fragile]{Text Representation Techniques}
    \begin{block}{Key Techniques}
        \begin{itemize}
            \item \textbf{Bag-of-Words (BoW)}: Represents text without considering grammar or word order.
            \item \textbf{TF-IDF (Term Frequency-Inverse Document Frequency)}: Evaluates word importance relative to the corpus; emphasizes informative words and reduces common words' weight.
        \end{itemize}
    \end{block}
    
    \begin{block}{Practical Example}
        Classifying tweets as positive or negative sentiment involves:
        \begin{itemize}
            \item Data Preparation: Collect labeled tweets, tokenize and preprocess the text.
            \item Feature Extraction: Use TF-IDF to convert text to numeric representation.
            \item Model Training: Split data and train a Logistic Regression model.
            \item Prediction: Evaluate accuracy and predict on new tweets.
        \end{itemize}
    \end{block}
\end{frame}

\begin{frame}[fragile]{Conclusion and Key Points}
    \begin{block}{Conclusion}
        Supervised learning techniques such as Logistic Regression and SVM are vital for text classification tasks in text mining. 
        Understanding these models along with effective text representation methods allows for efficient predictive systems.
    \end{block}
    
    \begin{block}{Key Points to Emphasize}
        \begin{itemize}
            \item Model Selection is crucial for performance.
            \item Performance Metrics include accuracy, precision, recall, and F1-score.
            \item Iterative Improvement is essential for refining models.
        \end{itemize}
    \end{block}
\end{frame}

\begin{frame}[fragile]{Code Snippet Example}
    \begin{lstlisting}[language=Python]
from sklearn.model_selection import train_test_split
from sklearn.feature_extraction.text import TfidfVectorizer
from sklearn.linear_model import LogisticRegression
from sklearn import metrics

# Sample data
documents = ["I love this product", "This is the worst thing ever", ...]
labels = [1, 0, ...]

# Train-test split
X_train, X_test, y_train, y_test = train_test_split(documents, labels, test_size=0.2)

# TF-IDF Vectorization
vectorizer = TfidfVectorizer()
X_train_tfidf = vectorizer.fit_transform(X_train)
X_test_tfidf = vectorizer.transform(X_test)

# Train Logistic Regression model
clf = LogisticRegression()
clf.fit(X_train_tfidf, y_train)

# Predictions
predictions = clf.predict(X_test_tfidf)

# Evaluation
print(metrics.classification_report(y_test, predictions))
    \end{lstlisting}
\end{frame}

\begin{frame}[fragile]
    \frametitle{Unsupervised Learning in Text Mining - Overview}
    \begin{block}{Overview of Unsupervised Learning}
        Unsupervised learning is a powerful approach in text mining, allowing the model to learn patterns from unlabelled data. This is particularly useful for discovering hidden structures in text data.
    \end{block}
    
    \begin{itemize}
        \item **Unlabeled Data:** Data without predefined categories (e.g., documents or tweets).
        \item **Objective:** Identify underlying structures, grouping similar documents based on content.
    \end{itemize}
\end{frame}

\begin{frame}[fragile]
    \frametitle{Unsupervised Learning in Text Mining - Clustering Methods}
    \begin{block}{Clustering Methods}
        Two prevalent clustering methods in text mining:
        \begin{itemize}
            \item \textbf{K-Means Clustering}
            \item \textbf{Hierarchical Clustering}
        \end{itemize}
    \end{block}
\end{frame}

\begin{frame}[fragile]
    \frametitle{K-Means Clustering}
    \begin{block}{Overview}
        K-means clustering partitions the dataset into K distinct clusters based on feature similarity.
    \end{block}
    
    \begin{enumerate}
        \item Initialization: Choose K initial centroids (randomly or using k-means++).
        \item Assignment Step: Assign each data point to the nearest centroid based on Euclidean distance.
        \item Update Step: Recalculate centroids by taking the mean of all points in each cluster.
        \item Repeat: Iterate until convergence.
    \end{enumerate}
    
    \begin{block}{Example}
        Grouping news articles into categories such as sports, politics, technology using k-means.
    \end{block}
    
    \begin{equation}
    d(\mathbf{x}, \mathbf{y}) = \sqrt{\sum_{i=1}^{n} (x_i - y_i)^2}
    \end{equation}
\end{frame}

\begin{frame}[fragile]
    \frametitle{Hierarchical Clustering}
    \begin{block}{Overview}
        Hierarchical clustering builds a tree of clusters (dendrogram) illustrating nested grouping of points.
    \end{block}
    
    \begin{itemize}
        \item \textbf{Agglomerative Approach:} Starts with each data point as its cluster, merging closest pairs iteratively.
        \item \textbf{Divisive Approach:} Starts with one cluster, splits into smaller clusters.
    \end{itemize}

    \begin{block}{Example}
        Analyzing social media posts to create a hierarchy of topics (e.g., health, fitness, illness).
    \end{block}
\end{frame}

\begin{frame}[fragile]
    \frametitle{Summary and Key Points}
    \begin{block}{Key Points}
        \begin{itemize}
            \item No labels required for unsupervised techniques.
            \item Applicable in market segmentation, recommendation systems, and topic modeling.
            \item Choice of distance metric impacts clustering outcomes (e.g., cosine similarity vs. Euclidean distance).
        \end{itemize}
    \end{block}
    
    \begin{block}{Conclusion}
        Unsupervised learning through clustering techniques like k-means and hierarchical clustering uncovers hidden patterns and organizes text data effectively.
    \end{block}
\end{frame}

\begin{frame}[fragile]
    \frametitle{K-Means Clustering Example Code}
    \begin{lstlisting}[language=python]
from sklearn.cluster import KMeans
from sklearn.feature_extraction.text import TfidfVectorizer

# Sample Documents
documents = ["This is a sports article.", "Political news today.", 
             "Technology advances in AI.", "Latest sports updates."]

# Convert to TF-IDF representation
vectorizer = TfidfVectorizer()
X = vectorizer.fit_transform(documents)

# Apply K-Means
kmeans = KMeans(n_clusters=2, random_state=42)
kmeans.fit(X)

# Cluster Assignment
print(kmeans.labels_)  # Output cluster labels for each document
    \end{lstlisting}
\end{frame}

\begin{frame}[fragile]{Natural Language Generation (NLG) - Introduction}
    \begin{block}{Introduction to Natural Language Generation}
        Natural Language Generation (NLG) is a subfield of Artificial Intelligence (AI) and Natural Language Processing (NLP) that focuses on the automatic generation of coherent and contextually relevant text from structured data. NLG bridges the gap between data analytics and natural language, enabling systems to communicate insights effectively and understandably.
    \end{block}
\end{frame}

\begin{frame}[fragile]{Natural Language Generation (NLG) - How It Works}
    \begin{enumerate}
        \item \textbf{Data Input:} Input data can be in various structured formats such as databases, spreadsheets, or JSON.
        \item \textbf{Content Selection:} The system determines what data to include based on context and relevance.
        \item \textbf{Document Structuring:} The content is organized into a coherent structure by identifying main points or themes.
        \item \textbf{Sentence Generation:} Using predefined templates or machine learning models, the system generates sentences describing the data.
        \item \textbf{Refinement:} The generated text is refined for grammar, style, and clarity, resulting in a finished piece of writing.
    \end{enumerate}
\end{frame}

\begin{frame}[fragile]{Natural Language Generation (NLG) - Example and Conclusion}
    \begin{block}{Example of NLG}
        Consider a sports analytics application that outputs a summary of a game:
        \begin{quote}
            \textbf{Input Data:} 
            \begin{itemize}
                \item Team A: 3 goals
                \item Team B: 2 goals
                \item Key Players: Player X (2 goals), Player Y (1 assist)
            \end{itemize}
            \textbf{Generated Output:} 
            "Team A won the match against Team B with a score of 3 to 2, thanks to Player X's outstanding performance, scoring 2 of the team's goals, while Player Y provided a crucial assist."
        \end{quote}
    \end{block}
    \begin{block}{Conclusion}
        NLG is a powerful tool in the text mining realm, enhancing the ability of machines to communicate complex data in human-friendly formats, thereby increasing the accessibility and usability of data insights.
    \end{block}
\end{frame}

\begin{frame}[fragile]
    \frametitle{Evaluation of Text Mining Models}
    \begin{block}{Objective}
        Understand key evaluation metrics (Precision, Recall, F1-score, and Confusion Matrix) used to assess the performance of text mining models.
    \end{block}
\end{frame}

\begin{frame}[fragile]
    \frametitle{1. Evaluation Metrics Overview}
    Evaluating the performance of text mining models is crucial to ensure their effectiveness in processing and extracting meaningful concepts from text data. Here, we will discuss four primary metrics:
\end{frame}

\begin{frame}[fragile]
    \frametitle{2. Precision}
    \begin{itemize}
        \item \textbf{Definition:} Measures the accuracy of the positive predictions made by the model. It is the ratio of true positives to the total positive predictions (true positives + false positives).
        \item \textbf{Formula:}  
        \begin{equation}
            \text{Precision} = \frac{TP}{TP + FP}
        \end{equation}
        \item \textbf{Example:} If a spam classifier identifies 80 emails as spam, but only 60 are actually spam:
        \begin{equation}
            \text{Precision} = \frac{60}{80} = 0.75 \, (75\%)
        \end{equation}
    \end{itemize}
\end{frame}

\begin{frame}[fragile]
    \frametitle{3. Recall (Sensitivity)}
    \begin{itemize}
        \item \textbf{Definition:} Quantifies the model's ability to identify all relevant instances. Measures the ratio of true positives to the total actual positives (true positives + false negatives).
        \item \textbf{Formula:}  
        \begin{equation}
            \text{Recall} = \frac{TP}{TP + FN}
        \end{equation}
        \item \textbf{Example:} In the same spam classifier scenario, if there are 100 actual spam emails and the model successfully identifies 60:
        \begin{equation}
            \text{Recall} = \frac{60}{100} = 0.60 \, (60\%)
        \end{equation}
    \end{itemize}
\end{frame}

\begin{frame}[fragile]
    \frametitle{4. F1-Score}
    \begin{itemize}
        \item \textbf{Definition:} The harmonic mean of precision and recall, providing a balance between the two metrics. Useful in cases of imbalanced datasets.
        \item \textbf{Formula:}  
        \begin{equation}
            \text{F1-Score} = 2 \times \frac{\text{Precision} \times \text{Recall}}{\text{Precision} + \text{Recall}}
        \end{equation}
        \item \textbf{Example:} Using the previous precision (0.75) and recall (0.60):
        \begin{equation}
            \text{F1-Score} = 2 \times \frac{0.75 \times 0.60}{0.75 + 0.60} \approx 0.6667 \, (66.67\%)
        \end{equation}
    \end{itemize}
\end{frame}

\begin{frame}[fragile]
    \frametitle{5. Confusion Matrix}
    \begin{itemize}
        \item \textbf{Definition:} A table summarizing the performance of a classification model displaying TP, TN, FP, FN.
        \item \textbf{Illustration:}
        \begin{verbatim}
                         Actual Positive  |  Actual Negative
        ------------------------------------------------------
        Predicted Positive |       TP       |        FP
        Predicted Negative |       FN       |        TN
        \end{verbatim}
        \item \textbf{Interpretation:} 
        \begin{itemize}
            \item TP: Correctly predicted positive instances
            \item TN: Correctly predicted negative instances
            \item FP: Incorrectly predicted positive instances
            \item FN: Incorrectly predicted negative instances
        \end{itemize}
    \end{itemize}
\end{frame}

\begin{frame}[fragile]
    \frametitle{Key Points and Conclusion}
    \begin{itemize}
        \item \textbf{Precision vs. Recall:} 
            \begin{itemize}
                \item Minimizing false positives (precise) might lower recall, and vice versa.
            \end{itemize}
        \item \textbf{F1-Score Usefulness:} 
            \begin{itemize}
                \item Useful when needing a balance between precision and recall, particularly in scenarios with class imbalance.
            \end{itemize}
        \item \textbf{Confusion Matrix Insights:} 
            \begin{itemize}
                \item Provides a more comprehensive view of the model's performance beyond simple accuracy.
            \end{itemize}
    \end{itemize}

    \begin{block}{Conclusion}
        Understanding and applying these evaluation metrics allows for better analysis and refinement of text mining models, ensuring they meet the desired accuracy and relevance in extracting information from text data.
    \end{block}
\end{frame}

\begin{frame}[fragile]
    \frametitle{Next Steps}
    Consider the ethical implications and data privacy concerns in text mining as we move towards our next chapter.
\end{frame}

\begin{frame}[fragile]
    \frametitle{Ethical Considerations in Text Mining}
    \begin{block}{Introduction to Ethical Considerations}
        Text mining presents numerous benefits but raises significant ethical concerns. 
        Addressing these issues is crucial for responsible data use and to foster public trust.
    \end{block}
\end{frame}

\begin{frame}[fragile]
    \frametitle{Key Ethical Issues in Text Mining}
    \begin{enumerate}
        \item \textbf{Data Privacy}
        \begin{itemize}
            \item Explanation: Utilizes personal data from various sources.
            \item Example: Mining customer reviews may expose personal opinions.
        \end{itemize}
        
        \item \textbf{Security Concerns}
        \begin{itemize}
            \item Explanation: Collected data might be vulnerable to breaches.
            \item Example: A healthcare organization must encrypt and securely store patient data.
        \end{itemize}
        
        \item \textbf{Algorithmic Bias}
        \begin{itemize}
            \item Explanation: Algorithms can reflect and amplify biases from training data.
            \item Example: A biased sentiment analysis model produces skewed results.
        \end{itemize}
    \end{enumerate}
\end{frame}

\begin{frame}[fragile]
    \frametitle{Best Practices for Ethical Text Mining}
    \begin{itemize}
        \item \textbf{Obtain Informed Consent:} Secure consent from individuals when using their data.
        \item \textbf{Anonymization:} Remove personally identifiable information (PII) before analysis.
        \item \textbf{Bias Evaluation:} Regularly assess algorithms for bias and improve models.
        \item \textbf{Implement Security Protocols:} Use encryption and access controls to safeguard data.
    \end{itemize}
\end{frame}

\begin{frame}[fragile]
    \frametitle{Summary and Key Points}
    \begin{block}{Summary}
        Ethical considerations in text mining are essential to maintaining integrity and public trust. 
        Addressing data privacy, security, and algorithmic bias is crucial for responsible practices.
    \end{block}

    \begin{itemize}
        \item Ethical issues are critical in text mining.
        \item Data privacy and security should be prioritized.
        \item Awareness of algorithmic bias is necessary.
        \item Best practices contribute to responsible text mining.
    \end{itemize}
\end{frame}

\begin{frame}[fragile]
    \frametitle{References (For further reading)}
    \begin{itemize}
        \item "Big Data: A Revolution That Will Transform How We Live, Work, and Think" by Viktor Mayer-Schönberger and Kenneth Cukier.
        \item "Weapons of Math Destruction" by Cathy O'Neil.
    \end{itemize}
\end{frame}

\begin{frame}[fragile]{Case Studies in Text Mining - Introduction}
    \begin{block}{Overview}
        Text mining extracts valuable insights from textual data. This presentation explores three notable case studies demonstrating the successful implementation of text mining techniques across different industries.
    \end{block}
\end{frame}

\begin{frame}[fragile]{Case Study 1: Healthcare - Predicting Disease Outbreaks}
    \begin{itemize}
        \item \textbf{Background:} Healthcare providers utilize data to anticipate epidemics.
        \item \textbf{Implementation:} Analysis of social media, news, and patient records.
        \item \textbf{Techniques Used:} 
        \begin{itemize}
            \item Natural Language Processing (NLP) to classify symptoms.
            \item Sentiment Analysis to gauge public concern.
        \end{itemize}
        \item \textbf{Outcomes:} 
        \begin{itemize}
            \item Flu outbreak predictions with over 85\% accuracy.
            \item Proactive healthcare responses implemented.
        \end{itemize}
    \end{itemize}
\end{frame}

\begin{frame}[fragile]{Key Points from Case Study 1}
    \begin{itemize}
        \item Text mining enables timely interventions in healthcare.
        \item NLP and sentiment analysis are crucial tools for understanding textual data.
    \end{itemize}
\end{frame}

\begin{frame}[fragile]{Case Study 2: Finance - Risk Assessment in Credit Scoring}
    \begin{itemize}
        \item \textbf{Background:} Traditional credit scoring often overlooks qualitative data.
        \item \textbf{Implementation:} Used text mining on customer feedback, social media, and reviews.
        \item \textbf{Techniques Used:} 
        \begin{itemize}
            \item Topic Modeling to uncover themes in customer feedback.
            \item Classification Algorithms for assessing borrower risk.
        \end{itemize}
        \item \textbf{Outcomes:} 
        \begin{itemize}
            \item 20\% reduction in default rates.
            \item Improved risk prediction through early identification of high-risk applicants.
        \end{itemize}
    \end{itemize}
\end{frame}

\begin{frame}[fragile]{Key Points from Case Study 2}
    \begin{itemize}
        \item Combining qualitative and quantitative data improves decision-making.
        \item Text mining enhances traditional credit scoring models.
    \end{itemize}
\end{frame}

\begin{frame}[fragile]{Case Study 3: Retail - Enhancing Customer Experience}
    \begin{itemize}
        \item \textbf{Background:} Retailers aim to enhance customer satisfaction and marketing strategies.
        \item \textbf{Implementation:} Applied text mining to customer reviews and feedback.
        \item \textbf{Techniques Used:} 
        \begin{itemize}
            \item Sentiment Analysis to assess product satisfaction.
            \item Clustering to group feedback for targeted marketing.
        \end{itemize}
        \item \textbf{Outcomes:} 
        \begin{itemize}
            \item 15\% increase in sales.
            \item Improved customer retention rates.
        \end{itemize}
    \end{itemize}
\end{frame}

\begin{frame}[fragile]{Key Points from Case Study 3}
    \begin{itemize}
        \item Text mining helps businesses understand customer preferences better.
        \item Effective sentiment analysis leads to improved marketing strategies.
    \end{itemize}
\end{frame}

\begin{frame}[fragile]{Conclusion}
    \begin{block}{Summary}
        These case studies highlight diverse applications of text mining in various industries. By utilizing text analysis, organizations can enhance decision-making, improve operational efficiency, and elevate customer experiences. Reflect on how these advancements might influence your work in data mining.
    \end{block}
\end{frame}

\begin{frame}[fragile]
    \frametitle{Summary of Key Learnings - Part 1}
    \begin{block}{1. Introduction to Text Mining}
        \begin{itemize}
            \item \textbf{Definition}: Text mining is the process of deriving high-quality information from text using NLP, data mining, and machine learning.
            \item \textbf{Importance}: Text mining helps analyze vast amounts of unstructured text data from sources like social media and reviews, facilitating informed decision-making across various industries.
        \end{itemize}
    \end{block}
\end{frame}

\begin{frame}[fragile]
    \frametitle{Summary of Key Learnings - Part 2}
    \begin{block}{2. Key Concepts Covered}
        \begin{itemize}
            \item \textbf{Text Preprocessing}:
                \begin{itemize}
                    \item \textit{Tokenization}: Splitting text into words or phrases.
                    \item \textit{Stop Word Removal}: Filtering out common words that add little meaning.
                    \item \textit{Stemming/Lemmatization}: Reducing words to their base form.
                \end{itemize}
            \item \textbf{Feature Extraction}:
                \begin{itemize}
                    \item \textit{Bag of Words}: Represents text as a set of word counts.
                    \item \textit{TF-IDF}: Reflects how important a word is to a document relative to a corpus.
                \end{itemize}
            \item \textbf{Text Classification}:
                \begin{itemize}
                    \item \textit{Sentiment Analysis}: Identifying sentiment as positive, negative, or neutral.
                    \item \textit{Spam Detection}: Classifying emails as spam or non-spam.
                \end{itemize}
        \end{itemize}
    \end{block}
\end{frame}

\begin{frame}[fragile]
    \frametitle{Summary of Key Learnings - Part 3}
    \begin{block}{3. Practical Applications}
        \begin{itemize}
            \item \textbf{Customer Feedback Analysis}: Mining reviews to understand sentiment and improve products.
            \item \textbf{Social Media Monitoring}: Understanding public opinion in real-time.
            \item \textbf{Automated Content Tagging}: Organizing and retrieving documents in large databases.
        \end{itemize}
    \end{block}
    
    \begin{block}{4. Implications for Data Mining}
        \begin{itemize}
            \item \textbf{Enhanced Decision Making}: Deriving actionable insights from unstructured data.
            \item \textbf{Improved Customer Engagement}: Tailoring marketing strategies based on customer interactions.
        \end{itemize}
    \end{block}
    
    \begin{block}{Key Points to Emphasize}
        \begin{itemize}
            \item The significance of preprocessing and feature extraction.
            \item Text mining's broad applicability across various domains.
            \item Continuous advancements in NLP and machine learning. 
        \end{itemize}
    \end{block}
\end{frame}

\begin{frame}[fragile]{Final Project Preparation - Overview}
    \begin{block}{Overview of Text Mining in Your Project}
        Text mining is the process of deriving high-quality information from text. 
        Incorporating text mining techniques can enhance your final project by uncovering patterns, trends, and insights that can make your analysis more impactful.
    \end{block}
\end{frame}

\begin{frame}[fragile]{Final Project Preparation - Guidelines}
    \begin{block}{Guidelines for Incorporating Text Mining Techniques}
        \begin{enumerate}
            \item \textbf{Select Your Textual Data:}
                \begin{itemize}
                    \item Identify a corpus (e.g., social media posts, research articles, customer reviews).
                    \item Ensure your data is relevant to your research question.
                \end{itemize}
            \item \textbf{Preprocessing Data:}
                \begin{itemize}
                    \item \textbf{Tokenization:} Break text into words or sentences.
                    \item \textbf{Normalization:} Convert text to lowercase and remove punctuation.
                    \item \textbf{Stop-word Removal:} Exclude common words (e.g., "and", "the") that do not add value.
                \end{itemize}
            \item \textbf{Feature Extraction:}
                \begin{itemize}
                    \item Use techniques like Term Frequency-Inverse Document Frequency (TF-IDF) to convert text into numerical features for analysis.
                    \item \textbf{Formula:}
                    \begin{equation}
                        \text{TF-IDF}(t, d) = \text{TF}(t, d) \times \log\left(\frac{N}{\text{DF}(t)}\right)
                    \end{equation}
                \end{itemize}
            \item \textbf{Applying Text Mining Techniques:}
                \begin{itemize}
                    \item \textbf{Sentiment Analysis:} Determine the sentiment from the text data.
                    \item \textbf{Topic Modeling:} Use algorithms like Latent Dirichlet Allocation (LDA) to identify topics within your corpus.
                \end{itemize}
        \end{enumerate}
    \end{block}
\end{frame}

\begin{frame}[fragile]{Final Project Preparation - Code Example}
    \begin{block}{Example Code for Preprocessing}
        \begin{lstlisting}[language=Python]
import nltk
from nltk.corpus import stopwords
from nltk.tokenize import word_tokenize

nltk.download('punkt')
nltk.download('stopwords')

text = "Text mining helps discover patterns in text data."
tokens = word_tokenize(text.lower())
filtered_tokens = [word for word in tokens if word not in set(stopwords.words('english'))]
        \end{lstlisting}
    \end{block}
\end{frame}

\begin{frame}[fragile]{Final Project Preparation - Expected Outcomes}
    \begin{block}{Expected Outcomes}
        \begin{itemize}
            \item Gain insights from unstructured text data that inform your conclusions.
            \item Clear visualizations (charts/graphs) summarizing findings effectively.
            \item Develop skills in data preprocessing, analysis, and interpretation relevant in various fields (business, healthcare, social science).
        \end{itemize}
    \end{block}
\end{frame}

\begin{frame}[fragile]{Final Project Preparation - Resources}
    \begin{block}{Resources for Further Learning}
        \begin{itemize}
            \item \textbf{Books:} "Speech and Language Processing" by Jurafsky \& Martin.
            \item \textbf{Online Courses:} Coursera's "Text Mining and Analytics".
            \item \textbf{Libraries:} Familiarize with Python libraries such as NLTK, SpaCy, and Scikit-learn for text mining tasks.
        \end{itemize}
    \end{block}
\end{frame}

\begin{frame}[fragile]{Final Project Preparation - Key Points}
    \begin{block}{Key Points to Emphasize}
        \begin{itemize}
            \item Text mining is a powerful methodological approach to extracting meaningful insights from vast amounts of textual data.
            \item Hands-on practice with real datasets enhances your learning experience and prepares you for future analytical challenges.
        \end{itemize}
    \end{block}
\end{frame}

\begin{frame}[fragile]
    \frametitle{Q\&A and Discussion on Text Mining}
    \begin{block}{Overview of Text Mining}
        Text mining is the process of extracting meaningful information from text using techniques from:
        \begin{itemize}
            \item Natural Language Processing (NLP)
            \item Data Mining
            \item Machine Learning
            \item Statistics
        \end{itemize}
    \end{block}
\end{frame}

\begin{frame}[fragile]
    \frametitle{Key Concepts to Discuss}
    \begin{enumerate}
        \item \textbf{Basic Definitions}:
            \begin{itemize}
                \item \textbf{Natural Language Processing (NLP)}: Interacting with computers using natural language.
                \item \textbf{Sentiment Analysis}: A technique to categorize sentiment as positive, negative, or neutral.
            \end{itemize}
        \item \textbf{Important Techniques}:
            \begin{itemize}
                \item \textbf{Tokenization}: Breaking text into individual terms.
                \item \textbf{Stemming and Lemmatization}: Reducing words to their root form.
                \item \textbf{Vectorization}: Converting text into numerical form using methods like Bag-of-Words, TF-IDF, or Word Embeddings.
            \end{itemize}
    \end{enumerate}
\end{frame}

\begin{frame}[fragile]
    \frametitle{Applications and Discussion Points}
    \begin{block}{Applications}
        \begin{itemize}
            \item Customer Feedback Analysis
            \item Social Media Monitoring
            \item Healthcare Data Analysis
        \end{itemize}
    \end{block}
    
    \begin{block}{Example Discussion Points}
        \begin{enumerate}
            \item Real-world applications in enhancing customer service using text mining.
            \item Challenges of dealing with ambiguity and context in text analysis.
        \end{enumerate}
    \end{block}
\end{frame}

\begin{frame}[fragile]
    \frametitle{Key Questions and Conclusion}
    \begin{block}{Key Questions to Consider}
        \begin{enumerate}
            \item What specific text mining technique do you find most valuable in your field?
            \item How can machine learning enhance traditional text mining methods?
            \item What limitations or ethical considerations exist in applying text mining technologies?
        \end{enumerate}
    \end{block}

    \begin{block}{Conclusion}
        We aim to clarify concepts, share experiences, and explore applications of text mining. Please feel free to pose questions or share insights!
    \end{block}
\end{frame}

\begin{frame}[fragile]{Code Snippet for Basic Text Processing}
\begin{lstlisting}[language=Python]
# Import necessary libraries
import pandas as pd
from sklearn.feature_extraction.text import TfidfVectorizer

# Sample text data
documents = ["I love programming.", "Python is intuitive.", "I enjoy solving problems."]

# Create TF-IDF Vectorizer
vectorizer = TfidfVectorizer()
tfidf_matrix = vectorizer.fit_transform(documents)

# Display the TF-IDF representation
print(pd.DataFrame(tfidf_matrix.toarray(), columns=vectorizer.get_feature_names_out()))
\end{lstlisting}
\begin{itemize}
    \item This code snippet demonstrates vectorization by converting text documents into a TF-IDF matrix, allowing further analysis.
\end{itemize}
\end{frame}


\end{document}