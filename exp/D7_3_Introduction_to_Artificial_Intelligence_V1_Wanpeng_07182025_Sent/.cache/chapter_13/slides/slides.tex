\documentclass[aspectratio=169]{beamer}

% Theme and Color Setup
\usetheme{Madrid}
\usecolortheme{whale}
\useinnertheme{rectangles}
\useoutertheme{miniframes}

% Additional Packages
\usepackage[utf8]{inputenc}
\usepackage[T1]{fontenc}
\usepackage{graphicx}
\usepackage{booktabs}
\usepackage{listings}
\usepackage{amsmath}
\usepackage{amssymb}
\usepackage{xcolor}
\usepackage{tikz}
\usepackage{pgfplots}
\pgfplotsset{compat=1.18}
\usetikzlibrary{positioning}
\usepackage{hyperref}

% Custom Colors
\definecolor{myblue}{RGB}{31, 73, 125}
\definecolor{mygray}{RGB}{100, 100, 100}
\definecolor{mygreen}{RGB}{0, 128, 0}
\definecolor{myorange}{RGB}{230, 126, 34}
\definecolor{mycodebackground}{RGB}{245, 245, 245}

% Set Theme Colors
\setbeamercolor{structure}{fg=myblue}
\setbeamercolor{frametitle}{fg=white, bg=myblue}
\setbeamercolor{title}{fg=myblue}
\setbeamercolor{section in toc}{fg=myblue}
\setbeamercolor{item projected}{fg=white, bg=myblue}
\setbeamercolor{block title}{bg=myblue!20, fg=myblue}
\setbeamercolor{block body}{bg=myblue!10}
\setbeamercolor{alerted text}{fg=myorange}

% Set Fonts
\setbeamerfont{title}{size=\Large, series=\bfseries}
\setbeamerfont{frametitle}{size=\large, series=\bfseries}
\setbeamerfont{caption}{size=\small}
\setbeamerfont{footnote}{size=\tiny}

% Document Start
\begin{document}

\frame{\titlepage}

\begin{frame}[fragile]
    \frametitle{Introduction to Capstone Presentations}
    \begin{block}{Overview}
        Capstone presentations are the final exercise in this course, showcasing student learnings and comprehensive understanding of the material covered.
    \end{block}
\end{frame}

\begin{frame}[fragile]
    \frametitle{Purpose and Key Elements}
    \begin{block}{Purpose of Capstone Presentations}
        The capstone presentation consolidates your knowledge and skills, enabling you to apply theoretical concepts to practical scenarios.
    \end{block}
    
    \begin{block}{Key Elements}
        \begin{itemize}
            \item \textbf{Integration of Knowledge:} Demonstrates connection of various course materials and skills.
            \item \textbf{Practical Application:} Highlights real-world implementation of learned AI tools and concepts.
            \item \textbf{Critical Thinking and Problem Solving:} Showcases ability to address complex problems innovatively.
            \item \textbf{Effective Communication:} Emphasizes clear and engaging articulation of complex information.
        \end{itemize}
    \end{block}
\end{frame}

\begin{frame}[fragile]
    \frametitle{Capstone Project Structure}
    \begin{block}{Project Structure}
        Your capstone presentation should include:
        \begin{enumerate}
            \item \textbf{Introduction:} Introduce your project topic and its significance.
            \item \textbf{Methodology:} Explain the processes and tools used (e.g., AI algorithms).
            \item \textbf{Results:} Present findings using data visualizations, graphs, or models.
            \item \textbf{Conclusion:} Summarize key takeaways and implications for future research.
        \end{enumerate}
    \end{block}

    \begin{block}{Final Thought}
        Capstone presentations are an opportunity to shine—take pride in your work, engage your audience, and prepare for discussion.
    \end{block}
\end{frame}

\begin{frame}[fragile]
    \frametitle{Objectives of Capstone Presentations - Overview}
    Capstone presentations are the culmination of your learning journey in this course. 
    They provide an opportunity to exhibit your understanding of complex AI concepts through practical applications, 
    showcasing both your depth of knowledge and your presentation skills.
\end{frame}

\begin{frame}[fragile]
    \frametitle{Objectives of Capstone Presentations - Key Objectives 1}
    \begin{enumerate}
        \item \textbf{Integration of Knowledge:}
        \begin{itemize}
            \item \textit{Concept Explanation:} Synthesizing learning from various modules to show how different AI concepts interconnect.
            \item \textit{Example:} Discuss the relationship between data preprocessing, feature selection, and model training in a machine learning project.
        \end{itemize}
        
        \item \textbf{Articulation of Complex AI Concepts:}
        \begin{itemize}
            \item \textit{Concept Explanation:} Clearly present intricate ideas, especially in a rapidly-evolving field like AI.
            \item \textit{Example:} Use analogies or illustrations, such as comparing neural networks to human brain processes, for better understanding.
        \end{itemize}
    \end{enumerate}
\end{frame}

\begin{frame}[fragile]
    \frametitle{Objectives of Capstone Presentations - Key Objectives 2}
    \begin{enumerate}
        \setcounter{enumi}{2}
        \item \textbf{Demonstration of Practical Application:}
        \begin{itemize}
            \item \textit{Concept Explanation:} Show how theoretical knowledge translates into real-world applications, emphasizing AI relevance.
            \item \textit{Example:} Illustrate how applying reinforcement learning to optimize logistics can lead to cost reductions and enhanced efficiency.
        \end{itemize}

        \item \textbf{Critical Thinking and Problem-Solving:}
        \begin{itemize}
            \item \textit{Concept Explanation:} Showcase analytical skills by discussing project challenges and solutions.
            \item \textit{Example:} Explain strategies for dealing with data quality issues, such as data augmentation or algorithm selection.
        \end{itemize}

        \item \textbf{Engagement with Audience:}
        \begin{itemize}
            \item \textit{Concept Explanation:} Promote dialogue with peers and faculty to encourage questions and feedback.
            \item \textit{Example:} Prepare anticipated questions and responses regarding methodology to foster engagement.
        \end{itemize}
    \end{enumerate}
\end{frame}

\begin{frame}[fragile]
    \frametitle{Structure of Presentations - Introduction}
    \begin{itemize}
        \item \textbf{Purpose}: Sets the stage and captures the audience's attention.
        \item \textbf{Key Elements}:
        \begin{itemize}
            \item \textbf{Context}: Describe the problem or opportunity your project addresses.
            \item \textbf{Objectives}: State the goals of your research.
            \item \textbf{Hook}: Use a compelling statistic or story to engage.
        \end{itemize}
    \end{itemize}
    \textit{Example}: "Today, we will explore how predictive analytics can enhance healthcare outcomes, leading to a 30\% reduction in misdiagnosis rates."
\end{frame}

\begin{frame}[fragile]
    \frametitle{Structure of Presentations - Methodology}
    \begin{itemize}
        \item \textbf{Purpose}: Outline how you conducted your research, providing validity to your findings.
        \item \textbf{Key Elements}:
        \begin{itemize}
            \item \textbf{Research Design}: Describe the type of research (qualitative, quantitative, etc.).
            \item \textbf{Data Collection}: Explain how data was gathered (surveys, experiments, etc.).
            \item \textbf{Analysis Methods}: Outline the techniques used for data analysis.
        \end{itemize}
    \end{itemize}
    \textit{Example}: "We utilized a mixed-methods approach, combining surveys of healthcare professionals with machine learning algorithms to analyze patient data."
\end{frame}

\begin{frame}[fragile]
    \frametitle{Structure of Presentations - Results, Discussion, and Conclusion}
    \begin{itemize}
        \item \textbf{Results}:
        \begin{itemize}
            \item \textbf{Purpose}: Summarize key findings clearly and concisely.
            \item \textbf{Key Elements}:
            \begin{itemize}
                \item \textbf{Data Presentation}: Use charts, graphs, and tables.
                \item \textbf{Summary of Findings}: Highlight critical results.
            \end{itemize}
        \end{itemize}
        \textit{Example}: "Our analysis revealed that 65\% of patients experienced improved diagnostic accuracy..."
        
        \item \textbf{Discussion}:
        \begin{itemize}
            \item Interpret results and acknowledge limitations.
            \item Discuss implications of findings.
        \end{itemize}
        
        \item \textbf{Conclusion}:
        \begin{itemize}
            \item Restate key findings and suggest future work.
            \item Call to action for the audience.
        \end{itemize}
    \end{itemize}
    \textit{Example}: "In conclusion, leveraging predictive analytics can revolutionize healthcare diagnostics..."
\end{frame}

\begin{frame}[fragile]
    \frametitle{Demonstrating AI Techniques - Introduction}
    \begin{block}{Overview}
        Artificial Intelligence (AI) encompasses various techniques such as Machine Learning (ML), Natural Language Processing (NLP), and Data Analysis. This presentation showcases practical applications of these techniques, highlighting their relevance and impact.
    \end{block}
\end{frame}

\begin{frame}[fragile]
    \frametitle{Demonstrating AI Techniques - Machine Learning}
    \begin{itemize}
        \item \textbf{Machine Learning (ML)}
            \begin{itemize}
                \item \textbf{Definition}: A subset of AI that enables systems to learn from data patterns and improve performance over time.
                \item \textbf{Example}: Predictive Modeling
                \begin{itemize}
                    \item Algorithms such as Decision Trees and Random Forests can forecast sales based on historical data.
                \end{itemize}
            \end{itemize}
        \item \textbf{Mathematical Representation}
            \begin{equation}
            Y = \beta_0 + \beta_1X_1 + \beta_2X_2 + ... + \beta_nX_n + \epsilon
            \end{equation}
            where $Y$ represents the target variable, $X$ are the predictors, $\beta$ are the coefficients, and $\epsilon$ is the error term.
    \end{itemize}
\end{frame}

\begin{frame}[fragile]
    \frametitle{Demonstrating AI Techniques - NLP and Data Analysis}
    \begin{itemize}
        \item \textbf{Natural Language Processing (NLP)}
            \begin{itemize}
                \item \textbf{Definition}: Focuses on the interaction between computers and human languages for processing and analyzing textual data.
                \item \textbf{Example}: Sentiment Analysis
                \begin{itemize}
                    \item Companies analyze customer reviews to gauge public opinion on products.
                \end{itemize}
            \end{itemize}
        \item \textbf{Code Snippet (Python)}
        \begin{lstlisting}[language=Python]
        from textblob import TextBlob
        
        text = "I love this product! It works great and has exceeded my expectations."
        analysis = TextBlob(text)
        sentiment = analysis.sentiment.polarity  # Ranges from -1 (negative) to 1 (positive)
        print(f"Sentiment Polarity: {sentiment}")
        \end{lstlisting}
    \end{itemize}

    \begin{itemize}
        \item \textbf{Data Analysis}
            \begin{itemize}
                \item \textbf{Definition}: The process of inspecting and modeling data to discover useful information.
                \item \textbf{Example}: Exploratory Data Analysis (EDA)
                \begin{itemize}
                    \item Identifying trends and correlations using visualization tools like Matplotlib or Seaborn.
                \end{itemize}
            \end{itemize}
        \item \textbf{Diagram of EDA:}
        Raw Data $\rightarrow$ Data Cleaning $\rightarrow$ Data Visualization $\rightarrow$ Insights
    \end{itemize}
\end{frame}

\begin{frame}[fragile]
    \frametitle{Demonstrating AI Techniques - Conclusion}
    \begin{itemize}
        \item \textbf{Key Points to Emphasize}
        \begin{itemize}
            \item \textbf{Interconnectedness}: AI techniques often work together; data analysis can enhance ML models by selecting essential features.
            \item \textbf{Real-World Impact}: Applications in healthcare, finance, and retail demonstrate AI's versatility.
            \item \textbf{Continuous Learning}: Encourage exploration of evolving AI techniques and their methodologies.
        \end{itemize}
        \item \textbf{Conclusion}: In your capstone presentations, highlight how these AI techniques solve real-world problems to convey both theoretical knowledge and practical significance.
    \end{itemize}
\end{frame}

\begin{frame}[fragile]
    \frametitle{Collaboration and Teamwork}
    \begin{block}{Importance of Collaboration in Group Projects}
        Collaboration is essential for successful group projects, especially in complex fields like AI. It allows for skill sharing and diverse perspectives, leading to more robust solutions.
    \end{block}
\end{frame}

\begin{frame}[fragile]
    \frametitle{Benefits of Collaboration}
    \begin{enumerate}
        \item \textbf{Enhanced Creativity} 
            \begin{itemize}
                \item Different viewpoints foster innovative ideas.
                \item \textit{Example:} Combining machine learning with natural language processing in an AI project.
            \end{itemize}
        
        \item \textbf{Improved Problem Solving}
            \begin{itemize}
                \item Collective brainstorming aids effective troubleshooting.
                \item \textit{Example:} Team members debugging code collaboratively.
            \end{itemize}
        
        \item \textbf{Increased Accountability}
            \begin{itemize}
                \item Regular check-ins encourage members to meet deadlines.
                \item \textit{Example:} Accountability through scheduled progress updates.
            \end{itemize}

        \item \textbf{Skill Development}
            \begin{itemize}
                \item Collaborating enhances individual skills.
                \item \textit{Example:} A programmer learns data visualization from a designer.
            \end{itemize}
    \end{enumerate}
\end{frame}

\begin{frame}[fragile]
    \frametitle{Best Practices for Effective Teamwork}
    \begin{enumerate}
        \item \textbf{Define Roles Clearly} 
            \begin{itemize}
                \item Assign roles according to strengths.
                \item \textit{Example:} Project manager, developer, and designer roles.
            \end{itemize}

        \item \textbf{Establish Communication Protocols}
            \begin{itemize}
                \item Use tools and regular meetings for updates.
                \item \textit{Example:} Slack for quick updates and Zoom for discussions.
            \end{itemize}

        \item \textbf{Practice Together}
            \begin{itemize}
                \item Hold rehearsals to get familiar with the presentation.
                \item \textit{Example:} Schedule mock presentations to refine content delivery.
            \end{itemize}

        \item \textbf{Embrace Constructive Feedback}
            \begin{itemize}
                \item Create an environment for open feedback.
                \item \textit{Example:} Discussing improvements after practice sessions.
            \end{itemize}

        \item \textbf{Utilize Technology Wisely}
            \begin{itemize}
                \item Use collaborative tools for efficiency.
                \item \textit{Example:} Google Docs for writing, Trello for task management.
            \end{itemize}
    \end{enumerate}
\end{frame}

\begin{frame}[fragile]
    \frametitle{Key Takeaways}
    \begin{itemize}
        \item Collaboration enriches project outcomes through diverse skills and ideas.
        \item Clearly defined roles, effective communication, and practice are vital for success.
        \item Embrace feedback and use technology to enhance collaboration.
    \end{itemize}
    \begin{block}{Conclusion}
        Fostering a collaborative culture can significantly boost team performance in project presentations requiring cohesive output.
    \end{block}
\end{frame}

\begin{frame}[fragile]
    \frametitle{Ethical Considerations in AI - Overview}
    \begin{block}{Objective}
        To understand and discuss the ethical implications of AI solutions in our projects, ensuring responsible and fair applications.
    \end{block}
    
    \begin{block}{Key Concepts}
        \begin{itemize}
            \item \textbf{Ethics in AI}: Moral principles guiding the development and utilization of AI technology.
            \item \textbf{Key Ethical Principles}:
                \begin{itemize}
                    \item Transparency
                    \item Fairness
                    \item Accountability
                    \item Privacy
                \end{itemize}
        \end{itemize}
    \end{block}
\end{frame}

\begin{frame}[fragile]
    \frametitle{Ethical Considerations in AI - Examples}
    \begin{block}{Examples of Ethical Considerations}
        \begin{itemize}
            \item \textbf{Facial Recognition Technology}: Enhances security but may lead to racial profiling and privacy infringements.
            \item \textbf{Algorithmic Bias}: Biased data can result in unfair practices, such as discriminatory hiring.
        \end{itemize}
    \end{block}
    
    \begin{block}{Discussion Points}
        \begin{itemize}
            \item Identifying Bias: Consider how data selection and model training introduce ethical challenges.
            \item Ethical Frameworks: Introduce frameworks such as Beneficence, Non-maleficence, and Justice.
        \end{itemize}
    \end{block}
\end{frame}

\begin{frame}[fragile]
    \frametitle{Ethical Assessment Formula}
    \begin{block}{Formula for Ethical Assessment}
        Risk Assessment:
        \begin{equation}
            \text{Risk Level} = \frac{\text{Likelihood of Harm} \times \text{Severity of Harm}}{\text{Mitigation Measures}}
        \end{equation}
        Use this formula to evaluate the potential impact of AI solutions in your projects.
    \end{block}
    
    \begin{block}{Key Points to Emphasize}
        \begin{itemize}
            \item Ethical implications must be foundational in project proposals.
            \item Collaboration is crucial for addressing ethical issues.
            \item Continuous monitoring post-implementation is necessary for ongoing ethical compliance.
        \end{itemize}
    \end{block}
    
    \begin{block}{Conclusion}
        Incorporating ethical considerations enhances the value and acceptance of AI solutions.
    \end{block}
\end{frame}

\begin{frame}[fragile]
    \frametitle{Presentation Skills - Learning Objectives}
    \begin{itemize}
        \item Understand key communication techniques for effective presentations.
        \item Develop strategies for engaging the audience and handling questions confidently.
    \end{itemize}
\end{frame}

\begin{frame}[fragile]
    \frametitle{Effective Communication During Presentations - Key Techniques}
    \begin{block}{Clarity and Structure}
        \begin{itemize}
            \item \textbf{Organize Your Content}: Begin with an introduction, followed by main points, and conclude with a summary.
            \item \textbf{Use Clear Language}: Avoid jargon unless essential, and explain technical terms.
        \end{itemize}
    \end{block}

    \begin{block}{Engaging the Audience}
        \begin{itemize}
            \item \textbf{Start with a Hook}: Capture attention with a question or a relevant story.
            \item \textbf{Incorporate Visual Aids}: Use slides, charts, and videos to illustrate points.
            \item \textbf{Interact with Your Audience}: 
                \begin{itemize}
                    \item Ask rhetorical questions.
                    \item Use live polls to gather feedback.
                \end{itemize}
        \end{itemize}
    \end{block}
\end{frame}

\begin{frame}[fragile]
    \frametitle{Managing Questions and Key Points}
    \begin{block}{Body Language and Voice}
        \begin{itemize}
            \item \textbf{Confident Posture}: Stand tall, shoulders back, avoid crossing arms.
            \item \textbf{Eye Contact}: Establish rapport by making eye contact with various audience members.
            \item \textbf{Vocal Variety}: Change tone, pitch, and pace to maintain interest.
        \end{itemize}
    \end{block}

    \begin{block}{Managing Questions}
        \begin{itemize}
            \item \textbf{Anticipate Questions}: Prepare for common queries related to your topic.
            \item \textbf{Repeat Questions}: Ensure everyone hears and understands before you answer.
            \item \textbf{Stay Calm and Composed}: It's okay to admit if you don't know an answer.
        \end{itemize}
    \end{block}

    \begin{block}{Conclusion}
        Effective presentation skills enhance how your ideas are received. By organizing your content and engaging your audience, you create a memorable presentation experience.
    \end{block}
\end{frame}

\begin{frame}[fragile]
    \frametitle{Assessment Criteria - Overview}
    \begin{block}{Objective}
        Understand the evaluation rubric for capstone presentations, focusing on the key aspects: clarity, depth, and engagement.
    \end{block}
\end{frame}

\begin{frame}[fragile]
    \frametitle{Assessment Criteria - Clarity}
    \begin{enumerate}
        \item \textbf{Clarity}  
        \begin{itemize}
            \item \textbf{Structure:} A clear presentation has a logical flow with an introduction, body, and conclusion.
            \item \textbf{Language:} Use straightforward language and avoid jargon. Visual aids should enhance understanding.
            \item \textbf{Example:} 
            \begin{quote}
                "In our study, we analyzed solar panel efficiency through a series of tests…"
            \end{quote}
        \end{itemize}
    \end{enumerate}
\end{frame}

\begin{frame}[fragile]
    \frametitle{Assessment Criteria - Depth \& Engagement}
    \begin{enumerate}
        \setcounter{enumi}{1} % Continue numbering from the previous frame
        \item \textbf{Depth}  
        \begin{itemize}
            \item \textbf{Research \& Analysis:} Demonstrate comprehensive understanding supported by data.
            \item \textbf{Critical Thinking:} Provide unique perspectives and address counterarguments with evidence.
            \item \textbf{Example:} Discuss both the economic and environmental impacts of solar energy adoption.
        \end{itemize}

        \item \textbf{Engagement}  
        \begin{itemize}
            \item \textbf{Interaction:} Encourage audience participation through questions or discussions.
            \item \textbf{Body Language \& Delivery:} Use confident body language, eye contact, and vocal variation.
            \item \textbf{Example:} 
            \begin{quote}
                "What obstacles do you think hinder the widespread use of solar energy?"
            \end{quote}
        \end{itemize}
    \end{enumerate}
\end{frame}

\begin{frame}[fragile]
    \frametitle{Group Dynamics and Roles - Understanding Group Dynamics}
    
    Group dynamics refers to the interactions and behaviors between individuals working collectively in a group setting. In a capstone presentation where collaboration is key, understanding these dynamics can significantly enhance both the process and outcome.
    
    \begin{itemize}
        \item Effective group dynamics encourage participation, foster creativity, and leverage diverse perspectives.
    \end{itemize}

\end{frame}

\begin{frame}[fragile]
    \frametitle{Group Dynamics and Roles - Distribution of Roles}
    
    \begin{block}{Role Assignment}
        Clearly defining roles within the group is essential to ensure that all members contribute effectively. Common roles include:
    \end{block}
    
    \begin{enumerate}
        \item **Leader/Facilitator**: Organizes meetings and coordinates efforts.
        \item **Researcher**: Gathers information and conducts analysis.
        \item **Content Creator**: Develops the presentation materials.
        \item **Presenter**: Delivers the content to the audience.
        \item **Timekeeper**: Monitors the presentation duration.
    \end{enumerate}

\end{frame}

\begin{frame}[fragile]
    \frametitle{Group Dynamics and Roles - Key Points and Conclusion}
    
    \begin{block}{Key Points to Emphasize}
        \begin{itemize}
            \item **Everyone Matters**: Ensuring active participation leads to a comprehensive presentation.
            \item **Conflict Resolution**: Foster open communication to strengthen dynamics.
            \item **Feedback Loop**: Regular feedback aligns efforts and improves output.
        \end{itemize}
    \end{block}
    
    \begin{block}{Conclusion}
        Understanding and managing group dynamics through effective role distribution is crucial for successful capstone presentations. Valuing each member's contribution fosters an innovative and engaging collaborative environment.
    \end{block}    

\end{frame}

\begin{frame}[fragile]
    \frametitle{Conclusion and Next Steps - Overview of Key Takeaways}
    
    \begin{block}{Collaborative Problem Solving}
        \begin{itemize}
            \item Teams showcased the ability to work cohesively, leveraging diverse skills.
            \item Example: Group A used natural language processing and machine learning to build an intelligent chatbot.
        \end{itemize}
    \end{block}
    
    \begin{block}{Application of AI Tools and Frameworks}
        \begin{itemize}
            \item Projects utilized frameworks like Scikit-Learn, TensorFlow, and PyTorch effectively.
            \item Example: Group B developed a predictive analytics model with TensorFlow.
        \end{itemize}
    \end{block}
    
    \begin{block}{Ethical Considerations in AI}
        \begin{itemize}
            \item Teams addressed the societal impacts of their solutions.
            \item Example: Group C analyzed biases in AI algorithms, emphasizing fairness.
        \end{itemize}
    \end{block}
\end{frame}

\begin{frame}[fragile]
    \frametitle{Conclusion and Next Steps - Future Applications}
    
    \begin{itemize}
        \item \textbf{Industry Readiness:} Skills are applicable in sectors like healthcare and finance.
        \item \textbf{Innovation in AI Fields:} Explore niche areas such as reinforcement learning or GANs.
        \begin{itemize}
            \item Example: GANs can innovate artistic applications or data augmentation.
        \end{itemize}
        \item \textbf{Continuous Learning:} Engage in lifelong learning through online courses.
        \begin{itemize}
            \item Consider platforms like Coursera or edX for specialized topics.
        \end{itemize}
    \end{itemize}
\end{frame}

\begin{frame}[fragile]
    \frametitle{Conclusion and Next Steps - Emphasis Points and Next Steps}
    
    \begin{itemize}
        \item Collaboration and teamwork are crucial for successful AI outcomes.
        \item Ethical considerations must be prioritized in AI systems.
        \item Ongoing education will enhance adaptability in the evolving AI landscape.
    \end{itemize}
    
    \textbf{Next Steps:}
    \begin{itemize}
        \item Engage with feedback from presentations to refine skills.
        \item Develop a portfolio showcasing your projects for job applications.
        \item Network with peers and professionals for internship or job opportunities.
    \end{itemize}
    
    \textbf{Concluding Remarks:} Your journey in this AI course prepares you for innovative contributions in diverse fields. Stay curious, ethical, and committed to growth in your AI journey!
\end{frame}


\end{document}