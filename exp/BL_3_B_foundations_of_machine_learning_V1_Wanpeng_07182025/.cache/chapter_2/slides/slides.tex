\documentclass[aspectratio=169]{beamer}

% Theme and Color Setup
\usetheme{Madrid}
\usecolortheme{whale}
\useinnertheme{rectangles}
\useoutertheme{miniframes}

% Additional Packages
\usepackage[utf8]{inputenc}
\usepackage[T1]{fontenc}
\usepackage{graphicx}
\usepackage{booktabs}
\usepackage{listings}
\usepackage{amsmath}
\usepackage{amssymb}
\usepackage{xcolor}
\usepackage{tikz}
\usepackage{pgfplots}
\pgfplotsset{compat=1.18}
\usetikzlibrary{positioning}
\usepackage{hyperref}

% Custom Colors
\definecolor{myblue}{RGB}{31, 73, 125}
\definecolor{mygray}{RGB}{100, 100, 100}
\definecolor{mygreen}{RGB}{0, 128, 0}
\definecolor{myorange}{RGB}{230, 126, 34}
\definecolor{mycodebackground}{RGB}{245, 245, 245}

% Set Theme Colors
\setbeamercolor{structure}{fg=myblue}
\setbeamercolor{frametitle}{fg=white, bg=myblue}
\setbeamercolor{title}{fg=myblue}
\setbeamercolor{section in toc}{fg=myblue}
\setbeamercolor{item projected}{fg=white, bg=myblue}
\setbeamercolor{block title}{bg=myblue!20, fg=myblue}
\setbeamercolor{block body}{bg=myblue!10}
\setbeamercolor{alerted text}{fg=myorange}

% Set Fonts
\setbeamerfont{title}{size=\Large, series=\bfseries}
\setbeamerfont{frametitle}{size=\large, series=\bfseries}
\setbeamerfont{caption}{size=\small}
\setbeamerfont{footnote}{size=\tiny}

% Code Listing Style
\lstdefinestyle{customcode}{
  backgroundcolor=\color{mycodebackground},
  basicstyle=\footnotesize\ttfamily,
  breakatwhitespace=false,
  breaklines=true,
  commentstyle=\color{mygreen}\itshape,
  keywordstyle=\color{blue}\bfseries,
  stringstyle=\color{myorange},
  numbers=left,
  numbersep=8pt,
  numberstyle=\tiny\color{mygray},
  frame=single,
  framesep=5pt,
  rulecolor=\color{mygray},
  showspaces=false,
  showstringspaces=false,
  showtabs=false,
  tabsize=2,
  captionpos=b
}
\lstset{style=customcode}

% Custom Commands
\newcommand{\hilight}[1]{\colorbox{myorange!30}{#1}}
\newcommand{\source}[1]{\vspace{0.2cm}\hfill{\tiny\textcolor{mygray}{Source: #1}}}
\newcommand{\concept}[1]{\textcolor{myblue}{\textbf{#1}}}
\newcommand{\separator}{\begin{center}\rule{0.5\linewidth}{0.5pt}\end{center}}

% Footer and Navigation Setup
\setbeamertemplate{footline}{
  \leavevmode%
  \hbox{%
  \begin{beamercolorbox}[wd=.3\paperwidth,ht=2.25ex,dp=1ex,center]{author in head/foot}%
    \usebeamerfont{author in head/foot}\insertshortauthor
  \end{beamercolorbox}%
  \begin{beamercolorbox}[wd=.5\paperwidth,ht=2.25ex,dp=1ex,center]{title in head/foot}%
    \usebeamerfont{title in head/foot}\insertshorttitle
  \end{beamercolorbox}%
  \begin{beamercolorbox}[wd=.2\paperwidth,ht=2.25ex,dp=1ex,center]{date in head/foot}%
    \usebeamerfont{date in head/foot}
    \insertframenumber{} / \inserttotalframenumber
  \end{beamercolorbox}}%
  \vskip0pt%
}

% Turn off navigation symbols
\setbeamertemplate{navigation symbols}{}

% Title Page Information
\title[Mathematical Foundations]{Chapter 2: Mathematical Foundations}
\subtitle{Essential Concepts for Machine Learning}
\author[J. Smith]{John Smith, Ph.D.}
\institute[University Name]{
  Department of Computer Science\\
  University Name\\
  \vspace{0.3cm}
  Email: email@university.edu\\
  Website: www.university.edu
}
\date{\today}

% Document Start
\begin{document}

\frame{\titlepage}

\begin{frame}[fragile]
    \titlepage
\end{frame}

\begin{frame}[fragile]
    \frametitle{Overview of Mathematical Concepts in Machine Learning}
    \begin{block}{Importance of Mathematics}
        Mathematics serves as the backbone of machine learning, providing essential tools for developing algorithms and understanding their behavior. 
        By mastering these foundations, students improve their ability to design, analyze, and interpret machine learning models effectively.
    \end{block}
\end{frame}

\begin{frame}[fragile]
    \frametitle{Key Mathematical Areas - Part 1}
    \begin{enumerate}
        \item \textbf{Linear Algebra}
            \begin{itemize}
                \item \textbf{Concept}: Studies vectors, matrices, and transformations.
                \item \textbf{Example}: Data often takes the form of matrices, with rows as samples and columns as features.
                \item \textbf{Key Point}: Matrix operations, like multiplication and eigendecomposition, are crucial for techniques like PCA.
            \end{itemize}
        \item \textbf{Calculus}
            \begin{itemize}
                \item \textbf{Concept}: Studies change through derivatives and integrals.
                \item \textbf{Example}: Gradient descent minimizes loss functions using derivatives.
                \item \textbf{Key Point}: Understanding derivatives helps in grasping how model parameters shape performance.
            \end{itemize}
    \end{enumerate}
\end{frame}

\begin{frame}[fragile]
    \frametitle{Key Mathematical Areas - Part 2}
    \begin{enumerate}
        \setcounter{enumi}{2} % Continue enumeration from the previous frame
        \item \textbf{Probability and Statistics}
            \begin{itemize}
                \item \textbf{Concept}: Studies uncertainty and data analysis.
                \item \textbf{Example}: Models like Naive Bayes use conditional probabilities for predictions.
                \item \textbf{Key Point}: Understanding distributions, expectation, and variance aids in interpreting performance metrics.
            \end{itemize}
        \item \textbf{Optimization}
            \begin{itemize}
                \item \textbf{Concept}: Finds best parameters to minimize a cost function.
                \item \textbf{Example}: Linear regression adjusts parameters to minimize squared errors.
                \item \textbf{Key Point}: Algorithms like stochastic gradient descent are essential for training models.
            \end{itemize}
    \end{enumerate}
\end{frame}

\begin{frame}[fragile]
    \frametitle{Conclusion and Additional Resources}
    \begin{block}{Conclusion}
        A strong mathematical foundation enables students to navigate machine learning complexities and develop analytical skills for data-driven problems. We will explore each area in-depth, ensuring relevance and application in machine learning contexts.
    \end{block}
    
    \begin{block}{Additional Resources}
        \begin{itemize}
            \item Books on Linear Algebra (e.g., "Linear Algebra and Its Applications" by Gilbert Strang)
            \item Calculus textbooks (e.g., "Calculus: Early Transcendentals" by James Stewart)
            \item Online courses on Probability and Statistics (e.g., Coursera or Khan Academy)
        \end{itemize}
    \end{block}
\end{frame}

\begin{frame}[fragile]
    \frametitle{Next Steps}
    \begin{block}{Preparation}
        Prepare for a deeper exploration of the learning objectives in our upcoming slides.
    \end{block}
    
    \begin{block}{Reminder}
        Engaging actively with mathematical principles is critical to mastering machine learning!
    \end{block}
\end{frame}

\begin{frame}[fragile]
    \frametitle{Learning Objectives - Overview}
    \begin{block}{Learning Objectives}
        Upon completion of this chapter, students will be able to:
    \end{block}
\end{frame}

\begin{frame}[fragile]
    \frametitle{Learning Objectives - Key Mathematical Concepts}
    \begin{enumerate}
        \item \textbf{Understand Key Mathematical Concepts}
        \begin{itemize}
            \item Grasp essential mathematical terminology and notations, especially in machine learning (vectors, matrices, functions).
            \item Recognize the significance of these concepts in formulating algorithms and models.
        \end{itemize}
        
        \item \textbf{Apply Mathematical Techniques to Problem Solving}
        \begin{itemize}
            \item Use algebraic techniques to manipulate and solve equations relevant to data analysis.
            \item Implement geometric interpretations of data points and transformations, including linear combinations in machine learning contexts.
        \end{itemize}
    \end{enumerate}
\end{frame}

\begin{frame}[fragile]
    \frametitle{Learning Objectives - Critical Thinking and Collaboration}
    \begin{enumerate}\setcounter{enumi}{2}
        \item \textbf{Develop Critical Thinking and Analytical Skills}
        \begin{itemize}
            \item Analyze mathematical problems to determine the best approach for solutions.
            \item Engage in discussions on the implications of mathematical decisions in modeling and their impact on machine learning outcomes.
        \end{itemize}
        
        \item \textbf{Collaborate and Communicate Effectively in Teams}
        \begin{itemize}
            \item Work with peers to discuss and solve mathematical problems, fostering teamwork.
            \item Present mathematical findings clearly to facilitate understanding in discussions.
        \end{itemize}
    \end{enumerate}
\end{frame}

\begin{frame}[fragile]
    \frametitle{Key Concepts to Master}
    \begin{itemize}
        \item \textbf{Linear Algebra}
        \begin{itemize}
            \item Understand vectors and matrices; calculate dot products, matrix multiplication, and use matrix inverses.
            \item \textit{Example:} If $A$ is a matrix and $x$ is a vector, the product $A \cdot x$ represents a transformation of the vector space.
        \end{itemize}
        
        \item \textbf{Calculus}
        \begin{itemize}
            \item Familiarity with derivatives and integrals is crucial for optimizing algorithms.
            \item \textit{Formula:} The derivative of a function \( f(x) \) represents the slope of the function at any point:
            \begin{equation}
                f'(x) = \lim_{{h \to 0}} \frac{{f(x+h) - f(x)}}{h}
            \end{equation}
        \end{itemize}
        
        \item \textbf{Probability and Statistics}
        \begin{itemize}
            \item Basic concepts such as distributions, mean, variance, and standard deviation are vital for understanding data behaviors and predictions.
        \end{itemize}
    \end{itemize}
\end{frame}

\begin{frame}[fragile]
    \frametitle{Emphases for Practical Application}
    \begin{itemize}
        \item \textbf{Real-world Examples}
        \begin{itemize}
            \item Discuss the use of linear regression for predictions, grounded in matrix operations.
        \end{itemize}
        
        \item \textbf{Problem Sets}
        \begin{itemize}
            \item Regular assignments that encourage the application of mathematical concepts to hypothetical scenarios in machine learning.
        \end{itemize}
    \end{itemize}
\end{frame}

\begin{frame}[fragile]
    \frametitle{Linear Algebra Basics - Introduction}
    \begin{itemize}
        \item This presentation covers the fundamentals of linear algebra.
        \item Focus areas include:
        \begin{itemize}
            \item Vectors
            \item Matrices
            \item Key operations relevant to machine learning
        \end{itemize}
    \end{itemize}
\end{frame}

\begin{frame}[fragile]
    \frametitle{Linear Algebra Basics - Introduction to Vectors}
    \begin{itemize}
        \item \textbf{Definition}: A vector is a mathematical object with both magnitude and direction. It represents data points in multi-dimensional space.
        \item \textbf{Notation}: Vectors are denoted by boldface letters (e.g., \textbf{v}) or with an arrow above (e.g., $\vec{v}$).
        \item \textbf{Example}: A vector in 2D space:
        \[
        \mathbf{v} = \begin{pmatrix}
        x \\
        y
        \end{pmatrix}
        \]
    \end{itemize}
\end{frame}

\begin{frame}[fragile]
    \frametitle{Linear Algebra Basics - Introduction to Matrices}
    \begin{itemize}
        \item \textbf{Definition}: A matrix is a rectangular array of numbers arranged in rows and columns, serving as a collection of vectors.
        \item \textbf{Notation}: Matrices are written in capital letters (e.g., \textbf{A}).
        \item \textbf{Example}: A 2x2 matrix:
        \[
        \mathbf{A} = \begin{pmatrix}
        a_{11} & a_{12} \\
        a_{21} & a_{22}
        \end{pmatrix}
        \]
        \item \textbf{Use in Machine Learning}: Matrices represent datasets, with rows as data points and columns as features.
    \end{itemize}
\end{frame}

\begin{frame}[fragile]
    \frametitle{Linear Algebra Basics - Key Operations}
    \begin{enumerate}
        \item \textbf{Vector Addition}:
        \[
        \mathbf{u} + \mathbf{v} = \begin{pmatrix}
        u_1 \\
        u_2
        \end{pmatrix} + \begin{pmatrix}
        v_1 \\
        v_2
        \end{pmatrix} = \begin{pmatrix}
        u_1 + v_1 \\
        u_2 + v_2
        \end{pmatrix}
        \]
        
        \item \textbf{Scalar Multiplication}:
        \[
        c \mathbf{v} = c \begin{pmatrix}
        x \\
        y
        \end{pmatrix} = \begin{pmatrix}
        cx \\
        cy
        \end{pmatrix}
        \]

        \item \textbf{Matrix Addition}:
        \[
        \mathbf{A} + \mathbf{B} = \begin{pmatrix}
        a_{11} & a_{12} \\
        a_{21} & a_{22}
        \end{pmatrix} + \begin{pmatrix}
        b_{11} & b_{12} \\
        b_{21} & b_{22}
        \end{pmatrix} = \begin{pmatrix}
        a_{11} + b_{11} & a_{12} + b_{12} \\
        a_{21} + b_{21} & a_{22} + b_{22}
        \end{pmatrix}
        \]
        
        \item \textbf{Matrix-Vector Multiplication}:
        \[
        \mathbf{A} \mathbf{v} = \begin{pmatrix}
        a_{11} & a_{12} \\
        a_{21} & a_{22}
        \end{pmatrix} \begin{pmatrix}
        x \\
        y
        \end{pmatrix} = \begin{pmatrix}
        a_{11}x + a_{12}y \\
        a_{21}x + a_{22}y
        \end{pmatrix}
        \]
    \end{enumerate}
\end{frame}

\begin{frame}[fragile]
    \frametitle{Linear Algebra Basics - Key Points and Summary}
    \begin{itemize}
        \item \textbf{Dimensions are Crucial}: Ensure proper conditions on dimensions for operations.
        \item \textbf{Geometric Interpretation}: Visualize vectors as arrows; interpret matrix operations as transformations.
        \item \textbf{Applications in Machine Learning}: Manipulating vectors and matrices is essential for data preprocessing and algorithm implementation.
    \end{itemize}
    
    \textbf{Summary}: Understanding vectors and matrices, along with their operations, lays the groundwork for more advanced machine learning topics.
\end{frame}

\begin{frame}[fragile]{Matrix Operations - Introduction}
    \begin{block}{Introduction to Matrix Operations}
        Matrices are foundational elements in many fields, particularly in algorithms for machine learning, computer graphics, and data analysis. 
        Understanding how to perform and apply matrix operations is crucial for computational tasks.
    \end{block}
\end{frame}

\begin{frame}[fragile]{Matrix Operations - Key Operations}
    \begin{block}{Key Operations}
        \begin{enumerate}
            \item \textbf{Matrix Addition}
            \begin{itemize}
                \item \textbf{Definition}: Two matrices can be added together if they have the same dimensions.
                \item \textbf{Formula}: 
                \[
                C = A + B \quad \text{where } C[i][j] = A[i][j] + B[i][j]
                \]
            \end{itemize}
            
            \item \textbf{Matrix Multiplication}
            \begin{itemize}
                \item \textbf{Definition}: The product of two matrices is defined when the number of columns in the first matrix matches the number of rows in the second matrix.
                \item \textbf{Formula}: 
                \[
                C = A \times B \quad \text{where } C[i][j] = \sum_{k} A[i][k] \times B[k][j]
                \]
            \end{itemize}
        \end{enumerate}
    \end{block}
\end{frame}

\begin{frame}[fragile]{Matrix Operations - Examples}
    \begin{block}{Example of Matrix Addition}
        Let 
        \[
        A = \begin{bmatrix}
        1 & 2 \\
        3 & 4
        \end{bmatrix}, \quad 
        B = \begin{bmatrix}
        5 & 6 \\
        7 & 8
        \end{bmatrix}
        \]
        Then,
        \[
        C = A + B = \begin{bmatrix}
        6 & 8 \\
        10 & 12
        \end{bmatrix}
        \]
    \end{block}
    
    \begin{block}{Example of Matrix Multiplication}
        Let 
        \[
        A = \begin{bmatrix}
        1 & 2 \\
        3 & 4
        \end{bmatrix}, \quad 
        B = \begin{bmatrix}
        5 & 6 \\
        7 & 8
        \end{bmatrix}
        \]
        Then,
        \[
        C = A \times B = \begin{bmatrix}
        19 & 22 \\
        43 & 50
        \end{bmatrix}
        \]
    \end{block}
\end{frame}

\begin{frame}[fragile]{Matrix Operations - Significance}
    \begin{block}{Significance in Algorithms}
        \begin{itemize}
            \item \textbf{Linear Regression}: Matrices are used to represent datasets and perform operations to derive the coefficients for prediction.
            \item \textbf{Neural Networks}: Operations such as matrix multiplication are fundamental in propagating inputs through layers and adjusting weights during training.
        \end{itemize}
    \end{block}
    
    \begin{block}{Key Points to Emphasize}
        \begin{itemize}
            \item Matrix operations like addition and multiplication serve as building blocks for complex algorithms.
            \item Ensure matrices have compatible dimensions before performing operations to avoid errors.
            \item Matrices can efficiently handle large datasets, enabling faster computations and clearer representations of data relationships.
        \end{itemize}
    \end{block}
\end{frame}

\begin{frame}[fragile]{Matrix Operations - Summary}
    \begin{block}{Summary}
        Mastering matrix operations is essential for pursuing advanced topics in computer science and artificial intelligence. 
        Understanding how to manipulate and apply matrices lays the groundwork for more complex mathematical modeling and algorithm development.
    \end{block}
\end{frame}

\begin{frame}[fragile]
    \frametitle{Vector Spaces and Transformations - Understanding Vector Spaces}
    \begin{block}{Definition}
        A \textbf{vector space} (or linear space) is a collection of vectors that can be added together and multiplied by scalars while satisfying specific properties. Formally, a set \( V \) is a vector space if:
    \end{block}
    \begin{itemize}
        \item \textbf{Closure under Addition}: For any vectors \( \mathbf{u}, \mathbf{v} \in V \), the vector \( \mathbf{u} + \mathbf{v} \in V \).
        \item \textbf{Closure under Scalar Multiplication}: For any vector \( \mathbf{u} \in V \) and any scalar \( c \), the vector \( c\mathbf{u} \in V \).
    \end{itemize}
    \begin{block}{Key Properties}
        \begin{itemize}
            \item \textbf{Additive Identity}: There exists a vector \( \mathbf{0} \) in \( V \) such that \( \mathbf{u} + \mathbf{0} = \mathbf{u} \).
            \item \textbf{Additive Inverses}: For every \( \mathbf{u} \in V \), there exists \( -\mathbf{u} \in V \) such that \( \mathbf{u} + (-\mathbf{u}) = \mathbf{0} \).
        \end{itemize}
    \end{block}
    \begin{block}{Example}
        The set of all 2D vectors \( \mathbb{R}^2 \).
    \end{block}
\end{frame}

\begin{frame}[fragile]
    \frametitle{Vector Spaces and Transformations - Bases and Dimension}
    \begin{block}{Bases}
        A \textbf{basis} of a vector space is a set of vectors that are linearly independent and span the entire space. The number of vectors in a basis is called the \textbf{dimension} of the vector space.
    \end{block}
    \begin{itemize}
        \item \textbf{Span}: A set of vectors spans a space if any vector in the space can be expressed as a combination of these vectors.
    \end{itemize}
    \begin{block}{Example}
        In \( \mathbb{R}^2 \), the vectors 
        \[
        \begin{pmatrix} 1 \\ 0 \end{pmatrix} \text{ and } \begin{pmatrix} 0 \\ 1 \end{pmatrix}
        \]
        form a basis.
    \end{block}
\end{frame}

\begin{frame}[fragile]
    \frametitle{Vector Spaces and Transformations - Linear Independence and Transformations}
    \begin{block}{Linear Independence}
        A set of vectors \( \{\mathbf{v}_1, \mathbf{v}_2, \ldots, \mathbf{v}_n\} \) is \textbf{linearly independent} if the only solution to the equation
        \[
        c_1 \mathbf{v}_1 + c_2 \mathbf{v}_2 + \ldots + c_n \mathbf{v}_n = \mathbf{0}
        \]
        is \( c_1 = c_2 = \ldots = c_n = 0 \).
    \end{block}
    \begin{itemize}
        \item \textbf{Example}: The vectors 
        \[
        \begin{pmatrix} 1 \\ 2 \end{pmatrix} \text{ and } \begin{pmatrix} 2 \\ 4 \end{pmatrix}
        \]
        are linearly dependent because the second vector is a multiple of the first.
    \end{itemize}
    \begin{block}{Transformations}
        A \textbf{linear transformation} \( T: V \to W \) satisfies:
        \begin{itemize}
            \item \( T(\mathbf{u} + \mathbf{v}) = T(\mathbf{u}) + T(\mathbf{v}) \) for all \( \mathbf{u}, \mathbf{v} \in V \).
            \item \( T(c\mathbf{u}) = cT(\mathbf{u}) \) for any scalar \( c \) and vector \( \mathbf{u} \in V \).
        \end{itemize}
        \begin{block}{Example}
            The transformation \( T(x, y) = (2x, 3y) \) is linear since it scales each component by a different factor.
        \end{block}
    \end{block}
\end{frame}

\begin{frame}[fragile]
    \frametitle{Probability Theory}
    \begin{block}{Overview}
        Foundational concepts of probability, random variables, and probability distributions.
    \end{block}
\end{frame}

\begin{frame}[fragile]
    \frametitle{1. Introduction to Probability}
    \begin{block}{Definition}
        Probability is a measure of the likelihood that an event will occur, ranging from 0 (impossible event) to 1 (certain event).
    \end{block}
    
    \begin{itemize}
        \item \textbf{Experiment:} A process that generates results (e.g., rolling a die).
        \item \textbf{Outcome:} A possible result of an experiment (e.g., rolling a 4).
        \item \textbf{Event:} A set of outcomes (e.g., rolling an even number).
    \end{itemize}
\end{frame}

\begin{frame}[fragile]
    \frametitle{2. Basic Probability Principles}
    \begin{itemize}
        \item \textbf{Probability of an Event (P):} 
        \begin{equation}
        P(A) = \frac{\text{Number of favorable outcomes}}{\text{Total number of outcomes}}
        \end{equation}
        \textbf{Example:} The probability of rolling a 4 on a fair six-sided die is \( P(4) = \frac{1}{6} \).
        
        \item \textbf{Complement of an Event:}
        \begin{equation}
        P(A') = 1 - P(A)
        \end{equation}
        \textbf{Example:} The probability of not rolling a 4 is \( P(\text{not 4}) = 1 - \frac{1}{6} = \frac{5}{6} \).
    \end{itemize}
\end{frame}

\begin{frame}[fragile]
    \frametitle{3. Random Variables}
    \begin{block}{Definition}
        A random variable is a numerical outcome of a random phenomenon.
    \end{block}
    
    \begin{itemize}
        \item \textbf{Types:}
        \begin{itemize}
            \item \textbf{Discrete Random Variables:} Countable values (e.g., number of goals in a soccer match).
            \item \textbf{Continuous Random Variables:} Infinite values (e.g., the height of students).
        \end{itemize}
        
        \item \textbf{Notation:} Let \( X \) represent a random variable. \( P(X = x) \) denotes the probability that \( X \) takes on the value \( x \).
    \end{itemize}
\end{frame}

\begin{frame}[fragile]
    \frametitle{4. Probability Distributions}
    \begin{block}{Definition}
        A probability distribution describes how probabilities are distributed among the possible values of a random variable.
    \end{block}
    
    \begin{itemize}
        \item \textbf{Discrete Probability Distribution:} 
            \begin{itemize}
                \item Represented by a probability mass function (PMF). 
                \item \textbf{Example:} For a fair die, the PMF \( P(X=x) \) is: 
                \[
                P(X=1) = P(X=2) = P(X=3) = P(X=4) = P(X=5) = P(X=6) = \frac{1}{6}
                \]
            \end{itemize}

        \item \textbf{Continuous Probability Distribution:}
            \begin{itemize}
                \item Represented by a probability density function (PDF).
                \item \textbf{Example:} The normal distribution is defined as:
                \begin{equation}
                f(x) = \frac{1}{\sigma\sqrt{2\pi}} e^{-\frac{(x - \mu)^2}{2\sigma^2}}
                \end{equation}
            \end{itemize}
    \end{itemize}
\end{frame}

\begin{frame}[fragile]
    \frametitle{5. Key Points and Applications}
    \begin{itemize}
        \item Understand the basic principles of probability and random variables.
        \item Familiarity with probability distributions is crucial for modeling real-world phenomena.
        \item Recognize the significance of PMF (for discrete variables) and PDF (for continuous variables).
    \end{itemize}
    
    \begin{block}{Applications in Machine Learning}
        Probability theory is foundational for algorithms, decision making, classification, and regression tasks in machine learning. Understanding data distributions is crucial for model selection and evaluation.
    \end{block}
\end{frame}

\begin{frame}[fragile]
    \frametitle{Conclusion}
    Probability theory provides essential tools for understanding uncertainty and making predictions. It lays the groundwork for advanced topics in data science and machine learning, bridging concepts to the next slide on important probability distributions.
    
    \begin{block}{Note}
        Ensure clear understanding through discussions and practical exercises regarding random variables and probability distributions before transitioning to the next slide.
    \end{block}
\end{frame}

\begin{frame}[fragile]
    \frametitle{Important Probability Distributions}
    \begin{block}{Overview of Key Distributions}
        Understanding probability distributions is crucial to statistical analysis and machine learning (ML) applications. 
    \end{block}
    We will discuss:
    \begin{itemize}
        \item Normal Distribution
        \item Binomial Distribution
    \end{itemize}
\end{frame}

\begin{frame}[fragile]
    \frametitle{Normal Distribution}
    \begin{block}{Definition}
        The Normal distribution, also known as the Gaussian distribution, is a continuous probability distribution characterized by its bell-shaped curve. It is defined by two parameters: the mean ($\mu$) and the standard deviation ($\sigma$).
    \end{block}
    \begin{block}{Formula}
        \begin{equation}
            f(x) = \frac{1}{\sigma\sqrt{2\pi}} e^{-\frac{(x - \mu)^2}{2\sigma^2}}
        \end{equation}
    \end{block}
    \begin{itemize}
        \item Symmetrical around the mean.
        \item Approximately 68\% of the data fall within one standard deviation ($\sigma$) of the mean.
        \item Many natural phenomena (e.g., heights, test scores) are normally distributed.
    \end{itemize}
    \begin{block}{Applications in ML}
        \begin{itemize}
            \item Feature normalization: Crucial for algorithms like Logistic Regression and Neural Networks.
            \item Assumptions in statistical modeling: Many techniques assume normality for hypothesis testing.
        \end{itemize}
    \end{block}
\end{frame}

\begin{frame}[fragile]
    \frametitle{Binomial Distribution}
    \begin{block}{Definition}
        The Binomial distribution represents the number of successes in a fixed number of independent Bernoulli trials (each trial has two outcomes: success or failure).
    \end{block}
    \begin{itemize}
        \item Parameters: 
            \begin{itemize}
                \item $n$: Number of trials
                \item $p$: Probability of success on each trial
            \end{itemize}
    \end{itemize}
    \begin{block}{Formula}
        \begin{equation}
            P(X = k) = \binom{n}{k} p^k (1-p)^{n-k}
        \end{equation}
    \end{block}
    \begin{itemize}
        \item Key Properties:
            \begin{itemize}
                \item Discrete distribution representing successes in $n$ trials.
                \item Mean ($\mu$) = $np$, Variance ($\sigma^2$) = $np(1-p)$.
            \end{itemize}
    \end{itemize}
    \begin{block}{Applications in ML}
        \begin{itemize}
            \item Classification tasks: Used in Naive Bayes models.
            \item A/B testing: Determining success rates across groups.
        \end{itemize}
    \end{block}
\end{frame}

\begin{frame}[fragile]
    \frametitle{Key Points to Emphasize}
    \begin{itemize}
        \item Normal Distribution: Essential for continuous data analysis and forms the basis for many statistical methods.
        \item Binomial Distribution: Vital for understanding discrete outcomes and predicting success/failure scenarios.
    \end{itemize}
    \begin{block}{Conclusion}
        Mastering these distributions enhances understanding of statistical concepts and equips you with tools for effective machine learning implementation.
    \end{block}
    \begin{block}{Discussion Question}
        Can you think of a scenario in machine learning that could leverage the normal or binomial distribution?
    \end{block}
\end{frame}

\begin{frame}[fragile]
    \frametitle{Statistics Fundamentals}
    \begin{block}{Overview}
        Statistics is vital for analyzing data and drawing meaningful conclusions. This slide covers five essential statistical measures: Mean, Median, Mode, Variance, and Standard Deviation.
    \end{block}
\end{frame}

\begin{frame}[fragile]
    \frametitle{Mean (Average)}
    \begin{itemize}
        \item \textbf{Definition}: The mean is the sum of all data points divided by the number of points.
        \begin{equation}
            \text{Mean} (\mu) = \frac{\sum_{i=1}^{n} x_i}{n}
        \end{equation}
        \item \textbf{Example}: Consider the data set: 4, 8, 6, 5, 3.
        \begin{itemize}
            \item Mean = (4 + 8 + 6 + 5 + 3) / 5 = 26 / 5 = 5.2
        \end{itemize}
    \end{itemize}
\end{frame}

\begin{frame}[fragile]
    \frametitle{Median and Mode}
    \begin{itemize}
        \item \textbf{Median}:
        \begin{itemize}
            \item \textbf{Definition}: The median is the middle value when arranged in ascending order.
            \item \textbf{Example}: For the set: 3, 4, 5, 6, 8, Median = 5.
            \item For the set: 3, 4, 5, 6, Median = (4 + 5) / 2 = 4.5.
        \end{itemize}
        
        \item \textbf{Mode}:
        \begin{itemize}
            \item \textbf{Definition}: The mode is the value that appears most frequently.
            \item \textbf{Example}: For the set: 1, 2, 2, 3, 4, Mode = 2.
        \end{itemize}
    \end{itemize}
\end{frame}

\begin{frame}[fragile]
    \frametitle{Variance and Standard Deviation}
    \begin{itemize}
        \item \textbf{Variance}:
        \begin{itemize}
            \item \textbf{Definition}: Measures how much values spread out from the mean.
            \begin{equation}
                \text{Variance} (\sigma^2) = \frac{\sum_{i=1}^{n} (x_i - \mu)^2}{n}
            \end{equation}
            \item \textbf{Example}: For 2, 4, 6: Mean = 4; Variance = [(2-4)² + (4-4)² + (6-4)²] / 3 = 2.67.
        \end{itemize}

        \item \textbf{Standard Deviation}:
        \begin{itemize}
            \item \textbf{Definition}: The square root of variance.
            \begin{equation}
                \text{Standard Deviation} (\sigma) = \sqrt{\text{Variance}}
            \end{equation}
            \item \textbf{Example}: Standard Deviation = √2.67 ≈ 1.63.
        \end{itemize}
    \end{itemize}
\end{frame}

\begin{frame}[fragile]
    \frametitle{Key Points and Conclusion}
    \begin{itemize}
        \item \textbf{Importance}: Understanding these measures aids in data analysis and informed decision-making.
        \item \textbf{Contextual Use}: Different measures can reveal varying aspects of data. For instance, the mean is sensitive to outliers, while the median better represents skewed distributions.
    \end{itemize}
    \begin{block}{Conclusion}
        These fundamental statistical measures form the basis for more advanced statistical methods and will be discussed in the next slide.
    \end{block}
\end{frame}

\begin{frame}[fragile]
    \frametitle{Statistical Inference - Overview}
    \begin{block}{Overview of Statistical Inference}
        Statistical inference involves drawing conclusions about a population based on a sample. It allows us to:
        \begin{itemize}
            \item Make estimates
            \item Test hypotheses
            \item Provide predictions while accounting for uncertainty
        \end{itemize}
    \end{block}
\end{frame}

\begin{frame}[fragile]
    \frametitle{Statistical Inference - Key Concepts}
    \begin{block}{Hypothesis Testing}
        \begin{itemize}
            \item \textbf{Definition}: A method for making decisions about a population parameter based on sample data.
            \item \textbf{Types of Hypotheses}:
                \begin{itemize}
                    \item \textbf{Null Hypothesis (H0)}: The statement being tested (e.g., no effect).
                    \item \textbf{Alternative Hypothesis (H1)}: The statement we want to prove (e.g., an effect exists).
                \end{itemize}
            \item \textbf{Example}: Testing a new drug's effect on blood pressure.
                \begin{itemize}
                    \item H0: The drug has no effect (mean change = 0).
                    \item H1: The drug has an effect (mean change $\neq 0$).
                \end{itemize}
        \end{itemize}
    \end{block}
\end{frame}

\begin{frame}[fragile]
    \frametitle{Statistical Inference - Key Concepts (cont.)}
    \begin{block}{Confidence Intervals}
        \begin{itemize}
            \item \textbf{Definition}: A range to estimate an unknown population parameter with a specified level of confidence.
            \item \textbf{Formula}:
            \begin{equation}
                \text{Confidence Interval} = \hat{p} \pm Z_{\alpha/2} \cdot \sqrt{\frac{\hat{p}(1 - \hat{p})}{n}}
            \end{equation}
            where \( \hat{p} \) is the sample proportion, \( Z_{\alpha/2} \) is the Z-value (e.g., 1.96 for 95%), and \( n \) is the sample size.
            \item \textbf{Example}: If 60\% support a candidate with a ±3\% margin, the 95\% confidence interval is [57\%, 63\%].
        \end{itemize}
    \end{block}
    
    \begin{block}{Significance Levels}
        \begin{itemize}
            \item \textbf{Definition}: The probability of rejecting the null hypothesis when it is true (Type I error), denoted as \( \alpha \).
            \item \textbf{Typical Values}: 0.05 (5\%), 0.01 (1\%).
            \item \textbf{Interpretation}: An \( \alpha \) of 0.05 indicates a 5\% chance of a Type I error.
            \item \textbf{Example}: If the p-value is 0.03 and \( \alpha = 0.05 \), we reject H0 (statistically significant).
        \end{itemize}
    \end{block}
\end{frame}

\begin{frame}[fragile]
    \frametitle{Key Points to Emphasize}
    \begin{itemize}
        \item Hypothesis testing aids in decision-making with sample data.
        \item Confidence intervals provide a likely range for the true parameter value.
        \item The significance level is crucial for determining threshold for null hypothesis rejection.
    \end{itemize}

    \begin{block}{Conclusion}
        Understanding these concepts is vital across various fields, enhancing critical thinking and analytical skills in data interpretation.
    \end{block}

    \begin{block}{Discussion}
        Consider real-world scenarios where these concepts apply. Facilitating teamwork can enhance understanding and interpretation.
    \end{block}
\end{frame}

\begin{frame}[fragile]
    \frametitle{Correlation and Regression}
    \begin{block}{Understanding Relationships Between Variables}
        This section covers two key statistical concepts:
        \begin{itemize}
            \item Correlation: Measures the strength and direction of a linear relationship between two variables.
            \item Regression: Models and analyzes the relationships between dependent and independent variables.
        \end{itemize}
    \end{block}
\end{frame}

\begin{frame}[fragile]
    \frametitle{What is Correlation?}
    \begin{block}{Definition}
        Correlation measures the strength and direction of a linear relationship between two variables.
    \end{block}
    \begin{itemize}
        \item Coefficient Range: The correlation coefficient \( r \) ranges from -1 to +1.
        \begin{itemize}
            \item \( r = 1 \): Perfect positive correlation
            \item \( r = -1 \): Perfect negative correlation
            \item \( r = 0 \): No correlation
        \end{itemize}
        \item Example: For study hours and exam scores, if \( r = 0.85 \), there is a strong positive correlation.
    \end{itemize}
\end{frame}

\begin{frame}[fragile]
    \frametitle{Understanding Regression}
    \begin{block}{Definition}
        Regression is a statistical method to model and analyze relationships between variables.
    \end{block}
    \begin{itemize}
        \item Simple Linear Regression: 
        \begin{equation}
            Y = \beta_0 + \beta_1 X + \epsilon
        \end{equation}
        Where:
        \begin{itemize}
            \item \( Y \) = dependent variable
            \item \( X \) = independent variable
            \item \( \beta_0 \) = y-intercept
            \item \( \beta_1 \) = slope of the line
            \item \( \epsilon \) = error term
        \end{itemize}
        \item Example: Predicting scores:
        \begin{equation}
            \text{Score} = 50 + 10 \times \text{Study Hours} + \epsilon
        \end{equation}
        Here, every additional hour increases the score by 10 points.
    \end{itemize}
\end{frame}

\begin{frame}[fragile]
    \frametitle{Key Points and Applications}
    \begin{itemize}
        \item Correlation does not imply causation.
        \item Interpreting Regression Coefficients: Understand how changes in an independent variable affect the dependent variable.
        \item Goodness of Fit: \( R^2 \) value indicates how well the model explains variability.
    \end{itemize}
    \begin{block}{Applications}
        \begin{itemize}
            \item In business: Relation between marketing spend and sales.
            \item In healthcare: Impact of lifestyle choices on health outcomes.
        \end{itemize}
    \end{block}
\end{frame}

\begin{frame}[fragile]{Applying Mathematical Foundations in ML - Introduction}
    \begin{itemize}
        \item Mathematics is essential in machine learning (ML).
        \item Provides tools for model development and evaluation.
        \item Understanding foundational concepts enables informed decisions.
    \end{itemize}
\end{frame}

\begin{frame}[fragile]{Applying Mathematical Foundations in ML - Key Concepts}
    \begin{block}{Key Mathematical Concepts}
        \begin{itemize}
            \item \textbf{Linear Algebra}:
              \begin{itemize}
                  \item Vector and Matrix Operations: Data representation.
                  \item Example: A data point represented as \( \mathbf{x} = [age, salary, spending\_score] \).
              \end{itemize}
            \item \textbf{Calculus}:
              \begin{itemize}
                  \item Derivatives for minimizing loss functions.
                  \item Example: Gradient Descent updates \( \theta \) using:
                  \begin{equation}
                      \theta := \theta - \alpha \nabla J(\theta)
                  \end{equation}
              \end{itemize}
            \item \textbf{Probability and Statistics}:
              \begin{itemize}
                  \item Probabilistic models for decision-making.
                  \item Statistical measures: Variance defined as:
                  \begin{equation}
                      \sigma^2 = \frac{1}{n} \sum (x_i - \mu)^2
                  \end{equation}
              \end{itemize}
        \end{itemize}
    \end{block}
\end{frame}

\begin{frame}[fragile]{Applying Mathematical Foundations in ML - Application and Conclusion}
    \begin{block}{Applying the Concepts}
        \begin{itemize}
            \item Develop models using Linear Algebra and Calculus.
            \item Evaluate models with Statistical Measures and Probability.
            \item Example Application Flow:
              \begin{enumerate}
                  \item Data Collection
                  \item Preprocessing
                  \item Model Selection
                  \item Training
                  \item Evaluation
              \end{enumerate}
        \end{itemize}
    \end{block}

    \begin{block}{Conclusion}
        \begin{itemize}
            \item Mastery of mathematical foundations empowers ML practitioners.
            \item Enhances collaborative discussions and problem-solving skills.
        \end{itemize}
    \end{block}
    
    \begin{block}{Code Snippet}
    \begin{lstlisting}[language=Python]
# Simple gradient descent implementation
def gradient_descent(X, y, theta, alpha, iterations):
    m = len(y)  # number of training examples
    for _ in range(iterations):
        predictions = X.dot(theta)
        errors = predictions - y
        theta -= (alpha/m) * (X.T.dot(errors))  # Update rule
    return theta
    \end{lstlisting}
    \end{block}
\end{frame}

\begin{frame}[fragile]
    \frametitle{Real-world Applications - Overview}
    Mathematics serves as the foundational framework for various machine learning algorithms.
    By applying mathematical concepts, we can effectively solve practical problems across different domains.
    
    \begin{block}{Key Points}
        \begin{itemize}
            \item Interdisciplinary nature of machine learning.
            \item Effectiveness of mathematical foundations in real-world applications.
            \item Importance of practical examples in understanding mathematical abstractions.
        \end{itemize}
    \end{block}
\end{frame}

\begin{frame}[fragile]
    \frametitle{Mathematical Concepts and Their Applications}
    \begin{enumerate}
        \item \textbf{Linear Algebra}
            \begin{itemize}
                \item \textbf{Concept}: Linear transformations, vectors, matrices.
                \item \textbf{Application}: Image processing, computer vision.
                \item \textbf{Example}: PCA reduces dimensionality using eigenvalues/vectors.
            \end{itemize}
        
        \item \textbf{Calculus}
            \begin{itemize}
                \item \textbf{Concept}: Derivatives and integrals.
                \item \textbf{Application}: Minimizing loss functions via gradient descent.
                \item \textbf{Example}: Backpropagation adjusts weights in neural networks.
            \end{itemize}
        
        \item \textbf{Probability and Statistics}
            \begin{itemize}
                \item \textbf{Concept}: Probability distributions, expectations, variance.
                \item \textbf{Application}: Making predictions and uncertainty estimation.
                \item \textbf{Example}: Naive Bayes classifiers use Bayes' theorem.
            \end{itemize}
        
        \item \textbf{Optimization}
            \begin{itemize}
                \item \textbf{Concept}: Finding extrema of a function.
                \item \textbf{Application}: Enhancing model efficiency and accuracy.
                \item \textbf{Example}: SVMs optimize the separating hyperplane.
            \end{itemize}
    \end{enumerate}
\end{frame}

\begin{frame}[fragile]
    \frametitle{Example: Predicting Housing Prices}
    \textbf{Problem}: Predict the price of houses based on features like size, number of bedrooms, and location.

    \begin{itemize}
        \item \textbf{Feature Representation}: Use vectors to represent features of each house.
        
        \item \textbf{Model}:
        \begin{equation}
            \hat{y} = \beta_0 + \beta_1 x_1 + \beta_2 x_2 + \ldots + \beta_n x_n
        \end{equation}
        Where \( \hat{y} \) is the predicted price, \( \beta \) are coefficients, and \( x \) are features.

        \item \textbf{Optimization}: 
        \begin{itemize}
            \item Minimize mean squared error using gradient descent.
        \end{itemize}
    \end{itemize}
\end{frame}

\begin{frame}[fragile]
    \frametitle{Conclusion and Key Takeaways}
    \begin{itemize}
        \item Mathematics is a powerful tool in machine learning.
        \item Leveraging mathematical concepts improves the performance of models across applications.
        \item Understanding these concepts empowers practitioners to tackle complex problems.
    \end{itemize}
    
    \textbf{Discussion Prompt:} How can we incorporate collaborative activities or thought experiments to enhance understanding in class?
\end{frame}

\begin{frame}[fragile]
    \frametitle{Ethical Considerations - Introduction}
    \begin{block}{Introduction to Ethics in Data Handling}
        Ethics in data handling refers to the moral principles guiding how we collect, manage, and use data. 
        Being aware of ethical implications is crucial for data scientists, statisticians, and machine learning practitioners.
    \end{block}
\end{frame}

\begin{frame}[fragile]
    \frametitle{Ethical Considerations - Importance}
    \begin{enumerate}
        \item \textbf{Trust and Transparency}
            \begin{itemize}
                \item Upholding ethical standards fosters trust among users and stakeholders.
                \item Example: Companies like Facebook faced backlash for unclear data usage.
            \end{itemize}
        \item \textbf{Fairness and Bias Mitigation}
            \begin{itemize}
                \item Algorithms can perpetuate biases in training data.
                \item Example: Hiring algorithms favoring male candidates due to biased data.
            \end{itemize}
        \item \textbf{Privacy and Security}
            \begin{itemize}
                \item Personal data must be collected with consent and comply with privacy laws (e.g., GDPR).
                \item Example: Health apps utilizing user data require consent and anonymization.
            \end{itemize}
    \end{enumerate}
\end{frame}

\begin{frame}[fragile]
    \frametitle{Ethical Considerations - Key Principles}
    \begin{block}{Key Ethical Principles}
        \begin{itemize}
            \item \textbf{Informed Consent}: Obtain user consent before data collection.
            \item \textbf{Data Minimization}: Gather only necessary data for specific purposes.
            \item \textbf{Accountability}: Create systems of accountability for data use and breaches.
            \item \textbf{Explainability}: Ensure algorithms are understandable to stakeholders.
        \end{itemize}
    \end{block}
    \begin{block}{Key Takeaway}
        Practicing ethics is essential for responsible data handling and algorithm development.
    \end{block}
\end{frame}

\begin{frame}[fragile]
    \frametitle{Collaboration in Learning - Importance of Teamwork}
    
    \begin{block}{Collaboration Enhances Learning Outcomes}
        \begin{itemize}
            \item Mathematical concepts can be complex and benefit from diverse perspectives.
            \item Working in teams allows students to share knowledge, clarify doubts, and reinforce their understanding by explaining concepts to others.
        \end{itemize}
    \end{block}
    
    \begin{block}{Key Reasons for Collaborative Learning}
        \begin{enumerate}
            \item Diverse Perspectives: Different methodologies enrich problem-solving.
            \item Skill Development: Fosters communication, conflict resolution, and critical thinking.
            \item Motivation and Engagement: Teamwork increases motivation and accountability.
        \end{enumerate}
    \end{block}
\end{frame}

\begin{frame}[fragile]
    \frametitle{Collaboration in Mathematics - Applications}
    
    \begin{block}{Real-World Applications of Collaborative Skills}
        \begin{itemize}
            \item \textbf{Data Science Teams:} Combines statisticians, data analysts, and experts to tackle complex data sets and produce actionable strategies.
            \item \textbf{Research Groups:} Collaborative research leads to innovative mathematical theories and solutions to unsolved problems.
        \end{itemize}
    \end{block}
    
    \begin{block}{Example of Effective Collaboration}
        \begin{itemize}
            \item Group Project: Analyzing trends in a dataset.
            \begin{enumerate}
                \item \textbf{Data Exploration:} Team members explore various statistical methods together.
                \item \textbf{Discussion of Findings:} Each student presents their analysis.
                \item \textbf{Collective Conclusion:} The team synthesizes findings into a cohesive report.
            \end{enumerate}
        \end{itemize}
    \end{block}
\end{frame}

\begin{frame}[fragile]
    \frametitle{Effective Collaboration in Math - Tips}
    
    \begin{block}{Tips for Effective Collaboration}
        \begin{itemize}
            \item \textbf{Establish Clear Roles:} Assign roles based on individual strengths (e.g., researcher, presenter, coder).
            \item \textbf{Encourage Open Communication:} Foster an environment where all members can share ideas and ask questions.
            \item \textbf{Use Collaborative Tools:} Utilize platforms (Google Docs, Trello) to share notes and track progress.
        \end{itemize}
    \end{block}

    \begin{block}{Conclusion}
        Collaboration in mathematics is essential for mastering concepts and developing skills for future career paths, leading to deepened understanding and preparedness for teamwork in real-world situations.
    \end{block}
    
    \begin{block}{Key Takeaway}
        Teamwork is vital for success in mathematics and future careers; it nurtures a collaborative mindset invaluable in academia and various professions.
    \end{block}
\end{frame}

\begin{frame}[fragile]
    \frametitle{Review and Reflection - Key Points Learned}
    
    \begin{enumerate}
        \item \textbf{Fundamentals of Mathematics in Machine Learning}:
        \begin{itemize}
            \item \textbf{Linear Algebra}: Essential for understanding data structures. Operations with matrices and vectors are foundational.
            \item \textbf{Calculus}: Vital for optimization problems, helping to find minimum or maximum values during model training.
            \item \textbf{Probability and Statistics}: Allows making predictions and validating models, improving decision-making.
        \end{itemize}
    \end{enumerate}
\end{frame}

\begin{frame}[fragile]
    \frametitle{Review and Reflection - Examples}
    
    \begin{block}{Key Examples}
        \begin{itemize}
            \item \textbf{Linear Algebra Example}: In linear regression, the equation $y = mx + b$ can be represented in matrix form for multiple outputs.
            \item \textbf{Calculus Example}: Gradient descent utilizes derivatives to update weights minimizing the loss function.
            \item \textbf{Probability Example}: The Naive Bayes classifier employs Bayes' theorem to compute class probabilities.
        \end{itemize}
    \end{block}
\end{frame}

\begin{frame}[fragile]
    \frametitle{Review and Reflection - Application and Takeaways}
    
    \begin{enumerate}
        \item \textbf{Application of Mathematics in Machine Learning}:
        \begin{itemize}
            \item Used in \textbf{Model Development} and \textbf{Data Preprocessing}, enhancing model performance.
            \item Important for \textbf{Performance Evaluation} with metrics like accuracy and precision.
        \end{itemize}
        
        \item \textbf{Key Takeaways}:
        \begin{itemize}
            \item \textbf{Collaboration is Key}: Sharing insights among peers fosters deeper understanding.
            \item \textbf{Critical Thinking}: Encouraged to analyze and question assumptions when applying mathematical concepts.
        \end{itemize}
    \end{enumerate}
\end{frame}

\begin{frame}[fragile]
    \frametitle{Next Steps - Part 1}
    \begin{block}{Advancing Your Learning in Machine Learning}
        After a solid grounding in mathematical foundations, the next step is to explore advanced topics in machine learning (ML). This path will deepen your understanding and equip you with practical skills for real-world applications.
    \end{block}
\end{frame}

\begin{frame}[fragile]
    \frametitle{Next Steps - Part 2}
    \begin{enumerate}
        \item \textbf{Diving Deeper into Advanced Topics}
            \begin{itemize}
                \item \textbf{Algorithms}
                \begin{itemize}
                    \item Support Vector Machines (SVM): Optimal hyperplane classification.
                    \item Neural Networks: Multilayer perceptrons and backpropagation.
                \end{itemize}
                \item \textbf{Statistics}
                \begin{itemize}
                    \item Bayesian Inference: Prior, likelihood, and posterior distributions.
                    \item Hypothesis Testing: Type I and Type II errors and p-values.
                \end{itemize}
            \end{itemize}
        \item \textbf{Exploring Key Frameworks}
            \begin{itemize}
                \item TensorFlow: Build and deploy deep learning models.
                \item Scikit-Learn: Easy implementation of standard ML algorithms.
            \end{itemize}
    \end{enumerate}
\end{frame}

\begin{frame}[fragile]
    \frametitle{Next Steps - Part 3}
    \begin{enumerate}
        \setcounter{enumi}{2}
        \item \textbf{Practical Applications}
            \begin{itemize}
                \item Projects: Engage in real-life projects including:
                \begin{itemize}
                    \item Data Preprocessing: Normalization and handling missing values.
                    \item Model Evaluation: Metrics like accuracy, precision, recall, F1-score, and ROC-AUC.
                \end{itemize}
            \end{itemize}
        \item \textbf{Engagement and Collaboration}
            \begin{itemize}
                \item Teamwork: Join study groups or collaborate on ML projects.
                \item Discussion Sessions: Organize forums to discuss challenges and insights.
            \end{itemize}
        \item \textbf{Key Points to Remember}
            \begin{itemize}
                \item Master foundational math for advanced concepts.
                \item Engage in collaborative learning and projects.
                \item Seek external resources for continual learning.
            \end{itemize}
    \end{enumerate}
\end{frame}

\begin{frame}[fragile]
    \frametitle{Next Steps - Conclusion}
    \begin{block}{Conclusion}
        By following these next steps, you will solidify your understanding of machine learning and be well-prepared for a career in this exciting domain. Embrace the journey of learning; your dedication is the key to success!
    \end{block}
    
    \begin{block}{Useful Formulas & Code Snippets}
        \begin{equation}
            L(y_i, \hat{y}_i) = \max(0, 1 - y_i \cdot f(x_i))
        \end{equation}
        \begin{equation}
            w = w - \eta \frac{\partial L}{\partial w}
        \end{equation}
    \end{block}
\end{frame}


\end{document}