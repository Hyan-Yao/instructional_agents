\documentclass{beamer}

% Theme choice
\usetheme{Madrid} % You can change to e.g., Warsaw, Berlin, CambridgeUS, etc.

% Encoding and font
\usepackage[utf8]{inputenc}
\usepackage[T1]{fontenc}

% Graphics and tables
\usepackage{graphicx}
\usepackage{booktabs}

% Code listings
\usepackage{listings}
\lstset{
basicstyle=\ttfamily\small,
keywordstyle=\color{blue},
commentstyle=\color{gray},
stringstyle=\color{red},
breaklines=true,
frame=single
}

% Math packages
\usepackage{amsmath}
\usepackage{amssymb}

% Colors
\usepackage{xcolor}

% TikZ and PGFPlots
\usepackage{tikz}
\usepackage{pgfplots}
\pgfplotsset{compat=1.18}
\usetikzlibrary{positioning}

% Hyperlinks
\usepackage{hyperref}

% Title information
\title{Week 6: Introduction to SQL for Data Analysis}
\author{Your Name}
\institute{Your Institution}
\date{\today}

\begin{document}

\frame{\titlepage}

\begin{frame}[fragile]
    \frametitle{Introduction to SQL for Data Analysis}
    \begin{block}{Overview of SQL's Role in Data Analysis}
        SQL (Structured Query Language) is a standard programming language for managing and manipulating relational databases.
    \end{block}
\end{frame}

\begin{frame}[fragile]
    \frametitle{Importance of SQL in Data Analysis - Part 1}
    \begin{enumerate}
        \item \textbf{Data Retrieval}:
        \begin{itemize}
            \item SQL allows efficient querying of large datasets.
            \item Example:
            \begin{lstlisting}[language=SQL]
SELECT customer_name, total_spent 
FROM sales_data 
WHERE purchase_date >= '2023-01-01';
            \end{lstlisting}
        \end{itemize}

        \item \textbf{Data Manipulation}:
        \begin{itemize}
            \item Insert, update, and delete records using SQL commands.
            \item Example:
            \begin{lstlisting}[language=SQL]
INSERT INTO sales_data (customer_name, total_spent, purchase_date) 
VALUES ('Alice', 150.00, '2023-03-15');
            \end{lstlisting}
        \end{itemize}
        
        \item \textbf{Data Aggregation}:
        \begin{itemize}
            \item Use aggregate functions like COUNT, SUM, AVG, and MAX.
            \item Example:
            \begin{lstlisting}[language=SQL]
SELECT SUM(total_spent) 
FROM sales_data;
            \end{lstlisting}
        \end{itemize}
    \end{enumerate}
\end{frame}

\begin{frame}[fragile]
    \frametitle{Importance of SQL in Data Analysis - Part 2}
    \begin{enumerate}
        \setcounter{enumi}{3}
        \item \textbf{Data Filtering}:
        \begin{itemize}
            \item Filter data with the WHERE clause.
            \item Example:
            \begin{lstlisting}[language=SQL]
SELECT customer_name 
FROM sales_data 
WHERE total_spent > 100.00;
            \end{lstlisting}
        \end{itemize}

        \item \textbf{Joins and Relationships}:
        \begin{itemize}
            \item Combine data from multiple tables with JOIN operations.
            \item Example:
            \begin{lstlisting}[language=SQL]
SELECT customers.customer_name, sales.total_spent 
FROM customers 
JOIN sales ON customers.customer_id = sales.customer_id;
            \end{lstlisting}
        \end{itemize}
    \end{enumerate}
\end{frame}

\begin{frame}[fragile]
    \frametitle{Conclusion and Next Steps}
    \begin{block}{Key Points}
        \begin{itemize}
            \item SQL is essential for data analysis due to its efficiency in managing vast datasets.
            \item Familiarity with SQL syntax enhances the ability to gain insights from data.
            \item Understanding SQL commands is foundational for data-driven decision-making.
        \end{itemize}
    \end{block}
    
    \begin{block}{Conclusion}
        In summary, SQL serves as a powerful tool enabling diverse data operations, crucial for effective data analysis.
    \end{block}
    
    \begin{block}{Next Steps}
        In the upcoming slide, we will delve deeper into SQL, covering key definitions and primary functions for data management.
    \end{block}
\end{frame}

\begin{frame}[fragile]
    \frametitle{What is SQL?}
    SQL, or Structured Query Language, is a powerful programming language specifically designed for managing and manipulating relational databases. It allows users to:
    \begin{itemize}
        \item Create, 
        \item Read, 
        \item Update, and 
        \item Delete (CRUD) data stored in a structured format.
    \end{itemize}
\end{frame}

\begin{frame}[fragile]
    \frametitle{Key Functions of SQL in Data Management}
    \begin{enumerate}
        \item \textbf{Data Querying:}
        \begin{itemize}
            \item Retrieve specific data using the \texttt{SELECT} statement.
            \item \textbf{Example:}
            \begin{lstlisting}[language=SQL]
SELECT name, age FROM users;
            \end{lstlisting}
        \end{itemize}

        \item \textbf{Data Insertion:}
        \begin{itemize}
            \item Add new records using the \texttt{INSERT} statement.
            \item \textbf{Example:}
            \begin{lstlisting}[language=SQL]
INSERT INTO users (name, age) VALUES ('Alice', 30);
            \end{lstlisting}
        \end{itemize}
    \end{enumerate}
\end{frame}

\begin{frame}[fragile]
    \frametitle{Key Functions Continued}
    \begin{enumerate}
        \setcounter{enumi}{2} % Continue the enumeration from the previous frame
        \item \textbf{Data Updating:}
        \begin{itemize}
            \item Modify existing records using the \texttt{UPDATE} statement.
            \item \textbf{Example:}
            \begin{lstlisting}[language=SQL]
UPDATE users SET age = 31 WHERE name = 'Alice';
            \end{lstlisting}
        \end{itemize}

        \item \textbf{Data Deletion:}
        \begin{itemize}
            \item Remove records using the \texttt{DELETE} statement.
            \item \textbf{Example:}
            \begin{lstlisting}[language=SQL]
DELETE FROM users WHERE name = 'Alice';
            \end{lstlisting}
        \end{itemize}
    \end{enumerate}
\end{frame}

\begin{frame}[fragile]
    \frametitle{Key Points and Summary}
    \begin{itemize}
        \item \textbf{Standardized Language:} 
        SQL is used in various database systems (e.g., MySQL, PostgreSQL, SQLite).
        \item \textbf{Data Integrity:} 
        Ensures data integrity through constraints and relationships in database schemas.
        \item \textbf{Powerful Analysis Tool:} 
        Enables complex queries for deep data analysis, essential for decision-making.
    \end{itemize}
    
    \textbf{Summary:} SQL is a backbone for data management, offering tools for efficient interaction with relational databases. In the next slide, we will explore key concepts such as databases, tables, and schemas.
\end{frame}

\begin{frame}[fragile]
    \frametitle{Key SQL Concepts - Databases}
    \begin{block}{1. Databases}
        \begin{itemize}
            \item \textbf{Definition}: A database is an organized collection of data, typically stored and accessed electronically.
            \item \textbf{Key Points}:
            \begin{itemize}
                \item Databases can be classified into several types, such as relational, NoSQL, and more.
                \item Common relational database management systems (RDBMS): MySQL, PostgreSQL, Oracle, Microsoft SQL Server.
            \end{itemize}
        \end{itemize}
    \end{block}
\end{frame}

\begin{frame}[fragile]
    \frametitle{Key SQL Concepts - Tables}
    \begin{block}{2. Tables}
        \begin{itemize}
            \item \textbf{Definition}: Tables are the fundamental building blocks of a database, consisting of rows and columns where data is stored.
            \item \textbf{Components}:
            \begin{itemize}
                \item \textbf{Rows}: Each row represents a unique entry, also known as records.
                \item \textbf{Columns}: Each column represents a specific type of data for all records, also known as fields or attributes.
            \end{itemize}
            \item \textbf{Example}:
            \begin{center}
            \begin{tabular}{|c|c|c|c|}
                \hline
                EmployeeID & FirstName & LastName & Department \\
                \hline
                1          & John      & Doe      & HR         \\
                2          & Jane      & Smith    & IT         \\
                3          & Mike      & Johnson  & Finance    \\
                \hline
            \end{tabular}
            \end{center}
        \end{itemize}
    \end{block}
\end{frame}

\begin{frame}[fragile]
    \frametitle{Key SQL Concepts - Schemas}
    \begin{block}{3. Schemas}
        \begin{itemize}
            \item \textbf{Definition}: A schema is a blueprint of the database, defining how data is organized and how relationships among data are handled.
            \item \textbf{Components}:
            \begin{itemize}
                \item Specifies tables, fields, data types, and relationships (e.g., one-to-many, many-to-many).
            \end{itemize}
            \item \textbf{Importance}: A well-defined schema contributes to data integrity and improves query performance.
            \item \textbf{Example}: A \texttt{Company} schema may include tables for \texttt{Employees}, \texttt{Departments}, and \texttt{Salaries}, outlining relationships such as departments containing multiple employees.
        \end{itemize}
    \end{block}
\end{frame}

\begin{frame}[fragile]
    \frametitle{Key SQL Concepts - Summary}
    \begin{block}{Summary}
        \begin{itemize}
            \item Understanding databases, tables, and schemas is crucial for efficient data analysis using SQL.
            \item These concepts form the foundation for more complex SQL queries and data manipulations.
        \end{itemize}
    \end{block}
    \begin{block}{Next Steps}
        In the next slide, we will explore Basic SQL Syntax, including commands like \texttt{SELECT}, \texttt{FROM}, \texttt{WHERE}, and \texttt{JOIN}, allowing us to interact with these key components effectively.
    \end{block}
\end{frame}

\begin{frame}[fragile]
    \frametitle{Key SQL Concepts - Code Snippet}
    \begin{block}{Code Snippet Example}
        To create a simple table in SQL:
        \begin{lstlisting}[language=SQL]
CREATE TABLE Employees (
    EmployeeID INT PRIMARY KEY,
    FirstName VARCHAR(50),
    LastName VARCHAR(50),
    Department VARCHAR(50)
);
        \end{lstlisting}
        This command initializes the \texttt{Employees} table with designated data types, specifying each column's intended content.
    \end{block}
\end{frame}

\begin{frame}[fragile]
    \frametitle{Basic SQL Syntax - Overview}
    \begin{block}{Overview of Fundamental SQL Commands}
        SQL (Structured Query Language) is the standard language used for managing and manipulating relational databases. Understanding the basic syntax is essential for effective data analysis. We will cover four fundamental SQL commands:
        \begin{itemize}
            \item \textbf{SELECT}
            \item \textbf{FROM}
            \item \textbf{WHERE}
            \item \textbf{JOIN}
        \end{itemize}
    \end{block}
\end{frame}

\begin{frame}[fragile]
    \frametitle{Basic SQL Syntax - SELECT and FROM}
    \begin{block}{1. SELECT}
        The \textbf{SELECT} statement is used to specify which columns of data you want to retrieve from a database. 

        \textbf{Syntax:}
        \begin{lstlisting}
SELECT column1, column2, ...
        \end{lstlisting}

        \textbf{Example:}
        \begin{lstlisting}
SELECT first_name, last_name FROM employees;
        \end{lstlisting}
        *This command retrieves the first and last names of employees from the "employees" table.*
    \end{block}
    
    \begin{block}{2. FROM}
        The \textbf{FROM} clause specifies the table from which to retrieve the data.

        \textbf{Syntax:}
        \begin{lstlisting}
FROM table_name
        \end{lstlisting}

        \textbf{Combining WITH SELECT:}
        \begin{lstlisting}
SELECT * FROM products;
        \end{lstlisting}
        *This command retrieves all columns from the "products" table.*
    \end{block}
\end{frame}

\begin{frame}[fragile]
    \frametitle{Basic SQL Syntax - WHERE and JOIN}
    \begin{block}{3. WHERE}
        The \textbf{WHERE} clause allows you to filter records based on specified conditions.

        \textbf{Syntax:}
        \begin{lstlisting}
WHERE condition
        \end{lstlisting}

        \textbf{Example:}
        \begin{lstlisting}
SELECT * FROM orders WHERE order_date >= '2023-01-01';
        \end{lstlisting}
        *This command retrieves all records from the "orders" table where the order date is on or after January 1, 2023.*
    \end{block}
    
    \begin{block}{4. JOIN}
        The \textbf{JOIN} clause is used to combine rows from two or more tables based on related columns. There are several types of JOINs: INNER JOIN, LEFT JOIN, RIGHT JOIN, and FULL JOIN.

        \textbf{Syntax (INNER JOIN Example):}
        \begin{lstlisting}
SELECT a.column1, b.column2
FROM table1 a
JOIN table2 b ON a.common_field = b.common_field;
        \end{lstlisting}

        \textbf{Example:}
        \begin{lstlisting}
SELECT employees.first_name, departments.department_name 
FROM employees 
JOIN departments ON employees.department_id = departments.id;
        \end{lstlisting}
        *This command retrieves the first names of employees along with their corresponding department names by joining the "employees" and "departments" tables based on the department ID.*
    \end{block}
\end{frame}

\begin{frame}[fragile]
    \frametitle{Basic SQL Syntax - Key Points and Summary}
    \begin{block}{Key Points to Emphasize}
        \begin{itemize}
            \item \textbf{SELECT} is essential for choosing data.
            \item \textbf{FROM} indicates the source of the data.
            \item \textbf{WHERE} helps filter results to meet specific criteria.
            \item \textbf{JOIN} is crucial for combining data from multiple tables effectively.
        \end{itemize}
    \end{block}
    
    \begin{block}{Summary}
        These basic SQL commands form the foundation of querying databases. Mastering them allows for more complex queries and effective data analysis as you delve deeper into SQL.
        
        By understanding these fundamental concepts, you will be well-equipped to start analyzing and retrieving data from relational databases!
    \end{block}
\end{frame}

\begin{frame}
    \frametitle{Querying Large Datasets - Overview}
    \begin{block}{Overview}
        Efficient querying of large datasets is vital for timely results in SQL. 
        This session covers two essential techniques:
        \begin{itemize}
            \item Indexes
            \item Partitions
        \end{itemize}
    \end{block}
\end{frame}

\begin{frame}
    \frametitle{Querying Large Datasets - Indexes}
    \begin{block}{What are Indexes?}
        An index is a data structure designed to speed up data retrieval operations at the cost of additional storage space and slower writes.
    \end{block}
    
    \begin{block}{How Indexes Work}
        \begin{itemize}
            \item **Structure**: Created on one or more columns of a table.
            \item **Types**:
                \begin{itemize}
                    \item Single-Column Indexes
                    \item Composite Indexes
                \end{itemize}
        \end{itemize}
    \end{block}
    
    \begin{block}{Key Points}
        \begin{itemize}
            \item Reduces query time.
            \item Increases storage requirements and can slow down write operations.
        \end{itemize}
    \end{block}
\end{frame}

\begin{frame}[fragile]
    \frametitle{Querying Large Datasets - Index Example}
    \begin{block}{Example of Creating an Index}
        The following SQL command demonstrates how to create an index:
        \begin{lstlisting}
CREATE INDEX idx_lastname ON customers (last_name);
        \end{lstlisting}
    \end{block}
\end{frame}

\begin{frame}
    \frametitle{Querying Large Datasets - Partitions}
    \begin{block}{What are Partitions?}
        Partitioning divides a large table into smaller segments, maintaining the logical structure while enhancing manageability.
    \end{block}
    
    \begin{block}{Why Use Partitions?}
        \begin{itemize}
            \item Improves query performance by scanning only relevant data.
            \item Makes data management tasks simpler.
        \end{itemize}
    \end{block}
    
    \begin{block}{Key Points}
        \begin{itemize}
            \item Targeted scans for increased efficiency.
            \item Eases data maintenance tasks.
        \end{itemize}
    \end{block}
\end{frame}

\begin{frame}[fragile]
    \frametitle{Querying Large Datasets - Partition Example}
    \begin{block}{Example of Partitioning}
        The following SQL command creates a partitioned table:
        \begin{lstlisting}
CREATE TABLE sales (
    sale_id INT,
    sale_date DATE,
    amount DECIMAL(10, 2)
)
PARTITION BY RANGE (YEAR(sale_date)) (
    PARTITION p2021 VALUES LESS THAN (2022),
    PARTITION p2022 VALUES LESS THAN (2023)
);
        \end{lstlisting}
    \end{block}
\end{frame}

\begin{frame}
    \frametitle{Querying Large Datasets - Conclusion}
    \begin{block}{Conclusion}
        Using indexes and partitions is essential for efficient querying of large datasets in SQL.
        By employing these techniques, we can significantly boost the performance of our database queries.
    \end{block}
    
    \begin{block}{Next Steps}
        Next, we will investigate how to use SQL filtering techniques like the `WHERE` clause to extract specific subsets of data.
    \end{block}
\end{frame}

\begin{frame}[fragile]
    \frametitle{Using SQL for Data Filtering - Introduction}
    \begin{itemize}
        \item Data filtering in SQL retrieves specific rows by applying conditions.
        \item Essential for focusing on relevant information in large datasets.
        \item The WHERE clause is the main tool for data filtering.
    \end{itemize}
\end{frame}

\begin{frame}[fragile]
    \frametitle{Using SQL for Data Filtering - The WHERE Clause}
    \begin{block}{WHERE Clause}
        The WHERE clause specifies conditions for records to be included in query results.
    \end{block}
    
    \begin{block}{Syntax}
        \begin{lstlisting}
SELECT column1, column2, ...
FROM table_name
WHERE condition;
        \end{lstlisting}
    \end{block}
\end{frame}

\begin{frame}[fragile]
    \frametitle{Using SQL for Data Filtering - Examples of WHERE}
    \begin{enumerate}
        \item \textbf{Basic Filtering:} Find employees in a certain department.
        \begin{lstlisting}
SELECT * 
FROM employees 
WHERE department = 'Sales';
        \end{lstlisting}
        
        \item \textbf{Comparison Operators:} Filter products over a certain price.
        \begin{lstlisting}
SELECT * 
FROM products 
WHERE price > 50;
        \end{lstlisting}
        
        \item \textbf{Multiple Conditions:} Combination of conditions using AND/OR.
        \begin{lstlisting}
SELECT * 
FROM orders 
WHERE order_date >= '2023-01-01' 
AND status = 'Shipped';
        \end{lstlisting}
    \end{enumerate}
\end{frame}

\begin{frame}[fragile]
    \frametitle{Using SQL for Data Filtering - Advanced Filtering}
    \begin{itemize}
        \item \textbf{Filtering with LIKE:} For pattern matching.
        \begin{lstlisting}
SELECT * 
FROM customers 
WHERE name LIKE 'A%';
        \end{lstlisting}
        \item \textbf{Filtering with IN:} Specify multiple values.
        \begin{lstlisting}
SELECT * 
FROM products 
WHERE category IN ('Electronics', 'Stationery');
        \end{lstlisting}
        \item \textbf{Handling NULL Values:}
        \begin{lstlisting}
SELECT * 
FROM employees 
WHERE termination_date IS NULL;
        \end{lstlisting}
    \end{itemize}
\end{frame}

\begin{frame}[fragile]
    \frametitle{Using SQL for Data Filtering - Key Points}
    \begin{itemize}
        \item The WHERE clause filters records before grouping or aggregation.
        \item Careful condition use enhances data retrieval speed in large datasets.
        \item Combine conditions logically for refined results.
        \item Always test queries to ensure expected results.
    \end{itemize}
\end{frame}

\begin{frame}[fragile]
    \frametitle{Using SQL for Data Filtering - Conclusion}
    \begin{itemize}
        \item Mastering WHERE clauses and filtering techniques is essential for data analysis in SQL.
        \item Enables focus on relevant dataset portions, enhancing insights and decision-making.
    \end{itemize}
\end{frame}

\begin{frame}[fragile]
    \frametitle{Aggregate Functions - Overview}
    \begin{block}{Definition}
        Aggregate functions are powerful tools in SQL that allow you to perform calculations on a set of values and return a single value. They are essential for summarizing data in databases, especially when analyzing large datasets.
    \end{block}
\end{frame}

\begin{frame}[fragile]
    \frametitle{Common Aggregate Functions}
    \begin{enumerate}
        \item \textbf{COUNT()}
        \begin{itemize}
            \item \textbf{Purpose}: Returns the number of rows that match a specified condition.
            \item \textbf{Syntax}:
            \begin{lstlisting}
SELECT COUNT(column_name) FROM table_name WHERE condition;
\end{lstlisting}
            \item \textbf{Example}:
            \begin{lstlisting}
SELECT COUNT(*) FROM employees WHERE department = 'Sales';
\end{lstlisting}
        \end{itemize}

        \item \textbf{SUM()}
        \begin{itemize}
            \item \textbf{Purpose}: Calculates the total sum of a numeric column.
            \item \textbf{Syntax}:
            \begin{lstlisting}
SELECT SUM(column_name) FROM table_name WHERE condition;
\end{lstlisting}
            \item \textbf{Example}:
            \begin{lstlisting}
SELECT SUM(salary) FROM employees WHERE department = 'Marketing';
\end{lstlisting}
        \end{itemize}
    \end{enumerate}
\end{frame}

\begin{frame}[fragile]
    \frametitle{Common Aggregate Functions (Cont'd)}
    \begin{enumerate}[resume]
        \item \textbf{AVG()}
        \begin{itemize}
            \item \textbf{Purpose}: Computes the average value of a numeric column.
            \item \textbf{Syntax}:
            \begin{lstlisting}
SELECT AVG(column_name) FROM table_name WHERE condition;
\end{lstlisting}
            \item \textbf{Example}:
            \begin{lstlisting}
SELECT AVG(salary) FROM employees;
\end{lstlisting}
        \end{itemize}

        \item \textbf{MIN()}
        \begin{itemize}
            \item \textbf{Purpose}: Finds the smallest value in a specified column.
            \item \textbf{Syntax}:
            \begin{lstlisting}
SELECT MIN(column_name) FROM table_name WHERE condition;
\end{lstlisting}
            \item \textbf{Example}:
            \begin{lstlisting}
SELECT MIN(salary) FROM employees;
\end{lstlisting}
        \end{itemize}

        \item \textbf{MAX()}
        \begin{itemize}
            \item \textbf{Purpose}: Returns the largest value in a specified column.
            \item \textbf{Syntax}:
            \begin{lstlisting}
SELECT MAX(column_name) FROM table_name WHERE condition;
\end{lstlisting}
            \item \textbf{Example}:
            \begin{lstlisting}
SELECT MAX(salary) FROM employees;
\end{lstlisting}
        \end{itemize}
    \end{enumerate}
\end{frame}

\begin{frame}[fragile]
    \frametitle{Key Points and SQL Code Snippet}
    \begin{block}{Key Points}
        \begin{itemize}
            \item Aggregate functions work on a set of rows and return a single value.
            \item They can be combined with the \texttt{GROUP BY} clause to summarize data.
            \item Extremely useful for reporting and data analysis, providing quick insights.
        \end{itemize}
    \end{block}

    \textbf{SQL Code Snippet with GROUP BY}:
    \begin{lstlisting}
SELECT department, COUNT(*) AS EmployeeCount, AVG(salary) AS AvgSalary
FROM employees
GROUP BY department;
\end{lstlisting}
\end{frame}

\begin{frame}
    \frametitle{Data Visualization with SQL}
    \begin{block}{Best Practices for Preparing Data for Visualization}
        Preparing data effectively is crucial for generating insights using visualization tools.
    \end{block}
\end{frame}

\begin{frame}[fragile]
    \frametitle{Best Practices - Understanding Your Data}
    \begin{enumerate}
        \item Understand your data structure:
        \begin{itemize}
            \item Identify types (numerical, categorical) and table relationships.
            \item \textbf{Example:} Sales data – recognize measures (sales amount) and dimensions (product name, region).
        \end{itemize}
    \end{enumerate}
\end{frame}

\begin{frame}[fragile]
    \frametitle{Best Practices - Summarizing Data}
    \begin{enumerate}
        \setcounter{enumi}{1}
        \item Employ aggregate functions:
        \begin{itemize}
            \item Use COUNT, SUM, AVG, MIN, MAX to summarize before visualization.
            \item \textbf{Example Query:}
            \begin{lstlisting}
SELECT 
    product_name, 
    SUM(sales_amount) AS total_sales
FROM 
    sales_data
GROUP BY 
    product_name;
            \end{lstlisting}
        \end{itemize}
    \end{enumerate}
\end{frame}

\begin{frame}[fragile]
    \frametitle{Best Practices - Filtering and Joining Data}
    \begin{enumerate}
        \setcounter{enumi}{2}
        \item Filter and limit data:
        \begin{itemize}
            \item Focus on necessary data to improve performance.
            \item \textbf{Example Query:}
            \begin{lstlisting}
SELECT * 
FROM sales_data 
WHERE year = 2023;
            \end{lstlisting}
        \end{itemize}
        
        \item Join data effectively:
        \begin{itemize}
            \item Use JOIN to enrich datasets.
            \item \textbf{Example Query:}
            \begin{lstlisting}
SELECT 
    customers.customer_name, 
    SUM(sales.sales_amount) AS total_sales
FROM 
    customers
JOIN 
    sales ON customers.id = sales.customer_id
GROUP BY 
    customers.customer_name;
            \end{lstlisting}
        \end{itemize}
    \end{enumerate}
\end{frame}

\begin{frame}[fragile]
    \frametitle{Best Practices - Derived Columns and Documentation}
    \begin{enumerate}
        \setcounter{enumi}{4}
        \item Create derived/calculated columns:
        \begin{itemize}
            \item Provide additional insights.
            \item \textbf{Example:} Profit margin calculation.
            \begin{lstlisting}
SELECT 
    product_name, 
    (SUM(sales_amount) - SUM(cost_amount)) / SUM(sales_amount) * 100 AS profit_margin
FROM 
    sales_data
GROUP BY 
    product_name;
            \end{lstlisting}
        \end{itemize}
        
        \item Maintain consistent naming conventions:
        \begin{itemize}
            \item Use clear, descriptive names for better understanding.
        \end{itemize}
        
        \item Document your queries:
        \begin{itemize}
            \item Include comments to clarify intent.
            \item \textbf{Example:}
            \begin{lstlisting}
-- Calculate total sales grouped by product
SELECT 
    product_name, 
    SUM(sales_amount) AS total_sales
FROM 
    sales_data
GROUP BY 
    product_name;
            \end{lstlisting}
        \end{itemize}
    \end{enumerate}
\end{frame}

\begin{frame}
    \frametitle{Key Points to Remember}
    \begin{itemize}
        \item Data preparation is crucial for effective visualization.
        \item Summarize and filter data to enhance clarity.
        \item Use JOINs to combine relevant datasets and enrich your analysis.
        \item Clear naming conventions and documentation improve collaboration and understanding.
    \end{itemize}
\end{frame}

\begin{frame}
    \frametitle{Summary}
    Preparing SQL data effectively influences the quality of visualizations in Tableau and Power BI. Following these best practices will enhance data-driven decision-making.
\end{frame}

\begin{frame}[fragile]
    \frametitle{SQL Best Practices - Overview}
    Understanding and applying best practices in SQL is crucial for efficient data handling, optimization of queries, and ensuring the reliability of the data analysis process.
    
    \begin{block}{Key Points}
        \begin{itemize}
            \item Use SELECT Wisely
            \item Use Proper Indexing
            \item Write Clear and Concise Queries
            \item Use Joins Judiciously
            \item Use Aggregations and GROUP BY Smartly
            \item Avoid Using Functions on Indexed Columns
            \item Limit the Number of Nested Queries
            \item Test and Optimize Regularly
        \end{itemize}
    \end{block}
\end{frame}

\begin{frame}[fragile]
    \frametitle{SQL Best Practices - Query Optimization}
    
    \begin{enumerate}
        \item \textbf{Use SELECT Wisely}
        \begin{itemize}
            \item Avoid SELECT *: Specify only necessary columns.
            \item \textit{Example:}
            \begin{lstlisting}
            SELECT first_name, last_name FROM employees;
            \end{lstlisting}
        \end{itemize}
        
        \item \textbf{Use Proper Indexing}
        \begin{itemize}
            \item Create indexes on frequently used columns.
            \item \textit{Example:}
            \begin{lstlisting}
            CREATE INDEX idx_employee_lastname ON employees(last_name);
            \end{lstlisting}
        \end{itemize}
    \end{enumerate}
\end{frame}

\begin{frame}[fragile]
    \frametitle{SQL Best Practices - Advanced Tips}
    
    \begin{enumerate}
        \setcounter{enumi}{2}
        \item \textbf{Write Clear and Concise Queries}
        \begin{itemize}
            \item Use proper formatting and indentation.
            \item \textit{Example:}
            \begin{lstlisting}
            SELECT 
                first_name, last_name 
            FROM 
                employees 
            WHERE 
                department_id = 3 
            ORDER BY 
                last_name;
            \end{lstlisting}
        \end{itemize}

        \item \textbf{Use Joins Judiciously}
        \begin{itemize}
            \item Limit results with appropriate WHERE clauses.
            \item \textit{Example:}
            \begin{lstlisting}
            SELECT 
                e.first_name, e.last_name, d.department_name 
            FROM 
                employees e
            JOIN 
                departments d ON e.department_id = d.id
            WHERE 
                d.location = 'New York';
            \end{lstlisting}
        \end{itemize}
    \end{enumerate}
\end{frame}

\begin{frame}[fragile]
    \frametitle{Conclusion}
    In this chapter, we've explored the fundamental concepts of SQL (Structured Query Language) for Data Analysis. By mastering SQL, you can efficiently query databases, manipulate data, and derive insightful analyses that are crucial for data-driven decision-making.
\end{frame}

\begin{frame}[fragile]
    \frametitle{Key Takeaways}
    \begin{enumerate}
        \item \textbf{Understanding SQL Basics:} Introduction to SQL syntax with key commands such as \texttt{SELECT}, \texttt{FROM}, \texttt{WHERE}, \texttt{JOIN}, and \texttt{GROUP BY}.
        
        \item \textbf{Data Manipulation:} Performing essential data manipulation tasks including filtering, sorting, and aggregating functions like \texttt{COUNT()}, \texttt{SUM()}, and \texttt{AVG()}.
        
        \item \textbf{Best Practices:} 
        \begin{itemize}
            \item Use of aliases for readability.
            \item Proper indexing for optimized query performance.
            \item Writing clear and concise queries to avoid redundancy.
        \end{itemize}
        
        \item \textbf{Use of Joins:} Importance of joins (INNER JOIN, LEFT JOIN, RIGHT JOIN) in combining datasets for comprehensive analyses.
        
        \item \textbf{Real-World Applications:} SQL applications in various domains like business intelligence, analytics, and data engineering.
    \end{enumerate}
\end{frame}

\begin{frame}[fragile]
    \frametitle{Further Resources}
    To enhance your SQL skills, consider the following resources:
    \begin{itemize}
        \item \textbf{Online Courses:}
        \begin{itemize}
            \item \texttt{Coursera: Databases and SQL for Data Science}
            \item \texttt{edX: Introduction to SQL}
        \end{itemize}
        
        \item \textbf{Books:}
        \begin{itemize}
            \item \texttt{"SQL for Data Analysis" by Cathy Tanimura}
            \item \texttt{"Learning SQL" by Alan Beaulieu}
        \end{itemize}
        
        \item \textbf{Interactive Platforms:}
        \begin{itemize}
            \item \texttt{LeetCode} for practice SQL queries.
            \item \texttt{Hackerrank} for SQL exercises.
        \end{itemize}
        
        \item \textbf{Community and Forums:}
        \begin{itemize}
            \item \texttt{Stack Overflow} for Q\&A on SQL.
            \item \texttt{Reddit: r/SQL} for discussions and resources.
        \end{itemize}
        
        \item \textbf{Tool-Specific Documentation:}
        \begin{itemize}
            \item \texttt{PostgreSQL Documentation}
            \item \texttt{MySQL Reference Manual}
        \end{itemize}
    \end{itemize}
\end{frame}

\begin{frame}[fragile]
    \frametitle{Summary}
    With these takeaways and resources, you are well-equipped to continue your journey in SQL for data analysis. Embrace practice, engage with the community, and keep exploring innovative ways to manipulate and analyze data. Happy querying!
\end{frame}


\end{document}