\documentclass[aspectratio=169]{beamer}

% Theme and Color Setup
\usetheme{Madrid}
\usecolortheme{whale}
\useinnertheme{rectangles}
\useoutertheme{miniframes}

% Additional Packages
\usepackage[utf8]{inputenc}
\usepackage[T1]{fontenc}
\usepackage{graphicx}
\usepackage{booktabs}
\usepackage{listings}
\usepackage{amsmath}
\usepackage{amssymb}
\usepackage{xcolor}
\usepackage{tikz}
\usepackage{pgfplots}
\pgfplotsset{compat=1.18}
\usetikzlibrary{positioning}
\usepackage{hyperref}

% Custom Colors
\definecolor{myblue}{RGB}{31, 73, 125}
\definecolor{mygray}{RGB}{100, 100, 100}
\definecolor{mygreen}{RGB}{0, 128, 0}
\definecolor{myorange}{RGB}{230, 126, 34}
\definecolor{mycodebackground}{RGB}{245, 245, 245}

% Set Theme Colors
\setbeamercolor{structure}{fg=myblue}
\setbeamercolor{frametitle}{fg=white, bg=myblue}
\setbeamercolor{title}{fg=myblue}
\setbeamercolor{section in toc}{fg=myblue}
\setbeamercolor{item projected}{fg=white, bg=myblue}
\setbeamercolor{block title}{bg=myblue!20, fg=myblue}
\setbeamercolor{block body}{bg=myblue!10}
\setbeamercolor{alerted text}{fg=myorange}

% Set Fonts
\setbeamerfont{title}{size=\Large, series=\bfseries}
\setbeamerfont{frametitle}{size=\large, series=\bfseries}
\setbeamerfont{caption}{size=\small}
\setbeamerfont{footnote}{size=\tiny}

% Document Start
\begin{document}

\frame{\titlepage}

\begin{frame}[fragile]
    \frametitle{Introduction to Final Project Work Days}
    \begin{block}{Overview}
        Final project workdays are dedicated periods within this course for intensive focus on final projects.
    \end{block}
\end{frame}

\begin{frame}[fragile]
    \frametitle{What Are Final Project Work Days?}
    Final project workdays allow students to:
    \begin{itemize}
        \item Consolidate learning
        \item Apply skills acquired
        \item Produce a comprehensive end product
    \end{itemize}
\end{frame}

\begin{frame}[fragile]
    \frametitle{Purpose of Final Project Work Days}
    \begin{enumerate}
        \item \textbf{Focused Time for Project Development}
        \begin{itemize}
            \item Allocate time for research, analysis, and implementation
        \end{itemize}
        \item \textbf{Facilitating Collaboration}
        \begin{itemize}
            \item Encourage peer interaction for refining project concepts
        \end{itemize}
        \item \textbf{Access to Resources and Guidance}
        \begin{itemize}
            \item Access instructors and TAs for assistance and clarification
        \end{itemize}
    \end{enumerate}
\end{frame}

\begin{frame}[fragile]
    \frametitle{Key Objectives}
    \begin{itemize}
        \item \textbf{Empower Independent Learning:}
        Encourage ownership of projects by applying course knowledge.
        \item \textbf{Promote Practical Application:}
        Bridge theory and practice, implementing data processing techniques.
        \item \textbf{Enhance Teamwork Skills:}
        Strengthen teamwork abilities crucial in professional settings.
    \end{itemize}
\end{frame}

\begin{frame}[fragile]
    \frametitle{Examples of Activities During Workdays}
    \begin{itemize}
        \item \textbf{Group Brainstorming:}
        Discuss ideas and challenges with peers.
        \item \textbf{Progress Presentations:}
        Share project progress and receive feedback.
        \item \textbf{Hands-on Workshops:}
        Participate in workshops on relevant skills and techniques.
    \end{itemize}
\end{frame}

\begin{frame}[fragile]
    \frametitle{Key Points to Remember}
    \begin{itemize}
        \item Use your time wisely: Plan ahead and set clear goals.
        \item Be open to feedback: Constructive criticism improves project quality.
        \item Engage actively: Maximize learning and collaboration benefits during discussions.
    \end{itemize}
\end{frame}

\begin{frame}[fragile]
    \frametitle{Conclusion}
    Final project workdays are essential for:
    \begin{itemize}
        \item Applying learning effectively
        \item Collaborating with instructors and peers
        \item Meeting project goals and showcasing knowledge and skills
    \end{itemize}
\end{frame}

\begin{frame}[fragile]{Learning Outcomes for Final Project}
    \begin{itemize}
        \item Understand the learning outcomes of your final project.
        \item Skills applicable in real-world scenarios.
    \end{itemize}
\end{frame}

\begin{frame}[fragile]{Learning Outcomes - Data Processing Proficiency}
    \begin{block}{1. Data Processing Proficiency}
        \begin{itemize}
            \item \textbf{Explanation:} Develop skills to gather, clean, analyze, and interpret data effectively.
            \item \textbf{Example:} Using Python's Pandas or R for data manipulation and visualization.
            \item \textbf{Key Point:} Mastering data processing is essential for data-driven decision-making in various industries.
        \end{itemize}
    \end{block}
\end{frame}

\begin{frame}[fragile]{Learning Outcomes - Governance, Ethics, and Teamwork}
    \begin{block}{2. Understanding of Governance}
        \begin{itemize}
            \item \textbf{Explanation:} Gain insights into project management frameworks and compliance.
            \item \textbf{Example:} Learning stakeholder engagement and risk management.
            \item \textbf{Key Point:} Good governance ensures accountability and alignment with organizational goals.
        \end{itemize}
    \end{block}
    
    \begin{block}{3. Ethical Considerations}
        \begin{itemize}
            \item \textbf{Explanation:} Explore ethical implications of data use, including privacy and equity.
            \item \textbf{Example:} Implementing anonymization techniques to protect identities.
            \item \textbf{Key Point:} Navigating ethical dilemmas fosters trust in research outcomes.
        \end{itemize}
    \end{block}
    
    \begin{block}{4. Teamwork}
        \begin{itemize}
            \item \textbf{Explanation:} Develop collaborative skills to work effectively in teams.
            \item \textbf{Example:} Participating in group discussions and integrating peer feedback.
            \item \textbf{Key Point:} Effective collaboration enhances project quality and promotes innovation.
        \end{itemize}
    \end{block}
\end{frame}

\begin{frame}[fragile]{Conclusion}
    \begin{itemize}
        \item Embrace the learning outcomes to tackle your project confidently.
        \item Skills acquired will be valuable beyond the course.
    \end{itemize}
\end{frame}

\begin{frame}[fragile]
    \frametitle{Project Structure and Requirements - Overview}
    \begin{block}{Overview of Project Components}
        The final project is an opportunity to synthesize and apply the knowledge and skills developed throughout the course. 
        It is structured to guide you through a comprehensive analytical process that enhances your understanding and practical application of concepts.
    \end{block}
\end{frame}

\begin{frame}[fragile]
    \frametitle{Project Structure and Requirements - Key Components}
    \begin{enumerate}
        \item \textbf{Project Proposal}
        \begin{itemize}
            \item \textbf{Description:} A formal document outlining your project objectives, research questions, methodology, and anticipated outcomes.
            \item \textbf{Requirements:}
            \begin{itemize}
                \item Title: A clear and concise title that reflects the project's scope.
                \item Objectives: Define what you aim to achieve.
                \item Methodology: Describe techniques for data collection and analysis.
                \item Timeline: Provide an estimated timeline for each stage.
            \end{itemize}
            \item \textbf{Example:} Outline data sources, analytical methods (e.g., regression analysis), and health outcomes to investigate for a project on community health data.
        \end{itemize}
        
        \item \textbf{Progress Report}
        \begin{itemize}
            \item \textbf{Description:} An update on your project's status, documenting progress against initial objectives and challenges encountered.
            \item \textbf{Requirements:}
            \begin{itemize}
                \item Overview: Brief summary of completed work.
                \item Challenges: Discuss obstacles or deviations from the plan.
                \item Next Steps: Clearly articulate what you will do next.
                \item Reflection: Consider successes and necessary adjustments.
            \end{itemize}
            \item \textbf{Example:} Document efforts to obtain data and alternative strategies if challenges arise in data collection.
        \end{itemize}
    \end{enumerate}
\end{frame}

\begin{frame}[fragile]
    \frametitle{Project Structure and Requirements - Continued}
    \begin{enumerate}[resume]
        \item \textbf{Final Presentation}
        \begin{itemize}
            \item \textbf{Description:} Share your findings with peers and instructors as the culmination of your project.
            \item \textbf{Requirements:}
            \begin{itemize}
                \item Content: Present key findings, methodology, and implications.
                \item Organization: Logical structure including introduction, main points, and conclusion.
                \item Visual Aids: Utilize slides or media to enhance understanding.
                \item Q\&A Session: Prepare to answer audience questions about your research.
            \end{itemize}
            \item \textbf{Example:} Use charts or graphs to present data findings visually, making results clear and accessible.
        \end{itemize}
    \end{enumerate}
    
    \begin{block}{Key Points to Emphasize}
        \begin{itemize}
            \item Each component builds upon the previous one; thorough preparation is key.
            \item Collaboration and feedback are crucial; seek input from peers and instructors.
            \item Adhere to deadlines for timely completion of all components.
        \end{itemize}
    \end{block}
\end{frame}

\begin{frame}[fragile]
    \frametitle{Timeline and Milestones - Overview}
    \begin{block}{Overview of the Project Timeline}
        The final project is structured around a series of key milestones that guide you from initial proposal stages to the project presentation. Understanding and managing these deadlines is critical for successful project completion.
    \end{block}
    \begin{block}{Key Milestones and Deliverables}
        \begin{enumerate}
            \item Project Proposal Submission
            \item Progress Report 1
            \item Midway Presentation
            \item Progress Report 2
            \item Final Draft Submission
            \item Final Presentation
            \item Final Project Submission
        \end{enumerate}
    \end{block}
\end{frame}

\begin{frame}[fragile]
    \frametitle{Timeline and Milestones - Milestones Details}
    \begin{itemize}
        \item \textbf{Project Proposal Submission}
            \begin{itemize}
                \item \textit{Due Date:} [Insert due date]
                \item Description: Outlines project’s objective, methodology, and outcomes.
            \end{itemize}

        \item \textbf{Progress Report 1}
            \begin{itemize}
                \item \textit{Due Date:} [Insert due date]
                \item Description: Brief report detailing progress and adjustments to plan.
            \end{itemize}

        \item \textbf{Midway Presentation}
            \begin{itemize}
                \item \textit{Due Date:} [Insert due date]
                \item Description: Share progress and gather feedback to refine approach.
            \end{itemize}

        \item \textbf{Progress Report 2}
            \begin{itemize}
                \item \textit{Due Date:} [Insert due date]
                \item Description: Comprehensive report providing updates and next steps.
            \end{itemize}

        \item \textbf{Final Draft Submission}
            \begin{itemize}
                \item \textit{Due Date:} [Insert due date]
                \item Description: Complete draft reflecting all efforts and feedback.
            \end{itemize}

        \item \textbf{Final Presentation}
            \begin{itemize}
                \item \textit{Due Date:} [Insert due date]
                \item Description: Formal presentation to showcase findings.
            \end{itemize}

        \item \textbf{Final Project Submission}
            \begin{itemize}
                \item \textit{Due Date:} [Insert due date]
                \item Description: Final version incorporating revisions and insights.
            \end{itemize}
    \end{itemize}
\end{frame}

\begin{frame}[fragile]
    \frametitle{Timeline and Milestones - Conclusion}
    \begin{block}{Key Points to Emphasize}
        \begin{itemize}
            \item \textbf{Stay Organized:} Break tasks into manageable steps and adhere to deadlines.
            \item \textbf{Proactive Communication:} Reach out to instructors and collaborate with peers for insights.
            \item \textbf{Adaptability:} Be prepared to adjust your project plan based on feedback.
        \end{itemize}
    \end{block}
    
    \begin{block}{Timeline Summary}
        Successful completion of the final project relies heavily on adhering to the timeline and meeting each milestone. A disciplined approach will lead to quality work that reflects your understanding and capabilities.
    \end{block}
\end{frame}

\begin{frame}[fragile]
    \frametitle{In-Class Support Mechanisms - Overview}
    \begin{block}{Overview of Support Available During Workdays}
        During our in-class workdays, students will have access to a variety of support systems designed to enhance collaboration, understanding, and progress on the final project. Here’s a breakdown of the support mechanisms available:
    \end{block}
\end{frame}

\begin{frame}[fragile]
    \frametitle{In-Class Support Mechanisms - Faculty Assistance}
    \begin{itemize}
        \item \textbf{One-on-One Guidance:}
            \begin{itemize}
                \item Faculty members will be present to provide personalized assistance. 
                \item Students can ask questions, seek clarification on concepts, or request feedback on their work.
                \item \textit{Example:} If a team is struggling with a particular aspect of their project proposal, they can schedule a brief consultation with the instructor to discuss their ideas and receive targeted advice.
            \end{itemize}
        
        \item \textbf{Clarification of Expectations:}
            \begin{itemize}
                \item Faculty can help clarify project criteria and expectations to ensure teams are on the right track.
                \item Discussing deliverables and alignment with the project rubric is key.
            \end{itemize}
    \end{itemize}
\end{frame}

\begin{frame}[fragile]
    \frametitle{In-Class Support Mechanisms - Collaboration and Resources}
    \begin{itemize}
        \item \textbf{Collaborative Opportunities:}
            \begin{itemize}
                \item Teams are encouraged to collaborate within their groups and across different groups.
                \item \textit{Example:} Teams might set up a "project-sharing" session to present their ideas, gaining valuable insights and constructive feedback.
                \item Interdisciplinary support may be provided by faculty from differing disciplines.
            \end{itemize}

        \item \textbf{Access to Resources:}
            \begin{itemize}
                \item Essential resources such as textbooks and academic papers will be accessible in the classroom.
                \item Technological tools, including computers and software, will be provided during workdays.
                \item \textit{Example:} Access to software like Excel or SPSS can significantly aid in conducting statistical analyses.
            \end{itemize}
    \end{itemize}

    \begin{block}{Key Points to Emphasize}
        \begin{itemize}
            \item Engage actively with faculty for personalized assistance.
            \item Collaborate to harness collective knowledge.
            \item Utilize resources to enhance project quality.
        \end{itemize}
    \end{block}
\end{frame}

\begin{frame}[fragile]
    \frametitle{Feedback and Assessment}
    % Introduction to the topic
    Feedback is essential in guiding your final project, helping you identify strengths and areas for improvement.
\end{frame}

\begin{frame}[fragile]
    \frametitle{Understanding the Feedback Mechanism}
    % Discuss the types of feedback
    \begin{block}{Types of Feedback}
        \begin{itemize}
            \item \textbf{Formative Feedback}: Ongoing feedback during project development.
            \item \textbf{Summative Feedback}: Evaluative feedback provided at project completion.
        \end{itemize}
    \end{block}
\end{frame}

\begin{frame}[fragile]
    \frametitle{Assessment of Progress}
    % Outline assessment methods
    Progress will be monitored through structured methods:
    \begin{itemize}
        \item \textbf{Weekly Check-ins}: Meetings with faculty to assess milestones.
        \item \textbf{Peer Reviews}: Evaluation of each other's work for diverse perspectives.
        \item \textbf{Progress Reports}: Written reports submitted midway to highlight achievements and strategies.
    \end{itemize}
\end{frame}

\begin{frame}[fragile]
    \frametitle{Criteria for Evaluation}
    % Discussion of evaluation criteria
    Evaluation will focus on:
    \begin{enumerate}
        \item \textbf{Creativity and Originality}: Innovation in the project.
        \item \textbf{Research and Evidence}: Use of credible sources.
        \item \textbf{Technical Execution}: Appropriate methods and tools.
        \item \textbf{Presentation Quality}: Clarity and engagement of the communication.
        \item \textbf{Collaboration and Team Dynamics}: Effectiveness of teamwork.
    \end{enumerate}
\end{frame}

\begin{frame}[fragile]
    \frametitle{Key Points and Conclusion}
    % Highlighting key takeaways
    \begin{itemize}
        \item \textbf{Timely Feedback}: Engage actively for better project outcomes.
        \item \textbf{Continuous Improvement}: Use feedback for enhancement.
        \item \textbf{Structured Evaluation}: Align project goals with evaluation criteria.
    \end{itemize}
    Remember, the feedback process enhances both your learning experience and the quality of your final project.
\end{frame}

\begin{frame}[fragile]
    \frametitle{Best Practices for Collaboration}
    % Introduction to effective collaboration
    Successful group projects depend on effective collaboration. By employing strategies for communication and task delegation, teams can enhance productivity, foster creativity, and meet project goals more efficiently.
\end{frame}

\begin{frame}[fragile]
    \frametitle{Key Practices for Effective Collaboration - Part 1}
    \begin{enumerate}
        \item \textbf{Establish Clear Communication}
            \begin{itemize}
                \item \textbf{Open Invitation for Ideas}: Encourage all team members to share their thoughts, ensuring everyone feels valued and included.
                \item \textbf{Regular Check-ins}: Schedule consistent meetings (weekly, bi-weekly) to maintain transparency about progress and address any concerns. Use tools like Zoom or Google Meet for virtual meetings.
                \item \textbf{Utilize Collaboration Tools}: Leverage platforms like Slack, Microsoft Teams, or Asana for ongoing discussions and updates, thus minimizing misunderstandings.
            \end{itemize}
            \item \textit{Example}: A design team uses Trello to track tasks, allowing members to comment and communicate on specific project elements as they progress.
 
        \item \textbf{Set Defined Roles and Responsibilities}
            \begin{itemize}
                \item \textbf{Assign Roles Based on Strengths}: Identify the skills and strengths of each member to ensure roles are assigned according to what each person does best, enhancing overall team performance.
                \item \textbf{Create a Responsibility Matrix}: Tools like RACI (Responsible, Accountable, Consulted, Informed) charts can clarify responsibilities and avoid overlap.
            \end{itemize}
            \item \textit{Example}: In a project team, one member focuses on research, another on design, while a third manages timelines, ensuring all aspects are covered without redundancy.
    \end{enumerate}
\end{frame}

\begin{frame}[fragile]
    \frametitle{Key Practices for Effective Collaboration - Part 2}
    \begin{enumerate}
        \setcounter{enumi}{2} % Continue from previous list
        \item \textbf{Foster a Collaborative Culture}
            \begin{itemize}
                \item \textbf{Encourage Constructive Feedback}: Create an environment where team members can give and receive feedback positively. This promotes growth and improves the end result.
                \item \textbf{Celebrate Milestones Together}: Acknowledge individual and team achievements to boost morale and motivation, whether through simple shout-outs or small rewards.
            \end{itemize}

        \item \textbf{Stay Organized and Document Everything}
            \begin{itemize}
                \item \textbf{Use Shared Documents}: Tools like Google Docs or Microsoft SharePoint allow team members to collaboratively edit and track versions of documents in real-time.
                \item \textbf{Maintain Task Lists}: Clearly defined task lists help in tracking progress. Use project management software to visualize workloads and deadlines.
            \end{itemize}
            \item \textit{Example}: A research group maintains a shared Google Drive folder to store all documents, logs, and revisions, ensuring everyone has access to the most current information.
    \end{enumerate}
\end{frame}

\begin{frame}[fragile]
    \frametitle{Conclusion: Emphasizing Successful Collaboration}
    To summarize, adopting these best practices—effective communication, defined roles, fostering collaboration, and organized documentation—can significantly enhance group dynamics and project outcomes. Implement these strategies within your teams to pave the way for successful collaboration during your final projects.

    \textbf{Key Points to Remember:}
    \begin{itemize}
        \item Clear communication is essential for minimizing misunderstandings.
        \item Assign roles based on individual strengths for optimal productivity.
        \item Foster a supportive environment where feedback is welcomed.
        \item Effective organization helps in tracking progress and maintaining clarity.
    \end{itemize}
\end{frame}

\begin{frame}[fragile]
    \frametitle{Common Challenges and Solutions - Introduction}
    \begin{block}{Overview}
    During the completion of your final project, you may encounter various challenges that can affect your progress or the quality of your work. Recognizing these challenges and implementing effective strategies to overcome them is crucial for a successful project outcome.
    \end{block}
\end{frame}

\begin{frame}[fragile]
    \frametitle{Common Challenges}
    \begin{enumerate}
        \item \textbf{Communication Barriers}
        \begin{itemize}
            \item Misunderstandings among team members can lead to duplication of effort.
            \item Example: A team member may not receive the latest version of a document.
        \end{itemize}

        \item \textbf{Time Management Issues}
        \begin{itemize}
            \item Balancing multiple responsibilities can lead to procrastination.
            \item Example: A team may underestimate the time needed for tasks.
        \end{itemize}

        \item \textbf{Conflicts Within the Team}
        \begin{itemize}
            \item Differing opinions and work styles can lead to disagreements.
            \item Example: One member prefers one design while another prefers a different approach.
        \end{itemize}
    \end{enumerate}
\end{frame}

\begin{frame}[fragile]
    \frametitle{Solutions to Common Challenges}
    \begin{enumerate}
        \item \textbf{Communication Barriers Solutions}
        \begin{itemize}
            \item \textbf{Regular Check-ins:} Schedule daily or weekly meetings.
            \item \textbf{Use Collaborative Tools:} Use tools like Trello or Slack.
        \end{itemize}

        \item \textbf{Time Management Solutions}
        \begin{itemize}
            \item \textbf{Set Milestones:} Break the project into smaller tasks.
            \item \textbf{Prioritize Tasks:} Use the Eisenhower Matrix for task distinction.
        \end{itemize}

        \item \textbf{Conflicts Solutions}
        \begin{itemize}
            \item \textbf{Establish Ground Rules:} Agree on communication norms.
            \item \textbf{Compromise:} Encourage open discussions and brainstorming.
        \end{itemize}

        \item \textbf{Resource Limitations Solutions}
        \begin{itemize}
            \item \textbf{Identify Needs Early:} Assess resource requirements at the start.
            \item \textbf{Seek Support:} Utilize university resources like libraries.
        \end{itemize}

        \item \textbf{Lack of Motivation Solutions}
        \begin{itemize}
            \item \textbf{Foster a Positive Environment:} Celebrate small wins.
            \item \textbf{Encourage Breaks:} Promote regular breaks to refresh.
        \end{itemize}
    \end{enumerate}
\end{frame}

\begin{frame}[fragile]
    \frametitle{Key Points and Conclusion}
    \begin{block}{Key Points to Emphasize}
    \begin{itemize}
        \item Recognize potential roadblocks early to develop proactive strategies.
        \item Collaboration and effective communication are vital.
        \item Be adaptable and ready to change strategies as needed.
    \end{itemize}
    \end{block}

    \begin{block}{Conclusion}
    By being aware of common challenges and employing the suggested solutions, you can enhance your project experience and work towards achieving your goals effectively.
    \end{block}
\end{frame}

\begin{frame}[fragile]
    \frametitle{Conclusion and Next Steps - Overview}
    % This frame provides a summary of key concepts discussed in the final project workdays.
    \begin{block}{Key Points}
        \begin{enumerate}
            \item Project Goals and Objectives
            \item Time Management
            \item Collaboration and Seeking Help
            \item Documenting Progress
            \item Adapting and Problem Solving
        \end{enumerate}
    \end{block}
    \begin{block}{Immediate Next Steps}
        \begin{enumerate}
            \item Review Project Guidelines
            \item Set Weekly Objectives
            \item Schedule Feedback Sessions
            \item Organize Research and Resources
            \item Prepare for Presentations
        \end{enumerate}
    \end{block}
\end{frame}

\begin{frame}[fragile]
    \frametitle{Conclusion and Next Steps - Key Points}
    % This frame elaborates on the key points from the final project workdays.
    \begin{itemize}
        \item \textbf{Project Goals and Objectives:}
        \begin{itemize}
            \item Define clear goals to guide research and development.
            \item Example: Research on solar cells might focus on efficiency and prototyping.
        \end{itemize}
        
        \item \textbf{Time Management:}
        \begin{itemize}
            \item Break work into manageable tasks with deadlines.
            \item Example: Schedule time for presentation content creation, design, and rehearsal.
        \end{itemize}
        
        \item \textbf{Collaboration and Seeking Help:}
        \begin{itemize}
            \item Engage with peers for feedback and idea exchange.
            \item Example: Create study groups to critique work.
        \end{itemize}
        
        \item \textbf{Documenting Progress:}
        \begin{itemize}
            \item Maintain a log to track advancements.
            \item Example Log Entry: “Week 1: Researched technologies, initiated prototype.”
        \end{itemize}
        
        \item \textbf{Adapting and Problem Solving:}
        \begin{itemize}
            \item Stay flexible to adjust strategies based on feedback.
            \item Key Takeaway: Regularly review and adapt project plans.
        \end{itemize}
    \end{itemize}
\end{frame}

\begin{frame}[fragile]
    \frametitle{Conclusion and Next Steps - Immediate Actions}
    % This frame outlines the immediate next steps for students.
    \begin{itemize}
        \item \textbf{Review Project Guidelines:}
        \begin{itemize}
            \item Ensure all project criteria are addressed.
        \end{itemize}
        
        \item \textbf{Set Weekly Objectives:}
        \begin{itemize}
            \item Break down the project into specific weekly tasks.
            \item Example: “By Week 2, complete literature review and project outline.”
        \end{itemize}
        
        \item \textbf{Schedule Feedback Sessions:}
        \begin{itemize}
            \item Plan meetings with instructors or peers for insights and criticisms.
        \end{itemize}
        
        \item \textbf{Organize Research and Resources:}
        \begin{itemize}
            \item Compile research materials in one accessible location.
            \item Consider tools like Google Drive or project management software.
        \end{itemize}
        
        \item \textbf{Prepare for Presentations:}
        \begin{itemize}
            \item Outline presentations focusing on objectives and outcomes.
            \item Use storytelling techniques to engage the audience.
        \end{itemize}
    \end{itemize}
\end{frame}


\end{document}