\documentclass[aspectratio=169]{beamer}

% Theme and Color Setup
\usetheme{Madrid}
\usecolortheme{whale}
\useinnertheme{rectangles}
\useoutertheme{miniframes}

% Additional Packages
\usepackage[utf8]{inputenc}
\usepackage[T1]{fontenc}
\usepackage{graphicx}
\usepackage{booktabs}
\usepackage{listings}
\usepackage{amsmath}
\usepackage{amssymb}
\usepackage{xcolor}
\usepackage{tikz}
\usepackage{pgfplots}
\pgfplotsset{compat=1.18}
\usetikzlibrary{positioning}
\usepackage{hyperref}

% Custom Colors
\definecolor{myblue}{RGB}{31, 73, 125}
\definecolor{mygray}{RGB}{100, 100, 100}
\definecolor{mygreen}{RGB}{0, 128, 0}
\definecolor{myorange}{RGB}{230, 126, 34}
\definecolor{mycodebackground}{RGB}{245, 245, 245}

% Set Theme Colors
\setbeamercolor{structure}{fg=myblue}
\setbeamercolor{frametitle}{fg=white, bg=myblue}
\setbeamercolor{title}{fg=myblue}
\setbeamercolor{section in toc}{fg=myblue}
\setbeamercolor{item projected}{fg=white, bg=myblue}
\setbeamercolor{block title}{bg=myblue!20, fg=myblue}
\setbeamercolor{block body}{bg=myblue!10}
\setbeamercolor{alerted text}{fg=myorange}

% Set Fonts
\setbeamerfont{title}{size=\Large, series=\bfseries}
\setbeamerfont{frametitle}{size=\large, series=\bfseries}
\setbeamerfont{caption}{size=\small}
\setbeamerfont{footnote}{size=\tiny}

% Document Start
\begin{document}

\frame{\titlepage}

\begin{frame}[fragile]
    \frametitle{Introduction to SQL for Data Retrieval}
    \begin{block}{Overview of SQL}
        Structured Query Language (SQL) is a standardized programming language used for managing and manipulating relational databases. It enables operations like data creation, updating, and retrieval.
    \end{block}
\end{frame}

\begin{frame}[fragile]
    \frametitle{Significance of SQL in Data Retrieval}
    \begin{itemize}
        \item \textbf{Data Access and Manipulation}: SQL provides powerful commands to query and manipulate data effectively.
        \item \textbf{Data Organization}: Define how data is structured and related within the database.
        \item \textbf{Standardization}: Once learned, SQL concepts are applicable across various DBMS like MySQL, PostgreSQL, SQL Server, and Oracle.
    \end{itemize}
\end{frame}

\begin{frame}[fragile]
    \frametitle{Key Components of SQL for Data Retrieval}
    \begin{enumerate}
        \item \textbf{SELECT Statement}:
            \begin{lstlisting}
SELECT first_name, last_name FROM employees;
            \end{lstlisting}
            Retrieves `first_name` and `last_name` from the `employees` table.

        \item \textbf{WHERE Clause}:
            \begin{lstlisting}
SELECT * FROM employees WHERE department = 'Sales';
            \end{lstlisting}
            Retrieves all columns where the department is 'Sales'.

        \item \textbf{ORDER BY Clause}:
            \begin{lstlisting}
SELECT first_name, last_name FROM employees ORDER BY last_name ASC;
            \end{lstlisting}
            Retrieves names sorted by last name alphabetically.
    \end{enumerate}
\end{frame}

\begin{frame}[fragile]
    \frametitle{Why Learn SQL for Data Retrieval?}
    \begin{itemize}
        \item \textbf{Efficiency}: SQL executes complex queries with simple statements.
        \item \textbf{Versatility}: Handles various data types and access patterns.
        \item \textbf{Foundational Skill}: Essential for careers in data analytics, data science, and database management.
    \end{itemize}
\end{frame}

\begin{frame}[fragile]
    \frametitle{Key Points to Remember}
    \begin{itemize}
        \item SQL stands for Structured Query Language and is essential for data retrieval.
        \item The SELECT statement is the cornerstone of data queries in SQL.
        \item SQL enables efficient data filtering, sorting, and manipulation.
        \item Mastery of SQL is crucial for various technology and analytics roles.
    \end{itemize}
\end{frame}

\begin{frame}[fragile]
    \frametitle{Practical SQL Query Example}
    Here’s a practical example of an SQL query that compiles several components:
    \begin{lstlisting}
SELECT employee_id, first_name, last_name, salary
FROM employees
WHERE salary > 50000 
ORDER BY last_name DESC;
    \end{lstlisting}
    This query retrieves employee IDs, first names, last names, and salaries for all employees earning more than $50,000, sorted by last names in descending order.
\end{frame}

\begin{frame}[fragile]
    \frametitle{Conclusion}
    By grasping these foundational concepts of SQL, you are setting the stage for advanced data retrieval techniques, which we will explore further in the next slide!
\end{frame}

\begin{frame}[fragile]
    \frametitle{Understanding SQL Syntax - Overview}
    \begin{block}{Overview of SQL Syntax}
        Structured Query Language (SQL) is the standard language used to communicate with relational databases. Its syntax is designed to be both powerful and easy to understand. This slide outlines the fundamental components of SQL syntax, focusing particularly on the SELECT statement, which is used for retrieving data.
    \end{block}
\end{frame}

\begin{frame}[fragile]
    \frametitle{Understanding SQL Syntax - Key Components}
    \begin{itemize}
        \item \textbf{SQL Keywords:} 
        \begin{itemize}
            \item SQL is not case-sensitive; however, it's common practice to write keywords in uppercase for better readability.
        \end{itemize}
        \item \textbf{Basic Structure:} 
        \begin{itemize}
            \item SQL statements usually consist of:
            \begin{lstlisting}
SELECT column1, column2, ...
FROM table_name
WHERE condition;
            \end{lstlisting}
        \end{itemize}
        \begin{itemize}
            \item \textbf{SELECT:} Specifies the columns to be returned in the result set.
            \item \textbf{FROM:} Indicates the table from which to retrieve the data.
            \item \textbf{WHERE:} Optional condition to filter results.
        \end{itemize}
    \end{itemize}
\end{frame}

\begin{frame}[fragile]
    \frametitle{Understanding SQL Syntax - Examples}
    \begin{block}{Example of a Basic SELECT Statement}
        Consider a sample database table called \texttt{employees} with columns \texttt{first_name}, \texttt{last_name}, and \texttt{department}.
        \begin{lstlisting}
SELECT first_name, last_name FROM employees;
        \end{lstlisting}
        \begin{itemize}
            \item This query retrieves the first and last names of all employees from the \texttt{employees} table.
        \end{itemize}
    \end{block}

    \begin{block}{Example with a WHERE Clause}
        To filter results by department:
        \begin{lstlisting}
SELECT first_name, last_name 
FROM employees 
WHERE department = 'IT';
        \end{lstlisting}
        \begin{itemize}
            \item This query retrieves only those records where the \texttt{department} is "IT".
        \end{itemize}
    \end{block}
\end{frame}

\begin{frame}[fragile]
  \frametitle{CRUD Operations - Overview}
  % Introduction to CRUD operations
  CRUD stands for Create, Read, Update, and Delete. These are the four basic operations that can be performed on data in a database. In the context of SQL, these operations allow users to interact with the data stored in the database effectively.
\end{frame}

\begin{frame}[fragile]
  \frametitle{CRUD Operations - Create}
  \begin{itemize}
    \item \textbf{Definition}: The operation used to insert new records into a database table.
    \item \textbf{SQL Example}:
    \begin{lstlisting}[language=SQL]
INSERT INTO Employees (FirstName, LastName, Age) VALUES ('John', 'Doe', 30);
    \end{lstlisting}
    \item \textbf{Key Point}: Creating records is essential for building and populating databases.
  \end{itemize}
\end{frame}

\begin{frame}[fragile]
  \frametitle{CRUD Operations - Read, Update, Delete}
  \begin{enumerate}
    \item \textbf{Read}
    \begin{itemize}
      \item \textbf{Definition}: This operation retrieves data from the database.
      \item \textbf{SQL Example}:
      \begin{lstlisting}[language=SQL]
SELECT * FROM Employees WHERE Age > 25;
      \end{lstlisting}
      \item \textbf{Key Point}: Reading data is vital for data analysis, reporting, and decision-making processes.
    \end{itemize}

    \item \textbf{Update}
    \begin{itemize}
      \item \textbf{Definition}: This operation modifies existing records in the database.
      \item \textbf{SQL Example}:
      \begin{lstlisting}[language=SQL]
UPDATE Employees SET Age = 31 WHERE FirstName = 'John' AND LastName = 'Doe';
      \end{lstlisting}
      \item \textbf{Key Point}: Updating records ensures the data remains current and accurate.
    \end{itemize}

    \item \textbf{Delete}
    \begin{itemize}
      \item \textbf{Definition}: This operation removes records from a database.
      \item \textbf{SQL Example}:
      \begin{lstlisting}[language=SQL]
DELETE FROM Employees WHERE Age < 18;
      \end{lstlisting}
      \item \textbf{Key Point}: Deleting records helps maintain data integrity and relevance.
    \end{itemize}
  \end{enumerate}
\end{frame}

\begin{frame}[fragile]
  \frametitle{The Role of CRUD Operations in Data Retrieval}
  \begin{itemize}
    \item \textbf{Data Management}: CRUD operations are fundamental for managing and maintaining data within a database.
    \item \textbf{User Interactions}: Applications often utilize CRUD operations behind the scenes to facilitate user interactions, data display, and manipulation.
    \item \textbf{Application Development}: CRUD operations are an integral part of many applications, enabling users to perform necessary actions on data.
  \end{itemize}
\end{frame}

\begin{frame}[fragile]
  \frametitle{Summary of CRUD Operations}
  \begin{itemize}
    \item CRUD operations form the backbone of data handling in SQL, serving essential purposes in creating, retrieving, updating, and deleting records.
    \item Understanding these operations is crucial for efficient database management and effective data retrieval strategies.
  \end{itemize}
  
  % Transition to next topic
  By mastering CRUD operations, you will gain a solid foundation for working with databases and retrieving data effectively in future SQL queries.
\end{frame}

\begin{frame}[fragile]
    \frametitle{Using SELECT Statement - Overview}
    \begin{block}{Overview}
        The SELECT statement is fundamental for data retrieval in SQL databases. 
        It allows users to query data, accessing specific information from one or 
        more tables in a database. Understanding how to effectively use the 
        SELECT statement is critical for performing CRUD operations, especially 
        the ``Read'' aspect.
    \end{block}
\end{frame}

\begin{frame}[fragile]
    \frametitle{Using SELECT Statement - Basic Syntax}
    \begin{block}{Basic Syntax}
        The basic syntax of a SELECT statement is as follows:
        \begin{lstlisting}[language=SQL]
SELECT column1, column2, ...
FROM table_name;
        \end{lstlisting}
        \begin{itemize}
            \item \texttt{SELECT}: Command used to specify which columns to retrieve.
            \item \texttt{FROM}: Indicates the table from which to pull the data.
        \end{itemize}
        \textbf{Example:}
        \begin{lstlisting}[language=SQL]
SELECT first_name, last_name
FROM employees;
        \end{lstlisting}
        This query retrieves the \texttt{first\_name} and \texttt{last\_name} 
        of all employees from the \texttt{employees} table.
    \end{block}
\end{frame}

\begin{frame}[fragile]
    \frametitle{Using SELECT Statement - Key Concepts}
    \begin{block}{Selecting All Columns}
        To retrieve all columns from a table, use the asterisk (*) symbol:
        \begin{lstlisting}[language=SQL]
SELECT * 
FROM table_name;
        \end{lstlisting}
        \textbf{Example:}
        \begin{lstlisting}[language=SQL]
SELECT * 
FROM employees;
        \end{lstlisting}
        This retrieves all information for every record in the \texttt{employees} table.
    \end{block}

    \begin{block}{Using Aliases}
        Aliases provide temporary names to tables or columns for the purpose of 
        a specific SQL query, improving readability.
        \begin{lstlisting}[language=SQL]
SELECT column1 AS alias_name
FROM table_name;
        \end{lstlisting}
        \textbf{Example:}
        \begin{lstlisting}[language=SQL]
SELECT first_name AS "First Name", last_name AS "Last Name"
FROM employees;
        \end{lstlisting}
        This will output columns labeled ``First Name'' and ``Last Name''.
    \end{block}
\end{frame}

\begin{frame}[fragile]
    \frametitle{Using SELECT Statement - Joins and Conclusion}
    \begin{block}{Retrieving Data from Multiple Tables (Joins)}
        To fetch data from multiple tables, SQL JOINs are used.
        
        \textbf{Example: Using INNER JOIN}
        \begin{lstlisting}[language=SQL]
SELECT employees.first_name, departments.department_name
FROM employees
INNER JOIN departments ON employees.department_id = departments.id;
        \end{lstlisting}
        This retrieves a list of employee first names along with their 
        respective department names.
    \end{block}

    \begin{block}{Key Points}
        \begin{itemize}
            \item \textbf{Clarity}: Use clear and meaningful column names (or aliases) for better understanding of results.
            \item \textbf{Efficiency}: Retrieve only the necessary columns/data to improve performance.
            \item \textbf{Flexibility}: The SELECT statement can be modified to include various conditions, joins, and sorts.
        \end{itemize}
    \end{block}

    \begin{block}{Conclusion}
        The SELECT statement is the gateway to data retrieval in SQL, offering 
        flexibility and power in how we access stored information. Familiarizing yourself 
        with its various components and usages will enable more sophisticated data 
        queries and analysis.
    \end{block}
\end{frame}

\begin{frame}[fragile]
    \frametitle{Filtering Data with WHERE Clause}
    % Introduction to WHERE clause
    \begin{block}{Understanding the WHERE Clause}
        The \textbf{WHERE clause} is a critical component of SQL that allows you to filter records based on specified conditions. 
        It ensures that only the data meeting certain criteria is retrieved, making your queries more efficient and focused.
    \end{block}
\end{frame}

\begin{frame}[fragile]
    \frametitle{Key Concepts of the WHERE Clause}
    % Overview of key concepts
    \begin{itemize}
        \item \textbf{Filtering Data:} Restricting the results of your SELECT statement to only those rows that fulfill a specific condition.
        \item \textbf{Conditions:} Can include comparisons, logical operators, and patterns for string matching.
        \item \textbf{Syntax:} The basic syntax is:
            \begin{lstlisting}[language=SQL]
SELECT column1, column2, ...
FROM table_name
WHERE condition;
            \end{lstlisting}
    \end{itemize}
\end{frame}

\begin{frame}[fragile]
    \frametitle{Examples of Using the WHERE Clause}
    % Examples illustrating the usage of WHERE clause
    \begin{enumerate}
        \item \textbf{Basic Filtering:} To retrieve all employees in `Sales`:
            \begin{lstlisting}[language=SQL]
SELECT *
FROM Employees
WHERE Department = 'Sales';
            \end{lstlisting}
        
        \item \textbf{Using Comparison Operators:} To find employees with a salary greater than $50,000:
            \begin{lstlisting}[language=SQL]
SELECT *
FROM Employees
WHERE Salary > 50000;
            \end{lstlisting}
        
        \item \textbf{Combining Conditions:} To select employees from `HR` with a salary under $40,000 or from `Sales` with a higher salary:
            \begin{lstlisting}[language=SQL]
SELECT *
FROM Employees
WHERE (Department = 'HR' AND Salary < 40000)
  OR (Department = 'Sales' AND Salary > 50000);
            \end{lstlisting}

        \item \textbf{Using LIKE for Pattern Matching:} To find names starting with "J":
            \begin{lstlisting}[language=SQL]
SELECT *
FROM Employees
WHERE Name LIKE 'J%';
            \end{lstlisting}
    \end{enumerate}
\end{frame}

\begin{frame}[fragile]
    \frametitle{Sorting Results with ORDER BY}
    \begin{block}{Introduction to ORDER BY}
        The \texttt{ORDER BY} clause in SQL allows you to sort the result set of your query by one or more columns. Sorting data enhances readability and helps in analyzing the results more effectively. Without sorting, data can appear unordered, making it difficult to interpret.
    \end{block}
\end{frame}

\begin{frame}[fragile]
    \frametitle{Syntax of ORDER BY}
    \begin{block}{Basic Syntax}
        The basic syntax for using the \texttt{ORDER BY} clause is as follows:
        \begin{lstlisting}
SELECT column1, column2, ...
FROM table_name
WHERE condition
ORDER BY column1 [ASC|DESC], column2 [ASC|DESC], ...;
        \end{lstlisting}
    \end{block}
    
    \begin{itemize}
        \item \textbf{ASC}: Sorts the data in ascending order (default).
        \item \textbf{DESC}: Sorts the data in descending order.
    \end{itemize}
\end{frame}

\begin{frame}[fragile]
    \frametitle{Key Points and Examples}
    \begin{block}{Key Points}
        \begin{itemize}
            \item \textbf{Multiple Columns}: You can sort by multiple columns by separating them with commas. The order of columns in the \texttt{ORDER BY} clause affects the sorting hierarchy.
            \item \textbf{Default Sorting}: If you do not specify \texttt{ASC} or \texttt{DESC}, SQL assumes \texttt{ASC}.
        \end{itemize}
    \end{block}
    
    \begin{block}{Examples}
        \begin{itemize}
            \item \textbf{Single Column Sort}:
            \begin{lstlisting}
SELECT first_name, last_name
FROM employees
ORDER BY last_name ASC;
            \end{lstlisting}
            
            \item \textbf{Multiple Column Sort}:
            \begin{lstlisting}
SELECT first_name, last_name, department
FROM employees
ORDER BY department ASC, last_name ASC;
            \end{lstlisting}
            
            \item \textbf{Descending Order}:
            \begin{lstlisting}
SELECT product_name, price
FROM products
ORDER BY price DESC;
            \end{lstlisting}
        \end{itemize}
    \end{block}
\end{frame}

\begin{frame}[fragile]
    \frametitle{Practical Application and Conclusion}
    \begin{block}{Practical Application}
        Sorting data is particularly useful when:
        \begin{itemize}
            \item Presenting results to stakeholders who need clearly organized data.
            \item Analyzing trends over time to identify patterns.
            \item Preparing data for reporting purposes.
        \end{itemize}
    \end{block}

    \begin{block}{Conclusion}
        Sorting results with the \texttt{ORDER BY} clause is a fundamental aspect of SQL that greatly enhances the clarity and usability of retrieved data. Utilize it to organize your datasets efficiently and effectively!
    \end{block}
\end{frame}

\begin{frame}[fragile]
    \frametitle{Aggregate Functions - Introduction}
    % Introduction to aggregate functions in SQL
    Aggregate functions are pivotal in SQL that allow for calculations over a set of values, returning a single summary value. 
    These functions are crucial for:
    \begin{itemize}
        \item Data analysis
        \item Reporting
        \item Deriving insights from large datasets
    \end{itemize}
\end{frame}

\begin{frame}[fragile]
    \frametitle{Common Aggregate Functions}
    % Overview of common SQL aggregate functions
    Here are some common aggregate functions:

    \begin{enumerate}
        \item \textbf{COUNT()}
            \begin{itemize}
                \item \textbf{Purpose}: Returns the total number of rows matching a specified criterion.
                \item \textbf{Syntax}: \texttt{COUNT(expression)}
                \item \textbf{Example}:
                \begin{lstlisting}
SELECT COUNT(*) AS total_employees FROM employees;
                \end{lstlisting}
                This counts all employees in the "employees" table.
            \end{itemize}

        \item \textbf{SUM()}
            \begin{itemize}
                \item \textbf{Purpose}: Calculates the total sum of a numeric column.
                \item \textbf{Syntax}: \texttt{SUM(column_name)}
                \item \textbf{Example}:
                \begin{lstlisting}
SELECT SUM(salary) AS total_salary FROM employees;
                \end{lstlisting}
                This computes the total salary expense for all employees.
            \end{itemize}
    \end{enumerate}
\end{frame}

\begin{frame}[fragile]
    \frametitle{Common Aggregate Functions - Part 2}
    % Continuation of common SQL aggregate functions
    Continuing with other aggregate functions:

    \begin{enumerate}
        \setcounter{enumi}{2} % Start from 3
        \item \textbf{AVG()}
            \begin{itemize}
                \item \textbf{Purpose}: Computes the average value of a numeric column.
                \item \textbf{Syntax}: \texttt{AVG(column_name)}
                \item \textbf{Example}:
                \begin{lstlisting}
SELECT AVG(salary) AS average_salary FROM employees;
                \end{lstlisting}
                This calculates the average salary among all employees.
            \end{itemize}

        \item \textbf{MAX()}
            \begin{itemize}
                \item \textbf{Purpose}: Finds the maximum value in a set of values.
                \item \textbf{Syntax}: \texttt{MAX(column_name)}
                \item \textbf{Example}:
                \begin{lstlisting}
SELECT MAX(salary) AS highest_salary FROM employees;
                \end{lstlisting}
                This retrieves the highest salary from the "employees" table.
            \end{itemize}

        \item \textbf{MIN()}
            \begin{itemize}
                \item \textbf{Purpose}: Identifies the minimum value in a set of values.
                \item \textbf{Syntax}: \texttt{MIN(column_name)}
                \item \textbf{Example}:
                \begin{lstlisting}
SELECT MIN(salary) AS lowest_salary FROM employees;
                \end{lstlisting}
                This locates the lowest salary among employees.
            \end{itemize}
    \end{enumerate}
\end{frame}

\begin{frame}[fragile]
    \frametitle{Key Points About Aggregate Functions}
    % Important notes regarding aggregate functions
    \begin{itemize}
        \item Aggregate functions simplify complex data into summary forms, making analysis more efficient.
        \item They can be combined with the \textbf{GROUP BY} clause to group rows sharing a property.
        \item Note that aggregate functions ignore NULL values in calculations.
    \end{itemize}
\end{frame}

\begin{frame}
    \frametitle{Grouping Data with GROUP BY}
    % Introduction to the GROUP BY clause in SQL
    The \texttt{GROUP BY} clause allows for the consolidation of identical data into groups.
    This is essential for performing aggregate functions on data rather than individual records.
\end{frame}

\begin{frame}[fragile]
    \frametitle{Understanding GROUP BY}
    \begin{itemize}
        \item Consolidates multiple rows with identical values into summary rows.
        \item Often used with aggregate functions: \texttt{COUNT}, \texttt{SUM}, \texttt{AVG}, \texttt{MAX}, \texttt{MIN}.
    \end{itemize}
    
    \textbf{Purpose of GROUP BY}:
    \begin{itemize}
        \item Analyze trends, patterns, or summaries within data.
    \end{itemize}
\end{frame}

\begin{frame}[fragile]
    \frametitle{Syntax and Example}
    % Showing the syntax and an example
    \textbf{Syntax}:
    \begin{lstlisting}
    SELECT column1, aggregate_function(column2)
    FROM table_name
    WHERE condition
    GROUP BY column1;
    \end{lstlisting}
    
    \textbf{Example}:
    \begin{lstlisting}
    SELECT product_id, SUM(quantity_sold) AS total_quantity
    FROM sales
    GROUP BY product_id;
    \end{lstlisting}
    \begin{block}{Result}
        \begin{tabular}{|c|c|}
            \hline
            product\_id & total\_quantity \\
            \hline
            1           & 300            \\
            2           & 150            \\
            3           & 450            \\
            \hline
        \end{tabular}
    \end{block}
\end{frame}

\begin{frame}[fragile]
    \frametitle{Advanced GROUP BY: HAVING Clause}
    % Discussing the HAVING clause in conjunction with GROUP BY
    \textbf{Using GROUP BY with HAVING}:
    \begin{itemize}
        \item The \texttt{HAVING} clause can filter groups created by \texttt{GROUP BY} based on aggregated results.
    \end{itemize}
    
    \textbf{Example}:
    \begin{lstlisting}
    SELECT product_id, SUM(quantity_sold) AS total_quantity
    FROM sales
    GROUP BY product_id
    HAVING SUM(quantity_sold) > 200;
    \end{lstlisting}
    
    \textbf{Key Points}:
    \begin{itemize}
        \item Every non-aggregated column in the \texttt{SELECT} statement must be included in the \texttt{GROUP BY}.
        \item Use \texttt{HAVING} for post-aggregation filtering.
    \end{itemize}
\end{frame}

\begin{frame}
    \frametitle{Practical Applications}
    % Discussing practical applications of GROUP BY
    \textbf{Practical Uses}:
    \begin{itemize}
        \item Analyzing sales performance by product.
        \item Summarizing customer purchases.
        \item Generating consolidated reports.
    \end{itemize}
    
    \textbf{Conclusion}:
    The \texttt{GROUP BY} clause is a powerful tool for data aggregation, enabling the transformation of detailed records into summary insights.
\end{frame}

\begin{frame}[fragile]
    \frametitle{Practice Exercise}
    % Practice exercise for students involving GROUP BY
    \textbf{Exercise}: 
    Write a query to calculate the average quantity sold per product 
    and filter out products with less than 50 total sales.
    
    \textit{Hint: Use \texttt{AVG} and \texttt{HAVING}.}
\end{frame}

\begin{frame}
    \frametitle{Joining Tables}
    % Brief overview of the topic.
    In SQL, joining tables allows us to combine rows from two or more tables based on a related column. 
    This is essential for retrieving meaningful data from relational databases where data is stored in different tables.
\end{frame}

\begin{frame}
    \frametitle{Types of Joins}
    % Introduction to different types of joins.
    \begin{block}{Types of Joins}
        \begin{enumerate}
            \item **INNER JOIN**
            \item **LEFT JOIN (LEFT OUTER JOIN)**
            \item **RIGHT JOIN (RIGHT OUTER JOIN)**
            \item **FULL OUTER JOIN**
        \end{enumerate}
    \end{block}
\end{frame}

\begin{frame}[fragile]
    \frametitle{INNER JOIN}
    \begin{itemize}
        \item \textbf{Description}: Returns only the rows that have matching values in both tables.
        \item \textbf{Use Case}: When you need only the data that exists in both tables.
        \item \textbf{Example}:
        \begin{lstlisting}[language=SQL]
SELECT A.CustomerID, A.CustomerName, B.OrderID
FROM Customers A
INNER JOIN Orders B ON A.CustomerID = B.CustomerID;
        \end{lstlisting}
        \item *This retrieves all customers who have placed orders.*
    \end{itemize}
\end{frame}

\begin{frame}[fragile]
    \frametitle{LEFT JOIN (LEFT OUTER JOIN)}
    \begin{itemize}
        \item \textbf{Description}: Returns all rows from the left table and the matched rows from the right table. If no match is found, NULLs are returned for columns from the right table.
        \item \textbf{Use Case}: When you want to include all records from the left table regardless of whether there are matches in the right table.
        \item \textbf{Example}:
        \begin{lstlisting}[language=SQL]
SELECT A.CustomerID, A.CustomerName, B.OrderID
FROM Customers A
LEFT JOIN Orders B ON A.CustomerID = B.CustomerID;
        \end{lstlisting}
        \item *This retrieves all customers, including those who haven’t placed any orders.*
    \end{itemize}
\end{frame}

\begin{frame}[fragile]
    \frametitle{RIGHT JOIN (RIGHT OUTER JOIN) \& FULL OUTER JOIN}
    % Details for RIGHT JOIN
    \begin{itemize}
        \item \textbf{RIGHT JOIN}: Returns all rows from the right table and the matched rows from the left table. If no match is found, NULLs are returned for columns from the left table.
        \item \textbf{Use Case}: Include all records from the right table.
        \item \textbf{Example}:
        \begin{lstlisting}[language=SQL]
SELECT A.CustomerID, A.CustomerName, B.OrderID
FROM Customers A
RIGHT JOIN Orders B ON A.CustomerID = B.CustomerID;
        \end{lstlisting}
        \item *This retrieves all orders, including orders that may not have associated customers.*
    \end{itemize}
    
    % Details for FULL OUTER JOIN
    \begin{itemize}
        \item \textbf{FULL OUTER JOIN}: Combines all records when there is a match in either left or right table records.
        \item \textbf{Example}:
        \begin{lstlisting}[language=SQL]
SELECT A.CustomerID, A.CustomerName, B.OrderID
FROM Customers A
FULL OUTER JOIN Orders B ON A.CustomerID = B.CustomerID;
        \end{lstlisting}
        \item *This retrieves all customers and all orders, providing a complete picture of the relationships.*
    \end{itemize}
\end{frame}

\begin{frame}
    \frametitle{Key Points to Remember}
    \begin{itemize}
        \item Joins are essential for combining related data across tables and implementing a relational structure.
        \item Understanding the difference between INNER, LEFT, RIGHT, and FULL JOINS is crucial in determining the results returned based on the data relationships.
        \item Always choose the appropriate join type based on the specific data retrieval needs to ensure accurate results.
    \end{itemize}
\end{frame}

\begin{frame}
    \frametitle{Conclusion}
    Mastering JOINs enables you to effectively navigate and analyze relational databases, making data retrieval more insightful and comprehensive. 
    Be sure to practice writing SQL queries using different JOIN types to reinforce your understanding!
\end{frame}

\begin{frame}[fragile]
    \frametitle{Subqueries - Introduction}
    A \textbf{subquery} is a query nested within another SQL query. It allows for complex data retrieval operations by enabling you to use the result of one query as a filter or input for another query. 
    \begin{itemize}
        \item Provides flexibility in data manipulation.
        \item Extremely useful with multiple tables or complicated criteria.
    \end{itemize}
\end{frame}

\begin{frame}[fragile]
    \frametitle{Subqueries - Types}
    \textbf{Types of Subqueries:}
    \begin{enumerate}
        \item \textbf{Single-Row Subquery:} Returns only one row as a result.
            \begin{block}{Example}
            \begin{lstlisting}
SELECT name 
FROM employees 
WHERE salary = (SELECT MAX(salary) FROM employees);
            \end{lstlisting}
            \end{block}
        
        \item \textbf{Multiple-Row Subquery:} Returns multiple rows.
            \begin{block}{Example}
            \begin{lstlisting}
SELECT name 
FROM employees 
WHERE department_id IN (SELECT id FROM departments WHERE location = 'New York');
            \end{lstlisting}
            \end{block}
        
        \item \textbf{Correlated Subquery:} Refers to a column from the outer query. Executes once for each row.
            \begin{block}{Example}
            \begin{lstlisting}
SELECT e1.name 
FROM employees e1 
WHERE e1.salary > (SELECT AVG(salary) FROM employees e2 WHERE e1.department_id = e2.department_id);
            \end{lstlisting}
            \end{block}
    \end{enumerate}
\end{frame}

\begin{frame}[fragile]
    \frametitle{Subqueries - Key Points and Tips}
    \textbf{Key Points to Emphasize:}
    \begin{itemize}
        \item Simplifies complex queries into manageable parts.
        \item May cause performance issues if not optimized, particularly correlated subqueries.
        \item Ensure expected results to avoid errors in main queries.
    \end{itemize}

    \textbf{Tips for Using Subqueries:}
    \begin{itemize}
        \item Use for derived values, not for simple joins.
        \item Test each subquery individually before integrating.
        \item Consider using \texttt{EXISTS} or \texttt{NOT EXISTS} when subqueries return no results.
    \end{itemize}
\end{frame}

\begin{frame}[fragile]
    \frametitle{Subqueries - Conclusion}
    Subqueries are essential tools for advanced data retrieval, enabling dynamic and precise SQL queries. 
    \begin{itemize}
        \item Mastering subqueries enhances your SQL skills.
        \item Facilitates complex data operations with ease.
        \item Lays foundation for advanced data retrieval practices.
    \end{itemize}
\end{frame}

\begin{frame}[fragile]
    \frametitle{Data Retrieval Best Practices - Introduction}
    \begin{block}{Overview}
        Effective data retrieval is crucial for optimizing database performance and ensuring that queries return the desired results quickly. This slide outlines best practices for writing efficient SQL queries, focusing on clarity, performance, and maintainability.
    \end{block}
\end{frame}

\begin{frame}[fragile]
    \frametitle{Best Practices for SQL Data Retrieval}
    \begin{enumerate}
        \item \textbf{Select Only Required Columns}
            \begin{itemize}
                \item Using \texttt{SELECT *} retrieves all columns, which can be inefficient.
                \item \textbf{Example:} 
                \begin{lstlisting}
SELECT first_name, last_name, department FROM employees;
                \end{lstlisting}
            \end{itemize}

        \item \textbf{Use WHERE Clauses Wisely}
            \begin{itemize}
                \item Filtering data reduces processing time.
                \item \textbf{Example:} 
                \begin{lstlisting}
SELECT first_name, last_name FROM employees WHERE department = 'Sales';
                \end{lstlisting}
            \end{itemize}

        \item \textbf{Limit Results with LIMIT/OFFSET}
            \begin{itemize}
                \item Limits rows returned to prevent overwhelming clients.
                \item \textbf{Example:} 
                \begin{lstlisting}
SELECT first_name, last_name FROM employees ORDER BY hire_date DESC LIMIT 10;
                \end{lstlisting}
            \end{itemize}
    \end{enumerate}
\end{frame}

\begin{frame}[fragile]
    \frametitle{Best Practices Continued}
    \begin{enumerate}
        \setcounter{enumi}{3} % Start from 4
        \item \textbf{Use Joins Effectively}
            \begin{itemize}
                \item Combine related data efficiently.
                \item \textbf{Example:} 
                \begin{lstlisting}
SELECT e.first_name, e.last_name, d.department_name 
FROM employees e 
JOIN departments d ON e.department_id = d.id;
                \end{lstlisting}
            \end{itemize}

        \item \textbf{Optimize Indexing}
            \begin{itemize}
                \item Use indexes on \texttt{WHERE}, \texttt{JOIN}, and \texttt{ORDER BY} columns.
            \end{itemize}

        \item \textbf{Avoid SELECT DISTINCT Unless Necessary}
            \begin{itemize}
                \item Can be resource-intensive. Use only to eliminate duplicates.
                \item \textbf{Example:} 
                \begin{lstlisting}
SELECT DISTINCT department FROM employees;
                \end{lstlisting}
            \end{itemize}

        \item \textbf{Regularly Review Query Performance}
            \begin{itemize}
                \item Use SQL performance tuning tools to optimize slow queries.
            \end{itemize}
    \end{enumerate}
\end{frame}

\begin{frame}[fragile]
    \frametitle{Case Study: Data Retrieval Scenario}
    \begin{block}{Introduction to the Case Study}
        In this case study, we will explore how a fictional retail company, "TechZone", effectively utilizes SQL queries to retrieve valuable insights from its customer and sales database.
    \end{block}
\end{frame}

\begin{frame}[fragile]
    \frametitle{Case Study: Scenario Overview}
    \begin{itemize}
        \item \textbf{Objective:} Analyze customer purchase behavior to develop targeted marketing strategies and improve inventory management.
    \end{itemize}

    \begin{block}{Database Tables Utilized}
        \begin{itemize}
            \item \textbf{Customers}
                \begin{itemize}
                    \item customer\_id
                    \item name
                    \item email
                    \item join\_date
                \end{itemize}
            \item \textbf{Orders}
                \begin{itemize}
                    \item order\_id
                    \item customer\_id
                    \item order\_date
                    \item total\_amount
                \end{itemize}
            \item \textbf{Products}
                \begin{itemize}
                    \item product\_id
                    \item product\_name
                    \item category
                    \item price
                \end{itemize}
        \end{itemize}
    \end{block}
\end{frame}

\begin{frame}[fragile]
    \frametitle{SQL Techniques Demonstrated}
    \begin{enumerate}
        \item \textbf{Basic Data Retrieval}
        \begin{lstlisting}[language=SQL]
SELECT * 
FROM Customers;
        \end{lstlisting}
        
        \item \textbf{Filtering Data with WHERE}
        \begin{lstlisting}[language=SQL]
SELECT * 
FROM Customers 
WHERE join_date > '2022-01-01';
        \end{lstlisting}

        \item \textbf{Joining Tables}
        \begin{lstlisting}[language=SQL]
SELECT C.name, O.order_id, O.total_amount 
FROM Customers C
JOIN Orders O 
ON C.customer_id = O.customer_id;
        \end{lstlisting}
    \end{enumerate}
\end{frame}

\begin{frame}[fragile]
    \frametitle{SQL Techniques Demonstrated (Cont.)}
    \begin{enumerate}
        \setcounter{enumi}{3}
        \item \textbf{Aggregating Data}
        \begin{lstlisting}[language=SQL]
SELECT C.name, SUM(O.total_amount) AS total_spent
FROM Customers C
JOIN Orders O ON C.customer_id = O.customer_id
GROUP BY C.name 
ORDER BY total_spent DESC;
        \end{lstlisting}

        \item \textbf{Date Functions to Analyze Trends}
        \begin{lstlisting}[language=SQL]
SELECT DATE_TRUNC('month', order_date) AS month, COUNT(order_id) AS order_count
FROM Orders
WHERE order_date >= '2023-01-01'
GROUP BY month
ORDER BY month;
        \end{lstlisting}
    \end{enumerate}
\end{frame}

\begin{frame}[fragile]
    \frametitle{Key Points and Conclusion}
    \begin{itemize}
        \item \textbf{Normalization:} Organizing data across different tables improves retrieval efficiency and management.
        \item \textbf{Performance Optimization:} Using indexing on common search fields enhances query performance.
        \item \textbf{Data Analysis:} SQL allows for complex analysis through aggregation and joining for informed decision-making.
        \item \textbf{Practical Application:} This case study clarifies how SQL can transform raw data into actionable insights.
    \end{itemize}

    \begin{block}{Conclusion}
        Understanding SQL for data retrieval empowers organizations like TechZone to make strategic decisions through data-driven insights.
    \end{block}
\end{frame}

\begin{frame}[fragile]
    \frametitle{Conclusion of SQL for Data Retrieval}
    % Summarizing key takeaways of SQL for effective data retrieval
    In conclusion, here are the key takeaways from our discussion on SQL for data retrieval:
    \begin{enumerate}
        \item \textbf{SQL Basics}: Understanding SQL syntax including commands such as \texttt{SELECT}, \texttt{WHERE}, \texttt{JOIN}, and \texttt{GROUP BY} is essential for effective data retrieval.
        
        \item \textbf{Techniques for Data Retrieval}:
        \begin{itemize}
            \item \textbf{Basic Queries}: 
                \begin{lstlisting}
SELECT first_name, last_name FROM employees WHERE department = 'Sales';
                \end{lstlisting}
            \item \textbf{Aggregations and Functions}: 
                \begin{lstlisting}
SELECT COUNT(*) FROM orders WHERE order_date >= '2023-01-01';
                \end{lstlisting}
            \item \textbf{Joining Tables}: 
                \begin{lstlisting}
SELECT customers.customer_name, orders.order_id 
FROM customers 
JOIN orders ON customers.customer_id = orders.customer_id;
                \end{lstlisting}
        \end{itemize}

        \item \textbf{Real-World Applications}: SQL enables data-driven decisions in business analytics, data science, and software development.
    \end{enumerate}
\end{frame}

\begin{frame}[fragile]
    \frametitle{Future Trends in SQL and Data Retrieval}
    % Discussing trends shaping the future landscape of SQL
    Looking ahead, several notable trends are emerging in SQL and data retrieval:
    \begin{itemize}
        \item \textbf{Emergence of NoSQL}: NoSQL databases like MongoDB and Cassandra are becoming popular for handling unstructured data.
        
        \item \textbf{Advanced Analytics and Machine Learning}: SQL integration with machine learning tools enhances data preparation for analysis.
        
        \item \textbf{Cloud Databases}: Platforms such as Amazon RDS and Google Cloud SQL provide scalability and flexibility.
        
        \item \textbf{Data Privacy and Security}: Compliance with regulations like GDPR makes secure data retrieval a priority.
        
        \item \textbf{SQL on Big Data Platforms}: SQL-like querying languages are essential for work with big data technologies like Apache Hive’s HQL.
    \end{itemize}
\end{frame}

\begin{frame}[fragile]
    \frametitle{Key Points to Remember}
    % Emphasizing key points for the audience
    As we conclude, here are the key points to remember:
    \begin{itemize}
        \item Mastering SQL basic commands lays the foundation for effective data retrieval.
        \item Combining techniques such as joins and aggregations allows for complex queries and insightful analysis.
        \item Staying informed about technological trends ensures your skills remain relevant in a changing landscape.
    \end{itemize}
    By staying updated, you can enhance your career prospects and adapt to future challenges in data management.
\end{frame}


\end{document}