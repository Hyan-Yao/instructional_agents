\documentclass[aspectratio=169]{beamer}

% Theme and Color Setup
\usetheme{Madrid}
\usecolortheme{whale}
\useinnertheme{rectangles}
\useoutertheme{miniframes}

% Additional Packages
\usepackage[utf8]{inputenc}
\usepackage[T1]{fontenc}
\usepackage{graphicx}
\usepackage{booktabs}
\usepackage{listings}
\usepackage{amsmath}
\usepackage{amssymb}
\usepackage{xcolor}
\usepackage{tikz}
\usepackage{pgfplots}
\pgfplotsset{compat=1.18}
\usetikzlibrary{positioning}
\usepackage{hyperref}

% Custom Colors
\definecolor{myblue}{RGB}{31, 73, 125}
\definecolor{mygray}{RGB}{100, 100, 100}
\definecolor{mygreen}{RGB}{0, 128, 0}
\definecolor{myorange}{RGB}{230, 126, 34}
\definecolor{mycodebackground}{RGB}{245, 245, 245}

% Set Theme Colors
\setbeamercolor{structure}{fg=myblue}
\setbeamercolor{frametitle}{fg=white, bg=myblue}
\setbeamercolor{title}{fg=myblue}
\setbeamercolor{section in toc}{fg=myblue}
\setbeamercolor{item projected}{fg=white, bg=myblue}
\setbeamercolor{block title}{bg=myblue!20, fg=myblue}
\setbeamercolor{block body}{bg=myblue!10}
\setbeamercolor{alerted text}{fg=myorange}

% Set Fonts
\setbeamerfont{title}{size=\Large, series=\bfseries}
\setbeamerfont{frametitle}{size=\large, series=\bfseries}
\setbeamerfont{caption}{size=\small}
\setbeamerfont{footnote}{size=\tiny}

% Footer and Navigation Setup
\setbeamertemplate{footline}{
  \leavevmode%
  \hbox{%
  \begin{beamercolorbox}[wd=.3\paperwidth,ht=2.25ex,dp=1ex,center]{author in head/foot}%
    \usebeamerfont{author in head/foot}\insertshortauthor
  \end{beamercolorbox}%
  \begin{beamercolorbox}[wd=.5\paperwidth,ht=2.25ex,dp=1ex,center]{title in head/foot}%
    \usebeamerfont{title in head/foot}\insertshorttitle
  \end{beamercolorbox}%
  \begin{beamercolorbox}[wd=.2\paperwidth,ht=2.25ex,dp=1ex,center]{date in head/foot}%
    \usebeamerfont{date in head/foot}
    \insertframenumber{} / \inserttotalframenumber
  \end{beamercolorbox}}%
  \vskip0pt%
}

% Title Page Information
\title[AI Ethics]{Chapter 6: AI Ethics and Social Implications}
\author[J. Smith]{John Smith, Ph.D.}
\institute[University Name]{
  Department of Computer Science\\
  University Name\\
  \vspace{0.3cm}
  Email: email@university.edu\\
  Website: www.university.edu
}
\date{\today}

% Document Start
\begin{document}

\frame{\titlepage}

\begin{frame}[fragile]
    \frametitle{Introduction to AI Ethics}
    \begin{block}{Overview of AI Ethics}
        Artificial Intelligence (AI) technologies are rapidly reshaping our world. However, as we innovate these powerful tools, it is crucial to consider the ethical frameworks that govern their development and implementation. AI ethics examines the moral implications and responsibilities associated with AI technologies, focusing on ensuring that their use aligns with societal values and norms.
    \end{block}
\end{frame}

\begin{frame}[fragile]
    \frametitle{Significance of AI Ethics}
    \begin{enumerate}
        \item \textbf{Trust and Safety}
            \begin{itemize}
                \item Ethical AI fosters public trust in technology.
                \item Addressing bias and ensuring transparency can minimize risks and enhance user safety.
                \item \textbf{Example:} In hiring algorithms, ethical considerations are necessary to prevent discrimination based on race, gender, or background.
            \end{itemize}
        
        \item \textbf{Accountability}
            \begin{itemize}
                \item AI systems often operate autonomously; therefore, determining accountability for decisions made by AI is crucial.
                \item \textbf{Example:} In autonomous vehicles, if an accident occurs, who is responsible? The manufacturer, the software developer, or the owner?
            \end{itemize}

        \item \textbf{Privacy and Surveillance}
            \begin{itemize}
                \item AI can process vast amounts of data, raising concerns about individual privacy rights.
                \item \textbf{Example:} Facial recognition technology can enhance security but also lead to mass surveillance and invasions of privacy if unchecked.
            \end{itemize}

        \item \textbf{Social Impact}
            \begin{itemize}
                \item AI can reinforce existing inequalities or create new ones.
                \item It’s essential to consider the wider societal effects, especially in vulnerable communities.
                \item \textbf{Example:} AI-driven financial algorithms can unintentionally disadvantage individuals without credit histories, further entrenching economic disparities.
            \end{itemize}
    \end{enumerate}
\end{frame}

\begin{frame}[fragile]
    \frametitle{Key Points and Conclusion}
    \begin{itemize}
        \item AI ethics encompasses a wide range of issues, including bias, accountability, privacy, and societal impact.
        \item Ethical considerations are not merely supplementary; they are integral to the responsible design and application of AI systems.
        \item Striking a balance between innovation and ethical standards is crucial for sustainable AI development.
    \end{itemize}
    
    \begin{block}{Conclusion}
        AI ethics is pivotal to harnessing the benefits of artificial intelligence while safeguarding against potential harms. The commitment to ethical practices in AI can foster not only technological advancement but also societal well-being and justice.
    \end{block}
\end{frame}

\begin{frame}[fragile]
    \frametitle{Ethical Frameworks in AI - Introduction}
    As artificial intelligence (AI) becomes more integrated into society, it is crucial to ground its development and application in ethical principles. Different ethical frameworks provide diverse perspectives on what constitutes 'right' behavior in the context of AI. This slide explores three primary ethical theories:
    
    \begin{itemize}
        \item Deontological ethics
        \item Consequentialism
        \item Virtue ethics
    \end{itemize}
\end{frame}

\begin{frame}[fragile]
    \frametitle{Ethical Frameworks in AI - Deontological Ethics}
    \textbf{Definition:} Rooted in the philosophy of Immanuel Kant, this framework emphasizes duty and adherence to rules. It posits that some actions are inherently right or wrong regardless of their outcomes. 
    
    \textbf{Key Concept:} 
    "Act according to that maxim whereby you can, at the same time, will that it should become a universal law." (Kant)
    
    \textbf{Application in AI:} 
    An AI system implementing a facial recognition feature must respect users' privacy rights, regardless of potential benefits in security or convenience.
    
    \textbf{Example:} 
    A self-driving car must adhere to traffic laws irrespective of a scenario where breaking a rule could save lives.
\end{frame}

\begin{frame}[fragile]
    \frametitle{Ethical Frameworks in AI - Consequentialism and Virtue Ethics}
    \textbf{Consequentialism:}
    \begin{itemize}
        \item \textbf{Definition:} This framework evaluates the morality of actions based on their outcomes, aiming to maximize overall happiness or utility.
        \item \textbf{Key Concept:} "The greatest good for the greatest number."
        \item \textbf{Application in AI:} AI applications should be assessed on their effectiveness in achieving positive outcomes.
        \item \textbf{Example:} An AI focusing resources on high-crime areas can lead to a significant crime reduction.
    \end{itemize}

    \textbf{Virtue Ethics:}
    \begin{itemize}
        \item \textbf{Definition:} This framework emphasizes the character of the moral agent, focusing on virtues like honesty and courage.
        \item \textbf{Key Concept:} "Act in accordance with virtue."
        \item \textbf{Application in AI:} Developers should embody virtues that promote fairness and transparency.
        \item \textbf{Example:} An AI recommending job candidates should promote equality and inclusivity.
    \end{itemize}
\end{frame}

\begin{frame}[fragile]
    \frametitle{Ethical Frameworks in AI - Key Points and Conclusion}
    \textbf{Key Points to Emphasize:}
    \begin{itemize}
        \item \textbf{Ethics in AI Development:} Understanding these frameworks guides AI developers in making morally sound decisions.
        \item \textbf{Cross-Applicability:} Each framework has its strengths and weaknesses; a combination may offer nuanced perspectives.
        \item \textbf{Real-world Impacts:} Choices made in applying these frameworks can significantly influence policies and public trust in technology.
    \end{itemize}
    
    \textbf{Conclusion:}
    Navigating AI ethics requires thoughtful consideration of these ethical frameworks. By embedding ethical reasoning into AI design and deployment, we can ensure these technologies contribute positively to society.
\end{frame}

\begin{frame}[fragile]
    \frametitle{Bias and Fairness in AI - Overview}
    \begin{block}{Understanding Bias in AI}
        \begin{itemize}
            \item **Bias**: Systematic errors in predictions by AI systems.
            \item Results in **unfair treatment** of certain groups.
        \end{itemize}
    \end{block}
    \begin{block}{Sources of Bias}
        \begin{enumerate}
            \item Data Bias
            \item Algorithmic Bias
            \item Human Bias
        \end{enumerate}
    \end{block}
\end{frame}

\begin{frame}[fragile]
    \frametitle{Sources of Bias in AI}
    \begin{itemize}
        \item \textbf{Data Bias:}
            \begin{itemize}
                \item Arises from the training data.
                \item \textit{Example: Facial recognition trained mostly on light-skinned images.}
            \end{itemize}
        \item \textbf{Algorithmic Bias:}
            \begin{itemize}
                \item Inherent biases in model prediction.
                \item \textit{Example: Logistic regression models may perpetuate biases.}
            \end{itemize}
        \item \textbf{Human Bias:}
            \begin{itemize}
                \item Encoded biases from human decision-making.
                \item \textit{Example: Historical hiring data leading to discrimination.}
            \end{itemize}
    \end{itemize}
\end{frame}

\begin{frame}[fragile]
    \frametitle{Importance of Fairness in AI}
    \begin{block}{Why Fairness is Essential}
        \begin{itemize}
            \item AI influences societal dimensions: employment, justice, finance.
            \item \textbf{Consequences of Bias:}
                \begin{itemize}
                    \item Disenfranchisement of marginalized groups.
                    \item Erosion of public trust in AI technologies.
                \end{itemize}
        \end{itemize}
    \end{block}
    \begin{block}{Measuring Fairness}
        \begin{itemize}
            \item Equality of Opportunity
            \item Demographic Parity
        \end{itemize}
    \end{block}
\end{frame}

\begin{frame}[fragile]
    \frametitle{Addressing Bias and Promoting Fairness}
    \begin{block}{Steps to Mitigate Bias}
        \begin{enumerate}
            \item Diverse Data Collection
            \item Bias Detection Tools
            \item Transparent Algorithms
            \item Regular Audits
        \end{enumerate}
    \end{block}
    \begin{block}{Conclusion}
        By addressing biases and promoting fairness, we can develop AI systems that serve all equitably.
    \end{block}
\end{frame}

\begin{frame}[fragile]
    \frametitle{Example Code for Fairness Assessment}
    \begin{lstlisting}[language=Python]
from sklearn.metrics import accuracy_score
from fairlearn.metrics import MetricFrame

# Sample predictions
y_true = [1, 0, 1, 1, 0]
y_pred = [1, 0, 0, 1, 0]

# Calculate accuracy
accuracy = accuracy_score(y_true, y_pred)

# Use MetricFrame for fairness assessment
metric_frame = MetricFrame(metrics={"accuracy": accuracy_score},
                           y_true=y_true,
                           sensitive_features=[1, 0, 1, 1, 0])
print(metric_frame.by_group)
    \end{lstlisting}
\end{frame}

\begin{frame}[fragile]
    \frametitle{Privacy Concerns - Overview}
    \begin{block}{Understanding Data Privacy in AI}
        As artificial intelligence (AI) systems become increasingly integrated into daily life, issues surrounding data privacy necessitate careful examination. 
        We will focus on three main elements: Data Collection, Data Storage, and User Consent.
    \end{block}
\end{frame}

\begin{frame}[fragile]
    \frametitle{Privacy Concerns - Data Collection}
    \begin{enumerate}
        \item \textbf{Data Collection}
        \begin{itemize}
            \item \textbf{Definition}: Process of gathering information from various sources (e.g., interactions, social media, records).
            \item \textbf{Example}: Healthcare apps collect user data for personalized health advice.
            \item \textbf{Concerns}: Unauthorized access, data breaches, and misuse of data.
        \end{itemize}
    \end{enumerate}
\end{frame}

\begin{frame}[fragile]
    \frametitle{Privacy Concerns - Data Storage and User Consent}
    \begin{enumerate}
        \setcounter{enumi}{1}
        \item \textbf{Data Storage}
        \begin{itemize}
            \item \textbf{Definition}: Methods used to save collected data; can be local or cloud-based.
            \item \textbf{Example}: Cloud AI services storing transactional data for machine learning.
            \item \textbf{Concerns}: Vulnerability to cyberattacks if stored improperly; importance of encryption and access controls.
        \end{itemize}

        \item \textbf{User Consent}
        \begin{itemize}
            \item \textbf{Definition}: Process by which individuals agree to the collection and use of their data.
            \item \textbf{Example}: Users must accept 'Terms and Conditions' for data practices when registering on online services.
            \item \textbf{Concerns}: Complexity of consent forms leads to misunderstandings and unintended permissions.
        \end{itemize}
    \end{enumerate}
\end{frame}

\begin{frame}[fragile]
    \frametitle{Key Points and Framework}
    \begin{block}{Key Points}
        \begin{itemize}
            \item \textbf{Transparency}: Organizations must be clear about data collection and protection measures.
            \item \textbf{User Control}: Users should manage their own data and have options for data collection.
            \item \textbf{Legal Framework}: Adherence to regulations like GDPR for data protection.
        \end{itemize}
    \end{block}

    \begin{block}{Privacy Risk Framework}
        Privacy Risk = Data Sensitivity x Exposure x Control
        \begin{itemize}
            \item \textbf{Data Sensitivity}: Classification of data; sensitive data requires higher protection.
            \item \textbf{Exposure}: Likelihood of unauthorized access to data.
            \item \textbf{Control}: Protective measures in place, such as encryption.
        \end{itemize}
    \end{block}
\end{frame}

\begin{frame}[fragile]
    \frametitle{Conclusion}
    Understanding privacy concerns in AI is vital for creating ethical and responsible applications. 
    By prioritizing transparency, user control, and adhering to legal standards, organizations can foster trust and secure user data.
\end{frame}

\begin{frame}[fragile]
    \frametitle{Accountability and Transparency - Part 1}
    \begin{block}{Accountability in AI Decision-Making}
        \textbf{Definition:} Accountability refers to the obligation of individuals, organizations, and systems to bear responsibility for the outcomes of their actions. In AI, this means ensuring creators and users are answerable for AI-driven decisions.
    \end{block}

    \begin{block}{Importance}
        \begin{itemize}
            \item \textbf{Ethical Responsibility:} Encourages consideration of societal impact by AI developers.
            \item \textbf{Trust Building:} Enhances public trust via ethical design and management of AI systems.
            \item \textbf{Legal Implications:} Addresses liability concerns when AI systems harm individuals.
        \end{itemize}
    \end{block}
\end{frame}

\begin{frame}[fragile]
    \frametitle{Accountability and Transparency - Part 2}
    \begin{block}{Example}
        In autonomous vehicles, accountability involves not only the technology but the companies and engineers behind it. Liability determination in accidents is critical—whether it falls on the manufacturer, software developer, or vehicle owner.
    \end{block}
    
    \begin{block}{Transparency in AI Systems}
        \textbf{Definition:} Transparency involves the clarity and openness of AI system processes, making them understandable and accessible to users and stakeholders.
    \end{block}
\end{frame}

\begin{frame}[fragile]
    \frametitle{Accountability and Transparency - Part 3}
    \begin{block}{Importance of Transparency}
        \begin{itemize}
            \item \textbf{Informed Decision-Making:} Users can make better choices by understanding AI conclusions.
            \item \textbf{Bias Detection:} Aids in identifying and rectifying biases in AI algorithms, fostering fairness.
            \item \textbf{Regulatory Compliance:} Many regulations mandate transparency in automated decisions.
        \end{itemize}
    \end{block}

    \begin{block}{Example}
        A transparent hiring algorithm would clarify why a resume was favored, allowing applicants to understand and challenge automated decisions.
    \end{block}
\end{frame}

\begin{frame}[fragile]
    \frametitle{Impact on Employment - Overview}
    \begin{block}{Slide Description}
        Assessment of how AI technologies can disrupt job markets and change the nature of work.
    \end{block}
\end{frame}

\begin{frame}[fragile]
    \frametitle{Understanding the Impact of AI on Employment}
    \begin{itemize}
        \item \textbf{Automation of Tasks:} Displacing jobs through automation of routine, manual labor.
        \item \textbf{Job Transformation:} Evolution of roles requiring new skills and adjustments to technology.
        \item \textbf{Creation of New Opportunities:} Emerging jobs in sectors like AI development, data analysis, and robotics.
    \end{itemize}
\end{frame}

\begin{frame}[fragile]
    \frametitle{Examples of AI's Influence on Employment}
    \begin{itemize}
        \item \textbf{Manufacturing:} Use of robots automates assembly line tasks, reducing manual labor positions.
        \item \textbf{Healthcare:} AI assists in diagnostics and patient management, affecting healthcare roles.
        \item \textbf{Transportation:} Self-driving technology reduces the need for drivers but increases demand for AI technicians.
    \end{itemize}
\end{frame}

\begin{frame}[fragile]
    \frametitle{Key Points to Emphasize}
    \begin{itemize}
        \item \textbf{Displacement vs. Creation:} AI displaces some jobs while creating new opportunities.
        \item \textbf{Need for Reskilling and Upskilling:} Workers must acquire new skills; emphasis on educational initiatives.
        \item \textbf{Inequality in Job Disruption:} Low-wage jobs more susceptible to automation, increasing income inequality.
    \end{itemize}
\end{frame}

\begin{frame}[fragile]
    \frametitle{Formulas Relevant to AI Impact on Employment}
    To model the impact of AI on employment dynamics, use the formula:
    \begin{equation}
        \text{Job Displacement Rate (JDR)} = \left( \frac{\text{Automated Jobs}}{\text{Total Jobs}} \right) \times 100
    \end{equation}
\end{frame}

\begin{frame}[fragile]
    \frametitle{Conclusion}
    \begin{block}{Takeaway}
        Understanding AI's dual impact on employment is crucial for workforce preparation. Integrative strategies focusing on education, skill development, and supportive policies will ensure a smooth transition into an AI-enhanced job market.
    \end{block}
\end{frame}

\begin{frame}[fragile]
    \frametitle{AI Governance and Regulation}
    \begin{block}{Overview}
        Overview of legislative and regulatory efforts aimed at governing AI technologies and protecting societal interests.
    \end{block}
\end{frame}

\begin{frame}[fragile]
    \frametitle{Concept of AI Governance}
    \begin{itemize}
        \item AI Governance refers to frameworks, policies, and practices guiding the development, deployment, and use of AI.
        \item Effective governance ensures:
        \begin{itemize}
            \item Accountability
            \item Transparency
            \item Ethical considerations
        \end{itemize}
    \end{itemize}
\end{frame}

\begin{frame}[fragile]
    \frametitle{Importance of AI Regulation}
    \begin{itemize}
        \item As AI technologies proliferate, unique societal concerns arise:
        \begin{itemize}
            \item \textbf{Bias and Discrimination:} Ensuring AI operates fairly across all groups.
            \item \textbf{Privacy Violations:} Protecting personal information from misuse.
            \item \textbf{Safety Risks:} Minimizing risks in critical sectors (e.g., healthcare, transportation).
        \end{itemize}
    \end{itemize}
\end{frame}

\begin{frame}[fragile]
    \frametitle{Key Legislation and Regulatory Efforts}
    \begin{enumerate}
        \item \textbf{The EU AI Act}
        \begin{itemize}
            \item Comprehensive framework proposed in 2021.
            \item \textbf{Risk-Based Approach:} Classifies AI systems into categories based on risk.
            \item \textbf{Mandatory Compliance:} High-risk systems must meet strict requirements.
        \end{itemize}
        
        \item \textbf{Algorithmic Accountability Act (USA)}
        \begin{itemize}
            \item Requires companies to assess algorithms for bias and fairness.
            \item \textbf{Impact Assessments:} Mandatory studies on biases.
            \item \textbf{Disclosure:} Public information on data usage.
        \end{itemize}
        
        \item \textbf{General Data Protection Regulation (GDPR)}
        \begin{itemize}
            \item Mandates data protection standards impacting AI applications.
            \item Key Components include user consent and the right to explanation.
        \end{itemize}
    \end{enumerate}
\end{frame}

\begin{frame}[fragile]
    \frametitle{Global Perspectives on AI Regulation}
    \begin{itemize}
        \item \textbf{China:} Advancing regulations emphasizing data governance and national security.
        \item \textbf{Singapore:} Implementing a Model AI Governance Framework for responsible AI deployment.
    \end{itemize}
\end{frame}

\begin{frame}[fragile]
    \frametitle{Challenges in AI Governance}
    \begin{itemize}
        \item \textbf{Evolving Technology:} Rapid AI advancements may outpace legislation.
        \item \textbf{International Collaboration:} Regulatory differences complicate compliance for global companies.
    \end{itemize}
\end{frame}

\begin{frame}[fragile]
    \frametitle{Conclusion}
    \begin{itemize}
        \item AI governance and regulation are essential for ensuring technological advancements benefit society while mitigating risks.
        \item Continuous adaptation and collaboration among stakeholders are crucial in the evolving landscape of AI regulation.
    \end{itemize}
\end{frame}

\begin{frame}[fragile]
  \frametitle{Case Studies in AI Ethics}
  \begin{block}{Overview of Ethical Dilemmas in AI}
    AI technologies introduce complex ethical dilemmas due to their capabilities, challenging societal norms, values, and legal frameworks. This section examines notable real-world cases that illustrate these challenges and the responses to them.
  \end{block}
\end{frame}

\begin{frame}[fragile]
  \frametitle{Case Study 1: COMPAS Recidivism Algorithm}
  \begin{itemize}
    \item \textbf{Background:} The Correctional Offender Management Profiling for Alternative Sanctions (COMPAS) tool predicts the likelihood of criminal reoffending, used in judicial contexts across the U.S.
    \item \textbf{Ethical Dilemma:} ProPublica's investigation revealed racial bias, showing Black individuals flagged as higher risk compared to white individuals, even with similar backgrounds.
    \item \textbf{Addressing the Dilemma:}
      \begin{itemize}
        \item Increased algorithmic transparency.
        \item Legislative pushes for disclosure of algorithmic decision-making processes.
        \item Adoption of practices for third-party bias evaluations.
      \end{itemize}
  \end{itemize}
\end{frame}

\begin{frame}[fragile]
  \frametitle{Case Study 2: Facial Recognition Technology in Law Enforcement}
  \begin{itemize}
    \item \textbf{Background:} Facial recognition systems are increasingly used by law enforcement for suspect identification.
    \item \textbf{Ethical Dilemma:} Privacy concerns and wrongful arrests, particularly affecting marginalized communities due to misidentifications.
    \item \textbf{Addressing the Dilemma:}
      \begin{itemize}
        \item City-level bans on facial recognition amid public protests.
        \item Development of ethical guidelines in collaboration with community stakeholders to ensure fair usage.
      \end{itemize}
  \end{itemize}
\end{frame}

\begin{frame}[fragile]
  \frametitle{Case Study 3: Autonomous Vehicles (AV) and Moral Machine Dilemmas}
  \begin{itemize}
    \item \textbf{Background:} Autonomous vehicles must make rapid decisions in accident scenarios, leading to moral programming choices.
    \item \textbf{Ethical Dilemma:} Should a car prioritize the safety of its passengers or pedestrians?
    \item \textbf{Addressing the Dilemma:}
      \begin{itemize}
        \item Public discourse and ethical frameworks including the "Moral Machine" project.
        \item Collaboration between automotive manufacturers and ethicists to create ethical decision algorithms.
      \end{itemize}
  \end{itemize}
\end{frame}

\begin{frame}[fragile]
  \frametitle{Key Points to Emphasize}
  \begin{enumerate}
    \item \textbf{Transparency and Accountability:} Growing demand for algorithmic transparency to prevent bias.
    \item \textbf{Public Engagement:} The vital role of community stakeholder involvement in ethical AI frameworks.
    \item \textbf{Legislation and Governance:} The importance of regulatory measures in guiding ethical AI usage and minimizing harm.
  \end{enumerate}
\end{frame}

\begin{frame}[fragile]
  \frametitle{Conclusion}
  \begin{block}{Summary}
    These case studies highlight the ongoing need for evaluation of AI technologies in real-world contexts. Establishing ethical guidelines is essential for ensuring equitable outcomes as AI technology continues to evolve.
  \end{block}
\end{frame}

\begin{frame}[fragile]
  \frametitle{References}
  \begin{itemize}
    \item ProPublica's report on COMPAS.
    \item Reports on facial recognition bans and community responses.
    \item The Moral Machine project on autonomous vehicle ethics.
  \end{itemize}
\end{frame}

\begin{frame}[fragile]
    \frametitle{Future Directions in AI Ethics}
    \begin{block}{Evolving Landscape of AI Ethics}
        As AI integrates into our lives, the ethics field is dynamically evolving, facing key trends and challenges.
    \end{block}
\end{frame}

\begin{frame}[fragile]
    \frametitle{Key Concepts - Part 1}
    \begin{enumerate}
        \item \textbf{Proliferation of AI Technologies}:
            \begin{itemize}
                \item Rapid growth of AI technologies (e.g., machine learning, neural networks) presents unique ethical challenges.
                \item Example: AI systems in recruitment can perpetuate biases.
            \end{itemize}
        
        \item \textbf{Data Privacy and Surveillance}:
            \begin{itemize}
                \item Increased data collection raises questions of user consent and surveillance.
                \item Example: GDPR in Europe emphasizes data privacy.
            \end{itemize}
    \end{enumerate}
\end{frame}

\begin{frame}[fragile]
    \frametitle{Key Concepts - Part 2}
    \begin{enumerate}
        \setcounter{enumi}{2} % continue the enumeration
        \item \textbf{Autonomous Systems and Decision-Making}:
            \begin{itemize}
                \item As AI makes decisions autonomously, moral responsibility becomes a key concern.
                \item Example: A self-driving car's decision between passenger and pedestrian safety raises accountability issues.
            \end{itemize}

        \item \textbf{Accountability and Transparency}:
            \begin{itemize}
                \item The "black box" nature of AI complicates accountability and calls for explainable decisions.
                \item Example: Regulatory efforts for algorithmic transparency aim to enable external audits.
            \end{itemize}
    \end{enumerate}
\end{frame}

\begin{frame}[fragile]
    \frametitle{Anticipated Challenges and Conclusion}
    \begin{itemize}
        \item \textbf{Ethical Framework Development}: Developing frameworks for ethical AI use is crucial.
        \item \textbf{Global Regulation Divergence}: Variations in regulations can lead to compliance challenges.
        \item \textbf{Inclusive AI}: Diverse representation in AI development is essential to reduce bias.
        \item \textbf{Socioeconomic Impacts}: The implications of AI on jobs necessitate reevaluating labor ethics.
    \end{itemize}

    \begin{block}{Conclusion}
        Addressing AI ethics involves recognizing the interplay between innovation, societal values, and ethical standards as we move towards an AI-driven world.
    \end{block}
\end{frame}

\begin{frame}[fragile]
  \frametitle{Conclusion and Reflection - Key Takeaways}
  \begin{enumerate}
    \item \textbf{Understanding AI Ethics:}
      \begin{itemize}
        \item AI ethics involves moral principles that guide the development and application of AI, addressing fairness, accountability, transparency, and societal impacts.
        \item Ethical implications are increasingly complex as AI evolves, with dilemmas surfacing in various fields such as healthcare and finance.
      \end{itemize}

    \item \textbf{Stakeholder Responsibility:}
      \begin{itemize}
        \item \textbf{AI Practitioners:} Must actively address ethical issues to prevent bias and harm.
        \item \textbf{Organizations:} Need to establish ethical guidelines that reflect societal values.
        \item \textbf{Regulators and Policymakers:} Should create frameworks to promote ethical AI practices while protecting public interests.
      \end{itemize}
  \end{enumerate}
\end{frame}

\begin{frame}[fragile]
  \frametitle{Conclusion and Reflection - Urgency and Examples}
  \begin{enumerate}
    \setcounter{enumi}{3} % Start numbering from 4
    \item \textbf{The Urgency of Ethical AI:}
      \begin{itemize}
        \item Rapid AI advancements highlight a critical need for ethical oversight and ongoing education for practitioners on ethical considerations.
      \end{itemize}

    \item \textbf{Real-world Examples:}
      \begin{itemize}
        \item \textbf{Facial Recognition Technology:} Biased outcomes affecting marginalized groups.
        \item \textbf{AI in Hiring:} Automated systems may perpetuate historical biases, necessitating critical evaluation of algorithms.
      \end{itemize}
  \end{enumerate}
\end{frame}

\begin{frame}[fragile]
  \frametitle{Conclusion and Reflection - Key Principles and Reflection}
  \begin{block}{Key Principles to Remember}
    \begin{itemize}
      \item Ethics should be integral to the entire AI development lifecycle.
      \item Essential components of ethical AI include transparency, accountability, and inclusivity.
    \end{itemize}
  \end{block}
  
  \textbf{Reflection:}
  \begin{itemize}
    \item Responsibilities of AI practitioners extend beyond technical skills to include ethical stewardship.
    \item Each stakeholder must commit to ethical standards that foster trust and justice in the digital landscape.
  \end{itemize}
\end{frame}


\end{document}