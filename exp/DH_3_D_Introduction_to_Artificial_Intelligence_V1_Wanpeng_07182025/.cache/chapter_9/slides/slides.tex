\documentclass[aspectratio=169]{beamer}

% Theme and Color Setup
\usetheme{Madrid}
\usecolortheme{whale}
\useinnertheme{rectangles}
\useoutertheme{miniframes}

% Additional Packages
\usepackage[utf8]{inputenc}
\usepackage[T1]{fontenc}
\usepackage{graphicx}
\usepackage{booktabs}
\usepackage{listings}
\usepackage{amsmath}
\usepackage{amssymb}
\usepackage{xcolor}
\usepackage{tikz}
\usepackage{pgfplots}
\pgfplotsset{compat=1.18}
\usetikzlibrary{positioning}
\usepackage{hyperref}

% Custom Colors
\definecolor{myblue}{RGB}{31, 73, 125}
\definecolor{mygray}{RGB}{100, 100, 100}
\definecolor{mygreen}{RGB}{0, 128, 0}
\definecolor{myorange}{RGB}{230, 126, 34}
\definecolor{mycodebackground}{RGB}{245, 245, 245}

% Set Theme Colors
\setbeamercolor{structure}{fg=myblue}
\setbeamercolor{frametitle}{fg=white, bg=myblue}
\setbeamercolor{title}{fg=myblue}
\setbeamercolor{section in toc}{fg=myblue}
\setbeamercolor{item projected}{fg=white, bg=myblue}
\setbeamercolor{block title}{bg=myblue!20, fg=myblue}
\setbeamercolor{block body}{bg=myblue!10}
\setbeamercolor{alerted text}{fg=myorange}

% Set Fonts
\setbeamerfont{title}{size=\Large, series=\bfseries}
\setbeamerfont{frametitle}{size=\large, series=\bfseries}
\setbeamerfont{caption}{size=\small}
\setbeamerfont{footnote}{size=\tiny}

% Code Listing Style
\lstdefinestyle{customcode}{
  backgroundcolor=\color{mycodebackground},
  basicstyle=\footnotesize\ttfamily,
  breakatwhitespace=false,
  breaklines=true,
  commentstyle=\color{mygreen}\itshape,
  keywordstyle=\color{blue}\bfseries,
  stringstyle=\color{myorange},
  numbers=left,
  numbersep=8pt,
  numberstyle=\tiny\color{mygray},
  frame=single,
  framesep=5pt,
  rulecolor=\color{mygray},
  showspaces=false,
  showstringspaces=false,
  showtabs=false,
  tabsize=2,
  captionpos=b
}
\lstset{style=customcode}

% Custom Commands
\newcommand{\hilight}[1]{\colorbox{myorange!30}{#1}}
\newcommand{\source}[1]{\vspace{0.2cm}\hfill{\tiny\textcolor{mygray}{Source: #1}}}
\newcommand{\concept}[1]{\textcolor{myblue}{\textbf{#1}}}
\newcommand{\separator}{\begin{center}\rule{0.5\linewidth}{0.5pt}\end{center}}

% Footer and Navigation Setup
\setbeamertemplate{footline}{
  \leavevmode%
  \hbox{%
  \begin{beamercolorbox}[wd=.3\paperwidth,ht=2.25ex,dp=1ex,center]{author in head/foot}%
    \usebeamerfont{author in head/foot}\insertshortauthor
  \end{beamercolorbox}%
  \begin{beamercolorbox}[wd=.5\paperwidth,ht=2.25ex,dp=1ex,center]{title in head/foot}%
    \usebeamerfont{title in head/foot}\insertshorttitle
  \end{beamercolorbox}%
  \begin{beamercolorbox}[wd=.2\paperwidth,ht=2.25ex,dp=1ex,center]{date in head/foot}%
    \usebeamerfont{date in head/foot}
    \insertframenumber{} / \inserttotalframenumber
  \end{beamercolorbox}}%
  \vskip0pt%
}

% Turn off navigation symbols
\setbeamertemplate{navigation symbols}{}

% Title Page Information
\title[Advanced AI Techniques]{Chapter 9: Advanced AI Techniques}
\subtitle{An In-depth Exploration}
\author[J. Smith]{John Smith, Ph.D.}
\institute[University Name]{
  Department of Computer Science\\
  University Name\\
  \vspace{0.3cm}
  Email: email@university.edu\\
  Website: www.university.edu
}
\date{\today}

% Document Start
\begin{document}

\frame{\titlepage}

\begin{frame}[fragile]
    \frametitle{Introduction to Chapter 9: Advanced AI Techniques}
    In this chapter, we delve into \textbf{Advanced AI Techniques} that enable machines to learn and make intelligent decisions. 
    As AI technology continues to evolve, understanding these advanced methodologies becomes critical for students and professionals alike.
\end{frame}

\begin{frame}[fragile]
    \frametitle{Key Concepts Overview}
    \begin{enumerate}
        \item \textbf{Deep Learning}
            \begin{itemize}
                \item Uses layered architectures called neural networks.
                \item Capable of learning from vast amounts of data.
                \item \textit{Example:} Convolutional Neural Networks (CNNs) excel in analyzing visual data.
            \end{itemize}
        
        \item \textbf{Reinforcement Learning (RL)}
            \begin{itemize}
                \item An agent learns through rewards or penalties.
                \item Mimics human trial and error learning.
                \item \textit{Example:} AlphaGo's superhuman level in the game of Go.
            \end{itemize}
        
        \item \textbf{Natural Language Processing (NLP)}
            \begin{itemize}
                \item Techniques for understanding and generating human language.
                \item \textit{Example:} ChatGPT generates human-like text and conversations.
            \end{itemize}
        
        \item \textbf{Generative Adversarial Networks (GANs)}
            \begin{itemize}
                \item Consists of two neural networks contesting with each other to generate new data.
                \item \textit{Example:} Creates realistic images of non-existent people.
            \end{itemize}
    \end{enumerate}
\end{frame}

\begin{frame}[fragile]
    \frametitle{Important Techniques and Algorithms}
    \begin{block}{Backpropagation in Neural Networks}
        The key algorithm for training deep learning models, using gradient descent to optimize weights:
        \begin{equation}
            w^{(new)} = w^{(old)} - \eta \frac{\partial L}{\partial w}
        \end{equation}
        where \( w \) represents the weights, \( \eta \) is the learning rate, and \( L \) is the loss function.
    \end{block}

    \begin{block}{Q-Learning in Reinforcement Learning}
        A value-based approach that updates its action-value function based on expected future rewards:
        \begin{equation}
            Q(s, a) = Q(s, a) + \alpha \left( r + \gamma \max_{a'} Q(s', a') - Q(s, a) \right)
        \end{equation}
    \end{block}
\end{frame}

\begin{frame}[fragile]
    \frametitle{Overview of Key Concepts}
    In this section, we will explore various advanced techniques used in Artificial Intelligence (AI) that enhance decision-making, improve learning efficiency, and expand the capabilities of AI systems.
\end{frame}

\begin{frame}[fragile]
    \frametitle{Key Concepts in Advanced AI Techniques}
    \begin{enumerate}
        \item \textbf{Deep Learning}
        \item \textbf{Reinforcement Learning (RL)}
        \item \textbf{Natural Language Processing (NLP)}
        \item \textbf{Generative Adversarial Networks (GANs)}
        \item \textbf{Transfer Learning}
    \end{enumerate}
\end{frame}

\begin{frame}[fragile]
    \frametitle{Deep Learning}
    \begin{itemize}
        \item \textbf{Explanation}: A subset of machine learning that uses neural networks with many layers to model complex patterns in large datasets.
        \item \textbf{Example}: Image recognition via distinguishing between cats and dogs.
        \item \textbf{Key Point}: Excels in feature extraction, reducing need for manual feature engineering.
    \end{itemize}
\end{frame}

\begin{frame}[fragile]
    \frametitle{Reinforcement Learning}
    \begin{itemize}
        \item \textbf{Explanation}: An agent learns to make decisions through trial and error by maximizing cumulative rewards in an environment.
        \item \textbf{Example}: Training a robot to navigate a maze using feedback (rewards).
        \item \textbf{Key Point}: Balances exploration (trying new actions) and exploitation (using known actions).
        \item \textbf{Basic Formula}:
        \begin{equation}
            R_t = r(s_t, a_t) + \gamma V(s_{t+1}),
        \end{equation}
        where \( R_t \) is the reward at time \( t \), \( r \) is the immediate reward function, \( \gamma \) is the discount factor, and \( V \) is the value function.
    \end{itemize}
\end{frame}

\begin{frame}[fragile]
    \frametitle{Natural Language Processing}
    \begin{itemize}
        \item \textbf{Explanation}: Enables computers to understand, interpret, and generate human language.
        \item \textbf{Example}: Sentiment analysis of social media posts (positive, negative, neutral).
        \item \textbf{Key Point}: Involves techniques like tokenization, stemming, and advanced models such as BERT.
    \end{itemize}
\end{frame}

\begin{frame}[fragile]
    \frametitle{Generative Adversarial Networks}
    \begin{itemize}
        \item \textbf{Explanation}: A framework where two neural networks (generator and discriminator) compete, creating synthetic instances of data.
        \item \textbf{Example}: Generating realistic images of non-existent faces.
        \item \textbf{Key Point}: Widely used for image generation, videos, and art synthesis.
    \end{itemize}
\end{frame}

\begin{frame}[fragile]
    \frametitle{Transfer Learning}
    \begin{itemize}
        \item \textbf{Explanation}: A model developed for a specific task is reused as a starting point for a second task.
        \item \textbf{Example}: Fine-tuning a pre-trained model on ImageNet for medical imaging tasks.
        \item \textbf{Key Point}: Reduces training time and computational resources needed.
    \end{itemize}
\end{frame}

\begin{frame}[fragile]
    \frametitle{Conclusion}
    As we delve further into these advanced AI techniques, we will analyze their algorithms, underlying mathematics, and real-world applications, preparing you to harness the potential of AI in innovative ways.
\end{frame}

\begin{frame}[fragile]
    \frametitle{Prepare for In-Depth Analysis}
    In the upcoming slides, we will break down each of these techniques further, examining their mathematical foundations, practical implementations, and evolution. This will equip you with the necessary skills to tackle real-world problems using advanced AI methods.
\end{frame}

\begin{frame}[fragile]
    \frametitle{Conclusion - Summary of Advanced AI Techniques}
    In this chapter, we explored several advanced techniques that enhance artificial intelligence systems. 
    These methods play crucial roles in making AI more efficient, accurate, and capable of tackling complex problems.
\end{frame}

\begin{frame}[fragile]
    \frametitle{Key Concepts - Part 1}
    \begin{itemize}
        \item \textbf{Deep Learning}: 
        \begin{itemize}
            \item A subset of machine learning using neural networks with multiple layers.
            \item \textbf{Example}: Image recognition tasks with convolutional neural networks.
        \end{itemize}
        
        \item \textbf{Reinforcement Learning}:
        \begin{itemize}
            \item Learning paradigm where agents learn from rewards or penalties.
            \item \textbf{Example}: AI playing video games based on scoring rewards.
        \end{itemize}
    \end{itemize}
\end{frame}

\begin{frame}[fragile]
    \frametitle{Key Concepts - Part 2}
    \begin{itemize}
        \item \textbf{Natural Language Processing (NLP)}:
        \begin{itemize}
            \item Techniques for machines to understand and generate human language.
            \item \textbf{Example}: Chatbots providing customer service.
        \end{itemize}
        
        \item \textbf{Generative Adversarial Networks (GANs)}:
        \begin{itemize}
            \item Competing neural networks (generator vs. discriminator).
            \item \textbf{Example}: Creating realistic images or artworks.
        \end{itemize}

        \item \textbf{Transfer Learning}:
        \begin{itemize}
            \item Reusing models trained on one task to expedite training on another.
            \item \textbf{Example}: Fine-tuning a language model on specialized datasets.
        \end{itemize}
    \end{itemize}
\end{frame}

\begin{frame}[fragile]
    \frametitle{Key Takeaways}
    \begin{itemize}
        \item \textbf{Interconnectedness}: Advanced AI techniques often intersect, e.g., deep learning in reinforcement learning.
        \item \textbf{Applications Across Industries}: Techniques used in healthcare (diagnostics) and finance (fraud detection).
        \item \textbf{Importance of Data}: Quality data significantly enhances model performance.
    \end{itemize}
\end{frame}

\begin{frame}[fragile]
    \frametitle{Illustrative Formula}
    \begin{block}{Loss Function in Deep Learning}
        \begin{equation}
            L = \frac{1}{N} \sum_{i=1}^{N} (y_i - \hat{y}_i)^2
        \end{equation}
        Where \(N\) is the number of samples, \(y_i\) is the true value, and \(\hat{y}_i\) is the predicted value.
    \end{block}
\end{frame}

\begin{frame}[fragile]
    \frametitle{Conclusion - Final Thoughts}
    The chapter emphasized that mastering these advanced AI techniques equips future AI practitioners with the tools needed to innovate and solve real-world problems effectively. Continuous learning and adaptation in this fast-evolving field are essential for leveraging AI's full potential.
\end{frame}


\end{document}