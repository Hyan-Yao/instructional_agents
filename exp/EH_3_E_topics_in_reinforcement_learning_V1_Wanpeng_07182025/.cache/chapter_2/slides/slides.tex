\documentclass[aspectratio=169]{beamer}

% Theme and Color Setup
\usetheme{Madrid}
\usecolortheme{whale}
\useinnertheme{rectangles}
\useoutertheme{miniframes}

% Additional Packages
\usepackage[utf8]{inputenc}
\usepackage[T1]{fontenc}
\usepackage{graphicx}
\usepackage{booktabs}
\usepackage{listings}
\usepackage{amsmath}
\usepackage{amssymb}
\usepackage{xcolor}
\usepackage{tikz}
\usepackage{pgfplots}
\pgfplotsset{compat=1.18}
\usetikzlibrary{positioning}
\usepackage{hyperref}

% Custom Colors
\definecolor{myblue}{RGB}{31, 73, 125}
\definecolor{mygray}{RGB}{100, 100, 100}
\definecolor{mygreen}{RGB}{0, 128, 0}
\definecolor{myorange}{RGB}{230, 126, 34}
\definecolor{mycodebackground}{RGB}{245, 245, 245}

% Set Theme Colors
\setbeamercolor{structure}{fg=myblue}
\setbeamercolor{frametitle}{fg=white, bg=myblue}
\setbeamercolor{title}{fg=myblue}
\setbeamercolor{section in toc}{fg=myblue}
\setbeamercolor{item projected}{fg=white, bg=myblue}
\setbeamercolor{block title}{bg=myblue!20, fg=myblue}
\setbeamercolor{block body}{bg=myblue!10}
\setbeamercolor{alerted text}{fg=myorange}

% Set Fonts
\setbeamerfont{title}{size=\Large, series=\bfseries}
\setbeamerfont{frametitle}{size=\large, series=\bfseries}
\setbeamerfont{caption}{size=\small}
\setbeamerfont{footnote}{size=\tiny}

% Code Listing Style
\lstdefinestyle{customcode}{
  backgroundcolor=\color{mycodebackground},
  basicstyle=\footnotesize\ttfamily,
  breakatwhitespace=false,
  breaklines=true,
  commentstyle=\color{mygreen}\itshape,
  keywordstyle=\color{blue}\bfseries,
  stringstyle=\color{myorange},
  numbers=left,
  numbersep=8pt,
  numberstyle=\tiny\color{mygray},
  frame=single,
  framesep=5pt,
  rulecolor=\color{mygray},
  showspaces=false,
  showstringspaces=false,
  showtabs=false,
  tabsize=2,
  captionpos=b
}
\lstset{style=customcode}

% Document Start
\begin{document}

\frame{\titlepage}

\begin{frame}[fragile]
    \frametitle{Introduction to Week 2 - Markov Decision Processes (MDPs)}
    \begin{block}{Overview}
        In this week, we will delve into the foundational concepts of Markov Decision Processes (MDPs), a crucial framework in decision-making and reinforcement learning. MDPs assist in modeling complex problems where outcomes are partly random and partly under the control of a decision-maker.
    \end{block}
\end{frame}

\begin{frame}[fragile]
    \frametitle{Key Concepts of MDPs}
    \begin{enumerate}
        \item \textbf{Definition of MDP:}
        \begin{itemize}
            \item A tuple $(S, A, P, R, \gamma)$ where:
            \begin{itemize}
                \item $S$: Finite set of states.
                \item $A$: Finite set of actions available at each state.
                \item $P$: Transition probability function.
                \item $R$: Reward function for actions taken in states.
                \item $\gamma$: Discount factor (0 to 1) for future rewards.
            \end{itemize}
        \end{itemize}
    \end{enumerate}
\end{frame}

\begin{frame}[fragile]
    \frametitle{Examples and Importance of MDPs}
    \begin{block}{Importance of MDPs}
        MDPs provide a formal framework for modeling decision-making in uncertain outcomes and are essential for developing algorithms in reinforcement learning.
    \end{block}

    \begin{block}{Example Scenario}
        Consider a robot navigating through a grid world:
        \begin{itemize}
            \item \textbf{States (S)}: Each cell in the grid represents a state.
            \item \textbf{Actions (A)}: The robot can move Up, Down, Left, or Right.
            \item \textbf{Transition (P)}: Probability of slipping to a different cell.
            \item \textbf{Rewards (R)}: Positive for reaching a goal, negative for hitting obstacles, and a small penalty for moves.
            \item \textbf{Discount Factor ($\gamma$)}: Values immediate rewards versus future rewards.
        \end{itemize}
    \end{block}
\end{frame}

\begin{frame}[fragile]
    \frametitle{Understanding MDPs Through Formulas}
    The \textbf{Bellman Equation} is central to solving MDPs:
    \begin{equation}
        V(s) = \max_{a \in A} \left( R(s, a) + \gamma \sum_{s' \in S} P(s'|s, a) V(s') \right)
    \end{equation}
    \begin{itemize}
        \item $V(s)$: Value function representing the maximum expected reward from state $s$.
        \item The equation indicates that the value of a state is the maximum expected reward obtainable by choosing an action $a$ and subsequently following the optimal policy.
    \end{itemize}
\end{frame}

\begin{frame}[fragile]
    \frametitle{Key Points to Emphasize about MDPs}
    \begin{itemize}
        \item MDPs are integral to robotics, economics, and artificial intelligence.
        \item Understanding MDP structure aids in developing efficient algorithms for decision-making problems.
        \item Real-world applications include autonomous navigation, resource management, and game playing.
    \end{itemize}
    In the next slide, we will provide an overview of the key concepts related to MDPs, including their characteristics, value iteration, and policy iteration methods.
\end{frame}

\begin{frame}[fragile]{Overview of Markov Decision Processes (MDPs) - Part 1}
    \frametitle{What is a Markov Decision Process (MDP)?}
    \begin{itemize}
        \item \textbf{Definition}: An MDP is a mathematical framework for modeling decision-making situations where outcomes are partly random and partly under the control of a decision-maker.
        \item \textbf{Key Components}:
        \begin{itemize}
            \item \textbf{States (S)}: Set of all possible states. For example, in a game, this could represent the current position of a player.
            \item \textbf{Actions (A)}: Set of all possible actions the agent can take, like moving left, right, or jumping.
            \item \textbf{Transition Function (P)}: Probabilities of moving from one state to another given a particular action, denoted as \(P(s'|s, a)\).
            \item \textbf{Reward Function (R)}: Assigns numerical values (rewards) to state-action pairs, symbolized as \(R(s, a)\).
            \item \textbf{Discount Factor (\(\gamma\))}: A value between 0 and 1 that determines the importance of future rewards.
        \end{itemize}
    \end{itemize}
\end{frame}

\begin{frame}[fragile]{Overview of Markov Decision Processes (MDPs) - Part 2}
    \frametitle{Key Concepts and Terminology}
    \begin{itemize}
        \item \textbf{Policy (\(\pi\))}: A strategy that the agent uses to decide actions based on the current state, represented as \(\pi(a|s)\).
        \item \textbf{Value Function (V)}: Measures the goodness of a state under a specific policy:
        \[
        V^{\pi}(s) = \mathbb{E} \left[ \sum_{t=0}^{\infty} \gamma^t R(s_t, a_t) \big| s_0 = s, \pi \right]
        \]
        \item \textbf{Optimal Policy (\(\pi^*\))}: The policy that maximizes the expected sum of rewards.
    \end{itemize}
\end{frame}

\begin{frame}[fragile]{Overview of Markov Decision Processes (MDPs) - Part 3}
    \frametitle{Example Scenario and Key Points}
    \begin{itemize}
        \item \textbf{Example Scenario}: 
        \begin{itemize}
            \item A grid world agent aims to reach a goal while avoiding obstacles. Each cell is a state, allowing movements of up, down, left, or right.
        \end{itemize}
        \item \textbf{Key Points}:
        \begin{itemize}
            \item MDPs are foundational for reinforcement learning algorithms.
            \item Components of MDPs motivate solution techniques, including:
            \begin{itemize}
                \item \textbf{Dynamic Programming}: Uses Bellman equations for value functions and optimal policies.
                \item \textbf{Monte Carlo Methods}: Estimates value functions through averaging returns from sampled episodes.
                \item \textbf{Temporal Difference Learning}: Combines ideas from both dynamic programming and Monte Carlo methods.
            \end{itemize}
        \end{itemize}
        \item \textbf{Conclusion}: Understanding MDPs facilitates learning about reinforcement learning algorithms and their application to real-world decisions.
    \end{itemize}
\end{frame}

\begin{frame}[fragile]{Conclusion - Summary of Key Concepts}
    In this chapter, we explored \textbf{Markov Decision Processes (MDPs)}, a mathematical framework for sequential decision-making under uncertainty. MDPs model environments where outcomes are partly random and partly controlled.
    
    The key components of MDPs are:
    \begin{itemize}
        \item \textbf{States (S)}: Various configurations of the environment.
        \item \textbf{Actions (A)}: Choices available to the decision-maker.
        \item \textbf{Transition Function (P)}: Probability of moving from one state to another after taking an action.
        \item \textbf{Reward Function (R)}: Feedback in terms of rewards for each transition.
        \item \textbf{Discount Factor ($\gamma$)}: Represents the importance of future rewards.
    \end{itemize}
\end{frame}

\begin{frame}[fragile]{Mathematical Representation of MDPs}
    The objective in an MDP is to find a policy $\pi$ that maximizes the expected sum of rewards over time. The \textbf{Optimal Value Function ($V^*$)} is defined as:
    \begin{equation}
    V^*(s) = \max_{\pi} \sum_{a \in A} \sum_{s' \in S} P(s'|s, a) [R(s, a) + \gamma V^*(s')]
    \end{equation}
\end{frame}

\begin{frame}[fragile]{Example and Applications}
    \textbf{Example:} Consider a grid world where an agent can move in four directions (up, down, left, right). The agent receives a reward for reaching a goal state (e.g., +10) and a penalty for falling into a trap (e.g., -10). 

    \begin{itemize}
        \item The agent’s goal is to find the optimal policy that maximizes its expected rewards while avoiding traps.
        \item \textbf{Key Points:}
        \begin{itemize}
            \item Understanding MDPs is crucial for fields like machine learning and robotics.
            \item \textbf{Value Iteration} and \textbf{Policy Iteration} are two primary algorithms used to find optimal policies.
            \item MDPs are applicable in various domains such as robotics (path planning), finance (investment decisions), and healthcare (treatment planning).
        \end{itemize}
    \end{itemize}
\end{frame}

\begin{frame}[fragile]{Conclusion}
    By mastering MDPs, you are prepared to tackle complex decision-making problems involving uncertainty. 

    \begin{block}{Next Steps}
        The next steps will involve implementing algorithms based on MDPs and applying them to practical scenarios for better understanding and real-world application.
    \end{block}
\end{frame}


\end{document}