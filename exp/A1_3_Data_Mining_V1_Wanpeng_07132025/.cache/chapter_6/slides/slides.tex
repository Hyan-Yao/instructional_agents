\documentclass[aspectratio=169]{beamer}

% Theme and Color Setup
\usetheme{Madrid}
\usecolortheme{whale}
\useinnertheme{rectangles}
\useoutertheme{miniframes}

% Additional Packages
\usepackage[utf8]{inputenc}
\usepackage[T1]{fontenc}
\usepackage{graphicx}
\usepackage{booktabs}
\usepackage{listings}
\usepackage{amsmath}
\usepackage{amssymb}
\usepackage{xcolor}
\usepackage{tikz}
\usepackage{pgfplots}
\pgfplotsset{compat=1.18}
\usetikzlibrary{positioning}
\usepackage{hyperref}

% Custom Colors
\definecolor{myblue}{RGB}{31, 73, 125}
\definecolor{mygray}{RGB}{100, 100, 100}
\definecolor{mygreen}{RGB}{0, 128, 0}
\definecolor{myorange}{RGB}{230, 126, 34}
\definecolor{mycodebackground}{RGB}{245, 245, 245}

% Set Theme Colors
\setbeamercolor{structure}{fg=myblue}
\setbeamercolor{frametitle}{fg=white, bg=myblue}
\setbeamercolor{title}{fg=myblue}
\setbeamercolor{section in toc}{fg=myblue}
\setbeamercolor{item projected}{fg=white, bg=myblue}
\setbeamercolor{block title}{bg=myblue!20, fg=myblue}
\setbeamercolor{block body}{bg=myblue!10}
\setbeamercolor{alerted text}{fg=myorange}

% Set Fonts
\setbeamerfont{title}{size=\Large, series=\bfseries}
\setbeamerfont{frametitle}{size=\large, series=\bfseries}
\setbeamerfont{caption}{size=\small}
\setbeamerfont{footnote}{size=\tiny}

% Custom Commands
\newcommand{\hilight}[1]{\colorbox{myorange!30}{#1}}
\newcommand{\concept}[1]{\textcolor{myblue}{\textbf{#1}}}
\newcommand{\separator}{\begin{center}\rule{0.5\linewidth}{0.5pt}\end{center}}

% Title Page Information
\title[Week 6: Deep Learning]{Week 6: Introduction to Deep Learning}
\author[J. Smith]{John Smith, Ph.D.}
\institute[University Name]{
  Department of Computer Science\\
  University Name\\
  \vspace{0.3cm}
  Email: email@university.edu\\
  Website: www.university.edu
}
\date{\today}

% Document Start
\begin{document}

\frame{\titlepage}

\begin{frame}[fragile]
    \titlepage
\end{frame}

\begin{frame}[fragile]
    \frametitle{What is Deep Learning?}
    \begin{block}{Definition}
        Deep Learning is a subfield of artificial intelligence (AI) that imitates the workings of the human brain to process data and create patterns for use in decision making.
    \end{block}
    
    \begin{itemize}
        \item Involves artificial neural networks (ANNs)
        \item Consists of multiple layers of nodes (neurons)
        \item Each layer refines output through complex computations
    \end{itemize}
\end{frame}

\begin{frame}[fragile]
    \frametitle{Why Do We Need Deep Learning?}
    \begin{itemize}
        \item \textbf{High-dimensional Data:} 
            \begin{itemize}
                \item Traditional models struggle with vast amounts of digital data.
                \item Deep learning excels in processing high-dimensional spaces.
            \end{itemize}
        
        \item \textbf{Complex Patterns:}
            \begin{itemize}
                \item Real-world problems often include intricate patterns and nonlinear relationships.
                \item Deep learning can capture these complexities effectively.
            \end{itemize}

        \item \textbf{Automation:}
            \begin{itemize}
                \item Automates feature engineering, reducing manual intervention.
                \item Models learn features automatically from raw data.
            \end{itemize}
    \end{itemize}
\end{frame}

\begin{frame}[fragile]
    \frametitle{Real-world Applications of Deep Learning}
    \begin{itemize}
        \item \textbf{Natural Language Processing (NLP):} 
            \begin{itemize}
                \item Applications like ChatGPT use deep learning for language understanding and generation.
            \end{itemize}
        
        \item \textbf{Computer Vision:}
            \begin{itemize}
                \item Powers technologies for image and video processing, including facial recognition and autonomous vehicles.
            \end{itemize}

        \item \textbf{Healthcare:}
            \begin{itemize}
                \item Assists in diagnosing diseases from medical images and personalizing treatment plans.
            \end{itemize}
    \end{itemize}
\end{frame}

\begin{frame}[fragile]
    \frametitle{Key Concepts Covered in This Week}
    \begin{enumerate}
        \item \textbf{Neural Networks:} Understanding the structure and function of neural networks.
        \item \textbf{Training Deep Learning Models:} 
            \begin{itemize}
                \item Exploring methods for training models including backpropagation and optimization techniques.
            \end{itemize}
        \item \textbf{Popular Architectures:} 
            \begin{itemize}
                \item Overview of CNNs for images and RNNs for sequences.
            \end{itemize}
        \item \textbf{Evaluation Metrics:} 
            \begin{itemize}
                \item Metrics for assessing model performance: accuracy, precision, recall, F1-score.
            \end{itemize}
        \item \textbf{Future Trends in Deep Learning:} 
            \begin{itemize}
                \item Insights into emerging applications such as reinforcement learning and transfer learning.
            \end{itemize}
    \end{enumerate}
\end{frame}

\begin{frame}[fragile]
    \frametitle{Summary and Transition}
    \begin{block}{Summary}
        This introduction to Deep Learning sets the stage for the upcoming sessions where we will explore powerful techniques integral to various fields. Emphasis will be on the iterative learning process and automation capabilities of deep learning.
    \end{block}
    
    \begin{block}{Transition}
        In the upcoming slide, we will provide a detailed overview of key concepts essential for understanding deep learning and its applications.
    \end{block}
\end{frame}

\begin{frame}[fragile]
    \frametitle{Overview of Key Concepts in Deep Learning}
    \begin{itemize}
        \item Definition of Deep Learning
        \item Motivation behind Deep Learning
        \item Core concepts
        \item Applications
        \item Key takeaways
    \end{itemize}
\end{frame}

\begin{frame}[fragile]
    \frametitle{What is Deep Learning?}
    \begin{block}{Definition}
        Deep Learning is a subset of Machine Learning that utilizes neural networks with many layers to analyze data and identify patterns. It mimics human learning through experience.
    \end{block}
    \begin{itemize}
        \item Proficient in handling vast amounts of data
        \item Automatically extracts features without human intervention
    \end{itemize}
\end{frame}

\begin{frame}[fragile]
    \frametitle{Motivation Behind Deep Learning}
    \begin{itemize}
        \item \textbf{Need for Complexity:} 
        \begin{itemize}
            \item Traditional algorithms struggle with high-dimensional data
            \item Deep Learning tackles complexities using multiple layers
        \end{itemize}
        \item \textbf{Increased Data Availability:} 
        \begin{itemize}
            \item Explosion of data from images, text, and audio
            \item Need for powerful analytical tools
        \end{itemize}
    \end{itemize}
    \begin{block}{Example}
        ChatGPT, powered by Deep Learning, utilizes vast datasets to understand and generate human-like responses, showcasing its capabilities in natural language processing.
    \end{block}
\end{frame}

\begin{frame}[fragile]
    \frametitle{Core Concepts}
    \begin{itemize}
        \item \textbf{Neural Networks:} Layers (input, hidden, output) with interconnected nodes (neurons)
        \item \textbf{Activation Functions:} Introduce non-linearity
        \begin{itemize}
            \item Sigmoid: \( f(x) = \frac{1}{1 + e^{-x}} \)
            \item ReLU: \( f(x) = \max(0, x) \)
        \end{itemize}
        \item \textbf{Backpropagation:} Updates weights by calculating error and minimizing it through gradient descent
    \end{itemize}
\end{frame}

\begin{frame}[fragile]
    \frametitle{Applications of Deep Learning}
    \begin{itemize}
        \item \textbf{Image Recognition:} Facial recognition in security systems
        \item \textbf{Natural Language Processing:} Tools like ChatGPT for text generation, question answering, translating languages
        \item \textbf{Healthcare Diagnostics:} Analyzing medical imaging to identify diseases
    \end{itemize}
\end{frame}

\begin{frame}[fragile]
    \frametitle{Key Takeaways}
    \begin{itemize}
        \item Deep Learning is driven by vast data and complexity
        \item Neural networks learn patterns through training and adapt based on feedback
        \item It's revolutionizing various fields with practical applications
    \end{itemize}
    \begin{block}{Summary}
        \begin{itemize}
            \item Addresses limitations in traditional computational approaches
            \item Ability to learn from large, unstructured datasets is a game changer
            \item Keeping up with innovations is crucial for relevance in technology fields
        \end{itemize}
    \end{block}
\end{frame}

\begin{frame}[fragile]
    \frametitle{Conclusion - Part 1}
    \begin{block}{Summary of Key Concepts in Deep Learning}
        Deep Learning has emerged as a powerful subset of Machine Learning that is essential for various applications today, from computer vision to natural language processing.
    \end{block}
    \begin{itemize}
        \item Understanding the Fundamentals
        \item Why Deep Learning?
        \item Training Deep Learning Models
        \item Real-world Applications
    \end{itemize}
\end{frame}

\begin{frame}[fragile]
    \frametitle{Conclusion - Part 2}
    \begin{block}{Understanding the Fundamentals}
        \begin{itemize}
            \item \textbf{Neural Networks:} Inspired by the human brain's architecture.
                \begin{itemize}
                    \item \textbf{Neurons:} Fundamental units that process input data.
                    \item \textbf{Layers:} Arranged in input, hidden, and output layers.
                    \item \textbf{Activation Functions:} Introduce non-linearities (e.g., ReLU, Sigmoid).
                \end{itemize}
        \end{itemize}
    \end{block}

    \begin{block}{Why Deep Learning?}
        \begin{itemize}
            \item \textbf{Complexity in Data:} Automatically extracts patterns from high-dimensional data.
            \item \textbf{Recent Successes:} Applications like ChatGPT demonstrate its capabilities for generating coherent text.
        \end{itemize}
    \end{block}
\end{frame}

\begin{frame}[fragile]
    \frametitle{Conclusion - Part 3}
    \begin{block}{Training Deep Learning Models}
        \begin{itemize}
            \item \textbf{Backpropagation and Optimization:} Gradient descent updates weights to minimize loss.
            \item \textbf{Overfitting Mitigation:} Strategies like dropout and regularization improve generalization.
        \end{itemize}
    \end{block}

    \begin{block}{Real-world Applications}
        \begin{itemize}
            \item \textbf{Healthcare:} Medical imaging for disease detection.
            \item \textbf{Autonomous Vehicles:} Sensory data processing for navigation.
            \item \textbf{Finance:} Fraud detection and algorithmic trading.
        \end{itemize}
    \end{block}
    
    \begin{block}{Conclusion}
        Deep Learning offers unprecedented innovation opportunities. Understanding its principles is essential for aspiring data scientists.
    \end{block}
\end{frame}


\end{document}