\documentclass[aspectratio=169]{beamer}

% Theme and Color Setup
\usetheme{Madrid}
\usecolortheme{whale}
\useinnertheme{rectangles}
\useoutertheme{miniframes}

% Additional Packages
\usepackage[utf8]{inputenc}
\usepackage[T1]{fontenc}
\usepackage{graphicx}
\usepackage{booktabs}
\usepackage{listings}
\usepackage{amsmath}
\usepackage{amssymb}
\usepackage{xcolor}
\usepackage{tikz}
\usepackage{pgfplots}
\pgfplotsset{compat=1.18}
\usetikzlibrary{positioning}
\usepackage{hyperref}

% Custom Colors
\definecolor{myblue}{RGB}{31, 73, 125}
\definecolor{mygray}{RGB}{100, 100, 100}
\definecolor{mygreen}{RGB}{0, 128, 0}
\definecolor{myorange}{RGB}{230, 126, 34}
\definecolor{mycodebackground}{RGB}{245, 245, 245}

% Set Theme Colors
\setbeamercolor{structure}{fg=myblue}
\setbeamercolor{frametitle}{fg=white, bg=myblue}
\setbeamercolor{title}{fg=myblue}
\setbeamercolor{section in toc}{fg=myblue}
\setbeamercolor{item projected}{fg=white, bg=myblue}
\setbeamercolor{block title}{bg=myblue!20, fg=myblue}
\setbeamercolor{block body}{bg=myblue!10}
\setbeamercolor{alerted text}{fg=myorange}

% Set Fonts
\setbeamerfont{title}{size=\Large, series=\bfseries}
\setbeamerfont{frametitle}{size=\large, series=\bfseries}
\setbeamerfont{caption}{size=\small}
\setbeamerfont{footnote}{size=\tiny}

% Code Listing Style
\lstdefinestyle{customcode}{
  backgroundcolor=\color{mycodebackground},
  basicstyle=\footnotesize\ttfamily,
  breakatwhitespace=false,
  breaklines=true,
  commentstyle=\color{mygreen}\itshape,
  keywordstyle=\color{blue}\bfseries,
  stringstyle=\color{myorange},
  numbers=left,
  numbersep=8pt,
  numberstyle=\tiny\color{mygray},
  frame=single,
  framesep=5pt,
  rulecolor=\color{mygray},
  showspaces=false,
  showstringspaces=false,
  showtabs=false,
  tabsize=2,
  captionpos=b
}
\lstset{style=customcode}

% Custom Commands
\newcommand{\hilight}[1]{\colorbox{myorange!30}{#1}}
\newcommand{\source}[1]{\vspace{0.2cm}\hfill{\tiny\textcolor{mygray}{Source: #1}}}
\newcommand{\concept}[1]{\textcolor{myblue}{\textbf{#1}}}
\newcommand{\separator}{\begin{center}\rule{0.5\linewidth}{0.5pt}\end{center}}

% Footer and Navigation Setup
\setbeamertemplate{footline}{
  \leavevmode%
  \hbox{%
  \begin{beamercolorbox}[wd=.3\paperwidth,ht=2.25ex,dp=1ex,center]{author in head/foot}%
    \usebeamerfont{author in head/foot}\insertshortauthor
  \end{beamercolorbox}%
  \begin{beamercolorbox}[wd=.5\paperwidth,ht=2.25ex,dp=1ex,center]{title in head/foot}%
    \usebeamerfont{title in head/foot}\insertshorttitle
  \end{beamercolorbox}%
  \begin{beamercolorbox}[wd=.2\paperwidth,ht=2.25ex,dp=1ex,center]{date in head/foot}%
    \usebeamerfont{date in head/foot}
    \insertframenumber{} / \inserttotalframenumber
  \end{beamercolorbox}}%
  \vskip0pt%
}

% Turn off navigation symbols
\setbeamertemplate{navigation symbols}{}

% Title Page Information
\title[Deep Q-Networks]{Deep Q-Networks}
\author[J. Smith]{John Smith, Ph.D.}
\institute[University Name]{
  Department of Computer Science\\
  University Name\\
  \vspace{0.3cm}
  Email: email@university.edu\\
  Website: www.university.edu
}
\date{\today}

% Document Start
\begin{document}

\frame{\titlepage}

\begin{frame}[fragile]
    \frametitle{Introduction to Deep Q-Networks}
    \begin{block}{Overview of Deep Q-Networks (DQNs)}
        Deep Q-Networks represent an integration of **Q-learning** and **deep learning**, allowing reinforcement learning agents to learn optimal strategies in complex environments.
    \end{block}
\end{frame}

\begin{frame}[fragile]
    \frametitle{Key Concepts}
    \begin{enumerate}
        \item \textbf{Q-learning Basics:} 
        \begin{itemize}
            \item A model-free algorithm that learns action values.
            \item Uses the Q-function \( Q(s, a) \) to estimate expected future rewards.
        \end{itemize}
        
        \item \textbf{Deep Learning's Contribution:}
        \begin{itemize}
            \item Traditional Q-learning uses tables for Q-values, limiting scalability.
            \item Utilizes deep neural networks to approximate the Q-function for high-dimensional inputs (e.g., images).
        \end{itemize}
    \end{enumerate}
\end{frame}

\begin{frame}[fragile]
    \frametitle{Significance in Reinforcement Learning}
    \begin{itemize}
        \item \textbf{Scalability:} 
        \begin{itemize}
            \item Can learn policies in vast state spaces (e.g., playing Atari games).
        \end{itemize}
        \item \textbf{Generalization:}
        \begin{itemize}
            \item Deep learning enables better performance across unseen states.
        \end{itemize}
    \end{itemize}
    
    \begin{block}{What We'll Cover in This Chapter}
        \begin{enumerate}
            \item Background on Q-learning.
            \item Architecture of DQNs.
            \item Training and optimization techniques.
            \item Applications and case studies.
        \end{enumerate}
    \end{block}
\end{frame}

\begin{frame}[fragile]
    \frametitle{Background on Q-learning - Overview}
    \begin{itemize}
        \item Q-learning is a value-based reinforcement learning algorithm.
        \item Enables an agent to learn optimal actions to maximize cumulative rewards.
        \item Model-free approach: learns directly from experience without a model of the environment.
    \end{itemize}
\end{frame}

\begin{frame}[fragile]
    \frametitle{Background on Q-learning - Key Concepts}
    \begin{itemize}
        \item \textbf{Agent}: The learner or decision-maker.
        \item \textbf{Environment}: The external system the agent interacts with.
        \item \textbf{State (s)}: Current situation representation of the agent.
        \item \textbf{Action (a)}: Any potential move the agent can make.
        \item \textbf{Reward (r)}: Feedback indicating the success of an action.
    \end{itemize}
\end{frame}

\begin{frame}[fragile]
    \frametitle{Background on Q-learning - Q-Value Function and Update Rule}
    \begin{block}{Q-Value Function}
        The Q-value function, denoted as \( Q(s, a) \), indicates the expected utility of taking action \( a \) in state \( s \).
    \end{block}
    
    \begin{block}{Update Formula}
        The core update rule for adjusting Q-values is:
        \[
        Q(s, a) \leftarrow Q(s, a) + \alpha \left( r + \gamma \max_{a'} Q(s', a') - Q(s, a) \right)
        \]
        Where:
        \begin{itemize}
            \item \( \alpha \) (learning rate): Determines how much new information overrides old information.
            \item \( r \): Reward received after taking action \( a \).
            \item \( \gamma \) (discount factor): Balances immediate and future rewards.
            \item \( s' \): New state after executing action \( a \).
        \end{itemize}
    \end{block}
\end{frame}

\begin{frame}[fragile]
    \frametitle{The Role of Neural Networks in DQNs - Overview}
    \begin{itemize}
        \item Deep Q-Networks (DQNs) utilize neural networks to estimate Q-values.
        \item Enables scaling to larger, more complex environments.
        \item Addresses limitations of traditional Q-learning due to the curse of dimensionality.
    \end{itemize}
\end{frame}

\begin{frame}[fragile]
    \frametitle{What are Q-values?}
    \begin{itemize}
        \item \textbf{Definition:} Q-values represent expected future rewards for taking action \( a \) in state \( s \).
        \item \textbf{Objective of DQNs:} Learn function \( Q^* \) to maximize expected rewards over time:
        \begin{equation}
            Q^* = \max \mathbb{E}[R_t | s_t, a_t]
        \end{equation}
    \end{itemize}
\end{frame}

\begin{frame}[fragile]
    \frametitle{How Neural Networks Help Reduce Complexity}
    \begin{enumerate}
        \item \textbf{Function Approximation:}
        \begin{itemize}
            \item Generalizes from limited experiences.
            \item Predicts Q-values for unvisited state-action pairs through experience replay memory.
        \end{itemize}

        \item \textbf{Handling High Dimensionality:}
        \begin{itemize}
            \item Neural networks process high-dimensional input (e.g., raw pixels).
            \item Transforms it into a lower-dimensional representation.
        \end{itemize}

        \item \textbf{Continuous Learning:}
        \begin{itemize}
            \item Updates improve approximations of Q-values over time.
        \end{itemize}
    \end{enumerate}
\end{frame}

\begin{frame}[fragile]
    \frametitle{Key Components of DQNs}
    \begin{itemize}
        \item \textbf{Input Layer:} Captures state representations (e.g., pixel data).
        \item \textbf{Hidden Layers:} Process input data, applying non-linear transformations.
        \item \textbf{Output Layer:} Represents Q-values for each potential action.
    \end{itemize}
\end{frame}

\begin{frame}[fragile]
    \frametitle{Example: Playing Atari Games}
    \begin{itemize}
        \item Consider the Atari game \textit{Breakout}:
        \begin{itemize}
            \item Inputs: Raw pixel frames.
            \item DQN processes frames to approximate Q-values for actions like:
            \begin{itemize}
                \item Move left
                \item Move right
                \item Fire
            \end{itemize}
            \item As gameplay progresses, the network refines its strategy based on experiences.
        \end{itemize}
    \end{itemize}
\end{frame}

\begin{frame}[fragile]
    \frametitle{Conclusion}
    \begin{itemize}
        \item Neural networks enable DQNs to approximate Q-values in complex environments.
        \item Facilitate development of intelligent systems that adapt and improve decision-making.
        \item Synthesis of deep learning and reinforcement learning empowers nuanced strategies in diverse scenarios.
    \end{itemize}
\end{frame}

\begin{frame}[fragile]
    \frametitle{DQN Architecture - Overview}
    \begin{itemize}
        \item Deep Q-Networks (DQNs) use a neural network architecture.
        \item They approximate Q-values in complex environments, enabling effective decision-making.
        \item The architecture consists of three main components:
        \begin{itemize}
            \item Input Layer
            \item Hidden Layers
            \item Output Layer
        \end{itemize}
    \end{itemize}
\end{frame}

\begin{frame}[fragile]
    \frametitle{DQN Architecture - Components}
    \begin{block}{1. Input Layer}
        \begin{itemize}
            \item **Purpose**: Receives the current state representation.
            \item **Examples**: 
            \begin{itemize}
                \item Game frame image
                \item Simple state variables: [position, speed]
            \end{itemize}
        \end{itemize}
    \end{block}

    \begin{block}{2. Hidden Layers}
        \begin{itemize}
            \item **Purpose**: Apply transformations to learn complex features.
            \item **Common Structures**: 
            \begin{itemize}
                \item Fully Connected Layers
                \item Convolutional Layers
            \end{itemize}
            \item **Activation Functions**: Typically use Rectified Linear Units (ReLU).
        \end{itemize}
    \end{block}

    \begin{block}{3. Output Layer}
        \begin{itemize}
            \item **Purpose**: Represents action value estimates.
            \item **Structure**: Number of neurons equals the number of possible actions.
            \item **Example**: In a game with four actions, outputs a four-dimensional vector.
        \end{itemize}
    \end{block}
\end{frame}

\begin{frame}[fragile]
    \frametitle{DQN Architecture - Key Points & Conclusion}
    \begin{itemize}
        \item **Q-values**: Quantify expected future rewards based on current state and actions taken.
        \item **Learning Objective**: Minimize difference between predicted and target Q-values.
        \[
        Q(s, a) \leftarrow R + \gamma \max_{a'} Q(s', a')
        \]
        \item **Example Setup**: Inputs a 4-dimensional state vector; uses two hidden layers with ReLU, outputs Q-values for four actions.
    \end{itemize}

    \begin{block}{Conclusion}
        Understanding DQN architecture is crucial for approximating Q-values effectively, improving agent decision-making in complex environments.
    \end{block}

    \begin{block}{Next Steps}
        Transition to the next slide: **Experience Replay** mechanism.
    \end{block}
\end{frame}

\begin{frame}[fragile]
    \frametitle{Experience Replay - Overview}
    \begin{itemize}
        \item Experience Replay is a technique for enhancing the training efficiency of Deep Q-Networks (DQNs).
        \item It stores past experiences and samples from them during training.
        \item Key benefits:
        \begin{itemize}
            \item Efficient learning from diverse past experiences.
            \item Improved stability in training.
            \item Enhanced convergence speed in learning.
        \end{itemize}
    \end{itemize}
\end{frame}

\begin{frame}[fragile]
    \frametitle{Experience Replay - How It Works}
    \begin{enumerate}
        \item \textbf{Experience Storage:}
        \begin{itemize}
            \item Tuple format: 
            \[
            (s_t, a_t, r_t, s_{t+1})
            \]
            \item Where \(s_t\) = state, \(a_t\) = action, \(r_t\) = reward, and \(s_{t+1}\) = next state.
        \end{itemize}
        
        \item \textbf{Replay Buffer:}
        \begin{itemize}
            \item Stores tuples, limited in size. Older experiences are discarded when exceeded.
        \end{itemize}
        
        \item \textbf{Sampling:}
        \begin{itemize}
            \item Experiences are randomly sampled to break correlations, stabilizing the training process.
        \end{itemize}
    \end{enumerate}
\end{frame}

\begin{frame}[fragile]
    \frametitle{Experience Replay - Benefits and Example}
    \begin{block}{Benefits}
        \begin{itemize}
            \item Efficient learning from a variety of past experiences.
            \item Stability in Q-value updates via averaged samples.
            \item Faster convergence to optimal strategies.
        \end{itemize}
    \end{block}
    
    \begin{block}{Example}
        Without Experience Replay:
        \begin{itemize}
            \item A unique situation encountered once may not result in effective learning.
        \end{itemize}
        With Experience Replay:
        \begin{itemize}
            \item The agent can revisit situations through stored experiences.
            \item E.g., 
            \[
            (s_t: \text{being pursued}, a_t: \text{run}, r_t: -1, s_{t+1}: \text{safe location})
            \]
        \end{itemize}
    \end{block}
    
\end{frame}

\begin{frame}[fragile]
    \frametitle{Target Network Updates - Overview}
    \begin{block}{Introduction}
        In Deep Q-Networks (DQNs), the instability in training is primarily due to the correlation between estimated Q-values and their corresponding targets. Target Networks are introduced to mitigate this issue.
    \end{block}
\end{frame}

\begin{frame}[fragile]
    \frametitle{Target Network Updates - Concepts}
    \begin{enumerate}
        \item \textbf{Target Networks vs. Main Networks:}
        \begin{itemize}
            \item DQNs utilize a Main Network (Q-network) for generating Q-values based on the current policy.
            \item The Target Network calculates target Q-values, aiding in the Main Network updates.
        \end{itemize}
        
        \item \textbf{Purpose of Target Networks:}
        \begin{itemize}
            \item Stabilizes training by providing a consistent target.
            \item Reduces correlation with rapidly changing Main Network parameters.
        \end{itemize}
    \end{enumerate}
\end{frame}

\begin{frame}[fragile]
    \frametitle{Target Network Updates - Mechanism and Benefits}
    \begin{block}{Update Mechanism}
        The Target Network is updated less frequently than the Main Network, typically every N steps:
        \begin{equation}
            Q_{\text{target}} \leftarrow Q_{\text{main}} \quad \text{(updated every N steps)}
        \end{equation}
    \end{block}

    \begin{block}{Benefits of Target Networks}
        \begin{itemize}
            \item \textbf{Stability in Learning:} Decoupling target values from turbulent Q-values results in stable training.
            \item \textbf{Reduced Variance:} The Target Network lowers the variance in updates, reducing oscillations or divergence.
        \end{itemize}
    \end{block}
\end{frame}

\begin{frame}[fragile]
    \frametitle{Target Network Updates - Example and Summary}
    \begin{block}{Example}
        Consider an agent playing a game using a single neural network:
        \begin{itemize}
            \item High variance in learning due to constant updates leads to possible learning failure.
            \item A Target Network allows for stable calculation of expected future rewards based on less volatile Q-values.
        \end{itemize}
    \end{block}

    \begin{block}{Key Points}
        \begin{itemize}
            \item Target Networks are essential for stabilizing learning in DQNs.
            \item Infrequent updates mitigate harmful correlations, aiding convergence.
            \item They strike a balance between exploration and convergence stability.
        \end{itemize}
    \end{block}
\end{frame}

\begin{frame}[fragile]
    \frametitle{Training Process of DQNs - Overview}
    In this section, we will detail the training process of Deep Q-Networks (DQNs), which includes:
    \begin{itemize}
        \item Data collection mechanism
        \item Dual-network architecture for updates
        \item Optimization of the network to minimize loss
    \end{itemize}
\end{frame}

\begin{frame}[fragile]
    \frametitle{Training Process of DQNs - Data Collection}
    \textbf{1. Data Collection}
    \begin{itemize}
        \item \textbf{Experience Replay Buffer}
        \begin{itemize}
            \item DQNs utilize an experience replay buffer to store agent experiences, which consist of tuples \((s_t, a_t, r_t, s_{t+1})\).
            \begin{itemize}
                \item \(s_t\): state at time \(t\)
                \item \(a_t\): action taken at state \(s_t\)
                \item \(r_t\): reward received after action \(a_t\)
                \item \(s_{t+1}\): next state after action
            \end{itemize}
            \item Random sampling from the buffer helps break correlation between consecutive samples, improving training stability.
        \end{itemize}
    \end{itemize}
\end{frame}

\begin{frame}[fragile]
    \frametitle{Training Process of DQNs - Example Data}
    \textbf{Example:}
    Consider an agent learning to play a game by collecting experiences:
    \begin{itemize}
        \item \(s_t\): Current game state (position, score)
        \item \(a_t\): Action taken (jump/move right)
        \item \(r_t\): Reward received (+1 for hitting a target, -1 for missing)
        \item \(s_{t+1}\): New game state after action
    \end{itemize}
\end{frame}

\begin{frame}[fragile]
    \frametitle{Training Process of DQNs - Network Updates}
    \textbf{2. Network Updates}
    DQNs employ two networks:
    \begin{itemize}
        \item \textbf{Main Network}
            \begin{itemize}
                \item Updated at every time step based on sampled experiences from the replay buffer.
            \end{itemize}
        \item \textbf{Target Network}
            \begin{itemize}
                \item Updated less frequently to stabilize training by providing consistent targets for Q-values.
            \end{itemize}
    \end{itemize}
    \textbf{Update Mechanism:}
    \begin{equation}
    Q(s_t, a_t) = \text{MainNetwork}(s_t)
    \end{equation}
    \begin{equation}
    Q^{target} = r_t + \gamma \max_a Q_{target}(s_{t+1}, a)
    \end{equation}
    where \(\gamma\) is the discount factor.
\end{frame}

\begin{frame}[fragile]
    \frametitle{Training Process of DQNs - Loss Minimization}
    \textbf{3. Loss Minimization}
    The goal of training is to minimize the loss:
    \begin{itemize}
        \item \textbf{Loss Function} 
        \begin{equation}
        L = \frac{1}{N} \sum (Q(s_t, a_t) - Q^{target})^2
        \end{equation}
        where \(N\) is the number of sampled experiences.
    \end{itemize}
    \textbf{Key Points:}
    \begin{itemize}
        \item Replay buffer provides diverse experiences for better learning.
        \item Target networks prevent oscillations and training instability.
        \item Periodic updates of the target network stabilize training.
    \end{itemize}
\end{frame}

\begin{frame}[fragile]
    \frametitle{Training Process of DQNs - Summary}
    \textbf{Summary:}
    The training process of DQNs is essential for learning effective policies in complex environments. 
    Key methods:
    \begin{itemize}
        \item Leveraging experience replay
        \item Maintaining target networks
        \item Minimizing loss through Mean Squared Error (MSE)
    \end{itemize}
\end{frame}

\begin{frame}[fragile]
    \frametitle{Training Process of DQNs - Code Snippet}
    \textbf{Brief Code Snippet (Pseudo-code):}
    \begin{lstlisting}[language=Python]
# Pseudo-code for training loop
for episode in range(num_episodes):
    state = env.reset()
    done = False
    while not done:
        action = select_action(state)  # Epsilon-greedy selection
        next_state, reward, done = env.step(action)
        replay_buffer.add(state, action, reward, next_state, done)
        
        # Sample mini-batch from replay buffer
        minibatch = replay_buffer.sample(batch_size)
        # Update main network
        for s, a, r, s_next, done in minibatch:
            target = r + (1 - done) * gamma * max(Q_target(s_next, a))
            loss = (Q_main(s, a) - target)**2
            # Backpropagation to update weights
    \end{lstlisting}
\end{frame}

\begin{frame}[fragile]
    \frametitle{Applications of Deep Q-Networks}
    \begin{block}{Overview}
        Deep Q-Networks (DQNs) leverage deep learning for decision-making in complex environments. Their profound impact is notable in various fields, especially gaming and robotics, showcasing the algorithm's adaptability and power in solving real-world problems.
    \end{block}
\end{frame}

\begin{frame}[fragile]
    \frametitle{Key Applications of DQNs}
    \begin{enumerate}
        \item \textbf{Gaming}
        \begin{itemize}
            \item \textbf{Breakout (Atari Game):}
            \begin{itemize}
                \item Learned to play at a superhuman level through trial and error.
                \item \textit{How?} Translates pixel data into states to maximize rewards.
            \end{itemize}
            \item \textbf{AlphaGo:}
            \begin{itemize}
                \item Pivotal in developing the AI that defeated a world champion in Go.
                \item \textit{How?} Combined deep reinforcement learning with tree search methods.
            \end{itemize}
        \end{itemize}
        
        \item \textbf{Robotics}
        \begin{itemize}
            \item \textbf{Autonomous Navigation:}
            \begin{itemize}
                \item Helps robots navigate environments and avoid obstacles.
            \end{itemize}
            \item \textbf{Manipulation Tasks:}
            \begin{itemize}
                \item Used for object manipulation, like picking and placing items.
            \end{itemize}
        \end{itemize}
        
        \item \textbf{Healthcare}
        \begin{itemize}
            \item \textbf{Personalized Treatment Planning:}
            \begin{itemize}
                \item Optimizes treatment plans based on patient data.
            \end{itemize}
        \end{itemize}
    \end{enumerate}
\end{frame}

\begin{frame}[fragile]
    \frametitle{Key Points to Emphasize}
    \begin{itemize}
        \item \textbf{Learning through Interaction:} DQNs learn optimal actions through environment interactions, often leading to novel strategies.
        \item \textbf{Generalization Capabilities:} DQNs generalize well to similar tasks, ideal for environments without explicit programming.
        \item \textbf{Combining Exploration and Exploitation:} Balance between exploring unknown actions and exploiting known rewards is crucial for success.
    \end{itemize}

    \begin{block}{Conclusion}
        Deep Q-Networks exemplify AI capabilities to master complex tasks and make autonomous decisions, with significant applications in gaming and robotics.
    \end{block}
\end{frame}

\begin{frame}[fragile]
    \frametitle{Challenges and Limitations of Deep Q-Networks (DQNs)}
    
    \begin{block}{Overview}
        Deep Q-Networks (DQNs) have revolutionized reinforcement learning, particularly in complex environments. However, their application is not without challenges. Understanding these issues is crucial for effective application and development of DQNs.
    \end{block}
\end{frame}

\begin{frame}[fragile]
    \frametitle{Common Challenges in Training DQNs - Part 1}
    
    \begin{enumerate}
        \item \textbf{Overfitting}
        \begin{itemize}
            \item \textbf{Explanation:} The model performs well on training data but fails to generalize to unseen data, exploiting specific patterns.
            \item \textbf{Example:} A DQN trained to play "Breakout" may perform poorly on new scenarios.
            \item \textbf{Mitigation:} 
                \begin{itemize}
                    \item Experience replay
                    \item Target network stabilization
                \end{itemize}
        \end{itemize}
        
        \item \textbf{Instability}
        \begin{itemize}
            \item \textbf{Explanation:} Variance in training performance due to interactions between the neural network and learning updates.
            \item \textbf{Example:} Sudden policy changes can lead to large fluctuations in Q-values.
            \item \textbf{Mitigation:} 
                \begin{itemize}
                    \item Experience replay buffers
                    \item Separate target networks
                \end{itemize}
        \end{itemize}
    \end{enumerate}
\end{frame}

\begin{frame}[fragile]
    \frametitle{Common Challenges in Training DQNs - Part 2}
    
    \begin{enumerate}[resume]
        \item \textbf{Sample Inefficiency}
        \begin{itemize}
            \item \textbf{Explanation:} DQNs require a large number of interactions to learn effectively, leading to sample inefficiency.
            \item \textbf{Example:} Learning an effective policy in a high-dimensional space may require millions of interaction steps, which is impractical.
            \item \textbf{Mitigation:} 
                \begin{itemize}
                    \item Prioritized experience replay
                \end{itemize}
        \end{itemize}
        
        \item \textbf{Real-World Limitations}
        \begin{itemize}
            \item \textbf{Explanation:} DQNs struggle with noise and unmodeled dynamics in real-world applications.
            \item \textbf{Example:} A robotic arm trained in simulation may fail due to variations in weight and unforeseen obstacles.
            \item \textbf{Mitigation:} 
                \begin{itemize}
                    \item Domain randomization
                \end{itemize}
        \end{itemize}
    \end{enumerate}
\end{frame}

\begin{frame}[fragile]
    \frametitle{Key Takeaways and Conclusion}

    \begin{itemize}
        \item \textbf{Key Challenges:} Overfitting, instability, sample inefficiency, and real-world limitations are critical in DQN training.
        \item \textbf{Importance of Understanding:} Addressing these challenges requires a combination of theoretical knowledge and practical application methods.
        \item \textbf{Effective Strategies:} Modifying training processes and using additional techniques enhance robustness and applicability of DQNs.
    \end{itemize}

    \begin{block}{Conclusion}
        While DQNs are powerful reinforcement learning tools, their inherent challenges require careful consideration and strategic approaches to optimize performance in both simulated and real environments.
    \end{block}
\end{frame}

\begin{frame}[fragile]
    \frametitle{Future Directions in Deep Q-Networks (DQNs)}
    \begin{block}{Overview}
        Deep Q-Networks (DQNs) have significantly advanced the field of reinforcement learning (RL). Ongoing research aims to address existing challenges and limitations while exploring enhancements in DQN architectures and methodologies.
    \end{block}
\end{frame}

\begin{frame}[fragile]
    \frametitle{Key Future Directions - Part 1}
    \begin{enumerate}
        \item \textbf{Enhanced Exploration Techniques}
        \begin{itemize}
            \item \textbf{Curiosity-Driven Exploration}: Rewards agents for exploring novel states.
            \item \textbf{Noisy Networks}: Introduces noise in action selection to diversify behavior.
        \end{itemize}
        \item \textbf{Improved Stability and Generalization}
        \begin{itemize}
            \item \textbf{Double Q-Learning}: Reduces overestimation bias with two value function estimators.
            \item \textbf{Dueling Network Architectures}: Separates value and advantage estimates for efficient learning.
        \end{itemize}
    \end{enumerate}
\end{frame}

\begin{frame}[fragile]
    \frametitle{Key Future Directions - Part 2}
    \begin{enumerate}\setcounter{enumi}{3}
        \item \textbf{Integration of Model-based Approaches}
        \begin{itemize}
            \item \textbf{Hybrid Models}: Combines model-free and model-based RL to enhance action selection.
        \end{itemize}
        \item \textbf{Scalability to Real-world Applications}
        \begin{itemize}
            \item \textbf{Hierarchical Reinforcement Learning}: Decomposes tasks for better decision-making in complex environments.
            \item \textbf{Real-Time Adaptation}: Allows DQNs to adapt to dynamic environments.
        \end{itemize}
    \end{enumerate}
\end{frame}

\begin{frame}[fragile]
    \frametitle{Q-Learning Update Rule}
    \begin{block}{Formula Reference}
        The Q-Learning update rule can be defined as follows:
        \begin{equation}
        Q(s, a) \leftarrow Q(s, a) + \alpha \left( r + \gamma \max_{a'} Q(s', a') - Q(s, a) \right)
        \end{equation}
    Where:
    \begin{itemize}
        \item $s$: current state
        \item $a$: action taken
        \item $r$: reward received
        \item $s'$: next state
        \item $\alpha$: learning rate
        \item $\gamma$: discount factor
    \end{itemize}
    \end{block}
\end{frame}

\begin{frame}[fragile]
    \frametitle{Closing Key Points}
    \begin{itemize}
        \item The evolution of DQNs emphasizes enhancements in exploration techniques, stability, generalization, and applicability to real-world problems.
        \item Future developments in architecture will lead to more robust and scalable agents, capable of thriving in complex and diverse environments.
    \end{itemize}
\end{frame}

\begin{frame}[fragile]
    \frametitle{Engagement Prompt}
    \begin{block}{Discussion}
        Consider potential applications in your field. How could these advancements in DQNs change the way agents interact and learn in your area of interest?
    \end{block}
\end{frame}

\begin{frame}[fragile]
    \frametitle{Conclusion - Part 1}
    \begin{block}{Summary of Key Takeaways from the Chapter on Deep Q-Networks (DQNs)}
        \begin{enumerate}
            \item \textbf{Understanding DQNs}:
                \begin{itemize}
                    \item DQNs integrate deep learning with Q-learning to learn effective policies.
                    \item A DQN approximates the Q-value function using a neural network.
                \end{itemize}
            \item \textbf{Key Components}:
                \begin{itemize}
                    \item \textbf{Experience Replay}: Breaks correlation between samples, stabilizing training.
                    \item \textbf{Target Network}: Periodically updated to stabilize training and mitigate Q-value divergence.
                \end{itemize}
        \end{enumerate}
    \end{block}
\end{frame}

\begin{frame}[fragile]
    \frametitle{Conclusion - Part 2}
    \begin{block}{Impact on Reinforcement Learning}
        \begin{enumerate}
            \item DQNs achieved human-level performance in complex environments like Atari and Go.
            \item The methodology has applications in various fields beyond gaming, including:
                \begin{itemize}
                    \item Robotics
                    \item Autonomous vehicles
                    \item Healthcare
                \end{itemize}
        \end{enumerate}
    \end{block}

    \begin{block}{Advantages \& Challenges}
        \begin{itemize}
            \item \textbf{Advantages}:
                \begin{itemize}
                    \item Generalization across states due to function approximation.
                    \item Handles high-dimensional input spaces, such as images.
                \end{itemize}
            \item \textbf{Challenges}:
                \begin{itemize}
                    \item Instability and divergence during training.
                    \item Requires careful hyperparameter tuning to optimize performance.
                \end{itemize}
        \end{itemize}
    \end{block}
\end{frame}

\begin{frame}[fragile]
    \frametitle{Conclusion - Part 3}
    \begin{block}{Future Directions}
        Ongoing research aims to enhance DQNs through:
        \begin{itemize}
            \item Integration of new techniques such as:
                \begin{itemize}
                    \item Double Q-learning
                    \item Prioritized experience replay
                    \item Dueling network architectures
                \end{itemize}
        \end{itemize}
    \end{block}

    \begin{block}{Key Formula}
        The Bellman equation for updating the action-value function \( Q(s, a) \):
        \begin{equation}
            Q(s, a) \leftarrow Q(s, a) + \alpha \left[ r + \gamma \max_{a'} Q(s', a') - Q(s, a) \right]
        \end{equation}
        where:
        \begin{itemize}
            \item \( \alpha \): learning rate
            \item \( r \): immediate reward
            \item \( \gamma \): discount factor
            \item \( s' \): next state after action \( a \)
        \end{itemize}
    \end{block}
\end{frame}

\begin{frame}[fragile]
    \frametitle{Q\&A Session}
    \begin{block}{Objective}
        Open the floor for questions and discussions regarding Deep Q-Networks (DQNs) and facilitate a deeper understanding of the material covered.
    \end{block}
\end{frame}

\begin{frame}[fragile]
    \frametitle{Overview of Deep Q-Networks}
    \begin{itemize}
        \item \textbf{What is a Deep Q-Network?}
        \begin{itemize}
            \item A type of reinforcement learning algorithm that utilizes deep learning to approximate the Q-value function.
            \item DQNs use neural networks to estimate the expected future rewards for each action taken in a given state.
        \end{itemize}
        \item \textbf{Key Concepts:}
        \begin{itemize}
            \item Q-Learning: An off-policy reinforcement learning algorithm that seeks to find the optimal action-selection policy.
            \item Experience Replay: A technique where a replay buffer stores past experiences to improve learning stability.
            \item Target Network: A separate Q-network used to stabilize the training process.
        \end{itemize}
    \end{itemize}
\end{frame}

\begin{frame}[fragile]
    \frametitle{Discussion Topics and Key Points}
    \begin{itemize}
        \item \textbf{Example Questions to Guide Discussion:}
        \begin{itemize}
            \item What challenges might arise in training a DQN?
            \item How does experience replay impact the efficiency of learning in DQNs?
            \item Can you explain the difference between the greedy policy and epsilon-greedy strategy?
        \end{itemize}
        \item \textbf{Key Points to Emphasize:}
        \begin{itemize}
            \item DQNs can converge to optimal strategies under certain conditions with careful tuning.
            \item The use of neural networks allows DQNs to generalize from fewer examples.
            \item DQNs have been successfully applied in various domains including gaming and robotics.
        \end{itemize}
    \end{itemize}
\end{frame}

\begin{frame}[fragile]
    \frametitle{Essential Formula}
    \begin{block}{Q-Learning Update Rule}
        \begin{equation}
        Q(s, a) \leftarrow Q(s, a) + \alpha \left( r + \gamma \max_{a'} Q(s', a') - Q(s, a) \right)
        \end{equation}
        Where:
        \begin{itemize}
            \item $s$ = current state
            \item $a$ = current action
            \item $r$ = reward received
            \item $s'$ = next state
            \item $\alpha$ = learning rate
            \item $\gamma$ = discount factor
        \end{itemize}
    \end{block}
\end{frame}

\begin{frame}[fragile]
    \frametitle{Code Snippet}
    \begin{lstlisting}[language=Python]
import numpy as np
import random
from collections import deque

class DQNAgent:
    def __init__(self):
        self.memory = deque(maxlen=2000)
        self.gamma = 0.95  # discount rate
        self.epsilon = 1.0  # exploration rate
        self.epsilon_min = 0.01
        self.epsilon_decay = 0.995

    def remember(self, state, action, reward, next_state, done):
        self.memory.append((state, action, reward, next_state, done))
    
    def replay(self, batch_size):
        minibatch = random.sample(self.memory, batch_size)
        # Update DQN here using Q-Learning update rule
    \end{lstlisting}
\end{frame}

\begin{frame}[fragile]
    \frametitle{Conclusion}
    \begin{block}{Summary}
        This session allows participants to clarify doubts, explore topics of interest, and strengthen their understanding of Deep Q-Networks. Encourage engagement and complex questions to further enrich the learning experience.
    \end{block}
\end{frame}


\end{document}