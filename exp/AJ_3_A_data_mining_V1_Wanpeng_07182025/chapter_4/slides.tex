\documentclass[aspectratio=169]{beamer}

% Theme and Color Setup
\usetheme{Madrid}
\usecolortheme{whale}
\useinnertheme{rectangles}
\useoutertheme{miniframes}

% Additional Packages
\usepackage[utf8]{inputenc}
\usepackage[T1]{fontenc}
\usepackage{graphicx}
\usepackage{booktabs}
\usepackage{listings}
\usepackage{amsmath}
\usepackage{amssymb}
\usepackage{xcolor}
\usepackage{tikz}
\usepackage{pgfplots}
\pgfplotsset{compat=1.18}
\usetikzlibrary{positioning}
\usepackage{hyperref}

% Custom Colors
\definecolor{myblue}{RGB}{31, 73, 125}
\definecolor{mygray}{RGB}{100, 100, 100}
\definecolor{mygreen}{RGB}{0, 128, 0}
\definecolor{myorange}{RGB}{230, 126, 34}
\definecolor{mycodebackground}{RGB}{245, 245, 245}

% Set Theme Colors
\setbeamercolor{structure}{fg=myblue}
\setbeamercolor{frametitle}{fg=white, bg=myblue}
\setbeamercolor{title}{fg=myblue}
\setbeamercolor{section in toc}{fg=myblue}
\setbeamercolor{item projected}{fg=white, bg=myblue}
\setbeamercolor{block title}{bg=myblue!20, fg=myblue}
\setbeamercolor{block body}{bg=myblue!10}
\setbeamercolor{alerted text}{fg=myorange}

% Set Fonts
\setbeamerfont{title}{size=\Large, series=\bfseries}
\setbeamerfont{frametitle}{size=\large, series=\bfseries}
\setbeamerfont{caption}{size=\small}
\setbeamerfont{footnote}{size=\tiny}

% Code Listing Style
\lstdefinestyle{customcode}{
  backgroundcolor=\color{mycodebackground},
  basicstyle=\footnotesize\ttfamily,
  breakatwhitespace=false,
  breaklines=true,
  commentstyle=\color{mygreen}\itshape,
  keywordstyle=\color{blue}\bfseries,
  stringstyle=\color{myorange},
  numbers=left,
  numbersep=8pt,
  numberstyle=\tiny\color{mygray},
  frame=single,
  framesep=5pt,
  rulecolor=\color{mygray},
  showspaces=false,
  showstringspaces=false,
  showtabs=false,
  tabsize=2,
  captionpos=b
}
\lstset{style=customcode}

% Custom Commands
\newcommand{\hilight}[1]{\colorbox{myorange!30}{#1}}
\newcommand{\source}[1]{\vspace{0.2cm}\hfill{\tiny\textcolor{mygray}{Source: #1}}}
\newcommand{\concept}[1]{\textcolor{myblue}{\textbf{#1}}}
\newcommand{\separator}{\begin{center}\rule{0.5\linewidth}{0.5pt}\end{center}}

% Footer and Navigation Setup
\setbeamertemplate{footline}{
  \leavevmode%
  \hbox{%
  \begin{beamercolorbox}[wd=.3\paperwidth,ht=2.25ex,dp=1ex,center]{author in head/foot}%
    \usebeamerfont{author in head/foot}\insertshortauthor
  \end{beamercolorbox}%
  \begin{beamercolorbox}[wd=.5\paperwidth,ht=2.25ex,dp=1ex,center]{title in head/foot}%
    \usebeamerfont{title in head/foot}\insertshorttitle
  \end{beamercolorbox}%
  \begin{beamercolorbox}[wd=.2\paperwidth,ht=2.25ex,dp=1ex,center]{date in head/foot}%
    \usebeamerfont{date in head/foot}
    \insertframenumber{} / \inserttotalframenumber
  \end{beamercolorbox}}%
  \vskip0pt%
}

% Turn off navigation symbols
\setbeamertemplate{navigation symbols}{}

% Title Page Information
\title[Classification Techniques]{Week 4: Classification Techniques}
\author[J. Smith]{John Smith, Ph.D.}
\institute[University Name]{
  Department of Computer Science\\
  University Name\\
  \vspace{0.3cm}
  Email: email@university.edu\\
  Website: www.university.edu
}
\date{\today}

% Document Start
\begin{document}

\frame{\titlepage}

\begin{frame}[fragile]
    \frametitle{Introduction to Classification Techniques}
    \begin{block}{Overview of Classification in Data Mining}
        Classification is a supervised learning technique that involves identifying the category or class of new observations based on training data.
    \end{block}
\end{frame}

\begin{frame}[fragile]
    \frametitle{Importance of Classification}
    \begin{enumerate}
        \item \textbf{Decision-Making}: Organizations can make data-driven decisions by categorizing data meaningfully. For example, banks identify default risks in loan applications.
        
        \item \textbf{Pattern Recognition}: Classification reveals patterns that may not be immediately obvious, improving patient care by categorizing patients based on health metrics.
        
        \item \textbf{Automation}: Classification automates processes, such as email spam filtering.
        
        \item \textbf{Forecasting}: Businesses use classification to predict future behaviors, like customer purchasing patterns.
    \end{enumerate}
\end{frame}

\begin{frame}[fragile]
    \frametitle{Common Classification Algorithms}
    \begin{itemize}
        \item \textbf{Logistic Regression}: For binary classification. 
        \begin{equation}
            P(Y=1|X) = \frac{1}{1 + e^{-(\beta_0 + \beta_1X_1 + \beta_2X_2 + ... + \beta_nX_n)}}
        \end{equation}
        
        \item \textbf{Decision Trees}: A structure that classifies data based on feature values.
        
        \item \textbf{Support Vector Machines (SVM)}: Finds the hyperplane that best separates different classes.
    \end{itemize}
\end{frame}

\begin{frame}[fragile]
    \frametitle{Example Scenario}
    Consider a dataset of emails with features such as word frequency, sender reputation, and the presence of links. A classification algorithm can be trained on labeled examples (spam or not spam) to classify new emails automatically.
\end{frame}

\begin{frame}[fragile]
    \frametitle{Key Points and Conclusion}
    \begin{block}{Key Points}
        \begin{itemize}
            \item Classification is essential for effective data mining.
            \item It enables better decision-making and pattern recognition.
            \item Familiarity with algorithms is crucial for real-world applications.
        \end{itemize}
    \end{block}
    
    As we proceed, we'll explore classification algorithms in detail, their applications, and performance assessment techniques. Make sure to grasp these concepts as they form the foundation for advanced data mining techniques!
\end{frame}

\begin{frame}[fragile]
    \frametitle{Learning Objectives - Overview}
    \begin{block}{Key Learning Objectives for This Week’s Class on Classification Techniques}
        \begin{enumerate}
            \item Understand the Concept of Classification
            \item Explore Common Classification Algorithms
            \item Implement Classification Models
            \item Utilize Evaluation Metrics
            \item Recognize Real-World Applications
        \end{enumerate}
    \end{block}
\end{frame}

\begin{frame}[fragile]
    \frametitle{Learning Objectives - Concept of Classification}
    \begin{block}{1. Understand the Concept of Classification}
        \begin{itemize}
            \item \textbf{Definition:} 
            Classification is a supervised learning technique where the model learns to categorize data into predefined classes based on input features.
            \item \textbf{Importance:} 
            Classification techniques are foundational in data mining and machine learning, enabling effective predictions in various applications such as spam detection, disease diagnosis, and customer segmentation.
        \end{itemize}
    \end{block}
\end{frame}

\begin{frame}[fragile]
    \frametitle{Learning Objectives - Algorithms and Implementations}
    \begin{block}{2. Explore Common Classification Algorithms}
        \begin{itemize}
            \item \textbf{Decision Trees:} Tree-like structures that make decisions based on feature values.
            \item \textbf{Logistic Regression:} Used for binary classification problems by estimating probabilities.
            \item \textbf{Support Vector Machines (SVM):} Finds the optimal hyperplane to separate classes.
            \item \textbf{K-Nearest Neighbors (KNN):} Classifies based on the majority class among the 'k' nearest neighbors in the feature space.
        \end{itemize}
    \end{block}

    \begin{block}{3. Implement Classification Models}
        Hands-On Training: Students will learn how to:
        \begin{itemize}
            \item Preprocess data (handling missing values, encoding categorical variables, scaling).
            \item Split datasets into training and testing sets.
            \item Train models using popular libraries (e.g., Scikit-learn in Python).
            \item Evaluate model performance using metrics such as accuracy, precision, recall, and F1 score.
        \end{itemize}
    \end{block}
\end{frame}

\begin{frame}[fragile]
    \frametitle{Learning Objectives - Evaluation Metrics and Applications}
    \begin{block}{4. Utilize Evaluation Metrics}
        \begin{itemize}
            \item \textbf{Model Assessment:} Understand and apply different evaluation metrics to assess classification model effectiveness:
            \begin{itemize}
                \item \textbf{Confusion Matrix:} Visual representation of true vs. predicted classifications.
                \item \textbf{Accuracy:} Proportion of true results among the total number of cases.
                \item \textbf{Precision \& Recall:} Measures to evaluate model relevance and completeness. 
            \end{itemize}
        \end{itemize}
    \end{block}

    \begin{block}{Example Calculation: Confusion Matrix}
        \[
        \begin{array}{|c|c|c|}
        \hline
        & \text{Predicted Positive} & \text{Predicted Negative} \\
        \hline
        \text{Actual Positive} & TP & FN \\
        \hline
        \text{Actual Negative} & FP & TN \\
        \hline
        \end{array}
        \]
        \textbf{TP:} True Positives, \textbf{TN:} True Negatives, \textbf{FP:} False Positives, \textbf{FN:} False Negatives.
    \end{block}
\end{frame}

\begin{frame}[fragile]
    \frametitle{Learning Objectives - Real-World Applications}
    \begin{block}{5. Recognize Real-World Applications}
        \begin{itemize}
            \item \textbf{Case Studies:} Discuss various scenarios where classification is applicable, including:
            \begin{itemize}
                \item \textbf{Medical Diagnosis:} Identifying diseases based on patient data.
                \item \textbf{Fraud Detection:} Classifying transactions to identify fraudulent activities.
                \item \textbf{Sentiment Analysis:} Classifying reviews as positive, negative, or neutral.
            \end{itemize}
        \end{itemize}
    \end{block}

    \begin{block}{Emphasis Points}
        Classifications will not only help in automating decisions but also enhance predictive capabilities across different domains. Mastery of classification techniques provides a robust foundation for advanced machine learning concepts.
    \end{block}
\end{frame}

\begin{frame}[fragile]
    \frametitle{Decision Trees - Introduction}
    \begin{block}{Definition}
        A Decision Tree is a flowchart-like structure used for classification and regression tasks. 
        It splits the dataset into branches, resembling a tree structure where:
        \begin{itemize}
            \item Each internal node represents a feature (or attribute).
            \item Each branch represents a decision rule.
            \item Each terminal node (or leaf) represents an outcome.
        \end{itemize}
    \end{block}
\end{frame}

\begin{frame}[fragile]
    \frametitle{Decision Trees - Structure}
    \begin{block}{Components}
        \begin{enumerate}
            \item \textbf{Root Node}: The top node representing the entire dataset.
            \item \textbf{Internal Nodes}: Represent tests on attributes that split the data based on feature values.
            \item \textbf{Branches}: Paths from nodes representing the outcome of a test.
            \item \textbf{Leaf Nodes}: Final outcomes or classifications predicted by the model.
        \end{enumerate}
    \end{block}
\end{frame}

\begin{frame}[fragile]
    \frametitle{Decision Trees - Classification Process}
    \begin{block}{Process Overview}
        \begin{enumerate}
            \item Start from the root and choose a feature to split the data.
            \item Apply decision rules recursively until a stopping criterion is met.
            \item Assign the majority class of the leaf node to the data point.
        \end{enumerate}
    \end{block}
    
    \begin{block}{Example}
        Classifying fruits based on features like color, size, and texture:
        \begin{itemize}
            \item Is the fruit red? 
            \begin{itemize}
                \item Yes: Check Size
                \begin{itemize}
                    \item Is Size large? 
                    \begin{itemize}
                        \item Yes: Leaf - "Apple"
                        \item No: Leaf - "Cherry"
                    \end{itemize}
                \end{itemize}
                \item No: Check Color
                \begin{itemize}
                    \item Is it yellow? 
                    \begin{itemize}
                        \item Yes: Leaf - "Banana"
                        \item No: Leaf - "Orange"
                    \end{itemize}
                \end{itemize}
            \end{itemize}
        \end{itemize}
    \end{block}
\end{frame}

\begin{frame}[fragile]
    \frametitle{Decision Trees - Key Points}
    \begin{itemize}
        \item \textbf{Interpretability}: Easy to visualize and interpret, suitable for non-experts.
        \item \textbf{Non-linear Relationships}: Can capture non-linear relationships between features and outcomes.
        \item \textbf{Overfitting}: Deep trees can overfit the training data; methods like pruning can help.
    \end{itemize}
\end{frame}

\begin{frame}[fragile]
    \frametitle{Decision Trees - Considerations}
    \begin{block}{Splitting Criteria}
        \begin{itemize}
            \item \textbf{Gini Impurity}: Measures node impurity; lower values are preferred.
            \item \textbf{Entropy}: Quantifies uncertainty in the dataset. 
        \end{itemize}
        A common formula for Gini Impurity:
        \begin{equation}
            Gini(p) = 1 - \sum (p_i^2)
        \end{equation}
        where \( p_i \) is the probability of a class in the node.
    \end{block}

    \begin{block}{Performance Measures}
        \begin{itemize}
            \item \textbf{Accuracy}: Proportion of correctly predicted instances.
            \item \textbf{Confusion Matrix}: A table to evaluate the performance of a classification model.
        \end{itemize}
    \end{block}
\end{frame}

\begin{frame}[fragile]
    \frametitle{Decision Trees - Summary}
    \begin{block}{Conclusions}
        Decision Trees are powerful tools for classification tasks, offering a combination of visual clarity and decision-making capabilities. Understanding their structure and operation is crucial for deploying effective machine learning models.
    \end{block}
\end{frame}

\begin{frame}[fragile]
    \frametitle{Building Decision Trees}
    \begin{block}{Introduction}
        Decision Trees are a popular method for classification in machine learning. 
        They structure decision-making visually and enable intuitive interpretation of results. 
        We will discuss the process of constructing a Decision Tree, focusing on data splitting and stopping criteria.
    \end{block}
\end{frame}

\begin{frame}[fragile]
    \frametitle{Step 1: Prepare Your Data}
    \begin{itemize}
        \item \textbf{Input Features:} Identify relevant features from your dataset. 
        \item \textbf{Target Variable:} Clearly define the target variable, e.g., 'buy' or 'not buy'.
    \end{itemize}
\end{frame}

\begin{frame}[fragile]
    \frametitle{Step 2: Choose a Splitting Criterion}
    Splitting criteria determine how to partition the data at each node. Common criteria include:
    \begin{itemize}
        \item \textbf{Gini Impurity:} 
        \begin{equation}
            Gini(D) = 1 - \sum_{i=1}^{c} p_i^2
        \end{equation}
        \item \textbf{Entropy:} 
        \begin{equation}
            Entropy(D) = - \sum_{i=1}^{c} p_i \log_2(p_i)
        \end{equation}
        \item \textbf{Information Gain:} 
        \begin{equation}
            IG(D, A) = Entropy(D) - \sum_{v \in Values(A)} \frac{|D_v|}{|D|} Entropy(D_v)
        \end{equation}
    \end{itemize}
\end{frame}

\begin{frame}[fragile]
    \frametitle{Step 3: Split the Data}
    \begin{itemize}
        \item Based on the chosen criterion, partition the dataset into subsets.
        \item Objective: create homogeneous subsets.
    \end{itemize}
    \textbf{Example:} Split based on age where Age < 30 results in 90\% 'Not Buy' and Age $\geq$ 30 results in 70\% 'Buy'.
\end{frame}

\begin{frame}[fragile]
    \frametitle{Step 4: Determine Stopping Criteria}
    Stopping criteria guide when to halt tree growth:
    \begin{itemize}
        \item \textbf{Maximum Depth:} Limit the number of splits.
        \item \textbf{Minimum Samples at Leaf Node:} Minimum samples to consider a leaf.
        \item \textbf{Purity Threshold:} Stop when nodes are sufficiently pure.
    \end{itemize}
\end{frame}

\begin{frame}[fragile]
    \frametitle{Conclusion}
    Building Decision Trees involves iterating through feature selection, evaluating split criteria, and applying stopping rules. Mastery enhances your ability to create effective models for classification tasks.
\end{frame}

\begin{frame}[fragile]
    \frametitle{Key Points to Emphasize}
    \begin{itemize}
        \item Understand and apply different splitting criteria (Gini, Entropy).
        \item Properly define stopping criteria to prevent overfitting.
        \item Each decision impacts model interpretability and accuracy.
    \end{itemize}
\end{frame}

\begin{frame}[fragile]
    \frametitle{Visual Aid Suggestion}
    Consider including a diagram that illustrates:
    \begin{itemize}
        \item The flow from decision nodes to leaf nodes.
        \item Highlight sample splits for clarity.
    \end{itemize}
\end{frame}

\begin{frame}[fragile]
    \frametitle{Further Reading}
    Explore Decision Tree algorithms such as:
    \begin{itemize}
        \item CART (Classification and Regression Trees)
        \item ID3 (Iterative Dichotomiser 3)
    \end{itemize}
    Further reading helps deepen your understanding of Decision Trees.
\end{frame}

\begin{frame}[fragile]
    \frametitle{Advantages of Decision Trees}
    \begin{enumerate}
        \item \textbf{Simplicity and Interpretability}
        \begin{itemize}
            \item Decision Trees are intuitive and easy to visualize.
            \item Example: A tree structure can illustrate decisions like playing outside based on weather and temperature.
        \end{itemize}

        \item \textbf{No Need for Data Normalization}
        \begin{itemize}
            \item Direct handling of numerical and categorical data without feature scaling.
        \end{itemize}
        
        \item \textbf{Handling Non-linear Relationships}
        \begin{itemize}
            \item Can model non-linear relationships and interactions, e.g., age and income influencing loan approval.
        \end{itemize}
        
        \item \textbf{Automatic Feature Selection}
        \begin{itemize}
            \item Automatically ranks feature importance, aiding in model simplification and reducing overfitting.
        \end{itemize}
        
        \item \textbf{Robust to Outliers}
        \begin{itemize}
            \item Splits focus on data grouping, minimizing the impact of outliers, e.g., extreme housing prices.
        \end{itemize}
    \end{enumerate}
\end{frame}

\begin{frame}[fragile]
    \frametitle{Limitations of Decision Trees}
    \begin{enumerate}
        \item \textbf{Overfitting}
        \begin{itemize}
            \item Trees can become overly complex, modeling noise rather than the pattern.
            \item Example: A tree based on very few samples might perform poorly on unseen data.
        \end{itemize}

        \item \textbf{Instability}
        \begin{itemize}
            \item Small changes in data can yield significantly different trees.
            \item Key Point: Techniques like bagging can mitigate this issue.
        \end{itemize}

        \item \textbf{Bias towards Dominant Classes}
        \begin{itemize}
            \item Imbalanced datasets may cause bias towards majority classes.
            \item Example: In medical diagnosis, a tree might excessively predict healthy outcomes in skewed data.
        \end{itemize}
    \end{enumerate}
\end{frame}

\begin{frame}[fragile]
    \frametitle{Limitations of Decision Trees - Continued}
    \begin{enumerate}
        \setcounter{enumi}{3}
        \item \textbf{Limited Performance on Imbalanced Datasets}
        \begin{itemize}
            \item Decision Trees may favor prevalent classes during splits.
            \item Solution: Techniques like stratified sampling can aid class balance.
        \end{itemize}

        \item \textbf{Greedy Nature}
        \begin{itemize}
            \item Splitting decisions are made based on short-term criteria rather than a holistic approach.
            \item Key Point: This can lead to suboptimal trees.
        \end{itemize}
    \end{enumerate}
\end{frame}

\begin{frame}[fragile]
    \frametitle{Summary of Decision Trees}
    \begin{block}{Overview}
        Decision Trees are intuitive and flexible classification tools, providing clear advantages such as simplicity and interpretability. However, considerations regarding overfitting, instability, and bias must be addressed. Utilizing pruning and ensemble methods can enhance their predictive performance.
    \end{block}
\end{frame}

\begin{frame}[fragile]
    \frametitle{k-Nearest Neighbors (k-NN) - Introduction}
    \begin{block}{Introduction}
        k-Nearest Neighbors (k-NN) is a simple yet powerful instance-based learning algorithm primarily used for classification tasks. 
        It is a lazy learner, meaning it does not create a model during the training phase but instead memorizes the training instances for prediction.
    \end{block}
\end{frame}

\begin{frame}[fragile]
    \frametitle{k-Nearest Neighbors (k-NN) - How the Algorithm Works}
    \begin{enumerate}
        \item \textbf{Training Phase:} Store all training examples in memory.
        \item \textbf{Classification Phase:}
        \begin{itemize}
            \item \textbf{Calculate Distance:} Measure distance between new input and stored training examples. Common metrics include:
            \begin{itemize}
                \item \textbf{Euclidean Distance:}
                \begin{equation}
                    d(x, y) = \sqrt{\sum_{i=1}^{n} (x_i - y_i)^2}
                \end{equation}
                \item \textbf{Manhattan Distance:}
                \begin{equation}
                    d(x, y) = \sum_{i=1}^{n} |x_i - y_i|
                \end{equation}
            \end{itemize}
            \item \textbf{Identify Neighbors:} Find the $k$ closest training examples.
            \item \textbf{Voting Mechanism:} Assign the class label that is most frequent among the $k$ neighbors.
        \end{itemize}
    \end{enumerate}
\end{frame}

\begin{frame}[fragile]
    \frametitle{k-Nearest Neighbors (k-NN) - Key Points}
    \begin{itemize}
        \item \textbf{Choosing k:} A small $k$ may lead to overfitting, while a large $k$ might cause underfitting. Use cross-validation to find the optimal $k$.
        \item \textbf{Scalability:} k-NN can be computationally intensive as datasets grow, requiring distance calculations for each query.
        \item \textbf{Feature Scaling:} Standardize or normalize data to ensure equal contribution of features to distance calculations.
    \end{itemize}
\end{frame}

\begin{frame}[fragile]
    \frametitle{k-Nearest Neighbors (k-NN) - Example Illustration}
    Consider a 2D dataset with two classes: red stars and blue circles. A new point (e.g., an orange square) is classified based on its nearest neighbors:
    \begin{itemize}
        \item If $k=3$ and two of the three nearest neighbors are red stars, the new point is classified as a red star.
    \end{itemize}
\end{frame}

\begin{frame}[fragile]
    \frametitle{k-Nearest Neighbors (k-NN) - Code Snippet}
    \begin{lstlisting}[language=Python]
from sklearn.neighbors import KNeighborsClassifier

# Sample Data
X = [[0, 0], [1, 1], [1, 0], [0, 1]]  # Features
y = [0, 1, 1, 0]                      # Labels (0 = Class A, 1 = Class B)

# Create k-NN model
model = KNeighborsClassifier(n_neighbors=3)

# Train the model
model.fit(X, y)

# Predict a new data point
new_data = [[0.5, 0.5]]
prediction = model.predict(new_data)
print(prediction)
    \end{lstlisting}
\end{frame}

\begin{frame}[fragile]
    \frametitle{k-Nearest Neighbors (k-NN) - Conclusion}
    \begin{block}{Conclusion}
        k-NN is an intuitive and effective algorithm for classification tasks, providing a solid baseline for many problems. 
        Its simplicity requires careful consideration of distance metrics, the choice of $k$, and its scalability with larger datasets.
    \end{block}
\end{frame}

\begin{frame}[fragile]
    \frametitle{Distance Metrics in k-NN}
    \begin{block}{Introduction}
        In the k-Nearest Neighbors (k-NN) algorithm, distance metrics are vital for measuring the similarity between data points.
        The choice of metric can greatly influence classification performance and accuracy.
    \end{block}
\end{frame}

\begin{frame}[fragile]
    \frametitle{1. Euclidean Distance}
    \begin{itemize}
        \item \textbf{Definition:} The Euclidean distance measures the "straight-line" distance between two points in Euclidean space:
        \begin{equation}
            d(x, y) = \sqrt{\sum_{i=1}^{n} (x_i - y_i)^2}
        \end{equation}
        \item \textbf{Example:} For points A(2, 3) and B(5, 7):
        \begin{equation}
            d(A, B) = \sqrt{(5-2)^2 + (7-3)^2} = \sqrt{9 + 16} = 5
        \end{equation}
    \end{itemize}
\end{frame}

\begin{frame}[fragile]
    \frametitle{2. Manhattan Distance}
    \begin{itemize}
        \item \textbf{Definition:} The Manhattan distance measures distance across a grid-like path:
        \begin{equation}
            d(x, y) = \sum_{i=1}^{n} |x_i - y_i|
        \end{equation}
        \item \textbf{Example:} For points A(2, 3) and B(5, 7):
        \begin{equation}
            d(A, B) = |5 - 2| + |7 - 3| = 3 + 4 = 7
        \end{equation}
    \end{itemize}
\end{frame}

\begin{frame}[fragile]
    \frametitle{3. Other Distance Metrics}
    \begin{itemize}
        \item \textbf{Minkowski Distance:} Generalized form that includes both Euclidean and Manhattan distances:
        \begin{equation}
            d(x, y) = \left( \sum_{i=1}^{n} |x_i - y_i|^p \right)^{1/p}
        \end{equation}
        \begin{itemize}
            \item Set \( p = 2 \) for Euclidean and \( p = 1 \) for Manhattan.
        \end{itemize}
        
        \item \textbf{Hamming Distance:} Used for categorical data, it measures differing positions between two equal-length strings.
    \end{itemize}
\end{frame}

\begin{frame}[fragile]
    \frametitle{Practical Considerations in k-NN}
    \begin{itemize}
        \item \textbf{Data Scaling:} Important for improving distance calculations, especially with Euclidean distance.
        \item \textbf{Choosing the Right Metric:} Consider data type (continuous vs. categorical) and the context of usage.
    \end{itemize}
\end{frame}

\begin{frame}[fragile]
    \frametitle{Conclusion}
    Understanding and selecting the appropriate distance metric is crucial for optimal k-NN implementation, as it directly affects classification accuracy. Grasping these metrics aids in making informed decisions that improve outcomes.
\end{frame}

\begin{frame}[fragile]
    \frametitle{Strengths and Weaknesses of k-NN}
    \begin{block}{Overview of k-NN}
        The k-NN algorithm is a simple, yet powerful classification method that operates based on the proximity of data points in a feature space. Given a new instance, k-NN identifies the 'k' closest training instances and assigns a class label based on majority voting among those neighbors.
    \end{block}
\end{frame}

\begin{frame}[fragile]
    \frametitle{Strengths of k-NN}
    \begin{enumerate}
        \item \textbf{Simplicity and Intuition}
        \begin{itemize}
            \item \textbf{Explanation:} k-NN is straightforward to understand and implement.
            \item \textbf{Example:} Classifying fruits based on size and color.
        \end{itemize}
        
        \item \textbf{No Training Phase}
        \begin{itemize}
            \item \textbf{Explanation:} No traditional training phase, making it fast to set up.
            \item \textbf{Example:} Ideal for applications where model acquisition is not feasible.
        \end{itemize}

        \item \textbf{Adaptability}
        \begin{itemize}
            \item \textbf{Explanation:} Easily adapts to changes by recalculating distances.
            \item \textbf{Example:} New data can be included seamlessly.
        \end{itemize}

        \item \textbf{Works Well with Unlabeled Data}
        \begin{itemize}
            \item \textbf{Explanation:} Useful in semi-supervised scenarios with partly labeled data.
            \item \textbf{Example:} Classifying users based on interaction without explicit labels.
        \end{itemize}
    \end{enumerate}
\end{frame}

\begin{frame}[fragile]
    \frametitle{Weaknesses of k-NN}
    \begin{enumerate}
        \item \textbf{Computational Complexity}
        \begin{itemize}
            \item \textbf{Explanation:} Requires calculating distances from the point being classified to every point in the training set.
            \item \textbf{Illustration:} Time complexity increases to O(n * d), where n is the number of instances and d is the number of dimensions.
        \end{itemize}

        \item \textbf{Sensitivity to Noise}
        \begin{itemize}
            \item \textbf{Explanation:} Irrelevant features or noisy data can significantly impact performance.
            \item \textbf{Example:} A mislabeled instance too close may incorrectly influence classification.
        \end{itemize}

        \item \textbf{Choice of 'k'}
        \begin{itemize}
            \item \textbf{Explanation:} Performance heavily depends on the choice of 'k'.
            \item \textbf{Example:} A small 'k' could be noisy, while a large 'k' may smooth distinctions.
        \end{itemize}

        \item \textbf{Curse of Dimensionality}
        \begin{itemize}
            \item \textbf{Explanation:} Increased features make distance metrics less meaningful.
            \item \textbf{Example:} In high-dimensional space, all points converge, reducing efficacy.
        \end{itemize}
    \end{enumerate}
\end{frame}

\begin{frame}[fragile]
    \frametitle{Support Vector Machines (SVM)}
    \begin{block}{Definition}
        Support Vector Machines (SVM) are supervised learning models used for classification and regression tasks. The primary aim is to find the optimal hyperplane that best separates data points of different classes in a multi-dimensional space.
    \end{block}
\end{frame}

\begin{frame}[fragile]
    \frametitle{Key Concepts}
    \begin{itemize}
        \item \textbf{Hyperplane}
            \begin{itemize}
                \item A hyperplane is a flat affine subspace of one dimension less than its ambient space (line in 2D, plane in 3D).
                \item An optimal hyperplane maximizes the margin between different classes, defined by the closest data points (support vectors).
            \end{itemize}
        \item \textbf{Support Vectors}
            \begin{itemize}
                \item Support vectors are the data points closest to the hyperplane that are critical in defining its position.
                \item Removing non-support vector data points does not affect the hyperplane, but removing support vectors does.
            \end{itemize}
    \end{itemize}
\end{frame}

\begin{frame}[fragile]
    \frametitle{SVM Mechanics}
    \begin{itemize}
        \item \textbf{Separating Classes:} SVM identifies the hyperplane that maximizes the margin between two classes in the dataset.
        \item \textbf{Mathematical Representation:}
            \begin{equation}
                w^T x + b = 0
            \end{equation}
            where \( w \) is the weight vector, \( x \) is the input feature vector, and \( b \) is a bias term.
        \item \textbf{Margin Definition:}
            \begin{equation}
                \text{Margin} = \frac{2}{||w||}
            \end{equation}
            Maximizing margin improves model performance on unseen data.
    \end{itemize}
\end{frame}

\begin{frame}[fragile]
    \frametitle{Example}
    \begin{block}{Binary Classification in 2D}
        Imagine two classes:
        \begin{itemize}
            \item Class A: Points in the lower-left quadrant.
            \item Class B: Points in the upper-right quadrant.
        \end{itemize}
        By plotting these data, SVM will choose the line that maximizes the gap between the two classes, illustrating the optimal hyperplane.
    \end{block}
\end{frame}

\begin{frame}[fragile]
    \frametitle{Emphasized Points}
    \begin{itemize}
        \item \textbf{Flexibility:} SVMs can handle linear and non-linear classification (to be discussed with kernel trick on next slide).
        \item \textbf{Robustness:} Margin maximization offers robustness, especially in high-dimensional spaces.
        \item \textbf{Performance:} SVMs are effective in high-dimensional settings where the number of dimensions exceeds the number of samples.
    \end{itemize}
\end{frame}

\begin{frame}[fragile]
    \frametitle{Summary}
    \begin{block}{}
        Support Vector Machines are powerful classification tools that effectively separate classes using hyperplanes. Support vectors ensure maximized margins, enhancing performance accuracy and data handling across diverse applications.
    \end{block}
\end{frame}

\begin{frame}[fragile]
    \frametitle{SVM Kernel Trick - Overview}
    The kernel trick is a mathematical technique used in Support Vector Machines (SVM) to enable linear classification in high-dimensional feature spaces without explicitly converting data into these spaces. 
    \begin{itemize}
        \item Allows handling of non-linearly separable data effectively.
    \end{itemize}
\end{frame}

\begin{frame}[fragile]
    \frametitle{SVM Kernel Trick - What is the Kernel Trick?}
    \begin{enumerate}
        \item \textbf{Basic Concept}:
        \begin{itemize}
            \item Utilizes kernel functions to compute the dot product of two data points in a high-dimensional space without needing explicit transformation.
            \item Enables SVM to classify non-linearly separable data in its original space.
        \end{itemize}
        \item \textbf{Functionality}:
        \begin{itemize}
            \item A kernel function \( K(x_i, x_j) \) computes the inner product \( \phi(x_i) \cdot \phi(x_j) \) in the transformed feature space.
        \end{itemize}
    \end{enumerate}
\end{frame}

\begin{frame}[fragile]
    \frametitle{SVM Kernel Trick - Types of Kernels}
    \begin{enumerate}
        \item \textbf{Linear Kernel}:
        \begin{itemize}
            \item Formula: \( K(x_i, x_j) = x_i \cdot x_j \)
            \item Best for linearly separable data.
        \end{itemize}
        \item \textbf{Polynomial Kernel}:
        \begin{itemize}
            \item Formula: \( K(x_i, x_j) = (x_i \cdot x_j + c)^d \)
            \item Captures interactions up to degree \( d \).
        \end{itemize}
        \item \textbf{Radial Basis Function (RBF) Kernel}:
        \begin{itemize}
            \item Formula: \( K(x_i, x_j) = e^{-\gamma \|x_i - x_j\|^2} \)
            \item Effective for complex, non-linearly separable data.
        \end{itemize}
        \item \textbf{Sigmoid Kernel}:
        \begin{itemize}
            \item Formula: \( K(x_i, x_j) = \tanh(\alpha (x_i \cdot x_j) + c) \)
            \item Mimics neural networks and is less commonly used.
        \end{itemize}
    \end{enumerate}
\end{frame}

\begin{frame}[fragile]
    \frametitle{SVM Kernel Trick - Role and Example}
    \begin{block}{Role of the Kernel Trick in SVM}
    \begin{itemize}
        \item \textbf{Transformation Without Computation}:
        \begin{itemize}
            \item Enables high-dimensional operations without expensive transformations.
        \end{itemize}
        \item \textbf{Flexibility}:
        \begin{itemize}
            \item Different kernels allow customization for various datasets.
        \end{itemize}
        \item \textbf{Improved Performance}:
        \begin{itemize}
            \item Separates complex datasets leading to enhanced classification accuracy.
        \end{itemize}
    \end{itemize}
    \end{block}

    \textbf{Example:}
    \begin{itemize}
        \item Linearly separable data: Use a linear kernel.
        \item Non-linearly separable data: Use RBF or polynomial kernels for complex boundaries.
    \end{itemize}
\end{frame}

\begin{frame}[fragile]
    \frametitle{SVM Kernel Trick - Key Points and Conclusion}
    \begin{itemize}
        \item Kernel trick simplifies SVM implementation and enhances robustness on classification tasks.
        \item Understanding kernel functions is crucial for maximizing SVM performance.
    \end{itemize}
    
    \textbf{Conclusion:}
    \begin{itemize}
        \item The kernel trick is essential for powerful classification in high-dimensional spaces and efficient processing of complex datasets.
    \end{itemize}
\end{frame}

\begin{frame}[fragile]
    \frametitle{Advantages and Disadvantages of SVM}
    Support Vector Machines (SVM) are a powerful class of supervised learning models that can be used for classification or regression tasks.
    They function by finding the hyperplane that best separates data points of different classes in a high-dimensional space.
\end{frame}

\begin{frame}[fragile]
    \frametitle{Advantages of SVM}
    \begin{enumerate}
        \item \textbf{Effective in High-Dimension Spaces}:
            \begin{itemize}
                \item SVM performs well when dimensions exceed the number of samples.
                \item \textit{Example}: Classifying documents with thousands of unique words.
            \end{itemize}
            
        \item \textbf{Robust to Overfitting}:
            \begin{itemize}
                \item Controls overfitting using the regularization parameter (C).
                \item \textit{Key Point}: Kernel selection and regularization manage generalization.
            \end{itemize}

        \item \textbf{Flexibility with Kernel Trick}:
            \begin{itemize}
                \item Handles non-linear classification via transformation to higher dimensions.
                \item \textit{Illustration}: Circularly separable data made linear using RBF kernel.
            \end{itemize}
            
        \item \textbf{Works Well on Both Linear and Non-Linear Data}:
            \begin{itemize}
                \item Efficiently manages complex relationships with appropriate kernels.
                \item \textit{Kernel Examples}: Linear, Polynomial, RBF.
            \end{itemize}
            
        \item \textbf{Clear Margin of Separation}:
            \begin{itemize}
                \item Aims to maximize the margin between classes for better classification performance.
                \item \textit{Formula}: Margin equals the distance to the closest support vectors of each class.
            \end{itemize}
    \end{enumerate}
\end{frame}

\begin{frame}[fragile]
    \frametitle{Disadvantages of SVM}
    \begin{enumerate}
        \item \textbf{Sensitivity to Noise}:
            \begin{itemize}
                \item SVM susceptible to outliers affecting hyperplane position.
                \item \textit{Key Point}: Proper noise handling is critical.
            \end{itemize}
            
        \item \textbf{Computationally Intensive}:
            \begin{itemize}
                \item Training SVM involves complex optimization, demanding on resources.
                \item \textit{Example}: Time complexity can be \(O(n^2)\) to \(O(n^3)\).
            \end{itemize}

        \item \textbf{Limited Interpretability}:
            \begin{itemize}
                \item Decision boundaries with non-linear kernels can be complex and opaque.
                \item \textit{Key Point}: SVM's "black box" nature can impact usability in domains requiring transparency.
            \end{itemize}

        \item \textbf{Memory Usage}:
            \begin{itemize}
                \item Significant memory required when training on large datasets due to support vectors.
                \item \textit{Mitigation Strategy}: Use of stochastic gradient descent (SGD) or approximations like Linear SVM.
            \end{itemize}
    \end{enumerate}
\end{frame}

\begin{frame}[fragile]
    \frametitle{Conclusion and Code Snippet}
    \begin{block}{Conclusion}
        SVM is a powerful tool in the machine learning toolkit, particularly effective for high-dimensional and complex datasets. 
        Understanding its advantages and disadvantages is crucial for selecting the right model for classification tasks.
    \end{block}
    
    \begin{block}{Code Snippet Example (Python)}
    \begin{lstlisting}[language=Python]
from sklearn import datasets
from sklearn import svm
from sklearn.model_selection import train_test_split
from sklearn.metrics import classification_report

# Load dataset
iris = datasets.load_iris()
X, y = iris.data, iris.target

# Split into training and testing sets
X_train, X_test, y_train, y_test = train_test_split(X, y, test_size=0.3, random_state=42)

# Create a SVM classifier with RBF kernel
classifier = svm.SVC(kernel='rbf', C=1.0, gamma='scale')

# Train the classifier
classifier.fit(X_train, y_train)

# Make predictions
predictions = classifier.predict(X_test)

# Evaluate the model
print(classification_report(y_test, predictions))
    \end{lstlisting}
    \end{block}
\end{frame}

\begin{frame}
    \frametitle{Application of Classification Techniques}
    Classification techniques are methods used to predict the category or class of data points based on input features. 
    \begin{block}{Common Classification Techniques}
        \begin{itemize}
            \item Decision Trees
            \item Support Vector Machines (SVM)
            \item Naïve Bayes
            \item K-Nearest Neighbors (KNN)
            \item Neural Networks
        \end{itemize}
    \end{block}
\end{frame}

\begin{frame}
    \frametitle{Practical Application Steps - Part 1}
    Applying classification techniques involves several steps:
    \begin{enumerate}
        \item \textbf{Data Preparation}
        \begin{itemize}
            \item \textbf{Collection:} Gather relevant data.
            \item \textbf{Cleaning:} Remove duplicates and handle missing values (e.g., using imputation).
            \item \textbf{Feature Selection:} Choose relevant variables that contribute to classification.
        \end{itemize}
        \textbf{Example:} In a dataset predicting species of flowers, use features like petal length, petal width, sepal length, and sepal width.
    \end{enumerate}
\end{frame}

\begin{frame}[fragile]
    \frametitle{Practical Application Steps - Part 2}
    \begin{enumerate}[resume]
        \item \textbf{Data Splitting}
        \begin{itemize}
            \item Split the dataset into training and test sets to evaluate model performance.
            \item Common split ratio: 70\% training, 30\% testing.
        \end{itemize}
        \textbf{Code Snippet:}
        \begin{lstlisting}[language=Python]
from sklearn.model_selection import train_test_split

# Example dataset containing features 'X' and labels 'y'
X_train, X_test, y_train, y_test = train_test_split(X, y, test_size=0.3, random_state=42)
        \end{lstlisting}
    \end{enumerate}
\end{frame}

\begin{frame}[fragile]
    \frametitle{Practical Application Steps - Part 3}
    \begin{enumerate}[resume]
        \item \textbf{Model Training}
        \begin{itemize}
            \item Select a classification algorithm and train the model on the training data.
            \item Tune hyperparameters for improved performance.
        \end{itemize}
        \textbf{Example: Using SVM}
        \begin{lstlisting}[language=Python]
from sklearn.svm import SVC

model = SVC(kernel='linear')  # Linear kernel
model.fit(X_train, y_train)
        \end{lstlisting}

        \item \textbf{Model Evaluation}
        \begin{itemize}
            \item Assess the model using the test set with metrics like accuracy, precision, recall, and F1 score.
            \item Use confusion matrix to visualize performance.
        \end{itemize}
        \textbf{Confusion Matrix Example:}
        \begin{verbatim}
    Actual    Predicted
                A   B
          A   [TP  FN]
          B   [FP  TN]
        \end{verbatim}
    \end{enumerate}
\end{frame}

\begin{frame}
    \frametitle{Key Points and Conclusion}
    \begin{itemize}
        \item Importance of proper data preparation and preprocessing.
        \item Selection of the right model based on the dataset characteristics.
        \item Continuous evaluation and improvement of the model based on performance metrics.
    \end{itemize}
    \textbf{Conclusion:} Applying classification techniques involves a systematic approach from data preparation to model deployment. Understanding each step is crucial for successfully predicting outcomes and making informed decisions based on data.
\end{frame}

\begin{frame}[fragile]
    \frametitle{Case Study in Classification Techniques}
    \begin{block}{Overview}
        In this case study, we explore how classification techniques are applied in the field of healthcare to predict patient diagnoses based on historical medical records. This example illustrates the use of various classification algorithms, their effectiveness, and key outcomes.
    \end{block}
\end{frame}

\begin{frame}[fragile]
    \frametitle{Problem Statement and Dataset}
    \begin{block}{Problem Statement}
        Predict whether a patient has diabetes based on several diagnostic measurements and patient data.
    \end{block}
    
    \begin{block}{Dataset}
        \begin{itemize}
            \item \textbf{Source}: Pima Indians Diabetes Database
            \item \textbf{Features}:
            \begin{itemize}
                \item Pregnancies
                \item Glucose
                \item Blood Pressure
                \item Skin Thickness
                \item Insulin
                \item BMI
                \item Age
                \item Outcome (1 indicates diabetes; 0 indicates no diabetes)
            \end{itemize}
        \end{itemize}
    \end{block}
\end{frame}

\begin{frame}[fragile]
    \frametitle{Classification Techniques Used}
    \begin{enumerate}
        \item \textbf{Logistic Regression}
        \begin{itemize}
            \item Description: A statistical model predicting binary outcomes.
            \item Formula:
            \begin{equation}
            P(Y=1|X) = \frac{1}{1 + e^{-(\beta_0 + \beta_1X_1 + \ldots + \beta_nX_n)}}
            \end{equation}
            \item Application: Estimate the probability of diabetes based on test results.
        \end{itemize}
        
        \item \textbf{Decision Trees}
        \begin{itemize}
            \item Description: A flowchart-like structure representing tests on features.
            \item Visualization: Simplifies decision-making highlighting important feature splits.
        \end{itemize}
        
        \item \textbf{Support Vector Machine (SVM)}
        \begin{itemize}
            \item Description: Finds a hyperplane that best separates classes.
            \item Application: Classifies complex, non-linear boundaries in patient data.
        \end{itemize}
        
        \item \textbf{Random Forest}
        \begin{itemize}
            \item Description: An ensemble of decision trees to improve accuracy.
            \item Application: Combines predictions from multiple trees for more accurate diabetes predictions.
        \end{itemize}
    \end{enumerate}
\end{frame}

\begin{frame}[fragile]
    \frametitle{Key Outcomes and Evaluation Metrics}
    \begin{block}{Key Outcomes}
        \begin{itemize}
            \item \textbf{Model Performance}: Evaluated based on accuracy, precision, and recall.
            \item \textbf{Best Model}: Random Forest achieved the highest accuracy at 85%.
            \item \textbf{Confusion Matrix}:
            \begin{itemize}
                \item True Positive (TP): 90
                \item True Negative (TN): 65
                \item False Positive (FP): 10
                \item False Negative (FN): 15
            \end{itemize}
        \end{itemize}
    \end{block}
    
    \begin{block}{Evaluation Metrics}
        \begin{itemize}
            \item \textbf{Accuracy}:
            \begin{equation}
            \text{Accuracy} = \frac{TP + TN}{TP + TN + FP + FN}
            \end{equation}
            \item \textbf{Precision}:
            \begin{equation}
            \text{Precision} = \frac{TP}{TP + FP}
            \end{equation}
            \item \textbf{Recall}:
            \begin{equation}
            \text{Recall} = \frac{TP}{TP + FN}
            \end{equation}
        \end{itemize}
    \end{block}
\end{frame}

\begin{frame}[fragile]
    \frametitle{Conclusion and Key Points}
    \begin{block}{Conclusion}
        This case study highlights the efficacy of various classification techniques in a real-world setting. By utilizing different models, we can derive valuable insights into patient health and improve decision-making in healthcare.
    \end{block}
    
    \begin{block}{Key Points to Emphasize}
        \begin{itemize}
            \item Classification techniques can effectively be used in healthcare for predictive analytics.
            \item The choice of model significantly impacts predictive performance.
            \item Evaluation metrics such as accuracy, precision, and recall are essential for assessing model effectiveness.
        \end{itemize}
    \end{block}
\end{frame}

\begin{frame}[fragile]
    \frametitle{Code Snippet Example}
    \begin{lstlisting}[language=Python]
from sklearn.ensemble import RandomForestClassifier
from sklearn.model_selection import train_test_split
from sklearn.metrics import confusion_matrix, classification_report

# Load data
data = ... # Load your dataset here

# Split the data
X = data.drop('Outcome', axis=1)
y = data['Outcome']
X_train, X_test, y_train, y_test = train_test_split(X, y, test_size=0.2, random_state=42)

# Train the model
model = RandomForestClassifier()
model.fit(X_train, y_train)

# Predictions
y_pred = model.predict(X_test)

# Evaluation
print(confusion_matrix(y_test, y_pred))
print(classification_report(y_test, y_pred))
    \end{lstlisting}
\end{frame}

\begin{frame}[fragile]
    \frametitle{Model Evaluation - Introduction}
    \begin{itemize}
        \item Model evaluation is crucial in machine learning, especially for classification models.
        \item It helps assess how effectively a model performs and identifies areas for improvement.
        \item Key metrics in model evaluation include:
        \begin{itemize}
            \item Accuracy
            \item Precision
            \item Recall
            \item Confusion Matrix
        \end{itemize}
    \end{itemize}
\end{frame}

\begin{frame}[fragile]
    \frametitle{Model Evaluation - Key Metrics}
    \begin{block}{Accuracy}
        \begin{itemize}
            \item \textbf{Definition:} Proportion of correctly classified instances.
            \item \textbf{Formula:}
            \[
            \text{Accuracy} = \frac{\text{True Positives} + \text{True Negatives}}{\text{Total Instances}}
            \]
            \item \textbf{Example:} If a model predicts 80 correct out of 100 samples:
            \[
            \text{Accuracy} = \frac{80}{100} = 0.8 \text{ or } 80\%
            \]
        \end{itemize}
    \end{block}

    \begin{block}{Precision}
        \begin{itemize}
            \item \textbf{Definition:} Measures the accuracy of positive predictions.
            \item \textbf{Formula:}
            \[
            \text{Precision} = \frac{\text{True Positives}}{\text{True Positives} + \text{False Positives}}
            \]
            \item \textbf{Example:} Out of 40 positive predictions, if 30 were correct:
            \[
            \text{Precision} = \frac{30}{30 + 10} = \frac{30}{40} = 0.75 \text{ or } 75\%
            \]
        \end{itemize}
    \end{block}
\end{frame}

\begin{frame}[fragile]
    \frametitle{Model Evaluation - Recall and Confusion Matrix}
    \begin{block}{Recall}
        \begin{itemize}
            \item \textbf{Definition:} Measures the ability of a model to find all relevant cases.
            \item \textbf{Formula:}
            \[
            \text{Recall} = \frac{\text{True Positives}}{\text{True Positives} + \text{False Negatives}}
            \]
            \item \textbf{Example:} If a model correctly identifies 30 out of 50 actual positives:
            \[
            \text{Recall} = \frac{30}{30 + 20} = \frac{30}{50} = 0.6 \text{ or } 60\%
            \]
        \end{itemize}
    \end{block}

    \begin{block}{Confusion Matrix}
        \begin{itemize}
            \item \textbf{Definition:} A table to visualize the performance of a classification model.
            \item \textbf{Structure:}
            \[
            \begin{array}{|c|c|c|}
            \hline
            & \text{Predicted Positive} & \text{Predicted Negative} \\
            \hline
            \text{Actual Positive} & \text{True Positive (TP)} & \text{False Negative (FN)} \\
            \hline
            \text{Actual Negative} & \text{False Positive (FP)} & \text{True Negative (TN)} \\
            \hline
            \end{array}
            \]
            \item \textbf{Example:}
            \[
            \begin{array}{|c|c|c|}
            \hline
            & \text{Predicted Yes} & \text{Predicted No} \\
            \hline
            \text{Actual Yes} & 50 & 10 \\
            \hline
            \text{Actual No} & 5 & 35 \\
            \hline
            \end{array}
            \]
        \end{itemize}
    \end{block}
\end{frame}

\begin{frame}[fragile]
    \frametitle{Ethical Considerations - Introduction}
    In the field of data classification, ethical considerations are paramount due to the sensitive nature of the data involved and the potential impact of biased algorithms on society. This slide explores two major ethical dimensions:
    \begin{itemize}
        \item \textbf{Data Privacy}
        \item \textbf{Algorithm Bias}
    \end{itemize}
\end{frame}

\begin{frame}[fragile]
    \frametitle{Ethical Considerations - Data Privacy}
    \textbf{Definition:} Data privacy refers to the proper handling, processing, and use of personal information collected from individuals.\\
    \textbf{Importance:} Classification techniques often require access to personal data, making it critical to ensure that privacy rights are respected.

    \begin{block}{Key Points}
        \begin{itemize}
            \item \textbf{Informed Consent:} Users should be informed about how their data will be used and should consent to its usage.
            \item \textbf{Anonymization:} Techniques such as data anonymization help protect individual identities while allowing data analysis.
            \item \textbf{Regulations:} Legislation like GDPR sets strict guidelines for data usage, protecting individuals' privacy.
        \end{itemize}
    \end{block}
    
    \textbf{Example:} When developing a credit scoring algorithm, organizations must ensure that personal financial data is collected and used with explicit consent.
\end{frame}

\begin{frame}[fragile]
    \frametitle{Ethical Considerations - Algorithm Bias}
    \textbf{Definition:} Algorithm bias occurs when a classification algorithm produces results that are systematically prejudiced due to flaws in the data, coding, or algorithm design. \\
    \textbf{Impact:} Bias can lead to unfair treatment of certain groups, reinforcing existing inequalities within society.

    \begin{block}{Sources of Bias}
        \begin{itemize}
            \item \textbf{Data Representation:} Poorly represented training data can lead to skewed predictions.
            \item \textbf{Historical Bias:} Algorithms trained on historical data may perpetuate past injustices.
        \end{itemize}
    \end{block}
    
    \begin{block}{Mitigation Strategies}
        \begin{itemize}
            \item \textbf{Diverse Training Datasets:} Include diverse demographics to reduce bias.
            \item \textbf{Regular Audits:} Conduct evaluations to identify and address potential biases.
            \item \textbf{Bias Detection Tools:} Use tools like Fairness Indicators to evaluate model performance.
        \end{itemize}
    \end{block}

    \textbf{Example:} A facial recognition system tested primarily on lighter-skinned individuals may misidentify those with darker skin tones.
\end{frame}

\begin{frame}[fragile]
    \frametitle{Conclusion - Core Topics Summary}
    
    In this week, we explored \textbf{Classification Techniques}, a crucial aspect of data mining that involves predicting class labels for new observations based on past data. Key concepts include:
    
    \begin{enumerate}
        \item \textbf{Classification Algorithms}
        \begin{itemize}
            \item \textbf{Decision Trees:} Simple and interpretable models that split datasets into branches based on feature values.
            \item \textbf{Support Vector Machines (SVM):} Effective in high-dimensional spaces, identifying the best hyperplane to separate classes.
            \item \textbf{Neural Networks:} Composed of interconnected nodes, these models excel at recognizing complex patterns in large datasets.
        \end{itemize}
        
        \item \textbf{Evaluation Metrics}
        \begin{itemize}
            \item \textbf{Accuracy:} Ratio of correctly predicted instances to total instances.
            \item \textbf{Precision \& Recall:} Metrics to evaluate performance, especially with imbalanced classes.
            \item \textbf{F1 Score:} Harmonic mean of precision and recall, used for imbalanced datasets.
            \item \textbf{Confusion Matrix:} Visual tool showcasing true positives, false positives, true negatives, and false negatives.
        \end{itemize}
    \end{enumerate}

\end{frame}

\begin{frame}[fragile]
    \frametitle{Conclusion - Core Topics Summary (cont.)}
    
    \begin{enumerate}[resume]
        \item \textbf{Cross-Validation}
        \begin{itemize}
            \item A method for assessing model performance by partitioning the dataset for training and validation, aiding in reducing overfitting.
        \end{itemize}

        \item \textbf{Ethical Considerations}
        \begin{itemize}
            \item Importance of data privacy and recognizing algorithmic bias to ensure models do not reinforce social inequities.
        \end{itemize}
    \end{enumerate}
\end{frame}

\begin{frame}[fragile]
    \frametitle{Conclusion - Relevance in Data Mining}
    
    Classification techniques are fundamental for various applications in data mining, including:
    
    \begin{itemize}
        \item \textbf{Predictive Analytics:} Utilized in healthcare (disease prediction), finance (credit scoring), and marketing (customer segmentation).
        \item \textbf{Pattern Recognition:} Enhances detection and classification of patterns in large datasets, assisting in image and speech recognition.
        \item \textbf{Risk Management:} Enables organizations to predict potential risks based on historical data.
    \end{itemize}

    \textbf{Key Points to Emphasize:}
    \begin{itemize}
        \item Adaptability of algorithms depending on the dataset and problem context.
        \item Continuous assessment of model performance is crucial for accuracy and fairness.
        \item Ensuring ethical data handling is vital for public trust and compliance.
    \end{itemize}
    
    \textbf{Closing Remarks:} Understanding classification techniques empowers data scientists and drives insights for better decision-making across industries.
\end{frame}


\end{document}