\documentclass{beamer}

% Theme choice
\usetheme{Madrid} % You can change to e.g., Warsaw, Berlin, CambridgeUS, etc.

% Encoding and font
\usepackage[utf8]{inputenc}
\usepackage[T1]{fontenc}

% Graphics and tables
\usepackage{graphicx}
\usepackage{booktabs}

% Code listings
\usepackage{listings}
\lstset{
    basicstyle=\ttfamily\small,
    keywordstyle=\color{blue},
    commentstyle=\color{gray},
    stringstyle=\color{red},
    breaklines=true,
    frame=single
}

% Math packages
\usepackage{amsmath}
\usepackage{amssymb}

% Colors
\usepackage{xcolor}

% TikZ and PGFPlots
\usepackage{tikz}
\usepackage{pgfplots}
\pgfplotsset{compat=1.18}
\usetikzlibrary{positioning}

% Hyperlinks
\usepackage{hyperref}

% Title information
\title{Chapter 1: Introduction to Cryptography}
\author{Your Name}
\institute{Your Institution}
\date{\today}

\begin{document}

\frame{\titlepage}

\begin{frame}[fragile]
    \frametitle{Introduction to Cryptography - Overview}
    \begin{block}{Overview of Cryptography}
        Cryptography is the science of encoding and decoding information to secure data against unauthorized access, modification, or attacks. 
        It serves as a cornerstone of information security in today's digital world, impacting our daily communications, financial transactions, and even governmental operations.
    \end{block}
\end{frame}

\begin{frame}[fragile]
    \frametitle{Introduction to Cryptography - Key Concepts}
    \begin{enumerate}
        \item \textbf{Confidentiality}
            \begin{itemize}
                \item Ensures that data is not accessible to unauthorized viewers.
                \item \textit{Example}: Techniques such as AES or RSA to protect sensitive emails or files.
            \end{itemize}
        \item \textbf{Integrity}
            \begin{itemize}
                \item Guarantees unaltered information during transmission.
                \item \textit{Example}: Hash functions like SHA-256 for file verification.
            \end{itemize}
        \item \textbf{Authentication}
            \begin{itemize}
                \item Confirms identities of users or systems in communication.
                \item \textit{Example}: Digital signatures for message source authentication.
            \end{itemize}
        \item \textbf{Non-repudiation}
            \begin{itemize}
                \item Prevents entities from denying their actions.
                \item \textit{Example}: Digital certificates establish identities in transactions.
            \end{itemize}
    \end{enumerate}
\end{frame}

\begin{frame}[fragile]
    \frametitle{Introduction to Cryptography - Significance}
    \begin{block}{The Significance of Cryptography}
        \begin{itemize}
            \item \textbf{Data Protection}: Safeguards sensitive information against breaches by transforming it into unreadable formats.
            \item \textbf{Secure Communications}: Essential for protocols like HTTPS, enabling safe internet transactions.
            \item \textbf{Enabling Trust}: Establishes reliable communications and identity verification in the digital economy.
        \end{itemize}
    \end{block}

    \begin{block}{Illustrative Example}
        \textbf{Encryption and Decryption}
        \begin{itemize}
            \item \textit{Scenario}: Alice wants to send a secret message to Bob.
                \begin{itemize}
                    \item Alice encrypts her message using Bob's public key.
                    \item Bob decrypts it using his private key.
                \end{itemize}
        \end{itemize}
    \end{block}
\end{frame}

\begin{frame}[fragile]
    \frametitle{Historical Context - Overview}
    \begin{block}{Overview of Cryptography}
        Cryptography, the practice of securing communication and information through encoding, has a rich history spanning thousands of years. 
        Understanding its evolution helps us appreciate its pivotal role in modern security practices.
    \end{block}
\end{frame}

\begin{frame}[fragile]
    \frametitle{Historical Context - Key Developments}
    \begin{enumerate}
        \item \textbf{Ancient Cryptography (2000 BC - 500 AD)}
            \begin{itemize}
                \item Example: The \textbf{Caesar Cipher}, shifting letters in the alphabet (e.g., a shift of 3 transforms 'A' to 'D').
                \item Significance: Essential for state secrets.
            \end{itemize}

        \item \textbf{The Middle Ages (500 - 1500 AD)}
            \begin{itemize}
                \item Advanced techniques: \textbf{Substitution} and \textbf{Transposition ciphers}.
                \item Example: The \textbf{Vigenère Cipher}, utilizing keywords for complexity.
                \item Impact: Enhanced communication in warfare and politics.
            \end{itemize}

        \item \textbf{The Renaissance Era (1500 - 1700 AD)}
            \begin{itemize}
                \item Development of printing-based ciphers.
                \item Key figure: Blaise de Vigenère improved security with the Vigenère square.
            \end{itemize}
    \end{enumerate}
\end{frame}

\begin{frame}[fragile]
    \frametitle{Historical Context - Modern Developments}
    \begin{enumerate}
        \setcounter{enumi}{3}
        \item \textbf{World War Periods (20th Century)}
            \begin{itemize}
                \item Milestone: Mechanical encoding machines like the \textbf{Enigma Machine}.
                \item Impact: Codebreakers like Alan Turing were crucial in intelligence warfare.
            \end{itemize}

        \item \textbf{The Digital Age (Late 20th Century - Present)}
            \begin{itemize}
                \item Breakthrough: Introduction of public key cryptography by Whitfield Diffie and Martin Hellman in 1976.
                \item Key Concept: Public Key Infrastructure (PKI) for secure data exchange.
            \end{itemize}

        \item \textbf{Modern Cryptography}
            \begin{itemize}
                \item Algorithms: Implementation of advanced algorithms like \textbf{AES} (Advanced Encryption Standard).
                \item Example: AES encrypts data into blocks of 128 bits using keys of 128, 192, or 256 bits.
            \end{itemize}
    \end{enumerate}
\end{frame}

\begin{frame}[fragile]
    \frametitle{Historical Context - Key Points}
    \begin{block}{Key Points to Emphasize}
        - Cryptography has evolved from simple ciphers to complex algorithms vital for internet security today.
        - Key historical figures, such as Julius Caesar and Alan Turing, made significant contributions.
        - Applications range from military communications to e-commerce, showing its impact on our digital lives.
    \end{block}
\end{frame}

\begin{frame}[fragile]
    \frametitle{Illustrative Example: Caesar Cipher}
    \begin{block}{Example Encoding}
        To encode the word "HELLO" with a shift of 3:
        \begin{align*}
            H & \to K \\
            E & \to H \\
            L & \to O \\
            L & \to O \\
            O & \to R \\
        \end{align*}
        \textbf{Encoded Result}: "KHOOR"
    \end{block}
\end{frame}

\begin{frame}[fragile]
    \frametitle{Conclusion and Transition}
    \begin{block}{Conclusion}
        By studying the historical context of cryptography, we gain insights into its foundational role in secure communication methods.
        With a solid understanding of its history, we can now delve into the core concepts that define cryptography today.
    \end{block}
\end{frame}

\begin{frame}[fragile]
    \frametitle{Core Concepts of Cryptography - Introduction}
    Cryptography is a fundamental aspect of information security that ensures the protection and authenticity of data. 
    In this presentation, we will explore the four core concepts that serve as the foundation for cryptographic practices:
    \begin{itemize}
        \item \textbf{Confidentiality}
        \item \textbf{Integrity}
        \item \textbf{Authentication}
        \item \textbf{Non-repudiation}
    \end{itemize}
\end{frame}

\begin{frame}[fragile]
    \frametitle{Core Concepts of Cryptography - 1. Confidentiality}
    \begin{block}{Definition}
        Confidentiality ensures that sensitive information is accessed only by authorized individuals and kept secret from unauthorized users.
    \end{block}
    
    \begin{block}{Illustration}
        Imagine sending a letter through the postal service. If the letter is in a sealed envelope, only the intended recipient can read its contents, maintaining its confidentiality.
    \end{block}
    
    \begin{itemize}
        \item Achieved through encryption techniques, which convert plaintext into ciphertext using algorithms and keys.
        \item Common algorithms include \textbf{AES} (Advanced Encryption Standard) and \textbf{RSA} (Rivest-Shamir-Adleman).
    \end{itemize}
\end{frame}

\begin{frame}[fragile]
    \frametitle{Core Concepts of Cryptography - 2. Integrity}
    \begin{block}{Definition}
        Integrity guarantees that data has not been altered or tampered with during storage or transmission.
    \end{block}
    
    \begin{block}{Example}
        Consider a file downloaded from the internet. If the file's integrity is maintained, what you download is exactly what was uploaded by the sender.
    \end{block}
    
    \begin{itemize}
        \item Integrity is often verified using hash functions, such as \textbf{SHA-256}, which produce a unique hash value for the data.
        \item If even a single bit in the data changes, the hash will be different, signaling potential tampering.
    \end{itemize}
\end{frame}

\begin{frame}[fragile]
    \frametitle{Core Concepts of Cryptography - 3. Authentication}
    \begin{block}{Definition}
        Authentication is the process of verifying the identity of a user, device, or entity before granting access to systems or information.
    \end{block}
    
    \begin{block}{Example}
        When you log in to your email account using a username and password, the system checks if those credentials match what’s stored, confirming your identity.
    \end{block}
    
    \begin{itemize}
        \item Methods include password-based authentication, two-factor authentication (2FA), and biometrics (e.g., fingerprint or facial recognition).
        \item Authentication ensures that users are who they claim to be, preventing unauthorized access.
    \end{itemize}
\end{frame}

\begin{frame}[fragile]
    \frametitle{Core Concepts of Cryptography - 4. Non-repudiation}
    \begin{block}{Definition}
        Non-repudiation ensures that a party cannot deny the authenticity of their signature, action, or transaction in a digital environment.
    \end{block}
    
    \begin{block}{Illustration}
        If you send an email with a digital signature, you cannot later claim you did not send that email, as the signature validates your identity.
    \end{block}
    
    \begin{itemize}
        \item Frequently implemented using digital signatures and cryptographic keys.
        \item Important in legal contexts to provide proof of origin and ensure accountability.
    \end{itemize}
\end{frame}

\begin{frame}[fragile]
    \frametitle{Core Concepts of Cryptography - Summary}
    Understanding these four concepts is crucial for anyone involved in information security and cryptography. 
    They form the backbone of secure communication and data protection efforts in an increasingly digital world. 
\end{frame}

\begin{frame}[fragile]
    \frametitle{Core Concepts of Cryptography - Closing}
    Next, we will dive into \textbf{Confidentiality} and explore its importance and the methods to maintain it.
\end{frame}

\begin{frame}[fragile]
    \frametitle{Confidentiality - Introduction}
    \begin{block}{What is Confidentiality?}
        Confidentiality is the principle that ensures sensitive information is accessed only by authorized individuals. This protects personal privacy and sensitive data from unauthorized access, safeguarding against potential misuse or breaches.
    \end{block}
\end{frame}

\begin{frame}[fragile]
    \frametitle{Confidentiality - Importance}
    \begin{itemize}
        \item \textbf{Protects Sensitive Information:} Prevents fraud and identity theft by safeguarding personal details, financial records, and proprietary information.
        
        \item \textbf{Maintains Trust:} Fostering trust among clients and stakeholders is essential for business relationships and reputation.
        
        \item \textbf{Regulatory Compliance:} Adhering to regulations like GDPR and HIPAA requires strict confidentiality measures to protect data privacy.
    \end{itemize}
\end{frame}

\begin{frame}[fragile]
    \frametitle{Confidentiality - Methods to Ensure}
    \begin{enumerate}
        \item \textbf{Encryption:}
        \begin{itemize}
            \item Transforms readable data into unreadable ciphertext.
            \item \textbf{Example:} AES (Advanced Encryption Standard)
            \begin{lstlisting}
    Plaintext -> AES Encryption -> Ciphertext
            \end{lstlisting}
        \end{itemize}
        
        \item \textbf{Access Control:} 
        \begin{itemize}
            \item Implements authentication mechanisms to restrict data access.
            \item \textbf{Examples:} User IDs and passwords, Multi-factor authentication (MFA).
        \end{itemize}
        
        \item \textbf{Data Masking:}
        \begin{itemize}
            \item Replaces sensitive data with anonymized values.
            \item \textbf{Example:} "John Doe, 123-45-6789" becomes "XXXX XXX, XXX-XX-XXXX".
        \end{itemize}
        
        \item \textbf{Secure Communication Protocols:}
        \begin{itemize}
            \item Ensures encrypted data transmission over networks.
            \item \textbf{Examples:} HTTPS, SSL/TLS.
        \end{itemize}
    \end{enumerate}
\end{frame}

\begin{frame}[fragile]
    \frametitle{Confidentiality - Key Points and Summary}
    \begin{itemize}
        \item Confidentiality is a cornerstone of information security.
        
        \item Various methods exist to protect data; the right choice depends on context and requirements.
        
        \item Regular updates and audits are critical to maintain confidentiality measures.
    \end{itemize}
    
    \begin{block}{Summary}
        Understanding and implementing robust confidentiality measures helps protect sensitive information, build trust with stakeholders, and comply with legal obligations. This sets the groundwork for the next topic: integrity.
    \end{block}
    
    \begin{block}{Final Note}
        Confidentiality is not just a technical requirement; it's fundamental to building and maintaining trust in any personal or professional relationship.
    \end{block}
\end{frame}

\begin{frame}[fragile]
    \frametitle{Integrity - Introduction}
    \textbf{Introduction to Data Integrity:}
    \begin{itemize}
        \item Data integrity refers to the accuracy, consistency, and reliability of data over its lifecycle. 
        \item It ensures that information remains unchanged and uncorrupted, thus maintaining its original intended value.
    \end{itemize}
\end{frame}

\begin{frame}[fragile]
    \frametitle{Integrity - Importance}
    \textbf{Importance of Data Integrity:}
    \begin{itemize}
        \item \textbf{Trustworthiness:} Builds trust in systems and organizations through reliable information.
        \item \textbf{Compliance:} Regulations (e.g., GDPR, HIPAA) mandate data integrity to protect sensitive information.
        \item \textbf{Risk Mitigation:} Reduces the likelihood of negligence or malicious activities by ensuring data is not altered.
    \end{itemize}
\end{frame}

\begin{frame}[fragile]
    \frametitle{Integrity - Mechanisms}
    \textbf{Mechanisms to Achieve Data Integrity:}
    \begin{enumerate}
        \item \textbf{Checksums:}
        \begin{itemize}
            \item Used to verify data integrity. Example formula:
            \[
            \text{Checksum}(D) = \sum_{i=1}^{n} b_i \mod m
            \]
        \end{itemize}
        
        \item \textbf{Hash Functions:}
        \begin{itemize}
            \item Produce a fixed-size string representing the data.
            \item Examples: MD5, SHA-1, SHA-256.
            \begin{center}
                \texttt{Original Data $\rightarrow$ Hash Function $\rightarrow$ Hash Value}
            \end{center}
        \end{itemize}
        
        \item \textbf{Digital Signatures:}
        \begin{itemize}
            \item Combine hashing and asymmetric encryption for integrity and authentication.
        \end{itemize}
        
        \item \textbf{Data Redundancy:}
        \begin{itemize}
            \item Store copies across multiple locations. Techniques: RAID and backups.
        \end{itemize}
    \end{enumerate}
\end{frame}

\begin{frame}[fragile]
    \frametitle{Integrity - Conclusion}
    \textbf{Conclusion:}
    \begin{itemize}
        \item Maintaining data integrity is essential in cryptography and cybersecurity.
        \item Employing mechanisms like checksums, hash functions, digital signatures, and data redundancy helps protect data from unauthorized changes.
    \end{itemize}

    \textbf{References:}
    \begin{itemize}
        \item Stallings, W. (2017). \textit{Cryptography and Network Security: Principles and Practice}.
        \item National Institute of Standards and Technology (NIST) Guidelines on Digital Signatures.
    \end{itemize}
\end{frame}

\begin{frame}[fragile]
    \frametitle{Authentication}
    Understanding authentication, its role in security, and the different techniques used.
\end{frame}

\begin{frame}[fragile]
    \frametitle{Understanding Authentication}
    Authentication is the process of verifying the identity of a user, device, or entity in a system. 

    \begin{block}{Importance of Authentication}
        \begin{itemize}
            \item \textbf{Trusted Access}: Ensures that sensitive information can only be accessed by authorized users.
            \item \textbf{Accountability}: Provides a trail of actions taken by authenticated users, which is essential for auditing and investigating breaches.
            \item \textbf{Foundation of Security}: Acts as the first line of defense in a multi-layer security strategy.
        \end{itemize}
    \end{block}
\end{frame}

\begin{frame}[fragile]
    \frametitle{Techniques of Authentication}
    Several techniques are used to achieve authentication:

    \begin{enumerate}
        \item \textbf{Password-Based Authentication} 
        \begin{itemize}
            \item \textit{Description}: Users provide a secret password to gain access.
            \item \textit{Example}: Logging into your email account using a password.
            \item \textit{Consideration}: Must ensure strong password policies (e.g., length, complexity).
        \end{itemize}

        \item \textbf{Two-Factor Authentication (2FA)}
        \begin{itemize}
            \item \textit{Description}: Requires two forms of verification, combining something the user knows (password) with something they possess (like a smartphone).
            \item \textit{Example}: Receiving a text message with a code after entering your password.
            \item \textit{Benefit}: Adds an additional layer of security beyond just a password.
        \end{itemize}
        
        \item \textbf{Biometric Authentication} 
        \begin{itemize}
            \item \textit{Description}: Uses unique biological attributes of a user, such as fingerprints or facial recognition.
            \item \textit{Example}: Unlocking a smartphone using your fingerprint.
            \item \textit{Advantage}: Difficult for others to replicate.
        \end{itemize}
    \end{enumerate}
\end{frame}

\begin{frame}[fragile]
    \frametitle{Further Techniques of Authentication}
    Continuing with additional techniques:

    \begin{enumerate}
        \setcounter{enumi}{3} % Continue enumerating from previous frame
        \item \textbf{Token-Based Authentication} 
        \begin{itemize}
            \item \textit{Description}: Users are issued a token, a device or software-generated code that must be presented for access.
            \item \textit{Example}: One-Time Password (OTP) generated by an app.
            \item \textit{Security Note}: Tokens can enhance security but require management.
        \end{itemize}

        \item \textbf{Public Key Infrastructure (PKI)} 
        \begin{itemize}
            \item \textit{Description}: Uses cryptographic pairs of public and private keys for validation.
            \item \textit{Example}: Secure email communication using digital signatures.
            \item \textit{Strength}: Provides robust security due to asymmetric encryption.
        \end{itemize}
    \end{enumerate}
\end{frame}

\begin{frame}[fragile]
    \frametitle{Key Points and Conclusion}
    \begin{block}{Key Points to Emphasize}
        \begin{itemize}
            \item \textbf{Usability vs. Security}: Striking a balance between ease of access and maintaining security.
            \item \textbf{Regular Updates and Audits}: Authentication methods require regular reviews to adapt to evolving threats.
            \item \textbf{User Education}: Teaching users about recognizing phishing attempts and maintaining secure passwords can greatly enhance overall security.
        \end{itemize}
    \end{block}

    \begin{block}{Conclusion}
        Authentication is crucial in validating identities and securing access to sensitive information. Understanding and implementing the right authentication techniques can significantly reduce the risk of unauthorized access to systems and data.
    \end{block}
\end{frame}

\begin{frame}[fragile]
    \frametitle{Non-repudiation - Definition}
    \begin{block}{Definition}
        \textbf{Non-repudiation} is a crucial concept in digital communications that ensures that a party in a transaction cannot deny the authenticity of their signature or the sending of a message. This provides proof of the integrity and origin of data, ensuring accountability of the sender's actions.
    \end{block}
\end{frame}

\begin{frame}[fragile]
    \frametitle{Non-repudiation - Significance}
    \begin{itemize}
        \item \textbf{Accountability:} Establishes trust in digital communications by holding senders accountable for their actions.
        \item \textbf{Legal Proof:} Protects against fraudulent claims and provides concrete evidence in case of disputes.
        \item \textbf{Fraud Prevention:} Deters malicious actions by ensuring that parties cannot deny involvement in transactions.
    \end{itemize}
\end{frame}

\begin{frame}[fragile]
    \frametitle{Non-repudiation - Mechanisms}
    \begin{itemize}
        \item \textbf{Digital Signatures:} 
            \begin{itemize}
                \item Unique cryptographic value generated by the sender using their private key.
                \item \textbf{Example:} If Alice sends a contract to Bob and digitally signs it, Alice cannot claim she did not send it; Bob can verify the signature.
            \end{itemize}
        \item \textbf{Timestamps:}
            \begin{itemize}
                \item Provides evidence of when communication occurred.
                \item \textbf{Example:} An online transaction with a timestamp is crucial to proving timing in disputes.
            \end{itemize}
    \end{itemize}
\end{frame}

\begin{frame}[fragile]
    \frametitle{Non-repudiation - Example Process}
    \begin{enumerate}
        \item Alice wishes to send a confidential document to Bob.
        \item Alice uses her private key to create a digital signature for the document.
        \item The signed document is sent to Bob along with Alice's public key.
        \item Bob verifies the signature with Alice's public key, confirming authenticity and integrity of the document.
    \end{enumerate}
\end{frame}

\begin{frame}[fragile]
    \frametitle{Non-repudiation - Summary}
    Non-repudiation is vital in digital communications by ensuring accountability and trust. It is accomplished through cryptographic means, establishing a secure foundation for online transactions in various domains such as e-commerce and online banking.
\end{frame}

\begin{frame}[fragile]
    \frametitle{Types of Cryptographic Algorithms}
    \begin{block}{Overview}
        Cryptography is essential for securing communications and data. There are three main types of cryptographic algorithms:
    \end{block}
    \begin{itemize}
        \item Symmetric Key Algorithms
        \item Asymmetric Key Algorithms
        \item Hash Functions
    \end{itemize}
    Each serves different purposes and is applicable in various scenarios.
\end{frame}

\begin{frame}[fragile]
    \frametitle{Symmetric Key Algorithms}
    \begin{block}{Definition}
        Symmetric algorithms use the same key for both encryption and decryption. Both parties must possess the secret key and keep it confidential.
    \end{block}
    \begin{itemize}
        \item \textbf{Examples}:
        \begin{itemize}
            \item AES (Advanced Encryption Standard)
            \item DES (Data Encryption Standard)
        \end{itemize}
        \item \textbf{Applications}:
        \begin{itemize}
            \item Data encryption in databases
            \item VPNs (Virtual Private Networks)
        \end{itemize}
        \item \textbf{Key Point}: 
        Symmetric algorithms are faster and less computationally intensive, making them suitable for large data volumes.
    \end{itemize}
\end{frame}

\begin{frame}[fragile]
    \frametitle{Asymmetric Key Algorithms}
    \begin{block}{Definition}
        Asymmetric algorithms use two keys: a public key for encryption and a private key for decryption. The public key can be shared openly, while the private key must remain secret.
    \end{block}
    \begin{itemize}
        \item \textbf{Examples}:
        \begin{itemize}
            \item RSA (Rivest-Shamir-Adleman)
            \item ECC (Elliptic Curve Cryptography)
        \end{itemize}
        \item \textbf{Applications}:
        \begin{itemize}
            \item Digital signatures
            \item Key exchanges in protocols (e.g., SSL/TLS)
        \end{itemize}
        \item \textbf{Key Point}: 
        While slower than symmetric algorithms, they provide secure key distribution essential for modern secure communications.
    \end{itemize}
\end{frame}

\begin{frame}[fragile]
    \frametitle{Hash Functions}
    \begin{block}{Definition}
        Hash functions take an input and return a fixed-size string of bytes. They are one-way; the original data cannot be reconstructed from the hash.
    \end{block}
    \begin{itemize}
        \item \textbf{Examples}:
        \begin{itemize}
            \item SHA-256 (Secure Hash Algorithm 256-bit)
            \item MD5 (Message Digest 5)
        \end{itemize}
        \item \textbf{Applications}:
        \begin{itemize}
            \item Data integrity checks
            \item Password storage (storing hashes instead of passwords)
        \end{itemize}
        \item \textbf{Key Point}: 
        Hash functions enhance data integrity and detect any data modification easily.
    \end{itemize}
\end{frame}

\begin{frame}[fragile]
    \frametitle{Summary of Key Points}
    \begin{itemize}
        \item \textbf{Symmetric Key}:
        \begin{itemize}
            \item Same key for encryption/decryption
            \item Fast, used for bulk data encryption
        \end{itemize}
        \item \textbf{Asymmetric Key}:
        \begin{itemize}
            \item Pair of keys (public/private)
            \item Secure key distribution, used for emails and certificates
        \end{itemize}
        \item \textbf{Hash Functions}:
        \begin{itemize}
            \item Produce a unique fixed-size output
            \item Ensure data integrity; one-way function
        \end{itemize}
    \end{itemize}
\end{frame}

\begin{frame}[fragile]
    \frametitle{Key Cryptographic Protocols}
    Cryptographic protocols are essential for secure communication over networks. 
    They utilize cryptographic algorithms to ensure:
    \begin{itemize}
        \item Confidentiality
        \item Integrity
        \item Authentication
    \end{itemize}
    Key protocols include:
    \begin{itemize}
        \item TLS/SSL (Transport Layer Security / Secure Sockets Layer)
        \item IPsec (Internet Protocol Security)
    \end{itemize}
\end{frame}

\begin{frame}[fragile]
    \frametitle{TLS/SSL (Transport Layer Security / Secure Sockets Layer)}
    \textbf{Definition}: 
    TLS is the modern protocol that succeeded SSL for secure communication.

    \textbf{Key Roles}:
    \begin{itemize}
        \item \textbf{Encryption}: Protects data by converting plaintext into ciphertext.
        \item \textbf{Authentication}: Verifies identity using digital certificates.
        \item \textbf{Data Integrity}: Uses MAC to ensure data has not been altered.
    \end{itemize}

    \textbf{Process Overview}:
    \begin{enumerate}
        \item Handshake establishes a secure connection.
        \item Session Key Generation creates a temporary symmetric key.
        \item Secure Communication enables encrypted data transfer.
    \end{enumerate}
\end{frame}

\begin{frame}[fragile]
    \frametitle{IPsec (Internet Protocol Security)}
    \textbf{Definition}: 
    IPsec secures IP communications by authenticating and encrypting each packet.

    \textbf{Key Roles}:
    \begin{itemize}
        \item \textbf{Confidentiality}: Encrypts data packets.
        \item \textbf{Integrity}: Ensures packets are unaltered during transit.
        \item \textbf{Authentication}: Verifies the sender's identity.
    \end{itemize}

    \textbf{Modes of Operation}:
    \begin{itemize}
        \item \textbf{Transport Mode}: Encrypts only the payload.
        \item \textbf{Tunnel Mode}: Encapsulates entire packets, used for VPNs.
    \end{itemize}
    
    \textbf{Example}: 
    IPsec is commonly utilized for establishing secure VPNs.
\end{frame}

\begin{frame}[fragile]
    \frametitle{Conclusion and Future Trends - Summary of Key Points}
    \begin{itemize}
        \item \textbf{Definition of Cryptography:} Securing information by transforming it into an unreadable format to protect against unauthorized access.
        \item \textbf{Importance of Secure Communication:} Protocols like TLS/SSL and IPsec ensure data confidentiality, integrity, and authentication.
    \end{itemize}
    
    \begin{block}{Key Concepts}
        \begin{itemize}
            \item \textbf{Symmetric Encryption:} Single key for both encryption and decryption (e.g., AES).
            \item \textbf{Asymmetric Encryption:} Pair of keys—public and private—for encryption and decryption (e.g., RSA).
            \item \textbf{Hash Functions:} Produce a fixed-size string from variable-sized input to ensure data integrity (e.g., SHA-256).
        \end{itemize}
    \end{block}
\end{frame}

\begin{frame}[fragile]
    \frametitle{Conclusion and Future Trends - Emerging Trends in Cryptography}
    \begin{enumerate}
        \item \textbf{Quantum Cryptography}
            \begin{itemize}
                \item \textbf{Overview:} Leverages quantum mechanics for theoretically secure communication systems.
                \item \textbf{Example:} Quantum Key Distribution (QKD) alerts parties if an eavesdropper tries to intercept the key.
            \end{itemize}
        
        \item \textbf{Post-Quantum Cryptography}
            \begin{itemize}
                \item \textbf{Need for Transition:} Traditional algorithms risk vulnerability with quantum computing advancements.
                \item \textbf{Ongoing Research:} NIST is working on standardizing post-quantum algorithms.
            \end{itemize}
        
        \item \textbf{Homomorphic Encryption}
            \begin{itemize}
                \item \textbf{What It Is:} Enables computations on encrypted data without decryption.
                \item \textbf{Use Case Example:} Querying encrypted patient records without exposing sensitive data.
            \end{itemize}
    \end{enumerate}
\end{frame}

\begin{frame}[fragile]
    \frametitle{Conclusion and Future Trends - Key Concepts and Conclusion}
    \begin{block}{Key Points to Emphasize}
        \begin{itemize}
            \item Evolving technology requires updated cryptographic practices to counter new threats.
            \item Collaboration among computer scientists, mathematicians, and cybersecurity experts is essential.
            \item Understanding cryptographic principles is not just technical; it’s about fostering secure communication environments.
        \end{itemize}
    \end{block}
    
    \begin{block}{Notable Formulas}
        \begin{itemize}
            \item \textbf{Symmetric Key Encryption:} 
            \begin{equation}
                C = E(K, P)
            \end{equation}
            \item \textbf{Asymmetric Key Encryption:} 
            \begin{equation}
                C = E(PubK, M) \quad \text{and} \quad M = D(PrivK, C)
            \end{equation}
        \end{itemize}
    \end{block}

    \textbf{Conclusion:} The future of cryptography holds exciting innovations for data protection in a rapidly changing technological landscape, highlighting the need for early understanding of these trends in cybersecurity.
\end{frame}


\end{document}