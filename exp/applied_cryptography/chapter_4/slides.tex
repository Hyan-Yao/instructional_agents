\documentclass{beamer}

% Theme choice
\usetheme{Madrid} % You can change to e.g., Warsaw, Berlin, CambridgeUS, etc.

% Encoding and font
\usepackage[utf8]{inputenc}
\usepackage[T1]{fontenc}

% Graphics and tables
\usepackage{graphicx}
\usepackage{booktabs}

% Code listings
\usepackage{listings}
\lstset{
basicstyle=\ttfamily\small,
keywordstyle=\color{blue},
commentstyle=\color{gray},
stringstyle=\color{red},
breaklines=true,
frame=single
}

% Math packages
\usepackage{amsmath}
\usepackage{amssymb}

% Colors
\usepackage{xcolor}

% TikZ and PGFPlots
\usepackage{tikz}
\usepackage{pgfplots}
\pgfplotsset{compat=1.18}
\usetikzlibrary{positioning}

% Hyperlinks
\usepackage{hyperref}

% Title information
\title{Chapter 4: Cryptographic Hash Functions}
\author{Your Name}
\institute{Your Institution}
\date{\today}

\begin{document}

\frame{\titlepage}

\begin{frame}[fragile]
    \frametitle{Introduction to Cryptographic Hash Functions}
    \begin{block}{What Are Cryptographic Hash Functions?}
        Cryptographic hash functions are specialized algorithms that transform input data of any length into a fixed-size string of characters, typically a hexadecimal number. The output, known as the \textbf{hash} or \textbf{digest}, plays a crucial role in various security applications.
    \end{block}
\end{frame}

\begin{frame}[fragile]
    \frametitle{Purpose and Importance}
    \begin{enumerate}
        \item \textbf{Data Integrity:}
            \begin{itemize}
                \item Ensures that any alteration of input data will result in a different hash.
                \item \textit{Example:} Verifying downloaded software by comparing provided hash with the computed hash.
            \end{itemize}
        \item \textbf{Authentication:}
            \begin{itemize}
                \item Used in password storage by hashing passwords.
                \item \textit{Example:} Hashing "mypassword" keeps it confidential.
            \end{itemize}
        \item \textbf{Digital Signatures and Certificates:}
            \begin{itemize}
                \item Ensures documents have not been altered and binds identity to a document.
            \end{itemize}
        \item \textbf{Efficiency:}
            \begin{itemize}
                \item Fast computation of hashes is crucial for applications like blockchain.
            \end{itemize}
    \end{enumerate}
\end{frame}

\begin{frame}[fragile]
    \frametitle{Key Characteristics}
    \begin{itemize}
        \item \textbf{Deterministic:} Same input yields the same hash output.
        \item \textbf{Fixed Output Length:} Output is of consistent length (e.g., 256 bits for SHA-256).
        \item \textbf{Pre-image Resistance:} Infeasible to reverse-engineer the original input from the hash.
        \item \textbf{Collision Resistance:} Challenging to find two different inputs that produce the same hash.
        \item \textbf{Avalanche Effect:} Small change in input results in drastically different hash output.
    \end{itemize}
\end{frame}

\begin{frame}[fragile]
    \frametitle{Popular Cryptographic Hash Functions}
    \begin{itemize}
        \item \textbf{SHA-256:} Widely used in security applications (e.g., Bitcoin).
        \item \textbf{SHA-3:} Latest member of the Secure Hash Algorithm family with enhanced security features.
        \item \textbf{MD5:} Previously popular, now insecure due to vulnerabilities.
    \end{itemize}
\end{frame}

\begin{frame}[fragile]
    \frametitle{Conclusion}
    Cryptographic hash functions are foundational for modern cybersecurity. They enable secure data storage, verification, and integrity checks, making knowledge of their properties crucial for safeguarding data in a digital world.
\end{frame}

\begin{frame}[fragile]
    \frametitle{What are Hash Functions?}
    % Definition of hash functions
    \begin{block}{Definition}
        Hash functions are mathematical algorithms that transform input data (or 'message') into a fixed-size string of characters, typically a hexadecimal number. This output is known as the hash value or digest.
    \end{block}
\end{frame}

\begin{frame}[fragile]
    \frametitle{Key Characteristics of Hash Functions}
    % Characteristics of hash functions
    \begin{enumerate}
        \item \textbf{Determinism:}
        \begin{itemize}
            \item A hash function consistently produces the same hash value for the same input.
            \item Example: Hashing "Hello, World!" yields the output \\ \texttt{65a8e27d8879283831b664bd8b7f0ad4}.
        \end{itemize}

        \item \textbf{Fixed Output Length:}
        \begin{itemize}
            \item The hash value remains of a fixed length regardless of input size.
            \item Example: Both "abc" and an entire book can yield hash values of 256 bits (e.g., SHA-256).
        \end{itemize}

        \item \textbf{Computational Efficiency:}
        \begin{itemize}
            \item Fast computation of hashes is crucial for applications like verifying data integrity.
            \item Example: Quickly hashing a document in a digital signature process.
        \end{itemize}
    \end{enumerate}
\end{frame}

\begin{frame}[fragile]
    \frametitle{Illustrative Example}
    % Example of hash function
    Consider the hash function SHA-256:
    \begin{block}{Input and Result}
        \begin{itemize}
            \item \textbf{Input Data:} "data123"
            \item \textbf{Resulting Hash:} \texttt{6f7c5b882d0a138e4f6fdd64e5400270adfef0e1cd2956c7f7e5c7e24fc60012}
        \end{itemize}
    \end{block}
    
    \begin{block}{Conclusion}
        Hash functions are critical in security applications, essential for understanding advanced cryptographic concepts.
    \end{block}
\end{frame}

\begin{frame}[fragile]
    \frametitle{Properties of Hash Functions - Introduction}
    \begin{block}{Key Properties of Hash Functions}
        Cryptographic hash functions are designed to take an input (or message) and produce a fixed-size string of bytes, typically a digest that appears random. For hash functions to be secure and effective, they must possess several crucial properties.
    \end{block}
\end{frame}

\begin{frame}[fragile]
    \frametitle{Properties of Hash Functions - Pre-image Resistance}
    \begin{enumerate}
        \item \textbf{Pre-image Resistance}
        \begin{itemize}
            \item \textbf{Definition:} Given a hash output \( h \), it should be computationally infeasible to find an input \( x \) such that \( \text{hash}(x) = h \).
            \item \textbf{Explanation:} If someone knows the hash value, they cannot easily reverse-engineer it to discover the original input.
            \item \textbf{Example:} If the hash of a password is stored, pre-image resistance ensures that knowing the hash alone does not allow easy recovery of the password.
        \end{itemize}
    \end{enumerate}
\end{frame}

\begin{frame}[fragile]
    \frametitle{Properties of Hash Functions - Second Pre-image Resistance and Collision Resistance}
    \begin{enumerate}
        \setcounter{enumi}{1}
        \item \textbf{Second Pre-image Resistance}
        \begin{itemize}
            \item \textbf{Definition:} Given an input \( x \) and its hash \( h \), it should be hard to find another input \( x' \) (where \( x' \neq x \)) such that \( \text{hash}(x') = h \).
            \item \textbf{Explanation:} Prevents the creation of a different file that hashes to the same value, thus preserving authenticity.
            \item \textbf{Example:} If a user signs a document, crafting a different document with the same hash value undermines the user's signature.
        \end{itemize}

        \item \textbf{Collision Resistance}
        \begin{itemize}
            \item \textbf{Definition:} It should be computationally infeasible to find any two distinct inputs \( x \) and \( y \) such that \( \text{hash}(x) = \text{hash}(y) \).
            \item \textbf{Explanation:} Collision resistance ensures that no two different messages have the same hash output.
            \item \textbf{Example:} If two transactions could yield the same hash, a fraudster could create a valid transaction proving the same funds were spent twice.
        \end{itemize}
    \end{enumerate}
\end{frame}

\begin{frame}[fragile]
    \frametitle{Properties of Hash Functions - Avalanche Effect}
    \begin{enumerate}
        \setcounter{enumi}{3}
        \item \textbf{Avalanche Effect}
        \begin{itemize}
            \item \textbf{Definition:} A small change in the input (even changing one bit) should result in a drastic change in the output hash.
            \item \textbf{Explanation:} Enhances security by making it difficult to predict output changes with minor input variations.
            \item \textbf{Example:} Hashing `abc` and `abc1` should yield drastically different results, demonstrating the property.
        \end{itemize}
    \end{enumerate}
\end{frame}

\begin{frame}[fragile]
    \frametitle{Key Points and Applications of Hash Functions}
    \begin{itemize}
        \item These properties are foundational for ensuring the security and reliability of cryptographic systems.
        \item Cryptographic hash functions are widely used in:
        \begin{itemize}
            \item Digital signatures
            \item Data integrity verification
            \item Password hashing
        \end{itemize}
        \item Real-world applications depend heavily on these properties to function correctly and securely.
    \end{itemize}
\end{frame}

\begin{frame}[fragile]
    \frametitle{The SHA Family of Algorithms - Overview}
    % Overview of the SHA family
    The Secure Hash Algorithm (SHA) family, developed by the National Security Agency (NSA), is a collection of cryptographic hash functions that play a critical role in data integrity and security. 
    \begin{itemize}
        \item Transforms input data (of any size) into a fixed-size output (the hash).
        \item Even minor changes in input produce significantly different outputs.
    \end{itemize}
\end{frame}

\begin{frame}[fragile]
    \frametitle{The SHA Family - Key Variants}
    % Key SHA variants
    \begin{enumerate}
        \item \textbf{SHA-1}
            \begin{itemize}
                \item Length: 160-bit hash value.
                \item Usage: Previously used for data integrity and digital signatures.
                \item Current Status: Weakened by vulnerabilities (collision attacks).
                \item Example: Input ``Hello'' yields a SHA-1 hash of \texttt{f5721d4...}.
            \end{itemize}
        
        \item \textbf{SHA-2}
            \begin{itemize}
                \item Sub-variants: SHA-224, SHA-256, SHA-384, SHA-512.
                \item Example: Input ``Hello'' yields a SHA-256 hash of \texttt{2cf24d...}.
                \item Strengths: More secure than SHA-1 with better collision resistance.
                \item Usage: Commonly used in security protocols like SSL/TLS.
            \end{itemize}
        
        \item \textbf{SHA-3}
            \begin{itemize}
                \item Introduced as an alternative to SHA-2 in 2012.
                \item Flexible output length (224, 256, 384, 512 bits).
                \item Architecture: Based on the Keccak sponge construction.
                \item Example: Input ``Hello'' yields a SHA-3 hash of \texttt{7c211...}.
            \end{itemize}
    \end{enumerate}
\end{frame}

\begin{frame}[fragile]
    \frametitle{Importance of the SHA Family}
    % Key characteristics and importance of SHA
    \begin{block}{Key Characteristics}
        \begin{itemize}
            \item Pre-image Resistance: Infeasible to retrieve input from hash output.
            \item Collision Resistance: Unlikely for different inputs to produce the same output.
            \item Avalanche Effect: Small changes in input result in drastic output changes.
        \end{itemize}
    \end{block}

    \begin{block}{Importance}
        \begin{itemize}
            \item Integrity Verification: Critical for confirming data integrity, used in software distribution and digital signatures.
            \item Security Protocols: Essential for SSL/TLS protocols safeguarding web communications.
        \end{itemize}
    \end{block}

    \begin{block}{Conclusion}
        Understanding the SHA algorithms is crucial for ensuring data integrity and secure communication, especially as the landscape of cybersecurity evolves.
    \end{block}
\end{frame}

\begin{frame}[fragile]
    \frametitle{SHA-1: Strengths and Vulnerabilities - Introduction}
    % Introduction to SHA-1
    \begin{itemize}
        \item \textbf{SHA-1} (Secure Hash Algorithm 1)
        \begin{itemize}
            \item Developed by the National Security Agency (NSA) and published by NIST in 1995.
            \item Produces a 160-bit hash value.
            \item Used in various security applications and protocols, including TLS, PGP, and SSH.
        \end{itemize}
    \end{itemize}
\end{frame}

\begin{frame}[fragile]
    \frametitle{SHA-1: Strengths and Applications}
    % Strengths and Applications of SHA-1
    \begin{block}{Strengths of SHA-1}
        \begin{enumerate}
            \item \textbf{Standardized Algorithm}: Widely adopted and trusted.
            \item \textbf{Speed}: Fast and efficient, suitable for high-performance systems.
            \item \textbf{Simplicity}: Easy to implement for developers.
        \end{enumerate}
    \end{block}

    \begin{block}{Applications of SHA-1}
        \begin{itemize}
            \item Digital signatures for integrity and authenticity.
            \item Version control systems (e.g., Git) for data consistency.
            \item Used in TLS certificates by early certificate authorities.
        \end{itemize}
    \end{block}
\end{frame}

\begin{frame}[fragile]
    \frametitle{SHA-1: Vulnerabilities and Summary}
    % Known vulnerabilities and summary
    \begin{block}{Known Vulnerabilities}
        \begin{enumerate}
            \item \textbf{Collision Attacks}:
                \begin{itemize}
                    \item Practical collisions demonstrated in 2005.
                    \item "SHAttered" attack in 2017 - two distinct files, same hash.
                \end{itemize}
            \item \textbf{Security Level}:
                \begin{itemize}
                    \item Reduced hash strength to approximately 63 bits.
                    \item Increased feasibility for brute force attacks.
                \end{itemize}
            \item \textbf{Deprecation}:
                \begin{itemize}
                    \item Major tech entities moved to stronger alternatives due to vulnerabilities.
                \end{itemize}
        \end{enumerate}
    \end{block}

    \begin{block}{Summary}
        SHA-1 played a vital role but is now obsolete due to vulnerabilities; organizations should transition to more secure algorithms to protect information integrity.
    \end{block}
\end{frame}

\begin{frame}[fragile]
    \frametitle{SHA-256 and SHA-3 - Overview}
    % Overview of SHA-256, SHA-3, and their importance
    In this slide, we'll explore SHA-256, a prominent member of the SHA-2 family, and SHA-3, the latest addition to the Secure Hash Algorithm suite. 
    Both algorithms serve crucial roles in maintaining data integrity and secure communications across various applications.
\end{frame}

\begin{frame}[fragile]
    \frametitle{SHA-256 (Secure Hash Algorithm 256-bit)}
    % Overview of SHA-256 definition, security features, and applications
    \begin{block}{Definition}
        SHA-256 is a cryptographic hash function that produces a 256-bit (32-byte) hash value. 
        It is part of the SHA-2 family, designed by the National Security Agency (NSA) to replace SHA-1.
    \end{block}

    \begin{itemize}
        \item \textbf{Security Features}:
            \begin{itemize}
                \item \textit{Collision Resistance}: Computationally infeasible to find two distinct inputs yielding the same output.
                \item \textit{Pre-image Resistance}: Hard to find an original input from a hash value.
                \item \textit{Second Pre-image Resistance}: Difficult to find a different input that matches a known hash.
            \end{itemize}
        
        \item \textbf{Practical Applications}:
            \begin{itemize}
                \item Digital Signatures for verifying authenticity in certificates.
                \item Blockchain technology to ensure transaction integrity.
            \end{itemize}
    \end{itemize}

\end{frame}

\begin{frame}[fragile]
    \frametitle{SHA-256 Hash Calculation Example}
    % Code snippet for SHA-256 hash calculation
    \begin{block}{Hash Calculation Example}
        \begin{lstlisting}[language=Python]
import hashlib

message = "Hello, World!"
sha256_hash = hashlib.sha256(message.encode()).hexdigest()
print(f"SHA-256 Hash: {sha256_hash}")
        \end{lstlisting}
        % Output example
        *Output*: SHA-256 Hash: `315f5bdb76d084c0c9b11e0f600bfb0a80b866c8e9186e8eb0b908774139f30f`
    \end{block}
\end{frame}

\begin{frame}[fragile]
    \frametitle{SHA-3 (Secure Hash Algorithm 3)}
    % Overview of SHA-3 definition, security features, and practical applications
    \begin{block}{Definition}
        SHA-3 is the third generation of the Secure Hash Algorithm family, standardized in 2015. 
        Unlike SHA-2, SHA-3 is built on the Keccak sponge construction.
    \end{block}

    \begin{itemize}
        \item \textbf{Security Features}:
            \begin{itemize}
                \item Offers high security against collision and pre-image attacks.
                \item \textit{Flexibility}: Supports variable output lengths (224, 256, 384, and 512 bits).
            \end{itemize}

        \item \textbf{Practical Applications}:
            \begin{itemize}
                \item Securing messaging, file integrity checks, and blockchain.
                \item Enhanced resistance against quantum attacks, making it more future-proof.
            \end{itemize}
    \end{itemize}

\end{frame}

\begin{frame}[fragile]
    \frametitle{SHA-3 Hash Calculation Example}
    % Code snippet for SHA-3 hash calculation
    \begin{block}{Hash Calculation Example}
        \begin{lstlisting}[language=Python]
import hashlib

message = "Hello, World!"
sha3_hash = hashlib.sha3_256(message.encode()).hexdigest()
print(f"SHA-3 Hash: {sha3_hash}")
        \end{lstlisting}
        % Output example
        *Output*: SHA-3 Hash: `a5b47e9dbdd6b496e9d3aa68500214c9a61a605bdf27e6cb3c34e87b8a5773bc`
    \end{block}
\end{frame}

\begin{frame}[fragile]
    \frametitle{Key Points and Conclusion}
    % Summary of key points and conclusion
    \begin{itemize}
        \item Both SHA-256 and SHA-3 offer strong security guarantees essential for today's digital ecosystem.
        \item SHA-256 is widely used but transitioning towards SHA-3 is recommended due to its enhanced design and flexibility.
        \item The choice between SHA-2 and SHA-3 depends on specific application requirements, such as output size and quantum threat resilience.
    \end{itemize}

    \begin{block}{Conclusion}
        Understanding SHA-256 and SHA-3 enhances our grasp of cryptographic principles. 
        These algorithms play significant roles in maintaining data integrity and security across digital platforms.
    \end{block}
\end{frame}

\begin{frame}[fragile]
    \frametitle{Applications of Cryptographic Hash Functions - Introduction}
    % Introduction to the role of cryptographic hash functions
    Cryptographic hash functions are vital in modern cybersecurity, converting input data into a fixed-size string that appears random. This property is essential for various applications that ensure data security.
\end{frame}

\begin{frame}[fragile]
    \frametitle{Applications of Cryptographic Hash Functions - Key Applications}
    % Key applications of cryptographic hash functions
    \begin{itemize}
        \item Data Integrity Verification
        \item Digital Signatures
        \item Password Hashing
    \end{itemize}
\end{frame}

\begin{frame}[fragile]
    \frametitle{Data Integrity Verification}
    \begin{block}{Definition}
        Ensures that data remains unchanged during storage or transmission.
    \end{block}
    \begin{itemize}
        \item When data is created, a hash is computed and stored alongside the data.
        \item Upon accessing the data later, the stored hash is compared to the recomputed hash.
    \end{itemize}
    \begin{exampleblock}{Example}
        A software download site might provide a hash value. Users hash the downloaded file to check it matches the provided hash.
    \end{exampleblock}
\end{frame}

\begin{frame}[fragile]
    \frametitle{Digital Signatures}
    \begin{block}{Definition}
        A digital signature uses hash functions to ensure authenticity and integrity of a message.
    \end{block}
    \begin{enumerate}
        \item A sender computes the hash of the message.
        \item The hash is encrypted with the sender's private key to create a digital signature.
        \item The receiver decrypts the signature with the sender's public key and verifies it against the message hash.
    \end{enumerate}
    \begin{exampleblock}{Example}
        Email clients use digital signatures to confirm that email content remains unaltered in transit.
    \end{exampleblock}
\end{frame}

\begin{frame}[fragile]
    \frametitle{Password Hashing}
    \begin{block}{Definition}
        Storing passwords securely to protect user accounts.
    \end{block}
    \begin{itemize}
        \item Systems hash user passwords instead of storing them in plaintext.
        \item On login, the entered password is hashed and compared to the stored hash.
    \end{itemize}
    \begin{exampleblock}{Example}
        If a user sets the password "SecurePass123", the system computes its hash and stores it instead of the plaintext password.
    \end{exampleblock}
    \begin{block}{Key Point}
        Use strong hashing algorithms and include a salt to prevent rainbow table attacks.
    \end{block}
\end{frame}

\begin{frame}[fragile]
    \frametitle{Summary and Conclusion}
    % Summary of applications and importance of cryptographic hash functions
    Cryptographic hash functions are essential for:
    \begin{itemize}
        \item Ensuring data integrity by verifying unchanged information.
        \item Creating digital signatures that authenticate sources and protect content.
        \item Safeguarding passwords from unauthorized access.
    \end{itemize}
    
    In our digital world, understanding these applications is crucial for data security professionals to protect information and verify its integrity.
\end{frame}

\begin{frame}
    \frametitle{Case Study: Practical Use Cases}
    \begin{block}{Overview}
        \begin{itemize}
            \item Analysis of real-world applications of hash functions.
            \item Focus on their impact on security in software systems.
        \end{itemize}
    \end{block}
\end{frame}

\begin{frame}
    \frametitle{Introduction to Cryptographic Hash Functions}
    \begin{itemize}
        \item Cryptographic hash functions ensure data integrity and authenticity.
        \item They convert an arbitrary input into a fixed-size string.
        \item The process is non-reversible, preventing retrieval of the original data.
    \end{itemize}
\end{frame}

\begin{frame}
    \frametitle{Real-World Applications}
    \begin{enumerate}
        \item Data Integrity Verification
        \item Digital Signatures
        \item Password Hashing
    \end{enumerate}
\end{frame}

\begin{frame}[fragile]
    \frametitle{1. Data Integrity Verification}
    \begin{itemize}
        \item \textbf{Example:} Download Verification
        \begin{itemize}
            \item Hash value (e.g., SHA-256) is provided when downloading software.
            \item Compute hash of downloaded file and compare with provided hash.
        \end{itemize}
        \item \textbf{Impact on Security:} 
        \begin{itemize}
            \item Ensures data received matches original data.
            \item Prevents corruption and unauthorized changes.
        \end{itemize}
    \end{itemize}
\end{frame}

\begin{frame}
    \frametitle{2. Digital Signatures}
    \begin{itemize}
        \item \textbf{Example:} Electronic Contracts
        \begin{itemize}
            \item Digital signatures verify sender's identity and document integrity.
            \item Document is hashed, and the hash is encrypted with the sender's private key.
        \end{itemize}
        \item \textbf{Impact on Security:} 
        \begin{itemize}
            \item Confirms message authenticity.
            \item Provides non-repudiation, preventing denial of message sending.
        \end{itemize}
    \end{itemize}
\end{frame}

\begin{frame}
    \frametitle{3. Password Hashing}
    \begin{itemize}
        \item \textbf{Example:} User Authentication
        \begin{itemize}
            \item Storing only hashed passwords (e.g., using bcrypt).
            \item During login, the provided password is hashed and compared with the stored hash.
        \end{itemize}
        \item \textbf{Impact on Security:}
        \begin{itemize}
            \item Shields original passwords even if the database is compromised.
            \item Enhances user security significantly.
        \end{itemize}
    \end{itemize}
\end{frame}

\begin{frame}[fragile]
    \frametitle{Illustrative Example: Code Snippet}
    \begin{lstlisting}[language=Python]
import hashlib

# Function to create a SHA-256 hash of a given input
def create_hash(input_data):
    # Encode the input data
    encoded_data = input_data.encode()
    # Create a new sha256 hash object
    hash_object = hashlib.sha256()
    # Update the hash object with the bytes-like object
    hash_object.update(encoded_data)
    # Return the hexadecimal digest of the hash
    return hash_object.hexdigest()

# Example usage
print(create_hash("Hello, World!"))  # Outputs: A591A6D40BF420404A513F898CAC38B99151B8D3
    \end{lstlisting}
\end{frame}

\begin{frame}
    \frametitle{Conclusion}
    \begin{itemize}
        \item Hash functions are essential for data integrity and securing communications.
        \item They play a crucial role in building trust and safeguarding credentials.
        \item As technology evolves, the reliance on robust hash functions will increase.
    \end{itemize}
\end{frame}

\begin{frame}[fragile]
    \frametitle{Future of Hash Functions in Cryptography}
    \begin{block}{Overview}
        The development of hash functions is vital for future-proofing security systems, especially with the rise of post-quantum cryptography (PQC).
    \end{block}
\end{frame}

\begin{frame}[fragile]
    \frametitle{Hash Functions and Their Properties}
    \begin{block}{What are Hash Functions?}
        A hash function is a one-way function that converts input data of any size into a fixed-size string of characters, which appears random.
    \end{block}
    \begin{itemize}
        \item \textbf{Deterministic:} Same input produces the same hash.
        \item \textbf{Fast Computation:} Quick to compute the hash for any input.
        \item \textbf{Pre-image Resistance:} Infeasible to retrieve the original input.
        \item \textbf{Collision Resistance:} Hard to find two inputs yielding the same hash.
    \end{itemize}
\end{frame}

\begin{frame}[fragile]
    \frametitle{Post-Quantum Cryptography (PQC)}
    \begin{block}{What is PQC?}
        Post-quantum cryptography refers to algorithms believed to be secure against quantum computer threats, which can solve problems like integer factorization much faster than classical computers.
    \end{block}
\end{frame}

\begin{frame}[fragile]
    \frametitle{Key Developments in Hash Functions}
    \begin{enumerate}
        \item \textbf{Enhanced Security Standards:}
            \begin{itemize}
                \item Transitioning from SHA-1 to SHA-256 to SHA-3 for better resistance against attacks.
            \end{itemize}
        \item \textbf{PQC-Compatible Hash Functions:}
            \begin{itemize}
                \item Developing hash functions that are robust against quantum attacks.
                \item Examples include candidates from NIST’s PQC project.
            \end{itemize}
        \item \textbf{Applications in Emerging Technologies:}
            \begin{itemize}
                \item Integration in blockchain technology and digital signatures for long-term security.
            \end{itemize}
    \end{enumerate}
\end{frame}

\begin{frame}[fragile]
    \frametitle{Examples of Hash Functions}
    \begin{block}{Integrity Assurance Example}
        Imagine sending a message: hash the message, send both original and hash. The receiver hashes the original again. If hashes match, message is unaltered.
    \end{block}
    \begin{block}{PQC Example}
        A hybrid approach combining traditional cryptography with quantum-resistant algorithms (like lattice-based hashes).
    \end{block}
\end{frame}

\begin{frame}[fragile]
    \frametitle{Key Takeaways}
    \begin{itemize}
        \item The impact of quantum computing on traditional cryptographic methods motivates the transition to future-proof hash functions.
        \item Continuous advancements are necessary to ensure resilience against evolving threats.
        \item Engagement with standards is crucial for developers to ensure compatibility within security protocols.
    \end{itemize}
\end{frame}

\begin{frame}[fragile]
    \frametitle{Conclusion and Further Reading}
    \begin{block}{Conclusion}
        The future of hash functions in cryptography holds promise through evolution and integration of post-quantum designs, ensuring long-term security.
    \end{block}
    \begin{itemize}
        \item NIST Post-Quantum Cryptography Standards
        \item Research on cryptographic primitives in the age of quantum computing.
    \end{itemize}
\end{frame}

\begin{frame}[fragile]
    \frametitle{Conclusion and Key Takeaways - Importance of Cryptographic Hash Functions}
    \begin{itemize}
        \item \textbf{Definition}: A cryptographic hash function transforms input data into a fixed-size string, creating a unique digest that ensures data integrity.
        \item \textbf{Key Properties}:
        \begin{itemize}
            \item Deterministic: Identical inputs yield identical outputs.
            \item Quick Computation: Computationally feasible to calculate the hash.
            \item Pre-image Resistance: Infeasible to reverse the hash.
            \item Collision Resistance: Unlikely for different inputs to produce the same hash.
            \item Avalanche Effect: Small input changes produce significantly different hashes.
        \end{itemize}
    \end{itemize}
\end{frame}

\begin{frame}[fragile]
    \frametitle{Conclusion and Key Takeaways - Applications and Relevance}
    \begin{enumerate}
        \item \textbf{Real-World Applications}:
        \begin{itemize}
            \item Data Integrity: Confirming file integrity through hash validation.
            \item Password Storage: Storing hashed passwords for security.
            \item Digital Signatures: Ensuring authenticity via hashing and encryption.
        \end{itemize}
        \item \textbf{Relevance in the Digital Age}:
        \begin{itemize}
            \item Evolving Threat Landscape: Need for secure hash functions in light of advancing threats.
            \item Regulatory Compliance: Required cryptographic practices for sensitive data protection.
        \end{itemize}
    \end{enumerate}
\end{frame}

\begin{frame}[fragile]
    \frametitle{Conclusion and Key Takeaways - Final Thoughts}
    \begin{itemize}
        \item \textbf{Foundational Role}: Essential in security protocols like SSL/TLS and cryptocurrencies.
        \item \textbf{Continuous Evolution}: Need for updates as vulnerabilities are discovered, particularly in post-quantum cryptography.
        \item \textbf{Importance in Cybersecurity}: Effective use of hash functions is critical for ensuring digital trust and securing sensitive information.
    \end{itemize}
    \begin{block}{Example}
        \textbf{SHA-256 Example:}
        Given an input message \( M \): "Hello, World!"  
        The SHA-256 hash function will produce:
        \begin{equation}
            \text{Hash}(M) = \text{4d186321c1a7f0f354b297e8914ab240}
        \end{equation}
        (in hexadecimal format)
    \end{block}
\end{frame}


\end{document}