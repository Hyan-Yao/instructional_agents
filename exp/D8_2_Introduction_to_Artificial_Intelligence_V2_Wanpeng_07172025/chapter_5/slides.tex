\documentclass[aspectratio=169]{beamer}

% Theme and Color Setup
\usetheme{Madrid}
\usecolortheme{whale}
\useinnertheme{rectangles}
\useoutertheme{miniframes}

% Additional Packages
\usepackage[utf8]{inputenc}
\usepackage[T1]{fontenc}
\usepackage{graphicx}
\usepackage{booktabs}
\usepackage{listings}
\usepackage{amsmath}
\usepackage{amssymb}
\usepackage{xcolor}
\usepackage{tikz}
\usepackage{pgfplots}
\pgfplotsset{compat=1.18}
\usetikzlibrary{positioning}
\usepackage{hyperref}

% Custom Colors
\definecolor{myblue}{RGB}{31, 73, 125}
\definecolor{mygray}{RGB}{100, 100, 100}
\definecolor{mygreen}{RGB}{0, 128, 0}
\definecolor{myorange}{RGB}{230, 126, 34}
\definecolor{mycodebackground}{RGB}{245, 245, 245}

% Set Theme Colors
\setbeamercolor{structure}{fg=myblue}
\setbeamercolor{frametitle}{fg=white, bg=myblue}
\setbeamercolor{title}{fg=myblue}
\setbeamercolor{section in toc}{fg=myblue}
\setbeamercolor{item projected}{fg=white, bg=myblue}
\setbeamercolor{block title}{bg=myblue!20, fg=myblue}
\setbeamercolor{block body}{bg=myblue!10}
\setbeamercolor{alerted text}{fg=myorange}

% Document Start
\begin{document}

\frame{\titlepage}

\begin{frame}[fragile]
    \frametitle{Introduction to Probabilistic Reasoning - Overview}
    \begin{block}{What is Probabilistic Reasoning?}
        Probabilistic reasoning uses probability theory to infer conclusions or make decisions under uncertainty. In AI, it helps evaluate and interpret uncertain information, facilitating informed decision-making.
    \end{block}
\end{frame}

\begin{frame}[fragile]
    \frametitle{Introduction to Probabilistic Reasoning - Importance in AI}
    \begin{itemize}
        \item \textbf{Handling Uncertainty:}
            \begin{itemize}
                \item Real-world scenarios are often uncertain (e.g., weather predictions, stock market behavior, medical diagnoses).
                \item Quantification and management of this uncertainty are critical.
            \end{itemize}
        \item \textbf{Flexible Modeling:}
            \begin{itemize}
                \item Represent complex relationships between variables efficiently.
                \item Example: Disease prediction models accounting for symptoms and risk factors.
            \end{itemize}
        \item \textbf{Incremental Learning:}
            \begin{itemize}
                \item Systems can update beliefs as new evidence arises.
                \item Essential for applications like recommendation systems and fraud detection.
            \end{itemize}
    \end{itemize}
\end{frame}

\begin{frame}[fragile]
    \frametitle{Key Concepts in Probabilistic Reasoning}
    \begin{block}{Bayes' Theorem}
        A fundamental rule to update probabilities based on new evidence:
        \begin{equation}
            P(A | B) = \frac{P(B | A) P(A)}{P(B)}
        \end{equation}
        Where:
        \begin{itemize}
            \item $P(A | B)$: Probability of event A given B.
            \item $P(B | A)$: Probability of event B given A.
            \item $P(A)$ and $P(B)$: Independent probabilities of A and B.
        \end{itemize}
    \end{block}

    \begin{block}{Examples of Probabilistic Reasoning in AI}
        \begin{itemize}
            \item \textbf{Spam Detection:} AI uses probabilistic reasoning to evaluate if an email is spam based on keywords and sender reputation.
            \item \textbf{Autonomous Vehicles:} Assess sensory inputs and navigate based on the probabilities of outcomes, like detecting obstacles.
        \end{itemize}
    \end{block}
\end{frame}

\begin{frame}[fragile]{Understanding Probability - Definition}
    \begin{block}{Definition of Probability}
        Probability is a numerical measure that quantifies the likelihood of an event occurring. It ranges from 0 to 1, where:
        \begin{itemize}
            \item \textbf{0} indicates an impossible event.
            \item \textbf{1} indicates a certain event.
        \end{itemize}
    \end{block}
    
    Mathematically, probability \( P(E) \) of an event \( E \) can be defined as:
    
    \begin{equation}
        P(E) = \frac{\text{Number of favorable outcomes}}{\text{Total number of outcomes}}
    \end{equation}
\end{frame}

\begin{frame}[fragile]{Understanding Probability - Role in Uncertainty}
    \begin{block}{Role of Probability in Uncertainty and Reasoning}
        \begin{itemize}
            \item \textbf{Dealing with Uncertainty}: Probability allows us to express uncertainties systematically in complex systems, helping us make informed decisions despite incomplete knowledge.
            
            \item \textbf{Reasoning Under Uncertainty}: In AI and statistical reasoning, probability aids in making predictions based on uncertain information, distinguishing between likely and unlikely events.
        \end{itemize}
    \end{block}
\end{frame}

\begin{frame}[fragile]{Understanding Probability - Key Points and Example}
    \begin{block}{Key Points to Emphasize}
        \begin{itemize}
            \item \textbf{Completeness and Coherence}: A good probabilistic model captures all relevant information and remains coherent.
            \item \textbf{Application in AI}: Used in machine learning, robotics, natural language processing, etc. E.g., spam detection uses probabilities to classify emails.
            \item \textbf{Bayesian Perspective}: Allows updating our beliefs in light of new evidence, forming a foundation for Bayesian networks.
        \end{itemize}
    \end{block}
  
    \textbf{Example:} Consider a six-sided die:
    \begin{equation}
        P(\text{rolling a 3}) = \frac{1}{6}
    \end{equation}
    This means there’s a 16.67\% chance of rolling a 3 on any single roll.
\end{frame}

\begin{frame}[fragile]{Understanding Probability - Conclusion}
    Understanding probability is integral to making rational decisions in uncertain situations. As we explore further into probabilistic reasoning and Bayesian networks, the principles laid out here will form the basis for more complex models and reasoning strategies.
\end{frame}

\begin{frame}[fragile]
  \frametitle{Key Terminology}
  \begin{block}{Understanding Key Terms}
    In probabilistic reasoning and Bayesian networks, it is crucial to familiarize ourselves with foundational concepts. 
    We will focus on:
    \begin{itemize}
      \item Random Variables
      \item Probability Distributions
      \item Events
    \end{itemize}
  \end{block}
\end{frame}

\begin{frame}[fragile]
  \frametitle{Key Terminology - Random Variables}
  \begin{block}{1. Random Variables}
    \textbf{Definition}: A random variable is a variable that can take on different values based on the outcome of a random phenomenon, denoted by a capital letter (e.g., $X$, $Y$).
  
    \textbf{Types}:
    \begin{itemize}
      \item \textbf{Discrete Random Variables}: Take on a countable number of values (e.g., the roll of a die).
      \item \textbf{Continuous Random Variables}: Can take on an infinite number of values within a range (e.g., weight, height).
    \end{itemize}

    \textbf{Example}: Let $X$ represent the result of rolling a fair six-sided die. The possible values for $X$ are $\{1, 2, 3, 4, 5, 6\}$.
  \end{block}
\end{frame}

\begin{frame}[fragile]
  \frametitle{Key Terminology - Probability Distributions}
  \begin{block}{2. Probability Distributions}
    \textbf{Definition}: A probability distribution describes how probabilities are distributed over the values of a random variable.
  
    \textbf{Types}:
    \begin{itemize}
      \item \textbf{Probability Mass Function (PMF)}: For discrete random variables, gives the probability that a random variable equals a specific value.
      \begin{equation}
        P(X = x) = p(x)
      \end{equation}
      \item \textbf{Probability Density Function (PDF)}: For continuous random variables, describes the likelihood of falling within a range.
    \end{itemize}

    \textbf{Example PMF for die:}
    \begin{equation}
      P(X = x) = 
      \begin{cases}
        \frac{1}{6} & \text{if } x \in \{1, 2, 3, 4, 5, 6\} \\
        0 & \text{otherwise}
      \end{cases}
    \end{equation}
  \end{block}
\end{frame}

\begin{frame}[fragile]
  \frametitle{Key Terminology - Events}
  \begin{block}{3. Events}
    \textbf{Definition}: An event is a specific outcome or collection of outcomes of a random variable. Events can be:
    \begin{itemize}
      \item \textbf{Simple Event}: A single outcome (e.g., $E = \{4\}$).
      \item \textbf{Compound Event}: Multiple outcomes (e.g., $E = \{2, 4, 6\}$).
    \end{itemize}
    
    \textbf{Key Points}:
    \begin{itemize}
      \item An event is a subset of the sample space (set of all possible outcomes).
      \item Probability of an event $E$ is calculated as:
      \begin{equation}
        P(E) = \sum_{x \in E} P(X = x)
      \end{equation}
    \end{itemize}

    \textbf{Note}: For continuous variables, integrate the PDF over the relevant interval.
  \end{block}
\end{frame}

\begin{frame}[fragile]
  \frametitle{Conclusion}
  Understanding these key terms lays the groundwork for more advanced topics in probability and Bayesian networks. 
  Mastering random variables, probability distributions, and events allows for deeper engagement with inference and decision-making under uncertainty.
  
  \textbf{Engagement Questions}:
  \begin{itemize}
    \item Can you think of a real-life scenario where you might use a random variable?
    \item What are some common events you encounter in daily life that can be modeled probabilistically?
  \end{itemize}
  
  This overview will assist as we move into deeper discussions on Bayesian inference in the next slide!
\end{frame}

\begin{frame}[fragile]
    \frametitle{Bayesian Inference - Introduction}
    \begin{block}{Definition}
        Bayesian inference is a statistical method that applies Bayes' theorem to update the probability of a hypothesis as more evidence becomes available.
    \end{block}
    \begin{itemize}
        \item Provides a rational approach to decision-making in uncertainty.
        \item Focuses on updating prior beliefs with new evidence.
    \end{itemize}
\end{frame}

\begin{frame}[fragile]
    \frametitle{Bayesian Inference - Key Concepts}
    \begin{enumerate}
        \item \textbf{Prior Probability (P(H))}: Initial belief before new evidence.
        \item \textbf{Likelihood (P(E | H))}: Probability of observing evidence given the hypothesis.
        \item \textbf{Posterior Probability (P(H | E))}: Updated probability after considering the new evidence.
        \item \textbf{Evidence (P(E))}: Total probability of observing the evidence under all hypotheses.
    \end{enumerate}
\end{frame}

\begin{frame}[fragile]
    \frametitle{Bayes' Theorem}
    \begin{block}{Formula}
        Bayes' theorem can be mathematically expressed as:
        \begin{equation}
        P(H | E) = \frac{P(E | H) \times P(H)}{P(E)}
        \end{equation}
    \end{block}
    \begin{itemize}
        \item \( P(H | E) \): Posterior probability
        \item \( P(E | H) \): Likelihood
        \item \( P(H) \): Prior probability
        \item \( P(E) \): Evidence
    \end{itemize}
\end{frame}

\begin{frame}[fragile]
    \frametitle{Bayesian Inference - Example Scenario}
    \begin{block}{Medical Diagnosis}
        Imagine a medical test for a disease with:
        \begin{itemize}
            \item \textbf{Prior Probability (P(Disease))}: Prevalence (e.g., 1\%).
            \item \textbf{Likelihood (P(Pos | Disease))}: Probability of a positive result if diseased (e.g., 90\%).
            \item \textbf{False Positive Rate (P(Pos | No Disease))}: Positive result without disease (e.g., 5\%).
        \end{itemize}
    \end{block}
    Suppose a patient tests positive. We want to calculate \( P(Disease | Pos) \).
\end{frame}

\begin{frame}[fragile]
    \frametitle{Bayesian Inference - Example Calculation}
    \begin{block}{Calculating Probabilities}
        \begin{align*}
            P(Disease) & = 0.01 \\
            P(Pos | Disease) & = 0.9 \\
            P(Pos | No Disease) & = 0.05 \\
            P(Pos) & = P(Pos | Disease) \cdot P(Disease) + P(Pos | No Disease) \cdot P(No Disease) \\
            & = 0.9 \times 0.01 + 0.05 \times 0.99 \\
            & = 0.009 + 0.0495 = 0.0585 \\
            P(Disease | Pos) & = \frac{P(Pos | Disease) \times P(Disease)}{P(Pos)} \\
            & = \frac{0.9 \cdot 0.01}{0.0585} \approx 0.1538 \text{ (15.38\%)}
        \end{align*}
    \end{block}
\end{frame}

\begin{frame}[fragile]
    \frametitle{Bayesian Inference - Key Points}
    \begin{itemize}
        \item Allows continual updating of beliefs with new evidence.
        \item Widely applicable in fields such as:
        \begin{itemize}
            \item Medical diagnosis
            \item Finance
            \item Artificial intelligence
        \end{itemize}
        \item Understanding priors and likelihoods is critical for accuracy.
    \end{itemize}
\end{frame}

\begin{frame}[fragile]
    \frametitle{Bayesian Inference - Conclusion}
    \begin{block}{Summary}
        Bayesian inference is a powerful statistical tool that informs decision-making through the application of Bayes' theorem, allowing for informed judgments based on evidence and prior knowledge.
    \end{block}
\end{frame}

\begin{frame}[fragile]
  \frametitle{Applications of Bayesian Inference}
  
  \begin{block}{Understanding Bayesian Inference}
    Bayesian inference is a statistical method that applies the principles of Bayes' theorem to update the probability estimates of a hypothesis as more evidence or information becomes available. This framework allows for a probabilistic approach to reasoning and decision-making in various fields.
  \end{block}
\end{frame}

\begin{frame}[fragile]
  \frametitle{Applications in Various Fields}
  
  \begin{enumerate}
    \item \textbf{Medicine}
      \begin{itemize}
        \item Disease Diagnosis: Bayesian inference is employed in medical diagnostics to update the probability of disease given new test results.
        \item \textbf{Example}: If a disease has a prevalence of 1\% and the test has a 95\% sensitivity and a 5\% false positive rate, the post-test probability of the disease can be calculated using Bayes' theorem:
        \begin{equation}
        P(Disease|Positive) = \frac{P(Positive|Disease) \times P(Disease)}{P(Positive)}
        \end{equation}
      \end{itemize}
    
    \item \textbf{Finance}
      \begin{itemize}
        \item Risk Assessment and Portfolio Management: Bayesian inference assists financial analysts in evaluating risks and returns.
        \item \textbf{Example}: A financier uses historical stock returns to estimate a new probabilistic model for a stock's future performance.
      \end{itemize}
    
    \item \textbf{Artificial Intelligence (AI)}
      \begin{itemize}
        \item Machine Learning Models: Bayesian methods are pivotal in machine learning, particularly in probabilistic models.
        \item \textbf{Example}: In a spam detection system, the Bayesian classifier evaluates the probability of an email being spam based on the features of the email.
      \end{itemize}
  \end{enumerate}
\end{frame}

\begin{frame}[fragile]
  \frametitle{Key Points and Conclusion}
  
  \begin{block}{Key Points to Emphasize}
    \begin{itemize}
      \item Bayesian inference provides a robust framework for incorporating new evidence into existing beliefs.
      \item It is applicable across diverse domains, enhancing decision-making under uncertainty.
      \item Real-world problems often involve multiple sources of uncertainty, perfectly addressed by Bayesian methods.
    \end{itemize}
  \end{block}

  \begin{block}{Conclusion}
    Bayesian inference serves as a powerful tool across medicine, finance, and AI, facilitating more informed and adaptable decisions by continuously integrating new evidence into existing hypotheses.
    
    \textit{Explore how Bayesian inference can improve your understanding and applications in your field of study or work.}
  \end{block}
\end{frame}

\begin{frame}[fragile]
    \frametitle{Bayesian Networks Introduction - Definition}
    \begin{block}{Definition of Bayesian Networks}
        \textbf{Bayesian Networks} are graphical models that represent a set of variables and their probabilistic dependencies using directed acyclic graphs (DAGs). 
        They provide a systematic way to compute the probability of certain outcomes based on prior knowledge or evidence.
    \end{block}
\end{frame}

\begin{frame}[fragile]
    \frametitle{Bayesian Networks Introduction - Structure}
    \begin{block}{Structure of Bayesian Networks}
        \begin{itemize}
            \item \textbf{Nodes}: Each node represents a random variable (discrete or continuous).
            \begin{itemize}
                \item Example: In a medical diagnosis context, nodes may represent symptoms and diseases.
            \end{itemize}
            \item \textbf{Directed Edges}: Edges (arrows) indicate a directed relationship.
            \begin{itemize}
                \item Example: If a disease (node A) causes a symptom (node B), the edge points as \(A \rightarrow B\).
            \end{itemize}
            \item \textbf{Conditional Probability Tables (CPTs)}: Quantifies the effect of parent nodes.
            \begin{itemize}
                \item Example: The CPT for a symptom may specify the probability of it being present given the state of the disease.
            \end{itemize}
        \end{itemize}
    \end{block}
\end{frame}

\begin{frame}[fragile]
    \frametitle{Bayesian Networks Introduction - Key Points}
    \begin{block}{Key Points}
        \begin{itemize}
            \item \textbf{DAG Structure}: Bayesian networks are acyclic, ensuring clear and well-defined influence.
            \item \textbf{Probabilistic Reasoning}: They allow updating beliefs using Bayes' theorem:
            \begin{equation}
                P(H|E) = \frac{P(E|H) \cdot P(H)}{P(E)}
            \end{equation}
            Where:
            \begin{itemize}
                \item \(P(H|E)\) = posterior probability
                \item \(P(E|H)\) = likelihood of evidence
                \item \(P(H)\) = prior probability
                \item \(P(E)\) = marginal probability
            \end{itemize}
        \end{itemize}
    \end{block}
\end{frame}

\begin{frame}[fragile]
    \frametitle{Bayesian Networks Introduction - Example Illustration}
    \begin{block}{Example Illustration}
        Consider a simple Bayesian Network for disease diagnosis based on symptoms:
        \begin{itemize}
            \item \textbf{Nodes}:
                \begin{itemize}
                    \item Disease (D)
                    \item Symptom1 (S1)
                    \item Symptom2 (S2)
                \end{itemize}
            \item \textbf{Directed Edges}:
                \begin{itemize}
                    \item \(D \rightarrow S1\)
                    \item \(D \rightarrow S2\)
                \end{itemize}
            \item \textbf{CPT for Symptom1}:
            \begin{equation}
                P(S1|D) =
                \begin{cases}
                    P(S1 = True | D = True) = 0.8 \\
                    P(S1 = True | D = False) = 0.1
                \end{cases}
            \end{equation}
            This illustrates how disease presence affects symptom likelihood.
        \end{itemize}
    \end{block}
\end{frame}

\begin{frame}[fragile]
    \frametitle{Components of Bayesian Networks - Introduction}
    \begin{block}{Introduction to Bayesian Networks}
        Bayesian Networks are graphical models that represent a set of variables and their conditional dependencies through a directed acyclic graph (DAG). These networks are useful for probabilistic reasoning, allowing us to infer unknown probabilities based on known information.
    \end{block}
\end{frame}

\begin{frame}[fragile]
    \frametitle{Components of Bayesian Networks - Nodes}
    \begin{itemize}
        \item \textbf{Definition:} A node represents a random variable, which can be:
        \begin{itemize}
            \item Binary (true/false)
            \item Discrete (a finite set of values)
            \item Continuous (any value in a range)
        \end{itemize}

        \item \textbf{Example:}
        \begin{itemize}
            \item In a medical diagnosis network, nodes may represent symptoms (e.g., "Cough", "Fever") and diseases (e.g., "Flu", "COVID-19").
        \end{itemize}

        \item \textbf{Key Point:} Each node contains information about the variable, represented as probabilities reflecting its possible states.
    \end{itemize}
\end{frame}

\begin{frame}[fragile]
    \frametitle{Components of Bayesian Networks - Directed Edges and CPTs}
    \textbf{2. Directed Edges:}
    \begin{itemize}
        \item \textbf{Definition:} Directed edges indicate the direction of influence between nodes. An edge from node A to node B implies that A influences B, making B conditionally dependent on A.
        
        \item \textbf{Example:}
        \begin{itemize}
            \item A directed edge from "Smoke" to "Cough" implies that smoking affects the likelihood of coughing.
        \end{itemize}
        
        \item \textbf{Key Point:} Directed edges establish a cause-and-effect relationship, modeling complex interactions.
    \end{itemize}

    \bigskip

    \textbf{3. Conditional Probability Tables (CPTs):}
    \begin{itemize}
        \item \textbf{Definition:} Each node has an associated CPT that quantifies how parent nodes affect it, providing probabilities of each state given the states of its parents.
        
        \item \textbf{Example:}
        \begin{itemize}
            \item For the "Cough" node with parent "Smoke": 
            \[
            P(Cough | Smoke=True) = 0.8, \quad P(Cough | Smoke=False) = 0.1 
            \]
        \end{itemize}
        
        \item \textbf{Key Point:} CPTs are essential for calculating joint probabilities and making inferences based on observed evidence.
    \end{itemize}
\end{frame}

\begin{frame}[fragile]
    \frametitle{Summary and Visual Representation}
    \begin{block}{Summary}
        \begin{itemize}
            \item \textbf{Nodes:} Represent random variables within the network.
            \item \textbf{Directed Edges:} Indicate influences or dependencies among variables.
            \item \textbf{Conditional Probability Tables:} Provide the framework detailing how each variable is affected by its parents.
        \end{itemize}
    \end{block}

    \bigskip

    \textbf{Visual Representation:} 
    \begin{itemize}
        \item Consider drawing a simple Bayesian network diagram with nodes such as "Smoke" and "Cough" showing directed edges and a sample CPT.
    \end{itemize}

    \bigskip

    \textbf{Next Slide:} We will explore the practical steps involved in constructing a Bayesian network tailored for specific problems.
\end{frame}

\begin{frame}[fragile]
    \frametitle{Constructing a Bayesian Network - Overview}
    \begin{block}{Definition}
        A Bayesian Network (BN) is a graphical model that represents a set of variables and their conditional dependencies through a directed acyclic graph (DAG). 
    \end{block}
    
    \begin{block}{Objective}
        This slide outlines the systematic steps to construct a Bayesian network specifically tailored for a given problem.
    \end{block}
\end{frame}

\begin{frame}[fragile]
    \frametitle{Constructing a Bayesian Network - Steps}
    \begin{enumerate}
        \item \textbf{Define the Problem and Identify Relevant Variables}
            \begin{itemize}
                \item Clearly state the problem to solve.
                \item Identify key variables affecting the outcome.
                \item \textit{Example}: For medical diagnosis, variables include symptoms, test results, and diseases.
            \end{itemize}

        \item \textbf{Determine the Structure of the Network}
            \begin{itemize}
                \item Establish relationships; draw directed edges from parents to children.
                \item \textit{Key Point}: Absence of an edge implies independence.
                \item \textit{Example Diagram}: If `Disease A` influences `Symptom X`, draw an edge from `Disease A` to `Symptom X`.
            \end{itemize}
    \end{enumerate}
\end{frame}

\begin{frame}[fragile]
    \frametitle{Constructing a Bayesian Network - Continued Steps}
    \begin{enumerate}
        \setcounter{enumi}{2}
        \item \textbf{Specify Conditional Probability Tables (CPTs)}
            \begin{itemize}
                \item Define probabilities for each variable, conditional on parent nodes.
                \item Specify unconditional probabilities if there are no parents.
                \item \textit{Example}: For `Symptom X`, create a CPT for $P(\text{Symptom X} | \text{Disease A})$.
            \end{itemize}

        \item \textbf{Validate the Network Structure}
            \begin{itemize}
                \item Verify against known data or expert knowledge.
                \item \textit{Key Point}: Ensure relationships and probabilities make sense.
            \end{itemize}

        \item \textbf{Refine and Iterate}
            \begin{itemize}
                \item Refine based on validation feedback.
                \item This iterative process enhances the model's accuracy.
                \item \textit{Example}: Adjust the network if new information suggests another variable influences `Symptom X`.
            \end{itemize}
    \end{enumerate}
\end{frame}

\begin{frame}[fragile]
    \frametitle{Constructing a Bayesian Network - Example and Conclusion}
    \begin{block}{Example Application: Medical Diagnosis}
        \begin{itemize}
            \item \textbf{Problem}: Diagnose disease based on symptoms and test results.
            \item \textbf{Variables}: Disease (D), Symptom 1 (S1), Symptom 2 (S2), Test Result (T).
            \item \textbf{CPTs}:
                \begin{itemize}
                    \item $P(D)$
                    \item $P(S1 | D)$
                    \item $P(S2 | D)$
                    \item $P(T | D, S1, S2)$
                \end{itemize}
        \end{itemize}
    \end{block}
    
    \begin{block}{Conclusion}
        Constructing a Bayesian network involves a systematic approach from defining the problem to validating and refining the model. These steps ensure that the network effectively represents the underlying uncertainties and relationships, facilitating reasoning and inference.
    \end{block}
\end{frame}

\begin{frame}[fragile]
    \frametitle{Inference in Bayesian Networks - Introduction}
    \begin{block}{Introduction to Inference}
        Inference in Bayesian networks refers to the process of deriving conclusions or predictions based on known information.
    \end{block}
    These networks are graphical models representing probabilistic relationships among a set of variables. 
    \begin{itemize}
        \item Nodes represent variables.
        \item Directed edges indicate dependencies.
    \end{itemize}
    This structure facilitates reasoning about uncertainty.
\end{frame}

\begin{frame}[fragile]
    \frametitle{Inference in Bayesian Networks - Types of Methods}
    \begin{block}{Types of Inference Methods}
        There are two main categories of inference methods in Bayesian networks:
    \end{block}
    \begin{enumerate}
        \item \textbf{Exact Inference}
        \item \textbf{Approximate Inference}
    \end{enumerate}
\end{frame}

\begin{frame}[fragile]
    \frametitle{Inference in Bayesian Networks - Exact Inference}
    \begin{block}{Exact Inference}
        \begin{itemize}
            \item \textbf{Definition:} Calculates exact probability of a query variable given evidence.
            \item \textbf{Common Algorithms:}
                \begin{itemize}
                    \item \textbf{Variable Elimination:} Systematically eliminates variables by summing out non-query variables.
                    \item \textbf{Junction Tree Algorithm:} Transforms network into tree structure for efficient marginal probability computation.
                \end{itemize}
        \end{itemize}
        \textbf{Example:} If it is known that it is raining, we can use exact inference to find the probability that the grass is wet.
    \end{block}
\end{frame}

\begin{frame}[fragile]
    \frametitle{Inference in Bayesian Networks - Approximate Inference}
    \begin{block}{Approximate Inference}
        \begin{itemize}
            \item \textbf{Definition:} Provides estimates rather than exact probabilities for large or complex networks.
            \item \textbf{Common Techniques:}
                \begin{itemize}
                    \item \textbf{Monte Carlo Methods:} Uses random sampling to approximate distributions.
                    \item \textbf{Variational Inference:} Approximates probability distribution with a simpler, tractable distribution.
                \end{itemize}
        \end{itemize}
        \textbf{Example:} Use Monte Carlo simulation to estimate diagnosis likelihood in an expansive network influenced by multiple factors.
    \end{block}
\end{frame}

\begin{frame}[fragile]
    \frametitle{Inference in Bayesian Networks - Key Points and Conclusion}
    \begin{block}{Key Points to Emphasize}
        \begin{itemize}
            \item Importance of Inference: Enables decision-making under uncertainty, crucial in medical diagnosis and risk assessment.
            \item Exact vs. Approximate: Use exact methods for smaller networks; choose approximate for larger networks.
            \item Graph Structure Matters: Design impacts inferencing efficiency.
        \end{itemize}
    \end{block}

    \begin{block}{Conclusion}
        Understanding inference enhances decision-making based on uncertain information. Mastery of methods allows better utilization of Bayesian networks in real-world applications.
    \end{block}
\end{frame}

\begin{frame}[fragile]
    \frametitle{Inference in Bayesian Networks - Further Reading}
    \begin{block}{Suggested Further Reading}
        \begin{itemize}
            \item Explore lecture notes on Variable Elimination and Junction Tree Algorithms.
            \item Review case studies demonstrating Bayesian inference in healthcare diagnostics.
        \end{itemize}
    \end{block}
\end{frame}

\begin{frame}[fragile]
  \frametitle{Example of a Bayesian Network}
  \begin{block}{Bayesian Networks Overview}
    A Bayesian Network (BN) is a graphical model representing a set of variables and their conditional dependencies using directed acyclic graphs. It helps understand how factors influence each other and allows probabilistic inferences about uncertain situations.
  \end{block}
\end{frame}

\begin{frame}[fragile]
  \frametitle{Practical Example: Diagnosis in Healthcare}
  \begin{block}{Key Variables in Our Bayesian Network}
    \begin{enumerate}
      \item \textbf{Disease (D)}: True (D=1) or False (D=0)
      \item \textbf{Symptom 1 (S1)}: Present (S1=1) or Absent (S1=0) (e.g., Cough)
      \item \textbf{Symptom 2 (S2)}: Present (S2=1) or Absent (S2=0) (e.g., Fever)
      \item \textbf{Test Result (T)}: Positive (T=1) or Negative (T=0)
    \end{enumerate}
  \end{block}

  \begin{block}{Structure of the Bayesian Network}
    The arrows indicate influence:
    \begin{center}
      D $\rightarrow$ S1 \\
      D $\rightarrow$ S2 \\
      D $\rightarrow$ T
    \end{center}
  \end{block}
\end{frame}

\begin{frame}[fragile]
  \frametitle{Conditional Probabilities and Inferences}
  \begin{block}{Prior Probability of Disease (P(D))}
    \begin{itemize}
      \item P(D=1) = 0.1 (10\% chance of having Disease A)
      \item P(D=0) = 0.9 (90\% chance of not having the disease)
    \end{itemize}
  \end{block}

  \begin{block}{Conditional Probabilities}
    \begin{itemize}
      \item P(S1 | D): 
        \begin{itemize}
          \item P(S1=1 | D=1) = 0.8 (80\% likely to cough if the disease is present)
          \item P(S1=1 | D=0) = 0.1 (10\% likely to cough if disease is absent)
        \end{itemize}
      \item P(S2 | D): 
        \begin{itemize}
          \item P(S2=1 | D=1) = 0.9 (90\% likely to have a fever if present)
          \item P(S2=1 | D=0) = 0.05 (5\% unlikely to have fever if absent)
        \end{itemize}
    \end{itemize}
  \end{block}

  \begin{block}{Test Result Probabilities (P(T|D))}
    \begin{itemize}
      \item P(T=1 | D=1) = 0.95 (95\% chance of positive test if disease present)
      \item P(T=1 | D=0) = 0.1 (10\% chance of positive test if disease absent)
    \end{itemize}
  \end{block}

  \begin{block}{Making Inferences}
    Questions like "What is the probability of Disease A given a cough and a positive test?" can be answered using Bayes' theorem.
  \end{block}
\end{frame}

\begin{frame}[fragile]
  \frametitle{Key Points}
  \begin{itemize}
    \item Bayesian Networks encapsulate relationships between variables.
    \item They integrate new evidence/preferences to update beliefs (posterior probability).
    \item They are particularly useful in healthcare for enhancing diagnostic decisions.
  \end{itemize}
\end{frame}

\begin{frame}[fragile]
    \frametitle{Advantages and Limitations - Overview}
    \begin{block}{Benefits of Bayesian Networks}
        Analyze the benefits and limitations of using Bayesian networks in various applications.
    \end{block}
\end{frame}

\begin{frame}[fragile]
    \frametitle{Advantages of Bayesian Networks}
    \begin{enumerate}
        \item \textbf{Intuitive Representation}
            \begin{itemize}
                \item Concept: Directed acyclic graphs (DAGs) represent variables and their probabilistic relationships visually.
                \item Example: In healthcare, nodes can signify diseases and symptoms, aiding clinicians in understanding complex relationships.
            \end{itemize}
        \item \textbf{Incorporation of Prior Knowledge}
            \begin{itemize}
                \item Concept: Integrates prior beliefs with new evidence.
                \item Example: Family history of a disease influences diagnosis probability based on test results.
            \end{itemize}
        \item \textbf{Flexible and Scalable}
            \begin{itemize}
                \item Concept: Models complex systems with interacting variables across diverse fields.
                \item Example: Scales easily without complete reinterpretation, unlike traditional methods.
            \end{itemize}
    \end{enumerate}
\end{frame}

\begin{frame}[fragile]
    \frametitle{Advantages of Bayesian Networks (cont.)}
    \begin{enumerate}
        \setcounter{enumi}{3} % To continue the numbering
        \item \textbf{Uncertainty Quantification}
            \begin{itemize}
                \item Concept: Systematic handling of uncertainty, clarifying evidence's influence on conclusions.
                \item Example: Patients without classic symptoms can still be diagnosed relevantly.
            \end{itemize}
        \item \textbf{Good for Dynamic Systems}
            \begin{itemize}
                \item Concept: Easily updates with new evidence over time.
                \item Example: In time-series analysis, they adapt with new data, useful for stock price predictions.
            \end{itemize}
    \end{enumerate}
\end{frame}

\begin{frame}[fragile]
    \frametitle{Limitations of Bayesian Networks}
    \begin{enumerate}
        \item \textbf{Computational Complexity}
            \begin{itemize}
                \item Concept: Increased variables lead to intractable computations for inference.
                \item Example: Large networks require significant processing time and memory.
            \end{itemize}
        \item \textbf{Dependency Assumptions}
            \begin{itemize}
                \item Concept: Assumes conditional independence that may not hold in real scenarios.
                \item Example: Conditional dependencies among diseases can lead to mismodeling.
            \end{itemize}
        \item \textbf{Data Requirements}
            \begin{itemize}
                \item Concept: Requires substantial data for accurate probability estimates.
                \item Example: Rare diseases with insufficient data yield poorly performing networks.
            \end{itemize}
    \end{enumerate}
\end{frame}

\begin{frame}[fragile]
    \frametitle{Limitations of Bayesian Networks (cont.)}
    \begin{enumerate}
        \setcounter{enumi}{3} % To continue the numbering
        \item \textbf{Difficulties in Model Specification}
            \begin{itemize}
                \item Concept: Design often needs domain expertise and is subjective.
                \item Example: Incorrect relationships lead to flawed predictions.
            \end{itemize}
        \item \textbf{Sensitivity to Prior Distributions}
            \begin{itemize}
                \item Concept: Results can be influenced heavily by prior distribution choices.
                \item Example: Non-representative priors may skew results towards them rather than observation.
            \end{itemize}
    \end{enumerate}
\end{frame}

\begin{frame}[fragile]
    \frametitle{Key Takeaways and Conclusion}
    \begin{block}{Key Takeaways}
        \begin{itemize}
            \item Bayesian networks are powerful for modeling relationships and uncertainty in complex systems.
            \item Consider limitations such as computational cost and assumptions about dependencies.
        \end{itemize}
    \end{block}
    \begin{block}{Conclusion}
        Understanding both advantages and limitations is crucial for optimizing the use of Bayesian networks in practical applications. This awareness helps leverage strengths while recognizing potential pitfalls.
    \end{block}
\end{frame}

\begin{frame}[fragile]
    \frametitle{Comparing Bayesian Networks with Other Models - Overview}
    \begin{block}{Key Concepts}
        \begin{itemize}
            \item **Bayesian Networks (BN)**: Directed acyclic graph representing conditional dependencies.
            \item **Markov Networks (MN)**: Undirected graphical model showing relationships without directional influences.
        \end{itemize}
    \end{block}
\end{frame}

\begin{frame}[fragile]
    \frametitle{Key Differences Between Bayesian and Markov Networks}
    \begin{enumerate}
        \item **Directionality**:
            \begin{itemize}
                \item Bayesian Networks: Directed edges indicate causality (e.g., A → B).
                \item Markov Networks: Undirected edges denote symmetric relationships (e.g., A - B).
            \end{itemize}
        \item **Type of Dependencies**:
            \begin{itemize}
                \item Bayesian Networks: Effective for conditional dependencies.
                \item Markov Networks: Capture global dependencies through potential functions.
            \end{itemize}
    \end{enumerate}
\end{frame}

\begin{frame}[fragile]
    \frametitle{Mathematical Factorization}
    \begin{block}{Factorization of Joint Distribution}
        \begin{itemize}
            \item \textbf{Bayesian Networks}:
                \[
                P(X_1, X_2, \ldots, X_n) = \prod_{i=1}^{n} P(X_i \mid \text{Parents}(X_i))
                \]
            \item \textbf{Markov Networks}:
                \[
                P(X_1, X_2, \ldots, X_n) = \frac{1}{Z} \prod_{c \in \text{Cliques}} \phi_c(X_c)
                \]
                where \(Z\) is a normalization constant.
        \end{itemize}
    \end{block}
\end{frame}

\begin{frame}[fragile]
    \frametitle{Practical Examples}
    \begin{block}{Bayesian Networks Example}
        A medical diagnosis system where symptoms influence diseases. Observing a symptom adjusts the probabilities of underlying causes.
    \end{block}

    \begin{block}{Markov Networks Example}
        In image segmentation, pixels influence neighboring pixels, maintaining consistent segments based on local information.
    \end{block}
\end{frame}

\begin{frame}[fragile]
    \frametitle{Key Points to Emphasize}
    \begin{itemize}
        \item **Causality**: Bayesian Networks are superior for causal inference due to explicit directional influences.
        \item **Symmetry and Locality**: Markov Networks excel in modeling local dependencies without prioritizing direction.
        \item **Computational Considerations**: Bayesian Networks often require sophisticated inference algorithms, while Markov Networks may involve simpler local computations.
    \end{itemize}
\end{frame}

\begin{frame}[fragile]
    \frametitle{Summary}
    Both Bayesian Networks and Markov Networks serve as powerful tools in probabilistic reasoning across various AI applications. Recognizing their structural differences aids in selecting appropriate models for specific challenges.
\end{frame}

\begin{frame}[fragile]
    \frametitle{Implementing Bayesian Networks in AI - Introduction}
    \begin{block}{Introduction to Bayesian Networks}
        Bayesian Networks (BNs) are powerful tools for representing and reasoning about uncertainty in AI systems. They model relationships between variables and allow us to compute probabilities, making them ideal for decision-making tasks.
    \end{block}
\end{frame}

\begin{frame}[fragile]
    \frametitle{Tools and Libraries for Implementing Bayesian Networks - Part 1}
    \begin{enumerate}
        \item \textbf{pgmpy}
        \begin{itemize}
            \item Description: A Python library specifically for probabilistic graphical models, including Bayesian Networks and Markov models.
            \item Key Features:
            \begin{itemize}
                \item Easy interface for constructing BNs
                \item Inference methods (e.g., variable elimination, belief propagation)
                \item Parameter learning from data
            \end{itemize}
            \item Example:
            \begin{lstlisting}[language=Python]
from pgmpy.models import BayesianModel
model = BayesianModel([('A', 'C'), ('B', 'C')])
model.add_cpds(cpd_A, cpd_B, cpd_C)  # Add Conditional Probability Distributions
            \end{lstlisting}
        \end{itemize}
        
        \item \textbf{BayesPy}
        \begin{itemize}
            \item Description: A flexible library for Bayesian inference in Python, particularly useful for more complex models and various inference algorithms.
            \item Key Features:
            \begin{itemize}
                \item Supports different types of variables and full Bayesian modeling
                \item Provides a graphical representation of models
            \end{itemize}
            \item Example:
            \begin{lstlisting}[language=Python]
import bayespy as bp
# Model and inference can be setup using BayesPy's structures and updates
            \end{lstlisting}
        \end{itemize}
    \end{enumerate}
\end{frame}

\begin{frame}[fragile]
    \frametitle{Tools and Libraries for Implementing Bayesian Networks - Part 2}
    \begin{enumerate}[resume]
        \item \textbf{Netica}
        \begin{itemize}
            \item Description: A commercial software tool for graphical probabilistic models with a user-friendly interface.
            \item Key Features:
            \begin{itemize}
                \item Graphical user interface to model BNs
                \item Extensive documentation and support
                \item Can be integrated into other programming environments (like Python)
            \end{itemize}
            \item Use Cases: Often utilized in industries for risk analysis, medical diagnosis, etc.
        \end{itemize}

        \item \textbf{Hugin}
        \begin{itemize}
            \item Description: Software for building, managing, and running BNs, it offers both a graphical interface and an API for programmatic access.
            \item Key Features:
            \begin{itemize}
                \item Efficient inference algorithms
                \item Can handle large networks efficiently
            \end{itemize}
        \end{itemize}

        \item \textbf{BNT (Bayesian Network Toolbox)}
        \begin{itemize}
            \item Description: A MATLAB toolbox for Bayesian networks, suitable for statistical analysis and research.
            \item Key Features:
            \begin{itemize}
                \item Implements multiple types of inference and learning algorithms
                \item User-friendly functions to create and manipulate BNs
            \end{itemize}
        \end{itemize}
    \end{enumerate}
\end{frame}

\begin{frame}[fragile]
    \frametitle{Key Considerations and Closing Notes}
    \begin{block}{Key Considerations When Choosing a Tool}
        \begin{itemize}
            \item \textbf{Complexity of the Model}: Choose different libraries based on simple or complex networks.
            \item \textbf{Data Sources}: Ease of integrating external data sources for inference and learning should be considered.
            \item \textbf{User Interface}: Some tools come with GUIs which might ease the learning curve.
            \item \textbf{Community and Support}: Popular libraries usually have wider community support and documentation.
        \end{itemize}
    \end{block}
    
    \begin{block}{Closing Notes}
        Bayesian networks provide a robust framework for reasoning under uncertainty. Selecting the right tool or library greatly impacts the efficiency and effectiveness of your implementation. Experimenting with multiple libraries could help in finding the best fit for your specific application in AI.
    \end{block}
\end{frame}

\begin{frame}
    \frametitle{Overview of Bayesian Networks}
    \begin{block}{Definition}
        Bayesian Networks (BNs) are graphical models that represent a set of variables and their conditional dependencies via a directed acyclic graph (DAG). Each node represents a random variable, while edges denote the dependencies between them.
    \end{block}
\end{frame}

\begin{frame}
    \frametitle{Applications of Bayesian Networks - Fraud Detection}
    \begin{itemize}
        \item \textbf{Concept}: BNs help identify fraudulent activities by evaluating relationships and probabilities in financial transactions.
        \item \textbf{Example}: 
        \begin{itemize}
            \item An online payment system notices a sudden large transaction from a different geographical location after a history of small transactions.
            \item The Bayesian network computes the probability of fraud based on past transaction patterns and alerts administrators.
        \end{itemize}
        \item \textbf{Key Point}: Adjusting belief thresholds allows organizations to update risk assessments as more data becomes available.
    \end{itemize}
\end{frame}

\begin{frame}
    \frametitle{Applications of Bayesian Networks - Risk Management}
    \begin{itemize}
        \item \textbf{Concept}: BNs model uncertainties in various risk factors (e.g., market and credit risks) to support decision-making.
        \item \textbf{Example}: 
        \begin{itemize}
            \item In insurance, a BN evaluates the likelihood of a claim based on factors like demographics, geographical data, and historical claims.
            \item This modeling helps insurers set premiums according to individual risk profiles.
        \end{itemize}
        \item \textbf{Key Point}: Decision-makers use BNs to simulate scenarios, enabling informed risk mitigation strategies.
    \end{itemize}
\end{frame}

\begin{frame}
    \frametitle{Key Features of Bayesian Networks}
    \begin{itemize}
        \item \textbf{Probabilistic in Nature}: Deal with uncertainty and reason under it.
        \item \textbf{Graphical Representation}: Simplifies modeling complex relationships.
        \item \textbf{Inference Capabilities}: Update beliefs with new evidence to enable dynamic decision-making.
    \end{itemize}
\end{frame}

\begin{frame}[fragile]
    \frametitle{Example Code Snippet (Python, pgmpy)}
    \begin{lstlisting}[language=Python]
from pgmpy.models import BayesianNetwork

# Define the structure of the Bayesian Network
model = BayesianNetwork([('Transaction_Amount', 'Fraud'),
                          ('Geographic_Location', 'Fraud'),
                          ('User_History', 'Fraud')])

# Define the Conditional Probability Tables (CPTs)
cpt_fraud =  ...
# Add CPDs to the model
model.add_cpds(cpt_fraud, ...)
    \end{lstlisting}
\end{frame}

\begin{frame}
    \frametitle{Conclusion}
    Bayesian networks are powerful tools in various fields, including finance, healthcare, and engineering. They model uncertainties efficiently and support data-driven decisions in complex variable relationships. Understanding their application contexts enhances their utility in solving real-world problems.
\end{frame}

\begin{frame}[fragile]
    \frametitle{Ethical Considerations in Probabilistic Reasoning and Bayesian Networks}
    \begin{block}{Introduction}
        Ethical considerations are paramount in Artificial Intelligence (AI), particularly concerning Bayesian networks and probabilistic reasoning. These tools are increasingly used in decision-making processes across various domains, warranting a thorough analysis of their implications.
    \end{block}
\end{frame}

\begin{frame}[fragile]
    \frametitle{Key Ethical Concerns}
    \begin{enumerate}
        \item \textbf{Bias and Fairness}
            \begin{itemize}
                \item Models can perpetuate biases from training data, leading to unfair outcomes.
                \item \textit{Example:} Credit scoring models may disadvantage certain demographic groups.
            \end{itemize}
        \item \textbf{Transparency}
            \begin{itemize}
                \item Complexity of Bayesian networks can lead to a lack of understanding of decision-making processes.
                \item \textit{Example:} Healthcare algorithms must be transparent to build trust with patients and doctors.
            \end{itemize}
        \item \textbf{Data Privacy}
            \begin{itemize}
                \item Use of personal data raises privacy concerns; misuse can lead to ethical violations.
                \item \textit{Example:} Predictive models based on demographic data can expose individuals without consent.
            \end{itemize}
        \item \textbf{Responsibility and Accountability}
            \begin{itemize}
                \item Accountability for AI decisions can be complex, particularly when decisions impact lives.
                \item \textit{Example:} Questions arise regarding accountability for community policing algorithms.
            \end{itemize}
    \end{enumerate}
\end{frame}

\begin{frame}[fragile]
    \frametitle{Mitigation Strategies and Conclusion}
    \begin{block}{Mitigation Strategies}
        \begin{enumerate}
            \item \textbf{Bias Detection and Correction} - Identify and rectify biases in datasets before deployment.
            \item \textbf{Explainable AI (XAI)} - Develop models that enhance transparency for better understanding.
            \item \textbf{User Consent and Data Governance} - Obtain informed consent and establish strict data protocols.
            \item \textbf{Collaboration with Ethicists} - Involve ethicists to address ethical dilemmas in AI development.
        \end{enumerate}
    \end{block}

    \begin{block}{Conclusion}
        Engaging with these ethical considerations is essential for responsible AI development. Prioritizing bias mitigation, ensuring transparency, respecting privacy, and clarifying accountability navigates ethical complexities in AI's application.
    \end{block}
\end{frame}

\begin{frame}[fragile]
  \frametitle{Conclusion and Future Directions - Key Points}

  \begin{enumerate}
    \item \textbf{Understanding of Probabilistic Reasoning}:
    \begin{itemize}
      \item Involves using probability theory to reason about uncertainty.
      \item Facilitates predictions, belief updates, and rational decisions with incomplete information.
    \end{itemize}
    
    \item \textbf{Bayesian Networks}:
    \begin{itemize}
      \item Directed acyclic graphs (DAGs) representing variables and their conditional dependencies.
      \item Serve as graphical models for easier visualization of complex relationships.
    \end{itemize}
    
    \item \textbf{Applications}:
    \begin{itemize}
      \item Medical diagnosis, risk assessment, natural language processing, and machine learning.
    \end{itemize}
    
    \item \textbf{Key Concepts}:
    \begin{itemize}
      \item \textbf{Bayes' Theorem}:
      \begin{equation}
      P(H|E) = \frac{P(E|H) \cdot P(H)}{P(E)}
      \end{equation}
      \item \textbf{Inference}: Use observed data to update beliefs.
      \item \textbf{Learning}: Methods like Maximum Likelihood Estimation (MLE) and Bayesian Estimation to adjust network parameters.
    \end{itemize}
  \end{enumerate}
\end{frame}

\begin{frame}[fragile]
  \frametitle{Conclusion and Future Directions - Future Trends}

  \begin{enumerate}
    \item \textbf{Integration with Deep Learning}:
    \begin{itemize}
      \item Merging probabilistic methods with deep learning for better interpretability and uncertainty quantification.
    \end{itemize}

    \item \textbf{Scalability}:
    \begin{itemize}
      \item Researching enhancements for efficient computation in large Bayesian networks.
    \end{itemize}

    \item \textbf{Explainable AI (XAI)}:
    \begin{itemize}
      \item Promoting interpretable models aligned with ethical AI practices.
    \end{itemize}

    \item \textbf{Healthcare Innovations}:
    \begin{itemize}
      \item Use in personalized medicine and predictive analytics for tailored treatments based on individual data.
    \end{itemize}

    \item \textbf{Automated Learning Algorithms}:
    \begin{itemize}
      \item Advancements supporting real-time updates in Bayesian networks, enhancing applicability in dynamic environments.
    \end{itemize}
  \end{enumerate}
\end{frame}

\begin{frame}[fragile]
  \frametitle{Conclusion and Future Directions - Overall Conclusion}

  \begin{block}{Conclusion}
    The field of probabilistic reasoning and Bayesian networks continues to evolve, offering significant research and application avenues. Understanding its concepts and potentials enhances AI capabilities and supports ethical AI development.
  \end{block}
\end{frame}


\end{document}