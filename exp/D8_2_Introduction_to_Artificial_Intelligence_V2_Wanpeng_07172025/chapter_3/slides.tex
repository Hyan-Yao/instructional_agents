\documentclass[aspectratio=169]{beamer}

% Theme and Color Setup
\usetheme{Madrid}
\usecolortheme{whale}
\useinnertheme{rectangles}
\useoutertheme{miniframes}

% Additional Packages
\usepackage[utf8]{inputenc}
\usepackage[T1]{fontenc}
\usepackage{graphicx}
\usepackage{booktabs}
\usepackage{listings}
\usepackage{amsmath}
\usepackage{amssymb}
\usepackage{xcolor}
\usepackage{tikz}
\usepackage{pgfplots}
\pgfplotsset{compat=1.18}
\usetikzlibrary{positioning}
\usepackage{hyperref}

% Custom Colors
\definecolor{myblue}{RGB}{31, 73, 125}
\definecolor{mygray}{RGB}{100, 100, 100}
\definecolor{mygreen}{RGB}{0, 128, 0}
\definecolor{myorange}{RGB}{230, 126, 34}
\definecolor{mycodebackground}{RGB}{245, 245, 245}

% Set Theme Colors
\setbeamercolor{structure}{fg=myblue}
\setbeamercolor{frametitle}{fg=white, bg=myblue}
\setbeamercolor{title}{fg=myblue}
\setbeamercolor{section in toc}{fg=myblue}
\setbeamercolor{item projected}{fg=white, bg=myblue}
\setbeamercolor{block title}{bg=myblue!20, fg=myblue}
\setbeamercolor{block body}{bg=myblue!10}
\setbeamercolor{alerted text}{fg=myorange}

% Set Fonts
\setbeamerfont{title}{size=\Large, series=\bfseries}
\setbeamerfont{frametitle}{size=\large, series=\bfseries}
\setbeamerfont{caption}{size=\small}
\setbeamerfont{footnote}{size=\tiny}

% Code Listing Style
\lstdefinestyle{customcode}{
  backgroundcolor=\color{mycodebackground},
  basicstyle=\footnotesize\ttfamily,
  breakatwhitespace=false,
  breaklines=true,
  commentstyle=\color{mygreen}\itshape,
  keywordstyle=\color{blue}\bfseries,
  stringstyle=\color{myorange},
  numbers=left,
  numbersep=8pt,
  numberstyle=\tiny\color{mygray},
  frame=single,
  framesep=5pt,
  rulecolor=\color{mygray},
  showspaces=false,
  showstringspaces=false,
  showtabs=false,
  tabsize=2,
  captionpos=b
}
\lstset{style=customcode}

% Custom Commands
\newcommand{\hilight}[1]{\colorbox{myorange!30}{#1}}
\newcommand{\source}[1]{\vspace{0.2cm}\hfill{\tiny\textcolor{mygray}{Source: #1}}}
\newcommand{\concept}[1]{\textcolor{myblue}{\textbf{#1}}}
\newcommand{\separator}{\begin{center}\rule{0.5\linewidth}{0.5pt}\end{center}}

% Footer and Navigation Setup
\setbeamertemplate{footline}{
  \leavevmode%
  \hbox{%
  \begin{beamercolorbox}[wd=.3\paperwidth,ht=2.25ex,dp=1ex,center]{author in head/foot}%
    \usebeamerfont{author in head/foot}\insertshortauthor
  \end{beamercolorbox}%
  \begin{beamercolorbox}[wd=.5\paperwidth,ht=2.25ex,dp=1ex,center]{title in head/foot}%
    \usebeamerfont{title in head/foot}\insertshorttitle
  \end{beamercolorbox}%
  \begin{beamercolorbox}[wd=.2\paperwidth,ht=2.25ex,dp=1ex,center]{date in head/foot}%
    \usebeamerfont{date in head/foot}
    \insertframenumber{} / \inserttotalframenumber
  \end{beamercolorbox}}%
  \vskip0pt%
}

% Turn off navigation symbols
\setbeamertemplate{navigation symbols}{}

% Title Page Information
\title[Academic Template]{Week 6-7: Multi-Agent Search and Game Playing}
\author[J. Smith]{John Smith, Ph.D.}
\institute[University Name]{
  Department of Computer Science\\
  University Name\\
  \vspace{0.3cm}
  Email: email@university.edu\\
  Website: www.university.edu
}
\date{\today}

% Document Start
\begin{document}

\frame{\titlepage}

\begin{frame}[fragile]
    \frametitle{Introduction to Multi-Agent Search and Game Playing}
    \begin{itemize}
        \item Overview of adversarial search 
        \item Core principles of game playing 
        \item Examples of multi-agent games 
        \item Importance of strategy 
        \item Key points to emphasize 
    \end{itemize}
\end{frame}

\begin{frame}[fragile]
    \frametitle{Overview of Adversarial Search in AI}
    \begin{block}{Adversarial Search}
        A type of search used in scenarios where multiple agents compete against one another, leading to what is known as a zero-sum game.
    \end{block}
    \begin{itemize}
        \item \textbf{Definition}: A situation in which one agent's gain is equivalent to another agent's loss.
        \item \textbf{Key Concepts}:
            \begin{itemize}
                \item \textbf{Agents}: Entities (e.g., players) that make decisions based on the current state of the game.
                \item \textbf{States}: Configurations of the game at any given time.
                \item \textbf{Actions}: Moves made by agents to transition from one state to another.
            \end{itemize}
    \end{itemize}
\end{frame}

\begin{frame}[fragile]
    \frametitle{Core Principles of Game Playing}
    \begin{itemize}
        \item \textbf{Minimax Algorithm}:
            \begin{itemize}
                \item A decision-making algorithm to minimize possible loss in a worst-case scenario.
                \item \textbf{Mechanism}: Evaluates nodes in a game tree to determine optimal plays.
                \item \textbf{Formula}:
                \begin{equation}
                \text{Value}(N) = \max_{\text{child}(N)} \text{Value}(\text{child}(N)) \text{ (Maximizing move)}
                \end{equation}
                \begin{equation}
                \text{Value}(N) = \min_{\text{child}(N)} \text{Value}(\text{child}(N)) \text{ (Minimizing move)}
                \end{equation}
            \end{itemize}
        \item \textbf{Alpha-Beta Pruning}:
            \begin{itemize}
                \item An optimization technique for the minimax algorithm.
                \item \textbf{Key Idea}: Eliminate branches in the search tree that won't influence the final decision, improving efficiency.
            \end{itemize}
    \end{itemize}
\end{frame}

\begin{frame}{Presentation Overview}
  \tableofcontents[hideallsubsections]
\end{frame}

\begin{frame}[fragile]
    \frametitle{Importance of Game Theory in AI}
    \begin{block}{Overview of Game Theory}
        Game theory is a mathematical framework that studies strategic interactions among rational decision-makers. In the context of AI, it models and analyzes scenarios where agents (players) make decisions affecting their own outcomes and those of others.
    \end{block}
\end{frame}

\begin{frame}[fragile]
    \frametitle{Key Concepts}
    \begin{itemize}
        \item \textbf{Players}: Individuals or entities making decisions (e.g., two opponents in chess).
        \item \textbf{Strategies}: Plans of action available to players (e.g., different moves in a game).
        \item \textbf{Payoffs}: Outcomes resulting from players' strategies, often represented in a matrix.
        \item \textbf{Nash Equilibrium}: A situation in a game where no player can benefit from changing their strategy while others hold constant.
    \end{itemize}
\end{frame}

\begin{frame}[fragile]
    \frametitle{Role of Game Theory in AI}
    \begin{itemize}
        \item \textbf{Strategic Decision-Making}: AI can use game theory to anticipate competitors' moves, especially in economics, auctions, and competitive games.
        
        \item \textbf{Multi-Agent Systems}: Game theory informs decision-making among multiple autonomous agents, such as in cooperation or competition.
    \end{itemize}
\end{frame}

\begin{frame}[fragile]
    \frametitle{Examples in AI Applications}
    \begin{itemize}
        \item \textbf{Robotics}: Coordinating multiple robots to achieve common goals (e.g., delivery, exploration) without interference.
        
        \item \textbf{Economics and Auctions}: Analyzing bidding strategies to predict competitors' behavior for maximizing revenue.
        
        \item \textbf{Games}: Algorithms for games like Chess or Go use game theory to predict opponent moves and devise strategies.
    \end{itemize}
\end{frame}

\begin{frame}[fragile]
    \frametitle{Illustrative Example: Tic-Tac-Toe}
    \begin{itemize}
        \item \textbf{Players}: X and O.
        \item \textbf{Strategies}: Each player has multiple possible moves per turn.
        \item \textbf{Payoffs}: Winning (1), Losing (-1), or Draw (0).
    \end{itemize}
    \begin{block}{Nash Equilibrium in Tic-Tac-Toe}
        If both players play optimally, the game results in a draw, reflecting a stable state where neither player can unilaterally change their strategy for a better outcome.
    \end{block}
\end{frame}

\begin{frame}[fragile]
    \frametitle{Conclusion and Key Points}
    Incorporating game theory into AI enhances agents' ability to make rational decisions in competitive environments. Agents systematically evaluate potential outcomes and strategies, leading to more robust AI applications.
    
    \begin{block}{Key Points to Emphasize}
        \begin{itemize}
            \item Game theory serves as the backbone for strategic interaction in AI applications.
            \item Understanding player behavior and strategy optimization is crucial for effective AI decision-making.
            \item Real-world applications extend beyond games to include economics, robotics, and social interactions.
        \end{itemize}
    \end{block}
\end{frame}

\begin{frame}[fragile]
    \frametitle{Nash Equilibrium Example}
    \begin{equation}
    \begin{array}{c|c|c}
    & O (Cooperate) & O (Defect) \\
    \hline
    X (Cooperate) & (2,2) & (0,3) \\
    \hline
    X (Defect) & (3,0) & (1,1) \\
    \end{array}
    \end{equation}
    In this matrix, each cell indicates the payoff for players X and O based on their respective strategies.
\end{frame}

\begin{frame}[fragile]
    \frametitle{Adversarial Search Fundamentals}
    % Introduction to adversarial search
    Adversarial search is a critical concept in artificial intelligence, particularly in game playing and decision making, involving multiple agents with potentially opposing goals.
\end{frame}

\begin{frame}[fragile]
    \frametitle{Key Concepts - Players and Strategies}
    \begin{block}{Players}
        - In adversarial games, players are the agents making decisions.\\
        - Each player aims to maximize their chances of winning while minimizing the opponent's chances.\\
        - \textbf{Example:} In chess, the two players (White and Black) have opposing objectives—checkmating the opponent's king.
    \end{block}
    
    \begin{block}{Strategies}
        - A strategy defines the approach a player takes to achieve their goal:
            \begin{itemize}
                \item \textbf{Deterministic:} Fixed actions based on predetermined rules.
                \item \textbf{Randomized:} Using probability to make decisions, which can avoid predictability.
            \end{itemize}
        - \textbf{Example:} In Tic-Tac-Toe, a player might choose a strategy of always taking the center square if available.
    \end{block}
\end{frame}

\begin{frame}[fragile]
    \frametitle{Key Concepts - Outcomes and Importance}
    \begin{block}{Outcomes}
        - Outcomes of an adversarial game can be classified into:
            \begin{enumerate}
                \item \textbf{Win:} One player achieves the objective and wins.
                \item \textbf{Loss:} The opponent successfully achieves their goal, resulting in a loss.
                \item \textbf{Draw:} Neither player can claim victory, common in perfect information games.
            \end{enumerate}
        - \textbf{Example:} In connect-four, if neither player can connect four pieces and the board is full, the game ends in a draw.
    \end{block}

    \begin{block}{Importance of Adversarial Search}
        - Adversarial search algorithms evaluate and choose moves based on possible future game states, predicting and countering an opponent's possible moves.
        - Commonly used in strategic games like chess, checkers, and Go.
    \end{block}
\end{frame}

\begin{frame}[fragile]
    \frametitle{Minimax Algorithm - Overview}
    \begin{block}{Definition}
        The \textbf{Minimax Algorithm} is a decision-making strategy used in two-player games to minimize the possible loss in a worst-case scenario. This algorithm is crucial for players in determining the best possible move under the assumption that their opponent plays optimally.
    \end{block}
\end{frame}

\begin{frame}[fragile]
    \frametitle{Minimax Algorithm - Key Concepts}
    \begin{itemize}
        \item \textbf{Players}:
        \begin{itemize}
            \item Maximizer: Aims to maximize their score (represented as ‘Max’).
            \item Minimizer: Aims to minimize the score of the maximizer (represented as ‘Min’).
        \end{itemize}
        
        \item \textbf{Game Tree}:
        \begin{itemize}
            \item A graphical representation of all possible moves in a game.
            \item Each node represents a game state with branches leading to subsequent game states.
        \end{itemize}
    \end{itemize}
\end{frame}

\begin{frame}[fragile]
    \frametitle{Minimax Algorithm - How It Works}
    \begin{enumerate}
        \item \textbf{Tree Construction}:
        \begin{itemize}
            \item Build a game tree starting from the current state where each level represents a player’s turn.
        \end{itemize}
        
        \item \textbf{Leaf Node Evaluation}:
        \begin{itemize}
            \item Evaluate the game outcome at leaf nodes (e.g., win, lose, draw) and assign numeric scores (e.g., +1 for win).
        \end{itemize}
        
        \item \textbf{Backpropagation of Values}:
        \begin{itemize}
            \item Propagate values back up the tree:
            \begin{itemize}
                \item Max selects the maximum value.
                \item Min selects the minimum value.
            \end{itemize}
        \end{itemize}
        
        \item \textbf{Optimal Move Selection}:
        \begin{itemize}
            \item The root will have the optimal value, guiding Max in move selection.
        \end{itemize}
    \end{enumerate}
\end{frame}

\begin{frame}[fragile]
    \frametitle{Minimax Algorithm - Example Visualization}
    Consider a simplified game decision represented in a tree:
    \begin{center}
    \begin{verbatim}
             Max
            /   \
          Min   Min
         / \     / \
        3   5   2   9
    \end{verbatim}
    \end{center}
    
    \begin{itemize}
        \item Leaf Nodes (evaluated scores): [3, 5] for left and [2, 9] for right.
        \item Min selects:
        \begin{itemize}
            \item Left: \textit{min}(3, 5) = 3
            \item Right: \textit{min}(2, 9) = 2
        \end{itemize}
        \item Max selects \textit{max}(3, 5) = 5.
    \end{itemize}
    Hence, the optimal move leads to the branch with value 5.
\end{frame}

\begin{frame}[fragile]
    \frametitle{Minimax Algorithm - Key Points}
    \begin{itemize}
        \item Assumes optimal and rational play from both players.
        \item Computationally intensive due to large branching factors.
        \item Includes enhancements like \textbf{Alpha-Beta Pruning} to improve efficiency by eliminating unnecessary branches.
    \end{itemize}
\end{frame}

\begin{frame}[fragile]
    \frametitle{Minimax Algorithm - Pseudocode}
    \begin{lstlisting}[language=Python]
function minimax(node, depth, isMaximizingPlayer):
    if (node is a terminal node or depth is 0):
        return evaluate(node)

    if (isMaximizingPlayer):
        bestValue = -∞
        for each child of node:
            bestValue = max(bestValue, minimax(child, depth - 1, false))
        return bestValue
    else:
        bestValue = +∞
        for each child of node:
            bestValue = min(bestValue, minimax(child, depth - 1, true))
        return bestValue
    \end{lstlisting}
\end{frame}

\begin{frame}[fragile]
    \frametitle{Alpha-Beta Pruning}
    \begin{block}{Overview}
        Alpha-Beta pruning is an optimization technique for the minimax algorithm used in decision-making processes for two-player games. It eliminates portions of the search tree, reducing the number of nodes evaluated.
    \end{block}
\end{frame}

\begin{frame}[fragile]
    \frametitle{Key Concepts}
    \begin{itemize}
        \item \textbf{Minimax Algorithm Recap}: 
          A recursive approach where the maximizer tries to maximize their score while the minimizer seeks to minimize it.
        \item \textbf{Alpha and Beta Values}:
          \begin{itemize}
            \item \textbf{Alpha ($\alpha$)}: The best value that the maximizer can guarantee at that level or above. Initially set to $-\infty$.
            \item \textbf{Beta ($\beta$)}: The best value that the minimizer can guarantee at that level or below. Initially set to $+\infty$.
          \end{itemize}
    \end{itemize}
\end{frame}

\begin{frame}[fragile]
    \frametitle{How Alpha-Beta Pruning Works}
    \begin{enumerate}
        \item \textbf{Node Evaluation}: As nodes are evaluated in the tree, $\alpha$ and $\beta$ values are updated.
        \item \textbf{Pruning Decision}: If at any point $\beta \leq \alpha$, further exploration of that branch is not needed, as it cannot influence the final decision.
    \end{enumerate}
\end{frame}

\begin{frame}[fragile]
    \frametitle{Efficiency and Example}
    \begin{block}{Efficiency}
        \begin{itemize}
            \item \textbf{Without Pruning}: The minimax algorithm has a worst-case time complexity of $O(b^d)$, where $b$ is the branching factor and $d$ is the depth of the tree.
            \item \textbf{With Alpha-Beta Pruning}: The effective branching factor can be reduced to $O(b^{(d/2)})$, significantly improving performance.
        \end{itemize}
    \end{block}

    \begin{block}{Example}
        \includegraphics[width=\textwidth]{example_tree.png} % Placeholder for diagram
        % Replace with actual code to include an illustrative diagram
    \end{block}
\end{frame}

\begin{frame}[fragile]
    \frametitle{Conclusion}
    \begin{block}{Key Points}
        \begin{itemize}
            \item Alpha-Beta pruning enhances the minimax algorithm's efficiency.
            \item It does not compromise the optimality of results, but significantly reduces unnecessary calculations.
            \item Understanding pruning applications is crucial for programmers working with strategic AI.
        \end{itemize}
    \end{block}
\end{frame}

\begin{frame}
  \frametitle{Evaluation Functions}
  \begin{block}{Overview}
    Discussing the importance of evaluation functions in estimating game positions.
  \end{block}
\end{frame}

\begin{frame}
  \frametitle{Understanding Evaluation Functions}
  \begin{itemize}
    \item \textbf{What are Evaluation Functions?} 
    \begin{itemize}
      \item Algorithms that assess and estimate the desirability of game positions.
      \item Provide a numerical value indicating how favorable a position is for a player.
    \end{itemize}
    
    \item \textbf{Why are Evaluation Functions Important?}
    \begin{itemize}
      \item Complexity Reduction: Helps intelligently prune moves in vast search spaces.
      \item Heuristic Guidance: Guides search algorithms (e.g., Minimax with alpha-beta pruning) by scoring game states.
    \end{itemize}
  \end{itemize}
\end{frame}

\begin{frame}
  \frametitle{Key Components of Evaluation Functions}
  \begin{enumerate}
    \item \textbf{Game-specific Heuristics}
    \begin{itemize}
      \item Different games need tailored evaluation strategies:
      \begin{itemize}
        \item In chess: material advantage, center control, king safety.
        \item In checkers: center control, piece mobility.
      \end{itemize}
    \end{itemize}

    \item \textbf{Score Representation}
    \begin{itemize}
      \item Positive values favor the maximizing player.
      \item Negative values favor the minimizing player.
      \item A score of zero indicates a neutral position.
    \end{itemize}
  \end{enumerate}
\end{frame}

\begin{frame}[fragile]
  \frametitle{Example: Chess Evaluation Function}
  \begin{block}{Sample Code}
    \begin{lstlisting}[language=Python]
def evaluate_chess_position(board):
    score = 0

    piece_values = {
        "p": 1,  # pawn
        "N": 3,  # knight
        "B": 3,  # bishop
        "R": 5,  # rook
        "Q": 9,  # queen
        "K": 0   # king (though king safety is crucial)
    }

    for piece in board.pieces():
        score += piece_values.get(piece.type, 0) * (1 if piece.color == "white" else -1)

    return score
    \end{lstlisting}
  \end{block}
  
  \begin{itemize}
    \item This function evaluates the board by summing up piece values, showing how heuristics can be coded into evaluation functions.
  \end{itemize}
\end{frame}

\begin{frame}
  \frametitle{Key Points to Emphasize}
  \begin{itemize}
    \item \textbf{Efficiency:} Transforms complex search problems into manageable tasks.
    \item \textbf{Customization:} Tailored functions enhance AI decision-making.
    \item \textbf{Not Perfect:} Approximations may fail to capture complete positional values; nuanced positions may require deeper analysis.
  \end{itemize}
\end{frame}

\begin{frame}
  \frametitle{Conclusion}
  \begin{block}{Summary}
    Evaluation functions are crucial for strategic decision-making in game AI. They simplify decision processes, enabling efficient computation and effective gameplay strategies.
  \end{block}
\end{frame}

\begin{frame}[fragile]
  \frametitle{Game Tree Representation - Overview}
  \begin{itemize}
    \item \textbf{Definition}: A game tree is a structured representation of all possible moves in a game from the perspective of players and their potential responses.
    \item \textbf{Structure}: 
    \begin{itemize}
      \item Each node represents a game state.
      \item The edges represent the possible moves.
      \item The root node represents the initial state of the game.
    \end{itemize}
  \end{itemize}
\end{frame}

\begin{frame}[fragile]
  \frametitle{Game Tree Representation - Components}
  \begin{enumerate}
    \item \textbf{Nodes}: 
      \begin{itemize}
        \item \textbf{Decision Nodes}: Where players make choices (e.g., player 1's turn).
        \item \textbf{Leaf Nodes}: Represent terminal states where the game has ended (e.g., win, lose, draw).
      \end{itemize}
    
    \item \textbf{Branches}: Connections representing the moves between states.

    \item \textbf{Levels}: The depth indicates the number of turns taken, alternating between players.
  \end{enumerate}
\end{frame}

\begin{frame}[fragile]
  \frametitle{Game Tree Representation - Example and Utilization}
  \textbf{Example: Tic-Tac-Toe}
  \begin{itemize}
    \item \textbf{Initial State}: The root node (empty board).
    \item \textbf{First Layers}: Player X makes a move, leading to multiple new nodes for each possible state.
    \item \textbf{Growth}: The tree expands exponentially as players continue until a win/lose/draw condition is met.
  \end{itemize}
  
  \textbf{Utilization in Search Strategies}
  \begin{itemize}
    \item \textbf{Minimax Algorithm}: A strategy to determine optimal moves by simulating future moves.
    \item \textbf{Alpha-Beta Pruning}: An optimization to reduce the number of nodes evaluated in the minimax algorithm by eliminating branches that won’t influence the decision.
  \end{itemize}
\end{frame}

\begin{frame}[fragile]
    \frametitle{Types of Games}
    \begin{block}{Classification of Games}
        Games can be classified into different categories based on various characteristics. Understanding these classifications helps us develop appropriate strategies and algorithms for game-playing AI. 
    \end{block}
    \begin{itemize}
        \item Deterministic vs. Stochastic Games
        \item Zero-Sum vs. Non-Zero-Sum Games
    \end{itemize}
\end{frame}

\begin{frame}[fragile]
    \frametitle{1. Deterministic vs. Stochastic Games}
    \begin{itemize}
        \item \textbf{Deterministic Games:}
            \begin{itemize}
                \item \textbf{Definition:} The outcome is fully determined by initial conditions and players' choices, with no element of chance.
                \item \textbf{Example:} Chess, where outcomes depend purely on the players' decisions.
            \end{itemize}
        
        \item \textbf{Stochastic Games:}
            \begin{itemize}
                \item \textbf{Definition:} Include randomness in the outcomes via chance events or random variables.
                \item \textbf{Example:} Backgammon, where the roll of dice introduces chance, influencing gameplay.
            \end{itemize}
    \end{itemize}
\end{frame}

\begin{frame}[fragile]
    \frametitle{2. Zero-Sum vs. Non-Zero-Sum Games}
    \begin{itemize}
        \item \textbf{Zero-Sum Games:}
            \begin{itemize}
                \item \textbf{Definition:} One player's gain is exactly equal to another's loss. The total benefit remains constant.
                \item \textbf{Example:} Poker; gains and losses net to zero.
            \end{itemize}
        
        \item \textbf{Non-Zero-Sum Games:}
            \begin{itemize}
                \item \textbf{Definition:} The sum of outcomes can be greater or lesser than zero, allowing for mutual gains or losses.
                \item \textbf{Example:} The Prisoner's Dilemma allows for cooperative strategies that can lead to better outcomes.
            \end{itemize}
    \end{itemize}
    \begin{block}{Key Points to Remember}
        \begin{itemize}
            \item Deterministic games have predictable outcomes; stochastic games incorporate chance.
            \item Zero-sum games involve direct competition; non-zero-sum games can yield cooperative outcomes.
        \end{itemize}
    \end{block}
\end{frame}

\begin{frame}[fragile]
    \frametitle{Introduction to Multi-Agent Systems}
    \begin{block}{Definition}
        A multi-agent system (MAS) is a framework in which multiple autonomous entities (agents) interact to achieve specific goals. These interactions often involve coordination and competition, making MAS a rich area for research and application in artificial intelligence (AI) and operations research.
    \end{block}
\end{frame}

\begin{frame}[fragile]
    \frametitle{Key Concepts}
    \begin{itemize}
        \item \textbf{Agents:}
        \begin{itemize}
            \item An agent perceives its environment through sensors and acts upon it through actuators.
            \item Agents can be software programs (e.g., chatbots) or physical robots (e.g., drones).
        \end{itemize}
        \item \textbf{Coordination:}
        \begin{itemize}
            \item Agents work together to achieve a common goal.
            \item Example: In a warehouse, multiple robots coordinate to efficiently transport items.
        \end{itemize}
        \item \textbf{Competition:}
        \begin{itemize}
            \item Agents with conflicting goals can hinder each other's success.
            \item Example: In chess, players strategize to outwit their opponents.
        \end{itemize}
    \end{itemize}
\end{frame}

\begin{frame}[fragile]
    \frametitle{Types of Interactions}
    \begin{itemize}
        \item \textbf{Cooperative Interactions:}
        \begin{itemize}
            \item Agents collaborate to maximize a collective reward.
            \item Example: Traffic management where cars communicate to avoid congestion.
        \end{itemize}
        \item \textbf{Competitive Interactions:}
        \begin{itemize}
            \item Agents aim to maximize their own rewards at the expense of others.
            \item Example: In esports, teams compete against each other for victory.
        \end{itemize}
    \end{itemize}
\end{frame}

\begin{frame}[fragile]
    \frametitle{Applications of Multi-Agent Systems}
    \begin{itemize}
        \item \textbf{Robotics:} Collaborative robots (cobots) in manufacturing.
        \item \textbf{Transportation:} Autonomous vehicles coordinating traffic flow.
        \item \textbf{Simulation:} Used in social systems to model and predict behavior.
    \end{itemize}
\end{frame}

\begin{frame}[fragile]
    \frametitle{Key Points and Conclusion}
    \begin{itemize}
        \item The dual nature of interactions (coordination vs. competition) is fundamental in designing MAS.
        \item Understanding interaction dynamics aids in developing effective algorithms.
        \item MAS have widespread applications in various domains like games, robotics, and social networks.
    \end{itemize}
    \begin{block}{Conclusion}
        Multi-agent systems represent a critical domain in AI, encompassing a variety of interactions and applications.
    \end{block}
\end{frame}

\begin{frame}[fragile]
    \frametitle{Example Visual Representation}
    Imagine a visual representation of a warehouse environment with multiple robots:
    \begin{itemize}
        \item Some robots are colored green indicating \textbf{cooperation} (e.g., moving items together).
        \item Others are colored red indicating \textbf{competition} (e.g., racing to pick up the same item).
    \end{itemize}
\end{frame}

\begin{frame}[fragile]
    \frametitle{Next Steps}
    Next, we will delve into \textbf{Cooperative vs. Non-Cooperative Games}, where we will explore the implications of agent interactions in game theory contexts.
\end{frame}

\begin{frame}[fragile]
    \frametitle{Cooperative vs. Non-Cooperative Games - Introduction}
    In multi-agent systems, understanding how agents interact with one another is crucial. The distinction between cooperative and non-cooperative games is fundamental in game theory, shaping the strategies agents employ in various scenarios.
\end{frame}

\begin{frame}[fragile]
    \frametitle{Cooperative Games}
    
    \begin{block}{Definition}
        Cooperative games are scenarios in which players can benefit from cooperating with one another. They can negotiate and form binding agreements to achieve mutual gains.
    \end{block}
    
    \begin{itemize}
        \item \textbf{Collaboration}: Players work together to achieve shared goals.
        \item \textbf{Coalitions}: Formation of groups to enhance outcomes.
        \item \textbf{Payoff Allocation}: The total benefit is distributed among players based on pre-agreed terms.
    \end{itemize}

    \begin{block}{Examples}
        \begin{itemize}
            \item Team Sports: Players on a soccer team collaborate to score goals and win matches.
            \item Business Partnerships: Companies cooperate in a joint venture to develop new products, sharing resources and profits.
        \end{itemize}
    \end{block}
\end{frame}

\begin{frame}[fragile]
    \frametitle{Non-Cooperative Games}
    
    \begin{block}{Definition}
        Non-cooperative games are characterized by players acting in their self-interest without the possibility of forming alliances or binding agreements. Each player aims to maximize their own payoff, often at the expense of others.
    \end{block}

    \begin{itemize}
        \item \textbf{Independence}: Players make decisions independently.
        \item \textbf{Strategies}: Focus on individual strategies and anticipated responses from other players.
        \item \textbf{Equilibrium Concepts}: The Nash Equilibrium is a common solution concept where no player benefits from changing their strategy unilaterally.
    \end{itemize}

    \begin{block}{Examples}
        \begin{itemize}
            \item Auction Bidding: Each bidder tries to win an item for the lowest cost, often outbidding others without collusion.
            \item Traffic Flow: Drivers choose routes based on their interest in minimizing travel time, creating competition for the most efficient paths.
        \end{itemize}
    \end{block}

    \begin{block}{Illustrative Diagram}
        Consider two players, P1 and P2, choosing between two strategies, A and B:
        \begin{equation}
        \begin{array}{c|c|c}
            & \text{P2 A} & \text{P2 B} \\
            \hline
            \text{P1 A} & 3, 3 & 0, 4 \\
            \text{P1 B} & 4, 0 & 1, 1 
        \end{array}
        \end{equation}
    \end{block}
\end{frame}

\begin{frame}[fragile]
    \frametitle{Key Points to Emphasize}
    \begin{itemize}
        \item \textbf{Nature of Interactions}: Cooperative games focus on collaboration, while non-cooperative games highlight individual strategies.
        \item \textbf{Impact on Strategy}: The type of game influences how agents will behave and their possible outcomes.
        \item \textbf{Real-World Implications}: Understanding these concepts is essential in fields like economics, political science, and artificial intelligence, influencing how systems are designed and managed.
    \end{itemize}
    
    Understanding these differences lays the groundwork for exploring real-world applications of game-playing agents, as will be discussed in the next section.
\end{frame}

\begin{frame}[fragile]
  \frametitle{Real-World Applications of Game Playing Agents}
  \begin{block}{Introduction}
    Game playing agents, based on multi-agent systems and strategic reasoning, have notable applications across various industries. By leveraging AI and game theory principles, these agents enhance decision-making and efficiency in complex environments.
  \end{block}
\end{frame}

\begin{frame}[fragile]
  \frametitle{Key Applications of Game Playing Agents}
  \begin{enumerate}
    \item \textbf{Finance}
    \begin{itemize}
      \item \textbf{Algorithmic Trading}: Simulate market conditions to decide on optimal buy/sell strategies, competing against other traders’ algorithms in a non-cooperative game scenario.
      \item \textbf{Risk Management}: Analyze potential risks and devise strategies to mitigate them by modeling market behaviors and predicting competitor actions.
    \end{itemize}

    \item \textbf{Robotics}
    \begin{itemize}
      \item \textbf{Multi-Robot Coordination}: Collaborate to accomplish tasks in environments with limited resources or objectives.
      \item \textbf{Path Planning}: Dynamically adjust routes based on the actions of other robots to avoid collisions.
    \end{itemize}

    \item \textbf{Entertainment}
    \begin{itemize}
      \item \textbf{Video Games}: Adapt AI opponents in real-time for an engaging player experience.
      \item \textbf{Interactive Storytelling}: Dictate narrative outcomes based on player choices, creating unique experiences.
    \end{itemize}
  \end{enumerate}
\end{frame}

\begin{frame}[fragile]
  \frametitle{Examples and Conclusions}
  \begin{block}{Examples of Game Playing Agents}
    \begin{itemize}
      \item \textbf{Finance Example}: An agent uses reinforcement learning to adapt trading strategies based on real-time market fluctuations.
      \item \textbf{Robotics Example}: In a search and rescue operation, drones use cooperative game strategies to effectively cover areas and share information.
      \item \textbf{Entertainment Example}: An AI character in a role-playing game evaluates players' strategies and adapts its actions for enhanced challenge.
    \end{itemize}
  \end{block}

  \begin{block}{Conclusion}
    Game playing agents enhance decision-making, optimize operations, and create engaging experiences across various sectors. Understanding their applications can drive better strategies for future challenges.
  \end{block}
\end{frame}

\begin{frame}[fragile]
  \frametitle{Key Points to Remember}
  \begin{itemize}
    \item Game playing agents apply strategic reasoning to solve real-world problems.
    \item Applications extend across finance, robotics, and entertainment.
    \item These agents adapt to environments, enhancing both efficiency and engagement.
  \end{itemize}
  \begin{block}{Further Inquiry}
    Consider ethical implications arising from the deployment of game playing agents in these fields.
  \end{block}
\end{frame}

\begin{frame}[fragile]
  \frametitle{Ethical Considerations in Game AI}
  \begin{block}{Introduction to Ethical AI in Gaming}
    As AI technologies permeate various aspects of game playing, it is crucial to examine the ethical implications of their integration. The use of AI can enhance gameplay and make it more engaging, but it also poses significant concerns.
  \end{block}
\end{frame}

\begin{frame}[fragile]
  \frametitle{Key Concepts}
  \begin{enumerate}
    \item \textbf{Fairness in AI}
      \begin{itemize}
        \item \textbf{Definition}: Fairness refers to the unbiased performance of AI systems, ensuring no player is unjustly favored or disadvantaged.
        \item \textbf{Importance}: An unbiased game AI guarantees matches are decided by player skill rather than algorithmic advantage.
        \item \textbf{Example}: AI-driven matchmaking systems should create balanced matches based on player skill, avoiding favoritism for experienced players.
      \end{itemize}
      
    \item \textbf{Bias in AI}
      \begin{itemize}
        \item \textbf{Definition}: Bias refers to unjust discrimination arising from data or algorithmic design.
        \item \textbf{Causes}: Can stem from unrepresentative training data or flawed algorithm design.
        \item \textbf{Example}: If AI chooses strategies based on limited demographics, it might disadvantage innovative players.
      \end{itemize}
  \end{enumerate}
\end{frame}

\begin{frame}[fragile]
  \frametitle{Ethical Challenges}
  \begin{itemize}
    \item \textbf{Algorithmic Transparency}: Players need to understand AI decision-making to foster trust and fairness.
    \item \textbf{Data Privacy}: Player data must be handled with respect to privacy; safeguarding personal information is essential.
    \item \textbf{Impact on Player Experience}: AI should enhance gameplay; overly aggressive or unfair AI can lead to frustration.
  \end{itemize}
\end{frame}

\begin{frame}[fragile]
  \frametitle{Key Points to Emphasize}
  \begin{itemize}
    \item \textbf{Balance vs. Rigging}: Ensuring that AI enhances gameplay without manipulating outcomes.
    \item \textbf{Continuous Monitoring}: Ethical practices require ongoing assessments of AI behavior to address bias.
    \item \textbf{Inclusivity}: Developers should strive for diverse representations in AI design.
  \end{itemize}
\end{frame}

\begin{frame}[fragile]
  \frametitle{Conclusion}
  Ethical considerations in game AI are critical for ensuring a fair, inclusive, and enjoyable gaming experience. Developers must actively address issues of fairness and bias to create responsible and ethical AI systems.
\end{frame}

\begin{frame}[fragile]
  \frametitle{Additional Resources}
  \begin{itemize}
    \item \textbf{Articles on AI Ethics}: Research papers and articles exploring ethical AI practices in gaming.
    \item \textbf{AI Fairness Frameworks}: Guidelines and tools for assessing fairness in AI implementation.
  \end{itemize}
\end{frame}

\begin{frame}[fragile]
    \frametitle{Building a Game-Playing Agent}
    \begin{block}{Introduction to Game-Playing Agents}
        - Game-playing agents make intelligent decisions in a game environment. \\
        - They explore possible game states using search techniques to select optimal moves.
    \end{block}
\end{frame}

\begin{frame}[fragile]
    \frametitle{Steps to Develop a Game-Playing Agent}
    \begin{enumerate}
        \item Define the Game Environment
        \item Represent the Game State
        \item Define Possible Actions
        \item Evaluate Game States
        \item Implement Search Algorithm
        \item Make the Move
    \end{enumerate}
\end{frame}

\begin{frame}[fragile]
    \frametitle{1. Define the Game Environment}
    \begin{itemize}
        \item Identify the rules, objectives, and components of the game.
        \item Example: Tic-Tac-Toe
        \begin{itemize}
            \item Players: Two (X and O)
            \item Objective: Get three in a row (horizontally, vertically, or diagonally)
        \end{itemize}
    \end{itemize}
\end{frame}

\begin{frame}[fragile]
    \frametitle{2. Represent the Game State}
    \begin{itemize}
        \item Use a suitable data structure to represent the board and players.
        \item Example: 2D Array for Tic-Tac-Toe
        \begin{lstlisting}
        board = [
          ['X', 'O', 'X'],
          [' ', 'X', 'O'],
          ['O', ' ', ' ']
        ]
        \end{lstlisting}
    \end{itemize}
\end{frame}

\begin{frame}[fragile]
    \frametitle{3. Define Possible Actions}
    \begin{itemize}
        \item Create a function to generate valid moves.
        \item Example: Available positions on the Tic-Tac-Toe board:
        \begin{lstlisting}
        def available_moves(board):
            return [(i, j) for i in range(3) for j in range(3) if board[i][j] == ' ']
        \end{lstlisting}
    \end{itemize}
\end{frame}

\begin{frame}[fragile]
    \frametitle{4. Evaluate Game States}
    \begin{itemize}
        \item Implement a heuristic evaluation function.
        \item Example: Score function for Tic-Tac-Toe:
        \begin{lstlisting}
        def score(board):
            for line in winning_lines:
                if line.count('X') == 3:
                    return 10
                elif line.count('O') == 3:
                    return -10
            return 0
        \end{lstlisting}
    \end{itemize}
\end{frame}

\begin{frame}[fragile]
    \frametitle{5. Implement Search Algorithm}
    \begin{itemize}
        \item Use a search algorithm like Minimax to explore possible moves recursively.
        \item The algorithm simulates all moves and selects the best one.
        
        \begin{lstlisting}
        def minimax(board, depth, is_maximizing):
            score = evaluate(board)
            if score == 10 or score == -10:
                return score
            if is_maximizing:
                best_score = -infinity
                for move in available_moves(board):
                    board[move[0]][move[1]] = 'X'
                    best_score = max(best_score, minimax(board, depth + 1, False))
                    board[move[0]][move[1]] = ' '
                return best_score
            else:
                best_score = infinity
                for move in available_moves(board):
                    board[move[0]][move[1]] = 'O'
                    best_score = min(best_score, minimax(board, depth + 1, True))
                    board[move[0]][move[1]] = ' '
                return best_score
        \end{lstlisting}
    \end{itemize}
\end{frame}

\begin{frame}[fragile]
    \frametitle{6. Make the Move}
    \begin{itemize}
        \item Select the optimal move after evaluating all possibilities via the search algorithm.
        \item Example: Choose the move with the highest score for 'X' after running Minimax.
    \end{itemize}
\end{frame}

\begin{frame}[fragile]
    \frametitle{Key Points to Emphasize}
    \begin{itemize}
        \item Define the game environment clearly.
        \item Efficient algorithms like Minimax support optimal decision-making.
        \item Accurate evaluation of game states influences outcomes significantly.
    \end{itemize}
\end{frame}

\begin{frame}[fragile]
    \frametitle{Usage of This Agent}
    \begin{itemize}
        \item This agent can play games like Tic-Tac-Toe, Chess, or Checkers.
        \item Follow similar principles while adapting to the complexity of each game environment.
    \end{itemize}
\end{frame}

\begin{frame}[fragile]
    \frametitle{Next Steps}
    \begin{itemize}
        \item Explore Challenges in Adversarial Search.
        \item Understand limitations and improvements in game-playing agents.
    \end{itemize}
\end{frame}

\begin{frame}[fragile]
    \frametitle{Challenges in Adversarial Search - Introduction}
    \begin{block}{Introduction}
        Adversarial search is a crucial component of game playing in AI, where agents compete against one another. However, significant challenges arise in this context. Understanding these challenges can help improve the effectiveness of game-playing strategies and algorithms.
    \end{block}
\end{frame}

\begin{frame}[fragile]
    \frametitle{Challenges in Adversarial Search - Key Challenges}
    \begin{enumerate}
        \item \textbf{Complexity of the Game Tree}
            \begin{itemize}
                \item The game tree represents all possible moves, growing exponentially as games progress. 
                \item In chess, approximately 10 million positions arise after just a few moves, leading to billions of outcomes.
                \item \textbf{Key Point:} Depth and breadth of the tree drastically increase computational load.
            \end{itemize}

        \item \textbf{Computational Limits}
            \begin{itemize}
                \item Evaluating every possible move is often impractical due to the vastness of the game tree.
                \item Techniques like Alpha-Beta Pruning help disregard branches of the tree that won't influence the final decision.
                \item \textbf{Key Point:} Efficient search algorithms are vital for managing computational resources.
            \end{itemize}
    \end{enumerate}
\end{frame}

\begin{frame}[fragile]
    \frametitle{Challenges in Adversarial Search - More Key Challenges}
    \begin{enumerate}
        \setcounter{enumi}{2} % Start this frame with the third challenge
        \item \textbf{Uncertainty and Incomplete Information}
            \begin{itemize}
                \item Many games involve uncertainty, complicating the search process.
                \item In poker, players can't see others' hands, making evaluation of moves challenging.
                \item \textbf{Key Point:} Strategies must adapt to incomplete information, using probability and deception.
            \end{itemize}

        \item \textbf{Dynamic Nature of Games}
            \begin{itemize}
                \item Opponents can change tactics, requiring adaptive strategies that respond in real-time.
                \item In real-time strategy games, tactics may shift based on the game state.
                \item \textbf{Key Point:} The need for adaptability increases algorithm complexity.
            \end{itemize}
        
        \item \textbf{Multi-Agent Cooperation and Competition}
            \begin{itemize}
                \item Some games require handling both cooperative and competitive dynamics.
                \item In negotiation games, players balance between alliances and competition.
                \item \textbf{Key Point:} Understanding motivations of other agents is essential. 
            \end{itemize}
    \end{enumerate}
\end{frame}

\begin{frame}[fragile]
    \frametitle{Challenges in Adversarial Search - Conclusion & Approaches}
    \begin{block}{Conclusion}
        Addressing these challenges is pivotal for enhancing performance in multi-agent systems in game-playing. Implementing advanced search strategies optimizes decision-making, improving responsiveness and effectiveness.
    \end{block}
    
    \begin{block}{Suggested Approaches}
        \begin{itemize}
            \item \textbf{Heuristics and Evaluation Functions:} Guide the search and simplify decisions using domain-specific knowledge.
            \item \textbf{Monte Carlo Methods:} Use simulations of game outcomes to inform strategic choices under uncertainty.
            \item \textbf{Reinforcement Learning:} Enable agents to learn optimal play through experience instead of explicit programming.
        \end{itemize}
    \end{block}
\end{frame}

\begin{frame}[fragile]
    \frametitle{Future Trends in Game AI}
    \begin{block}{Introduction}
        Game Artificial Intelligence (AI) is evolving rapidly, reflecting advancements in technology and the changing landscape of game development and experiences. This section explores emerging trends and research directions in AI for gaming and their future implications.
    \end{block}
\end{frame}

\begin{frame}[fragile]
    \frametitle{Key Emerging Trends - Part 1}
    \begin{enumerate}
        \item \textbf{Enhanced Multi-Agent Systems}
            \begin{itemize}
                \item Future game AI will leverage multi-agent systems for collaborative or competitive interactions.
                \item Example: NPCs learning from each other and adjusting their strategies based on player behavior.
            \end{itemize}

        \item \textbf{Deep Reinforcement Learning (DRL)}
            \begin{itemize}
                \item Combines deep learning with reinforcement learning for optimal strategy learning.
                \item Example: OpenAI's AlphaZero mastering Chess and Go through self-play.
            \end{itemize}
    \end{enumerate}
\end{frame}

\begin{frame}[fragile]
    \frametitle{Key Emerging Trends - Part 2}
    \begin{enumerate}
        \setcounter{enumi}{2} % Continue numbering from previous frame
        \item \textbf{Procedural Content Generation (PCG)}
            \begin{itemize}
                \item AI creates game content automatically, providing unique gameplay experiences.
                \item Example: "No Man's Sky" and "Spelunky" feature procedurally generated environments.
            \end{itemize}

        \item \textbf{Improved Natural Language Processing (NLP)}
            \begin{itemize}
                \item NLP enhances player and NPC interactions through conversational interfaces.
                \item Example: RPGs where players verbally interact with characters for immersive storytelling.
            \end{itemize}

        \item \textbf{Ethical AI and Fairness}
            \begin{itemize}
                \item Ensures AI functions ethically, providing fair challenges and respecting player agency.
                \item Example: Algorithms to prevent "pay-to-win" scenarios, ensuring balanced gameplay.
            \end{itemize}
    \end{enumerate}
\end{frame}

\begin{frame}[fragile]
    \frametitle{Implications for Game Development}
    \begin{itemize}
        \item \textbf{Player Engagement:} Advanced AI systems enhance interactions, keeping players invested.
        \item \textbf{Adaptive Difficulty:} AI can adjust difficulty in real-time based on player skill levels.
        \item \textbf{Customizable Experiences:} Players receive tailored experiences that adapt to preferences via intelligent modeling.
    \end{itemize}
\end{frame}

\begin{frame}[fragile]
    \frametitle{Conclusion and Key Takeaways}
    \begin{block}{Conclusion}
        Game AI is for transformative advancements. Embracing trends like multi-agent systems, deep reinforcement learning, and ethical AI can lead to innovative and immersive gaming experiences. The future of gaming will alter game design and player interactions in these dynamic environments.
    \end{block}

    \begin{itemize}
        \item Key trends include multi-agent systems, deep reinforcement learning, and procedural content generation.
        \item Player engagement and customization are significant implications of advancements.
        \item Ethical AI considerations are crucial for fair and enjoyable gameplay.
    \end{itemize}
\end{frame}

\begin{frame}[fragile]
  \frametitle{Conclusion and Key Takeaways - Part 1}
  \begin{block}{Recap of Multi-Agent Search and Game Playing}
    \begin{enumerate}
      \item \textbf{Understanding Multi-Agent Systems (MAS)}
        \begin{itemize}
          \item Multi-agent systems consist of multiple interacting autonomous agents capable of decision-making.
          \item In competitive scenarios, each agent employs its own strategy, necessitating sophisticated algorithms.
        \end{itemize}

      \item \textbf{Search Techniques in Multi-Agent Environments}
        \begin{itemize}
          \item \textit{Adversarial Search}: Used in competitive scenarios (e.g., chess) where agents anticipate opponents' moves.
            \begin{itemize}
              \item \textit{Minimax Algorithm}: Aims to minimize the possible loss in worst-case scenarios.
              \item \textit{Alpha-Beta Pruning}: Eliminates branches in minimax that are not optimal, enhancing efficiency.
            \end{itemize}
        \end{itemize}
    \end{enumerate}
  \end{block}
\end{frame}

\begin{frame}[fragile]
  \frametitle{Conclusion and Key Takeaways - Part 2}
  \begin{block}{Cooperative vs. Competitive Strategies}
    \begin{itemize}
      \item \textbf{Cooperative}: Agents collaborate to achieve a common goal.
        \begin{itemize}
          \item \textit{Example}: Players in a soccer game coordinate to score a goal.
        \end{itemize}
        
      \item \textbf{Competitive}: Agents aim to outperform each other, needing distinct algorithms and heuristics.
    \end{itemize}

    \begin{block}{Applications in AI}
      \begin{itemize}
        \item \textbf{Game Playing}: Crucial for developing intelligent agents in various games.
        \item \textbf{Real-World Scenarios}: Extends to economics, robotics, and negotiation, utilizing game theory and multi-agent systems.
      \end{itemize}
    \end{block}
  \end{block}
\end{frame}

\begin{frame}[fragile]
  \frametitle{Conclusion and Key Takeaways - Part 3}
  \begin{block}{Future Trends in Game AI}
    \begin{itemize}
      \item Research will focus on:
        \begin{itemize}
          \item Enhancing machine learning within multi-agent frameworks.
          \item Developing adaptive agents that learn from opponents' strategies.
          \item Addressing ethical considerations in AI-driven competitive environments.
        \end{itemize}
    \end{itemize}
  \end{block}

  \begin{block}{Key Points to Emphasize}
    \begin{itemize}
      \item Multi-agent systems highlight the balance between cooperation and competition in AI.
      \item Search techniques are critical for analyzing and anticipating agent behaviors in adversarial contexts.
      \item Understanding these concepts equips us for emerging AI trends, fostering the development of sophisticated systems.
    \end{itemize}
  \end{block}
\end{frame}


\end{document}