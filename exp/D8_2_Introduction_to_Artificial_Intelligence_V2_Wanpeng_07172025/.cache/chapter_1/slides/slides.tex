\documentclass[aspectratio=169]{beamer}

% Theme and Color Setup
\usetheme{Madrid}
\usecolortheme{whale}
\useinnertheme{rectangles}
\useoutertheme{miniframes}

% Additional Packages
\usepackage[utf8]{inputenc}
\usepackage[T1]{fontenc}
\usepackage{graphicx}
\usepackage{booktabs}
\usepackage{listings}
\usepackage{amsmath}
\usepackage{amssymb}
\usepackage{xcolor}
\usepackage{tikz}
\usepackage{pgfplots}
\pgfplotsset{compat=1.18}
\usetikzlibrary{positioning}
\usepackage{hyperref}

% Custom Colors
\definecolor{myblue}{RGB}{31, 73, 125}
\definecolor{mygray}{RGB}{100, 100, 100}
\definecolor{mygreen}{RGB}{0, 128, 0}
\definecolor{myorange}{RGB}{230, 126, 34}
\definecolor{mycodebackground}{RGB}{245, 245, 245}

% Set Theme Colors
\setbeamercolor{structure}{fg=myblue}
\setbeamercolor{frametitle}{fg=white, bg=myblue}
\setbeamercolor{title}{fg=myblue}
\setbeamercolor{section in toc}{fg=myblue}
\setbeamercolor{item projected}{fg=white, bg=myblue}
\setbeamercolor{block title}{bg=myblue!20, fg=myblue}
\setbeamercolor{block body}{bg=myblue!10}
\setbeamercolor{alerted text}{fg=myorange}

% Footer and Navigation Setup
\setbeamertemplate{navigation symbols}{}
\setbeamertemplate{footline}{
  \leavevmode%
  \hbox{%
  \begin{beamercolorbox}[wd=.3\paperwidth,ht=2.25ex,dp=1ex,center]{author in head/foot}%
    \usebeamerfont{author in head/foot}\insertshortauthor
  \end{beamercolorbox}%
  \begin{beamercolorbox}[wd=.5\paperwidth,ht=2.25ex,dp=1ex,center]{title in head/foot}%
    \usebeamerfont{title in head/foot}\insertshorttitle
  \end{beamercolorbox}%
  \begin{beamercolorbox}[wd=.2\paperwidth,ht=2.25ex,dp=1ex,center]{date in head/foot}%
    \usebeamerfont{date in head/foot}
    \insertframenumber{} / \inserttotalframenumber
  \end{beamercolorbox}}%
  \vskip0pt%
}

% Title Page Information
\title[AI and Agent Architectures]{Week 1-2: Introduction to AI and Agent Architectures}
\author[J. Smith]{John Smith, Ph.D.}
\institute[University Name]{
  Department of Computer Science\\
  University Name\\
  \vspace{0.3cm}
  Email: email@university.edu\\
  Website: www.university.edu
}
\date{\today}

% Document Start
\begin{document}

\frame{\titlepage}

\begin{frame}[fragile]
    \frametitle{Introduction to AI - Overview}
    \begin{block}{What is Artificial Intelligence (AI)?}
        AI refers to the simulation of human intelligence processes by machines, particularly computer systems. This includes:
        \begin{itemize}
            \item Learning: The acquisition of information and rules for using it.
            \item Reasoning: Using the rules to reach approximate or definite conclusions.
            \item Self-correction.
        \end{itemize}
    \end{block}
\end{frame}

\begin{frame}[fragile]
    \frametitle{Introduction to AI - Significance}
    \begin{block}{Significance of AI}
        \begin{itemize}
            \item Transformative Impact: AI technologies are reshaping industries by enhancing efficiency, reducing human errors, and enabling new capabilities.
            \item Applications Across Sectors:
            \begin{itemize}
                \item Healthcare: AI algorithms that analyze medical data and assist in diagnostics.
                \item Finance: Automated trading and fraud detection.
                \item Transportation: Development of self-driving cars.
                \item Manufacturing: Robotics and predictive maintenance.
            \end{itemize}
        \end{itemize}
    \end{block}
\end{frame}

\begin{frame}[fragile]
    \frametitle{Introduction to AI - Relevance and Conclusion}
    \begin{block}{Relevance of AI in Today's World}
        \begin{itemize}
            \item Data Explosion: Vast amounts of data provide a rich learning environment.
            \item Accessibility: AI tools like virtual assistants are becoming part of daily life.
            \item Employment Landscape: AI creates new job opportunities while automating tasks.
        \end{itemize}
    \end{block}

    \begin{block}{Key Points to Emphasize}
        \begin{itemize}
            \item Human-like Abilities: Performing tasks requiring human intelligence.
            \item Types of AI:
            \begin{itemize}
                \item Narrow AI: Specialized systems.
                \item General AI: Hypothetical systems capable of broader understanding.
            \end{itemize}
        \end{itemize}
    \end{block}

    \begin{block}{Conclusion}
        Understanding AI is crucial as its integration into daily life continues to expand, influencing our interactions with technology and shaping the future of industries globally.
    \end{block}
\end{frame}

\begin{frame}[fragile]{History of AI - Key Milestones}
    \begin{enumerate}
        \item \textbf{Foundation and Early Developments (1950s)}
            \begin{itemize}
                \item \textbf{Alan Turing (1950)}: Proposed the Turing Test.
                \item \textbf{Dartmouth Conference (1956)}: Marked the birth of AI; the term "Artificial Intelligence" was coined.
            \end{itemize}
        
        \item \textbf{Expansion and Initial Success (1960s-1970s)}
            \begin{itemize}
                \item Early Programs: ELIZA and SHRDLU demonstrated conversation simulation and language understanding.
                \item Symbolic AI: Focused on rule-based systems mimicking human-like reasoning.
            \end{itemize}
        
        \item \textbf{The AI Winter (1974-1980; late 1980s)}
            \begin{itemize}
                \item Funding Cuts: Disillusionment led to reduced funding.
                \item Research Stagnation: Limited computational power hindered progress.
            \end{itemize}
    \end{enumerate}
\end{frame}

\begin{frame}[fragile]{History of AI - Revival and New Approaches}
    \begin{enumerate}
        \item \textbf{Revival and New Approaches (1980s-1990s)}
            \begin{itemize}
                \item \textbf{Expert Systems}: MYCIN showcased practical applications in diagnosing diseases.
                \item \textbf{Neural Networks}: Rediscovery led by researchers like Geoffrey Hinton, exploring human brain-like learning.
            \end{itemize}
        
        \item \textbf{Rise of Machine Learning (1990s-2000s)}
            \begin{itemize}
                \item Data-driven AI: Shift to learning from data with algorithms like decision trees.
                \item \textbf{Deep Blue}: IBM’s Deep Blue defeated Garry Kasparov in chess (1997).
            \end{itemize}
    \end{enumerate}
\end{frame}

\begin{frame}[fragile]{History of AI - Present Day Advancements}
    \begin{enumerate}
        \item \textbf{Present Day Advancements (2010s-Present)}
            \begin{itemize}
                \item \textbf{Deep Learning}: Breakthroughs in neural networks, especially CNNs for images and RNNs for language.
                \item \textbf{AI in Daily Life}: Applications in voice assistants, recommendation systems, and self-driving vehicles.
                \item \textbf{Ethics and Accountability}: Focus on ethical implications and responsible usage.
            \end{itemize}
    \end{enumerate}
    
    \begin{block}{Key Points to Emphasize}
        \begin{itemize}
            \item Evolution of AI from rules to sophisticated neural networks.
            \item Historical setbacks shaped the field and promoted innovation.
            \item AI's convergence with Big Data significantly impacts society.
        \end{itemize}
    \end{block}
\end{frame}

\begin{frame}[fragile]{Core AI Concepts - Search Strategies}
    \frametitle{Core AI Concepts - Search Strategies}
    \begin{block}{Definition}
        Search strategies are systematic methods to explore the problem space to find solutions or optimal paths.
    \end{block}
    
    \begin{itemize}
        \item \textbf{Key Types:}
        \begin{itemize}
            \item \textbf{Uninformed Search:} No additional information about states.
                \begin{itemize}
                    \item \textbf{Example:} Depth-First Search (DFS), Breadth-First Search (BFS)
                \end{itemize}
            \item \textbf{Informed Search:} Uses heuristics to guide the search process.
                \begin{itemize}
                    \item \textbf{Example:} A* algorithm: \( f(n) = g(n) + h(n) \)
                    \begin{itemize}
                        \item \( g(n) \) = cost to reach node \( n \)
                        \item \( h(n) \) = estimated cost from \( n \) to the goal
                    \end{itemize}
                \end{itemize}
        \end{itemize}
    \end{itemize}
\end{frame}

\begin{frame}[fragile]{Core AI Concepts - Logic Reasoning}
    \frametitle{Core AI Concepts - Logic Reasoning}
    \begin{block}{Definition}
        Logic reasoning involves using formal logical systems to derive conclusions based on given premises.
    \end{block}

    \begin{itemize}
        \item \textbf{Types of Logic:}
        \begin{itemize}
            \item \textbf{Propositional Logic:} Deals with statements that are either true or false.
                \begin{itemize}
                    \item \textbf{Example:} "If it rains, then the ground is wet" as \( P \to Q \)
                \end{itemize}
            \item \textbf{First-Order Logic (FOL):} Extends propositional logic with predicates and quantifiers.
                \begin{itemize}
                    \item \textbf{Example:} "Everyone loves AI" can be \( \forall x, \text{Human}(x) \to \text{Loves}(x, \text{AI}) \)
                \end{itemize}
        \end{itemize}
    \end{itemize}
\end{frame}

\begin{frame}[fragile]{Core AI Concepts - Probabilistic Models}
    \frametitle{Core AI Concepts - Probabilistic Models}
    \begin{block}{Definition}
        Probabilistic models deal with uncertainty, allowing predictions or decisions in the face of incomplete information.
    \end{block}

    \begin{itemize}
        \item \textbf{Key Components:}
        \begin{itemize}
            \item \textbf{Bayesian Networks:} Graphical model representing variables and their conditional dependencies.
                \begin{itemize}
                    \item \textbf{Example:} Modeling diseases given symptoms in diagnostics.
                \end{itemize}
            \item \textbf{Markov Decision Processes (MDPs):} Framework for modeling decision-making with randomness.
                \begin{itemize}
                    \item \textbf{Key Concept:} Bellman Equation: 
                    \begin{equation}
                        V(s) = \max_a \left( R(s, a) + \gamma \sum_{s'} P(s'|s, a)V(s') \right)
                    \end{equation}
                    \item Where \( R \) represents rewards, \( \gamma \) is the discount factor, and \( P \) defines state transitions.
                \end{itemize}
        \end{itemize}
    \end{itemize}
\end{frame}

\begin{frame}[fragile]
    \frametitle{Understanding Agents}
    
    \begin{block}{What is an Agent in AI?}
        An **agent** in Artificial Intelligence (AI) is an entity that perceives its environment and takes actions to achieve specific goals. Agents can range from simple programs to complex systems operating in dynamic environments.
    \end{block}
    
    \begin{block}{Core Concepts Related to Agents}
        \begin{itemize}
            \item \textbf{Perception:} Agents perceive their environment through sensors (e.g. cameras, microphones).
            \item \textbf{Action:} Based on perceptions, agents act on their environment using actuators (e.g. moving, communicating).
            \item \textbf{Autonomy:} Agents make decisions without human intervention.
            \item \textbf{Goal-oriented Behavior:} Agents aim to achieve specific objectives.
        \end{itemize}
    \end{block}
\end{frame}

\begin{frame}[fragile]
    \frametitle{Types of Agents}
    
    \begin{block}{Foundational Categories}
        Some basic types of agents include:
        \begin{itemize}
            \item \textbf{Simple Reflex Agents:} Operate on condition-action rules (e.g. thermostat adjusts heating based on readings).
            \item \textbf{Model-Based Agents:} Maintain internal representations of the world (e.g. self-driving cars using models of their surroundings).
        \end{itemize}
    \end{block}
    
    \begin{block}{Role of Agents in System Design}
        \begin{itemize}
            \item Enable modular design for intelligent systems.
            \item Provide scalability and flexibility in system architecture.
            \item Allow distribution of tasks for efficient problem-solving.
        \end{itemize}
    \end{block}
\end{frame}

\begin{frame}[fragile]
    \frametitle{Illustrative Example and Summary}
    
    \begin{block}{Example: Autonomous Drone}
        \begin{itemize}
            \item \textbf{Sensors:} Uses cameras, GPS, and LIDAR to perceive its environment.
            \item \textbf{Actions:} Can fly, capture images, or avoid obstacles.
            \item \textbf{Autonomy:} Operates independently based on inputs and objectives (e.g. monitoring a forest).
        \end{itemize}
    \end{block}
    
    \begin{block}{Summary}
        Understanding agents is foundational for AI. Their ability to perceive, act towards goals, and function independently makes them integral in intelligent systems.
    \end{block}
\end{frame}

\begin{frame}[fragile]
    \frametitle{Types of Agents - Introduction}
    In the field of Artificial Intelligence, agents are entities capable of:
    \begin{itemize}
        \item Perceiving their environment through sensors,
        \item Acting upon that environment through actuators,
        \item Processing information to make decisions.
    \end{itemize}
    Understanding different types of agents is vital as they exhibit distinct behaviors based on their design.
\end{frame}

\begin{frame}[fragile]
    \frametitle{Types of Agents - Simple Reflex Agents}
    \begin{block}{1. Simple Reflex Agents}
        \begin{itemize}
            \item \textbf{Definition}: Operate solely on current percepts using condition-action rules known as \textit{reflex rules}.
            \item \textbf{How They Work}: Map the current state of the environment directly to actions.
            \item \textbf{Example}: A thermostat that turns on the heater when the temperature drops below a set threshold.
        \end{itemize}
    \end{block}
    
    \textbf{Key Concept}: Simple reflex agents do not consider the history of previous states, limiting functionality in complex environments.
\end{frame}

\begin{frame}[fragile]
    \frametitle{Types of Agents - Model-Based and Goal-Based Agents}
    \begin{block}{2. Model-Based Agents}
        \begin{itemize}
            \item \textbf{Definition}: Maintain an internal state based on the history of percepts for informed decision making.
            \item \textbf{How They Work}: Utilize a model of the world to track ongoing and predict future states.
            \item \textbf{Example}: A self-driving car that tracks obstacles even if temporarily out of sight.
        \end{itemize}
        
        \textbf{Key Concept}: More flexible than simple reflex agents, handling dynamic environments by considering past actions and observations.
    \end{block}

    \begin{block}{3. Goal-Based Agents}
        \begin{itemize}
            \item \textbf{Definition}: Act to fulfill specific goals, evaluating actions based on their contribution to these goals.
            \item \textbf{How They Work}: Plan sequences of actions to achieve goals by evaluating possible outcomes.
            \item \textbf{Example}: A chess-playing AI evaluating different moves to checkmate the opponent.
        \end{itemize}

        \textbf{Key Concept}: Incorporate knowledge about desirable outcomes, allowing for strategic planning.
    \end{block}
\end{frame}

\begin{frame}[fragile]
    \frametitle{Types of Agents - Summary}
    \begin{itemize}
        \item \textbf{Demand for Complexity}: Agents evolve from simple reflex actions to complex reasoning as tasks become more complicated.
        \item \textbf{Capabilities}: The effectiveness of an agent is determined by its ability to perceive, process information, and act towards achieving objectives.
    \end{itemize}
    
    \textit{Understanding these types of agents provides foundational insight into design decisions for AI systems.}
\end{frame}

\begin{frame}[fragile]
    \frametitle{Agent Architectures}
    \begin{block}{Overview of Agent Architectures}
        Agent architectures are frameworks that dictate how agents operate, make decisions, and interact with their environments. They can be categorized into three primary types:
    \end{block}
    \begin{enumerate}
        \item Reactive Architectures
        \item Deliberative Architectures
        \item Hybrid Architectures
    \end{enumerate}
\end{frame}

\begin{frame}[fragile]
    \frametitle{Reactive Architectures}
    \begin{block}{Definition}
        Reactive architectures respond to environmental stimuli quickly without internal reasoning.
    \end{block}
    \begin{exampleblock}{Example}
        A simple reflex agent like a thermostat that activates heating when the temperature drops below a threshold.
    \end{exampleblock}
    \begin{itemize}
        \item \textbf{Key Characteristics:}
        \begin{itemize}
            \item Simplicity: Minimal computation; decisions based on sensory input.
            \item Speed: Quick reactions beneficial in dynamic environments.
        \end{itemize}
    \end{itemize}
    \begin{block}{Illustration}
        Input (Environmental Change) $\rightarrow$ Rule Check (If-Then) $\rightarrow$ Output (Action)
    \end{block}
\end{frame}

\begin{frame}[fragile]
    \frametitle{Deliberative and Hybrid Architectures}
    \begin{block}{Deliberative Architectures}
        \begin{itemize}
            \item \textbf{Definition:} Involve internal models and reasoning for decision-making.
            \item \textbf{Example:} A robot that uses maps to navigate efficiently.
            \item \textbf{Key Characteristics:}
            \begin{itemize}
                \item Complexity: High-level reasoning, loops, conditionals.
                \item Strategic: Plans for long-term goals.
            \end{itemize}
        \end{itemize}
        \begin{block}{Illustration}
            Input (State of the World) $\rightarrow$ Internal Model $\rightarrow$ Decision $\rightarrow$ Output (Action)
        \end{block}
    \end{block}
    
    \begin{block}{Hybrid Architectures}
        \begin{itemize}
            \item \textbf{Definition:} Combine reactive and deliberative elements.
            \item \textbf{Example:} An autonomous vehicle that quickly reacts to obstacles while planning its route.
            \item \textbf{Key Characteristics:}
            \begin{itemize}
                \item Flexibility: Adapts to contexts for rapid responses.
                \item Efficiency: Minimizes weaknesses of both architectures.
            \end{itemize}
        \end{itemize}
    \end{block}
\end{frame}

\begin{frame}[fragile]
    \frametitle{Summary of Key Points}
    \begin{itemize}
        \item Agent architecture determines behavior: 
        Understanding architectures aids in designing agents suited for specific environments.
        \item \textbf{Reactive}: Fast, rule-based responses ideal for simpler tasks.
        \item \textbf{Deliberative}: Slow, thoughtful planning suitable for complex tasks.
        \item \textbf{Hybrid}: Combines the best of both worlds for robust agent performance.
    \end{itemize}
\end{frame}

\begin{frame}[fragile]
    \frametitle{Transition to Next Slide}
    Next, we will explore \textbf{AI Algorithms} which empower these architectures, focusing on search, planning, decision-making, and learning functionalities.
\end{frame}

\begin{frame}[fragile]{AI Algorithms Overview}
    \frametitle{AI Algorithms Overview}
    
    \begin{block}{Introduction}
    AI algorithms are fundamental to the functioning of intelligent systems. They can be broadly categorized into four main areas:
    \begin{itemize}
        \item Search
        \item Planning
        \item Decision Making
        \item Learning
    \end{itemize}
    \end{block}
\end{frame}

\begin{frame}[fragile]{1. Search Algorithms}
    \frametitle{Search Algorithms}

    \begin{block}{Definition}
    Algorithms designed to explore and locate solutions in problem-solving environments.
    \end{block}
    
    \begin{block}{Common Types}
        \begin{itemize}
            \item \textbf{Uninformed Search}: Searches without additional information (e.g., Breadth-First Search, Depth-First Search).
            \item \textbf{Informed Search}: Uses heuristics to guide the search (e.g., A* algorithm, Greedy Search).
        \end{itemize}
    \end{block}
    
    \begin{block}{Example}
        A* Algorithm finds the shortest path in a map by evaluating the cost to reach a particular node and the estimated cost to the destination.
    \end{block}
\end{frame}

\begin{frame}[fragile]{Key Algorithms in AI}
    \frametitle{Key Algorithms in AI}
    
    \begin{block}{2. Planning Algorithms}
        Planning algorithms generate a sequence of actions to achieve specific goals, based on predictions of future states.
        \begin{itemize}
            \item \textbf{Common Approaches}:
                \begin{itemize}
                    \item STRIPS
                    \item PDDL
                \end{itemize}
            \item \textbf{Example}:
                A robot planning to retrieve an object must consider its current position, obstacles, and the end goal.
        \end{itemize}
    \end{block}

    \begin{block}{3. Decision-Making Algorithms}
        Algorithms that assess various possible actions and select the most appropriate one.
        \begin{itemize}
            \item \textbf{Common Techniques}:
                \begin{itemize}
                    \item Markov Decision Processes (MDPs)
                    \item Decision Trees
                \end{itemize}
            \item \textbf{Example}:
                An autonomous vehicle uses MDPs to decide whether to stop at a traffic light.
        \end{itemize}
    \end{block}
\end{frame}

\begin{frame}[fragile]{4. Learning Algorithms}
    \frametitle{4. Learning Algorithms}

    \begin{block}{Definition}
    Algorithms designed to enable systems to learn from data and adapt their responses based on previous interactions.
    \end{block}
    
    \begin{block}{Common Types}
        \begin{itemize}
            \item Supervised Learning
            \item Unsupervised Learning
            \item Reinforcement Learning
        \end{itemize}
    \end{block}
    
    \begin{block}{Example}
    A recommendation system utilizes supervised learning to predict user preferences based on historical data.
    \end{block}

    \begin{block}{Key Points}
        \begin{itemize}
            \item Each category serves specific purposes and is often interrelated.
            \item Real-world applications showcase the power of these algorithms.
        \end{itemize}
    \end{block}
\end{frame}

\begin{frame}[fragile]{Application of AI Algorithms - Overview}
    \frametitle{Application of AI Algorithms}
    \begin{block}{Introduction}
    AI algorithms analyze data, make decisions, and learn from experiences. They tackle complex problems across various fields. This presentation demonstrates their real-life applications.
    \end{block}
\end{frame}

\begin{frame}[fragile]{Application of AI Algorithms - Natural Language Processing}
    \frametitle{Natural Language Processing (NLP)}
    \begin{itemize}
        \item \textbf{Concept:} NLP allows machines to understand and respond to human language.
        
        \item \textbf{Application Example:} Chatbots in customer service effectively address queries.
        
        \item \textbf{Key Points:}
        \begin{itemize}
            \item Enables real-time communication.
            \item Increases customer satisfaction with instant responses.
        \end{itemize}
    \end{itemize}
    \includegraphics[width=0.7\textwidth]{flowchart_example.png} % Placeholder for flowchart
\end{frame}

\begin{frame}[fragile]{Application of AI Algorithms - Other Applications}
    \frametitle{Various Applications of AI Algorithms}
    \begin{enumerate}
        \item \textbf{Recommendation Systems}
        \begin{itemize}
            \item Concept: Analyze user behavior and preferences.
            \item Example: Netflix recommends shows based on viewing history.
            \item Key Points:
            \begin{itemize}
                \item Personalized experiences increase engagement.
                \item Collaborative filtering improves over time.
            \end{itemize}
        \end{itemize}
        
        \item \textbf{Autonomous Vehicles}
        \begin{itemize}
            \item Concept: Combine computer vision and sensor data for navigation.
            \item Example: Waymo and Tesla's self-driving cars.
            \item Key Points:
            \begin{itemize}
                \item Improves safety by reducing human error.
                \item Deep learning algorithms assist in object detection.
            \end{itemize}
        \end{itemize}

        \item \textbf{Healthcare Diagnosis}
        \begin{itemize}
            \item Concept: Analyze medical data for disease diagnosis.
            \item Example: Image recognition identifies anomalies in scans.
            \item Key Points:
            \begin{itemize}
                \item Speeds up diagnostics.
                \item Comparable accuracy to expert radiologists.
            \end{itemize}
        \end{itemize}
        
        \item \textbf{Financial Services}
        \begin{itemize}
            \item Concept: Analyze trends and predict market movements.
            \item Example: Fraud detection systems monitor transactions.
            \item Key Points:
            \begin{itemize}
                \item Enhances banking security.
                \item Machine learning adapts to new fraud patterns.
            \end{itemize}
        \end{itemize}
    \end{enumerate}
\end{frame}

\begin{frame}[fragile]{Application of AI Algorithms - Conclusion}
    \frametitle{Conclusion}
    The application of AI algorithms is transforming industries, providing innovative solutions to complex challenges. Understanding these applications highlights the significance of AI in our daily lives and future developments.

    \textbf{Next Steps:} Continue to the next slide for an in-depth exploration of \textit{Markov Decision Processes (MDPs)} and their role in decision-making processes in AI.
\end{frame}

\begin{frame}[fragile]{Markov Decision Processes (MDPs)}
    \frametitle{Markov Decision Processes (MDPs)}
    \begin{block}{Definition}
        A \textbf{Markov Decision Process (MDP)} is a mathematical framework for modeling decision-making environments where outcomes are partly random and partly under the control of an agent. 
    \end{block}
    \begin{itemize}
        \item Foundation of reinforcement learning and AI.
        \item Models sequential decision-making problems.
    \end{itemize}
\end{frame}

\begin{frame}[fragile]{Components of MDPs}
    \frametitle{Components of MDPs}
    An MDP consists of:
    \begin{enumerate}
        \item \textbf{States (S)}: Set of all possible situations.
        \item \textbf{Actions (A)}: Set of all possible actions the agent can take.
        \item \textbf{Transition Model (P)}: Probability of state transition given an action.
            \begin{equation}
                P(s' | s, a)
            \end{equation}
        \item \textbf{Rewards (R)}: Feedback on actions, typically scalar values.
        \item \textbf{Discount Factor ($\gamma$)}: Value between 0 and 1 that discounts future rewards.
        \begin{itemize}
            \item Closer to 0 favors immediate rewards.
            \item Closer to 1 values future rewards similarly to immediate ones.
        \end{itemize}
    \end{enumerate}
\end{frame}

\begin{frame}[fragile]{The Role of MDPs in Decision-Making}
    \frametitle{The Role of MDPs in Decision-Making}
    MDPs:
    \begin{itemize}
        \item Provide a structure for modeling complex decision-making processes.
        \item Allow agents to evaluate strategies for maximum cumulative rewards.
        \item Manage uncertainty in outcomes, suitable for real-world applications.
    \end{itemize}
\end{frame}

\begin{frame}[fragile]{Example of MDP in Use}
    \frametitle{Example of MDP in Use}
    Consider a robot navigating a maze:
    \begin{itemize}
        \item \textbf{States (S)}: Each position in the maze (start, goal, walls).
        \item \textbf{Actions (A)}: Move up, down, left, right.
        \item \textbf{Transition Model (P)}:
            \begin{itemize}
                \item 90\% chance to move left, 10\% to hit a wall.
            \end{itemize}
        \item \textbf{Rewards (R)}: +50 for reaching the goal, -1 for each move.
        \item \textbf{Discount Factor ($\gamma$)}: Set to 0.9.
    \end{itemize}
    The MDP framework helps the robot determine the optimal path in the maze.
\end{frame}

\begin{frame}[fragile]{Key Points to Emphasize}
    \frametitle{Key Points to Emphasize}
    \begin{itemize}
        \item MDPs formalize decision-making under uncertainty.
        \item Components: states, actions, transition probabilities, rewards, and a discount factor.
        \item Widely applicable in AI, particularly in reinforcement learning frameworks.
    \end{itemize}
    By understanding MDPs, students can analyze and create algorithms for decision-making in uncertain environments.
\end{frame}

\begin{frame}[fragile]
    \frametitle{Reinforcement Learning}
    \begin{block}{Definition}
        Reinforcement Learning (RL) is a type of Machine Learning where an agent learns to make decisions by taking actions in an environment to maximize cumulative rewards through trial and error.
    \end{block}
    \begin{block}{Key Concepts in RL}
        \begin{itemize}
            \item \textbf{Agent}: The learner or decision-maker.
            \item \textbf{Environment}: Everything the agent interacts with, providing feedback.
            \item \textbf{Action}: A choice made by the agent affecting the environment.
            \item \textbf{State}: The current situation of the agent within the environment.
            \item \textbf{Reward}: Feedback received after an action, indicating its effectiveness.
        \end{itemize}
    \end{block}
\end{frame}

\begin{frame}[fragile]
    \frametitle{How Reinforcement Learning Works}
    \begin{block}{Exploration vs. Exploitation}
        The agent must balance exploring new actions to discover their rewards with exploiting known actions that yield high rewards.
    \end{block}
    \begin{block}{Q-Learning}
        A popular model-free method where the agent updates its knowledge of the value of actions in states:
        \begin{equation}
            Q(s, a) \leftarrow Q(s, a) + \alpha \left[r + \gamma \max_{a'} Q(s', a') - Q(s, a)\right]
        \end{equation}
        where:
        \begin{itemize}
            \item $Q(s, a)$ = value of action $a$ in state $s$
            \item $\alpha$ = learning rate
            \item $r$ = reward received after taking action $a$
            \item $\gamma$ = discount factor for future rewards
            \item $s'$ = new state after the action
        \end{itemize}
    \end{block}
\end{frame}

\begin{frame}[fragile]
    \frametitle{Applications and Example}
    \begin{block}{Applications of Reinforcement Learning}
        \begin{itemize}
            \item \textbf{Robotics}: Teaching robots tasks by learning from interactions.
            \item \textbf{Game Playing}: Success in strategic games (e.g., AlphaGo).
            \item \textbf{Autonomous Vehicles}: Real-time decision-making for self-driving cars.
            \item \textbf{Healthcare}: Personalizing treatment plans using patient data.
        \end{itemize}
    \end{block}
    \begin{block}{Example Scenario}
        Imagine training a dog (the agent) to fetch a ball. Success leads to a treat (reward), while failure results in no treat or a negative cue. The dog learns that retrieving the ball leads to rewards.
    \end{block}
\end{frame}

\begin{frame}[fragile]{Analyzing AI Models}
    Evaluating AI models involves assessing their correctness, performance, and applicability. 
    Understanding these dimensions helps ensure they deliver valuable outcomes in practical scenarios.
\end{frame}

\begin{frame}[fragile]{1. Correctness}
    \begin{block}{Definition}
        Correctness assesses whether a model produces accurate and reliable outputs based on given inputs.
    \end{block}
    
    \begin{itemize}
        \item \textbf{Methods of Evaluation}:
            \begin{itemize}
                \item Validation Data Sets: Separate unseen data to test model predictions.
                \item Cross-Validation: Techniques like k-fold cross-validation mitigate overfitting.
                \item Performance Metrics:
                    \begin{itemize}
                        \item Accuracy: Proportion of correct predictions.
                        \item Precision \& Recall: Critical in classification tasks.
                        \item F1 Score: Useful for imbalanced datasets.
                    \end{itemize}
            \end{itemize}
        \item \textbf{Example}: A classification model for disease diagnosis evaluated by true positives and true negatives from patient data.
    \end{itemize}
\end{frame}

\begin{frame}[fragile]{2. Performance}
    \begin{block}{Definition}
        Performance measures the efficiency of an AI model, focusing on speed and resource usage.
    \end{block}
    
    \begin{itemize}
        \item \textbf{Key Metrics}:
            \begin{itemize}
                \item Inference Time: Time taken for predictions.
                \item Throughput: Predictions made per unit of time.
            \end{itemize}
        \item \textbf{Resource Utilization}: 
            Measures memory consumption and computational power during training and inference.
        \item \textbf{Example}: A recommendation system's time to generate recommendations for a thousand users.
    \end{itemize}
\end{frame}

\begin{frame}[fragile]{3. Applicability}
    \begin{block}{Definition}
        Applicability refers to the usability of the AI model in specific scenarios.
    \end{block}
    
    \begin{itemize}
        \item \textbf{Evaluation Techniques}:
            \begin{itemize}
                \item Use Case Testing: Implementing in real or simulated environments.
                \item User Feedback: Collecting end-user insights on utility and usability.
                \item Comparison with Existing Solutions: Benchmarking against other models.
            \end{itemize}
        \item \textbf{Example}: Evaluating an AI chatbot's effectiveness in handling customer queries in a live service environment.
    \end{itemize}
\end{frame}

\begin{frame}[fragile]{Key Points to Remember}
    \begin{itemize}
        \item Evaluating AI models encompasses accuracy, performance, and real-world applicability.
        \item A variety of metrics is essential for a comprehensive view of a model's effectiveness.
        \item Real-world testing and user feedback inform the practicality of AI models.
    \end{itemize}
\end{frame}

\begin{frame}[fragile]{Metrics Summary Table}
    \begin{tabular}{|l|l|}
        \hline
        \textbf{Metric} & \textbf{Description} \\
        \hline
        Accuracy & Correct predictions / Total predictions \\
        Precision & True Positives / (True Positives + False Positives) \\
        Recall & True Positives / (True Positives + False Negatives) \\
        F1 Score & 2 * (Precision * Recall) / (Precision + Recall) \\
        Inference Time & Time taken to generate predictions \\
        Throughput & Predictions per second \\
        Resource Utilization & Memory and CPU usage during model operation \\
        \hline
    \end{tabular}
\end{frame}

\begin{frame}[fragile]
    \frametitle{Machine Learning Integration - Overview}
    \begin{block}{Definition}
        Machine Learning (ML) is a subset of artificial intelligence (AI) that focuses on building systems that learn from data to make predictions or decisions without being explicitly programmed.
    \end{block}

    \begin{itemize}
        \item \textbf{Key Concepts in Machine Learning}:
        \begin{enumerate}
            \item Supervised Learning: Learning from labeled training data.
            \item Unsupervised Learning: Finding patterns in unlabeled data.
            \item Reinforcement Learning: Learning from interaction and feedback.
        \end{enumerate}
    \end{itemize}
\end{frame}

\begin{frame}[fragile]
    \frametitle{Machine Learning Concepts - Examples}
    \begin{itemize}
        \item \textbf{Supervised Learning Example}:
        \begin{block}{Spam Filter}
            A spam filter that identifies spam emails based on labeled examples.
        \end{block}

        \item \textbf{Unsupervised Learning Example}:
        \begin{block}{Customer Segmentation}
            Analysis of customer purchasing behavior to identify distinct segments.
        \end{block}

        \item \textbf{Reinforcement Learning Example}:
        \begin{block}{Game-playing AI}
            An AI that learns strategies through trial and error in a gaming environment.
        \end{block}
    \end{itemize}
\end{frame}

\begin{frame}[fragile]
    \frametitle{Deep Learning Fundamentals}
    \begin{block}{Definition}
        Deep Learning is a specialized area within ML that uses neural networks with many layers to model complex patterns in data.
    \end{block}

    \begin{itemize}
        \item \textbf{Key Points of Deep Learning}:
        \begin{enumerate}
            \item Neural Networks: Layers of interconnected nodes (neurons).
            \item Activation Functions: Determine output of a node, e.g., ReLU.
            \item Training Process: Requires large datasets and significant computing power.
        \end{enumerate}
    \end{itemize}
\end{frame}

\begin{frame}[fragile]
    \frametitle{Integration of Machine Learning into AI}
    \begin{itemize}
        \item \textbf{Data-Driven Decision Making}:
        \begin{block}{Example}
            Self-driving cars analyze sensor data to make navigation decisions.
        \end{block}
        
        \item \textbf{Adaptability}:
        \begin{block}{Framework}
            Input → Machine Learning Model → Output (Predictions/Decisions)
        \end{block}

        \item \textbf{Example Use Case: AI Chatbots}:
        \begin{block}{Flow}
            User Input → NLP Processing (ML) → Response Generation (AI)
        \end{block}
    \end{itemize}
\end{frame}

\begin{frame}[fragile]
    \frametitle{Conclusion}
    \begin{block}{Summary}
        Machine Learning and Deep Learning are key components that empower AI systems to learn from data, adapt, and make intelligent decisions autonomously. Understanding their integration lays the groundwork for developing advanced AI solutions.
    \end{block}
\end{frame}

\begin{frame}[fragile]
    \frametitle{Key Formula}
    \begin{equation}
        L(y, \hat{y}) = -\frac{1}{N} \sum_{i=1}^{N} y_i \log(\hat{y_i})
    \end{equation}
    Where $L$ is the loss, $y$ is the true label, $\hat{y}$ is the predicted probability, and $N$ is the number of samples.
\end{frame}

\begin{frame}[fragile]
    \frametitle{Sample Code Snippet}
    \begin{lstlisting}[language=Python]
# Sample Python code for a simple machine learning model using scikit-learn
from sklearn.model_selection import train_test_split
from sklearn.linear_model import LogisticRegression

# Sample data
X = [[1, 2], [2, 3], [3, 4], [4, 5]]
y = [0, 0, 1, 1]

# Split data into train and test
X_train, X_test, y_train, y_test = train_test_split(X, y, test_size=0.2)

# Initialize model
model = LogisticRegression()

# Fit model
model.fit(X_train, y_train)

# Make predictions
predictions = model.predict(X_test)
    \end{lstlisting}
\end{frame}

\begin{frame}[fragile]
    \frametitle{Ethics in AI}
    \begin{block}{Introduction}
        Ethics in AI refers to the moral implications and responsibilities associated with the creation and implementation of AI technologies. 
        As AI systems advance, it is essential to address the ethical considerations that arise.
    \end{block}
\end{frame}

\begin{frame}[fragile]
    \frametitle{Key Ethical Considerations - Bias in AI}
    \begin{itemize}
        \item \textbf{Definition}: Systematic favoritism or prejudice in data or algorithms leading to unfair treatment.
        \item \textbf{Causes}:
            \begin{itemize}
                \item \textbf{Training Data}: Reflects societal prejudices, perpetuating biases (e.g., facial recognition systems error rates).
                \item \textbf{Algorithm Design}: Omission of key variables can skew results.
            \end{itemize}
    \end{itemize}
    \begin{block}{Example}
        Hiring algorithms trained on historical data may discriminate against women if previous biases exist.
    \end{block}
\end{frame}

\begin{frame}[fragile]
    \frametitle{Key Ethical Considerations - Privacy Concerns}
    \begin{itemize}
        \item \textbf{Definition}: Protection of personal data and ethical management by AI systems.
        \item \textbf{Challenges}:
            \begin{itemize}
                \item \textbf{Data Collection}: Raises concerns of consent and data ownership.
                \item \textbf{Surveillance}: Use in monitoring infringes on privacy rights (CCTV with facial recognition).
            \end{itemize}
    \end{itemize}
    \begin{block}{Example}
        Smart Assistants like Amazon Echo collect voice data, leading to potential privacy infringements if not handled transparently.
    \end{block}
\end{frame}

\begin{frame}[fragile]
    \frametitle{Key Points to Emphasize}
    \begin{itemize}
        \item Importance of \textbf{Diverse Data}: Essential for reducing bias in AI models.
        \item \textbf{Transparency and Consent}: Organizations must ensure clarity about data collection and usage.
        \item \textbf{Regulatory Frameworks}: Ongoing legislation discussions (e.g., GDPR in Europe) aim to protect individual rights.
    \end{itemize}
\end{frame}

\begin{frame}[fragile]
    \frametitle{Conclusion}
    Ethics in AI is critical for fostering trust and ensuring equitable outcomes. 
    By addressing bias and respecting privacy, we can leverage AI for societal benefit while minimizing harm.
\end{frame}

\begin{frame}[fragile]
    \frametitle{Collaboration in AI Projects - Overview}
    \begin{block}{Importance of Collaboration}
        \begin{itemize}
            \item Multidisciplinary Nature of AI
            \item Enhancing Creativity
        \end{itemize}
    \end{block}
    \begin{block}{Teamwork Dynamics}
        \begin{itemize}
            \item Communication Skills
            \item Role Distribution
            \item Peer Review
        \end{itemize}
    \end{block}
    \begin{block}{Importance of Presentation Skills}
        \begin{itemize}
            \item Sharing Ideas and Progress
            \item Gaining Feedback
            \item Knowledge Transfer
        \end{itemize}
    \end{block}
\end{frame}

\begin{frame}[fragile]
    \frametitle{Collaboration in AI Projects - Details}
    \textbf{1. Multidisciplinary Nature of AI:}
    \begin{itemize}
        \item AI projects require expertise from various fields (data science, software engineering, etc.)
        \item Leverages diverse skill sets for innovative solutions.
        \item \textit{Example:} Healthcare AI projects involve data scientists, healthcare professionals, and ethicists.
    \end{itemize}

    \textbf{2. Enhancing Creativity:}
    \begin{itemize}
        \item Teamwork fosters creativity through diverse perspectives.
        \item \textit{Illustration:} Collaboration between computer vision engineers and UI designers leads to innovative applications.
    \end{itemize}
\end{frame}

\begin{frame}[fragile]
    \frametitle{Teamwork Dynamics and Presentation Skills}
    \textbf{Teamwork Dynamics:}
    \begin{itemize}
        \item \textbf{Communication Skills:} 
        \begin{itemize}
            \item Essential for team dynamics and clarity.
            \item Regular meetings and collaborative tools enhance communication.
        \end{itemize}
        
        \item \textbf{Role Distribution:} 
        \begin{itemize}
            \item Clearly defined roles improve efficiency.
            \item \textit{Example Roles:}
            \begin{itemize}
                \item Data Engineer: Prepares data pipelines.
                \item Data Scientist: Develops models.
                \item Project Manager: Coordinates timelines.
            \end{itemize}
        \end{itemize}

        \item \textbf{Peer Review:} 
        \begin{itemize}
            \item Encourages quality work and learning.
            \item Regular code reviews mitigate risks.
        \end{itemize}
    \end{itemize}
    
    \textbf{Importance of Presentation Skills:}
    \begin{itemize}
        \item Sharing ideas effectively influences decision-making.
        \item Feedback during presentations facilitates iterative improvement.
        \item Knowledge transfer is enhanced through effective presentations.
    \end{itemize}
\end{frame}

\begin{frame}[fragile]
    \frametitle{Future of AI - Overview}
    \begin{block}{Introduction to the Future of AI}
        The future of Artificial Intelligence (AI) holds immense potential to transform industries, enhance human capabilities, and address global challenges. Understanding the future trends in AI and their societal implications is crucial as technology continues to evolve.
    \end{block}
\end{frame}

\begin{frame}[fragile]
    \frametitle{Future of AI - Key Trends}
    \begin{enumerate}
        \item \textbf{Advancements in Machine Learning}:
            \begin{itemize}
               \item AI models are becoming more sophisticated, enhancing predictive capabilities through deep learning and neural networks.
               \item Example: Automated decision-making systems in healthcare utilize predictive analytics to improve patient outcomes.
            \end{itemize}
            
        \item \textbf{Natural Language Processing (NLP)}:
            \begin{itemize}
               \item Enhancements in NLP lead to better human-machine interactions and understanding of context, sentiment, and intent.
               \item Example: Virtual assistants like Google Assistant and Siri evolve to hold more nuanced conversations.
            \end{itemize}
            
        \item \textbf{Ethical AI and Governance}:
            \begin{itemize}
                \item Increased focus on ethical considerations, responsible AI use, and bias mitigation.
                \item Example: Organizations establish AI ethics boards to ensure transparency and fairness in AI applications.
            \end{itemize}

        \item \textbf{AI in Automation and Robotics}:
            \begin{itemize}
                \item Robots and AI systems are automating routine tasks across sectors such as manufacturing and logistics.
                \item Example: Autonomous delivery drones are tested for food and package delivery to enhance service speed and efficiency.
            \end{itemize}
    
        \item \textbf{AI in Predictive Analytics and Decision Support}:
            \begin{itemize}
                \item AI systems are increasingly used for data-driven decision-making and providing insights from large data sets.
                \item Example: Financial institutions use AI to detect fraudulent activity in real-time by analyzing transaction patterns.
            \end{itemize}
    \end{enumerate}
\end{frame}

\begin{frame}[fragile]
    \frametitle{Future of AI - Societal Implications}
    \begin{enumerate}
        \item \textbf{Job Displacement vs. Job Creation}:
            \begin{itemize}
                \item AI may displace certain jobs but also creates new roles focused on oversight and management of AI systems.
                \item \textbf{Key Point}: Upskilling and reskilling will be essential for the workforce to adapt to these changes.
            \end{itemize}
            
        \item \textbf{Privacy and Security Challenges}:
            \begin{itemize}
                \item Increased data collection by AI systems raises concerns about privacy, security, and surveillance.
                \item \textbf{Key Point}: Policies must evolve to safeguard individual rights in the age of AI.
            \end{itemize}
            
        \item \textbf{AI and Inequality}:
            \begin{itemize}
                \item AI technology risks widening the gap between those with access to AI tools and those without.
                \item \textbf{Key Point}: Ensuring equitable access to AI technology is critical for inclusive benefits.
            \end{itemize}
    \end{enumerate}
\end{frame}

\begin{frame}[fragile]
    \frametitle{Recap of Key Points}
    \begin{enumerate}
        \item \textbf{What is Artificial Intelligence (AI)?}
        \begin{itemize}
            \item Simulation of human intelligence processes by machines.
            \item Key processes: learning, reasoning, and self-correction.
        \end{itemize}
        
        \item \textbf{Types of AI:}
        \begin{itemize}
            \item \textbf{Narrow AI:} Task-specific (e.g., chatbots).
            \item \textbf{General AI:} Hypothetical, capable of human-like tasks.
            \item \textbf{Superintelligent AI:} Surpasses human intelligence.
        \end{itemize}
    \end{enumerate}
\end{frame}

\begin{frame}[fragile]
    \frametitle{Agent Architectures and Implications}
    \begin{enumerate}
        \setcounter{enumi}{3} % Continue enumeration
        \item \textbf{Agent Architectures:}
        \begin{itemize}
            \item \textbf{Reactive Agents:} Respond to stimuli without memory.
            \item \textbf{Deliberative Agents:} Use world models for planning.
            \item \textbf{Hybrid Agents:} Combine reactive and deliberative features.
        \end{itemize}

        \item \textbf{Implications of AI:}
        \begin{itemize}
            \item Increased efficiency in fields like healthcare and finance.
            \item Ethical considerations: bias and transparency in decision-making.
        \end{itemize}
    \end{enumerate}
\end{frame}

\begin{frame}[fragile]
    \frametitle{Examples and Q\&A}
    \begin{block}{Examples and Illustrations}
        \begin{itemize}
            \item Examples of Narrow AI: AI-driven personal assistants (Siri, Alexa).
            \item Example of an Agent Architecture: A self-navigating robot (Deliberative Agent) using GPS.
        \end{itemize}
    \end{block}

    \begin{block}{Key Points to Emphasize}
        \begin{itemize}
            \item Distinction between different types of AI.
            \item Agent architecture influences AI system behavior.
            \item Importance of considering ethical implications of AI.
        \end{itemize}
    \end{block}

    \begin{block}{Q\&A Session}
        \begin{itemize}
            \item Open the floor for questions.
            \item Discuss real-world applications of AI.
            \item Clarify doubts regarding agent architectures.
        \end{itemize}
    \end{block}
\end{frame}


\end{document}