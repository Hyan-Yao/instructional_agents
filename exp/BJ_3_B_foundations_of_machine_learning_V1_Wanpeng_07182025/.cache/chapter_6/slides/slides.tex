\documentclass[aspectratio=169]{beamer}

% Theme and Color Setup
\usetheme{Madrid}
\usecolortheme{whale}
\useinnertheme{rectangles}
\useoutertheme{miniframes}

% Additional Packages
\usepackage[utf8]{inputenc}
\usepackage[T1]{fontenc}
\usepackage{graphicx}
\usepackage{booktabs}
\usepackage{listings}
\usepackage{amsmath}
\usepackage{amssymb}
\usepackage{xcolor}
\usepackage{tikz}
\usepackage{pgfplots}
\pgfplotsset{compat=1.18}
\usetikzlibrary{positioning}
\usepackage{hyperref}

% Custom Colors
\definecolor{myblue}{RGB}{31, 73, 125}
\definecolor{mygray}{RGB}{100, 100, 100}
\definecolor{mygreen}{RGB}{0, 128, 0}
\definecolor{myorange}{RGB}{230, 126, 34}
\definecolor{mycodebackground}{RGB}{245, 245, 245}

% Set Theme Colors
\setbeamercolor{structure}{fg=myblue}
\setbeamercolor{frametitle}{fg=white, bg=myblue}
\setbeamercolor{title}{fg=myblue}
\setbeamercolor{section in toc}{fg=myblue}
\setbeamercolor{item projected}{fg=white, bg=myblue}
\setbeamercolor{block title}{bg=myblue!20, fg=myblue}
\setbeamercolor{block body}{bg=myblue!10}
\setbeamercolor{alerted text}{fg=myorange}

% Set Fonts
\setbeamerfont{title}{size=\Large, series=\bfseries}
\setbeamerfont{frametitle}{size=\large, series=\bfseries}
\setbeamerfont{caption}{size=\small}
\setbeamerfont{footnote}{size=\tiny}

% Document Start
\begin{document}

\frame{\titlepage}

\begin{frame}[fragile]
    \frametitle{Introduction to Unsupervised Learning}
    \begin{block}{What is Unsupervised Learning?}
        Unsupervised learning is a type of machine learning that analyzes and interprets data without pre-defined labels. 
        Unlike supervised learning, it explores patterns and structures within the input data itself.
    \end{block}
\end{frame}

\begin{frame}[fragile]
    \frametitle{Importance of Unsupervised Learning}
    \begin{enumerate}
        \item \textbf{Data Exploration:} Helps discover hidden patterns and insights in data that are not immediately obvious.
        \item \textbf{Dimensionality Reduction:} Simplifies datasets by reducing feature space while retaining essential information.
        \item \textbf{Preprocessing:} Aids in grouping similar data points to enhance the performance of supervised learning algorithms.
    \end{enumerate}
\end{frame}

\begin{frame}[fragile]
    \frametitle{Focus on Clustering}
    \begin{block}{Clustering Defined}
        Clustering is a pivotal unsupervised learning technique that involves grouping similar data points based on defined similarity measures. 
        The core idea is that objects within the same cluster are more similar to each other than to those in other clusters.
    \end{block}
    \begin{itemize}
        \item \textbf{No Labels:} Operates on unlabelled data.
        \item \textbf{Applications:} Customer segmentation, image recognition, and anomaly detection.
        \item \textbf{Algorithms:} Common algorithms include K-Means, Hierarchical Clustering, and DBSCAN.
    \end{itemize}
\end{frame}

\begin{frame}[fragile]
    \frametitle{Examples of Clustering}
    \begin{itemize}
        \item \textbf{Customer Segmentation:} Businesses analyze purchase behavior data to group customers for targeted marketing strategies.
        \item \textbf{Image Segmentation:} In computer vision, clustering partitions images into segments for further analysis.
    \end{itemize}
\end{frame}

\begin{frame}[fragile]
    \frametitle{Summary}
    Unsupervised learning, particularly through clustering, is essential for organizing and understanding complex datasets. 
    It enables data scientists and analysts to extract meaningful insights without predefined labels.
\end{frame}

\begin{frame}[fragile]
    \frametitle{What is Clustering? - Definition}
    \begin{block}{Definition of Clustering}
        Clustering is a fundamental unsupervised learning technique in machine learning that groups similar data points into clusters.
    \end{block}
    \begin{block}{Key Distinction from Supervised Learning}
        \begin{itemize}
            \item \textbf{Supervised Learning:} Uses labeled datasets to predict outcomes.
            \item \textbf{Unsupervised Learning:} Focuses on identifying structures within unlabeled data.
        \end{itemize}
    \end{block}
\end{frame}

\begin{frame}[fragile]
    \frametitle{Clustering in Unsupervised Learning}
    \begin{block}{Role of Clustering}
        \begin{itemize}
            \item \textbf{Data Exploration:} Helps explore and understand data by segmenting it into meaningful groups.
            \item \textbf{Pattern Recognition:} Identifies patterns and anomalies without prior knowledge.
            \item \textbf{Preprocessing Step:} Simplifies datasets to improve model performance in data analysis pipelines.
        \end{itemize}
    \end{block}
\end{frame}

\begin{frame}[fragile]
    \frametitle{Applications and Examples}
    \begin{block}{Applications of Clustering}
        Clustering is widely used in fields such as:
        \begin{itemize}
            \item \textbf{Marketing:} Customer segmentation
            \item \textbf{Biology:} Gene grouping
            \item \textbf{Social Science:} Community detection
        \end{itemize}
    \end{block}

    \begin{block}{Example}
        Consider a dataset of customers' purchasing habits. Clustering can identify segments like:
        \begin{itemize}
            \item Frequent buyers
            \item Occasional shoppers
            \item Discount seekers
        \end{itemize}
        This segmentation aids targeted marketing strategies.
    \end{block}
\end{frame}

\begin{frame}[fragile]
    \frametitle{Types of Clustering}
    \begin{block}{Introduction to Clustering Methods}
        Clustering is a key technique in unsupervised learning, allowing us to group data points based on similarities without pre-labeled outputs.
        Among various clustering methods, two of the most prominent are \textbf{K-Means clustering} and \textbf{Hierarchical clustering}.
        Understanding these methods is crucial for effectively analyzing data patterns.
    \end{block}
\end{frame}

\begin{frame}[fragile]
    \frametitle{1. K-Means Clustering}
    \begin{block}{Definition}
        K-Means is a centroid-based clustering algorithm that partitions data into $k$ distinct clusters. Each cluster is defined by its centroid (the mean of all points in the cluster).
    \end{block}
    
    \begin{block}{How It Works}
        \begin{enumerate}
            \item \textbf{Initialization:} Choose $k$ initial centroids randomly from the dataset.
            \item \textbf{Assignment:} Allocate each data point to the nearest centroid based on Euclidean distance.
            \item \textbf{Update:} Recalculate the centroids as the mean of all points assigned to each cluster.
            \item \textbf{Repeat:} Iterate the Assignment and Update steps until centroids stabilize.
        \end{enumerate}
    \end{block}

    \begin{block}{Key Points}
        \begin{itemize}
            \item Requires the number of clusters $k$ to be specified beforehand.
            \item Sensitive to outliers—can skew centroids.
            \item Efficient and scalable for large datasets.
        \end{itemize}
    \end{block}
\end{frame}

\begin{frame}[fragile]
    \frametitle{K-Means: Euclidean Distance}
    The formula for Euclidean Distance used in K-Means is:

    \begin{equation}
        d(x_i, c_j) = \sqrt{\sum_{m=1}^{n} (x_{im} - c_{jm})^2}
    \end{equation}
    
    where \(x_i\) is the $i$th data point, \(c_j\) is the $j$th centroid, and \(n\) is the number of features.
\end{frame}

\begin{frame}[fragile]
    \frametitle{2. Hierarchical Clustering}
    \begin{block}{Definition}
        Hierarchical clustering builds a hierarchy of clusters either in an agglomerative (bottom-up) or divisive (top-down) manner.
    \end{block}
    
    \begin{block}{How It Works}
        \begin{itemize}
            \item \textbf{Agglomerative Approach:}
              \begin{enumerate}
                  \item Start with each data point as its own cluster.
                  \item Merge the closest pairs of clusters until only one cluster remains or a desired number of clusters is achieved.
              \end{enumerate}
            \item \textbf{Divisive Approach:}
              \begin{enumerate}
                  \item Start with one cluster that contains all data points.
                  \item Iteratively split the most dissimilar cluster until each point is individual or a desired number of clusters is formed.
              \end{enumerate}
        \end{itemize}
    \end{block}
\end{frame}

\begin{frame}[fragile]
    \frametitle{Hierarchical Clustering: Key Points}
    \begin{block}{Key Points}
        \begin{itemize}
            \item Produces a dendrogram that visually represents the merging process.
            \item Does not require a predetermined number of clusters.
            \item Can be computationally expensive for large datasets due to its complexity (O(n³) for agglomerative).
        \end{itemize}
    \end{block}

    \begin{block}{Dendrogram Example}
        A dendrogram is a tree diagram that illustrates the arrangement of the clusters. The vertical axis represents the distance at which clusters are merged.
    \end{block}
\end{frame}

\begin{frame}[fragile]
    \frametitle{Summary}
    \begin{itemize}
        \item \textbf{K-Means} is ideal for large datasets and requires pre-defining the number of clusters.
        \item \textbf{Hierarchical Clustering} provides a detailed structure of data relationships but is more computationally intensive.
    \end{itemize}
    
    Understanding these clustering methods allows you to choose the most suitable approach for your data analysis requirements effectively!
\end{frame}

\begin{frame}[fragile]
    \frametitle{K-Means Clustering: Overview}
    \begin{block}{What is K-Means Clustering?}
        K-Means Clustering is an unsupervised learning algorithm that partitions a dataset into *k* distinct, non-overlapping subgroups or clusters. The main objective is to group similar data points together while ensuring that data points in different clusters are as dissimilar as possible.
    \end{block}
    
    \begin{block}{Purpose of K-Means Clustering}
        \begin{itemize}
            \item \textbf{Data Segmentation}: Dividing data into meaningful segments for analysis.
            \item \textbf{Pattern Recognition}: Identifying patterns and structures in unlabelled data.
            \item \textbf{Dimensionality Reduction}: Reducing complexity by grouping data points for easier visualization.
        \end{itemize}
    \end{block}
\end{frame}

\begin{frame}[fragile]
    \frametitle{How K-Means Works}
    \begin{enumerate}
        \item \textbf{Initialization}: Select *k* initial centroids randomly from the dataset.
        \item \textbf{Assignment}: Assign each data point to the nearest centroid, forming *k* clusters.
        \item \textbf{Update}: Recalculate the centroids as the mean of all data points in each cluster.
        \item \textbf{Iteration}: Repeat the assignment and update steps until centroids do not change significantly or a predetermined number of iterations is reached.
    \end{enumerate}
\end{frame}

\begin{frame}[fragile]
    \frametitle{Applications and Key Points}
    \begin{block}{Applications of K-Means Clustering}
        \begin{itemize}
            \item \textbf{Market Segmentation}: Identifying customer segments.
            \item \textbf{Image Compression}: Reducing color complexity in images.
            \item \textbf{Anomaly Detection}: Finding outliers in data sets.
            \item \textbf{Document Clustering}: Grouping documents for topic modeling.
        \end{itemize}
    \end{block}

    \begin{block}{Key Points}
        \begin{itemize}
            \item \textbf{Choice of K}: The number of clusters (*k*) should be chosen carefully.
            \item \textbf{Scalability}: K-Means is efficient for large datasets but may face challenges with very high dimensions.
            \item \textbf{Sensitivity to Initial Conditions}: The outcome can depend on how centroids are initialized.
        \end{itemize}
    \end{block}
\end{frame}

\begin{frame}[fragile]
    \frametitle{K-Means Objective Function}
    The K-Means objective function aims to minimize the within-cluster sum of squares (WCSS):
    \begin{equation}
        J = \sum_{i=1}^k \sum_{x \in C_i} \| x - \mu_i \|^2
    \end{equation}
    Where:
    \begin{itemize}
        \item \( J \) = total distance metric (inertia)
        \item \( k \) = number of clusters
        \item \( C_i \) = data points in cluster \( i \)
        \item \( \mu_i \) = centroid of cluster \( i \)
        \item \( \| x - \mu_i \|^2 \) = squared Euclidean distance between a point \( x \) and the centroid \( \mu_i \)
    \end{itemize}
\end{frame}

\begin{frame}[fragile]
    \frametitle{Conclusion and Code Snippet}
    \begin{block}{Conclusion}
        K-Means Clustering is a powerful method for finding natural groupings in data. Its efficiency and simplicity make it a popular choice across various fields.
    \end{block}
    
    \begin{block}{Code Snippet (Python Example)}
        \begin{lstlisting}[language=Python]
from sklearn.cluster import KMeans

# Sample data: 2D Points
data = [[1, 2], [1, 4], [1, 0],
        [4, 2], [4, 4], [4, 0]]

# Applying K-Means
kmeans = KMeans(n_clusters=2, random_state=0).fit(data)

# Getting cluster labels
labels = kmeans.labels_
print(labels)
        \end{lstlisting}
    \end{block}
\end{frame}

\begin{frame}[fragile]
    \frametitle{K-Means Algorithm: Overview}
    \begin{block}{Overview of the K-Means Algorithm}
        K-Means is a popular clustering algorithm in unsupervised learning that partitions data into K distinct clusters based on feature similarity. The process involves iterating through a series of steps until convergence occurs, where cluster assignments no longer change.
    \end{block}
\end{frame}

\begin{frame}[fragile]
    \frametitle{K-Means Algorithm: Steps}
    \begin{enumerate}
        \item \textbf{Initialization:}
        \begin{itemize}
            \item Choose the number of clusters (K).
            \item Randomly select K data points as initial centroids.
        \end{itemize}
        
        \begin{block}{Key Point}
            Proper initialization can significantly affect the algorithm's performance and the final clusters.
        \end{block}
    \end{enumerate}
\end{frame}

\begin{frame}[fragile]
    \frametitle{K-Means Algorithm: Assignment Step}
    \begin{enumerate}[resume]
        \item \textbf{Assignment Step:}
        \begin{itemize}
            \item Assign each data point to the nearest centroid using the Euclidean distance:
            \begin{equation}
                \text{distance}(x_i, C_k) = \sqrt{\sum (x_{ij} - C_{kj})^2}
            \end{equation}
            \item Repeat for all data points, forming K clusters based on proximity to centroids.
        \end{itemize}
        
        \begin{block}{Illustration}
            Imagine points scattered on a 2D plane, each colored based on the closest centroid.
        \end{block}
    \end{enumerate}
\end{frame}

\begin{frame}[fragile]
    \frametitle{K-Means Algorithm: Update Step}
    \begin{enumerate}[resume]
        \item \textbf{Update Step:}
        \begin{itemize}
            \item Recalculate centroids:
            \begin{equation}
                C_k^{new} = \frac{1}{|C_k|} \sum_{x_j \in C_k} x_j
            \end{equation}
            \item Return to assignment step and repeat until convergence or a maximum number of iterations is reached.
        \end{itemize}
        
        \begin{block}{Example}
            An initial assignment may change after centroid updates, necessitating a re-evaluation of clusters.
        \end{block}
    \end{enumerate}
\end{frame}

\begin{frame}[fragile]
    \frametitle{K-Means Algorithm: Conclusion}
    \begin{block}{Conclusion}
        The K-Means algorithm iteratively refines cluster assignments and centroids until stable clusters are achieved. It is crucial to understand these steps for effective application across various data scenarios.
    \end{block}
    
    \begin{block}{Note}
        The algorithm can be sensitive to initial centroid placement, outliers, and the chosen K value; hence careful consideration is essential in interpreting results.
    \end{block}
\end{frame}

\begin{frame}[fragile]
    \frametitle{Choosing K: The Number of Clusters}
    \begin{block}{Understanding K in Clustering}
        In clustering algorithms, particularly K-Means, one of the most critical decisions is determining the optimal number of clusters, K. The choice of K significantly influences the results and interpretation of the clustering process.
    \end{block}
\end{frame}

\begin{frame}[fragile]
    \frametitle{Methods for Choosing K}
    \begin{enumerate}
        \item \textbf{Elbow Method}
        \begin{itemize}
            \item \textbf{Concept}: Identify the point where adding more clusters yields diminishing returns in terms of variance explained.
            \item \textbf{Process}:
                \begin{enumerate}
                    \item Run K-Means for a range of K values (e.g., from 1 to 10).
                    \item Calculate the total within-cluster sum of squares (WCSS) for each K.
                    \item Plot K against WCSS.
                    \item Look for the "elbow" point in the plot.
                \end{enumerate}
            \item \textbf{Interpretation}: The elbow point indicates the most suitable K, balancing model complexity and accuracy.
            \item \textbf{Example}: K=3 may be optimal if you see a significant drop in WCSS up to K=3 and minimal change afterward.
        \end{itemize}
    \end{enumerate}
\end{frame}

\begin{frame}[fragile]
    \frametitle{Methods for Choosing K (cont.)}
    \begin{enumerate}
        \setcounter{enumi}{1}
        \item \textbf{Silhouette Score}
        \begin{itemize}
            \item \textbf{Concept}: Measures how similar an object is to its own cluster compared to other clusters.
            \item \textbf{Calculation}:
                \begin{itemize}
                    \item Compute the average distance to points in the same cluster (a) and the average distance to points in the next nearest cluster (b).
                    \item Silhouette score for a point: 
                    \begin{equation}
                        s(i) = \frac{b(i) - a(i)}{\max(a(i), b(i))}
                    \end{equation}
                \end{itemize}
            \item \textbf{Range}: 
                \begin{itemize}
                    \item \textbf{+1}: Well-clustered data points.
                    \item \textbf{0}: Points near cluster boundaries.
                    \item \textbf{-1}: Misclassified points.
                \end{itemize}
            \item \textbf{Example}: K=4 is optimal if the average silhouette score peaks at K=4.
        \end{itemize}
    \end{enumerate}
\end{frame}

\begin{frame}[fragile]
    \frametitle{Key Points to Emphasize}
    \begin{itemize}
        \item Determining K is essential for effective clustering and influences result interpretability.
        \item Both the Elbow Method and Silhouette Score are widely used and can yield different K values; considering both methods for validation is useful.
        \item Visualization of clustering results is important for qualitatively assessing cluster separation.
    \end{itemize}
\end{frame}

\begin{frame}[fragile]
    \frametitle{K-Means Clustering Example}
    A real-world example illustrating the K-Means clustering process with visualizations.
\end{frame}

\begin{frame}[fragile]
    \frametitle{Overview of K-Means Clustering}
    \begin{itemize}
        \item K-Means is an unsupervised learning algorithm for partitioning data into \(K\) distinct clusters.
        \item Objective: To categorize data points such that those within the same cluster are similar to each other.
    \end{itemize}
\end{frame}

\begin{frame}[fragile]
    \frametitle{The K-Means Clustering Process}
    \begin{enumerate}
        \item \textbf{Initialization}
            \begin{itemize}
                \item Choose number of clusters \(K\).
                \item Randomly select \(K\) initial centroids.
            \end{itemize}
        \item \textbf{Assignment Step}
            \begin{itemize}
                \item Assign each data point to the nearest centroid using Euclidean distance:
                \begin{equation}
                    \text{Distance} = \sqrt{(x_2 - x_1)^2 + (y_2 - y_1)^2}
                \end{equation}
            \end{itemize}
        \item \textbf{Update Step}
            \begin{itemize}
                \item Calculate new centroids:
                \begin{equation}
                    \mu_j = \frac{1}{n_j} \sum_{x_i \in C_j} x_i
                \end{equation}
                \item \(n_j\) = number of points in cluster \(j\), \(C_j\) = points in cluster \(j\).
            \end{itemize}
        \item \textbf{Iteration}
            \begin{itemize}
                \item Repeat Assignment and Update steps until convergence.
            \end{itemize}
    \end{enumerate}
\end{frame}

\begin{frame}[fragile]
    \frametitle{Real-World Example: Customer Segmentation}
    \begin{itemize}
        \item Retail business segments customers based on purchasing behavior.
        \item Features:
            \begin{itemize}
                \item \textbf{X-axis}: Amount spent annually
                \item \textbf{Y-axis}: Frequency of purchases
            \end{itemize}
    \end{itemize}
    \begin{block}{Illustration Steps}
        \begin{enumerate}
            \item \textbf{Initial State}: Scattered points representing customers with random centroids.
            \item \textbf{After Assignment}: Customers grouped around nearest centroids.
            \item \textbf{After Update}: New centroids calculated as averages of their assigned customers.
        \end{enumerate}
    \end{block}
\end{frame}

\begin{frame}[fragile]
    \frametitle{Key Points to Emphasize}
    \begin{itemize}
        \item \textbf{Cluster Formation}: Clusters form around centroids, showcasing natural groupings.
        \item \textbf{Flexibility}: Applicable to various datasets; versatile clustering tool.
        \item \textbf{Scalability}: Efficient for large datasets but sensitive to choice of \(K\) and initialization.
    \end{itemize}
\end{frame}

\begin{frame}[fragile]
    \frametitle{Important Considerations}
    \begin{itemize}
        \item \textbf{Choosing \(K\)}: Critical for meaningful results.
        \item \textbf{Sensitivity to Initialization}: Different initial centroids yield varied outcomes.
        \item \textbf{Limitations}: Assumes spherical clusters; may struggle with non-globular shapes or varying density.
    \end{itemize}
\end{frame}

\begin{frame}[fragile]
    \frametitle{Conclusion}
    K-Means clustering effectively identifies patterns by grouping similar items. In applications like customer segmentation, it helps businesses tailor strategies to specific groups, enhancing satisfaction and profitability.
\end{frame}

\begin{frame}[fragile]
    \frametitle{Strengths and Limitations of K-Means - Overview}
    \begin{block}{K-Means Clustering Overview}
        K-Means is a widely used unsupervised learning algorithm that partitions data into K distinct clusters based on feature similarity. 
        The algorithm works iteratively to assign data points to clusters and update cluster centroids until convergence.
    \end{block}
\end{frame}

\begin{frame}[fragile]
    \frametitle{Strengths of K-Means}
    \begin{enumerate}
        \item \textbf{Computational Efficiency}
        \begin{itemize}
            \item Fast performance with linear time complexity of \(O(n \times K \times i)\).
            \item Scalable to large datasets across various applications.
        \end{itemize}
        
        \item \textbf{Ease of Implementation}
        \begin{itemize}
            \item Simple and intuitive algorithm compared to other clustering methods.
        \end{itemize}
        
        \item \textbf{Versatility}
        \begin{itemize}
            \item Applicable in multiple domains, suited for spherical clusters with equal size.
            \item Flexible definition of the number of clusters (K).
        \end{itemize}
    \end{enumerate}
\end{frame}

\begin{frame}[fragile]
    \frametitle{Limitations of K-Means}
    \begin{enumerate}
        \item \textbf{Sensitivity to Initial Conditions}
        \begin{itemize}
            \item Choice of initial centroids impacts clustering results.
            \item Techniques like K-Means++ improve initialization.
        \end{itemize}
        
        \item \textbf{Fixed Number of Clusters (K)}
        \begin{itemize}
            \item Requires pre-specification of K, which may not be known upfront.
            \item Methods like the elbow method can help estimate K.
        \end{itemize}
        
        \item \textbf{Assumption of Spherical Clusters}
        \begin{itemize}
            \item Assumes similar cluster size and spherical shapes, leading to poor results in complex datasets.
        \end{itemize}
        
        \item \textbf{Sensitivity to Noisy Data and Outliers}
        \begin{itemize}
            \item Outliers can skew centroids, resulting in inaccurate clustering outcomes.
        \end{itemize}
    \end{enumerate}
\end{frame}

\begin{frame}[fragile]
    \frametitle{Key Takeaway Points and Example Formula}
    \begin{block}{Key Takeaway Points}
        \begin{itemize}
            \item K-Means is powerful and efficient but has its constraints.
            \item Understanding strengths and limitations aids in refining clustering strategies.
        \end{itemize}
    \end{block}
    
    \begin{block}{Example Formula}
        The objective function of K-Means is to minimize the within-cluster sum of squares (WCSS):
        \begin{equation}
        J = \sum_{i=1}^{K} \sum_{x \in C_i} ||x - \mu_i||^2 
        \end{equation}
        Where:
        \begin{itemize}
            \item \(J\) = Total cost (within-cluster sum of squares)
            \item \(C_i\) = Cluster \(i\)
            \item \(x\) = Data point in cluster \(C_i\)
            \item \(\mu_i\) = Centroid of cluster \(C_i\)
        \end{itemize}
    \end{block}
\end{frame}

\begin{frame}[fragile]
    \frametitle{Conclusion}
    In summary, while K-Means is a valuable clustering tool, understanding its strengths and limitations is crucial for effective application in data-driven analysis.
\end{frame}

\begin{frame}[fragile]
    \frametitle{Hierarchical Clustering: Overview}
    \begin{block}{What is Hierarchical Clustering?}
        Hierarchical clustering is an unsupervised learning technique that groups a set of objects into clusters based on their similarity.
        \begin{itemize}
            \item Unlike K-Means, it builds a hierarchy of clusters.
            \item Provides a more informative structure.
        \end{itemize}
    \end{block}
\end{frame}

\begin{frame}[fragile]
    \frametitle{Approaches to Hierarchical Clustering}
    \begin{enumerate}
        \item \textbf{Agglomerative Approach (Bottom-Up)}
        \begin{itemize}
            \item \textbf{Description}: Starts with each data point as its own cluster and merges them iteratively.
            \item \textbf{Steps}:
            \begin{itemize}
                \item Compute distances between clusters.
                \item Merge closest clusters using a link function (e.g., single, complete).
                \item Repeat until one cluster remains or desired count is achieved.
            \end{itemize}
            \item \textbf{Example}: Points A, B, C, and D merge based on proximity to form clusters like \{A, B\} and \{C, D\}.
        \end{itemize}

        \item \textbf{Divisive Approach (Top-Down)}
        \begin{itemize}
            \item \textbf{Description}: Begins with one cluster and recursively splits into smaller clusters.
            \item \textbf{Steps}:
            \begin{itemize}
                \item Start with one cluster of all data points.
                \item Split using a distance metric.
                \item Continue until clusters meet a criterion (e.g., minimum size).
            \end{itemize}
            \item \textbf{Example}: Starting from one cluster, splitting might occur first into major groups, then refined further.
        \end{itemize}
    \end{enumerate}
\end{frame}

\begin{frame}[fragile]
    \frametitle{Key Points on Hierarchical Clustering}
    \begin{itemize}
        \item \textbf{Flexibility}:
            \begin{itemize}
                \item Does not require the number of clusters to be specified in advance.
                \item Enables exploratory data analysis.
            \end{itemize}
        \item \textbf{Dendrogram Representation}:
            \begin{itemize}
                \item Visualizes the clustering process.
                \item Illustrates the order and distance of merges or splits.
            \end{itemize}
        \item \textbf{Distance Metrics}:
            \begin{itemize}
                \item Common metrics include:
                \begin{itemize}
                    \item Euclidean
                    \item Manhattan
                    \item Cosine distance
                \end{itemize}
            \end{itemize}
    \end{itemize}
\end{frame}

\begin{frame}[fragile]
    \frametitle{Hierarchical Clustering: Dendrograms - Introduction}
    \begin{block}{What is a Dendrogram?}
        A dendrogram is a tree-like diagram that visually represents the arrangement of clusters formed by hierarchical clustering. It illustrates how data points are grouped and displays the relationships among clusters based on similarities or distances.
    \end{block}
\end{frame}

\begin{frame}[fragile]
    \frametitle{Hierarchical Clustering: Dendrograms - Structure}
    \begin{itemize}
        \item \textbf{Leaves}: Represent individual data points or observations.
        \item \textbf{Branches}: Connect the leaves, showing hierarchical relationships.
        \item \textbf{Height}: Indicates distance or dissimilarity between clusters; higher branches suggest greater distance.
    \end{itemize}
\end{frame}

\begin{frame}[fragile]
    \frametitle{Hierarchical Clustering: Dendrograms - How They Work}
    To create a dendrogram, a hierarchical clustering algorithm is utilized. Two main approaches:
    \begin{enumerate}
        \item \textbf{Agglomerative}: Start with each data point as its own cluster, merging the closest until one cluster remains.
        \item \textbf{Divisive}: Begin with all data points as one cluster and recursively split until each point is its own cluster.
    \end{enumerate}
\end{frame}

\begin{frame}[fragile]
    \frametitle{Hierarchical Clustering: Dendrograms - Example Interpretation}
    Consider three data points: A, B, and C.
    \begin{itemize}
        \item If A and B are more similar to each other than to C, the dendrogram illustrates that A and B merge before C, indicating a closer cluster.
    \end{itemize}
\end{frame}

\begin{frame}[fragile]
    \frametitle{Hierarchical Clustering: Dendrograms - Key Points}
    \begin{itemize}
        \item \textbf{Cluster Formation}: Dendrograms show step-by-step cluster formation based on similarity.
        \item \textbf{Thresholding}: The optimal number of clusters can be determined by cutting the dendrogram at a certain height.
        \item \textbf{Versatility}: Applies in various domains (e.g., biology, marketing).
    \end{itemize}
\end{frame}

\begin{frame}[fragile]
    \frametitle{Hierarchical Clustering: Dendrograms - Conclusion}
    Dendrograms are powerful tools in hierarchical clustering, making relationships among clusters clear and interpretable. Understanding dendrograms aids in extracting meaningful insights from data.
\end{frame}

\begin{frame}[fragile]
    \frametitle{Hierarchical Clustering: Algorithm Steps}
    \textbf{Understanding Hierarchical Clustering}
    
    Hierarchical clustering is a method of clustering that seeks to build a hierarchy of clusters. 
    It is divided into two main strategies: \textbf{Agglomerative} and \textbf{Divisive}.
\end{frame}

\begin{frame}[fragile]
    \frametitle{Agglomerative Clustering: Bottom-Up Approach}
    \begin{enumerate}
        \item \textbf{Initialization}:
        \begin{itemize}
            \item Treat each data point as a single cluster. If there are \( n \) data points, you begin with \( n \) clusters.
        \end{itemize}
        
        \item \textbf{Distance Calculation}:
        \begin{itemize}
            \item Compute the pairwise distances (e.g., Euclidean distance) between each cluster.
        \end{itemize}

        \item \textbf{Merge Clusters}:
        \begin{itemize}
            \item Identify the two closest clusters and merge them into a single cluster.
        \end{itemize}

        \item \textbf{Update Distances}:
        \begin{itemize}
            \item Recalculate distances using a defined linkage criterion (e.g., single, complete, or average linkage).
        \end{itemize}

        \item \textbf{Repeat}:
        \begin{itemize}
            \item Continue steps 2-4 until all points are merged into a single cluster.
        \end{itemize}

        \item \textbf{Dendrogram Creation}:
        \begin{itemize}
            \item Construct a dendrogram to visualize the merging process, depicting the order and distance of merges.
        \end{itemize}
    \end{enumerate}
\end{frame}

\begin{frame}[fragile]
    \frametitle{Divisive Clustering: Top-Down Approach}
    \begin{enumerate}
        \item \textbf{Initialization}:
        \begin{itemize}
            \item Start with a single cluster that includes all data points.
        \end{itemize}

        \item \textbf{Splitting Clusters}:
        \begin{itemize}
            \item Determine the most heterogeneous cluster using variance or distance.
        \end{itemize}

        \item \textbf{Create Subclusters}:
        \begin{itemize}
            \item Split the selected cluster into two or more clusters based on a specified criterion.
        \end{itemize}

        \item \textbf{Repeat}:
        \begin{itemize}
            \item Continue splitting clusters until each data point is its own cluster or a stopping criterion is met.
        \end{itemize}

        \item \textbf{Dendrogram Creation}:
        \begin{itemize}
            \item Create a dendrogram to visualize the splits, similar to agglomerative clustering.
        \end{itemize}
    \end{enumerate}
\end{frame}

\begin{frame}[fragile]
    \frametitle{Key Points to Emphasize}
    \begin{block}{Distance Metrics}
        Critical to the algorithm’s performance. Common metrics include:
        \begin{itemize}
            \item \textbf{Euclidean}: \( d(x, y) = \sqrt{\sum (x_i - y_i)^2} \)
            \item \textbf{Manhattan}: \( d(x, y) = \sum |x_i - y_i| \)
        \end{itemize}
    \end{block}
    
    \begin{block}{Linkage Criteria Examples}
        \begin{itemize}
            \item \textbf{Single Linkage}: Minimum distance between clusters.
            \item \textbf{Complete Linkage}: Maximum distance between clusters.
            \item \textbf{Average Linkage}: Average distance between all pairs of points in the clusters.
        \end{itemize}
    \end{block}

    \begin{block}{Dendrogram Interpretation}
        Heights of merges indicate the dissimilarity between clusters.
    \end{block}
\end{frame}

\begin{frame}[fragile]
    \frametitle{Example of Hierarchical Clustering}
    Imagine we have five data points A, B, C, D, and E. 

    \begin{enumerate}
        \item Start with: \{A\}, \{B\}, \{C\}, \{D\}, \{E\}.
        \item Merge closest points (e.g., \{A\} and \{B\}), resulting in: \{\{A, B\}, \{C\}, \{D\}, \{E\}\}.
        \item Update distances between the new cluster and remaining points and repeat until all points are merged into one cluster.
        \item The final dendrogram visualizes the structure of the clusters.
    \end{enumerate}
    
    Through understanding these steps, you can effectively apply and analyze hierarchical clustering methods in data analysis. 
    For practical implementation, Python libraries (like Scikit-learn) can significantly streamline the clustering process.
\end{frame}

\begin{frame}[fragile]
    \frametitle{Strengths of Hierarchical Clustering}
    
    \begin{itemize}
        \item \textbf{No Need for Predefined Clusters}:
        \begin{itemize}
            \item Constructs a hierarchy of clusters.
            \item The number of clusters can be determined visually via a dendrogram.
        \end{itemize}
        
        \item \textbf{Intuitive and Informative}:
        \begin{itemize}
            \item Dendrograms visualize cluster formation and relationships.
            \item Aids in understanding the data structure.
        \end{itemize}
        
        \item \textbf{Works with Different Distances}:
        \begin{itemize}
            \item Can utilize various distance metrics (e.g., Euclidean, Manhattan).
            \item Provides flexibility based on dataset characteristics.
        \end{itemize}
        
        \item \textbf{Suitable for Small Datasets}:
        \begin{itemize}
            \item Particularly effective for small to medium-sized datasets.
            \item Less computational load, yielding good results with few data points.
        \end{itemize}
    \end{itemize}
\end{frame}

\begin{frame}[fragile]
    \frametitle{Limitations of Hierarchical Clustering}
    
    \begin{itemize}
        \item \textbf{Scalability}:
        \begin{itemize}
            \item O(n$^2$) time complexity makes it inefficient for large datasets.
            \item K-Means has a linear time complexity, making it more suitable for larger datasets.
        \end{itemize}
        
        \item \textbf{Sensitivity to Noise and Outliers}:
        \begin{itemize}
            \item Heavily influenced by noise, can distort clustering results.
            \item K-Means can be more robust to outliers.
        \end{itemize}
        
        \item \textbf{Inability to Reassess Clusters}:
        \begin{itemize}
            \item Once clusters are formed, decisions cannot be revisited.
            \item K-Means allows for reassignment of points, enabling better optimization.
        \end{itemize}
        
        \item \textbf{Dendrogram Interpretation Complexity}:
        \begin{itemize}
            \item Interpretation can be complex with large datasets.
            \item Harder to determine optimal number of clusters.
        \end{itemize}
    \end{itemize}
\end{frame}

\begin{frame}[fragile]
    \frametitle{Key Points and Conclusion}
    
    \begin{block}{Key Points to Emphasize}
        \begin{itemize}
            \item Flexibility in cluster formation comes with limitations regarding efficiency and robustness.
            \item Dendrograms are useful for exploratory analysis but have performance limitations with larger data sets.
        \end{itemize}
    \end{block}
    
    \begin{block}{Conclusion}
        When selecting between hierarchical clustering and K-Means, consider:
        \begin{itemize}
            \item Dataset characteristics and size.
            \item Available computational resources.
            \item Trade-offs to enhance clustering strategies and decision-making.
        \end{itemize}
    \end{block}
\end{frame}

\begin{frame}[fragile]
    \frametitle{Practical Considerations in Clustering}
    \begin{block}{Understanding Clustering Challenges}
        Clustering is a powerful unsupervised learning technique for grouping similar data points. However, challenges exist that need to be considered to improve the effectiveness of clustering algorithms.
    \end{block}
\end{frame}

\begin{frame}[fragile]
    \frametitle{Curse of Dimensionality}
    \begin{block}{Definition}
        The "curse of dimensionality" describes how adding more dimensions (features) to a dataset renders the distance between data points less meaningful.
    \end{block}
    \begin{itemize}
        \item Clusters can become sparse in high-dimensional spaces.
        \item Distance metrics (e.g., Euclidean distance) may lose effectiveness, leading to misleading results.
    \end{itemize}
    \begin{block}{Example}
        In 2 dimensions, points may be easily separated:
        \begin{center}
            \includegraphics[width=0.3\textwidth]{2D_plot.png} % Placeholder for 2D plot
        \end{center}
        In a 100-dimensional space, points become almost equidistant.
    \end{block}
\end{frame}

\begin{frame}[fragile]
    \frametitle{Impact of Feature Scaling}
    \begin{block}{Definition}
        Feature scaling is the normalization or standardization of data ranges before clustering. It significantly influences clustering results.
    \end{block}
    \begin{itemize}
        \item Without scaling, larger range features may dominate distance calculations (e.g., meters vs. centimeters).
    \end{itemize}
    \begin{block}{Methods of Scaling}
        \begin{itemize}
            \item \textbf{Standardization (Z-score normalization)}:
            \begin{equation}
                z = \frac{(x - \mu)}{\sigma}
            \end{equation}
            where $\mu$ is mean and $\sigma$ is standard deviation.
            \item \textbf{Normalization (Min-Max scaling)}:
            \begin{equation}
                x' = \frac{(x - x_{min})}{(x_{max} - x_{min})}
            \end{equation}
            Rescales feature to [0, 1].
        \end{itemize}
    \end{block}
    \begin{block}{Example}
        Clustering customer data based on age and income:
        \begin{itemize}
            \item \textbf{Before Scaling}: Age: [20, 30, 40], Income: [20,000, 150,000, 300,000]
            \item \textbf{After Scaling (Min-Max)}: Age: [0, 0.5, 1], Income: [0, 0.67, 1]
        \end{itemize}
    \end{block}
\end{frame}

\begin{frame}[fragile]
    \frametitle{Key Points and Conclusion}
    \begin{itemize}
        \item The "curse of dimensionality" shows that more features do not always guarantee better clustering.
        \item Preprocessing steps like feature scaling are essential for equitable feature contributions.
        \item Consider dimensionality reduction techniques (e.g., PCA) for improved performance.
    \end{itemize}
    \begin{block}{Conclusion}
        Addressing practical challenges in clustering is essential for achieving meaningful and interpretable results. Proper data preparation increases the chances of uncovering insightful patterns.
    \end{block}
\end{frame}

\begin{frame}[fragile]
    \frametitle{Applications of Clustering - Overview}
    Clustering is a powerful unsupervised learning technique used across various fields to group similar data points together based on their features without pre-existing labels. Here, we explore notable applications of clustering in several domains:
\end{frame}

\begin{frame}[fragile]
    \frametitle{Applications of Clustering - Marketing}
    \begin{itemize}
        \item \textbf{Customer Segmentation:} Businesses use clustering to identify distinct customer segments based on purchasing behavior, preferences, and demographics.
        \begin{itemize}
            \item \textbf{Example:} A retail company might cluster customers into groups like "frequent buyers," "seasonal shoppers," and "occasional browsers," allowing targeted marketing strategies for each group.
        \end{itemize}
        
        \item \textbf{Recommendation Systems:} Clustering aids in designing recommendation engines by grouping similar products or users.
        \begin{itemize}
            \item \textbf{Example:} E-commerce platforms use clustering to suggest products that similar users have bought, enhancing personalized shopping experiences.
        \end{itemize}
    \end{itemize}
\end{frame}

\begin{frame}[fragile]
    \frametitle{Applications of Clustering - Biology and Social Sciences}
    \begin{block}{1. Biology}
        \begin{itemize}
            \item \textbf{Genomic Clustering:} In bioinformatics, clustering is used to categorize genes or proteins with similar expression patterns.
            \begin{itemize}
                \item \textbf{Example:} Clustering genes based on their expression levels can help identify genes involved in similar biological processes or disease states.
            \end{itemize}
            
            \item \textbf{Ecology:} Clustering helps in understanding species distributions by grouping geographic areas with similar ecological characteristics.
            \begin{itemize}
                \item \textbf{Example:} Clusters can identify regions with similar habitat types and species richness, aiding in conservation efforts.
            \end{itemize}
        \end{itemize}
    \end{block}

    \begin{block}{2. Social Sciences}
        \begin{itemize}
            \item \textbf{Survey Analysis:} Clustering techniques help researchers identify patterns among survey responses to group similar opinions.
            \begin{itemize}
                \item \textbf{Example:} A lifestyle survey may reveal clusters of respondents with similar attitudes towards health and fitness.
            \end{itemize}
            
            \item \textbf{Social Network Analysis:} Clustering algorithms analyze social networks to discover communities within a larger network.
            \begin{itemize}
                \item \textbf{Example:} Identification of tightly connected friend groups on social media platforms can inform marketing strategies.
            \end{itemize}
        \end{itemize}
    \end{block}
\end{frame}

\begin{frame}[fragile]
    \frametitle{Conclusion}
    \begin{itemize}
        \item Clustering is \textbf{versatile} and applicable in diverse fields.
        \item It enables \textbf{data-driven decision-making} by providing insights about natural groupings in data.
        \item Understanding clusters helps organizations tailor services and strategies for specific audience segments, increasing efficiency and satisfaction.
    \end{itemize}
    
    Clustering plays a crucial role in transforming data into actionable insights across various domains—from targeted marketing strategies to ecological conservation. Its ability to unveil hidden patterns is invaluable in today's data-driven world.
\end{frame}

\begin{frame}[fragile]
    \frametitle{Conclusion and Summary - Part 1}

    \begin{block}{Understanding Unsupervised Learning and Clustering}
        \begin{itemize}
            \item \textbf{Unsupervised Learning Overview:}
            \begin{itemize}
                \item A type of machine learning where models are trained on data without labeled outcomes.
                \item Seeks to identify patterns and structures within the data.
            \end{itemize}
            \item \textbf{Key Techniques:} 
            \begin{itemize}
                \item Clustering and Association are the primary methods in unsupervised learning.
            \end{itemize}
        \end{itemize}
    \end{block}
\end{frame}

\begin{frame}[fragile]
    \frametitle{Conclusion and Summary - Part 2}

    \begin{block}{What is Clustering?}
        \begin{itemize}
            \item A process of grouping objects such that objects in the same group (cluster) are more similar to each other than those in different groups.
            \item \textbf{Common Algorithms:}
            \begin{itemize}
                \item \textbf{K-Means:} Partitions data into K clusters based on feature similarity.
                \item \textbf{Hierarchical Clustering:} Builds a tree of clusters based on data points' distances.
                \item \textbf{DBSCAN:} Density-based clustering that groups together closely packed points.
            \end{itemize}
        \end{itemize}
    \end{block}
\end{frame}

\begin{frame}[fragile]
    \frametitle{Conclusion and Summary - Part 3}

    \begin{block}{Key Takeaways}
        \begin{enumerate}
            \item \textbf{Pattern Recognition:} Clustering aids in identifying natural groupings, facilitating exploratory data analysis.
            \item \textbf{Real-World Applications:} Used in Marketing, Biology, and Social Sciences for segmentation and classification.
            \item \textbf{Dimensionality Reduction:} Simplifies data complexity for analysis and visualization.
            \item \textbf{Evaluation Methods:} 
            \begin{itemize}
                \item Metrics include Silhouette Score, Dunn Index, and Within-Cluster Sum of Squares.
            \end{itemize}
        \end{enumerate}
    \end{block}

    \begin{block}{Importance of Clustering Techniques}
        \begin{itemize}
            \item \textbf{Data Simplification:} Reduces noise and highlights important signals by grouping similar data points.
            \item \textbf{Insight Generation:} Reveals hidden relationships and trends for informed decision-making.
            \item \textbf{Foundation for Other Methods:} Serves as a precursor for supervised learning tasks.
        \end{itemize}
    \end{block}
\end{frame}

\begin{frame}[fragile]
  \frametitle{Discussion Questions on Clustering in Unsupervised Learning - Introduction}
  Clustering is a powerful technique in unsupervised learning where data points are grouped based on similarity. 
  Understanding clustering not only enhances your knowledge of machine learning but also encourages critical thinking about data patterns and applications. 
  Here, we explore several discussion questions to stimulate conversation and deepen comprehension of the concepts covered in this chapter.
\end{frame}

\begin{frame}[fragile]
  \frametitle{Discussion Questions - Key Concepts}
  \begin{enumerate}
    \item \textbf{What is Clustering?}
        \begin{itemize}
            \item How would you define clustering in your own words?
            \item Why is clustering considered an unsupervised learning technique?
        \end{itemize}

    \item \textbf{Common Clustering Algorithms:}
        \begin{itemize}
            \item Compare and contrast any two clustering algorithms discussed in the chapter, such as K-Means and Hierarchical clustering.
            \item What are the strengths and weaknesses of each algorithm?
        \end{itemize}
  \end{enumerate}
\end{frame}

\begin{frame}[fragile]
  \frametitle{Discussion Questions - More Concepts}
  \begin{enumerate}
    \setcounter{enumi}{2}
    \item \textbf{Distance Metrics:}
        \begin{itemize}
            \item Discuss how different distance metrics (Euclidean, Manhattan, Cosine) affect the results of clustering.
            \item Which metric do you think is the most suitable for text data, and why?
        \end{itemize}
        
    \item \textbf{Real-World Applications:}
        \begin{itemize}
            \item In what real-world scenarios can clustering be utilized? Think of examples outside the typical applications like customer segmentation.
            \item How could organizations benefit from insights gained through clustering?
        \end{itemize}

    \item \textbf{Challenges and Limitations:}
        \begin{itemize}
            \item What challenges do you foresee when applying clustering algorithms to large datasets?
            \item How do factors like outliers and noise impact clustering outcomes?
        \end{itemize}

    \item \textbf{Evaluation of Clustering:}
        \begin{itemize}
            \item What methods can be employed to evaluate the effectiveness of clustering results? Discuss metrics such as Silhouette Score or Davies-Bouldin Index.
            \item How would you add an interpretive layer to the clustering output?
        \end{itemize}
  \end{enumerate}
\end{frame}

\begin{frame}[fragile]
  \frametitle{Conclusion}
  Engaging with these questions will help solidify your understanding of clustering in unsupervised learning. 
  Consider the implications of your answers for both theoretical knowledge and practical applications in data science.
\end{frame}


\end{document}