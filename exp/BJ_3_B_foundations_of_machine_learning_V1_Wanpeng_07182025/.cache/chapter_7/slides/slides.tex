\documentclass[aspectratio=169]{beamer}

% Theme and Color Setup
\usetheme{Madrid}
\usecolortheme{whale}
\useinnertheme{rectangles}
\useoutertheme{miniframes}

% Additional Packages
\usepackage[utf8]{inputenc}
\usepackage[T1]{fontenc}
\usepackage{graphicx}
\usepackage{booktabs}
\usepackage{listings}
\usepackage{amsmath}
\usepackage{amssymb}
\usepackage{xcolor}
\usepackage{tikz}
\usepackage{pgfplots}
\pgfplotsset{compat=1.18}
\usetikzlibrary{positioning}
\usepackage{hyperref}

% Custom Colors
\definecolor{myblue}{RGB}{31, 73, 125}
\definecolor{mygray}{RGB}{100, 100, 100}
\definecolor{mygreen}{RGB}{0, 128, 0}
\definecolor{myorange}{RGB}{230, 126, 34}
\definecolor{mycodebackground}{RGB}{245, 245, 245}

% Set Theme Colors
\setbeamercolor{structure}{fg=myblue}
\setbeamercolor{frametitle}{fg=white, bg=myblue}
\setbeamercolor{title}{fg=myblue}
\setbeamercolor{section in toc}{fg=myblue}
\setbeamercolor{item projected}{fg=white, bg=myblue}
\setbeamercolor{block title}{bg=myblue!20, fg=myblue}
\setbeamercolor{block body}{bg=myblue!10}
\setbeamercolor{alerted text}{fg=myorange}

% Set Fonts
\setbeamerfont{title}{size=\Large, series=\bfseries}
\setbeamerfont{frametitle}{size=\large, series=\bfseries}
\setbeamerfont{caption}{size=\small}
\setbeamerfont{footnote}{size=\tiny}

% Footer and Navigation Setup
\setbeamertemplate{footline}{
  \leavevmode%
  \hbox{%
    \begin{beamercolorbox}[wd=.3\paperwidth,ht=2.25ex,dp=1ex,center]{author in head/foot}%
      \usebeamerfont{author in head/foot}\insertshortauthor
    \end{beamercolorbox}%
    \begin{beamercolorbox}[wd=.5\paperwidth,ht=2.25ex,dp=1ex,center]{title in head/foot}%
      \usebeamerfont{title in head/foot}\insertshorttitle
    \end{beamercolorbox}%
    \begin{beamercolorbox}[wd=.2\paperwidth,ht=2.25ex,dp=1ex,center]{date in head/foot}%
      \usebeamerfont{date in head/foot}
      \insertframenumber{} / \inserttotalframenumber
    \end{beamercolorbox}}%
  \vskip0pt%
}

% Turn off navigation symbols
\setbeamertemplate{navigation symbols}{}

% Title Page Information
\title[Feature Engineering and Selection]{Week 7: Feature Engineering and Selection}
\author[J. Smith]{John Smith, Ph.D.}
\institute[University Name]{
  Department of Computer Science\\
  University Name\\
  \vspace{0.3cm}
  Email: email@university.edu\\
  Website: www.university.edu
}
\date{\today}

% Document Start
\begin{document}

\frame{\titlepage}

\begin{frame}[fragile]
    \frametitle{Introduction to Feature Engineering - Overview}
    \begin{block}{What are Features?}
        In the context of machine learning, \textbf{features} are individual measurable properties or characteristics of the data used by the model. They are the input variables that the model learns from to make predictions.
    \end{block}
    
    \begin{block}{Importance of Features}
        \begin{enumerate}
            \item \textbf{Model Performance:} The features chosen directly affect the model’s ability to learn and generalize from the training data.
            \item \textbf{Data Representation:} Features encapsulate underlying patterns in the data.
            \item \textbf{Dimensionality Reduction:} Selecting relevant features reduces model complexity, leading to faster training times.
        \end{enumerate}
    \end{block}
\end{frame}

\begin{frame}[fragile]
    \frametitle{Introduction to Feature Engineering - Examples}
    \begin{block}{Types of Features}
        \begin{itemize}
            \item \textbf{Numerical Features:} Continuous variables (e.g., height, weight).
            \item \textbf{Categorical Features:} Qualitative variables (e.g., color, type of vehicle).
        \end{itemize}
    \end{block}
    
    \begin{block}{Key Points to Emphasize}
        \begin{itemize}
            \item Feature quality significantly improves model predictions.
            \item Feature engineering can involve:
            \begin{itemize}
                \item Polynomial features (e.g., \(x^2\)).
                \item Interaction terms (e.g., product of two numerical variables).
            \end{itemize}
        \end{itemize}
    \end{block}
\end{frame}

\begin{frame}[fragile]
    \frametitle{Introduction to Feature Engineering - Simplified Example}
    \begin{block}{Example Scenario}
        Imagine building a model to predict whether a customer will buy a bicycle based on:
        \begin{itemize}
            \item Age (numerical feature)
            \item Has Children (categorical feature)
        \end{itemize}
    \end{block}
    
    \begin{block}{Feature Engineering}
        \begin{itemize}
            \item \textbf{Original Features:} Age, Has Children.
            \item \textbf{Engineered Feature:} Age Bracket (e.g., Child under 18, Adult, Senior).
        \end{itemize}
    \end{block}
    
    \begin{block}{Conclusion}
        Feature engineering is a critical step in the machine learning pipeline that can significantly enhance predictive capabilities.
    \end{block}
\end{frame}

\begin{frame}[fragile]
    \frametitle{Understanding Features - Definition}
    \begin{block}{Features}
        In machine learning, features are individual measurable properties or characteristics used as input to the model. They represent the information that the model uses to make predictions.
    \end{block}
    
    \begin{itemize}
        \item \textbf{Example:} In predicting house prices, features can include:
        \begin{itemize}
            \item Size of the house (in square feet)
            \item Number of bedrooms
            \item Location (zip code)
            \item Year built
        \end{itemize}
    \end{itemize}
\end{frame}

\begin{frame}[fragile]
    \frametitle{Understanding Features - Importance}
    \begin{block}{Role in Machine Learning}
        Features play a pivotal role in determining the model's capability to learn from data. High-quality features enable the model to discern patterns and make accurate predictions.
    \end{block}

    \begin{itemize}
        \item Feature Set: The collection of all features used in the model is termed as the feature set.
    \end{itemize}

    \begin{block}{Impact on Model Performance}
        \begin{itemize}
            \item Good Features: Well-chosen features significantly improve model accuracy.
            \item Poor Features: Irrelevant or noisy features can lead to overfitting and poor generalization.
        \end{itemize}
    \end{block}
\end{frame}

\begin{frame}[fragile]
    \frametitle{Understanding Features - Key Points}
    \begin{itemize}
        \item \textbf{Feature Quality:} Prioritize selected features based on their relationship with the target variable.
        \item \textbf{Feature Engineering:} Transforming raw data into meaningful features is essential for model success.
        \item \textbf{Feature Selection:} Identifying relevant features for training enhances performance and prevents overfitting.
    \end{itemize}

    \begin{block}{Example of Feature Impact}
        \begin{itemize}
            \item \textbf{Model A:} Uses one feature (e.g., age of customers).
            \item \textbf{Model B:} Uses multiple features (e.g., age, income, prior purchase behavior).
            \item \textbf{Performance Insight:} Model B is likely to outperform Model A due to a richer feature set.
        \end{itemize}
    \end{block}
\end{frame}

\begin{frame}[fragile]
    \frametitle{Understanding Features - Conclusion}
    \begin{block}{Conclusion}
        Features are the backbone of machine learning. Their quality and appropriateness are directly correlated with the effectiveness of the model.
    \end{block}
    \begin{itemize}
        \item Crafting high-quality features is essential for building robust, predictive models.
        \item Understanding and utilizing features effectively positions your machine learning projects for success! 
    \end{itemize}
\end{frame}

\begin{frame}[fragile]{Feature Types - Introduction}
    In machine learning, features are individual measurable properties or characteristics used as input for models. Understanding the different types of features is crucial for effective feature engineering and selection, as it influences how models interpret the data.
\end{frame}

\begin{frame}[fragile]{Feature Types - Numerical and Categorical}
    \begin{block}{1. Numerical Features}
        \begin{itemize}
            \item \textbf{Definition}: Quantitative data measured on a numerical scale; can be discrete or continuous.
            \item \textbf{Examples}:
                \begin{itemize}
                    \item Age: 25 years
                    \item Salary: \$60,000
                \end{itemize}
            \item \textbf{Usage}: Used in regression analysis, subject to mathematical operations (e.g., mean, standard deviation).
        \end{itemize}
    \end{block}
    
    \begin{block}{2. Categorical Features}
        \begin{itemize}
            \item \textbf{Definition}: Qualitative data representing categories or groups, non-numeric, can be nominal or ordinal.
            \item \textbf{Examples}:
                \begin{itemize}
                    \item Nominal: Gender (Male, Female)
                    \item Ordinal: Education Level (High School, Bachelor's, Master's)
                \end{itemize}
            \item \textbf{Usage}: Typically encoded using one-hot encoding or label encoding for machine learning algorithms.
        \end{itemize}
    \end{block}
\end{frame}

\begin{frame}[fragile]{Feature Types - Textual and Image}
    \begin{block}{3. Textual Features}
        \begin{itemize}
            \item \textbf{Definition}: Unstructured data in text form requiring special processing for effective modeling.
            \item \textbf{Examples}:
                \begin{itemize}
                    \item Reviews: "Great product!"
                    \item Sentences: "Machine learning is interesting."
                \end{itemize}
            \item \textbf{Usage}: Techniques like Bag of Words, TF-IDF, or word embeddings (e.g., Word2Vec) are often employed.
        \end{itemize}
    \end{block}
    
    \begin{block}{4. Image Features}
        \begin{itemize}
            \item \textbf{Definition}: Data represented in pixel format, commonly used in computer vision.
            \item \textbf{Examples}:
                \begin{itemize}
                    \item A photo of a cat consisting of thousands of pixels.
                \end{itemize}
            \item \textbf{Usage}: Techniques like Convolutional Neural Networks (CNNs) are used to automatically extract features from images.
        \end{itemize}
    \end{block}
\end{frame}

\begin{frame}[fragile]{Conclusion and Key Points}
    \begin{itemize}
        \item The choice of feature type affects model performance and interpretability.
        \item Understanding the nature of features facilitates proper preprocessing steps such as normalization, encoding, and extraction.
        \item Different modeling techniques leverage different types of features; for instance, linear regression works primarily with numerical features, while deep learning can handle textual and image data directly.
    \end{itemize}
\end{frame}

\begin{frame}[fragile]
    \frametitle{Feature Extraction Techniques - Introduction}
    \begin{block}{Overview}
        Feature extraction is a crucial step in the data preprocessing phase of machine learning. It transforms raw data into a format suitable for modeling, capturing the underlying structure and semantics for improved algorithm efficiency.
    \end{block}
\end{frame}

\begin{frame}[fragile]
    \frametitle{Feature Extraction Techniques - PCA}
    \begin{block}{Definition}
        PCA (Principal Component Analysis) is a statistical method used to reduce data dimensionality while preserving variance.
    \end{block}

    \begin{itemize}
        \item \textbf{Steps:}
        \begin{itemize}
            \item Standardization: Center data by subtracting the mean.
            \item Covariance Matrix: Identify feature relationships.
            \item Eigenvalues & Eigenvectors: Determine principal components.
            \item Projection: Map data to a reduced-dimensional space.
        \end{itemize}
        \item \textbf{Key Benefits:}
        \begin{itemize}
            \item Minimizes noise.
            \item Improves model performance.
            \item Works best with numerical data.
        \end{itemize}
    \end{itemize}
\end{frame}

\begin{frame}[fragile]
    \frametitle{Feature Extraction Techniques - LDA and TF-IDF}
    \begin{block}{Linear Discriminant Analysis (LDA)}
        \begin{itemize}
            \item \textbf{Definition:} A supervised technique to find a combination of features for class separation.
            \item \textbf{Process:}
            \begin{itemize}
                \item Compute class means.
                \item Calculate within-class and between-class scatter matrices.
                \item Maximize the ratio of between-class variance to within-class variance.
            \end{itemize}
            \item \textbf{Key Insights:}
            \begin{itemize}
                \item Effective for classification tasks.
                \item Sensitive to class separability.
            \end{itemize}
        \end{itemize}
    \end{block}

    \begin{block}{Term Frequency-Inverse Document Frequency (TF-IDF)}
        \begin{itemize}
            \item \textbf{Definition:} A statistic reflecting the importance of a word in a document relative to a corpus.
            \item \textbf{Components:}
            \begin{itemize}
                \item \text{TF}(t,d) = \frac{\text{Number of times term } t \text{ appears in document } d}{\text{Total number of terms in document } d}
                \item \text{IDF}(t) = \log\left(\frac{\text{Total number of documents}}{\text{Number of documents containing term } t}\right)
                \item \text{TF-IDF}(t,d) = \text{TF}(t,d) \times \text{IDF}(t)
            \end{itemize}
            \item \textbf{Key Use:}
            \begin{itemize}
                \item Highlights unique words in documents.
                \item Useful for text classification and clustering.
            \end{itemize}
        \end{itemize}
    \end{block}
\end{frame}

\begin{frame}[fragile]
    \frametitle{Feature Selection}
    \textbf{Definition:} Explain the process of selecting a subset of relevant features.
\end{frame}

\begin{frame}[fragile]
    \frametitle{Definition of Feature Selection}
    Feature selection is the process of identifying and selecting a subset of relevant features (or predictors) from the original set of features used in a dataset. 
    \begin{itemize}
        \item A critical step in data preprocessing.
        \item Aims to improve the efficiency of algorithms by focusing on the most informative attributes.
    \end{itemize}
\end{frame}

\begin{frame}[fragile]
    \frametitle{Key Concepts in Feature Selection}
    \begin{block}{Relevance}
        Features that provide useful information regarding the target variable.
    \end{block}
    
    \begin{block}{Redundancy}
        Features that provide little additional value for prediction when other features are already included.
    \end{block}
\end{frame}

\begin{frame}[fragile]
    \frametitle{Why is Feature Selection Important?}
    \begin{enumerate}
        \item \textbf{Reduces Overfitting}
        \begin{itemize}
            \item Overfitting occurs when a model learns noise instead of underlying patterns.
            \item Keeping only relevant features improves the model's ability to generalize.
            \item \textbf{Example:} Including irrelevant features like the color of the front door can mislead models in predicting house prices.
        \end{itemize}
        
        \item \textbf{Improves Model Performance}
        \begin{itemize}
            \item Simplifying the model leads to faster computations.
            \item Enhances predictive accuracy by focusing on strongest signals.
            \item \textbf{Illustration:} Weight and engine size are relevant to fuel efficiency, while the car's radio type is not.
        \end{itemize}

        \item \textbf{Enhances Interpretability}
        \begin{itemize}
            \item Easier to understand models facilitate stakeholder decision-making.
            \item \textbf{Key Point:} Simpler models reduce inference errors.
        \end{itemize}
    \end{enumerate}
\end{frame}

\begin{frame}[fragile]
    \frametitle{Conclusion}
    Feature selection is an essential approach in the machine learning workflow. 
    \begin{itemize}
        \item Reduces overfitting
        \item Enhances model performance
        \item Improves interpretability
    \end{itemize}
    This process sets the foundation for subsequent modeling techniques that will be discussed in the next slide.
\end{frame}

\begin{frame}[fragile]
    \frametitle{Next Steps}
    Look forward to exploring various popular feature selection techniques in the next slide, such as:
    \begin{itemize}
        \item Filter Methods
        \item Wrapper Methods
        \item Embedded Methods
    \end{itemize}
    These techniques provide structured ways to carry out the selection process effectively.
\end{frame}

\begin{frame}[fragile]{Feature Selection Techniques - Overview}
    \begin{block}{Key Concepts in Feature Selection}
        Feature selection is a critical step in the machine learning process. It involves choosing a subset of relevant features from the dataset to enhance model accuracy and efficiency. This process can also minimize overfitting, reduce training time, and improve the interpretability of the models.
    \end{block}
\end{frame}

\begin{frame}[fragile]{Popular Feature Selection Methods}
    We can categorize feature selection techniques into three main types:
    \begin{itemize}
        \item Filter Methods
        \item Wrapper Methods
        \item Embedded Methods
    \end{itemize}
\end{frame}

\begin{frame}[fragile]{Filter Methods}
    \begin{block}{Overview}
        Filter methods assess the relevance of features based on their statistical properties and correlation with the target variable, independent of any machine learning model.
    \end{block}
    
    \begin{itemize}
        \item \textbf{Advantages}: Quick to compute and can handle large datasets efficiently.
        \item \textbf{Common Techniques}:
        \begin{itemize}
            \item Correlation Coefficients
            \item Chi-Square Test
            \item ANOVA (Analysis of Variance)
        \end{itemize}
    \end{itemize}
    \begin{block}{Example}
        For a dataset predicting house prices, "square footage" may have a strong correlation with price, suggesting inclusion, while features with low correlation, such as "color of the house," can be excluded.
    \end{block}
\end{frame}

\begin{frame}[fragile]{Wrapper Methods}
    \begin{block}{Overview}
        Wrapper methods evaluate model performance using a certain subset of features, selecting the best combination based on performance metrics.
    \end{block}
    
    \begin{itemize}
        \item \textbf{Advantages}: Usually provide better accuracy as they are model-specific.
        \item \textbf{Common Techniques}:
        \begin{itemize}
            \item Forward Selection
            \item Backward Elimination
            \item Recursive Feature Elimination (RFE)
        \end{itemize}
    \end{itemize}
    \begin{block}{Example}
        Using a decision tree, you begin with all features, assess accuracy, and iteratively remove less important features to achieve optimal model accuracy.
    \end{block}
\end{frame}

\begin{frame}[fragile]{Embedded Methods}
    \begin{block}{Overview}
        Embedded methods incorporate feature selection into the model training phase, selecting features based on their importance as determined by the model itself.
    \end{block}
    
    \begin{itemize}
        \item \textbf{Advantages}: Combine benefits of filter and wrapper methods, improving computational efficiency and model performance.
        \item \textbf{Common Techniques}:
        \begin{itemize}
            \item Lasso (L1 Regularization)
            \item Decision Trees and Ensemble Methods
        \end{itemize}
    \end{itemize}
    \begin{block}{Example}
        In Lasso regression, features with coefficients shrunk to zero are excluded from the final model, simplifying the model while maintaining performance.
    \end{block}
\end{frame}

\begin{frame}[fragile]{Key Points to Emphasize}
    \begin{itemize}
        \item Filter methods are efficient but may miss interactions.
        \item Wrapper methods, while thorough, are computationally expensive.
        \item Embedded methods strike a balance between efficiency and effectiveness.
        \item Always validate the selected features' impact on model performance using techniques like cross-validation.
    \end{itemize}
\end{frame}

\begin{frame}
    \frametitle{Practical Examples of Feature Engineering}
    \begin{block}{Introduction}
        Feature engineering is the process of using domain knowledge to create features (variables) that make machine learning algorithms work better.
    \end{block}
\end{frame}

\begin{frame}
    \frametitle{Importance of Feature Engineering}
    \begin{itemize}
        \item Enhances model performance: Well-engineered features can significantly boost accuracy.
        \item Reduces overfitting: Helps in improving the generalization of the model.
        \item Facilitates interpretability: Better features lead to easier understanding of model predictions.
    \end{itemize}
\end{frame}

\begin{frame}
    \frametitle{E-Commerce Example: Clickthrough Rate Prediction}
    \begin{block}{Task}
        Predict whether users will click on ads.
    \end{block}
    \begin{block}{Feature Engineering}
        \begin{itemize}
            \item User behavior features: Time spent on site, previous click history, segmenting users.
            \item Ad features: Placement on webpage, visual appeal, contextual relevance.
        \end{itemize}
    \end{block}
\end{frame}

\begin{frame}
    \frametitle{Healthcare Example: Disease Outcome Prediction}
    \begin{block}{Task}
        Predict patient outcomes based on clinical data.
    \end{block}
    \begin{block}{Feature Engineering}
        \begin{itemize}
            \item Interaction features: Combining age and cholesterol levels.
            \item Temporal features: Features reflecting change over time.
        \end{itemize}
    \end{block}
\end{frame}

\begin{frame}
    \frametitle{Finance Example: Credit Default Prediction}
    \begin{block}{Task}
        Assess risk of borrowers defaulting on loans.
    \end{block}
    \begin{block}{Feature Engineering}
        \begin{itemize}
            \item Lag features: Incorporating past payment history.
            \item Categorical encoding: Transforming categorical variables into numerical values.
        \end{itemize}
    \end{block}
\end{frame}

\begin{frame}
    \frametitle{NLP Example: Sentiment Analysis}
    \begin{block}{Task}
        Identify sentiment in text data (positive, negative, neutral).
    \end{block}
    \begin{block}{Feature Engineering}
        \begin{itemize}
            \item Text features: Techniques like TF-IDF to convert text documents.
            \item N-grams: Features based on word sequences for contextual understanding.
        \end{itemize}
    \end{block}
\end{frame}

\begin{frame}
    \frametitle{Key Points}
    \begin{itemize}
        \item Feature engineering is context-dependent: Tailored approaches needed for different domains.
        \item Quality over quantity: High-quality features are more effective than numerous mediocre ones.
        \item Iterative process: Experimentation and validation are essential for refining features.
    \end{itemize}
\end{frame}

\begin{frame}[fragile]
    \frametitle{Code Snippet for Feature Engineering}
    \begin{lstlisting}[language=Python]
import pandas as pd

# Sample DataFrame
data = pd.DataFrame({'age': [25, 32, 47, 51],
                     'cholesterol_level': [240, 230, 220, 190],
                     'default': [0, 1, 1, 0]})

# Create interaction feature
data['age_cholesterol_interaction'] = data['age'] * data['cholesterol_level']

# Display the updated DataFrame
print(data)
    \end{lstlisting}
\end{frame}

\begin{frame}
    \frametitle{Conclusion}
    Effective feature engineering can dramatically enhance the efficacy of machine learning models across various applications. Understanding the underlying data and creatively manipulating features can lead to superior predictive capabilities.
\end{frame}

\begin{frame}[fragile]
    \frametitle{Case Study: Feature Engineering Impact}
    \begin{block}{Introduction to Feature Engineering}
        Feature engineering is the process of using domain knowledge to extract features from raw data that help improve the performance of machine learning models.
    \end{block}
\end{frame}

\begin{frame}[fragile]
    \frametitle{Case Study Overview: Predicting House Prices}
    \begin{itemize}
        \item \textbf{Background:} A real estate company aimed to enhance their predictive model for house prices based on attributes such as:
        \begin{itemize}
            \item Number of bedrooms
            \item Square footage
            \item Location
        \end{itemize}
        \item Initial models using raw features achieved a prediction accuracy of only 60\%.
    \end{itemize}
\end{frame}

\begin{frame}[fragile]
    \frametitle{Key Feature Engineering Steps}
    \begin{enumerate}
        \item \textbf{Combining Features:}
            \begin{itemize}
                \item Example: Combine 'number of bedrooms' and 'number of bathrooms' into 'total bathrooms'.
                \item Impact: Captured meaningful information about home functionality.
            \end{itemize}
        \item \textbf{Creating Interaction Features:}
            \begin{itemize}
                \item Example: Interaction between 'square footage' and 'location quality' creates 'quality per square foot'.
                \item Impact: Highlights pricing inefficiencies in large homes in less desirable locations.
            \end{itemize}
        \item \textbf{Log Transformations:}
            \begin{itemize}
                \item Example: Apply log transformation to 'price'.
                \item Impact: Improved linear relationships between features and the target variable.
            \end{itemize}
        \item \textbf{Encoding Categorical Variables:}
            \begin{itemize}
                \item Example: One-hot encoding for 'neighborhood'.
                \item Impact: Prevented misinterpretation of numerical relationships.
            \end{itemize}
        \item \textbf{Handling Missing Values:}
            \begin{itemize}
                \item Example: Imputed missing 'age of the house' using the mean of similar properties.
                \item Impact: Maintained data integrity and model robustness.
            \end{itemize}
    \end{enumerate}
\end{frame}

\begin{frame}[fragile]
    \frametitle{Model Performance Improvement}
    \begin{itemize}
        \item \textbf{Before Feature Engineering:}
            \begin{itemize}
                \item Accuracy: 60\%
                \item Model Type: Linear Regression 
            \end{itemize}
        \item \textbf{After Feature Engineering:}
            \begin{itemize}
                \item Accuracy: 85\%
                \item Model Type: Gradient Boosting 
            \end{itemize}
    \end{itemize}
    \begin{block}{Key Point}
        Tailored features facilitated the model's ability to learn complex patterns more effectively, showcasing the impact of feature engineering.
    \end{block}
\end{frame}

\begin{frame}[fragile]
    \frametitle{Conclusion and Key Takeaways}
    \begin{itemize}
        \item Thoughtful feature engineering dramatically enhances model performance by transforming raw data into valuable insights for prediction.
        \item Domain expertise is crucial in identifying and creating meaningful features.
        \item Continuous evaluation and iteration of features can lead to substantial performance improvements.
    \end{itemize}
\end{frame}

\begin{frame}[fragile]
    \frametitle{Further Exploration}
    \begin{block}{Reflection}
        As you progress, consider how feature engineering can apply to your own datasets.
        \begin{itemize}
            \item What new features can you create?
            \item How can interactions or data transformations improve your models?
        \end{itemize}
    \end{block}
\end{frame}

\begin{frame}[fragile]
    \frametitle{(Optional) Code Snippet for Python using Pandas}
    \begin{lstlisting}[language=Python]
import pandas as pd
import numpy as np

# Example: Creating a new feature
df['total_bathrooms'] = df['num_bedrooms'] + df['num_bathrooms']

# Log transformation
df['log_price'] = np.log(df['price'])

# One-hot encoding
df = pd.get_dummies(df, columns=['neighborhood'], drop_first=True)
    \end{lstlisting}
\end{frame}

\begin{frame}[fragile]
    \frametitle{Challenges in Feature Engineering - Introduction}
    \begin{block}{Key Challenges}
        Feature engineering presents several difficulties that can hamper model performance. 
        This includes:
        \begin{itemize}
            \item Data bias
            \item Dimensionality issues
            \item Feature correlation
        \end{itemize}
    \end{block}
\end{frame}

\begin{frame}[fragile]
    \frametitle{Challenges in Feature Engineering - Data Bias}
    \begin{block}{Data Bias}
        \begin{itemize}
            \item \textbf{Definition:} Data bias occurs when the dataset used for model training does not represent the actual population.
            \item \textbf{Example:} Training a model with images of domestic cats may lead to poor performance on wild cats.
            \item \textbf{Implications:} Models may inherit biases, producing skewed predictions and reinforcing stereotypes.
        \end{itemize}
    \end{block}
\end{frame}

\begin{frame}[fragile]
    \frametitle{Challenges in Feature Engineering - Dimensionality Issues}
    \begin{block}{Dimensionality Issues}
        \begin{itemize}
            \item \textbf{Definition:} High dimensionality complicates the modeling process and can lead to overfitting.
            \item \textbf{Curse of Dimensionality:} Increased dimensions result in sparse data, making it difficult for models to learn.
            \item \textbf{Example:} A dataset with hundreds of features may lead to poor performance on unseen data if only a few are informative.
        \end{itemize}
        \begin{block}{Addressing Dimensionality}
            Use techniques like:
            \begin{itemize}
                \item Principal Component Analysis (PCA)
                \item Recursive Feature Elimination (RFE)
            \end{itemize}
        \end{block}
    \end{block}
\end{frame}

\begin{frame}[fragile]
    \frametitle{Challenges in Feature Engineering - Feature Correlation}
    \begin{block}{Feature Correlation}
        \begin{itemize}
            \item \textbf{Definition:} Correlated features add redundancy to the model.
            \item \textbf{Example:} Including both size in square feet and number of rooms when predicting house prices.
            \item \textbf{Solution:} Identify and exclude redundant features using:
            \begin{itemize}
                \item Variance Inflation Factor (VIF)
            \end{itemize}
        \end{itemize}
    \end{block}
\end{frame}

\begin{frame}[fragile]
    \frametitle{Dimensionality Reduction and VIF Example}
    \begin{block}{Variance Inflation Factor (VIF) Calculation}
        \begin{lstlisting}[language=Python]
from statsmodels.stats.outliers_influence import variance_inflation_factor
import pandas as pd

# Assuming df is your DataFrame with features
X = df.values
vif_data = pd.DataFrame()
vif_data["Feature"] = df.columns
vif_data["VIF"] = [variance_inflation_factor(X, i) for i in range(X.shape[1])]

# Display the VIF DataFrame
print(vif_data)
        \end{lstlisting}
    \end{block}
\end{frame}

\begin{frame}[fragile]
    \frametitle{Summary - Challenges in Feature Engineering}
    \begin{itemize}
        \item Feature engineering involves navigating significant challenges: data bias, dimensionality, and feature correlation.
        \item Recognizing and addressing these challenges is vital for robust model creation.
        \item Applying appropriate techniques enhances feature engineering effectiveness.
    \end{itemize}
\end{frame}

\begin{frame}[fragile]
    \frametitle{Best Practices in Feature Engineering - Introduction}
    \begin{block}{Overview}
        Feature engineering is the process of transforming raw data into meaningful features that enhance model performance. 
        Engaging in best practices is crucial for maximizing effectiveness.
    \end{block}
\end{frame}

\begin{frame}[fragile]
    \frametitle{Best Practices in Feature Engineering - Data Understanding}
    \begin{itemize}
        \item \textbf{Understand Your Data:}
        \begin{itemize}
            \item Exploratory Data Analysis (EDA): Gain insights into distribution, missing values, outliers, and correlations.
            \item Visualization: Use histograms, box plots, and correlation matrices to visualize relationships and distributions.
        \end{itemize}
        \item \textbf{Example:} 
        \begin{itemize}
            \item A heat map can visualize correlations between features and target variables, identifying beneficial features.
        \end{itemize}
    \end{itemize}
\end{frame}

\begin{frame}[fragile]
    \frametitle{Best Practices in Feature Engineering - Feature Creation}
    \begin{itemize}
        \item \textbf{Feature Creation:}
        \begin{itemize}
            \item Domain Knowledge: Utilize expertise to create meaningful features (interactions, aggregations, transformations).
            \item Transformations: Common transformations include scaling, encoding categorical variables, and logarithmic transformations.
        \end{itemize}
        \item \textbf{Code Snippet:}
        \begin{lstlisting}[language=Python]
        # Scaling numeric features
        from sklearn.preprocessing import StandardScaler
        scaler = StandardScaler()
        scaled_features = scaler.fit_transform(data[['feature1', 'feature2']])
        \end{lstlisting}
    \end{itemize}
\end{frame}

\begin{frame}[fragile]
    \frametitle{Best Practices in Feature Engineering - Feature Selection}
    \begin{itemize}
        \item \textbf{Feature Selection:}
        \begin{itemize}
            \item Filter Methods: Statistical tests (e.g., Chi-square, ANOVA) filter irrelevant features.
            \item Wrapper Methods: Algorithms like Recursive Feature Elimination (RFE) evaluate model performance.
            \item Embedded Methods: Use models like LASSO for automatic feature selection during training.
        \end{itemize}
        \item \textbf{Example:} 
        \begin{itemize}
            \item LASSO regression builds a predictive model while selecting a subset of features.
        \end{itemize}
    \end{itemize}
\end{frame}

\begin{frame}[fragile]
    \frametitle{Best Practices in Feature Engineering - Avoiding Overfitting}
    \begin{itemize}
        \item \textbf{Avoiding Overfitting:}
        \begin{itemize}
            \item Cross-Validation: Implement k-fold cross-validation to ensure model performance consistency.
            \item Regularization Techniques: Use L1 or L2 regularization to prevent overly complex models.
        \end{itemize}
    \end{itemize}
\end{frame}

\begin{frame}[fragile]
    \frametitle{Best Practices in Feature Engineering - Continuous Iteration}
    \begin{itemize}
        \item \textbf{Continuous Iteration:}
        \begin{itemize}
            \item Feedback Loop: Incorporate new data and insights to refine features.
            \item Monitoring Model Performance: Regularly assess performance metrics (accuracy, precision, recall) and adjust features.
        \end{itemize}
    \end{itemize}
\end{frame}

\begin{frame}[fragile]
    \frametitle{Best Practices in Feature Engineering - Key Points and Conclusion}
    \begin{itemize}
        \item Key Points:
        \begin{itemize}
            \item Feature engineering is iterative and adaptable.
            \item Focus on the relationship between features and target variables.
            \item Utilize domain knowledge to craft meaningful features.
        \end{itemize}
        \item \textbf{Conclusion:} 
        \begin{itemize}
            \item Effective feature engineering significantly impacts model performance and accuracy.
            \item Adhering to best practices leads to better predictions and insights.
        \end{itemize}
    \end{itemize}
\end{frame}

\begin{frame}[fragile]{Conclusion - Feature Engineering and Selection}
    \begin{block}{Understanding Feature Engineering}
        Feature engineering is the process of selecting, modifying, or creating features from raw data to improve ML model performance.
    \end{block}
\end{frame}

\begin{frame}[fragile]{Conclusion - Importance of Feature Engineering}
    \begin{itemize}
        \item \textbf{Improves Model Performance:} Well-engineered features enhance accuracy and reliability of predictions.
        \item \textbf{Reduces Overfitting:} Selecting relevant features lowers model complexity and mitigates overfitting.
        \item \textbf{Enhances Interpretability:} Meaningful features help stakeholders understand model predictions.
        \item \textbf{Algorithm Compatibility:} Certain features work better with specific algorithms.
    \end{itemize}
\end{frame}

\begin{frame}[fragile]{Conclusion - Effects of Feature Selection}
    \begin{itemize}
        \item \textbf{Dimensionality Reduction:} Reduces inputs to improve computational efficiency.
        \item \textbf{Avoids Irrelevant Features:} Techniques like Recursive Feature Elimination help eliminate noise.
    \end{itemize}
\end{frame}

\begin{frame}[fragile]{Conclusion - Key Points to Emphasize}
    \begin{itemize}
        \item \textbf{Iterative Process:} Continuous evaluation and adjustments are essential.
        \item \textbf{Domain Knowledge:} Understanding context leads to better feature selection.
        \item \textbf{Tools and Techniques:} Employ PCA, feature interactions, and encoding strategies.
    \end{itemize}
\end{frame}

\begin{frame}[fragile]{Conclusion - Example to Illustrate Impact}
    \begin{block}{Example: Predicting House Prices}
        By creating the feature "price per room," we enhance the model's predictive accuracy compared to using raw features like the number of rooms, location, and age of the house.
    \end{block}
\end{frame}

\begin{frame}[fragile]{Conclusion - Summary}
    In conclusion, feature engineering and selection are crucial for improving ML algorithms' predictive power, interpretability, and operational efficiency. 
    \newline
    Next, we will explore various techniques and tools for effective feature engineering and selection in practice.
\end{frame}

\begin{frame}[fragile]{Conclusion - Discussion}
    Feel free to engage with the next slide, where we will open the floor for questions and discussions regarding these techniques!
\end{frame}

\begin{frame}
    \frametitle{Q\&A on Feature Engineering and Selection Techniques}
    \begin{block}{Introduction to Q\&A Session}
        \begin{itemize}
            \item \textbf{Objective}: Clarify doubts and promote discussions on feature engineering and selection techniques.
            \item \textbf{Importance}: Engaging in Q\&A facilitates a deeper understanding of how features influence model performance.
        \end{itemize}
    \end{block}
\end{frame}

\begin{frame}
    \frametitle{Key Concepts to Reflect On}
    \begin{enumerate}
        \item \textbf{Feature Engineering}:
            \begin{itemize}
                \item Using domain knowledge to create or modify features for model enhancement.
                \item \textbf{Examples}:
                    \begin{itemize}
                        \item Converting categorical variables (e.g., One-Hot Encoding).
                        \item Creating interaction features (e.g., multiplying two variables).
                    \end{itemize}
            \end{itemize}
        \item \textbf{Feature Selection}:
            \begin{itemize}
                \item Identifying a subset of relevant features for building a model.
                \item \textbf{Techniques}:
                    \begin{itemize}
                        \item **Filter Methods**: Statistical tests for correlation (e.g., Chi-square Test).
                        \item **Wrapper Methods**: Using a predictive model for feature evaluation (e.g., Recursive Feature Elimination).
                        \item **Embedded Methods**: Feature selection as part of the model construction (e.g., Lasso regression).
                    \end{itemize}
            \end{itemize}
    \end{enumerate}
\end{frame}

\begin{frame}[fragile]
    \frametitle{Example Discussion Points and Code Snippets}
    \begin{block}{Example Discussion Points}
        \begin{itemize}
            \item \textbf{Real-World Applications}: How does feature engineering impact models in fields like finance, healthcare, or e-commerce?
            \item \textbf{Challenges Faced}: Issues with high-dimensional datasets and solutions through feature selection.
        \end{itemize}
    \end{block}

    \begin{block}{Example of One-Hot Encoding in Python}
    \begin{lstlisting}[language=Python]
import pandas as pd

# Sample categorical data
data = {'Category': ['A', 'B', 'A', 'C']}
df = pd.DataFrame(data)

# One-hot encoding
df_encoded = pd.get_dummies(df, columns=['Category'])
print(df_encoded)
    \end{lstlisting}
    \end{block}

    \begin{block}{Feature Importance from a Model}
    \begin{lstlisting}[language=Python]
from sklearn.ensemble import RandomForestClassifier

# Assuming X_train and y_train are defined
model = RandomForestClassifier()
model.fit(X_train, y_train)
importance = model.feature_importances_
print(importance)
    \end{lstlisting}
    \end{block}
\end{frame}

\begin{frame}[fragile]
    \frametitle{Resources and Further Reading - Introduction}
    \begin{block}{Introduction to Feature Engineering and Selection}
        Feature Engineering and Selection are critical steps in the machine learning pipeline that significantly impact model performance. Below are curated resources for you to deepen your understanding of these concepts.
    \end{block}
\end{frame}

\begin{frame}[fragile]
    \frametitle{Books for In-Depth Understanding}
    \begin{enumerate}
        \item \textbf{"Feature Engineering for Machine Learning: Principles and Techniques for Data Scientists"} by Alice Zheng and Amanda Casari
        \begin{itemize}
            \item Overview of feature engineering techniques and strategies.
            \item Case studies demonstrating practical applications.
        \end{itemize}
        
        \item \textbf{"Pattern Recognition and Machine Learning"} by Christopher Bishop
        \begin{itemize}
            \item Comprehensive coverage of probabilistic graphical models and their applications.
            \item Insight into feature representation in various contexts.
        \end{itemize}
    \end{enumerate}
\end{frame}

\begin{frame}[fragile]
    \frametitle{Online Courses and Research Papers}
    \begin{block}{Online Courses and Tutorials}
        \begin{enumerate}
            \item \textbf{Coursera: "Feature Engineering for Machine Learning"}
            \begin{itemize}
                \item Hands-on exercises on feature extraction, transformation, and selection techniques.
            \end{itemize}
            
            \item \textbf{Kaggle: "Feature Engineering"}
            \begin{itemize}
                \item Community-driven tutorials and examples that explore good practices in feature engineering for real-world datasets.
            \end{itemize}
        \end{enumerate}
    \end{block}
    
    \begin{block}{Research Papers to Explore}
        \begin{enumerate}
            \item \textbf{“The Elements of Statistical Learning”} by Hastie, Tibshirani, and Friedman
            \begin{itemize}
                \item A foundational text that discusses various aspects of statistical learning, including feature selection.
            \end{itemize}
            
            \item \textbf{"Feature Selection for High-Dimensional Data: A Fast Correlation-Based Filter Solution"}
            \begin{itemize}
                \item Provides insights into filter-based feature selection methods and algorithms.
            \end{itemize}
        \end{enumerate}
    \end{block}
\end{frame}

\begin{frame}[fragile]
    \frametitle{Key Points and Example Code}
    \begin{block}{Key Points to Emphasize}
        \begin{itemize}
            \item Importance of Feature Engineering: Good feature engineering can lead to better model accuracy and efficiency. It's often more crucial than the algorithm itself.
            \item Specific Techniques: Understand techniques such as normalization, scaling, encoding categorical variables, and polynomial feature creation.
            \item Selecting Features: Emphasize methods such as Recursive Feature Elimination (RFE), LASSO regression, and Tree-based methods for feature selection.
        \end{itemize}
    \end{block}
    
    \begin{block}{Example Code Snippet (Feature Scaling using Python)}
    \begin{lstlisting}[language=Python]
from sklearn.preprocessing import StandardScaler
import pandas as pd

# Example DataFrame
data = pd.DataFrame({'feature1': [100, 200, 300], 'feature2': [0.1, 0.2, 0.3]})

# Standardizing the features
scaler = StandardScaler()
scaled_data = scaler.fit_transform(data)

print(scaled_data)
    \end{lstlisting}
    \end{block}
\end{frame}

\begin{frame}[fragile]
    \frametitle{Engagement Tip and Conclusion}
    \begin{block}{Engagement Tip}
        Consider working on a hands-on project where you apply these techniques to a dataset of your choice. Utilize platforms like Kaggle to find datasets and apply feature engineering practices as presented in these resources!
    \end{block}
    
    \begin{block}{Conclusion}
        These resources will not only enhance your theoretical knowledge but also allow you to apply feature engineering and selection principles effectively in practical scenarios. Happy learning!
    \end{block}
\end{frame}


\end{document}