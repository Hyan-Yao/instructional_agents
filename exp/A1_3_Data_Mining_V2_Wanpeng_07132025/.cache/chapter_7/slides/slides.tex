\documentclass[aspectratio=169]{beamer}

% Theme and Color Setup
\usetheme{Madrid}
\usecolortheme{whale}
\useinnertheme{rectangles}
\useoutertheme{miniframes}

% Additional Packages
\usepackage[utf8]{inputenc}
\usepackage[T1]{fontenc}
\usepackage{graphicx}
\usepackage{booktabs}
\usepackage{listings}
\usepackage{amsmath}
\usepackage{amssymb}
\usepackage{xcolor}
\usepackage{tikz}
\usepackage{pgfplots}
\pgfplotsset{compat=1.18}
\usetikzlibrary{positioning}
\usepackage{hyperref}

% Custom Colors
\definecolor{myblue}{RGB}{31, 73, 125}
\definecolor{mygray}{RGB}{100, 100, 100}
\definecolor{mygreen}{RGB}{0, 128, 0}
\definecolor{myorange}{RGB}{230, 126, 34}
\definecolor{mycodebackground}{RGB}{245, 245, 245}

% Set Theme Colors
\setbeamercolor{structure}{fg=myblue}
\setbeamercolor{frametitle}{fg=white, bg=myblue}
\setbeamercolor{title}{fg=myblue}
\setbeamercolor{section in toc}{fg=myblue}
\setbeamercolor{item projected}{fg=white, bg=myblue}
\setbeamercolor{block title}{bg=myblue!20, fg=myblue}
\setbeamercolor{block body}{bg=myblue!10}
\setbeamercolor{alerted text}{fg=myorange}

% Set Fonts
\setbeamerfont{title}{size=\Large, series=\bfseries}
\setbeamerfont{frametitle}{size=\large, series=\bfseries}
\setbeamerfont{caption}{size=\small}
\setbeamerfont{footnote}{size=\tiny}

% Footer and Navigation Setup
\setbeamertemplate{footline}{
  \leavevmode%
  \hbox{%
  \begin{beamercolorbox}[wd=.3\paperwidth,ht=2.25ex,dp=1ex,center]{author in head/foot}%
    \usebeamerfont{author in head/foot}\insertshortauthor
  \end{beamercolorbox}%
  \begin{beamercolorbox}[wd=.5\paperwidth,ht=2.25ex,dp=1ex,center]{title in head/foot}%
    \usebeamerfont{title in head/foot}\insertshorttitle
  \end{beamercolorbox}%
  \begin{beamercolorbox}[wd=.2\paperwidth,ht=2.25ex,dp=1ex,center]{date in head/foot}%
    \usebeamerfont{date in head/foot}
    \insertframenumber{} / \inserttotalframenumber
  \end{beamercolorbox}}%
  \vskip0pt%
}

% Turn off navigation symbols
\setbeamertemplate{navigation symbols}{}

% Title Page Information
\title[Week 7: Introduction to Neural Networks]{Week 7: Introduction to Neural Networks}
\author[J. Smith]{John Smith, Ph.D.}
\institute[University Name]{
  Department of Computer Science\\
  University Name\\
}
\date{\today}

\begin{document}

\frame{\titlepage}

\begin{frame}[fragile]
    \titlepage
\end{frame}

\begin{frame}[fragile]
    \frametitle{Overview}
    
    \begin{itemize}
        \item Neural networks are fundamental to artificial intelligence (AI).
        \item They are crucial in data mining due to their ability to analyze and learn from complex datasets.
        \item Applications include natural language processing (NLP) models such as ChatGPT.
    \end{itemize}
\end{frame}

\begin{frame}[fragile]
    \frametitle{Significance of Neural Networks in Data Mining}
    
    \begin{enumerate}
        \item \textbf{Data Complexity:}
        \begin{itemize}
            \item Traditional algorithms often struggle with high-dimensional and nonlinear data.
            \item Neural networks learn patterns from large datasets without explicit programming.
            \item \textbf{Example:} Analyze patient records to identify health trends.
        \end{itemize}
        
        \item \textbf{Feature Extraction:}
        \begin{itemize}
            \item They learn the best features for classification through their processing layers.
            \item \textbf{Illustration:} In image recognition, earlier layers detect edges; deeper layers identify complex shapes.
        \end{itemize}
        
        \item \textbf{Scalability:}
        \begin{itemize}
            \item Neural networks can scale with data size, ideal for big data applications.
            \item \textbf{Example:} Used by companies like Google and Facebook for content recommendations based on user behavior.
        \end{itemize}
    \end{enumerate}
\end{frame}

\begin{frame}[fragile]
    \frametitle{Recent AI Applications: ChatGPT}

    \begin{itemize}
        \item \textbf{ChatGPT} exemplifies neural network applications that leverage data mining.
        \item Key tasks include:
        \begin{itemize}
            \item \textbf{Text Generation:} Producing fluent, contextually relevant text responses.
            \item \textbf{Understanding Context:} Identifying user intent via input analysis.
        \end{itemize}
    \end{itemize}
    
    \begin{block}{Benefits from Data Mining}
        \begin{itemize}
            \item \textbf{Learning Patterns:} It learns linguistic patterns and user preferences from vast datasets.
            \item \textbf{Continuous Improvement:} Improves its ability to understand and generate natural language over time.
        \end{itemize}
    \end{block}
\end{frame}

\begin{frame}[fragile]
    \frametitle{Key Points to Emphasize}
    
    \begin{itemize}
        \item Neural networks connect large datasets to actionable insights in data mining.
        \item They automate pattern extraction, reducing the need for extensive feature engineering.
        \item Applications like ChatGPT illustrate the powerful combination of neural networks and data mining, transforming human-machine interactions.
    \end{itemize}
\end{frame}

\begin{frame}[fragile]
    \frametitle{Conclusion}
    
    \begin{itemize}
        \item Understanding neural networks is crucial for navigating the AI landscape.
        \item They empower organizations to extract valuable insights from complex data structures.
        \item As they evolve, neural networks will continue to play a pivotal role in data mining and AI applications.
    \end{itemize}
\end{frame}

\begin{frame}[fragile]
    \frametitle{Motivations for Neural Networks in Data Mining}
    \begin{block}{Understanding the Necessity of Neural Networks}
        Data mining is the process of discovering patterns and extracting useful information from large sets of data. As datasets grow increasingly complex, neural networks provide powerful tools for analysis.
    \end{block}
\end{frame}

\begin{frame}[fragile]
    \frametitle{Key Advantages of Neural Networks}
    \begin{enumerate}
        \item \textbf{Handling High-Dimensional Data}
            \begin{itemize}
                \item Neural networks process complex, high-dimensional inputs efficiently.
                \item \textit{Example:} Image recognition tasks distinguish objects through hierarchical feature learning.
            \end{itemize}
        
        \item \textbf{Non-Linear Relationships}
            \begin{itemize}
                \item Neural networks approximate complex non-linear relationships via activation functions.
                \item \textit{Illustration:} The ReLU activation function enables learning of intricate patterns:
                \begin{equation}
                    \text{ReLU}(x) = \max(0, x)
                \end{equation}
            \end{itemize}
    \end{enumerate}
\end{frame}

\begin{frame}[fragile]
    \frametitle{Further Advantages}
    \begin{enumerate}
        \setcounter{enumi}{2}
        \item \textbf{Adaptability and Scalability}
            \begin{itemize}
                \item Neural networks improve with more data, adapting to dynamic environments.
                \item \textit{Example:} ChatGPT learns from user interactions in real-time.
            \end{itemize}
        
        \item \textbf{Feature Engineering Automation}
            \begin{itemize}
                \item Neural networks reduce the need for manual feature selection and preprocessing.
                \item This automation accelerates analysis, yielding deeper insights efficiently.
            \end{itemize}
        
        \item \textbf{Enhanced Predictive Modeling}
            \begin{itemize}
                \item Neural networks excel in forecasting by uncovering patterns in historical data.
                \item \textit{Example:} Predicting stock prices in finance through market trend analysis.
            \end{itemize}
    \end{enumerate}
\end{frame}

\begin{frame}[fragile]
    \frametitle{Basic Concepts of Neural Networks - Introduction}
    Neural networks are a cornerstone of modern artificial intelligence. They enable computers to:
    \begin{itemize}
        \item Learn from data,
        \item Recognize patterns, and
        \item Make decisions similar to the human brain.
    \end{itemize}
    Understanding the basic components is essential for exploring advanced AI applications.
\end{frame}

\begin{frame}[fragile]
    \frametitle{Basic Concepts of Neural Networks - Neurons}
    \textbf{Neurons}
    \begin{itemize}
        \item Definition: Basic building blocks of neural networks, mimicking biological neurons.
        \item Function: Receive inputs, process them, and produce output.
        \item Example: In image recognition, a neuron might respond strongly to edges or textures.
    \end{itemize}
    \begin{block}{Key Representation}
        \begin{equation}
            Output = ActivationFunction \left( \sum (Weight_i \times Input_i) + Bias \right)
        \end{equation}
    \end{block}
\end{frame}

\begin{frame}[fragile]
    \frametitle{Basic Concepts of Neural Networks - Layers}
    \textbf{Layers}
    \begin{itemize}
        \item \textbf{Input Layer}: Receives input data; each node corresponds to a feature.
        \item \textbf{Hidden Layers}: Where computation occurs; can vary in number and neurons.
        \item \textbf{Output Layer}: Produces the final output (e.g., classification or regression).
    \end{itemize}
    \textbf{Example:} In a cat vs. dog classifier, the input layer might process features from images, hidden layers learn features like fur texture, and the output layer provides the classification.
\end{frame}

\begin{frame}[fragile]
    \frametitle{Basic Concepts of Neural Networks - Activation Functions}
    \textbf{Activation Functions}
    \begin{itemize}
        \item Purpose: Introduce non-linearity, enabling learning of complex patterns.
        \item Common Functions:
        \begin{itemize}
            \item \textbf{Sigmoid}: \( f(x) = \frac{1}{1 + e^{-x}} \) 
            \item \textbf{ReLU (Rectified Linear Unit)}: \( f(x) = \max(0, x) \) 
            \item \textbf{Softmax}: Used for multi-class classification, normalizing to probabilities.
        \end{itemize}
    \end{itemize}
    % Diagram would be included here in a graphical form.
\end{frame}

\begin{frame}[fragile]
    \frametitle{How Neural Networks Mimic Human Brain Functionality}
    Neural networks exhibit several parallels with human brain functionality:
    \begin{itemize}
        \item \textbf{Parallel Processing}: Evaluating many inputs at once, similar to the brain.
        \item \textbf{Learning through Experience}: Adjust weights based on feedback, much like human learning.
        \item \textbf{Adaptability}: Can adapt to new information, similar to how the brain adjusts to contexts.
    \end{itemize}
\end{frame}

\begin{frame}[fragile]
    \frametitle{Conclusion and Key Points}
    \textbf{Conclusion:}
    \begin{itemize}
        \item Neural networks are powerful tools in AI for analyzing complex data sets.
        \item Understanding neurons, layers, and activation functions is crucial for developing AI systems.
    \end{itemize}
    \textbf{Key Points to Emphasize:}
    \begin{itemize}
        \item Analogy between artificial and biological neurons enhances understanding.
        \item Flexibility of networks' layers and activation functions is key to their robustness.
        \item Foundation knowledge is essential before advanced architectures and applications.
    \end{itemize}
\end{frame}

\begin{frame}[fragile]
    \frametitle{Architecture of Neural Networks - Introduction}
    Neural networks are a crucial component of artificial intelligence, serving as powerful models for learning complex patterns from data.
    
    \begin{itemize}
        \item Explore the structure of neural networks: input, hidden, and output layers.
        \item The architecture affects the model's learning capability and application.
    \end{itemize}
\end{frame}

\begin{frame}[fragile]
    \frametitle{Layer Types in Neural Networks}
    
    \begin{enumerate}
        \item \textbf{Input Layer}
            \begin{itemize}
                \item Represents features of the input data
                \item Example: Each pixel value in an image
                \item Key Point: Number of nodes corresponds to the input feature vector size.
            \end{itemize}
        \item \textbf{Hidden Layer(s)}
            \begin{itemize}
                \item Processes data through weights and activations
                \item Example: One or more layers transforming inputs into higher-level representations
                \item Key Point: More layers can learn more complex patterns.
            \end{itemize}
        \item \textbf{Output Layer}
            \begin{itemize}
                \item Produces final results (predictions or classifications)
                \item Example: A node for binary classification using a sigmoid function
                \item Key Point: Structure depends on specific tasks (e.g., classification vs. regression).
            \end{itemize}
    \end{enumerate}
\end{frame}

\begin{frame}[fragile]
    \frametitle{Common Neural Network Architectures}

    \begin{enumerate}
        \item \textbf{Feedforward Neural Network}
            \begin{itemize}
                \item Simple architecture with data flowing one direction.
                \item Use Case: Basic classification tasks.
            \end{itemize}
        \item \textbf{Convolutional Neural Network (CNN)}
            \begin{itemize}
                \item Processes grid-like data (e.g., images) using convolutional layers.
                \item Use Case: Image recognition and processing, such as facial feature analysis.
            \end{itemize}
        \item \textbf{Recurrent Neural Network (RNN)}
            \begin{itemize}
                \item Maintains information over time for sequential data.
                \item Use Case: Language models and time series predictions.
            \end{itemize}
        \item \textbf{Transformer Networks}
            \begin{itemize}
                \item Uses self-attention mechanisms for data processing.
                \item Use Case: NLP tasks, such as those used by ChatGPT.
            \end{itemize}
    \end{enumerate}
\end{frame}

\begin{frame}[fragile]
    \frametitle{Summary and Key Points}
    
    \begin{block}{Summary}
        \begin{itemize}
            \item Neural networks comprise input, hidden, and output layers with distinct roles.
            \item The architecture greatly impacts learning capability and application.
            \item Understanding layer types and functions is essential for effective model design.
        \end{itemize}
    \end{block}
    
    \begin{block}{Key Points to Remember}
        \begin{itemize}
            \item Input nodes represent features.
            \item Hidden layers enable complex transformations.
            \item Output layers deliver predictions, tailored to the problem type.
        \end{itemize}
    \end{block}
\end{frame}

\begin{frame}[fragile]
    \frametitle{Training Neural Networks - Introduction}
    Training a neural network involves teaching it to make predictions or classifications based on input data. This process includes two main steps: 
    \begin{itemize}
        \item \textbf{Forward Propagation}
        \item \textbf{Backpropagation}
    \end{itemize}
    \begin{block}{Key Points}
        - Forward propagation computes the output from input data.
        - Backpropagation updates the model based on the errors.
    \end{block}
\end{frame}

\begin{frame}[fragile]
    \frametitle{Training Neural Networks - Forward Propagation}
    \textbf{Definition:} Forward propagation is where input data is passed through the neural network layers, producing an output.

    \textbf{How it Works:}
    \begin{itemize}
        \item Each neuron receives inputs multiplied by weights.
        \item A bias term is added, and the result is passed through an activation function (e.g., ReLU, Sigmoid).
    \end{itemize}

    \textbf{Mathematical Representation:}
    \begin{equation}
        y = f(W \cdot x + b)
    \end{equation}
    Where:
    \begin{itemize}
        \item \(y\) is the output,
        \item \(W\) is the weight,
        \item \(x\) is the input,
        \item \(b\) is the bias,
        \item \(f\) is the activation function.
    \end{itemize}
\end{frame}

\begin{frame}[fragile]
    \frametitle{Training Neural Networks - Backpropagation and Loss Function}
    \textbf{1. Loss Function:}
    \begin{itemize}
        \item \textbf{Purpose:} Measures how well the predicted output matches the actual target.
        \item \textbf{Common Loss Functions:}
            \begin{itemize}
                \item Mean Squared Error (MSE) for regression tasks.
                \item Cross-Entropy Loss for classification tasks.
            \end{itemize}
    \end{itemize}

    \textbf{Formula for MSE:}
    \begin{equation}
        \text{MSE} = \frac{1}{n} \sum_{i=1}^{n} (y_i - \hat{y}_i)^2
    \end{equation}

    \textbf{2. Backpropagation:}
    \begin{itemize}
        \item \textbf{Definition:} Updates the weights by calculating the gradient of the loss function.
        \item \textbf{Gradient Descent Update Rule:}
        \begin{equation}
            W = W - \eta \cdot \nabla L
        \end{equation}
    \end{itemize}
\end{frame}

\begin{frame}[fragile]
    \frametitle{Training Neural Networks - Optimization}
    \textbf{Purpose:}
    \begin{itemize}
        \item Optimization algorithms (like Stochastic Gradient Descent, Adam) help minimize the loss function efficiently.
    \end{itemize}

    \textbf{Key Points:}
    \begin{itemize}
        \item The choice of optimizer and learning rate greatly impacts convergence.
        \item Proper tuning can lead to better performance of the neural network.
    \end{itemize}

    \begin{block}{Key Takeaways}
        - Training combines forward propagation for computations and backpropagation for updates.
        - The loss function provides feedback for training.
        - Optimizers play a crucial role in training efficiency.
    \end{block}
\end{frame}

\begin{frame}[fragile]
    \frametitle{Applications of Neural Networks in Data Mining}
    
    \begin{block}{Overview of Data Mining}
        Data mining is the process of discovering patterns and extracting valuable information from large datasets. 
        It often involves techniques from statistics, machine learning, and database systems.
    \end{block}

    \begin{itemize}
        \item \textbf{Important Aspects of Data Mining:}
        \begin{itemize}
            \item \textbf{Decision Making:} Informed decisions based on data-driven insights.
            \item \textbf{Trend Analysis:} Identifying patterns to predict future behaviors.
            \item \textbf{Automation:} Streamlining processes through advanced algorithms.
        \end{itemize}
    \end{itemize}
\end{frame}

\begin{frame}[fragile]
    \frametitle{Neural Networks in Data Mining}

    \begin{block}{Overview}
        Neural networks, inspired by the human brain, are effective in data mining for:
    \end{block}

    \begin{enumerate}
        \item \textbf{Classification}
        \begin{itemize}
            \item \textbf{Definition:} Assigning items to target categories based on features.
            \item \textbf{Example:} Classifying emails as “spam” or “not spam.”
        \end{itemize}
        
        \item \textbf{Regression}
        \begin{itemize}
            \item \textbf{Definition:} Predicting a continuous output variable based on input features.
            \item \textbf{Example:} Predicting house prices based on various features.
            \item \textbf{Formula:}
            \begin{equation}
                y = w_1x_1 + w_2x_2 + b
            \end{equation}
        \end{itemize}

        \item \textbf{Clustering}
        \begin{itemize}
            \item \textbf{Definition:} Grouping similar data points together.
            \item \textbf{Example:} Customer segmentation for targeted marketing.
        \end{itemize}
    \end{enumerate}
\end{frame}

\begin{frame}[fragile]
    \frametitle{Key Points and Conclusion}

    \begin{itemize}
        \item \textbf{Scalability:} Neurons can handle large datasets, ideal for big data.
        \item \textbf{Flexibility:} Applicable for structured, unstructured, and semi-structured data.
        \item \textbf{Performance:} Often outperforms traditional methods in tasks like image recognition and natural language processing.
    \end{itemize}

    \vspace{1em}
    \begin{block}{Conclusion}
        Neural networks revolutionized data mining with techniques for classification, regression, and clustering, enabling actionable insights from complex datasets.
    \end{block}

    \vspace{1em}
    \textbf{Next Up:} Ethical Considerations – Understanding implications regarding data privacy and biases in AI models.
\end{frame}

\begin{frame}[fragile]
    \frametitle{Ethical Implications of Using Neural Networks}
    \textbf{Introduction}
    \begin{itemize}
        \item Neural Networks are becoming increasingly integrated into sectors like healthcare, finance, and social media.
        \item Ethical implications of their use warrant careful consideration.
        \item Key concerns: 
        \begin{itemize}
            \item Data Privacy
            \item Bias in AI Models
        \end{itemize}
    \end{itemize}
\end{frame}

\begin{frame}[fragile]
    \frametitle{Key Ethical Considerations - Data Privacy}
    \begin{itemize}
        \item \textbf{Definition}: Handling and protecting personal data to prevent unauthorized access and use.
        \item \textbf{Concern}: 
        \begin{itemize}
            \item Neural networks require vast amounts of data, potentially including sensitive information.
        \end{itemize}
        \item \textbf{Examples}:
        \begin{itemize}
            \item Health Data: Potential breaches in predicting patient outcomes.
            \item Facial Recognition: Capture of images without consent.
        \end{itemize}
        \item \textbf{Best Practices}:
        \begin{itemize}
            \item Strong encryption methods
            \item Data anonymization 
            \item Data access controls
        \end{itemize}
    \end{itemize}
\end{frame}

\begin{frame}[fragile]
    \frametitle{Key Ethical Considerations - Bias in AI Models}
    \begin{itemize}
        \item \textbf{Definition}: Prejudice in data collection, model training, or results interpretation leading to unfair outcomes.
        \item \textbf{Concern}:
        \begin{itemize}
            \item Neural networks can perpetuate or exacerbate biases present in the training data.
        \end{itemize}
        \item \textbf{Examples}:
        \begin{itemize}
            \item Hiring Algorithms: Bias against female candidates due to male-dominated training data.
            \item Credit Scoring: Disadvantage for minority groups based on biased data.
        \end{itemize}
        \item \textbf{Mitigating Bias}:
        \begin{itemize}
            \item Diverse datasets
            \item Regular audits for bias
            \item Engage diverse development teams
        \end{itemize}
    \end{itemize}
\end{frame}

\begin{frame}[fragile]
    \frametitle{Importance of Ethical Considerations}
    \begin{itemize}
        \item Ethical implications are essential for fostering trust between technology and users.
        \item Transparent practices enhance adoption and create positive societal impacts of AI.
    \end{itemize}
    \textbf{Conclusion}
    \begin{itemize}
        \item Prioritizing data privacy and mitigating bias leads to equitable AI systems.
        \item Ethical considerations must be integral to AI development and deployment.
    \end{itemize}
    \textbf{Key Takeaways}:
    \begin{itemize}
        \item Protect Personal Data
        \item Address Bias
        \item Build Trust through Transparency
    \end{itemize}
\end{frame}

\begin{frame}[fragile]
    \frametitle{Outline of Key Points}
    \begin{enumerate}
        \item Definition of Data Privacy and Bias
        \item Reasons for Ethical Concerns
        \item Examples of Ethical Issues
        \item Strategies to Mitigate Issues
        \item Importance of Ethical Practices in AI 
    \end{enumerate}
    \textbf{Summary:}  
    \begin{itemize}
        \item Recognizing and addressing ethical implications leads to responsible neural network development.
    \end{itemize}
\end{frame}

\begin{frame}[fragile]
    \frametitle{Conclusion and Future Directions - Overview}
    Neural networks are integral to the evolution of data mining and AI. They effectively analyze vast data sets, recognize complex patterns, and support predictive analytics, driving innovations across various sectors.
    
    \begin{itemize}
        \item Role of neural networks in processing complex data
        \item Importance in decision-making through data-driven insights
    \end{itemize}
\end{frame}

\begin{frame}[fragile]
    \frametitle{Motivations for Development}
    The continuous advancement of neural networks is prompted by the following factors:
    
    \begin{itemize}
        \item **Increasing Complexity of Data**: Neural networks tackle complex data types, including images, audio, and text.
        
        \item **Need for Data-Driven Insights**: Businesses and researchers depend on neural networks for extracting valuable information that informs strategic decisions.
    \end{itemize}
\end{frame}

\begin{frame}[fragile]
    \frametitle{Recent Advances and Research Opportunities}
    The field is witnessing significant advancements:
    
    \begin{itemize}
        \item **Transformers in NLP**: Revolutionizing natural language tasks through state-of-the-art architectures like ChatGPT.
        
        \item **Enhanced Computer Vision**: CNNs contribute to breakthroughs in areas such as medical image analysis and autonomous vehicles.
    \end{itemize}

    \begin{block}{Key Areas for Future Research}
        \begin{enumerate}
            \item Explainability and Ethics
            \item Robustness and Adversarial Training
            \item Federated Learning
            \item Integration of Multi-modal Data
        \end{enumerate}
    \end{block}
\end{frame}

\begin{frame}[fragile]
    \frametitle{Conclusion}
    Neural networks have fundamentally changed data mining and AI. The focus moving forward should be on:
    
    \begin{itemize}
        \item Addressing ethical concerns and improving explainability
        \item Enhancing model robustness against adversarial inputs
        \item Leveraging diverse data types for richer AI solutions
    \end{itemize}
    
    \begin{block}{Closing Thought}
        The evolution of neural networks will shape future AI innovations, emphasizing the necessity for ethical and effective applications in real-world scenarios.
    \end{block}
\end{frame}


\end{document}