\documentclass{beamer}

% Theme choice
\usetheme{Madrid} % You can change to e.g., Warsaw, Berlin, CambridgeUS, etc.

% Encoding and font
\usepackage[utf8]{inputenc}
\usepackage[T1]{fontenc}

% Graphics and tables
\usepackage{graphicx}
\usepackage{booktabs}

% Code listings
\usepackage{listings}
\lstset{
basicstyle=\ttfamily\small,
keywordstyle=\color{blue},
commentstyle=\color{gray},
stringstyle=\color{red},
breaklines=true,
frame=single
}

% Math packages
\usepackage{amsmath}
\usepackage{amssymb}

% Colors
\usepackage{xcolor}

% TikZ and PGFPlots
\usepackage{tikz}
\usepackage{pgfplots}
\pgfplotsset{compat=1.18}
\usetikzlibrary{positioning}

% Hyperlinks
\usepackage{hyperref}

% Title information
\title{Chapter 15: Course Review and Final Exam Preparation}
\author{Your Name}
\institute{Your Institution}
\date{\today}

\begin{document}

\frame{\titlepage}

\begin{frame}[fragile]
    \frametitle{Introduction to Course Review and Final Exam Preparation}
    \begin{block}{Overview}
        This presentation will cover the purpose, structure, and significance of the course review session, 
        aimed at consolidating your understanding before the final exam.
    \end{block}
\end{frame}

\begin{frame}[fragile]
    \frametitle{Purpose of the Course Review}
    \begin{enumerate}
        \item \textbf{Consolidation of Knowledge:}
        \begin{itemize}
            \item Revisiting key concepts enhances understanding and retention.
            \item Example: Re-reviewing supply and demand principles and IS-LM models in economics.
        \end{itemize}

        \item \textbf{Identifying Gaps:}
        \begin{itemize}
            \item Discover areas of less confidence to focus study efforts.
            \item Example: Struggling with calculus concepts can guide you to seek clarification.
        \end{itemize}

        \item \textbf{Exam Preparation:}
        \begin{itemize}
            \item Familiarize with exam formats and question types to enhance study plans.
            \item Example: Practicing mixed-format questions similar to past exams.
        \end{itemize}
    \end{enumerate}
\end{frame}

\begin{frame}[fragile]
    \frametitle{Structure of the Review Session}
    \begin{enumerate}
        \item \textbf{Interactive Discussions:}
        \begin{itemize}
            \item Engage in guided discussions that solidify understanding.
            \item Example: Peer teach-backs to explain concepts to each other.
        \end{itemize}
        
        \item \textbf{Review of Key Topics:}
        \begin{itemize}
            \item Focus on major themes and critical topics.
            \item Example: Revisiting Maslow’s Hierarchy of Needs in psychology.
        \end{itemize}
        
        \item \textbf{Practice Questions:}
        \begin{itemize}
            \item Engage with sample questions to mimic final exam structure.
            \item Example: Working through sample equations in mathematics.
        \end{itemize}
    \end{enumerate}
\end{frame}

\begin{frame}[fragile]
    \frametitle{Learning Objectives - Overview}
    In this session, we will outline the specific learning objectives designed to prepare you for the final exam. Our focus will be on solidifying your understanding of core concepts learned throughout the course and aligning our review activities with the expectations of the final assessment.
\end{frame}

\begin{frame}[fragile]
    \frametitle{Learning Objectives - Key Objectives}
    \begin{enumerate}
        \item \textbf{Consolidate Core Concepts}:
            \begin{itemize}
                \item Review fundamental theories such as supervised, unsupervised, and reinforcement learning.
                \item Example: Distinguish between supervised learning with labeled data and unsupervised learning with unlabeled data.
            \end{itemize}
        \item \textbf{Identify Important Topics}:
            \begin{itemize}
                \item Recognize key topics likely to appear on the final exam (e.g., algorithms, metrics).
                \item Example: Understand evaluation metrics like accuracy, precision, recall, and F1-score.
            \end{itemize}
    \end{enumerate}
\end{frame}

\begin{frame}[fragile]
    \frametitle{Learning Objectives - Additional Goals}
    \begin{enumerate}[resume]
        \item \textbf{Practice Problem-Solving}:
            \begin{itemize}
                \item Engage in exercises mimicking final exam questions.
                \item Example: Solve a case study question on model selection for a specific dataset.
            \end{itemize}
        \item \textbf{Clarify Common Misunderstandings}:
            \begin{itemize}
                \item Address misconceptions related to course material.
            \end{itemize}
        \item \textbf{Preparation Strategies}:
            \begin{itemize}
                \item Develop effective study strategies (e.g., time management, problem-solving).
                \item Example: Create a study schedule for each major topic.
            \end{itemize}
    \end{enumerate}
\end{frame}

\begin{frame}[fragile]
    \frametitle{Learning Objectives - Key Points}
    \begin{block}{Key Points to Emphasize}
        \begin{itemize}
            \item \textbf{Integration of Theory and Practice}: Connect theoretical knowledge with practical applications.
            \item \textbf{Active Participation}: Encourage questions and discussions.
            \item \textbf{Resource Utilization}: Make use of lecture notes, textbooks, and online materials.
        \end{itemize}
    \end{block}

    \begin{block}{Expected Format of the Final Exam}
        The final exam will consist of:
        \begin{itemize}
            \item Multiple-choice questions.
            \item Short answer questions.
            \item Case study analysis.
        \end{itemize}
        Key focus areas will include definitions, practical applications, and analytical thinking.
    \end{block}
\end{frame}

\begin{frame}[fragile]
    \frametitle{Review of Core Concepts}
    \begin{block}{Core Concepts of Machine Learning}
        Machine Learning (ML) is a subset of artificial intelligence that enables systems to learn from data and improve their performance over time without being explicitly programmed. 
        We will review three fundamental types of machine learning: \textit{Supervised Learning}, \textit{Unsupervised Learning}, and \textit{Reinforcement Learning}.
    \end{block}
\end{frame}

\begin{frame}[fragile]
    \frametitle{Supervised Learning}
    \begin{block}{Definition}
        In supervised learning, the algorithm is trained on labeled data, where each training example is paired with an output label.
    \end{block}
    
    \begin{itemize}
        \item \textbf{Key Components:}
            \begin{itemize}
                \item \textbf{Training Data:} Set of input-output pairs
                \item \textbf{Model:} A function mapping inputs to outputs based on learned patterns
            \end{itemize}
        
        \item \textbf{Common Algorithms:}
            \begin{itemize}
                \item Linear Regression
                \item Logistic Regression
                \item Decision Trees
            \end{itemize}
        
        \item \textbf{Example:} Predicting house prices based on features like size and location.
    \end{itemize}
\end{frame}

\begin{frame}[fragile]
    \frametitle{Unsupervised Learning}
    \begin{block}{Definition}
        Unsupervised learning involves training on data without labeled outputs, aiming to uncover hidden patterns or structures.
    \end{block}
    
    \begin{itemize}
        \item \textbf{Key Components:}
            \begin{itemize}
                \item \textbf{Training Data:} Unlabeled data needing insights
                \item \textbf{Model:} Discovers relationships and groupings
            \end{itemize}
        
        \item \textbf{Common Algorithms:}
            \begin{itemize}
                \item K-Means Clustering
                \item Principal Component Analysis (PCA)
            \end{itemize}
        
        \item \textbf{Example:} Segmenting customers based on purchasing habits without predefined categories.
    \end{itemize}
\end{frame}

\begin{frame}[fragile]
    \frametitle{Reinforcement Learning}
    \begin{block}{Definition}
        Reinforcement learning (RL) focuses on training agents to make decisions through rewards and penalties for desirable and undesirable actions.
    \end{block}
    
    \begin{itemize}
        \item \textbf{Key Components:}
            \begin{itemize}
                \item \textbf{Agent:} The decision-maker
                \item \textbf{Environment:} The context of the agent's operation
                \item \textbf{Rewards:} Feedback used for shaping the learning process
            \end{itemize}
        
        \item \textbf{Common Algorithms:}
            \begin{itemize}
                \item Q-Learning
                \item Deep Q-Networks (DQN)
            \end{itemize}
        
        \item \textbf{Example:} Training a robot to navigate a maze with rewards for reaching exits and penalties for obstacles.
    \end{itemize}
\end{frame}

\begin{frame}[fragile]
    \frametitle{Key Points to Emphasize}
    \begin{itemize}
        \item \textbf{Data Labeling:} Critical in supervised learning, absent in unsupervised learning.
        \item \textbf{Objective:}
            \begin{itemize}
                \item Supervised: Predict outcomes
                \item Unsupervised: Find patterns
                \item Reinforcement: Maximize cumulative reward
            \end{itemize}
        \item \textbf{Applications:} Different methods serve distinct real-world needs, from analytics to autonomous systems.
    \end{itemize}
\end{frame}

\begin{frame}[fragile]
    \frametitle{Conclusion}
    Understanding the differences among supervised, unsupervised, and reinforcement learning is crucial for selecting the appropriate ML technique for specific problems. Make sure to review these concepts as you prepare for the final exam.
    
    \begin{block}{Note}
        Feel free to reach out with any questions as we review these concepts further!
    \end{block}
\end{frame}

\begin{frame}[fragile]
    \frametitle{Mathematical Foundations - Overview}
    The principles of mathematics form the backbone of machine learning. Key areas include:
    \begin{itemize}
        \item \textbf{Linear Algebra}
        \item \textbf{Statistics}
        \item \textbf{Probability}
    \end{itemize}
    Understanding these concepts is crucial for building effective models and interpreting their outputs.
\end{frame}

\begin{frame}[fragile]
    \frametitle{Mathematical Foundations - Linear Algebra}
    \begin{block}{Definition}
        A branch of mathematics concerning vector spaces and linear mappings between these spaces.
    \end{block}
    
    \begin{block}{Key Concepts}
        \begin{itemize}
            \item \textbf{Vectors}: Objects representing data points in multi-dimensional space.
            \begin{itemize}
                \item Example: A vector \( \mathbf{x} = [x_1, x_2, \ldots, x_n] \)
            \end{itemize}
            \item \textbf{Matrices}: A rectangular array of numbers that can represent linear transformations or dataset features.
            \begin{itemize}
                \item Example: A dataset with \(m\) samples and \(n\) features is represented as a matrix \( \mathbf{X} \) of size \( m \times n \).
            \end{itemize}
        \end{itemize}
    \end{block}
    
    \begin{block}{Applications in ML}
        \begin{itemize}
            \item Transformations and projections of data
            \item Calculations in algorithms like Principal Component Analysis (PCA)
        \end{itemize}
    \end{block}
\end{frame}

\begin{frame}[fragile]
    \frametitle{Mathematical Foundations - Statistics and Probability}
    \begin{block}{Statistics}
        \begin{itemize}
            \item \textbf{Definition}: The science of collecting, analyzing, interpreting, and presenting empirical data.
            \item \textbf{Key Concepts}:
                \begin{itemize}
                    \item \textbf{Descriptive Statistics}:
                        \begin{itemize}
                            \item Measures of central tendency (mean, median, mode)
                            \item Measures of variability (variance, standard deviation)
                            \item Example: Mean of a dataset \( \text{Mean} = \frac{1}{n}\sum_{i=1}^{n} x_i \)
                        \end{itemize}
                        
                    \item \textbf{Inferential Statistics}:
                        \begin{itemize}
                            \item Drawing conclusions about a population based on sample data.
                            \item Hypothesis testing and confidence intervals.
                        \end{itemize}
                \end{itemize}
            \item \textbf{Applications in ML}:
                \begin{itemize}
                    \item Understanding underlying data distributions
                    \item Evaluating model performance with metrics like accuracy and F1 score
                \end{itemize}
        \end{itemize}
    \end{block}
    
    \begin{block}{Probability}
        \begin{itemize}
            \item \textbf{Definition}: The study of randomness and uncertainty, quantifying how likely events are to occur.
            \item \textbf{Key Concepts}:
                \begin{itemize}
                    \item \textbf{Probability Distributions}:
                        \begin{itemize}
                            \item Normal distribution, Bernoulli distribution, and others
                            \item Example: A normal distribution is characterized by its mean \( \mu \) and standard deviation \( \sigma \): \( \mathcal{N}(\mu, \sigma^2) \).
                        \end{itemize}
                    \item \textbf{Bayes' Theorem}:
                        \begin{equation}
                            P(A|B) = \frac{P(B|A) P(A)}{P(B)}
                        \end{equation}
                \end{itemize}
            \item \textbf{Applications in ML}:
                \begin{itemize}
                    \item Making predictions with uncertainty 
                    \item Bayesian networks and Markov models
                \end{itemize}
        \end{itemize}
    \end{block}
\end{frame}

\begin{frame}[fragile]
    \frametitle{Mathematical Foundations - Key Points and Conclusion}
    \begin{block}{Key Points to Emphasize}
        \begin{itemize}
            \item Mastery of linear algebra enables a better understanding of data representation and transformations.
            \item Statistics provides tools for data interpretation and model validation.
            \item Probability is essential for managing uncertainty and decision-making.
        \end{itemize}
    \end{block}
    
    \begin{block}{Conclusion}
        A strong grasp of these mathematical foundations will enhance your machine learning capabilities, allowing you to create robust models, make informed decisions, and interpret results effectively.
    \end{block}
\end{frame}

\begin{frame}[fragile]
    \frametitle{Programming Proficiency in Machine Learning - Overview}
    \begin{block}{Overview of Programming Languages for Machine Learning}
        \begin{itemize}
            \item \textbf{Python}: 
            \begin{itemize}
                \item Primary language for machine learning.
                \item Intuitive syntax and community support.
                \item Supported by numerous libraries and frameworks.
            \end{itemize}
            \item \textbf{R}: 
            \begin{itemize}
                \item Popular for statistical analysis and data visualization.
                \item Extensive packages for various modeling techniques.
            \end{itemize}
            \item \textbf{Java}: 
            \begin{itemize}
                \item Used in large-scale enterprise applications.
                \item Libraries like Weka for data mining applications.
            \end{itemize}
            \item \textbf{Julia}: 
            \begin{itemize}
                \item High performance in numerical analysis.
                \item Good integration with machine learning libraries.
            \end{itemize}
        \end{itemize}
    \end{block}
\end{frame}

\begin{frame}[fragile]
    \frametitle{Programming Proficiency in Machine Learning - Libraries}
    \begin{block}{Essential Libraries for Machine Learning}
        \begin{itemize}
            \item \textbf{TensorFlow}:
            \begin{itemize}
                \item Developed by Google for deep learning tasks.
                \item Deployable on various platforms.
                \item Example Usage:
                \begin{lstlisting}[language=Python]
import tensorflow as tf

# Define a simple neural network
model = tf.keras.Sequential([
    tf.keras.layers.Dense(64, activation='relu', input_shape=(32,)),
    tf.keras.layers.Dense(10)
])
model.compile(optimizer='adam', 
              loss='sparse_categorical_crossentropy', 
              metrics=['accuracy'])
                \end{lstlisting}
            \end{itemize}
            \item \textbf{scikit-learn}:
            \begin{itemize}
                \item Robust for classical machine learning algorithms.
                \item Clear tools for data mining and analysis.
                \item Example Usage:
                \begin{lstlisting}[language=Python]
from sklearn.model_selection import train_test_split
from sklearn.ensemble import RandomForestClassifier
from sklearn.metrics import accuracy_score

# Load dataset
X, y = ...  # Feature matrix and labels
X_train, X_test, y_train, y_test = train_test_split(X, y, test_size=0.2)

# Build and train the model
model = RandomForestClassifier()
model.fit(X_train, y_train)

# Make predictions
predictions = model.predict(X_test)
accuracy = accuracy_score(y_test, predictions)
print(f"Accuracy: {accuracy:.2f}")
                \end{lstlisting}
            \end{itemize}
        \end{itemize}
    \end{block}
\end{frame}

\begin{frame}[fragile]
    \frametitle{Programming Proficiency in Machine Learning - Key Points}
    \begin{block}{Key Points to Emphasize}
        \begin{itemize}
            \item \textbf{Practical Skills}: Proficiency in Python or R is essential, along with familiarity with libraries like TensorFlow and scikit-learn.
            \item \textbf{Interpreting Results}: Understanding model outputs requires knowledge of the underlying mathematical principles and statistics.
            \item \textbf{Deployment and Integration}: Knowing how to deploy models in production environments enhances their utility.
        \end{itemize}
    \end{block}

    \begin{block}{Conclusion}
        Gaining proficiency in programming and relevant libraries is crucial for tackling machine learning challenges effectively. Focus on coding examples, understanding library functions, and the workflow of building and deploying machine learning models.
    \end{block}
\end{frame}

\begin{frame}[fragile]
    \frametitle{Practical Problem Solving}
    \begin{block}{Understanding Problem Formulation in Machine Learning}
        Formulating machine learning problems effectively is crucial for leveraging algorithms to solve real-world issues. A well-defined problem helps in selecting appropriate models, features, and performance metrics.
    \end{block}
\end{frame}

\begin{frame}[fragile]
    \frametitle{Key Steps in Formulating a Machine Learning Problem}
    \begin{enumerate}
        \item \textbf{Define the Objective:}
        \begin{itemize}
            \item Determine what you want to achieve (e.g., classification, regression).
            \item Example: Predicting house prices based on features like location, size, and amenities.
        \end{itemize}
        
        \item \textbf{Identify the Data Requirements:}
        \begin{itemize}
            \item Gather relevant datasets that include necessary features and target variables.
            \item Example: Use a dataset from Kaggle containing features of houses sold in a particular region.
        \end{itemize}

        \item \textbf{Select the Features:}
        \begin{itemize}
            \item Choose the most relevant input variables for your problem.
            \item Example: Features might include square footage, number of bedrooms, and proximity to schools.
        \end{itemize}
        
        \item \textbf{Choose the Right Model:}
        \begin{itemize}
            \item Select an appropriate algorithm based on the problem type.
            \item Example: Use linear regression for house price prediction or decision trees for classification tasks.
        \end{itemize}
        
        \item \textbf{Determine Evaluation Metrics:}
        \begin{itemize}
            \item Establish how you will measure the performance of your model.
            \item Example: For regression, use Mean Squared Error (MSE) or R² score; for classification, consider accuracy, precision, recall, or F1 score.
        \end{itemize}
    \end{enumerate}
\end{frame}

\begin{frame}[fragile]
    \frametitle{Evaluating Algorithm Performance on Real-World Datasets}
    \begin{enumerate}
        \item \textbf{Train-Test Split:}
        \begin{itemize}
            \item Divide your dataset into a training set and a testing set (e.g., 80/20 split) to assess how well your model generalizes.
        \end{itemize}
        
        \item \textbf{Cross-Validation:}
        \begin{itemize}
            \item Implement techniques like k-fold cross-validation to provide a better estimate of model performance.
            \item This involves dividing the data into k subsets and training/testing k separate times.
        \end{itemize}
        
        \item \textbf{Performance Metrics:}
        \begin{itemize}
            \item Collect performance metrics from the test set and analyze them.
            \item Use graphs to visualize metrics (like ROC curves for classification).
        \end{itemize}
    \end{enumerate}

    \begin{block}{Example Code Snippet (Python Using Scikit-Learn)}
    \begin{lstlisting}[language=Python]
from sklearn.model_selection import train_test_split
from sklearn.metrics import accuracy_score
from sklearn.naive_bayes import GaussianNB

# Sample Data
X = [...]  # Features
y = [...]  # Labels

# Split Data
X_train, X_test, y_train, y_test = train_test_split(X, y, test_size=0.2, random_state=42)

# Model Training
model = GaussianNB()
model.fit(X_train, y_train)

# Predictions
y_pred = model.predict(X_test)

# Evaluate Performance
accuracy = accuracy_score(y_test, y_pred)
print(f"Accuracy: {accuracy:.2f}")
    \end{lstlisting}
    \end{block}
\end{frame}

\begin{frame}[fragile]
    \frametitle{Ethical Considerations - Overview}
    \begin{block}{Key Ethical Considerations in Machine Learning}
        In the rapidly evolving field of machine learning (ML), ethical considerations have become crucial. 
        We will review three primary concerns: bias, fairness, and accountability.
    \end{block}
\end{frame}

\begin{frame}[fragile]
    \frametitle{Ethical Considerations - Bias}
    \begin{block}{1. Bias in Machine Learning}
        \textbf{Definition:} Bias refers to systematic errors that unfairly favor one group over another. This can arise from:
        \begin{itemize}
            \item \textbf{Data Bias:} When training data is not representative.
            \item \textbf{Algorithmic Bias:} Certain algorithms may inherently favor particular outcomes.
        \end{itemize}
        
        \textbf{Example:}
        A hiring algorithm trained on historical data may perpetuate gender inequality by favoring male candidates due to biased historical hiring patterns.
    \end{block}
\end{frame}

\begin{frame}[fragile]
    \frametitle{Ethical Considerations - Fairness}
    \begin{block}{2. Fairness in Machine Learning}
        \textbf{Definition:} Fairness ensures ML outcomes do not favor or disadvantage specific groups. Important in applications like:
        \begin{itemize}
            \item Lending
            \item Hiring
            \item Law enforcement
        \end{itemize}
        
        \textbf{Types of Fairness:}
        \begin{itemize}
            \item \textbf{Individual Fairness:} Similar individuals receive similar outcomes.
            \item \textbf{Group Fairness:} Performance metrics should be similar across demographic groups.
        \end{itemize}
        
        \textbf{Example:} A fair credit scoring model should equitably predict loan eligibility across races and genders.
    \end{block}
\end{frame}

\begin{frame}[fragile]
    \frametitle{Ethical Considerations - Accountability}
    \begin{block}{3. Accountability in Machine Learning}
        \textbf{Definition:} Accountability emphasizes being answerable for ML model decisions. This includes:
        \begin{itemize}
            \item \textbf{Transparency:} Understanding how and why decisions are made.
            \item \textbf{Traceability:} Tracing decisions back to specific data.
            \item \textbf{Redress Mechanisms:} Clear channels for recourse when unjust outcomes occur.
        \end{itemize}
        
        \textbf{Example:} In healthcare, protocols must exist to address accountability for misdiagnoses caused by ML models.
    \end{block}
\end{frame}

\begin{frame}[fragile]
    \frametitle{Ethical Considerations - Key Takeaway}
    \begin{block}{Key Takeaway}
        Addressing ethical considerations in machine learning is a moral imperative. Future ML practitioners must advocate for ethical standards that ensure:
        \begin{itemize}
            \item Fairness
            \item Accountability
            \item Elimination of bias
        \end{itemize}
    \end{block}
    
    \begin{block}{Further Reflections}
        \begin{itemize}
            \item How can you incorporate these ethical considerations into your own projects?
            \item What frameworks or guidelines exist to help address these dilemmas?
        \end{itemize}
    \end{block}
    
    \begin{block}{Summary}
        Understanding and addressing bias, fairness, and accountability is vital for building ethical AI systems. Keep these considerations in mind for your final projects.
    \end{block}
\end{frame}

\begin{frame}[fragile]
    \frametitle{Final Project Review - Overview of Expectations}
    \begin{block}{Overview}
        The final project synthesizes concepts and skills learned throughout the course, reflecting industry standards in machine learning. Key expectations include:
    \end{block}
    \begin{enumerate}
        \item \textbf{Project Selection:} Relevant problem or dataset with industry significance.
        \item \textbf{Methodology:} Application of machine learning algorithms, clear definition of approach.
        \item \textbf{Implementation:} Use of programming languages and libraries, adherence to coding best practices.
        \item \textbf{Analysis \& Evaluation:} Thorough model analysis with metrics such as confusion matrices and ROC curves.
        \item \textbf{Ethical Considerations:} Addressing biases and ensuring fairness, accountability, and transparency.
        \item \textbf{Presentation:} Clear delivery of problem statement, methodology, results, and effective visuals.
    \end{enumerate}
\end{frame}

\begin{frame}[fragile]
    \frametitle{Final Project Review - Example Projects}
    \begin{block}{Industry-Relevant Project Examples}
        Here are some examples of projects that align with industry standards:
    \end{block}
    \begin{itemize}
        \item \textbf{Predictive Maintenance:} Predict failures using sensor data, incorporating time series analysis.
        \item \textbf{Customer Sentiment Analysis:} Analyze social media for sentiment using natural language processing (NLP).
        \item \textbf{Image Classification Task:} Implement a convolutional neural network (CNN) for known datasets like CIFAR-10 or MNIST.
    \end{itemize}
\end{frame}

\begin{frame}[fragile]
    \frametitle{Final Project Review - Implementation Example}
    \begin{block}{Implementation Snippet}
        Here is a snippet illustrating simple linear regression in Python using Scikit-learn:
    \end{block}
    \begin{lstlisting}[language=Python]
import pandas as pd
from sklearn.model_selection import train_test_split
from sklearn.linear_model import LinearRegression
from sklearn.metrics import mean_squared_error

# Load dataset
data = pd.read_csv('data.csv')
X = data[['feature1', 'feature2']] # predictors
y = data['target'] # response variable

# Split the dataset
X_train, X_test, y_train, y_test = train_test_split(X, y, test_size=0.2)

# Train the model
model = LinearRegression()
model.fit(X_train, y_train)

# Predictions
predictions = model.predict(X_test)

# Evaluation
mse = mean_squared_error(y_test, predictions)
print(f'Mean Squared Error: {mse}')
    \end{lstlisting}
\end{frame}

\begin{frame}[fragile]
    \frametitle{Study Strategies for Final Exam - Part 1}
    \begin{enumerate}
        \item \textbf{Active Learning Techniques:}
            \begin{itemize}
                \item \textbf{Concept Mapping:} Create visual representations for better understanding of relationships.
                \item \textbf{Practice Problems:} Regularly solve past exam questions to reinforce learning.
            \end{itemize}
        
        \item \textbf{Spaced Repetition:}
            \begin{itemize}
                \item Study in short sessions over time to enhance retention.
                \item \textbf{Example:} Study 1 hour daily instead of 6 hours before the exam.
            \end{itemize}
    \end{enumerate}
\end{frame}

\begin{frame}[fragile]
    \frametitle{Study Strategies for Final Exam - Part 2}
    \begin{enumerate}
        \setcounter{enumi}{3}
        \item \textbf{Study Groups:}
            \begin{itemize}
                \item Collaborate with peers for discussion and quizzing.
                \item \textbf{Tip:} Assign topics to each member for better understanding.
            \end{itemize}
        
        \item \textbf{Use of Mnemonics:}
            \begin{itemize}
                \item Create acronyms or phrases for complex terms.
                \item \textbf{Example:} "Please Excuse My Dear Aunt Sally" for order of operations.
            \end{itemize}
        
        \item \textbf{Resource Utilization:}
            \begin{itemize}
                \item Use textbooks, lecture notes, and online platforms for supplementary resources.
                \item \textbf{Example:} Use LeetCode to practice algorithm-related coding problems.
            \end{itemize}
    \end{enumerate}
\end{frame}

\begin{frame}[fragile]
    \frametitle{Study Strategies for Final Exam - Part 3}
    \begin{enumerate}
        \setcounter{enumi}{6}
        \item \textbf{Prioritize Understanding Over Memorization:}
            \begin{itemize}
                \item Grasp concepts thoroughly and be able to explain them.
                \item \textbf{Interactive Element:} Try teaching a concept to a peer.
            \end{itemize}
        
        \item \textbf{Create a Study Schedule:}
            \begin{itemize}
                \item Develop a timeline with specific topics and breaks included.
                \item \textbf{Example Schedule:}
                    \begin{itemize}
                        \item Week 1: Review Basics
                        \item Week 2: Focus on Advanced Topics
                        \item Week 3: Mock Exams
                    \end{itemize}
            \end{itemize}
        
        \item \textbf{Conclusion:}
            \begin{block}{}
                Integrate these strategies for enhanced understanding and retention.
                Tailor these methods to your personal learning style for optimal results!
            \end{block}
    \end{enumerate}
\end{frame}

\begin{frame}[fragile]
    \frametitle{Questions and Answers - Part 1}
    \textbf{Open Discussion on Final Exam Preparation}

    \begin{block}{Key Focus Areas for Discussion}
        \begin{enumerate}
            \item Understanding Exam Format
            \item Effective Study Techniques
            \item Reviewing Course Material
            \item Formulating Study Groups
            \item Addressing Exam Anxiety
        \end{enumerate}
    \end{block}
\end{frame}

\begin{frame}[fragile]
    \frametitle{Questions and Answers - Part 2}
    \textbf{Effective Study Techniques}
    
    \begin{itemize}
        \item \textbf{Active recall}:
            Test yourself rather than just reading notes.
        \item \textbf{Spaced repetition}:
            Distribute study sessions over time for better retention.
            \begin{itemize}
                \item Example: Revisit key topics every 1-2 days if studying for a week.
            \end{itemize}
    \end{itemize}
    
    \textbf{Reviewing Course Material}:
    \begin{itemize}
        \item Prioritize high-yield topics based on tests and syllabus.
        \item Create a topic checklist or summary sheet.
    \end{itemize}
\end{frame}

\begin{frame}[fragile]
    \frametitle{Questions and Answers - Part 3}
    \textbf{Example Q\&A Discussion Starters}
    
    \begin{itemize}
        \item \textbf{Q}: What are the best resources for understanding complex topics? \\
              \textbf{A}: Textbooks, online resources, peer discussions, and office hours.
        \item \textbf{Q}: How much time should I dedicate to each subject? \\
              \textbf{A}: Consider familiarity; allocate time based on exam topic weight.
        \item \textbf{Q}: How can I best utilize past exams or practice questions? \\
              \textbf{A}: Practice under timed conditions and review mistakes to improve.
    \end{itemize}

    \textbf{Key Points to Emphasize}:
    \begin{itemize}
        \item Engage actively.
        \item Collaborate with peers.
        \item Practice self-care.
    \end{itemize}
\end{frame}


\end{document}