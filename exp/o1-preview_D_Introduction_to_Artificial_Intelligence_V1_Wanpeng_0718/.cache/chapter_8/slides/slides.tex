\documentclass{beamer}

% Theme choice
\usetheme{Madrid} % You can change to e.g., Warsaw, Berlin, CambridgeUS, etc.

% Encoding and font
\usepackage[utf8]{inputenc}
\usepackage[T1]{fontenc}

% Graphics and tables
\usepackage{graphicx}
\usepackage{booktabs}

% Code listings
\usepackage{listings}
\lstset{
basicstyle=\ttfamily\small,
keywordstyle=\color{blue},
commentstyle=\color{gray},
stringstyle=\color{red},
breaklines=true,
frame=single
}

% Math packages
\usepackage{amsmath}
\usepackage{amssymb}

% Colors
\usepackage{xcolor}

% TikZ and PGFPlots
\usepackage{tikz}
\usepackage{pgfplots}
\pgfplotsset{compat=1.18}
\usetikzlibrary{positioning}

% Hyperlinks
\usepackage{hyperref}

% Title information
\title{Week 8: Data Integrity and Algorithmic Bias}
\author{Your Name}
\institute{Your Institution}
\date{\today}

\begin{document}

\frame{\titlepage}

\begin{frame}[fragile]
    \frametitle{Introduction to Data Integrity and Algorithmic Bias}
    \begin{block}{Overview}
        This slide introduces two critical concepts in the realm of Artificial Intelligence (AI): \textbf{Data Integrity} and \textbf{Algorithmic Bias}. Understanding these concepts is fundamental for anyone working with AI, as they directly impact the reliability, fairness, and effectiveness of AI systems.
    \end{block}
\end{frame}

\begin{frame}[fragile]
    \frametitle{Data Integrity}
    \begin{itemize}
        \item \textbf{Definition}: Data integrity refers to the accuracy, consistency, and reliability of data throughout its lifecycle. It ensures that data remains unchanged during transmission and storage, accurately reflecting the real-world constructs it represents.
        
        \item \textbf{Importance}: High data integrity ensures that AI models make decisions based on reliable information, which translates to better performance and trustworthiness. Insecure or corrupted data can lead to faulty algorithms and incorrect conclusions.
        
        \item \textbf{Example}: Consider a healthcare AI that predicts patient outcomes based on historical data. If the data contains errors (e.g., incorrect patient records), the AI's predictions may be inaccurate, potentially leading to harmful decisions.
    \end{itemize}
\end{frame}

\begin{frame}[fragile]
    \frametitle{Algorithmic Bias}
    \begin{itemize}
        \item \textbf{Definition}: Algorithmic bias refers to systematic and unfair discrimination that occurs when AI algorithms produce prejudiced results due to flawed data or design. This bias can arise from various sources, including biased training data and design choices.
        
        \item \textbf{Implications}: Algorithmic bias can perpetuate and amplify existing disparities in real life, leading to unfair treatment of certain groups based on race, gender, age, or other characteristics.
        
        \item \textbf{Example}: A hiring algorithm trained on historical employment data may learn biases present in that data against specific demographics, leading to unfair hiring practices that disadvantage those groups.
    \end{itemize}
\end{frame}

\begin{frame}[fragile]
    \frametitle{Interconnections and Conclusions}
    \begin{block}{Key Points}
        \begin{itemize}
            \item Data integrity and algorithmic bias are interconnected. Poor data integrity can exacerbate biases, while reliable data is crucial for fair algorithms.
            \item The implications of neglecting data integrity can be damaging, affecting business decisions, social justice, and individual lives. Addressing algorithmic bias is essential for building equitable AI systems.
        \end{itemize}
    \end{block}
    
    \begin{block}{Conclusion}
        In summary, maintaining data integrity and addressing algorithmic bias is essential for developing trustworthy AI systems. Awareness of these issues enables practitioners to create more accurate, fair, and effective AI solutions that benefit all users.
    \end{block}
\end{frame}

\begin{frame}[fragile]
    \frametitle{Suggested Further Reading}
    \begin{itemize}
        \item "Weapons of Math Destruction" by Cathy O'Neil
        \item "Data Integrity in Data Lakes: A Roadmap" (Research Paper)
    \end{itemize}
\end{frame}

\begin{frame}[fragile]
    \frametitle{Understanding Data Integrity - Definition}
    \begin{block}{Definition of Data Integrity}
        Data Integrity refers to the accuracy, consistency, and reliability of data throughout its lifecycle. This includes how data is:
        \begin{itemize}
            \item Created
            \item Stored
            \item Maintained
            \item Modified
        \end{itemize}
        When data integrity is intact, the information remains valid and trustworthy.
    \end{block}
\end{frame}

\begin{frame}[fragile]
    \frametitle{Understanding Data Integrity - Significance}
    \begin{block}{Significance of Data Integrity in AI Systems}
        \begin{itemize}
            \item \textbf{Foundation for Learning:} AI systems depend on the quality of data. High integrity data helps models learn correct patterns and make accurate predictions.
            \item \textbf{Decision Making:} Reliable data leads to better decision-making. Flawed data can result in misguided or harmful decisions.
            \item \textbf{Trustworthiness:} Consistent and accurate data is essential for user acceptance of AI systems. Poor data integrity fosters skepticism among stakeholders.
        \end{itemize}
    \end{block}
\end{frame}

\begin{frame}[fragile]
    \frametitle{Understanding Data Integrity - Impact on Performance}
    \begin{block}{Impact on Algorithm Performance}
        \begin{itemize}
            \item \textbf{Influence on Outcomes:} Algorithms trained on low integrity data can produce biased or inaccurate results. For example:
            \begin{itemize}
                \item Bias can occur if underrepresented groups are omitted from training data.
            \end{itemize}
            \item \textbf{Performance Metrics:} Key metrics like accuracy, precision, recall, and F1 score can suffer due to flawed data, indicating suboptimal algorithm performance.
        \end{itemize}
    \end{block}
\end{frame}

\begin{frame}[fragile]
    \frametitle{Understanding Data Integrity - Examples}
    \begin{block}{Examples to Illustrate}
        \begin{enumerate}
            \item \textbf{Healthcare AI:} Accurate patient data is crucial to predict outcomes. Missing or erroneous data can lead to severe clinical misjudgments.
            \item \textbf{Financial Algorithms:} In fraud detection, inaccurate transaction data can hinder the system's ability to identify fraudulent activities.
        \end{enumerate}
    \end{block}
\end{frame}

\begin{frame}[fragile]
    \frametitle{Understanding Data Integrity - Key Points}
    \begin{block}{Key Points to Emphasize}
        \begin{itemize}
            \item High data integrity is crucial for effective AI systems.
            \item Data flaws can lead to poor performance and biased outcomes.
            \item Regular audits and robust data governance are essential for maintaining integrity.
        \end{itemize}
    \end{block}
\end{frame}

\begin{frame}[fragile]
    \frametitle{Understanding Data Integrity - Next Steps}
    \begin{block}{Remember}
        To ensure data integrity, organizations should implement:
        \begin{itemize}
            \item Robust data governance frameworks
            \item Defined data management processes
            \item Validation techniques
        \end{itemize}
    \end{block}
    With a solid understanding of data integrity, we can now explore the various types of data integrity issues in the next slide.
\end{frame}

\begin{frame}[fragile]
    \frametitle{Types of Data Integrity Issues}
    
    \begin{block}{Overview}
        Data integrity is crucial for ensuring the reliability and accuracy of data used in artificial intelligence systems. 
        We will explore four main types of data integrity issues:
        \begin{enumerate}
            \item Accuracy
            \item Consistency
            \item Completeness
            \item Timeliness
        \end{enumerate}
    \end{block}
\end{frame}

\begin{frame}[fragile]
    \frametitle{Types of Data Integrity Issues - Accuracy}
    
    \begin{block}{Definition}
        Accuracy refers to how closely data values align with the true values or real-world phenomena they represent.
    \end{block}
    
    \begin{itemize}
        \item Inaccurate data can lead to poor decision-making and skewed analysis results.
        \item Common sources of inaccuracies include human errors, outdated information, and sensor malfunctions.
    \end{itemize}
    
    \begin{block}{Example}
        Consider a dataset that records the temperatures of various cities. If the temperature for New York City is recorded as 150°F, this data point is inaccurate and would mislead analyses on weather trends.
    \end{block}
\end{frame}

\begin{frame}[fragile]
    \frametitle{Types of Data Integrity Issues - Consistency}
    
    \begin{block}{Definition}
        Consistency refers to the uniformity of data across different datasets or within the same dataset over time.
    \end{block}
    
    \begin{itemize}
        \item Inconsistent data can arise when data is entered or formatted differently in various places.
        \item It can result in conflicting information, impacting reliability and user trust.
    \end{itemize}
    
    \begin{block}{Example}
        If one dataset indicates that a customer is from "New York" while another says "NY," these inconsistencies can create confusion in customer analysis processes.
    \end{block}
\end{frame}

\begin{frame}[fragile]
    \frametitle{Types of Data Integrity Issues - Completeness and Timeliness}
    
    \begin{block}{Completeness}
        Completeness measures whether all required data is present in a dataset; it evaluates whether the dataset is filled with all necessary entries.
        \begin{itemize}
            \item Missing data can lead to incomplete analyses, affecting the outcomes of machine learning models.
            \item Completeness is often assessed by identifying missing fields or records.
        \end{itemize}
        \begin{block}{Example}
            In a healthcare dataset, if patient medical histories are only partially recorded (e.g., missing allergy information), it can lead to ineffective treatment recommendations based on incomplete information.
        \end{block}
    \end{block}
    
    \begin{block}{Timeliness}
        Timeliness refers to the relevance of the data in relation to the current context or period. Data must be up-to-date to be valuable.
        \begin{itemize}
            \item Stale data can lead to erroneous conclusions or ineffective actions, especially in fast-moving scenarios like finance or social media.
            \item Regular updates and audits are essential to maintain timeliness.
        \end{itemize}
        \begin{block}{Example}
            If a dataset containing stock prices is updated only once a week, it may not accurately reflect current market conditions, leading to poor investment decisions.
        \end{block}
    \end{block}
\end{frame}

\begin{frame}[fragile]
    \frametitle{Types of Data Integrity Issues - Summary and Wrap-Up}
    
    \begin{block}{Summary}
        Addressing these data integrity issues—accuracy, consistency, completeness, and timeliness—is essential for maintaining the trustworthiness of data used in AI systems. 
        By ensuring robust data integrity, organizations can enhance the performance and reliability of their algorithms, ultimately achieving better outcomes.
    \end{block}

    \begin{block}{Wrap-Up}
        As we transition to the next slide, we will discuss algorithmic bias and how it can be influenced by the integrity of the data used. 
        Remember, even the best algorithms can achieve suboptimal performance if the foundational data is flawed.
    \end{block}
\end{frame}

\begin{frame}[fragile]
    \frametitle{Introduction to Algorithmic Bias - Definition}
    \begin{block}{Definition of Algorithmic Bias}
        Algorithmic bias refers to systematic and unfair discrimination embedded within algorithm-driven decisions. This occurs when algorithms produce results that are skewed due to the input data or the design of the algorithm itself. This bias can lead to unfair treatment of individuals or groups, particularly those from marginalized backgrounds.
    \end{block}
\end{frame}

\begin{frame}[fragile]
    \frametitle{Introduction to Algorithmic Bias - Manifestation}
    \begin{block}{How Algorithmic Bias Manifests}
        \begin{enumerate}
            \item \textbf{Data Bias:}
            \begin{itemize}
                \item Example: AI system for hiring trained on historical data may favor male candidates over qualified female candidates.
            \end{itemize}
            
            \item \textbf{Feature Selection Bias:}
            \begin{itemize}
                \item Illustration: A loan approval algorithm using income and zip code may discriminate based on geographic socio-economic status.
            \end{itemize}
            
            \item \textbf{Interpretation Bias:}
            \begin{itemize}
                \item Example: Image recognition systems may misinterpret images from particular racial backgrounds due to lack of diverse training data.
            \end{itemize}
            
            \item \textbf{Model Design Choices:}
            \begin{itemize}
                \item Key Point: Developers' choices, such as prioritizing accuracy over fairness, can lead to biased outcomes in important areas like healthcare.
            \end{itemize}
        \end{enumerate}
    \end{block}
\end{frame}

\begin{frame}[fragile]
    \frametitle{Introduction to Algorithmic Bias - Key Points and Conclusion}
    \begin{block}{Key Points to Emphasize}
        \begin{itemize}
            \item Algorithmic bias affects individual outcomes and perpetuates larger systematic inequalities in society.
            \item Understanding algorithmic bias is essential for developers and stakeholders to ensure fair AI implementation.
        \end{itemize}
    \end{block}
    
    \begin{block}{Conclusion}
        Addressing algorithmic bias requires concerted efforts in recognizing biases within data, re-evaluating design choices, and implementing strategies for bias mitigation. Awareness leads to more ethical and equitable AI systems.
    \end{block}
    
    \begin{block}{Further Investigation}
        In the following slides, we will explore the sources of algorithmic bias, equipping you with knowledge to identify and combat these issues in your own work with AI technologies.
    \end{block}
\end{frame}

\begin{frame}[fragile]
    \frametitle{Sources of Algorithmic Bias - Introduction}
    \begin{block}{Definition}
    Algorithmic bias refers to the systematic favoritism or discrimination that can occur in AI-driven algorithms due to various factors.
    \end{block}
    \begin{itemize}
        \item Bias originates not just from the algorithms themselves, but also from the data and design choices.
        \item Identifying sources of bias is crucial for developing fair and reliable algorithms.
    \end{itemize}
\end{frame}

\begin{frame}[fragile]
    \frametitle{Sources of Algorithmic Bias - Data Selection}
    \begin{block}{Definition}
    Bias can originate from the datasets used to train algorithms.
    \end{block}
    \begin{itemize}
        \item \textbf{Example:} A facial recognition system trained mostly on lighter-skinned individuals may misidentify darker-skinned individuals.
        \item \textbf{Key Points:}
        \begin{itemize}
            \item \textbf{Sample Bias:} Underrepresented or overrepresented groups in training data can lead models to favor those groups.
            \item \textbf{Historical Data Bias:} Historical data reflecting past prejudices perpetuates existing inequalities.
        \end{itemize}
    \end{itemize}
\end{frame}

\begin{frame}[fragile]
    \frametitle{Sources of Algorithmic Bias - Cultural Assumptions and Model Design}
    \begin{itemize}
        \item \textbf{Cultural Assumptions:}
        \begin{block}{Definition}
        Cultural norms and values can skew algorithm interpretations.
        \end{block}
        \begin{itemize}
            \item \textbf{Example:} A language processing algorithm may favor idioms from one culture, leading to misunderstandings.
            \item \textbf{Key Points:}
            \begin{itemize}
                \item \textbf{Normative Bias:} Assumptions about what is “normal” can exclude alternative perspectives.
                \item \textbf{Stereotyping:} Algorithms may reinforce cultural stereotypes through biased data interpretations.
            \end{itemize}
        \end{itemize}
        
        \item \textbf{Model Design Choices:}
        \begin{block}{Definition}
        Model formulation and parameter choices can introduce bias.
        \end{block}
        \begin{itemize}
            \item \textbf{Example:} Choosing between logistic regression and deep learning can affect model bias levels.
            \item \textbf{Key Points:}
            \begin{itemize}
                \item \textbf{Feature Engineering:} Decisions on feature inclusion/exclusion can alter outcomes significantly.
                \item \textbf{Optimization Goals:} Prioritizing accuracy may compromise fairness.
            \end{itemize}
        \end{itemize}
    \end{itemize}
\end{frame}

\begin{frame}[fragile]
    \frametitle{Case Studies: Data Integrity and Algorithmic Bias}
    
    \begin{block}{Introduction to Data Integrity and Algorithmic Bias}
        \begin{itemize}
            \item \textbf{Data Integrity}: Accuracy, consistency, and reliability of data throughout its lifecycle. Poor integrity leads to erroneous outcomes.
            \item \textbf{Algorithmic Bias}: Occurs when algorithms yield systematically prejudiced results due to flawed assumptions in the machine learning process, especially in training data.
        \end{itemize}
    \end{block}
\end{frame}

\begin{frame}[fragile]
    \frametitle{Case Study 1: Criminal Justice - Compas Algorithm}
    
    \begin{itemize}
        \item \textbf{Overview}: Used to assess risk of reoffending in jurisdictions.
        \item \textbf{Issues}: Investigation revealed bias favoring white defendants over black.
        \item \textbf{Consequences}: Misleading risk scores resulted in harsher sentences for minorities, raising ethical concerns about fairness.
    \end{itemize}
    
    \begin{block}{Key Point}
        Flawed data reflecting historical biases perpetuates systemic inequalities.
    \end{block}
\end{frame}

\begin{frame}[fragile]
    \frametitle{Case Study 2: Healthcare - IBM Watson for Oncology}
    
    \begin{itemize}
        \item \textbf{Overview}: Provided treatment recommendations based on clinical data.
        \item \textbf{Issues}: Trained on limited historical data, biasing recommendations towards certain demographics.
        \item \textbf{Consequences}: Variability in care; some patients received suboptimal treatment due to biased data inputs.
    \end{itemize}
    
    \begin{block}{Key Point}
        Insufficient data diversity can lead to biased outcomes, affecting patient health equity.
    \end{block}
\end{frame}

\begin{frame}[fragile]
    \frametitle{Case Study 3: Employment - Resume Screening Systems}
    
    \begin{itemize}
        \item \textbf{Overview}: AI used for sorting resumes and selecting candidates.
        \item \textbf{Issues}: Historical hiring data used, encoding biases (e.g., gender, ethnicity).
        \item \textbf{Consequences}: Women and minority candidates rated lower, limiting opportunities and reinforcing diversity issues.
    \end{itemize}
    
    \begin{block}{Key Point}
        Algorithms learning from historical data may perpetuate existing biases rather than eliminate them.
    \end{block}
\end{frame}

\begin{frame}[fragile]
    \frametitle{Conclusion and Key Takeaways}
    
    \begin{block}{Real-World Implications}
        These case studies highlight the significant impacts of data integrity and algorithmic bias across various sectors like justice, healthcare, and employment.
    \end{block}
    
    \begin{block}{Call for Action}
        Evaluate data sources and implement fairness checks to mitigate bias and ensure equitable outcomes.
    \end{block}

    \begin{itemize}
        \item \textbf{Awareness}: Acknowledge potential bias in datasets and design.
        \item \textbf{Strategy}: Promote data cleansing, diverse data sourcing, and ongoing assessments.
        \item \textbf{Ethics}: Foster accountability in AI deployment to protect against disenfranchisement.
    \end{itemize}
\end{frame}

\begin{frame}[fragile]
    \frametitle{Consequences of Poor Data Practices}
    \begin{block}{Understanding Data Integrity and Algorithmic Bias}
        Data integrity refers to the accuracy, consistency, and reliability of data throughout its lifecycle. Algorithmic bias arises when algorithms produce systematically prejudiced results due to incorrect or unrepresentative data input.
    \end{block}
\end{frame}

\begin{frame}[fragile]
    \frametitle{Ethical Implications}
    \begin{enumerate}
        \item \textbf{Discrimination:}
        \begin{itemize}
            \item Algorithms trained on biased data can lead to unfair treatment of certain groups (e.g., racial, gender).
            \item \textit{Example:} A hiring algorithm may prioritize candidates from a particular demographic while ignoring others.
        \end{itemize}

        \item \textbf{Lack of Accountability:}
        \begin{itemize}
            \item Poor data practices can create a “black box” scenario where decisions are made without transparency, making it difficult to identify who is responsible for harmful outputs.
        \end{itemize}

        \item \textbf{Erosion of Trust:}
        \begin{itemize}
            \item When data integrity is compromised, public trust in institutions erodes, leading to societal skepticism.
        \end{itemize}
    \end{enumerate}
\end{frame}

\begin{frame}[fragile]
    \frametitle{Societal Impacts}
    \begin{enumerate}
        \item \textbf{Misinformation Spread:}
        \begin{itemize}
            \item Data that is not accurate can lead to the dissemination of false information, exacerbating social divisions.
            \item \textit{Illustration:} Misinformation regarding COVID-19 treatment options can have dire public health consequences.
        \end{itemize}

        \item \textbf{Resource Misallocation:}
        \begin{itemize}
            \item Poor data can misdirect resources to ineffective programs. For instance, in healthcare, treatments may be prioritized based on flawed data analysis.
        \end{itemize}

        \item \textbf{Legal Consequences:}
        \begin{itemize}
            \item Organizations may face lawsuits due to unfair algorithms causing harm, leading to significant financial and reputational damage.
            \item \textit{Example:} Legal cases against companies due to biased loan approval systems that discriminate against minorities.
        \end{itemize}
    \end{enumerate}
\end{frame}

\begin{frame}[fragile]
    \frametitle{Key Points and Conclusion}
    \begin{enumerate}
        \item \textbf{Data Quality Matters:} The foundation of robust AI and decision-making systems lies in high-quality, representative data.
        
        \item \textbf{Awareness is Crucial:} Stakeholders must acknowledge and actively mitigate bias and integrity issues—it's not just a technical challenge but an ethical one.
        
        \item \textbf{Interdisciplinary Approach:} Addressing these issues requires collaboration among technologists, ethicists, and community representatives to create fair algorithms.
    \end{enumerate}
    
    \textbf{Conclusion:} The consequences of poor data practices are profound. Understanding and mitigating the ethical implications and societal impacts is essential for responsible data use, ensuring technology serves all sectors of society equitably.
\end{frame}

\begin{frame}[fragile]
    \frametitle{Best Practices for Ensuring Data Integrity - Introduction}
    \begin{block}{Introduction to Data Integrity}
        Data integrity refers to the accuracy, consistency, and reliability of data throughout its lifecycle. In AI systems, maintaining data integrity is crucial for ensuring valid results, decision-making processes, and ethical standards.
    \end{block}
\end{frame}

\begin{frame}[fragile]
    \frametitle{Best Practices for Ensuring Data Integrity - Strategies}
    \begin{enumerate}
        \item \textbf{Data Governance Framework}
        \begin{itemize}
            \item Implement a robust data governance strategy defining roles, responsibilities, and processes for data management.
            \item Example: Establishing a Data Steward role to oversee data quality and compliance.
        \end{itemize}
        
        \item \textbf{Data Validation Techniques}
        \begin{itemize}
            \item Use validation techniques to check for data entry errors.
            \item Techniques include:
                \begin{itemize}
                    \item \textbf{Range Checks}: Ensure data falls within a predefined range (e.g., age of a user must be between 0-120).
                    \item \textbf{Format Checks}: Verify that data adheres to specified formats (e.g., email addresses must include "@" symbol).
                \end{itemize}
        \end{itemize}
        
        \item \textbf{Regular Audits and Monitoring}
        \begin{itemize}
            \item Conduct regular audits of datasets to assess quality and discover inconsistencies.
            \item Utilize scripts for automated monitoring (e.g., Python scripts flagging anomalies).
        \end{itemize}
    \end{enumerate}
\end{frame}

\begin{frame}[fragile]
    \frametitle{Best Practices for Ensuring Data Integrity - Continued}
    \begin{enumerate}[resume]
        \item \textbf{Version Control for Datasets}
        \begin{itemize}
            \item Implement version control systems for datasets to track changes over time.
            \item Example: Tools like Git or DVC (Data Version Control) can help keep track of historical data.
        \end{itemize}
        
        \item \textbf{Data Encryption and Access Control}
        \begin{itemize}
            \item Encrypt sensitive data to prevent unauthorized access.
            \item Implement strict access control measures ensuring only authorized personnel can modify data. 
            \item Example: Role-based access control (RBAC) allows different users varying levels of access based on roles.
        \end{itemize}

        \item \textbf{Data Annotation and Documentation}
        \begin{itemize}
            \item Maintain clear documentation for datasets, including descriptions, sources, and changes made over time.
        \end{itemize}

    \end{enumerate}
\end{frame}

\begin{frame}[fragile]
    \frametitle{Best Practices for Ensuring Data Integrity - Training and Example}
    \begin{enumerate}[resume]
        \item \textbf{Engage in Continuous Training}
        \begin{itemize}
            \item Train staff involved in data handling on best practices for maintaining data integrity.
            \item Emphasize ethical implications of data handling and its effect on algorithmic bias.
        \end{itemize}
    \end{enumerate}

    \begin{block}{Key Points to Emphasize}
        \begin{itemize}
            \item Data integrity is essential for ethical AI practices and successful outcomes.
            \item Proactive measures like validation and version control are critical in maintaining data accuracy.
            \item Continuous monitoring and training create an environment prioritizing integrity and accountability.
        \end{itemize}
    \end{block}
    
    \begin{block}{Example Scenario}
        Consider an AI model predicting loan eligibility. If applicant data lacks integrity, it can lead to unfair decisions. Applying best practices, such as data validation and audits, can maintain high data integrity, resulting in fair outcomes.
    \end{block}
\end{frame}

\begin{frame}[fragile]
    \frametitle{Mitigating Algorithmic Bias - Introduction}
    \begin{block}{Introduction to Algorithmic Bias}
        Algorithmic bias refers to systematic and unfair discrimination that occurs when an algorithm produces prejudiced results due to flawed assumptions in the machine learning process. 
        This can perpetuate stereotypes and lead to inequitable outcomes in critical areas like hiring, law enforcement, and lending.
    \end{block}
\end{frame}

\begin{frame}[fragile]
    \frametitle{Mitigating Algorithmic Bias - Methods}
    \begin{block}{Methods to Identify and Reduce Bias}
        \begin{enumerate}
            \item \textbf{Diverse Datasets}
                \begin{itemize}
                    \item Ensuring datasets reflect a wide variety of demographics can minimize bias.
                    \item \textit{Example}: Training on a dataset predominantly of one ethnicity may not perform well for others.
                    \item \textbf{Key Point}: Regularly assess and update datasets to include diverse examples.
                \end{itemize}
                
            \item \textbf{Fairness Auditing}
                \begin{itemize}
                    \item Conduct audits to examine algorithms for biased outcomes.
                    \item \textit{Example}: A fairness audit of a credit scoring algorithm may show inaccuracies for low-income applicants.
                    \item \textbf{Key Metrics for Auditing}:
                        \begin{itemize}
                            \item Statistical Parity
                            \item Equal Opportunity
                        \end{itemize}
                \end{itemize}
        \end{enumerate}
    \end{block}
\end{frame}

\begin{frame}[fragile]
    \frametitle{Mitigating Algorithmic Bias - Techniques}
    \begin{block}{Bias Mitigation Techniques}
        Methods to correct bias during model training and evaluation:
        \begin{itemize}
            \item \textbf{Pre-processing}: Modify training data (e.g., oversampling underrepresented groups).
            \item \textbf{In-processing}: Incorporate fairness constraints during model training (e.g., adjust loss function).
            \item \textbf{Post-processing}: Adjust outputs after predictions for equitable outcomes.
        \end{itemize}
        \textit{Example}: Using a modified loss function that penalizes biases can minimize disparities in outputs.
    \end{block}

    \begin{block}{Continuous Monitoring}
        \begin{itemize}
            \item Regular monitoring of algorithms post-deployment is critical.
            \item Create feedback loops for users to report biased results. 
            \item Involve domain experts for ongoing insights.
        \end{itemize}
    \end{block}
\end{frame}

\begin{frame}[fragile]
    \frametitle{Mitigating Algorithmic Bias - Conclusion}
    \begin{block}{Conclusion}
        By employing diverse datasets, conducting fairness audits, implementing bias mitigation techniques, 
        and ensuring continuous monitoring, organizations can significantly reduce algorithmic bias. 
        This fosters fairer and more inclusive AI systems, essential for maximizing AI effectiveness in a diverse society.
    \end{block}

    \begin{block}{References for Further Reading}
        \begin{itemize}
            \item Barocas, S., Hardt, M., \& Narayanan, A. (2019). Fairness and Machine Learning.
            \item Mehrabi, N., Morstatter, F., Saxena, N., et al. (2019). A Survey on Bias and Fairness in Machine Learning.
        \end{itemize}
    \end{block}
\end{frame}

\begin{frame}[fragile]
    \frametitle{Conclusion and Future Perspectives - Key Points Summary}

    \begin{enumerate}
        \item \textbf{Data Integrity}:
            \begin{itemize}
                \item Defined as the accuracy, consistency, and reliability of data throughout its lifecycle.
                \item Ensures that data remains unchanged and trustworthy during collection, processing, and storage.
            \end{itemize}
        \item \textbf{Algorithmic Bias}:
            \begin{itemize}
                \item Arises from unfair datasets or flawed model assumptions.
                \item Can lead to unjust decisions in areas like hiring, lending, and law enforcement.
            \end{itemize}
        \item \textbf{Mitigating Bias}:
            \begin{itemize}
                \item Methods include using diverse datasets, conducting fairness audits, and implementing corrective algorithms.
                \item Continuous monitoring is essential for minimizing bias.
            \end{itemize}
    \end{enumerate}
\end{frame}

\begin{frame}[fragile]
    \frametitle{Conclusion and Future Perspectives - Importance of Continuous Evaluation}

    \begin{itemize}
        \item Data and algorithms are dynamic, evolving with new information and societal norms; therefore, practices should adapt continually.
        \item Regular evaluations help identify emerging biases or integrity issues for timely interventions.
    \end{itemize}

    \begin{block}{Illustration}
        Consider the healthcare sector: As medical research advances, patient demographics change, and public health needs evolve, AI-based diagnostic tools must be regularly updated to ensure relevance.
    \end{block}
\end{frame}

\begin{frame}[fragile]
    \frametitle{Conclusion and Future Perspectives - Future Perspectives and Final Thoughts}

    \begin{block}{Future Perspectives}
        \begin{itemize}
            \item \textbf{Data Governance}: Establish a robust framework for accountability in data management.
            \item \textbf{Ongoing Education}: Stakeholders should be educated on the impacts of bias and data integrity.
        \end{itemize}
    \end{block}

    \begin{block}{Final Thoughts}
        As technology advances, the landscapes of data collection and algorithmic deployment will also change. An adaptive approach to data integrity and bias mitigation is critical.
    \end{block}

    \begin{quote}
        ``A data-driven future demands that our approaches to integrity and fairness evolve alongside our technologies.''
    \end{quote}
\end{frame}


\end{document}