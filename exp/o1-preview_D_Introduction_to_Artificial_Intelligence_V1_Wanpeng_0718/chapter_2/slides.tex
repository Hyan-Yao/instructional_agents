\documentclass{beamer}

% Theme choice
\usetheme{Madrid} % You can change to e.g., Warsaw, Berlin, CambridgeUS, etc.

% Encoding and font
\usepackage[utf8]{inputenc}
\usepackage[T1]{fontenc}

% Graphics and tables
\usepackage{graphicx}
\usepackage{booktabs}

% Code listings
\usepackage{listings}
\lstset{
    basicstyle=\ttfamily\small,
    keywordstyle=\color{blue},
    commentstyle=\color{gray},
    stringstyle=\color{red},
    breaklines=true,
    frame=single
}

% Math packages
\usepackage{amsmath}
\usepackage{amssymb}

% Colors
\usepackage{xcolor}

% TikZ and PGFPlots
\usepackage{tikz}
\usepackage{pgfplots}
\pgfplotsset{compat=1.18}
\usetikzlibrary{positioning}

% Hyperlinks
\usepackage{hyperref}

% Title information
\title{Week 2: Core Concepts of AI: Machine Learning}
\author{Your Name}
\institute{Your Institution}
\date{\today}

\begin{document}

\frame{\titlepage}

\begin{frame}[fragile]
    \frametitle{Introduction to Machine Learning}
    \begin{block}{What is Machine Learning?}
        Machine Learning (ML) is a subset of artificial intelligence (AI) that empowers systems to learn and improve from experiences without being explicitly programmed.
    \end{block}
    \begin{block}{Significance of Machine Learning in AI}
        ML involves the development of algorithms that allow computers to recognize patterns and make decisions based on data.
    \end{block}
\end{frame}

\begin{frame}[fragile]
    \frametitle{Significance of Machine Learning in AI - Key Benefits}
    \begin{enumerate}
        \item \textbf{Automation of Tasks:}
            \begin{itemize}
                \item Improves efficiency by automating repetitive tasks (e.g., email filtering).
            \end{itemize}
        \item \textbf{Data-Driven Insights:}
            \begin{itemize}
                \item Extracts valuable insights from large data sets (e.g., predicting diagnostics in healthcare).
            \end{itemize}
        \item \textbf{Personalization:}
            \begin{itemize}
                \item Enhances user experience through tailored content recommendations (e.g., Netflix).
            \end{itemize}
        \item \textbf{Adaptive Learning:}
            \begin{itemize}
                \item Algorithms improve as more data is collected (e.g., self-driving cars).
            \end{itemize}
        \item \textbf{Wide Range of Applications:}
            \begin{itemize}
                \item Applicable in diverse fields such as speech recognition and financial forecasting.
            \end{itemize}
    \end{enumerate}
\end{frame}

\begin{frame}[fragile]
    \frametitle{Key Points and Formulas in Machine Learning}
    \begin{block}{Key Points}
        \begin{itemize}
            \item \textbf{Learning from Data:} Primary capability of ML is to learn patterns and make predictions.
            \item \textbf{Types of Learning:}
                \begin{itemize}
                    \item Supervised Learning
                    \item Unsupervised Learning
                    \item Reinforcement Learning
                \end{itemize}
            \item \textbf{Model Evaluation:} Assess accuracy and precision.
        \end{itemize}
    \end{block}
    \begin{block}{Basic Formula for Linear Regression:}
        \begin{equation}
            y = mx + b
        \end{equation}
        Where \(y\) is the predicted output, \(m\) is the slope, \(x\) is the input feature, and \(b\) is the intercept.
    \end{block}
\end{frame}

\begin{frame}[fragile]
    \frametitle{Conclusion}
    Machine Learning is a transformative force in AI. Its ability to adapt and learn from data leads to better, personalized experiences and efficiency across various sectors. Understanding ML principles is essential for grasping the broader landscape of artificial intelligence.
\end{frame}

\begin{frame}[fragile]
    \frametitle{History of Machine Learning - Introduction}
    \begin{block}{Overview}
        Machine Learning (ML) is a pivotal area within Artificial Intelligence (AI). Its foundations stretch back several decades, reflecting an evolution marked by both theoretical advancements and practical applications. 
    \end{block}
    \begin{itemize}
        \item Understand the history of ML to contextualize its current state and future potential.
    \end{itemize}
\end{frame}

\begin{frame}[fragile]
    \frametitle{History of Machine Learning - Key Milestones}
    \begin{enumerate}
        \item \textbf{1950s: The Birth of AI and Early Learning Algorithms}
            \begin{itemize}
                \item Alan Turing proposed the Turing Test (1950).
                \item Arthur Samuel coined the term "Machine Learning" (1959) while developing a checkers program.
            \end{itemize}
        \item \textbf{1960s: Symbolic AI and Early Neural Networks}
            \begin{itemize}
                \item Focus on symbolic AI and rule-based systems.
                \item Development of the Perceptron Algorithm by Frank Rosenblatt (1958).
            \end{itemize}
        \item \textbf{1970s: The AI Winter}
            \begin{itemize}
                \item Diminished interest due to unmet expectations and limited computing power.
            \end{itemize}
    \end{enumerate}
\end{frame}

\begin{frame}[fragile]
    \frametitle{History of Machine Learning - Continued Milestones}
    \begin{enumerate}[resume]
        \item \textbf{1980s: Renewed Interest and Advancements}
            \begin{itemize}
                \item Resurgence sparked by the backpropagation algorithm (1986) by Geoffrey Hinton.
            \end{itemize}
        \item \textbf{1990s: Expansion to Various Domains}
            \begin{itemize}
                \item Applications in finance, healthcare, and marketing.
                \item Emergence of Support Vector Machines (SVMs) and decision trees.
            \end{itemize}
        \item \textbf{2000s: Big Data and Algorithmic Refinements}
            \begin{itemize}
                \item Rise of the internet and vast data generation reshaped ML.
                \item Introduction of ensemble methods and development of tools like Weka.
            \end{itemize}
        \item \textbf{2010s: Deep Learning Revolution}
            \begin{itemize}
                \item Significant advancements in computer vision and natural language processing.
            \end{itemize}
        \item \textbf{2020s and Beyond: Democratization of AI}
            \begin{itemize}
                \item Focus on ethical AI and open-source frameworks like TensorFlow.
            \end{itemize}
    \end{enumerate}
\end{frame}

\begin{frame}[fragile]
    \frametitle{Core Concepts of Machine Learning - Overview}
    Machine Learning (ML) is a subset of Artificial Intelligence that enables systems to learn from data to improve performance on a specific task over time. 
    The three primary types of learning in machine learning are:
    \begin{itemize}
        \item Supervised Learning
        \item Unsupervised Learning
        \item Reinforcement Learning
    \end{itemize}
    Each approach has its unique use cases and techniques.
\end{frame}

\begin{frame}[fragile]
    \frametitle{Core Concepts of Machine Learning - Supervised Learning}
    \textbf{Supervised Learning}
    \begin{itemize}
        \item \textbf{Definition}: In supervised learning, the algorithm is trained on a labeled dataset, meaning the input data is paired with the correct output.
        \item \textbf{Key Points}:
        \begin{itemize}
            \item Utilizes a known output to guide the learning process.
            \item The goal is to construct a model that can predict the output for new data.
        \end{itemize}
        \item \textbf{Examples}:
        \begin{itemize}
            \item Classification: Classifying emails as 'spam' or 'not spam'.
            \item Regression: Predicting house prices based on features like size and location.
        \end{itemize}
    \end{itemize}
\end{frame}

\begin{frame}[fragile]
    \frametitle{Core Concepts of Machine Learning - Supervised Learning Examples}
    \textbf{Examples in Detail}
    \begin{itemize}
        \item **Classification**: Classifying emails as 'spam' or 'not spam'.
        \begin{itemize}
            \item **Diagram**: A set of labeled emails with inputs (features) such as subject line, sender, and content.
        \end{itemize}
        \item **Regression**: Predicting house prices based on features like size and location.
        \begin{equation}
        Y = aX + b
        \end{equation}
        Where \(Y\) is the output (price), \(X\) are input features (size, location), \(a\) is the weight, and \(b\) is the intercept.
    \end{itemize}
\end{frame}

\begin{frame}[fragile]
    \frametitle{Core Concepts of Machine Learning - Unsupervised Learning}
    \textbf{Unsupervised Learning}
    \begin{itemize}
        \item \textbf{Definition}: This type of learning deals with unlabeled data, allowing the model to learn the underlying structure without explicit output guidance.
        \item \textbf{Key Points}:
        \begin{itemize}
            \item The model identifies patterns, groupings, or anomalies in the dataset.
            \item Useful for exploratory data analysis.
        \end{itemize}
        \item \textbf{Examples}:
        \begin{itemize}
            \item Clustering: Grouping customers into segments based on purchasing behavior.
            \item Dimensionality Reduction: Reducing the number of features in a dataset while preserving its structure.
        \end{itemize}
    \end{itemize}
\end{frame}

\begin{frame}[fragile]
    \frametitle{Core Concepts of Machine Learning - Reinforcement Learning}
    \textbf{Reinforcement Learning}
    \begin{itemize}
        \item \textbf{Definition}: An agent learns to make decisions by taking actions in an environment to maximize a reward signal.
        \item \textbf{Key Points}:
        \begin{itemize}
            \item Focus on learning a policy that maps states of the environment to actions.
            \item Involves exploration and exploitation.
        \end{itemize}
        \item \textbf{Examples}:
        \begin{itemize}
            \item Game Playing: Training an agent (e.g., AlphaGo) to play chess or Go.
            \item Robotics: Teaching robots to navigate through a space with obstacles while maximizing efficiency.
        \end{itemize}
    \end{itemize}
\end{frame}

\begin{frame}[fragile]
    \frametitle{Core Concepts of Machine Learning - Summary}
    Understanding these core concepts—supervised, unsupervised, and reinforcement learning—provides the foundation for exploring various machine learning algorithms and their applications. 
    Each type of learning addresses unique challenges and can be applied across myriad domains, from finance to healthcare.
\end{frame}

\begin{frame}[fragile]
    \frametitle{Machine Learning Algorithms - Overview}
    \begin{itemize}
        \item Explore fundamental algorithms: 
        \begin{itemize}
            \item Linear Regression
            \item Decision Trees
            \item Neural Networks
        \end{itemize}
        \item Unique characteristics and applications.
    \end{itemize}
\end{frame}

\begin{frame}[fragile]
    \frametitle{Machine Learning Algorithms - Linear Regression}
    
    \begin{block}{Definition}
        Linear regression predicts a continuous outcome variable based on predictor variables, assuming a linear relationship.
    \end{block}
    
    \begin{block}{Key Formula}
        \begin{equation}
            Y = b_0 + b_1 X_1 + b_2 X_2 + \ldots + b_n X_n + \epsilon
        \end{equation}
        where:
        \begin{itemize}
            \item $Y$ = predicted value
            \item $b_0$ = intercept
            \item $b_i$ = coefficients of the features
            \item $\epsilon$ = error term
        \end{itemize}
    \end{block}
    
    \begin{block}{Use Case Example}
        Predicting house prices based on size, location, and bedrooms:
        \begin{equation}
            \text{Price} = b_0 + b_1 \times \text{Size}
        \end{equation}
    \end{block}
\end{frame}

\begin{frame}[fragile]
    \frametitle{Machine Learning Algorithms - Decision Trees and Neural Networks}
    
    \begin{block}{Decision Trees}
        \begin{itemize}
            \item Flowchart-like structure with nodes as features/tests.
            \item Easy to interpret; handles numerical and categorical data.
            \item Prone to overfitting without regulation.
        \end{itemize}
    \end{block}
    
    \begin{block}{Use Case Example}
        Classifying emails as spam or not using decision rules:
        \begin{itemize}
            \item Is the subject line urgent?
            \item Does it contain specific keywords?
        \end{itemize}
    \end{block}

    \begin{block}{Neural Networks}
        \begin{itemize}
            \item Inspired by the human brain; consists of interconnected nodes (neurons).
            \item Learns non-linear relationships, requires substantial data and resources.
        \end{itemize}
    \end{block}

    \begin{block}{Use Case Example}
        Recognizing handwritten digits with the MNIST dataset.
    \end{block}
\end{frame}

\begin{frame}[fragile]
    \frametitle{Key Points to Emphasize}
    \begin{itemize}
        \item \textbf{Linear Regression}: Best for predicting continuous outcomes with linear relationships.
        \item \textbf{Decision Trees}: Provide an interpretable model for classification/regression, but beware of overfitting.
        \item \textbf{Neural Networks}: Excel in large datasets for complex problems, advancing fields like computer vision and natural language processing.
    \end{itemize}
    
    \begin{block}{Next Steps}
        Overview provides foundational understanding; next we will discuss the critical role of data in machine learning, focusing on data quality and preprocessing.
    \end{block}
\end{frame}

\begin{frame}[fragile]
    \frametitle{Data in Machine Learning - Overview}
    \begin{block}{Understanding the Role and Importance}
        This section covers the critical aspects of data quality, integrity, and preprocessing in machine learning (ML).
    \end{block}
\end{frame}

\begin{frame}[fragile]
    \frametitle{Data in Machine Learning - Importance of Data Quality}
    \begin{itemize}
        \item \textbf{Definition}: Data quality refers to the accuracy, completeness, and reliability of the data used in ML.
        \item \textbf{Why It Matters}: High-quality data leads to better model performance and more accurate predictions.
    \end{itemize}
    \begin{block}{Key Points}
        \begin{itemize}
            \item \textbf{Accuracy}: Data must be correct and error-free.
            \item \textbf{Completeness}: All required data points must be present.
            \item \textbf{Consistency}: Data should be uniform across datasets and stages.
        \end{itemize}
    \end{block}
    \begin{exampleblock}{Example}
        A healthcare model trained on incorrect patient age entries may yield false conclusions about health outcomes.
    \end{exampleblock}
\end{frame}

\begin{frame}[fragile]
    \frametitle{Data in Machine Learning - Data Integrity and Preprocessing}
    \begin{itemize}
        \item \textbf{Data Integrity}: Involves maintaining and assuring the accuracy and consistency of data throughout its lifecycle.
        \begin{itemize}
            \item \textbf{Validity}: Data must meet defined requirements.
            \item \textbf{Protection}: Safeguarding against unauthorized access and corruption.
        \end{itemize}
        \item \textbf{Preprocessing Data}: Transforming raw data into a clean format for ML models.
        \begin{itemize}
            \item \textbf{Data Cleaning}: Removing duplicates, filling missing values, correcting inconsistencies.
            \item \textbf{Normalization}: Scaling to a uniform range (e.g., Min-Max scaling).
            \item \textbf{Encoding Categorical Variables}: Converting categories into numerical format using techniques such as one-hot encoding.
        \end{itemize}
    \end{itemize}
    \begin{block}{Example Code Snippet (Python)}
    \begin{lstlisting}[language=Python]
import pandas as pd
from sklearn.preprocessing import MinMaxScaler, OneHotEncoder

# Load dataset
data = pd.read_csv('data.csv')

# Handling missing values
data.fillna(data.mean(), inplace=True)

# Normalization
scaler = MinMaxScaler()
data[['Feature1', 'Feature2']] = scaler.fit_transform(data[['Feature1', 'Feature2']])

# One-hot encoding for categorical variables
data = pd.get_dummies(data, columns=['Category'])
    \end{lstlisting}
    \end{block}
\end{frame}

\begin{frame}[fragile]
    \frametitle{Data in Machine Learning - Conclusion}
    \begin{block}{Key Takeaways}
        \begin{itemize}
            \item Data integrity and quality are crucial for effective machine learning.
            \item Proper preprocessing enhances the readiness of data for ML algorithms.
        \end{itemize}
    \end{block}
    \begin{block}{Next Steps}
        In the next slide, we will explore how these principles apply across industries through case studies.
    \end{block}
\end{frame}

\begin{frame}[fragile]
    \frametitle{Applications of Machine Learning}
    Machine Learning (ML) is revolutionizing industries by enabling systems to learn from data, identify patterns, and make decisions with minimal human intervention. 
    This slide showcases several key applications of ML across different sectors, highlighting its transformative power.
\end{frame}

\begin{frame}[fragile]
    \frametitle{1. Healthcare}
    \begin{block}{Case Study: Predictive Analytics in Patient Care}
        \begin{itemize}
            \item \textbf{Concept}: ML algorithms analyze patient data to predict health outcomes.
            \item \textbf{Example}: Hospitals use ML models to predict patient readmission rates based on historical data, enabling preventive care measures.
            \item \textbf{Impact}: Improved patient outcomes and reduced healthcare costs.
        \end{itemize}
    \end{block}
\end{frame}

\begin{frame}[fragile]
    \frametitle{2. Finance}
    \begin{block}{Case Study: Fraud Detection}
        \begin{itemize}
            \item \textbf{Concept}: ML models are used to identify anomalous behavior in transactions.
            \item \textbf{Example}: Credit card companies deploy ML techniques to analyze transaction patterns in real-time, flagging potentially fraudulent activities.
            \item \textbf{Impact}: Enhanced security for financial transactions and protection for consumers.
        \end{itemize}
    \end{block}
\end{frame}

\begin{frame}[fragile]
    \frametitle{3. Retail}
    \begin{block}{Case Study: Recommendation Systems}
        \begin{itemize}
            \item \textbf{Concept}: ML analyzes customer behavior to recommend products.
            \item \textbf{Example}: E-commerce platforms like Amazon use collaborative filtering algorithms to suggest products based on browsing history and user ratings.
            \item \textbf{Impact}: Increased sales and improved customer satisfaction through personalized shopping experiences.
        \end{itemize}
    \end{block}
\end{frame}

\begin{frame}[fragile]
    \frametitle{4. Transportation}
    \begin{block}{Case Study: Autonomous Vehicles}
        \begin{itemize}
            \item \textbf{Concept}: ML algorithms process massive amounts of data from sensors and cameras to navigate safely.
            \item \textbf{Example}: Companies like Tesla and Waymo use deep learning to interpret the environment, make driving decisions, and adapt to changing road conditions.
            \item \textbf{Impact}: Potential for safer roads and reductions in traffic accidents.
        \end{itemize}
    \end{block}
\end{frame}

\begin{frame}[fragile]
    \frametitle{5. Manufacturing}
    \begin{block}{Case Study: Predictive Maintenance}
        \begin{itemize}
            \item \textbf{Concept}: ML predicts equipment failures before they occur by analyzing sensor data.
            \item \textbf{Example}: Factories implement ML to track machinery health and schedule maintenance, reducing downtime.
            \item \textbf{Impact}: Significant cost savings and improved operational efficiency.
        \end{itemize}
    \end{block}
\end{frame}

\begin{frame}[fragile]
    \frametitle{Key Points to Emphasize}
    \begin{itemize}
        \item \textbf{Versatility}: Machine Learning can be applied across diverse fields, highlighting its flexibility.
        \item \textbf{Data Dependency}: Effectiveness relies on the quality of data—data cleaning and preprocessing is crucial.
        \item \textbf{Continuous Improvement}: ML models improve over time as they learn from new data, leading to better accuracy and efficiency.
    \end{itemize}
\end{frame}

\begin{frame}[fragile]
    \frametitle{Conclusion}
    Understanding the various applications of Machine Learning is essential as it transforms industries fundamentally. These case studies illustrate the potential of ML to create efficient, intelligent systems that enhance service delivery and operational effectiveness across sectors.
\end{frame}

\begin{frame}[fragile]
    \frametitle{Ethical Considerations in Machine Learning}
    \begin{block}{Introduction}
        As Machine Learning (ML) technologies advance, the ethical implications of their deployment become increasingly significant. These considerations revolve around:
        \begin{itemize}
            \item How algorithms are designed
            \item The data used in training
            \item The impact of their decisions on society
        \end{itemize}
    \end{block}
\end{frame}

\begin{frame}[fragile]
    \frametitle{Ethical Considerations: Bias in Algorithms}
    \begin{block}{Definition}
        Bias in ML refers to the systematic favoritism or discrimination that can arise from flawed training data or model assumptions.
    \end{block}
    
    \begin{itemize}
        \item \textbf{Examples:}
        \begin{itemize}
            \item \textbf{Hiring Algorithms:} If historical hiring data includes candidates from a specific demographic, the model may favor those traits, sidelining qualified candidates from other backgrounds.
            \item \textbf{Facial Recognition:} Studies indicate higher error rates for individuals with darker skin tones, leading to unfair accusations or wrongful arrests.
        \end{itemize}
        
        \item \textbf{Key Point:} Addressing bias is crucial for fairness in decision-making and representation across diverse populations.
    \end{itemize}
\end{frame}

\begin{frame}[fragile]
    \frametitle{Ethical Considerations: Data Privacy}
    \begin{block}{Definition}
        Data privacy concerns arise regarding how personal data is collected, stored, and used in ML applications.
    \end{block}
    
    \begin{itemize}
        \item \textbf{Challenges:}
        \begin{itemize}
            \item \textbf{Informed Consent:} Users may not fully understand how their data will be used, leading to ethical dilemmas.
            \item \textbf{Data Breaches:} High-profile cases like Cambridge Analytica highlight risks of unauthorized access to personal information.
        \end{itemize}
        
        \item \textbf{Key Point:} Prioritize user consent, data anonymization, and robust security measures to protect personal data.
    \end{itemize}
\end{frame}

\begin{frame}[fragile]
    \frametitle{Ethical Considerations: Transparency and Fairness}
    \begin{block}{Transparency and Accountability}
        ML systems can function as black boxes, making the decision-making process opaque. It is crucial that stakeholders understand how decisions are made, especially in critical areas like healthcare and criminal justice.
    \end{block}
    
    \begin{itemize}
        \item \textbf{Importance:}
        \begin{itemize}
            \item Establish accountability for actions and outcomes of automated systems.
        \end{itemize}
        
        \item \textbf{Fairness Principle:} Ensure ML applications do not perpetuate inequalities through ongoing evaluation and adjustment.
        
        \item \textbf{Frameworks:} Implement fairness-aware algorithms (e.g., equal opportunity and demographic parity).
    \end{itemize}
\end{frame}

\begin{frame}[fragile]
    \frametitle{Conclusion and Key Takeaways}
    \begin{block}{Conclusion}
        Addressing these ethical considerations is essential for responsible AI development. By fostering awareness of biases, protecting data privacy, promoting transparency, and ensuring fairness, we can create ML systems that serve all individuals equitably and justly.
    \end{block}
    
    \begin{itemize}
        \item Recognize bias in algorithms and its societal impact.
        \item Uphold data privacy standards to protect individuals.
        \item Promote transparency and accountability in ML systems.
        \item Strive for fairness and equity in all ML applications.
    \end{itemize}
\end{frame}

\begin{frame}[fragile]
    \frametitle{Code Example}
    \begin{lstlisting}[language=Python]
from fairness import FairnessMetric

# Example of checking fairness in predictions
predictions = model.predict(X_test)
fairness_evaluation = FairnessMetric(y_test, predictions)
print(fairness_evaluation.check_fairness())
    \end{lstlisting}
\end{frame}

\begin{frame}[fragile]
    \frametitle{Hands-on Workshop: Implementing ML Models}
    \begin{block}{Overview}
        In this workshop, you will gain practical experience with Machine Learning (ML) by implementing models using popular Python libraries: \textbf{Scikit-learn} and \textbf{TensorFlow}. You will be guided through real-world examples to solidify your understanding of key concepts.
    \end{block}
    
    \begin{block}{Objectives}
        \begin{enumerate}
            \item Understand the basic workflow of building ML models.
            \item Familiarize yourself with Scikit-learn for classical ML tasks.
            \item Explore TensorFlow for building deep learning models.
            \item Apply learned techniques to a mini-project.
        \end{enumerate}
    \end{block}
\end{frame}

\begin{frame}[fragile]
    \frametitle{Key Concepts - Machine Learning Workflow}
    \begin{itemize}
        \item \textbf{Data Collection}: Gather relevant data for the problem.
        \item \textbf{Preprocessing}: Clean and prepare the data (handling missing values, encoding categorical variables).
        \item \textbf{Model Selection}: Choose an appropriate ML algorithm (e.g., linear regression, decision trees).
        \item \textbf{Training}: Fit the model to the training dataset.
        \item \textbf{Evaluation}: Assess model performance using metrics (e.g., accuracy, F1-score).
        \item \textbf{Prediction}: Use the model to make predictions on new data.
    \end{itemize}
\end{frame}

\begin{frame}[fragile]
    \frametitle{Key Concepts - Scikit-learn and TensorFlow}

    \begin{block}{Scikit-learn (Classical ML)}
        Used for simpler ML tasks, such as predicting housing prices.
        \begin{lstlisting}[language=Python]
        from sklearn.model_selection import train_test_split
        from sklearn.linear_model import LinearRegression
        from sklearn.metrics import mean_squared_error
        
        # Example dataset
        X = [[1], [2], [3], [4]] # Features
        y = [1, 2, 3, 4]         # Target
        
        # Split data
        X_train, X_test, y_train, y_test = train_test_split(X, y, test_size=0.2)
        
        # Create model
        model = LinearRegression()
        
        # Train model
        model.fit(X_train, y_train)
        
        # Make predictions
        predictions = model.predict(X_test)
        
        # Evaluate
        mse = mean_squared_error(y_test, predictions)
        print('Mean Squared Error:', mse)
        \end{lstlisting}
    \end{block}

    \begin{block}{TensorFlow (Deep Learning)}
        Utilized for complex tasks like image and speech recognition.
        \begin{lstlisting}[language=Python]
        import tensorflow as tf
        from tensorflow.keras import layers, models

        # Define model
        model = models.Sequential([
            layers.Flatten(input_shape=(28, 28)),
            layers.Dense(128, activation='relu'),
            layers.Dense(10, activation='softmax')
        ])

        model.compile(optimizer='adam',
                      loss='sparse_categorical_crossentropy',
                      metrics=['accuracy'])
        
        # Train the model
        model.fit(train_images, train_labels, epochs=5)
        \end{lstlisting}
    \end{block}
\end{frame}

\begin{frame}[fragile]
    \frametitle{Key Points to Emphasize}
    \begin{itemize}
        \item \textbf{Importance of Data Preprocessing}: Data quality directly impacts model performance.
        \item \textbf{Model Evaluation}: Different tasks require different evaluation metrics.
        \item \textbf{Experimentation}: Encourage experimenting with different algorithms and parameters to find the best model.
    \end{itemize}
\end{frame}

\begin{frame}[fragile]
    \frametitle{Mini-Project}
    \begin{block}{Task}
        Using a dataset of your choice, implement a classification or regression model.
    \end{block}
    
    \begin{block}{Goal}
        Apply the ML workflow—collect data, preprocess, train a model, and evaluate its performance.
    \end{block}
\end{frame}

\begin{frame}[fragile]
    \frametitle{Conclusion}
    This hands-on workshop reinforces theoretical concepts discussed in the previous session on ethical considerations in ML, empowering you with practical skills vital for your journey in AI and ML.
\end{frame}

\begin{frame}[fragile]
    \frametitle{Collaborative Projects - Introduction}
    Collaborative projects provide students with the opportunity to:
    \begin{itemize}
        \item Apply theoretical concepts in a practical setting.
        \item Develop teamwork and collaboration skills.
        \item Gain hands-on experience in machine learning.
    \end{itemize}
\end{frame}

\begin{frame}[fragile]
    \frametitle{Collaborative Projects - Objectives}
    \begin{enumerate}
        \item \textbf{Practical Application:} Solidify your understanding of machine learning concepts by applying them to real-world problems.
        \item \textbf{Team Dynamics:} Experience teamwork to enhance skills in communication, project management, and problem-solving.
        \item \textbf{Tangible Outcomes:} Produce a project that showcases your learning, serving as a portfolio piece for future opportunities.
    \end{enumerate}
\end{frame}

\begin{frame}[fragile]
    \frametitle{Collaborative Projects - Project Ideas}
    \begin{itemize}
        \item \textbf{Predictive Modeling:} Develop a model predicting outcomes.
            \begin{itemize}
                \item \textit{Example Task:} Build a regression model to forecast retail sales based on historical data.
            \end{itemize}
        \item \textbf{Image Classification:} Create a model to categorize images.
            \begin{itemize}
                \item \textit{Example Task:} Train a CNN on CIFAR-10 to classify image categories.
            \end{itemize}
        \item \textbf{Natural Language Processing (NLP):} Analyze text data to derive insights.
            \begin{itemize}
                \item \textit{Example Task:} Build a sentiment analysis tool for classifying movie reviews.
            \end{itemize}
    \end{itemize}
\end{frame}

\begin{frame}[fragile]
    \frametitle{Preparing for Further Studies and Careers in AI - Introduction}
    As we close our exploration of Machine Learning, it is essential to consider how these newly acquired skills can be leveraged in further studies and potential career pathways in Artificial Intelligence (AI).
\end{frame}

\begin{frame}[fragile]
    \frametitle{Key Concepts to Leverage}
    \begin{enumerate}
        \item \textbf{Understanding Machine Learning Algorithms}
            \begin{itemize}
                \item Familiarity with key algorithms: supervised, unsupervised, and reinforcement learning.
                \item Example: Applying linear regression models to real-world datasets to demonstrate practical skills.
            \end{itemize}
        \item \textbf{Data Preprocessing Skills}
            \begin{itemize}
                \item Techniques such as data cleaning, normalization, and transformation.
                \item \textit{Illustration}: Show a flowchart of a data preprocessing pipeline.
            \end{itemize}
        \item \textbf{Programming Proficiency}
            \begin{itemize}
                \item Leverage languages like Python.
                \item Familiarity with libraries: Scikit-learn, TensorFlow, and PyTorch.
            \end{itemize}
        \item \textbf{Understanding of Discrete Mathematics \& Statistics}
            \begin{itemize}
                \item Enhances ability to grasp algorithms and model evaluations effectively.
            \end{itemize}
    \end{enumerate}
\end{frame}

\begin{frame}[fragile]
    \frametitle{Career Opportunities}
    \begin{enumerate}
        \item \textbf{Data Scientist}
            \begin{itemize}
                \item Responsibilities: Analyze complex data for decision-making.
                \item Example: Creating predictive models to forecast sales.
            \end{itemize}
        \item \textbf{Machine Learning Engineer}
            \begin{itemize}
                \item Focus: Designing, building, and deploying ML models.
                \item Skills: Software engineering and familiarity with cloud platforms.
            \end{itemize}
        \item \textbf{AI Research Scientist}
            \begin{itemize}
                \item Engages in research on new algorithms and AI technologies.
                \item Requires strong theoretical foundation, often needing advanced degrees.
            \end{itemize}
        \item \textbf{Business Analyst}
            \begin{itemize}
                \item Utilizes ML insights for strategic business decisions.
            \end{itemize}
    \end{enumerate}
\end{frame}

\begin{frame}[fragile]
    \frametitle{Further Studies and Networking}
    \textbf{Further Studies:}
    \begin{itemize}
        \item Pursue Advanced Degrees: Consider a Master's or PhD specializing in AI or Data Science.
        \item Online Courses and Certifications: Platforms like Coursera, edX, or Udacity for specialized courses.
    \end{itemize}
    
    \textbf{Networking and Real-World Experience:}
    \begin{itemize}
        \item Join AI Meetups and Workshops: Network with professionals and stay updated on industry trends.
        \item Internships: Gain practical experience and insight into business implementations of AI solutions.
    \end{itemize}
\end{frame}

\begin{frame}[fragile]
    \frametitle{Conclusion}
    The skills and knowledge gained from your understanding of Machine Learning are foundational for academic growth and essential in navigating and contributing to the evolving landscape of AI careers. Engage actively in projects, seek deeper knowledge, and remain on the lookout for networking opportunities.
\end{frame}


\end{document}