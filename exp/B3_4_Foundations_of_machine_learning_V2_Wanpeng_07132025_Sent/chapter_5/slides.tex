\documentclass[aspectratio=169]{beamer}

% Theme and Color Setup
\usetheme{Madrid}
\usecolortheme{whale}
\useinnertheme{rectangles}
\useoutertheme{miniframes}

% Additional Packages
\usepackage[utf8]{inputenc}
\usepackage[T1]{fontenc}
\usepackage{graphicx}
\usepackage{booktabs}
\usepackage{listings}
\usepackage{amsmath}
\usepackage{amssymb}
\usepackage{xcolor}
\usepackage{tikz}
\usepackage{pgfplots}
\pgfplotsset{compat=1.18}
\usetikzlibrary{positioning}
\usepackage{hyperref}

% Custom Colors
\definecolor{myblue}{RGB}{31, 73, 125}
\definecolor{mygray}{RGB}{100, 100, 100}
\definecolor{mygreen}{RGB}{0, 128, 0}
\definecolor{myorange}{RGB}{230, 126, 34}
\definecolor{mycodebackground}{RGB}{245, 245, 245}

% Set Theme Colors
\setbeamercolor{structure}{fg=myblue}
\setbeamercolor{frametitle}{fg=white, bg=myblue}
\setbeamercolor{title}{fg=myblue}
\setbeamercolor{section in toc}{fg=myblue}
\setbeamercolor{item projected}{fg=white, bg=myblue}
\setbeamercolor{block title}{bg=myblue!20, fg=myblue}
\setbeamercolor{block body}{bg=myblue!10}
\setbeamercolor{alerted text}{fg=myorange}

% Set Fonts
\setbeamerfont{title}{size=\Large, series=\bfseries}
\setbeamerfont{frametitle}{size=\large, series=\bfseries}
\setbeamerfont{caption}{size=\small}
\setbeamerfont{footnote}{size=\tiny}

% Code Listing Style
\lstdefinestyle{customcode}{
  backgroundcolor=\color{mycodebackground},
  basicstyle=\footnotesize\ttfamily,
  breakatwhitespace=false,
  breaklines=true,
  commentstyle=\color{mygreen}\itshape,
  keywordstyle=\color{blue}\bfseries,
  stringstyle=\color{myorange},
  numbers=left,
  numbersep=8pt,
  numberstyle=\tiny\color{mygray},
  frame=single,
  framesep=5pt,
  rulecolor=\color{mygray},
  showspaces=false,
  showstringspaces=false,
  showtabs=false,
  tabsize=2,
  captionpos=b
}
\lstset{style=customcode}

% Custom Commands
\newcommand{\hilight}[1]{\colorbox{myorange!30}{#1}}
\newcommand{\source}[1]{\vspace{0.2cm}\hfill{\tiny\textcolor{mygray}{Source: #1}}}
\newcommand{\concept}[1]{\textcolor{myblue}{\textbf{#1}}}
\newcommand{\separator}{\begin{center}\rule{0.5\linewidth}{0.5pt}\end{center}}

% Footer and Navigation Setup
\setbeamertemplate{footline}{
  \leavevmode%
  \hbox{%
  \begin{beamercolorbox}[wd=.3\paperwidth,ht=2.25ex,dp=1ex,center]{author in head/foot}%
    \usebeamerfont{author in head/foot}\insertshortauthor
  \end{beamercolorbox}%
  \begin{beamercolorbox}[wd=.5\paperwidth,ht=2.25ex,dp=1ex,center]{title in head/foot}%
    \usebeamerfont{title in head/foot}\insertshorttitle
  \end{beamercolorbox}%
  \begin{beamercolorbox}[wd=.2\paperwidth,ht=2.25ex,dp=1ex,center]{date in head/foot}%
    \usebeamerfont{date in head/foot}
    \insertframenumber{} / \inserttotalframenumber
  \end{beamercolorbox}}%
  \vskip0pt%
}

% Turn off navigation symbols
\setbeamertemplate{navigation symbols}{}

% Title Page Information
\title[Chapter 5: Classification Algorithms]{Chapter 5: Classification Algorithms}
\author[J. Smith]{John Smith, Ph.D.}
\institute[University Name]{
  Department of Computer Science\\
  University Name\\
  \vspace{0.3cm}
  Email: email@university.edu\\
  Website: www.university.edu
}
\date{\today}

% Document Start
\begin{document}

\frame{\titlepage}

\begin{frame}[fragile]
    \titlepage
\end{frame}

\begin{frame}[fragile]
    \frametitle{What are Classification Algorithms?}
    \begin{block}{Definition}
        Classification algorithms are a subset of machine learning techniques used to predict categories or classes based on input data. They provide powerful ways to analyze complex data and make informed decisions.
    \end{block}
\end{frame}

\begin{frame}[fragile]
    \frametitle{Why Classification is Important?}
    \begin{itemize}
        \item \textbf{Real-World Applications:}
        \begin{itemize}
            \item \textbf{Email Filtering:} Spam vs. non-spam classification helps users manage their inboxes effectively.
            \item \textbf{Medical Diagnosis:} Assist in classifying diseases based on patient data, aiding in faster and accurate diagnosis.
            \item \textbf{Sentiment Analysis:} Classifies public opinion from social media data (positive, neutral, negative).
        \end{itemize}
        \item \textbf{Business Intelligence:} 
        Companies utilize customer classification for targeted marketing strategies, enhancing customer engagement and retention.
        \item \textbf{Ease of Interpretation:} 
        Classification algorithms often produce models that are easier to interpret, making it easier for stakeholders to understand predictions.
    \end{itemize}
\end{frame}

\begin{frame}[fragile]
    \frametitle{Key Concepts in Classification}
    \begin{itemize}
        \item \textbf{Binary vs. Multi-Class Classification:}
        \begin{itemize}
            \item \textbf{Binary Classification:} Classifying instances into two distinct classes (e.g., Yes/No).
            \item \textbf{Multi-Class Classification:} Involves more than two classes (e.g., classifying types of fruits: apple, banana, orange).
        \end{itemize}
        \item \textbf{Training and Testing:} 
        The model is trained on labeled data (known outcomes) to learn patterns and then tested on unseen data to evaluate its accuracy.
    \end{itemize}
\end{frame}

\begin{frame}[fragile]
    \frametitle{Common Classification Algorithms}
    \begin{itemize}
        \item \textbf{Decision Trees:} 
        Visual flowchart-like structures that can classify data based on feature values.
        \item \textbf{Support Vector Machines (SVM):} 
        Finds the best hyperplane that separates different classes in high-dimensional space.
        \item \textbf{K-Nearest Neighbors (KNN):} 
        Classifies a data point based on how its neighbors are classified.
        \item \textbf{Neural Networks:} 
        Deep learning models effective for complex classification tasks such as image or speech recognition.
    \end{itemize}
\end{frame}

\begin{frame}[fragile]
    \frametitle{Key Points to Emphasize}
    \begin{itemize}
        \item \textbf{Versatility:} 
        Classification algorithms can be applied to a variety of industries and domains.
        \item \textbf{Performance Metrics:} 
        Accuracy, precision, recall, and F1 score are essential in evaluating classifier performance.
        \item \textbf{Data Quality:} 
        The quality and quantity of the training data significantly impact the performance of classification algorithms.
    \end{itemize}
\end{frame}

\begin{frame}[fragile]
    \frametitle{Engaging Question for Reflection}
    \begin{block}{Consider this}
        How would the world change if algorithms could classify and predict behaviors based on limited data? What ethical considerations should we keep in mind as we leverage these technologies?
    \end{block}
\end{frame}

\begin{frame}[fragile]
    \frametitle{Conclusion}
    Classification algorithms are fundamental in transforming raw data into meaningful insights, driving innovation across numerous industries. Understanding their workings sets the base for further exploration in machine learning.
\end{frame}

\begin{frame}[fragile]
    \frametitle{Understanding Classification}
    Classification is a fundamental supervised learning technique used in machine learning, where the goal is to categorize data points into predefined classes based on their features.
\end{frame}

\begin{frame}[fragile]
    \frametitle{Definition of Classification in Machine Learning}
    \begin{itemize}
        \item Classification involves training a model on a labeled dataset (input-output pairs).
        \item The model learns to recognize patterns and make predictions on unseen data.
    \end{itemize}
\end{frame}

\begin{frame}[fragile]
    \frametitle{Key Concepts}
    \begin{itemize}
        \item \textbf{Supervised Learning}: The model is trained with input-output pairs.
        \item \textbf{Labels vs. Features}:
        \begin{itemize}
            \item Labels: Outcomes to predict (e.g., spam or not spam).
            \item Features: Input variables (e.g., characteristics of an email).
        \end{itemize}
        \item \textbf{Decision Boundary}: A line or surface that separates different classes in the feature space.
    \end{itemize}
\end{frame}

\begin{frame}[fragile]
    \frametitle{Examples of Classification}
    \begin{enumerate}
        \item \textbf{Email Filtering}: Classifying emails as spam or non-spam based on keywords or sender information.
        \item \textbf{Medical Diagnosis}: Predicting whether a patient has a specific disease based on medical test results.
        \item \textbf{Sentiment Analysis}: Assigning sentiment labels to social media posts based on words used.
    \end{enumerate}
\end{frame}

\begin{frame}[fragile]
    \frametitle{Applications of Classification}
    \begin{itemize}
        \item \textbf{Social Media}: Categorizing posts to enhance user experience.
        \item \textbf{Finance}: Fraud detection systems classify transactions as legitimate or fraudulent.
        \item \textbf{Healthcare}: Automating diagnosis through image classification.
    \end{itemize}
\end{frame}

\begin{frame}[fragile]
    \frametitle{Key Points to Emphasize}
    \begin{itemize}
        \item Vital for decision-making processes across various domains.
        \item Designed to handle binary and multi-class problems.
        \item Revolutionizes industries by improving efficiency and accuracy.
    \end{itemize}
\end{frame}

\begin{frame}[fragile]
    \frametitle{Visual Representation}
    \begin{block}{Diagram Description}
        Consider a diagram where:
        \begin{itemize}
            \item X-axis: Feature 1 (e.g., "Email Length")
            \item Y-axis: Feature 2 (e.g., "Frequency of Promo Words")
            \item Segments indicate classes such as "Spam" and "Not Spam," delineated by a decision boundary.
        \end{itemize}
    \end{block}
\end{frame}

\begin{frame}[fragile]
    \frametitle{Conclusion}
    Understanding classification is essential for leveraging machine learning. By mastering this concept, you can explore various classification algorithms such as Decision Trees and k-Nearest Neighbors (k-NN), which will be discussed in the next slide.
\end{frame}

\begin{frame}[fragile]
    \frametitle{Common Classification Algorithms - Introduction}
    \begin{block}{Introduction to Classification Algorithms}
        Classification algorithms are fundamental tools in machine learning that categorize data into predefined groups based on their characteristics. 
        They are widely used across various fields, such as:
        \begin{itemize}
            \item Finance for credit scoring
            \item Healthcare for disease identification
            \item Marketing for customer segmentation
            \item And many more...
        \end{itemize}
    \end{block}
\end{frame}

\begin{frame}[fragile]
    \frametitle{Common Classification Algorithms - Decision Trees}
    \begin{block}{1. Decision Trees}
        \begin{itemize}
            \item \textbf{Overview}: A flowchart-like model that splits data into branches to make decisions based on features.
            \item \textbf{Example}: 
            \begin{itemize}
                \item Classifying whether a fruit is an apple or an orange might involve asking, "Is the fruit red?"
                \item If yes, then proceed with "Is it round?" to classify it as an apple, amidst further questions for oranges.
            \end{itemize}
            \item \textbf{Key Points}:
            \begin{itemize}
                \item Easy to visualize and interpret.
                \item Capable of handling both numerical and categorical data.
                \item May create overfitting; thus, pruning techniques are beneficial.
            \end{itemize}
        \end{itemize}
    \end{block}
\end{frame}

\begin{frame}[fragile]
    \frametitle{Common Classification Algorithms - k-Nearest Neighbors}
    \begin{block}{2. k-Nearest Neighbors (k-NN)}
        \begin{itemize}
            \item \textbf{Overview}: An instance-based learning algorithm that classifies a data point based on the classes of its nearest neighbors.
            \item \textbf{Example}: 
            \begin{itemize}
                \item For classifying a new type of flower, k-NN would look at the closest 'k' types of flowers (e.g., using features like petal length and width) and select the majority class (e.g., identification as ‘Setosa’ if most neighbors are Setosa).
            \end{itemize}
            \item \textbf{Key Points}:
            \begin{itemize}
                \item No training phase; it retains the training dataset.
                \item The choice of 'k' is crucial; a small 'k' can be sensitive to noise, while a large 'k' may smooth out distinctions.
                \item Best applied to small to medium-sized datasets.
            \end{itemize}
        \end{itemize}
    \end{block}
\end{frame}

\begin{frame}[fragile]
    \frametitle{Summary and Engagement Question}
    \begin{block}{Summary}
        Understanding and selecting the right classification algorithm is crucial for building effective predictive models. 
        Both Decision Trees and k-NN offer unique advantages for different types of data and problems. 
        As you explore these algorithms, consider real-life applications and their role in solving complex challenges.
    \end{block}

    \begin{block}{Engagement Question}
        How might you use these algorithms in a project related to your field of interest?
        Using relatable examples helps you see practical applications and strengthens your understanding.
    \end{block}
\end{frame}

\begin{frame}[fragile]
    \frametitle{Decision Trees Overview - Introduction}
    \begin{block}{Definition}
        A Decision Tree is a supervised machine learning algorithm used for classification and regression tasks. It models decisions based on asking a series of questions to partition the data into subsets, ultimately leading to predictions.
    \end{block}
\end{frame}

\begin{frame}[fragile]
    \frametitle{Decision Trees Overview - Structure}
    \begin{itemize}
        \item \textbf{Root Node:}
          \begin{itemize}
              \item The top-most node representing the entire dataset.
              \item Example: "Is the fruit citrus?"
          \end{itemize}
        
        \item \textbf{Internal Nodes:}
          \begin{itemize}
              \item Decision points based on feature values to split the data.
              \item Example: "Is the fruit yellow?" 
          \end{itemize}
        
        \item \textbf{Leaf Nodes:}
          \begin{itemize}
              \item Terminal nodes providing the final output or class label.
              \item Example: "This is a lemon."
          \end{itemize}
    \end{itemize}
\end{frame}

\begin{frame}[fragile]
    \frametitle{Decision Trees Overview - Components}
    \begin{itemize}
        \item \textbf{Branches:}
          \begin{itemize}
              \item Connect the nodes, indicating the flow from question to answer.
          \end{itemize}
        
        \item \textbf{Decision Criteria:}
          \begin{itemize}
              \item Methods for splitting data.
              \begin{itemize}
                  \item \textbf{Gini Impurity:} Measures mislabeling frequency.
                  \item \textbf{Entropy:} Quantifies randomness in the dataset.
              \end{itemize}
          \end{itemize}
    \end{itemize}
\end{frame}

\begin{frame}[fragile]
    \frametitle{Decision Trees Overview - Key Points}
    \begin{itemize}
        \item \textbf{Interpretability:} Easy to visualize, each path from root to leaf represents clear rules.
        \item \textbf{Versatility:} Applicable for both classification and regression tasks.
        \item \textbf{Overfitting:} Common issue mitigated by techniques like pruning.
    \end{itemize}
\end{frame}

\begin{frame}[fragile]
    \frametitle{Decision Trees Overview - Example Scenario}
    \textbf{Problem Statement:}
    You want to determine if a customer will buy a car based on their attributes.

    \begin{block}{Sample Decision Tree}
    \begin{verbatim}
                 [Is age > 30?]
                 /             \
              Yes/               \No
              [Is income > 50k?]   [Does credit score > 700?]
              /        \             /                  \
             Yes       No         Yes                   No
          [Will Buy]  [No Buy]  [Will Buy]           [No Buy]
    \end{verbatim}
    \end{block}

    In this tree, decisions are made based on age, income, or credit score leading to the prediction.
\end{frame}

\begin{frame}[fragile]
    \frametitle{Decision Trees Overview - Conclusion}
    Decision Trees provide a powerful tool for making data-driven predictions due to their straightforward structure and ability to represent complex decisions in an interpretable way.
\end{frame}

\begin{frame}[fragile]
    \frametitle{How Decision Trees Work}
    \begin{block}{Understanding Decision Trees}
        A Decision Tree is a powerful model used for classification and regression tasks in machine learning, resembling human decision-making through a tree-like structure.
    \end{block}
    \begin{itemize}
        \item Internal nodes represent features or attributes.
        \item Branches symbolize decision rules.
        \item Leaf nodes present outcomes or class labels.
    \end{itemize}
\end{frame}

\begin{frame}[fragile]
    \frametitle{Key Concepts of Decision Trees}
    \begin{enumerate}
        \item \textbf{Tree Structure}
            \begin{itemize}
                \item \textbf{Root Node}: Represents the entire dataset.
                \item \textbf{Internal Nodes}: Decision points based on feature values.
                \item \textbf{Leaf Nodes}: Final output or class labels.
            \end{itemize}
        \item \textbf{Splitting Features}
            \begin{itemize}
                \item Recursively split data to maximize information gain.
                \item Criteria include Gini impurity or entropy.
            \end{itemize}
    \end{enumerate}
\end{frame}

\begin{frame}[fragile]
    \frametitle{Step-by-Step Prediction Process}
    \begin{enumerate}
        \item \textbf{Start at Root}: Begin at the root node.
        \item \textbf{Evaluate Feature}: Check feature value at the current node.
        \item \textbf{Follow Decision Path}: Move along the branch for that feature's value.
        \item \textbf{Continue Until Leaf Node}: Repeat until reaching a leaf.
        \item \textbf{Classify Based on Leaf Node}: The class label at the leaf node is the prediction.
    \end{enumerate}
\end{frame}

\begin{frame}[fragile]
    \frametitle{Example Illustration}
    \begin{block}{Scenario}
        Consider a dataset of animals with features like "Has Fur," "Can Fly," and "Lays Eggs."
    \end{block}
    \begin{itemize}
        \item \textbf{Root Node}: "Has Fur?"
            \begin{itemize}
                \item \textbf{Yes}: Go to "Is it a mammal?"
                \item \textbf{No}: "Can Fly?"
                    \begin{itemize}
                        \item \textbf{Yes}: "Is it a bird?"
                        \item \textbf{No}: Classify as "Reptile"
                    \end{itemize}
            \end{itemize}
        \item \textbf{Leaf Nodes}: "Mammal," "Bird," "Reptile"
    \end{itemize}
\end{frame}

\begin{frame}[fragile]
    \frametitle{Key Points to Emphasize}
    \begin{itemize}
        \item \textbf{Interpretability}: Highly interpretable, easy to understand decision-making.
        \item \textbf{Handling Non-linearity}: Captures complex relationships between features.
        \item \textbf{Prone to Overfitting}: Can become overly complex, fitting noise in the data.
    \end{itemize}
\end{frame}

\begin{frame}[fragile]
    \frametitle{Practical Application}
    \begin{lstlisting}[language=Python]
from sklearn.tree import DecisionTreeClassifier

# Sample dataset (features: weight, height; target: type of fruit)
X = [[150, 7], [140, 6], [130, 5], [120, 4]]
y = ['Apple', 'Apple', 'Orange', 'Orange']

# Create model
model = DecisionTreeClassifier()
model.fit(X, y)

# Predict
prediction = model.predict([[135, 5]])  # Returns the predicted type of fruit
    \end{lstlisting}
\end{frame}

\begin{frame}
    \frametitle{Creating Decision Trees}
    \begin{block}{Overview}
        Step-by-step process for creating a Decision Tree, from dataset selection to implementation.
    \end{block}
\end{frame}

\begin{frame}[fragile]
    \frametitle{Step-by-Step Process for Creating a Decision Tree}
    \begin{enumerate}
        \item \textbf{Dataset Selection}
            \begin{itemize}
                \item Choose a relevant dataset.
                \item Example: Predicting customer purchases based on age, income, and gender.
            \end{itemize}

        \item \textbf{Data Preprocessing}
            \begin{itemize}
                \item Clean the data: Handle missing values, remove duplicates.
                \item Encode categorical data: E.g., Male = 0, Female = 1.
            \end{itemize}
    \end{enumerate}
\end{frame}

\begin{frame}[fragile]
    \frametitle{Building and Evaluating the Decision Tree}
    \begin{enumerate}[resume]
        \setcounter{enumi}{3}
        \item \textbf{Splitting the Dataset}
            \begin{itemize}
                \item Typical split: 70\% training, 30\% testing.
            \end{itemize}

        \item \textbf{Building the Decision Tree}
            \begin{itemize}
                \item Use Gini impurity or information gain for splitting.
                \item Example Python code:
                \begin{lstlisting}[language=Python]
                from sklearn.tree import DecisionTreeClassifier

                model = DecisionTreeClassifier(criterion='gini') # Gini criterion
                model.fit(X_train, y_train)  # Fit the model
                \end{lstlisting}
            \end{itemize}
        
        \item \textbf{Model Evaluation}
            \begin{itemize}
                \item Assess performance using accuracy, precision, recall.
            \end{itemize}

        \item \textbf{Implementation}
            \begin{itemize}
                \item Deploy the model in real-world applications.
            \end{itemize}
    \end{enumerate}
\end{frame}

\begin{frame}[fragile]
    \frametitle{Key Points and Example Code}
    \begin{block}{Key Points to Remember}
        \begin{itemize}
            \item Easy to interpret and visualize.
            \item Proper data preprocessing is crucial.
            \item Overfitting can be prevented with pruning.
            \item Visualize model behavior for insights.
        \end{itemize}
    \end{block}

    \begin{block}{Example Python Code Snippet}
    \begin{lstlisting}[language=Python]
    from sklearn.model_selection import train_test_split
    from sklearn.tree import DecisionTreeClassifier, plot_tree

    X = df[['age', 'income', 'gender']]
    y = df['purchase']
    X_train, X_test, y_train, y_test = train_test_split(X, y, test_size=0.3, random_state=42)

    model = DecisionTreeClassifier(criterion='entropy')
    model.fit(X_train, y_train)

    plot_tree(model, feature_names=X.columns)
    \end{lstlisting}
    \end{block}
\end{frame}

\begin{frame}[fragile]
    \frametitle{Advantages of Decision Trees - Introduction}
    \begin{block}{Introduction to Decision Trees}
        Decision Trees are a popular classification method in machine learning. They represent decisions and their possible consequences in a tree-like model, making them intuitive and easy to understand.
    \end{block}
\end{frame}

\begin{frame}[fragile]
    \frametitle{Advantages of Decision Trees - Key Advantages}
    \begin{itemize}
        \item Easy to Understand and Interpret
        \item Requires Little Data Preparation
        \item Handles Both Classification and Regression Tasks
        \item Non-Parametric Nature
        \item Can Capture Non-linear Relationships
        \item Feature Importance Availability
    \end{itemize}
\end{frame}

\begin{frame}[fragile]
    \frametitle{Advantages of Decision Trees - Detailed Explanation}
    \begin{enumerate}
        \item \textbf{Easy to Understand and Interpret}
            \begin{itemize}
                \item Flowchart-like structure aids comprehension.
                \item Example: "Is the weather sunny?" leads to decisions about playing tennis.
            \end{itemize}
        \item \textbf{Requires Little Data Preparation}
            \begin{itemize}
                \item No normalization needed; handles numerical and categorical data.
                \item Example: Mixed data types (age and income level) are easily addressed.
            \end{itemize}
        \item \textbf{Handles Both Classification and Regression Tasks}
            \begin{itemize}
                \item Adaptable for class predictions and continuous value predictions.
                \item Example: Classifying houses by price brackets or predicting exact prices.
            \end{itemize}
    \end{enumerate}
\end{frame}

\begin{frame}[fragile]
    \frametitle{Advantages of Decision Trees - Remaining Advantages}
    \begin{enumerate}[resume]
        \item \textbf{Non-Parametric Nature}
            \begin{itemize}
                \item No assumptions about data distribution; versatile for various datasets.
            \end{itemize}
        \item \textbf{Can Capture Non-linear Relationships}
            \begin{itemize}
                \item Tree structure accommodates complex feature relationships.
                \item Example: Using multiple factors (age, purchasing history) for customer segmentation.
            \end{itemize}
        \item \textbf{Feature Importance Availability}
            \begin{itemize}
                \item Evaluates the importance of features in predictions.
                \item Illustration: "Age" identified as significant in purchasing behavior.
            \end{itemize}
    \end{enumerate}
\end{frame}

\begin{frame}[fragile]
    \frametitle{Advantages of Decision Trees - Conclusion and Call to Action}
    \begin{block}{Conclusion}
        Decision Trees offer strengths such as ease of interpretation, minimal data preparation, versatility, and the ability to capture complex relationships. These features make them essential for data scientists and analysts.
    \end{block}
    \begin{block}{Call to Action}
        \begin{itemize}
            \item Reflect on how the intuitive nature of Decision Trees impacts their real-world application.
            \item Consider scenarios where Decision Trees might outperform more complex models.
        \end{itemize}
    \end{block}
\end{frame}

\begin{frame}[fragile]
    \frametitle{Limitations of Decision Trees}
    \begin{block}{Overview}
        Decision Trees are a popular choice for classification tasks due to their easy interpretation and versatility. However, they are not without their weaknesses. This slide explores several notable limitations that can affect their performance.
    \end{block}
\end{frame}

\begin{frame}[fragile]
    \frametitle{Key Limitations of Decision Trees - Part 1}
    \begin{enumerate}
        \item \textbf{Overfitting}:
            \begin{itemize}
                \item Explanation: Decision Trees can easily become overly complex, capturing noise in the training data rather than generalizing well to unseen data.
                \item Example: A tree that splits on many minor characteristics (e.g., color of the product) may fail to classify new transactions accurately.
                \item Solution: Pruning or setting a maximum depth can mitigate overfitting.
            \end{itemize}
        
        \item \textbf{Instability}:
            \begin{itemize}
                \item Explanation: Small changes in the training data can lead to very different tree structures.
                \item Example: Adding a single new data point may significantly change the tree structure, affecting reliability.
                \item Solution: Ensemble methods like Random Forests combine multiple trees for more stable predictions.
            \end{itemize}
    \end{enumerate}
\end{frame}

\begin{frame}[fragile]
    \frametitle{Key Limitations of Decision Trees - Part 2}
    \begin{enumerate}
        \setcounter{enumi}{2} % continue numbering from the previous frame
        \item \textbf{Bias towards Certain Features}:
            \begin{itemize}
                \item Explanation: Decision Trees are biased towards features with more levels.
                \item Example: A variable with 100 unique categories may dominate the splitting process.
                \item Solution: Feature engineering and selection can create more balanced models.
            \end{itemize}
        
        \item \textbf{Poor Performance on Imbalanced Data}:
            \begin{itemize}
                \item Explanation: Sensitive to class distribution; may become biased towards majority class.
                \item Example: A dataset with 95\% healthy cases and 5\% sick cases may lead to mostly predicting healthy.
                \item Solution: Resampling techniques can improve model performance.
            \end{itemize}

        \item \textbf{Limited to Axis-Parallel Splits}:
            \begin{itemize}
                \item Explanation: Only creates splits perpendicular to the axes, limiting modeling of complex relationships.
                \item Example: A Decision Tree may struggle with circular or complex decision boundaries.
                \item Solution: Consider using complex models like Support Vector Machines or Neural Networks for nonlinear data.
            \end{itemize}
    \end{enumerate}
\end{frame}

\begin{frame}[fragile]
    \frametitle{Conclusion}
    \begin{block}{Conclusion}
        Despite their intuitive nature, Decision Trees have several limitations that can impact their effectiveness. Understanding these weaknesses allows data scientists to choose the right modeling approach and apply strategies to enhance performance, ensuring more robust and reliable predictions.
    \end{block}
    \begin{block}{Questions}
        Feel free to ask any questions or share your thoughts about Decision Trees and their limitations!
    \end{block}
\end{frame}

\begin{frame}[fragile]
    \frametitle{Introduction to k-Nearest Neighbors (k-NN)}
    \begin{block}{Overview of k-NN Algorithm}
        k-Nearest Neighbors (k-NN) is a powerful and intuitive classification algorithm used in machine learning. It is based on the concept of proximity, where similar data points are located near each other in the feature space.
    \end{block}
\end{frame}

\begin{frame}[fragile]
    \frametitle{What is k-NN?}
    \begin{itemize}
        \item k-NN is a \textbf{non-parametric} classifier, meaning it makes no assumptions about the underlying data distribution.
        \item It classifies an unseen data point based on the majority class among its k closest neighbors in the training dataset.
    \end{itemize}
\end{frame}

\begin{frame}[fragile]
    \frametitle{How Does k-NN Work?}
    \begin{itemize}
        \item For a given data point, the algorithm calculates the \textbf{distance} to all other data points in the training set.
        \item Common distance metrics include:
        \begin{itemize}
            \item Euclidean distance
            \item Manhattan distance
            \item Minkowski distance
        \end{itemize}
        \item It identifies the k closest points (neighbors) based on these distances.
        \item The algorithm assigns the majority class among these k neighbors to the new data point.
    \end{itemize}
\end{frame}

\begin{frame}[fragile]
    \frametitle{Example Explanation}
    Imagine you want to classify flowers based on features like petal length and width.

    \textbf{Training Set:}
    \begin{itemize}
        \item Flower 1 (A): (1, 2)
        \item Flower 2 (A): (2, 3)
        \item Flower 3 (B): (3, 3)
        \item Flower 4 (B): (5, 5)
    \end{itemize}

    \textbf{New Flower:} (2, 5)
    \begin{enumerate}
        \item Calculate distances to all training flowers.
        \item Identify the k (e.g., k=3) closest neighbors.
        \item Classify based on the majority class among the neighbors.
    \end{enumerate}
\end{frame}

\begin{frame}[fragile]
    \frametitle{Key Points to Emphasize}
    \begin{itemize}
        \item \textbf{Value of k:} A smaller k makes the algorithm sensitive to noise; a larger k smoothens the decision boundary.
        \item \textbf{Distance Metric:} The effectiveness of k-NN heavily depends on the distance metric used.
        \item \textbf{No Training Phase:} k-NN simply stores the dataset without a distinct training phase.
    \end{itemize}
\end{frame}

\begin{frame}[fragile]
    \frametitle{Summary}
    k-NN is a simple yet effective algorithm that relies on the proximity of data points for classification. It applies to various applications, including:
    \begin{itemize}
        \item Pattern recognition
        \item Recommendation systems
    \end{itemize}
    \newline
    As you move on to the next slide about \textbf{"How k-NN Works,"} consider these questions:
    \begin{itemize}
        \item How does changing k affect classification?
        \item What real-world applications can benefit from using k-NN?
    \end{itemize}
\end{frame}

\begin{frame}[fragile]
    \frametitle{How k-NN Works - Overview}
    \begin{itemize}
        \item The k-Nearest Neighbors (k-NN) algorithm is a classification method used in machine learning.
        \item It classifies data points based on the majority class of their 'k' nearest neighbors.
    \end{itemize}
\end{frame}

\begin{frame}[fragile]
    \frametitle{How k-NN Works - Key Concepts}
    \begin{enumerate}
        \item \textbf{Distance Measurement}
            \begin{itemize}
                \item k-NN uses metrics like Euclidean distance to find nearest points.
                \item Example (Euclidean Distance):
                    \begin{equation}
                        \text{Distance} = \sqrt{(x_2 - x_1)^2 + (y_2 - y_1)^2}
                    \end{equation}
            \end{itemize}

        \item \textbf{Finding Neighbors}
            \begin{itemize}
                \item Identifies 'k' closest points from the training dataset.
                \item Example: Choosing a dish based on 3 closest friends' recommendations.
            \end{itemize}

        \item \textbf{Majority Voting}
            \begin{itemize}
                \item Class label assigned based on majority vote from neighbors.
                \item Illustration: New flower classified as "Rose" if 4 out of 6 neighbors are "Rose."
            \end{itemize}
    \end{enumerate}
\end{frame}

\begin{frame}[fragile]
    \frametitle{How k-NN Works - Example}
    \begin{block}{Scenario}
        Classifying fruits based on features like weight and sweetness:
        \begin{itemize}
            \item Apple: (150g, 8)
            \item Orange: (200g, 6)
            \item Banana: (120g, 7)
        \end{itemize}
    \end{block}
    
    \begin{block}{New Data Point}
        A fruit with features (130g, 7):
        \begin{itemize}
            \item Neighbors: Apple (150g, 8), Banana (120g, 7), Orange (200g, 6)
            \item Majority Vote: 1 Apple, 1 Banana (tie, odd k required)
        \end{itemize}
    \end{block}
\end{frame}

\begin{frame}[fragile]
    \frametitle{How k-NN Works - Key Points}
    \begin{itemize}
        \item \textbf{Choosing k}
            \begin{itemize}
                \item Small 'k' susceptible to noise; large 'k' may include irrelevant points.
            \end{itemize}
        \item \textbf{Scalability}
            \begin{itemize}
                \item Computationally expensive with larger datasets due to distance calculations.
            \end{itemize}
    \end{itemize}
    
    \begin{block}{Final Note}
        k-NN is intuitive and non-parametric. Experimentation with 'k' and distance metrics can yield valuable insights.
    \end{block}
\end{frame}

\begin{frame}[fragile]
    \frametitle{Choosing the Right k Value - Understanding 'k' in k-NN}
    \begin{block}{Overview}
        The k-Nearest Neighbors (k-NN) algorithm is a simple yet powerful classification method. 
        The choice of 'k', the number of nearest neighbors to consider, plays a crucial role in model performance. 
        Selecting the right 'k' can significantly impact both the accuracy and the generalization of the model.
    \end{block}
\end{frame}

\begin{frame}[fragile]
    \frametitle{Choosing the Right k Value - General Guidelines}
    \begin{enumerate}
        \item \textbf{Start with Small Values}
        \begin{itemize}
            \item Begin with a small value of 'k' (e.g., 1 or 3). 
            \item A smaller 'k' makes the model very sensitive to noise in the training data.
        \end{itemize}

        \item \textbf{Use Cross-Validation}
        \begin{itemize}
            \item Implement k-fold cross-validation to test various values of 'k'.
            \item Ensures the selected 'k' performs well across different subsets of the data.
        \end{itemize}

        \item \textbf{Consider the Size of the Dataset}
        \begin{itemize}
            \item Smaller datasets may benefit from a smaller 'k'; larger datasets typically require larger 'k' for robustness.
        \end{itemize}

        \item \textbf{Assess Overfitting vs. Underfitting}
        \begin{itemize}
            \item If 'k' is too large, the model may generalize too much, missing important structures.
            \item If 'k' is too small, the model may too closely fit the training data and capture noise.
        \end{itemize}
    \end{enumerate}
\end{frame}

\begin{frame}[fragile]
    \frametitle{Choosing the Right k Value - Example Scenario}
    \begin{block}{Example Scenario}
        Consider using k-NN to classify flowers based on features like sepal and petal dimensions.
    \end{block}
    \begin{itemize}
        \item \textbf{For k=1}: 
        \begin{itemize}
            \item Classifies based on the nearest neighbor, possibly leading to overfitting if an outlier is close.
        \end{itemize}

        \item \textbf{For k=5}: 
        \begin{itemize}
            \item Considers the majority class among 5 nearest neighbors, reducing the impact of outliers.
        \end{itemize}

        \item \textbf{For k=20}: 
        \begin{itemize}
            \item May lose important distinctions, leading to underfitting.
        \end{itemize}
    \end{itemize}
    \begin{block}{Key Takeaway}
        Testing various 'k' values may reveal that \textbf{k=3} offers the best balance of accuracy.
    \end{block}
\end{frame}

\begin{frame}[fragile]
    \frametitle{Choosing the Right k Value - Key Takeaways}
    \begin{itemize}
        \item The selection of 'k' directly impacts model performance.
        \item Utilize cross-validation to determine the optimal 'k'.
        \item Adjust 'k' based on dataset size and complexity.
        \item A balanced approach yields better classification results.
    \end{itemize}
\end{frame}

\begin{frame}[fragile]
    \frametitle{Advantages of k-NN}
    \begin{block}{Introduction to k-NN (k-Nearest Neighbors)}
        k-NN is a simple yet effective classification algorithm used in machine learning. It operates based on a straightforward principle: an object is classified based on how its neighbors are classified.
    \end{block}
\end{frame}

\begin{frame}[fragile]
    \frametitle{Key Advantages of k-NN - Part 1}
    \begin{enumerate}
        \item \textbf{Simplicity and Intuitiveness}
            \begin{itemize}
                \item k-NN is one of the easiest algorithms to understand and implement, requiring no complex mathematics.
                \item \textit{Example:} A new student looks at their nearest peers to decide who to be friends with.
            \end{itemize}
        
        \item \textbf{Versatility}
            \begin{itemize}
                \item Applicable for both classification and regression tasks across various fields.
                \item \textit{Example:} In healthcare, predicting a patient’s risk based on similar historical records.
            \end{itemize}
        
        \item \textbf{No Training Phase}
            \begin{itemize}
                \item Unlike other algorithms, k-NN directly uses the training dataset for classification without a training phase.
                \item \textit{Illustration:} Like a library where books are shelved for easy access.
            \end{itemize}
    \end{enumerate}
\end{frame}

\begin{frame}[fragile]
    \frametitle{Key Advantages of k-NN - Part 2}
    \begin{enumerate}
        \setcounter{enumi}{3}
        \item \textbf{Flexibility in Distance Metrics}
            \begin{itemize}
                \item Different distance metrics can be used, making it adaptable to various data types.
                \item \textit{Example:} Classifying flowers based on features like petal length using appropriate distance metrics.
            \end{itemize}

        \item \textbf{Handles Multi-Class Problems}
            \begin{itemize}
                \item Can classify instances into multiple classes without requiring algorithm changes.
                \item \textit{Illustration:} Separating apples, oranges, and bananas in fruit classification.
            \end{itemize}

        \item \textbf{Good Performance with Large Datasets}
            \begin{itemize}
                \item Predictions improve with larger datasets due to more information.
                \item \textit{Example:} Accurate classification in image recognition tasks using thousands of labeled images.
            \end{itemize}
    \end{enumerate}
\end{frame}

\begin{frame}[fragile]
    \frametitle{Key Points and Conclusion}
    \begin{block}{Key Points}
        \begin{itemize}
            \item k-NN requires no training phase, making it ideal for rapid prototyping.
            \item Its simplicity allows easy implementation and interpretation by non-experts.
            \item With proper feature selection, k-NN can yield impressive results, despite challenges with high-dimensional data.
        \end{itemize}
    \end{block}

    \begin{block}{Conclusion}
        k-NN is a powerful and intuitive classification technique suited for various applications. Understanding its strengths helps practitioners leverage its full potential when solving classification problems.
    \end{block}
\end{frame}

\begin{frame}[fragile]
    \frametitle{Limitations of k-NN}
    \begin{block}{Overview}
        While k-Nearest Neighbors (k-NN) is a popular and straightforward classification algorithm, it has several limitations that can impact its performance and applicability in various scenarios.
    \end{block}
\end{frame}

\begin{frame}[fragile]
    \frametitle{Limitations of k-NN - Part 1}
    \begin{enumerate}
        \item \textbf{Computationally Intensive:}
            \begin{itemize}
                \item \textbf{Challenge:} Requires calculating the distance from the query sample to every point in the training set, leading to high computational costs.
                \item \textbf{Example:} Predicting the class of a new observation in a dataset with 1,000,000 entries can be time-consuming.
            \end{itemize}
        
        \item \textbf{Sensitivity to Irrelevant Features:}
            \begin{itemize}
                \item \textbf{Challenge:} Irrelevant or redundant features skew distance calculations, leading to misclassification.
                \item \textbf{Example:} Including unrelated demographic information may mislead the algorithm.
            \end{itemize}
    \end{enumerate}
\end{frame}

\begin{frame}[fragile]
    \frametitle{Limitations of k-NN - Part 2}
    \begin{enumerate}
        \setcounter{enumi}{2} % Continue numbering from previous frame
        \item \textbf{Curse of Dimensionality:}
            \begin{itemize}
                \item \textbf{Challenge:} In high-dimensional spaces, data points become sparse, making "nearness" less meaningful.
                \item \textbf{Illustration:} Navigating in a 100-dimensional space complicates finding truly "near" points.
            \end{itemize}
        
        \item \textbf{Choice of k:}
            \begin{itemize}
                \item \textbf{Challenge:} The number of neighbors (k) significantly impacts model performance. 
                \begin{itemize}
                    \item A small k (e.g., k=1) may be sensitive to noise.
                    \item A large k (e.g., k=50) may lead to generalization and underfitting.
                \end{itemize}
            \end{itemize}

        \item \textbf{Memory Intensive:}
            \begin{itemize}
                \item \textbf{Challenge:} k-NN is a lazy learner that stores all training samples, requiring substantial memory.
                \item \textbf{Consideration:} This can make k-NN impractical in environments with limited memory.
            \end{itemize}
    \end{enumerate}
\end{frame}

\begin{frame}[fragile]
    \frametitle{Key Points and Conclusion}
    \begin{block}{Key Points to Emphasize}
        \begin{itemize}
            \item Effective classification with k-NN requires consideration of dataset size, feature selection, dimensionality, the choice of k, and memory constraints.
            \item Preprocessing steps (like feature scaling) may enhance performance and help mitigate some of its limitations.
        \end{itemize}
    \end{block}

    \begin{block}{Conclusion}
        Understanding the limitations of k-NN is crucial for classifier selection. Acknowledging these challenges aids in making informed decisions on classifier usage and improving classification workflows.
    \end{block}
\end{frame}

\begin{frame}[fragile]
    \frametitle{Comparison of Decision Trees and k-NN - Overview}
    Both Decision Trees and k-Nearest Neighbors (k-NN) are popular classification algorithms in machine learning. They have distinct characteristics and are suitable for different types of data and applications.
\end{frame}

\begin{frame}[fragile]
    \frametitle{Comparison of Decision Trees and k-NN - Decision Trees}
    \begin{block}{Definition}
        A Decision Tree is a flowchart-like tree structure that makes decisions based on answering questions about the features of the data. 
    \end{block}
    
    \begin{itemize}
        \item \textbf{When to Use:}
            \begin{itemize}
                \item Structured Data: Best for datasets with clear hierarchical structures or categorical features.
                \item Interpretability: Ideal when the model needs to be interpretable (easy to visualize).
                \item No Assumption of Data Distribution: Can handle arbitrary distributions and is robust to outliers.
            \end{itemize}
        \item \textbf{Example:} In deciding whether to approve a loan, it considers factors like income level, credit score, and employment status.
    \end{itemize}
\end{frame}

\begin{frame}[fragile]
    \frametitle{Comparison of Decision Trees and k-NN - k-NN}
    \begin{block}{Definition}
        k-NN is a non-parametric method for classification and regression, classifying data points based on their 'k' nearest neighbors.
    \end{block}
    
    \begin{itemize}
        \item \textbf{When to Use:}
            \begin{itemize}
                \item Instance-based Learning: Ideal for small to moderate datasets where each instance carries significance.
                \item Continuous Data: Particularly useful for numerical or continuous data.
                \item Non-linear Boundaries: Effective for datasets with complex, non-linear decision boundaries.
            \end{itemize}
        \item \textbf{Example:} In a movie recommendation system, k-NN recommends movies to a user based on similar viewing histories of other users.
    \end{itemize}
\end{frame}

\begin{frame}[fragile]
    \frametitle{Comparison of Decision Trees and k-NN - Side-by-Side Comparison}
    \begin{tabular}{|l|l|l|}
        \hline
        \textbf{Feature} & \textbf{Decision Trees} & \textbf{k-NN} \\ \hline
        Model Type & Tree-based & Instance-based \\ \hline
        Training Time & Quick (build tree structure once) & Slow (needs to compute distances on the fly) \\ \hline
        Storage Requirement & Low (tree structure) & High (stores all instances) \\ \hline
        Interpretability & High (easy to visualize and explain) & Low (difficult to interpret) \\ \hline
        Data Type & Best for categorical data & Best for numerical/continuous data \\ \hline
        Scalability & Scales well for larger datasets & Computationally expensive with large datasets \\ \hline
        Robustness & Sensitive to noise and overfitting & Sensitive to irrelevant features and scale \\ \hline
    \end{tabular}
\end{frame}

\begin{frame}[fragile]
    \frametitle{Comparison of Decision Trees and k-NN - Key Points}
    \begin{itemize}
        \item \textbf{Select based on Data Structure:} 
            \begin{itemize}
                \item Decision Trees for categorical and structured data.
                \item k-NN for datasets with complex, non-linear relationships.
            \end{itemize}
        \item \textbf{Interpretability vs. Accuracy:} 
            \begin{itemize}
                \item Decision Trees are more interpretable.
                \item k-NN may produce better accuracy for certain distributions.
            \end{itemize}
        \item \textbf{Performance Factors:} 
            \begin{itemize}
                \item Consider dataset size; large datasets may lead to inefficiencies with k-NN.
            \end{itemize}
    \end{itemize}
\end{frame}

\begin{frame}[fragile]
    \frametitle{Conclusion}
    In summary, both Decision Trees and k-NN have their advantages and limitations. Understand the specific requirements of your data and the task to select the appropriate algorithm for classification tasks.
\end{frame}

\begin{frame}[fragile]
    \frametitle{Evaluation Metrics for Classification - Introduction}
    \begin{itemize}
        \item Understanding model performance is crucial in classification problems.
        \item Unlike regression, classification requires specific metrics for evaluation.
        \item Key evaluation metrics include:
        \begin{itemize}
            \item Accuracy
            \item Precision
            \item Recall
            \item F1 Score
        \end{itemize}
    \end{itemize}
\end{frame}

\begin{frame}[fragile]
    \frametitle{Evaluation Metrics for Classification - Accuracy}
    \begin{block}{Definition}
        Accuracy is the ratio of correctly predicted instances to the total instances.
    \end{block}
    
    \begin{block}{Formula}
        \begin{equation}
        \text{Accuracy} = \frac{\text{True Positives} + \text{True Negatives}}{\text{Total Instances}}
        \end{equation}
    \end{block}
    
    \begin{block}{Example}
        In a medical test for a disease:
        \begin{itemize}
            \item True Positives (TP): 70
            \item True Negatives (TN): 20
            \item False Positives (FP): 5
            \item False Negatives (FN): 5
        \end{itemize}
        \[
        \text{Accuracy} = \frac{70 + 20}{70 + 20 + 5 + 5} = \frac{90}{100} = 0.90 \text{ or } 90\%
        \]
    \end{block}
\end{frame}

\begin{frame}[fragile]
    \frametitle{Evaluation Metrics for Classification - Precision, Recall, and F1 Score}
    
    \begin{block}{Precision}
        \begin{itemize}
            \item Definition: Correctness of positive predictions.
            \item Formula:
            \[
            \text{Precision} = \frac{\text{True Positives}}{\text{True Positives} + \text{False Positives}}
            \]
            \item Example: 
            \[
            \text{Precision} = \frac{70}{70 + 5} = 0.9333 \text{ or } 93.33\%
            \]
        \end{itemize}
    \end{block}
    
    \begin{block}{Recall}
        \begin{itemize}
            \item Definition: Ability to identify all relevant cases.
            \item Formula:
            \[
            \text{Recall} = \frac{\text{True Positives}}{\text{True Positives} + \text{False Negatives}}
            \]
            \item Example:
            \[
            \text{Recall} = \frac{70}{70 + 5} = 0.9333 \text{ or } 93.33\%
            \]
        \end{itemize}
    \end{block}
    
    \begin{block}{F1 Score}
        \begin{itemize}
            \item Definition: Combines Precision and Recall.
            \item Formula:
            \[
            \text{F1 Score} = 2 \times \frac{\text{Precision} \times \text{Recall}}{\text{Precision} + \text{Recall}}
            \]
            \item Example:
            \[
            \text{F1 Score} = 0.9333
            \]
        \end{itemize}
    \end{block}
\end{frame}

\begin{frame}[fragile]
    \frametitle{Conclusion and Key Takeaways - Overview}
    \begin{itemize}
        \item Explored two prominent classification algorithms: 
        \textbf{Decision Trees} and \textbf{k-Nearest Neighbors (k-NN)}.
        \item Essential tools in the data scientist's toolkit for predictive modeling.
        \item Understanding strengths and weaknesses is crucial for effective application.
    \end{itemize}
\end{frame}

\begin{frame}[fragile]
    \frametitle{Conclusion and Key Takeaways - Decision Trees}
    \begin{itemize}
        \item \textbf{Concept}: Flowchart-like structure for classification.
        \item \textbf{Key Characteristics}:
        \begin{itemize}
            \item \textbf{Interpretability}: Easy to understand for both technical and non-technical stakeholders.
            \item \textbf{Structure}: Handles both numerical and categorical data.
        \end{itemize}
        \item \textbf{Example}: Predicting customer purchase based on age and income.
    \end{itemize}
\end{frame}

\begin{frame}[fragile]
    \frametitle{Conclusion and Key Takeaways - k-NN and Key Points}
    \begin{itemize}
        \item \textbf{Concept}: Non-parametric, instance-based learning algorithm.
        \item \textbf{Key Characteristics}:
        \begin{itemize}
            \item \textbf{Simplicity}: Easy implementation and understanding.
            \item \textbf{Flexibility}: Works well with multi-class classification.
        \end{itemize}
        \item \textbf{Key Takeaways}:
        \begin{enumerate}
            \item Complementary algorithms for different contexts.
            \item Manage overfitting in Decision Trees, and carefully choose $k$ in k-NN.
            \item Evaluate effectiveness using accuracy, precision, recall, and F1 score.
            \item Visualize data for better understanding of classification.
        \end{enumerate}
    \end{itemize}
\end{frame}

\begin{frame}[fragile]
    \frametitle{Conclusion and Key Takeaways - Final Thoughts}
    \begin{itemize}
        \item No single algorithm is superior for all tasks.
        \item Understand the nuances, strengths, and limitations of both algorithms.
        \item Mastery of these concepts empowers you to tackle real-world classification challenges effectively!
    \end{itemize}
\end{frame}


\end{document}