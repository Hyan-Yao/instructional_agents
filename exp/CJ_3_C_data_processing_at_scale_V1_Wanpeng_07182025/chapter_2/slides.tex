\documentclass[aspectratio=169]{beamer}

% Theme and Color Setup
\usetheme{Madrid}
\usecolortheme{whale}
\useinnertheme{rectangles}
\useoutertheme{miniframes}

% Additional Packages
\usepackage[utf8]{inputenc}
\usepackage[T1]{fontenc}
\usepackage{graphicx}
\usepackage{booktabs}
\usepackage{listings}
\usepackage{amsmath}
\usepackage{amssymb}
\usepackage{xcolor}
\usepackage{tikz}
\usepackage{pgfplots}
\pgfplotsset{compat=1.18}
\usetikzlibrary{positioning}
\usepackage{hyperref}

% Custom Colors
\definecolor{myblue}{RGB}{31, 73, 125}
\definecolor{mygray}{RGB}{100, 100, 100}
\definecolor{mygreen}{RGB}{0, 128, 0}
\definecolor{myorange}{RGB}{230, 126, 34}
\definecolor{mycodebackground}{RGB}{245, 245, 245}

% Set Theme Colors
\setbeamercolor{structure}{fg=myblue}
\setbeamercolor{frametitle}{fg=white, bg=myblue}
\setbeamercolor{title}{fg=myblue}
\setbeamercolor{section in toc}{fg=myblue}
\setbeamercolor{item projected}{fg=white, bg=myblue}
\setbeamercolor{block title}{bg=myblue!20, fg=myblue}
\setbeamercolor{block body}{bg=myblue!10}
\setbeamercolor{alerted text}{fg=myorange}

% Set Fonts
\setbeamerfont{title}{size=\Large, series=\bfseries}
\setbeamerfont{frametitle}{size=\large, series=\bfseries}
\setbeamerfont{caption}{size=\small}
\setbeamerfont{footnote}{size=\tiny}

% Code Listing Style
\lstdefinestyle{customcode}{
  backgroundcolor=\color{mycodebackground},
  basicstyle=\footnotesize\ttfamily,
  breakatwhitespace=false,
  breaklines=true,
  commentstyle=\color{mygreen}\itshape,
  keywordstyle=\color{blue}\bfseries,
  stringstyle=\color{myorange},
  numbers=left,
  numbersep=8pt,
  numberstyle=\tiny\color{mygray},
  frame=single,
  framesep=5pt,
  rulecolor=\color{mygray},
  showspaces=false,
  showstringspaces=false,
  showtabs=false,
  tabsize=2,
  captionpos=b
}
\lstset{style=customcode}

% Custom Commands
\newcommand{\hilight}[1]{\colorbox{myorange!30}{#1}}
\newcommand{\source}[1]{\vspace{0.2cm}\hfill{\tiny\textcolor{mygray}{Source: #1}}}
\newcommand{\concept}[1]{\textcolor{myblue}{\textbf{#1}}}
\newcommand{\separator}{\begin{center}\rule{0.5\linewidth}{0.5pt}\end{center}}

% Footer and Navigation Setup
\setbeamertemplate{footline}{
  \leavevmode%
  \hbox{%
  \begin{beamercolorbox}[wd=.3\paperwidth,ht=2.25ex,dp=1ex,center]{author in head/foot}%
    \usebeamerfont{author in head/foot}\insertshortauthor
  \end{beamercolorbox}%
  \begin{beamercolorbox}[wd=.5\paperwidth,ht=2.25ex,dp=1ex,center]{title in head/foot}%
    \usebeamerfont{title in head/foot}\insertshorttitle
  \end{beamercolorbox}%
  \begin{beamercolorbox}[wd=.2\paperwidth,ht=2.25ex,dp=1ex,center]{date in head/foot}%
    \usebeamerfont{date in head/foot}
    \insertframenumber{} / \inserttotalframenumber
  \end{beamercolorbox}}%
  \vskip0pt%
}

% Turn off navigation symbols
\setbeamertemplate{navigation symbols}{}

% Title Page Information
\title[Overview of Data Processing Tools and Techniques]{Week 2: Overview of Data Processing Tools and Techniques}
\author[J. Smith]{John Smith, Ph.D.}
\institute[University Name]{
  Department of Computer Science\\
  University Name\\
  \vspace{0.3cm}
  Email: email@university.edu\\
  Website: www.university.edu
}
\date{\today}

% Document Start
\begin{document}

\frame{\titlepage}

\begin{frame}[fragile]
    \frametitle{Introduction to Data Processing Tools}
    \begin{block}{Overview of Data Processing Tools in Criminal Justice}
        Data processing tools play a pivotal role by enabling professionals to efficiently collect, analyze, and interpret vast amounts of data. They are essential for:
    \end{block}
    \begin{itemize}
        \item \textbf{Data-Driven Decisions}: Making informed choices based on statistical evidence.
        \item \textbf{Efficiency}: Automating tedious data processes saves time.
        \item \textbf{Transparency and Accountability}: Ensures consistent practices in judicial processes.
    \end{itemize}
\end{frame}

\begin{frame}[fragile]
    \frametitle{Key Data Processing Tools}
    In this chapter, we will explore three powerful data processing tools commonly used in the field of criminal justice:
    \begin{enumerate}
        \item \textbf{R}
        \item \textbf{Python}
        \item \textbf{Tableau}
    \end{enumerate}
\end{frame}

\begin{frame}[fragile]
    \frametitle{Overview of R}
    \begin{block}{Description}
        R is a programming language and software environment dedicated to statistical computing and graphics.
    \end{block}
    \begin{block}{Use in Criminal Justice}
        \begin{itemize}
            \item Conducts statistical analysis of crime data (e.g., regression analysis).
            \item Visualizes data patterns through plots and graphs.
        \end{itemize}
    \end{block}
    \begin{block}{Example}
        \begin{lstlisting}[language=R]
        library(ggplot2)
        ggplot(data = crime_data, aes(x = Year, y = CrimeRate)) +
          geom_line() + 
          labs(title="Crime Rate Trends", x="Year", y="Crime Rate")
        \end{lstlisting}
    \end{block}
\end{frame}

\begin{frame}[fragile]
    \frametitle{Overview of Python}
    \begin{block}{Description}
        Python is a high-level programming language known for its readability and versatility.
    \end{block}
    \begin{block}{Use in Criminal Justice}
        \begin{itemize}
            \item Data manipulation and analysis via libraries like Pandas and NumPy.
            \item Machine learning applications for crime hotspot predictions.
        \end{itemize}
    \end{block}
    \begin{block}{Example}
        \begin{lstlisting}[language=Python]
        import pandas as pd
        crime_data = pd.read_csv('crime_data.csv')
        hotspot = crime_data.groupby('Location').size().sort_values(ascending=False)
        print(hotspot.head(10))
        \end{lstlisting}
    \end{block}
\end{frame}

\begin{frame}[fragile]
    \frametitle{Overview of Tableau}
    \begin{block}{Description}
        Tableau is a powerful data visualization tool that helps users create interactive graphs and dashboards.
    \end{block}
    \begin{block}{Use in Criminal Justice}
        \begin{itemize}
            \item Presents findings to stakeholders in a visually engaging manner.
            \item Combines data from multiple sources for comprehensive analysis.
        \end{itemize}
    \end{block}
    \begin{block}{Example}
        Creation of a dashboard displaying crime statistics for better insights into the impact of various initiatives.
    \end{block}
\end{frame}

\begin{frame}[fragile]{Learning Objectives - Part 1}
    \frametitle{Learning Objectives for Week 2}

    This week's learning objectives are designed to provide you with a solid foundation in the tools and techniques essential for data processing, particularly within the context of criminal justice. By the end of this week, you should be able to:
    
    \begin{enumerate}
        \item \textbf{Understand Key Data Processing Tools}
            \begin{itemize}
                \item \textbf{R}: A programming language and environment suitable for statistical computing and graphics. You will learn why R is favored in academic and professional settings for data analysis.
                \item \textbf{Python}: Known for its readability and versatility, Python is widely used for data manipulation and analysis. Understanding its libraries will enhance your ability to work with large datasets.
                \item \textbf{Tableau}: A powerful visualization tool that helps in creating interactive data visualizations. You will grasp how this can aid in conveying complex information effectively.
            \end{itemize}
    \end{enumerate}
\end{frame}

\begin{frame}[fragile]{Learning Objectives - Part 2}
    \frametitle{Learning Objectives for Week 2 (Continued)}

    \begin{enumerate}
        \setcounter{enumi}{1} % Resuming enumeration from the previous frame
        \item \textbf{Execute Basic Commands in R and Python}
            \begin{itemize}
                \item In \textbf{R}:
                    \begin{lstlisting}[language=R]
summary(data)  # Provides a statistical summary of a dataset
                    \end{lstlisting}
                    Key Functionality: Data manipulation using dplyr, and data visualization using ggplot2.

                \item In \textbf{Python}:
                    \begin{lstlisting}[language=Python]
df.describe()  # Returns a summary of statistics for a DataFrame
                    \end{lstlisting}
                    Key Libraries: Pandas for data manipulation, Matplotlib, and Seaborn for visualization.
                    \item \textit{Practice executing these commands and understanding their output for real datasets.}
            \end{itemize}
        
        \item \textbf{Apply Data Analysis Techniques}
            \begin{itemize}
                \item \textbf{Descriptive Analysis}: Learn to summarize and describe data characteristics (e.g., means, medians, modes, standard deviations).
                
                \item \textbf{Inferential Analysis}: Gain insights about a population based on a sample - Techniques to Explore: Hypothesis testing and correlation analysis.
                
                \item \textit{Case Study Application}: Analyze a dataset from the criminal justice sector to identify trends and potential biases.
            \end{itemize}
    \end{enumerate}
\end{frame}

\begin{frame}[fragile]{Key Points and Conclusion}
    \frametitle{Key Points and Conclusion}

    \begin{block}{Key Points to Emphasize}
        - Mastering these tools will empower you to manipulate and analyze data effectively, enabling insightful decision-making in criminal justice contexts.
        - Practical, hands-on experience with R and Python commands will build your confidence in data processing.
    \end{block}

    \begin{block}{Conclusion}
        This week will set the stage for your journey into data-driven analysis. Embrace these learning objectives as your guide to developing proficiency with essential data processing tools and techniques.
    \end{block}

    \textit{Note: Please ensure you practice using these commands and techniques throughout the week. Hands-on experience is crucial for solidifying your understanding of data processing in R and Python.}
\end{frame}

\begin{frame}
    \frametitle{Data Processing Fundamentals}
    \begin{block}{Definition}
        Data processing refers to the collection, transformation, analysis, and presentation of data to extract meaningful information. In the context of large-scale datasets, especially in criminal justice, efficient data processing is crucial for supporting decision-making, identifying patterns, and improving accountability.
    \end{block}
\end{frame}

\begin{frame}
    \frametitle{Key Concepts in Data Processing}
    \begin{enumerate}
        \item \textbf{Data Collection}:
        \begin{itemize}
            \item Gathering raw data from various sources, such as police records, court documents, or social media.
            \item \textit{Example}: A police department collects data on reported crimes, including location, type, and time of occurrence.
        \end{itemize}
        
        \item \textbf{Data Transformation}:
        \begin{itemize}
            \item Converting raw data into a usable format. This often involves cleaning, normalizing, or aggregating data.
            \item \textit{Example}: Removing duplicate records or correcting typographical errors in crime reports to ensure accuracy.
        \end{itemize}
        
        \item \textbf{Data Analysis}:
        \begin{itemize}
            \item Applying statistical and analytical techniques to understand trends and relationships within the data.
            \item \textit{Example}: Analyzing crime rates over time to assess the effectiveness of a new community policing strategy.
        \end{itemize}

        \item \textbf{Data Presentation}:
        \begin{itemize}
            \item Displaying results in a comprehensible format, such as dashboards, reports, or visualizations.
            \item \textit{Example}: Creating a heat map of crime hotspots to visualize areas with high criminal activity for policy-making.
        \end{itemize}
    \end{enumerate}
\end{frame}

\begin{frame}
    \frametitle{Relevance in Criminal Justice}
    \begin{itemize}
        \item \textbf{Informed Decision-Making}: Ensures that law enforcement and justice officials make decisions based on factual insights.
        \item \textbf{Predictive Policing}: Techniques can anticipate criminal activities and allocate resources effectively, enhancing public safety.
        \item \textbf{Accountability and Transparency}: Better monitoring of crime trends and performance leads to improved accountability.
    \end{itemize}
\end{frame}

\begin{frame}[fragile]
    \frametitle{Sample Code Snippet (Python)}
    \begin{lstlisting}[language=Python]
import pandas as pd

# Collecting data
data = pd.read_csv('crime_data.csv')

# Data Transformation: Cleaning and Normalizing
data['crime_type'] = data['crime_type'].str.lower()

# Data Analysis: Grouping by crime type
crime_counts = data['crime_type'].value_counts()

# Data Presentation: Displaying results
print(crime_counts)
    \end{lstlisting}
\end{frame}

\begin{frame}
    \frametitle{Conclusion}
    \begin{block}{Summary}
        Data processing is not just a technical requirement; it's an essential skill that enables criminal justice professionals to derive actionable insights from large datasets. Understanding these fundamental concepts sets the stage for deeper engagement with data processing tools in upcoming sessions.
    \end{block}
\end{frame}

\begin{frame}[fragile]
    \frametitle{Introduction to R - Overview}
    \begin{block}{Overview of R as a Data Processing Tool}
        R is a powerful programming language and environment specifically designed for statistical computing and data analysis. It is widely used in various fields, including data science, bioinformatics, and social sciences, due to its flexibility, rich set of packages, and extensive community support.
    \end{block}
\end{frame}

\begin{frame}[fragile]
    \frametitle{Introduction to R - Key Features}
    \begin{itemize}
        \item \textbf{Open-source:} R is free to download and use, promoting widespread accessibility.
        \item \textbf{Comprehensive:} Provides a wide array of statistical and graphical techniques.
        \item \textbf{Extensible:} Users can create and share packages that enhance R's capabilities.
    \end{itemize}
\end{frame}

\begin{frame}[fragile]
    \frametitle{Introduction to R - Common Functions}
    Here are a few fundamental R functions:
    \begin{itemize}
        \item \texttt{read.csv()} - Imports CSV files into R.
        \begin{lstlisting}
data <- read.csv("datafile.csv")
        \end{lstlisting}
        
        \item \texttt{summary()} - Provides a statistical summary of a dataset.
        \begin{lstlisting}
summary(data)
        \end{lstlisting}
        
        \item \texttt{str()} - Displays the structure of the data frame.
        \begin{lstlisting}
str(data)
        \end{lstlisting}
        
        \item Basic statistics: \texttt{mean(), median(), sd()}.
        \begin{lstlisting}
avg_value <- mean(data$column_name)
        \end{lstlisting}
    \end{itemize}
\end{frame}

\begin{frame}[fragile]
    \frametitle{Introduction to R - Common Packages}
    Key packages for data analysis in R:
    \begin{itemize}
        \item \texttt{dplyr} - Used for data manipulation.
        \begin{lstlisting}
library(dplyr)
filtered_data <- data %>% filter(column_name > value)
        \end{lstlisting}
        
        \item \texttt{ggplot2} - A powerful package for data visualization.
        \begin{lstlisting}
library(ggplot2)
ggplot(data, aes(x=column_x, y=column_y)) + geom_point()
        \end{lstlisting}
        
        \item \texttt{tidyr} - Assists in tidying data.
        \begin{lstlisting}
library(tidyr)
tidy_data <- gather(data, key, value, -column_to_exclude)
        \end{lstlisting}
    \end{itemize}
\end{frame}

\begin{frame}[fragile]
    \frametitle{Introduction to R - Example Application}
    \begin{block}{Example Application in Criminal Justice}
        Consider a dataset of crime statistics in a city. R can be employed to analyze trends and visualize results.
    \end{block}
    \begin{enumerate}
        \item \textbf{Import Data:}
        \begin{lstlisting}
crime_data <- read.csv("crime_data.csv")
        \end{lstlisting}
        
        \item \textbf{Data Summary:}
        \begin{lstlisting}
summary(crime_data)
        \end{lstlisting}
        
        \item \textbf{Visualization:}
        \begin{lstlisting}
ggplot(crime_data, aes(x=Year, y=CrimeRate)) + geom_line()
        \end{lstlisting}
    \end{enumerate}
\end{frame}

\begin{frame}[fragile]
    \frametitle{Introduction to R - Key Points}
    \begin{itemize}
        \item R is essential for statisticians and data analysts due to its broad functionalities.
        \item Mastering R's packages enhances analytical capabilities.
        \item Hands-on experience with R functions and packages is invaluable for practical learning.
    \end{itemize}
\end{frame}

\begin{frame}[fragile]
    \frametitle{Introduction to R - Conclusion}
    R serves as a fundamental tool for data analysis, especially in the context of large-scale datasets like those in criminal justice. Familiarity with its functions and packages is crucial for effective data processing.
\end{frame}

\begin{frame}[fragile]
    \frametitle{Basic Commands in R - Overview}
    R is a powerful tool for data processing and analysis. It offers a variety of commands that simplify tasks such as:
    \begin{itemize}
        \item Importing data
        \item Creating and manipulating data frames
        \item Performing basic statistical operations
    \end{itemize}
    Understanding these commands is essential for efficient data analysis.
\end{frame}

\begin{frame}[fragile]
    \frametitle{Basic Commands in R - Key Commands}
    \begin{enumerate}
        \item \textbf{Importing Data}:
        \begin{lstlisting}[language=R]
        data <- read.csv("data_file.csv")
        \end{lstlisting}

        \item \textbf{Creating Data Frames}:
        \begin{lstlisting}[language=R]
        df <- data.frame(
          Name = c("Alice", "Bob", "Charlie"),
          Age = c(25, 30, 35),
          Gender = c("F", "M", "M")
        )
        \end{lstlisting}
    \end{enumerate}
\end{frame}

\begin{frame}[fragile]
    \frametitle{Basic Commands in R - Data Access and Stats}
    \begin{enumerate}[resume]
        \item \textbf{Accessing Data Frame Elements}:
        \begin{itemize}
            \item Access a column by name: 
            \begin{lstlisting}[language=R]
            df$Age  # Retrieves the Age column
            \end{lstlisting}
            \item Access a specific row: 
            \begin{lstlisting}[language=R]
            df[1, ]  # Retrieves the first row
            \end{lstlisting}
        \end{itemize}

        \item \textbf{Basic Statistical Operations}:
        \begin{itemize}
            \item Mean: 
            \begin{lstlisting}[language=R]
            mean(df$Age)  # Calculates the average age
            \end{lstlisting}
            \item Standard Deviation: 
            \begin{lstlisting}[language=R]
            sd(df$Age)  # Calculates the standard deviation of ages
            \end{lstlisting}
            \item Summary Statistics: 
            \begin{lstlisting}[language=R]
            summary(df)  # Provides summary of each column in df
            \end{lstlisting}
        \end{itemize}
    \end{enumerate}
\end{frame}

\begin{frame}[fragile]
    \frametitle{Basic Commands in R - Data Manipulation}
    \begin{itemize}
        \item \textbf{Data Manipulation}:
        Use the \texttt{dplyr} package for advanced data manipulation (installation needed):
        \begin{lstlisting}[language=R]
        install.packages("dplyr")
        library(dplyr)

        # Filter data for ages greater than 28
        filtered_data <- filter(df, Age > 28)
        \end{lstlisting}
    \end{itemize}
\end{frame}

\begin{frame}[fragile]
    \frametitle{Basic Commands in R - Summary and Final Thoughts}
    \begin{itemize}
        \item Familiarity with basic R commands enhances your ability to manipulate and analyze data effectively.
        \item Key skills include importing data, creating data frames, and performing statistical operations.
        \item Start practicing these commands to become proficient in handling datasets.
    \end{itemize}
    \textbf{Next Steps:} Overview of Python capabilities alongside R.
\end{frame}

\begin{frame}
    \frametitle{Introduction to Python}
    \begin{block}{Overview of Python as a Data Processing Tool}
        Python is a powerful and versatile programming language that has gained immense popularity in data analysis and processing due to its readability and ease of use. It is particularly well-suited for both beginners and advanced users in data science, thanks to its vast ecosystem of libraries and frameworks.
    \end{block}
\end{frame}

\begin{frame}
    \frametitle{Key Capabilities of Python}
    \begin{itemize}
        \item \textbf{Ease of Learning}: Python’s syntax is clear and straightforward, making it an accessible option for those new to programming and data analysis.
        \item \textbf{Integrative Nature}: Python can be easily integrated with other languages like R, SQL, and Java, making it a collaborative tool for diverse data processing tasks.
        \item \textbf{Portability}: Python runs on various operating systems (Windows, macOS, Linux), making it versatile for different working environments.
        \item \textbf{Community Support}: A large and active community provides robust support and a wealth of tutorials, making troubleshooting and learning easier.
    \end{itemize}
\end{frame}

\begin{frame}[fragile]
    \frametitle{Essential Libraries for Data Processing - Pandas}
    \begin{itemize}
        \item \textbf{Purpose}: Primarily used for data manipulation and analysis.
        \item \textbf{Key Features}:
        \begin{itemize}
            \item \textbf{Data Structures}: Introduces powerful data structures like Series and DataFrames, which simplify data organization.
            \item \textbf{Data Handling}: Provides methods for reading/writing data in formats like CSV, Excel, etc.
        \end{itemize}
    \end{itemize}
    \begin{block}{Example}
    \begin{lstlisting}[language=Python]
import pandas as pd
# Loading a CSV file into a DataFrame
df = pd.read_csv('data.csv')
print(df.head())  # Displays the first 5 rows of the DataFrame
    \end{lstlisting}
    \end{block}
\end{frame}

\begin{frame}[fragile]
    \frametitle{Essential Libraries for Data Processing - NumPy and SciPy}
    \begin{enumerate}
        \item \textbf{NumPy}:
        \begin{itemize}
            \item \textbf{Purpose}: Used for numerical computations and handling arrays.
            \item \textbf{Key Features}:
            \begin{itemize}
                \item \textbf{N-dimensional Arrays}: Supports powerful array objects for efficient data manipulation.
                \item \textbf{Mathematical Functions}: Contains a wide array of mathematical functions for operations on arrays.
            \end{itemize}
        \end{itemize}
        \begin{block}{Example}
        \begin{lstlisting}[language=Python]
import numpy as np
# Creating a NumPy array
arr = np.array([1, 2, 3, 4])
print(np.mean(arr))  # Calculates the mean
        \end{lstlisting}
        \end{block}
        
        \item \textbf{SciPy}:
        \begin{itemize}
            \item \textbf{Purpose}: Builds on NumPy and provides additional functionality for scientific computing.
            \item \textbf{Key Features}:
            \begin{itemize}
                \item \textbf{Optimization}: Offers algorithms for optimization, integration, interpolation, eigenvalue problems, and more.
                \item \textbf{Statistical Functions}: Contains modules for statistical analysis.
            \end{itemize}
        \end{itemize}
        \begin{block}{Example}
        \begin{lstlisting}[language=Python]
from scipy import stats
# Performing a t-test
t_stat, p_value = stats.ttest_1samp(arr, 1.5)
print(f't-statistic: {t_stat}, p-value: {p_value}')
        \end{lstlisting}
        \end{block}
    \end{enumerate}
\end{frame}

\begin{frame}
    \frametitle{Key Points to Emphasize}
    \begin{itemize}
        \item \textbf{Interoperability}: Python's libraries can work together seamlessly.
        \item \textbf{Data Handling Techniques}: Emphasize the difference in data handling capabilities between Python and other languages like R.
        \item \textbf{Community and Resources}: Utilize online resources and communities for support.
    \end{itemize}
\end{frame}

\begin{frame}
    \frametitle{Conclusion}
    Understanding Python and its libraries like Pandas, NumPy, and SciPy lays the groundwork for effective data analysis and manipulation. Prepare for the next slide, where we will delve into basic commands in Python to further enhance your data processing toolkit!
\end{frame}

\begin{frame}[fragile]
    \frametitle{Basic Commands in Python - Overview}
    In this section, we will explore basic commands in Python that are essential for data manipulation and analysis. 
    We will use libraries like Pandas and NumPy to handle data, particularly focusing on:
    \begin{itemize}
        \item Loading data
        \item Working with DataFrames
        \item Executing statistical methods
    \end{itemize}
\end{frame}

\begin{frame}[fragile]
    \frametitle{Basic Commands in Python - Loading Data}
    The first step in data analysis is to load your data into Python. 
    We commonly use the Pandas library for this purpose.
    
    \begin{block}{Example Code}
    \begin{lstlisting}[language=Python]
import pandas as pd

# Load a CSV file into a DataFrame
data = pd.read_csv('datafile.csv')

# Display the first few rows of the DataFrame
print(data.head())
    \end{lstlisting}
    \end{block}

    Here, \texttt{pd.read\_csv()} reads a CSV file and converts it into a DataFrame. 
    The \texttt{head()} method shows the first five rows of the loaded data.
\end{frame}

\begin{frame}[fragile]
    \frametitle{Basic Commands in Python - DataFrames and Statistics}
    \textbf{Working with DataFrames}

    A DataFrame is a two-dimensional, size-mutable, and potentially heterogeneous tabular data structure with labeled axes (rows and columns).
    
    \begin{itemize}
        \item \textbf{View DataFrame Shape:}
        \begin{lstlisting}[language=Python]
print(data.shape)  # Outputs the number of rows and columns
        \end{lstlisting}

        \item \textbf{Selecting a Column:}
        \begin{lstlisting}[language=Python]
ages = data['age']  # Extracts the 'age' column
        \end{lstlisting}

        \item \textbf{Filtering Data:}
        \begin{lstlisting}[language=Python]
filtered_data = data[data['age'] > 30]  # Filters rows where age > 30
        \end{lstlisting}
    \end{itemize}

    \textbf{Executing Statistical Methods}

    Python provides a variety of statistical functions through the Pandas library.
    
    \begin{itemize}
        \item \textbf{Mean:}
        \begin{lstlisting}[language=Python]
mean_age = data['age'].mean()  # Calculates the average age
        \end{lstlisting}
        \item \textbf{Standard Deviation:}
        \begin{lstlisting}[language=Python]
std_age = data['age'].std()  # Calculates the standard deviation of ages
        \end{lstlisting}
        \item \textbf{Descriptive Statistics:}
        \begin{lstlisting}[language=Python]
stats_summary = data.describe()  # Generates a summary of key statistics
        \end{lstlisting}
    \end{itemize}
    
    \textbf{Key Point:} Always consider the context of your analysis before diving deeper.
\end{frame}

\begin{frame}[fragile]
    \frametitle{Introduction to Tableau}
    Overview of Tableau for data visualization and reporting in the context of criminal justice data.
\end{frame}

\begin{frame}[fragile]
    \frametitle{Overview of Tableau}
    \begin{itemize}
        \item Tableau is a powerful data visualization tool that creates interactive and shareable dashboards.
        \item It connects to various data sources, transforming raw data into understandable visual formats.
        \item In criminal justice, Tableau provides essential insights aiding informed decision-making.
    \end{itemize}
\end{frame}

\begin{frame}[fragile]
    \frametitle{Key Concepts of Tableau}
    \begin{itemize}
        \item \textbf{Data Connectivity:} Imports data from sources like Excel, SQL databases, and cloud services.
        \item \textbf{Visualization:} Utilizes a drag-and-drop interface for creating complex graphics without coding.
        \item \textbf{Interactivity:} Interactive features for exploring data dimensions through filtering and highlighting.
        \item \textbf{Sharing and Collaboration:} Dashboards can be published for real-time collaboration with stakeholders.
    \end{itemize}
\end{frame}

\begin{frame}[fragile]
    \frametitle{Examples in Criminal Justice}
    \begin{enumerate}
        \item \textbf{Crime Rate Analysis:} Visualizes crime rates over time to identify trends.
        \item \textbf{Incident Reports Visualization:} Geographically visualizes incidents to pinpoint hotspots.
        \item \textbf{Recidivism Studies:} Analyzes recidivism data across demographics for targeted interventions.
    \end{enumerate}
\end{frame}

\begin{frame}[fragile]
    \frametitle{Key Points to Emphasize}
    \begin{itemize}
        \item \textbf{User-Friendly Interface:} Accessible for users with limited technical skills.
        \item \textbf{Real-Time Data Reinforcement:} Supports immediate data analysis crucial for decision-making.
        \item \textbf{Beautiful Visualizations:} Conveys complex data clearly, enhancing stakeholder understanding.
    \end{itemize}
\end{frame}

\begin{frame}[fragile]
    \frametitle{Conclusion}
    In summary, Tableau is essential for transforming complex criminal justice data into visual representations.
    \begin{block}{Enhancing Policy Making}
        By familiarizing yourself with Tableau, you can contribute to more informed policies and practices in criminal justice.
        \newline
        Next slide: Basic Visualization Techniques in Tableau.
    \end{block}
\end{frame}

\begin{frame}[fragile]
    \frametitle{Basic Visualization Techniques in Tableau - Introduction}
    \begin{block}{Introduction to Visualization in Tableau}
        Tableau is a powerful tool for creating visualizations to explore and present data effectively. 
        Effective data visualization helps in identifying patterns, trends, and insights within raw data.
    \end{block}
\end{frame}

\begin{frame}[fragile]
    \frametitle{Basic Visualization Techniques - Concepts}
    \begin{block}{Basic Visualization Techniques}
        \begin{enumerate}
            \item \textbf{Creating a Worksheet}
                \begin{itemize}
                    \item Open Tableau and connect to your data source (e.g., Excel, CSV).
                    \item Click on the "Sheet" tab to create a new worksheet.
                \end{itemize}
            \item \textbf{Using Dimensions and Measures}
                \begin{itemize}
                    \item \textbf{Dimensions}: Qualitative fields (e.g., categories).
                    \item \textbf{Measures}: Quantitative fields (e.g., totals).
                    \item Drag a dimension (e.g., "Offense Type") to Rows and a measure (e.g., "Number of Incidents") to Columns.
                \end{itemize}
        \end{enumerate}
    \end{block}
\end{frame}

\begin{frame}[fragile]
    \frametitle{Basic Visualization Types}
    \begin{block}{Types of Basic Visualizations}
        \begin{itemize}
            \item \textbf{Bar Charts}: Compare values across categories 
                \begin{itemize}
                    \item Example: Crimes per category.
                \end{itemize}
            \item \textbf{Line Charts}: Display trends over time 
                \begin{itemize}
                    \item Example: Crime rates over the past decade.
                \end{itemize}
            \item \textbf{Pie Charts}: Show proportions 
                \begin{itemize}
                    \item Example: Percentage of total crimes by type.
                \end{itemize}
            \item \textbf{Maps}: Visualize geographic data 
                \begin{itemize}
                    \item Example: Crime hotspots in a city.
                \end{itemize}
        \end{itemize}
    \end{block}
\end{frame}

\begin{frame}[fragile]
    \frametitle{Creating a Bar Chart}
    \begin{block}{Steps to Create a Bar Chart}
        \begin{enumerate}
            \item Drag “Offense Type” to Rows shelf.
            \item Drag “Number of Incidents” to Columns shelf.
            \item Tableau generates a bar chart.
            \item Customize the chart:
                \begin{itemize}
                    \item Add colors to differentiate categories.
                    \item Sort bars in descending order.
                    \item Label the bars to show values directly.
                \end{itemize}
        \end{enumerate}
    \end{block}
\end{frame}

\begin{frame}[fragile]
    \frametitle{Key Features and Tips}
    \begin{block}{Key Features to Enhance Visuals}
        \begin{itemize}
            \item \textbf{Filters}: Narrow down data (e.g., by year).
            \item \textbf{Tooltips}: Additional insights on hover.
            \item \textbf{Dashboards}: Combine multiple visualizations.
        \end{itemize}
    \end{block}

    \begin{block}{Tips for Effective Visualizations}
        \begin{itemize}
            \item Keep it Simple: Avoid clutter.
            \item Use Color Wisely: Ensure visibility and consistency.
            \item Label Clearly: Include titles and axes.
            \item Know Your Audience: Tailor visuals accordingly.
        \end{itemize}
    \end{block}
\end{frame}

\begin{frame}[fragile]
    \frametitle{Conclusion and Example Code}
    \begin{block}{Conclusion}
        Mastering basic visualization techniques in Tableau enhances communication of statistical findings. Engaging with data visually promotes better understanding and action.
    \end{block}
    
    \begin{block}{Example Codes/Snippets in Tableau}
        To create calculated fields: 
        \begin{lstlisting}
IF [Offense Type] = 'Assault' THEN 1 ELSE 0 END
        \end{lstlisting}
        To create a trend line:
        \begin{itemize}
            \item Right-click on the visualization → “Add Trend Line.”
        \end{itemize}
    \end{block}
\end{frame}

\begin{frame}[fragile]
    \frametitle{Data Analysis Techniques - Overview}
    \begin{block}{Importance}
        Data analysis techniques are crucial for extracting meaningful insights from large datasets, especially in fields like criminal justice.
    \end{block}
    \begin{itemize}
        \item Identify patterns, trends, and relationships.
        \item Assist law enforcement and policymakers in decision-making.
    \end{itemize}
\end{frame}

\begin{frame}[fragile]
    \frametitle{Data Analysis Techniques - Key Statistical Methods}
    \begin{enumerate}
        \item \textbf{Descriptive Statistics}
            \begin{itemize}
                \item \textbf{Definition}: Summarizes and describes dataset features.
                \item \textbf{Common Measures}:
                    \begin{itemize}
                        \item Mean: Average value
                        \item Median: Middle value
                        \item Mode: Most frequently occurring value
                        \item Standard Deviation: Measure of variation
                    \end{itemize}
                \item \textbf{Example}: Average crime rate across districts.
            \end{itemize}
        \item \textbf{Inferential Statistics}
            \begin{itemize}
                \item \textbf{Definition}: Infers conclusions about a population from a sample.
                \item \textbf{Common Techniques}:
                    \begin{itemize}
                        \item Hypothesis Testing
                        \item Confidence Intervals
                    \end{itemize}
                \item \textbf{Example}: Analyzing crime rate changes due to new strategies.
            \end{itemize}
    \end{enumerate}
\end{frame}

\begin{frame}[fragile]
    \frametitle{Data Analysis Techniques - More Methods}
    \begin{enumerate}[resume]
        \item \textbf{Correlation and Regression Analysis}
            \begin{itemize}
                \item \textbf{Definition}: Examines relationships between variables.
                \item \textbf{Correlation Coefficient (r)}: Strength and direction of relationship.
                \item \textbf{Linear Regression}: Models relationships between dependent and independent variables.
                \item \textbf{Example}: Correlation between unemployment and property crime.
            \end{itemize}
        \item \textbf{Time Series Analysis}
            \begin{itemize}
                \item \textbf{Definition}: Analyzes data collected over time.
                \item \textbf{Techniques}:
                    \begin{itemize}
                        \item Moving Averages
                        \item ARIMA Models
                    \end{itemize}
                \item \textbf{Example}: Forecasting future crime trends with past monthly data.
            \end{itemize}
    \end{enumerate}
\end{frame}

\begin{frame}[fragile]
    \frametitle{Data Analysis Techniques - Real-World Applications}
    \begin{itemize}
        \item \textbf{Crime Prediction}: Identify future hotspots and resource allocation.
        \item \textbf{Recidivism Analysis}: Factors contributing to repeat offenses for better rehabilitation.
        \item \textbf{Risk Assessment}: Evaluating the likelihood of reoffending to inform judicial decisions.
    \end{itemize}

    \begin{block}{Summary Points}
        \begin{itemize}
            \item Enables insights into large datasets.
            \item Influences policy and operational decisions.
            \item Leads to proactive measures in crime prevention.
        \end{itemize}
    \end{block}
\end{frame}

\begin{frame}[fragile]
    \frametitle{Key Statistical Formulas}
    \begin{itemize}
        \item \textbf{Mean}: 
        \begin{equation}
            \text{Mean} = \frac{\Sigma x}{n}
        \end{equation}
        \item \textbf{Standard Deviation}:
        \begin{equation}
            \sigma = \sqrt{\frac{\Sigma (x - \mu)^{2}}{N}}
        \end{equation}
        \item \textbf{Correlation (r)}:
        \begin{equation}
            r = \frac{\text{Cov}(X,Y)}{\sigma_{X} \sigma_{Y}}
        \end{equation}
    \end{itemize}
    \begin{block}{Conclusion}
        By applying these techniques, criminal justice professionals can enhance decision-making processes.
    \end{block}
\end{frame}

\begin{frame}[fragile]
    \frametitle{Ethical Considerations in Data Processing - Overview}
    In the age of big data and advanced analytics, ethical considerations hold paramount importance, especially in fields such as criminal justice. Handling data responsibly not only ensures compliance with laws but also builds trust within the community. 

    \begin{block}{Key Ethical Issues}
        \begin{itemize}
            \item Data Privacy and Consent
            \item Data Accuracy and Integrity
            \item Transparency and Accountability
            \item Non-discrimination and Fairness
        \end{itemize}
    \end{block}
\end{frame}

\begin{frame}[fragile]
    \frametitle{Ethical Considerations in Data Processing - Privacy Laws: Focus on GDPR}
    The General Data Protection Regulation (GDPR), enacted by the European Union in 2018, sets stringent guidelines for the collection and processing of personal data. Its implications extend to various sectors, including criminal justice.

    \begin{enumerate}
        \item \textbf{Rights of Individuals:}
            \begin{itemize}
                \item Right to Access: Individuals can request access to their data and how it is processed.
                \item Right to Erasure: Also known as the "right to be forgotten," individuals can request data deletion when it's no longer necessary.
                \item Right to Data Portability: Individuals can transfer their data from one service provider to another.
            \end{itemize}
        \item \textbf{Lawful Processing:}
            \begin{itemize}
                \item Data processing must have a legal basis, such as consent or legitimate interests.
                \item In criminal justice, processing of sensitive data must minimize risks to privacy.
            \end{itemize}
    \end{enumerate}
\end{frame}

\begin{frame}[fragile]
    \frametitle{Ethical Considerations in Data Processing - Implications for Criminal Justice}
    \begin{itemize}
        \item \textbf{Data Collection:} 
            \begin{itemize}
                \item Law enforcement agencies must ensure that they only collect data that is necessary and proportionate to their objectives.
                \item \textit{Example:} Collecting data about a suspect's criminal history only if it directly pertains to an ongoing investigation.
            \end{itemize}
        \item \textbf{Data Retention:}
            \begin{itemize}
                \item Agencies must have clear policies on how long data can be retained.
                \item \textit{Example:} Auto-deleting records that have exceeded the retention period to comply with GDPR.
            \end{itemize}
        \item \textbf{Data Sharing:}
            \begin{itemize}
                \item Collaboration must be transparent, with explicit guidelines on sharing personal data.
                \item \textit{Example:} Sharing crime data between police departments while respecting individuals' privacy rights.
            \end{itemize}
    \end{itemize}
\end{frame}

\begin{frame}[fragile]
    \frametitle{Integrating Technology in Data Processing - Overview}
    Integrating the right technology into data processing is essential for enhancing efficiency, accuracy, and effectiveness. This involves carefully selecting tools and techniques that align with specific processing tasks, organizational needs, and ethical considerations.
\end{frame}

\begin{frame}[fragile]
    \frametitle{Integrating Technology in Data Processing - Key Concepts}
    \begin{enumerate}
        \item \textbf{Importance of Technology in Data Processing}
            \begin{itemize}
                \item \textbf{Efficiency}: Automation of repetitive tasks saves time and reduces human error.
                \item \textbf{Scalability}: Handles increasing data volumes with minimal workload increase.
                \item \textbf{Insight Generation}: Tools can uncover patterns in data that are hard to identify manually.
            \end{itemize}
        \item \textbf{Selecting Appropriate Technological Solutions}
            \begin{itemize}
                \item \textbf{Needs Assessment}: Define objectives and expected outcomes.
                \item \textbf{Tool Evaluation}: Assess tools for features, user-friendliness, integration, and cost.
                \item \textbf{Compliance and Ethics}: Ensure tools comply with data privacy laws (e.g., GDPR).
            \end{itemize}
    \end{enumerate}
\end{frame}

\begin{frame}[fragile]
    \frametitle{Integrating Technology in Data Processing - Implementation and Evaluation}
    \begin{enumerate}
        \setcounter{enumi}{2}
        \item \textbf{Implementation Process}
            \begin{itemize}
                \item \textbf{Planning}: Outline deployment with timelines, resources, and responsibilities.
                \item \textbf{Training}: Offer training sessions for staff.
                \item \textbf{Pilot Testing}: Test technology on a smaller scale before a full rollout.
            \end{itemize}
        \item \textbf{Continuous Evaluation and Improvement}
            \begin{itemize}
                \item \textbf{Feedback Loop}: Regularly collect user feedback.
                \item \textbf{Updates and Maintenance}: Keep tools updated for new features and cybersecurity.
            \end{itemize}
    \end{enumerate}
\end{frame}

\begin{frame}[fragile]
    \frametitle{Integrating Technology in Data Processing - Example Scenario}
    \textbf{Scenario}: A criminal justice agency needs to analyze crime data to detect trends.
    
    \textbf{Selected Technology}:
    \begin{itemize}
        \item \textbf{Tool}: Tableau for visualization and reporting.
        \item \textbf{Methods}: Use SQL for database management and Python with Pandas for statistical analysis.
    \end{itemize}
    
    \textbf{Implementation Steps}:
    \begin{enumerate}
        \item Conduct training for officers on Tableau dashboards.
        \item Launch a pilot project analyzing a subset of crime data.
        \item Refine tools based on user feedback and rerun analyses.
    \end{enumerate}
\end{frame}

\begin{frame}[fragile]
    \frametitle{Integrating Technology in Data Processing - Key Points}
    \begin{itemize}
        \item Align technology selection with organizational goals and compliance requirements.
        \item Implementation plans should be adaptive for iterative improvements.
        \item Effective training is crucial for maximizing the potential of data processing technologies.
    \end{itemize}
    
    \textbf{Conclusion}: Integrating technology efficiently enhances data processing capabilities, supporting informed decision-making and fostering a data-driven culture that prioritizes accuracy and ethics.
\end{frame}

\begin{frame}[fragile]
    \frametitle{Collaborative Approaches to Data Analysis - Overview}
    \begin{itemize}
        \item Importance of interdisciplinary collaboration in criminal justice.
        \item Combining expertise from various fields to tackle complex data processing challenges.
        \item Enhancing insights and decision-making through collaborative data analysis.
    \end{itemize}
\end{frame}

\begin{frame}[fragile]
    \frametitle{Understanding Collaborative Data Analysis}
    \begin{block}{Key Points}
        \begin{itemize}
            \item Collaboration involves experts from multiple disciplines: law enforcement, criminology, data science, etc.
            \item Enriches data analysis and improves decision-making capabilities.
            \item Diverse perspectives enhance understanding of complex issues.
        \end{itemize}
    \end{block}
\end{frame}

\begin{frame}[fragile]
    \frametitle{Addressing Data Challenges in Criminal Justice}
    \begin{enumerate}
        \item **Complex Data Sets:**
            \begin{itemize}
                \item Includes crime reports, social media feeds, and demographic data.
            \end{itemize}
        \item **Challenging Questions:**
            \begin{itemize}
                \item Requires insights from various fields to analyze effectively.
            \end{itemize}
        \item **Limitations of Single Discipline:**
            \begin{itemize}
                \item Risk of overlooking key social, legal, and technological factors.
            \end{itemize}
    \end{enumerate}
\end{frame}

\begin{frame}[fragile]
    \frametitle{Enhancing Insight and Decision-Making}
    \begin{block}{Collaborative Insights}
        \begin{itemize}
            \item Combining quantitative and qualitative data improves understanding.
            \item Example: Data analysts and sociologists exploring socioeconomic factors and crime correlation.
        \end{itemize}
    \end{block}
\end{frame}

\begin{frame}[fragile]
    \frametitle{Practical Examples of Collaboration}
    \begin{enumerate}
        \item **Case Study 1: Gun Violence Analysis**
            \begin{itemize}
                \item Data scientists analyze patterns; public health experts provide community insights.
            \end{itemize}
        \item **Case Study 2: Predictive Policing**
            \begin{itemize}
                \item Law enforcement and data analysts develop tools; advocates ensure ethical standards.
            \end{itemize}
    \end{enumerate}
\end{frame}

\begin{frame}[fragile]
    \frametitle{Conclusion and Key Takeaway}
    \begin{block}{Conclusion}
        \begin{itemize}
            \item Interdisciplinary collaboration is essential in criminal justice data analysis.
            \item Professionals can create more efficient systems for crime prevention and community outreach.
        \end{itemize}
    \end{block}
    \begin{block}{Key Takeaway}
        Encouraging interdisciplinary frameworks improves data analysis quality and actions taken based on it.
    \end{block}
\end{frame}

\begin{frame}[fragile]
  \frametitle{Summary and Key Takeaways - Overview}
  \begin{block}{Importance of Data Processing}
    Data processing is vital in criminal justice for transforming raw data into valuable information that:
    \begin{itemize}
      \item Aids decision-making
      \item Enhances investigative processes
      \item Supports legal procedures
    \end{itemize}
  \end{block}
\end{frame}

\begin{frame}[fragile]
  \frametitle{Summary and Key Takeaways - Key Tools}
  \begin{enumerate}
    \item \textbf{Spreadsheets:}
    \begin{itemize}
      \item Used for data organization and basic statistical analysis.
      \item \textit{Example:} A police department tracks crime rates in Excel.
    \end{itemize}

    \item \textbf{Databases:}
    \begin{itemize}
      \item Essential for storing and managing large volumes of data.
      \item \textit{Example:} SQL to manage case files and evidence.
    \end{itemize}

    \item \textbf{Data Analysis Software:}
    \begin{itemize}
      \item R and Python for advanced statistical analysis.
      \item \textit{Code Snippet:}
      \begin{lstlisting}
      import pandas as pd
      data = pd.read_csv('crime_data.csv')
      summary = data.describe()
      print(summary)
      \end{lstlisting}
    \end{itemize}
  \end{enumerate}
\end{frame}

\begin{frame}[fragile]
  \frametitle{Summary and Key Takeaways - Techniques and Collaboration}
  \begin{block}{Techniques Utilized}
    \begin{itemize}
      \item \textbf{Data Cleaning:} Ensures data integrity before analysis.
      \item \textbf{Statistical Analysis:} Techniques to forecast crime hotspots.
      \begin{itemize}
        \item \textit{Practical Application:} Predictive analysis helps allocate resources effectively.
      \end{itemize}
    \end{itemize}
  \end{block}

  \begin{block}{Collaborative Approaches}
    Emphasizes the importance of interdisciplinary collaboration among:
    \begin{itemize}
      \item Law enforcement
      \item Data analysts
      \item Social scientists
    \end{itemize}
  \end{block}
\end{frame}

\begin{frame}[fragile]
  \frametitle{Summary and Key Takeaways - Final Thoughts}
  \begin{itemize}
    \item Data processing tools are crucial for effective crime analysis.
    \item Mastery of basic and advanced tools is essential for law enforcement.
    \item Accurate data cleaning and analysis impact prevention strategies.
    \item Collaboration enhances data processing outcomes in criminal justice.
  \end{itemize}

  \begin{block}{Conclusion}
    Engaging with data processing methods improves organizational efficiency and aids justice through informed decisions.
  \end{block}
\end{frame}

\begin{frame}[fragile]
    \frametitle{Questions and Discussion - Objective}
    \begin{block}{Objective}
        To facilitate an interactive session where students can seek clarification and deepen their understanding of data processing tools and techniques discussed in the previous slides, especially regarding their practical applications in the field of criminal justice.
    \end{block}
\end{frame}

\begin{frame}[fragile]
    \frametitle{Questions and Discussion - Encouraging Active Participation}
    \begin{itemize}
        \item \textbf{Open the Floor for Questions:}
            \begin{itemize}
                \item Invite students to ask questions about any topic covered in the chapter.
                \item Encourage specific questions related to data processing applications, tools, or techniques.
            \end{itemize}
        \item \textbf{Discussion Prompts:}
            \begin{itemize}
                \item “What was the most surprising data processing technique you learned? Why?”
                \item “Can you think of a scenario in criminal justice where data processing made a significant impact?”
                \item “What are some challenges associated with the data processing tools we've discussed?”
            \end{itemize}
    \end{itemize}
\end{frame}

\begin{frame}[fragile]
    \frametitle{Questions and Discussion - Key Concepts}
    \begin{block}{Clarifying Key Concepts}
        \begin{itemize}
            \item \textbf{Data Processing Tools:}
            \begin{itemize}
                \item Software and hardware used to collect, manipulate, and analyze data.
                \item Examples: Excel, SQL databases, Tableau.
            \end{itemize}
            \item \textbf{Techniques:}
            \begin{itemize}
                \item Methods such as data cleaning, data analysis, and data visualization.
            \end{itemize}
            \item \textbf{Application in Criminal Justice:}
            \begin{itemize}
                \item Tools support real-world investigations, evidence management, and crime trend analysis.
            \end{itemize}
        \end{itemize}
    \end{block}
\end{frame}

\begin{frame}[fragile]
    \frametitle{Questions and Discussion - Concrete Examples}
    \begin{itemize}
        \item \textbf{Example 1: Data Visualization in Crime Analysis}
            \begin{itemize}
                \item Discuss how GIS (Geographic Information Systems) visualize crime hotspots for law enforcement.
            \end{itemize}
        \item \textbf{Example 2: Data Cleaning Techniques}
            \begin{itemize}
                \item Importance of cleaning data before analysis, as seen with police reports or court records.
            \end{itemize}
        \item \textbf{Example 3: SQL Queries}
            \begin{lstlisting}[language=SQL]
SELECT * FROM arrests
WHERE crime_type = 'Theft'
AND date >= '2023-01-01';
            \end{lstlisting}
            \item Discuss the importance of querying data effectively.
    \end{itemize}
\end{frame}

\begin{frame}[fragile]
    \frametitle{Questions and Discussion - Conclusions}
    \begin{block}{Conclusion}
        Encourage students to share personal insights or experiences related to data processing. 
        Reaffirm the importance of mastering these tools and techniques for their future in criminal justice.
    \end{block}
\end{frame}

\begin{frame}[fragile]
    \frametitle{Facilitating the Discussion}
    \begin{itemize}
        \item \textbf{Listen Actively:} Validate each question and provide thoughtful responses.
        \item \textbf{Create a Safe Environment:} Encourage all students to share their understanding and misconceptions.
        \item \textbf{Guide the Conversation:} Steer discussions back to the chapter's objectives and applications as needed.
    \end{itemize}
\end{frame}


\end{document}