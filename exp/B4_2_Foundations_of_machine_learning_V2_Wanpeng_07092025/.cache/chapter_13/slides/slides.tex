\documentclass[aspectratio=169]{beamer}

% Theme and Color Setup
\usetheme{Madrid}
\usecolortheme{whale}
\useinnertheme{rectangles}
\useoutertheme{miniframes}

% Additional Packages
\usepackage[utf8]{inputenc}
\usepackage[T1]{fontenc}
\usepackage{graphicx}
\usepackage{booktabs}
\usepackage{listings}
\usepackage{amsmath}
\usepackage{amssymb}
\usepackage{xcolor}
\usepackage{tikz}
\usepackage{pgfplots}
\pgfplotsset{compat=1.18}
\usetikzlibrary{positioning}
\usepackage{hyperref}

% Custom Colors
\definecolor{myblue}{RGB}{31, 73, 125}
\definecolor{mygray}{RGB}{100, 100, 100}
\definecolor{mygreen}{RGB}{0, 128, 0}
\definecolor{myorange}{RGB}{230, 126, 34}
\definecolor{mycodebackground}{RGB}{245, 245, 245}

% Set Theme Colors
\setbeamercolor{structure}{fg=myblue}
\setbeamercolor{frametitle}{fg=white, bg=myblue}
\setbeamercolor{title}{fg=myblue}
\setbeamercolor{section in toc}{fg=myblue}
\setbeamercolor{item projected}{fg=white, bg=myblue}
\setbeamercolor{block title}{bg=myblue!20, fg=myblue}
\setbeamercolor{block body}{bg=myblue!10}
\setbeamercolor{alerted text}{fg=myorange}

% Set Fonts
\setbeamerfont{title}{size=\Large, series=\bfseries}
\setbeamerfont{frametitle}{size=\large, series=\bfseries}
\setbeamerfont{caption}{size=\small}
\setbeamerfont{footnote}{size=\tiny}

% Code Listing Style
\lstdefinestyle{customcode}{
  backgroundcolor=\color{mycodebackground},
  basicstyle=\footnotesize\ttfamily,
  breakatwhitespace=false,
  breaklines=true,
  commentstyle=\color{mygreen}\itshape,
  keywordstyle=\color{blue}\bfseries,
  stringstyle=\color{myorange},
  numbers=left,
  numbersep=8pt,
  numberstyle=\tiny\color{mygray},
  frame=single,
  framesep=5pt,
  rulecolor=\color{mygray},
  showspaces=false,
  showstringspaces=false,
  showtabs=false,
  tabsize=2,
  captionpos=b
}
\lstset{style=customcode}

% Custom Commands
\newcommand{\hilight}[1]{\colorbox{myorange!30}{#1}}
\newcommand{\source}[1]{\vspace{0.2cm}\hfill{\tiny\textcolor{mygray}{Source: #1}}}
\newcommand{\concept}[1]{\textcolor{myblue}{\textbf{#1}}}
\newcommand{\separator}{\begin{center}\rule{0.5\linewidth}{0.5pt}\end{center}}

% Footer and Navigation Setup
\setbeamertemplate{footline}{
  \leavevmode%
  \hbox{%
  \begin{beamercolorbox}[wd=.3\paperwidth,ht=2.25ex,dp=1ex,center]{author in head/foot}%
    \usebeamerfont{author in head/foot}\insertshortauthor
  \end{beamercolorbox}%
  \begin{beamercolorbox}[wd=.5\paperwidth,ht=2.25ex,dp=1ex,center]{title in head/foot}%
    \usebeamerfont{title in head/foot}\insertshorttitle
  \end{beamercolorbox}%
  \begin{beamercolorbox}[wd=.2\paperwidth,ht=2.25ex,dp=1ex,center]{date in head/foot}%
    \usebeamerfont{date in head/foot}
    \insertframenumber{} / \inserttotalframenumber
  \end{beamercolorbox}}%
  \vskip0pt%
}

% Turn off navigation symbols
\setbeamertemplate{navigation symbols}{}

% Title Page Information
\title[Week 13: Reinforcement Learning]{Week 13: Reinforcement Learning}
\author[J. Smith]{John Smith, Ph.D.}
\institute[University Name]{
  Department of Computer Science\\
  University Name\\
  \vspace{0.3cm}
  Email: email@university.edu\\
  Website: www.university.edu
}
\date{\today}

% Document Start
\begin{document}

\frame{\titlepage}

\begin{frame}[fragile]
  \titlepage
\end{frame}

\begin{frame}[fragile]
  \frametitle{Overview of Reinforcement Learning}
  
  \begin{block}{What is Reinforcement Learning (RL)?}
    Reinforcement Learning is a branch of machine learning focused on how agents ought to take actions in an environment to maximize cumulative rewards.
    Unlike other forms of machine learning, RL is characterized by its interactive approach, where an agent learns to make decisions through trial and error.
  \end{block}

\end{frame}

\begin{frame}[fragile]
  \frametitle{Key Concepts of Reinforcement Learning}

  \begin{enumerate}
    \item **Agent**: The learner or decision-maker (e.g., a robot, software bot, or game-playing AI).
    \item **Environment**: The setting in which the agent operates and interacts; it includes everything the agent can observe and act upon.
    \item **State ($s$)**: A specific situation in which the agent finds itself at any given time.
    \item **Action ($a$)**: Choices available to the agent that affect the state of the environment. 
    \item **Reward ($r$)**: Feedback from the environment after each action, guiding the agent's learning.
    \item **Policy ($\pi$)**: A strategy used by the agent to determine the next action based on the current state.
    \item **Value Function ($V$)**: A prediction of future rewards for each state, helping the agent evaluate how good it is to be in a given state.
  \end{enumerate}

\end{frame}

\begin{frame}[fragile]
  \frametitle{Illustration of RL Concepts}

  \begin{block}{Example}
    Consider a self-driving car (Agent) navigating a city (Environment). 
    The car observes traffic lights and road conditions (State), 
    decides to turn left or go straight (Action), 
    and receives a score based on speed and safety (Reward).
  \end{block}
  
\end{frame}

\begin{frame}[fragile]
  \frametitle{The Learning Process}
  
  Reinforcement Learning can be described as a cycle:
  
  \begin{itemize}
    \item **Explore**: The agent tries different actions to see what happens.
    \item **Exploit**: The agent utilizes knowledge gained from prior experiences to maximize rewards.
  \end{itemize}

  This ongoing process is often managed using strategies like epsilon-greedy, where the agent balances exploration and exploitation.

\end{frame}

\begin{frame}[fragile]
  \frametitle{Significance in Machine Learning}
  
  \begin{itemize}
    \item **Real-World Applications**: 
      \begin{itemize}
        \item **Game Playing**: AlphaGo defeating human champions in the game Go.
        \item **Robotics**: Training robots to perform tasks in unstructured environments.
        \item **Healthcare**: Personalized treatment planning based on patient responses.
      \end{itemize}
    \item **Dynamic Systems Handling**: 
      RL excels in scenarios where optimal solutions are not pre-defined and environments are dynamic and uncertain.
  \end{itemize}

\end{frame}

\begin{frame}[fragile]
  \frametitle{Key Points and Summary}

  \begin{block}{Key Points to Emphasize}
    \begin{itemize}
      \item Differentiation from Supervised and Unsupervised Learning: RL learns from consequences of actions without direct supervision.
      \item Continuous Learning: RL models integrate experiences over time, allowing adaptation and improvement based on past interactions.
    \end{itemize}
  \end{block}

  In conclusion, Reinforcement Learning represents a powerful method for enabling machines to learn autonomously through a trial-and-error approach. Its diverse applications indicate its critical importance and potential for future technological advancements.
  
\end{frame}

\begin{frame}[fragile]
    \frametitle{What is Reinforcement Learning?}
    \begin{block}{Definition of Reinforcement Learning (RL)}
        Reinforcement Learning is a type of machine learning where an \textbf{agent} learns to make decisions by taking specific actions in an environment to maximize cumulative \textbf{rewards}. 
        The agent learns from past experiences, continuously refining its strategies through trial and error without explicit supervision.
    \end{block}
\end{frame}

\begin{frame}[fragile]
    \frametitle{How RL Differs from Other Learning Paradigms}
    \begin{enumerate}
        \item \textbf{Supervised Learning}:
            \begin{itemize}
                \item \textbf{Definition}: Algorithms learn from labeled datasets with known outputs.
                \item \textbf{Example}: Predicting house prices based on known features.
                \item \textbf{Key Point}: Direct feedback allows for error minimization.
            \end{itemize}
        \item \textbf{Unsupervised Learning}:
            \begin{itemize}
                \item \textbf{Definition}: Explores unlabeled data to find patterns without feedback.
                \item \textbf{Example}: Clustering customers based on purchasing behavior.
                \item \textbf{Key Point}: Focuses on understanding data's structure.
            \end{itemize}
        \item \textbf{Reinforcement Learning}:
            \begin{itemize}
                \item \textbf{Essence}: Learning through interaction with an environment rather than from fixed datasets.
                \item \textbf{Feedback Mechanism}: Agents receive rewards or penalties to learn effective actions.
                \item \textbf{Example}: A robot learning to navigate a maze by receiving rewards for reaching exits.
            \end{itemize}
    \end{enumerate}
\end{frame}

\begin{frame}[fragile]
    \frametitle{Core Elements of Reinforcement Learning}
    \begin{itemize}
        \item \textbf{Agent}: The learner or decision-maker.
        \item \textbf{Environment}: The space or context within which the agent operates.
        \item \textbf{Actions}: The choices available to the agent at any given time.
        \item \textbf{Reward}: Feedback signal evaluating the outcome of the agent's actions.
    \end{itemize}

    \begin{block}{Summary Points}
        \begin{itemize}
            \item RL focuses on learning through interaction and rewards, differing significantly from supervised and unsupervised learning.
            \item Emphasizes exploration vs. exploitation of known actions.
            \item Essential for applications like robotics, game playing, and self-driving cars.
        \end{itemize}
    \end{block}
\end{frame}

\begin{frame}[fragile]
  \frametitle{Core Components of Reinforcement Learning - Overview}
  \begin{block}{Key Elements of Reinforcement Learning}
    Reinforcement Learning (RL) is an area of machine learning where an agent learns to make decisions by interacting with its environment. The key components that define this process are:
  \end{block}
\end{frame}

\begin{frame}[fragile]
  \frametitle{Core Components - Agent and Environment}
  \begin{enumerate}
    \item \textbf{Agent}
      \begin{itemize}
        \item \textbf{Definition}: The agent is the learner or decision-maker that interacts with the environment.
        \item \textbf{Example}: In a game of chess, the player (human or neural network) is the agent making decisions on moves.
      \end{itemize}
    
    \item \textbf{Environment}
      \begin{itemize}
        \item \textbf{Definition}: The environment encompasses everything the agent interacts with, providing the necessary context.
        \item \textbf{Example}: In chess, the board and pieces represent the environment; the state includes the position of pieces.
      \end{itemize}
  \end{enumerate}
\end{frame}

\begin{frame}[fragile]
  \frametitle{Core Components - Actions and Rewards}
  \begin{enumerate}
    \setcounter{enumi}{2} % Continue numbering from the previous frame
    \item \textbf{Actions}
      \begin{itemize}
        \item \textbf{Definition}: Actions are the choices made by the agent in response to observed states.
        \item \textbf{Example}: In chess, actions include specific moves like moving a knight or capturing a piece.
      \end{itemize}

    \item \textbf{Rewards}
      \begin{itemize}
        \item \textbf{Definition}: Rewards are feedback signals received by the agent after actions, guiding behavior.
        \item \textbf{Example}: In sports, scoring points is a reward; a penalty for fouling is a negative reward.
      \end{itemize}
  \end{enumerate}
\end{frame}

\begin{frame}[fragile]
  \frametitle{Illustrative Diagram and Key Points}
  \begin{block}{Interaction Diagram}
    \centering
    \includegraphics[width=0.6\linewidth]{interaction_diagram} % Placeholder for diagram
    \\
    \textit{[Agent]} --(action)--> \textit{[Environment]} --(state + reward)--> \textit{[Agent]}
  \end{block}
  
  \begin{itemize}
    \item Reinforcement Learning operates on trial and error; the agent explores actions and learns from feedback.
    \item The goal is to maximize cumulative rewards over time through strategic decision-making.
    \item Understanding these components is crucial for developing effective reinforcement learning algorithms.
  \end{itemize}
\end{frame}

\begin{frame}[fragile]
  \frametitle{Example Scenario}
  \begin{block}{Training a Robot to Navigate a Maze}
    \begin{itemize}
      \item \textbf{Agent}: The robot.
      \item \textbf{Environment}: The maze, including walls, paths, and the goal location.
      \item \textbf{Actions}: Move forward, turn left, turn right, and stop.
      \item \textbf{Rewards}: Positive for reaching the goal, negative for hitting walls or taking too long.
    \end{itemize}
  \end{block}
  \begin{block}{Conclusion}
    Grasping these core components will provide a foundational understanding of how reinforcement learning systems operate.
  \end{block}
\end{frame}

\begin{frame}[fragile]
    \frametitle{The Reinforcement Learning Process - Overview}
    In reinforcement learning (RL), an \textbf{agent} learns to make decisions by interacting with an \textbf{environment}. 
    The objective is to choose actions that maximize cumulative \textbf{rewards} over time. 

    This process encompasses several key steps:
    \begin{itemize}
        \item Agent, Environment, Actions, State, Reward
        \item Interaction loop consisting of initialization, action selection, state transition, and updating knowledge.
    \end{itemize}
\end{frame}

\begin{frame}[fragile]
    \frametitle{The Reinforcement Learning Process - Components}
    \begin{enumerate}
        \item \textbf{Agent}: The learner or decision-maker.
        \item \textbf{Environment}: The context or scenario where the agent operates.
        \item \textbf{Actions (A)}: Choices made by the agent to influence the environment.
        \item \textbf{State (S)}: A representation of the current situation of the environment.
        \item \textbf{Reward (R)}: Feedback from the environment in response to the agent's action, typically a scalar value.
    \end{enumerate}
\end{frame}

\begin{frame}[fragile]
    \frametitle{The Reinforcement Learning Process - Interaction}
    \textbf{How the Interaction Works:}
    \begin{enumerate}
        \item \textbf{Initialization}: The agent starts in initial state \( S_0 \).
        \item \textbf{Action Selection}: The agent chooses an action \( A_t \) based on its policy \( \pi \).
        \item \textbf{State Transition}: After action \( A_t \), the environment responds with a new state \( S_{t+1} \).
        \item \textbf{Receiving Rewards}: The agent receives a reward \( R_t \) based on the action and resulting state.
        \item \textbf{Update Knowledge}: The agent updates its policy \( \pi \) based on received rewards.
        \item \textbf{Iterative Learning}: The process is repeated for multiple episodes to optimize actions.
    \end{enumerate}
    
    \begin{block}{Key Points}
        - Exploration vs. Exploitation: Balance new actions and known rewards.
        - Cumulative Reward: Aim to maximize total reward over time.
        - Trial and Error: Fundamental learning through repeated interactions.
    \end{block}
\end{frame}

\begin{frame}[fragile]
    \frametitle{Markov Decision Processes (MDPs) - Introduction}
    \begin{block}{What are MDPs?}
        Markov Decision Processes (MDPs) provide a mathematical framework for modeling decision-making scenarios where outcomes are partly random and partly under the control of a decision-maker (the agent). 
    \end{block}
    \begin{itemize}
        \item MDPs are foundational in Reinforcement Learning (RL).
        \item They enable agents to make a sequence of decisions in a dynamic environment.
    \end{itemize}
\end{frame}

\begin{frame}[fragile]
    \frametitle{Markov Decision Processes (MDPs) - Key Components}
    An MDP is defined by the following key elements:
    \begin{enumerate}
        \item \textbf{States (S)}: A finite set of states representing all possible situations.
        \item \textbf{Actions (A)}: A finite set of actions available to the agent.
        \item \textbf{Transition Model (P)}: Defines the probability of moving between states given an action.
        \item \textbf{Rewards (R)}: Assigns a numerical value received after transitioning between states.
        \item \textbf{Discount Factor ($\gamma$)}: Determines the importance of future rewards.
    \end{enumerate}
\end{frame}

\begin{frame}[fragile]
    \frametitle{Understanding MDPs through an Example}
    \begin{block}{Example Scenario}
        Imagine an agent navigating a grid towards a target while avoiding obstacles.
    \end{block}
    \begin{itemize}
        \item \textbf{States (S)}: Cells in the grid (e.g., S = \{A1, A2, A3,...\}).
        \item \textbf{Actions (A)}: {up, down, left, right}.
        \item \textbf{Transition Model (P)}: E.g., from state A1 to A2 with probability 0.8.
        \item \textbf{Rewards (R)}: +10 for reaching the target, -1 for hitting obstacles.
        \item \textbf{Discount Factor ($\gamma$)}: If $\gamma = 0.9$, emphasizes future rewards.
    \end{itemize}
\end{frame}

\begin{frame}[fragile]
    \frametitle{Markov Decision Processes (MDPs) - Key Takeaways}
    \begin{itemize}
        \item MDPs define the learning problem for many RL algorithms.
        \item The transition model and rewards guide optimal strategy learning.
        \item Understanding MDPs is crucial for advanced concepts like policy evaluation.
    \end{itemize}
    \begin{block}{Key Formula}
        \begin{equation}
            V(s) = R(s) + \gamma \sum_{s'} P(s' | s, a) V(s')
        \end{equation}
        This estimates expected return from state $s$.
    \end{block}
\end{frame}

\begin{frame}[fragile]
    \frametitle{Exploration vs Exploitation}
    \begin{block}{Understanding the Exploration-Exploitation Dilemma}
        In Reinforcement Learning (RL), the agent faces a fundamental challenge known as the exploration-exploitation dilemma. This dilemma is crucial for efficient learning and decision-making in uncertain environments.
    \end{block}
\end{frame}

\begin{frame}[fragile]
    \frametitle{Definitions}
    \begin{itemize}
        \item \textbf{Exploration}: The agent tries out new strategies or actions to discover their potential rewards. It focuses on gaining information that can improve future decision-making.
        
        \item \textbf{Exploitation}: The agent uses its current knowledge to choose the best-known actions that yield the highest immediate reward.
    \end{itemize}
\end{frame}

\begin{frame}[fragile]
    \frametitle{Key Concepts}
    \begin{itemize}
        \item \textbf{Balance}: Successful reinforcement learning requires finding a balance between exploration (trying new things) and exploitation (using known strategies).
        
        \item \textbf{Trade-off}: 
        \begin{itemize}
            \item If an agent never explores, it may miss out on better long-term rewards.
            \item Conversely, if it only exploits, it may get stuck in suboptimal solutions without learning new strategies.
        \end{itemize}
    \end{itemize}
\end{frame}

\begin{frame}[fragile]
    \frametitle{Visual Representation}
    \centering
    \includegraphics[width=0.7\textwidth]{exploration_exploitation_graph.png} % Placeholder for actual graph image
    \footnotesize{In this illustration, the exploration and exploitation strategies lead to different reward outcomes on the graph.}
\end{frame}

\begin{frame}[fragile]
    \frametitle{Practical Example}
    Imagine a simple video game where an agent must choose between two paths:
    \begin{enumerate}
        \item \textbf{Path A}: Historically gives 60\% success (high exploitation).
        \item \textbf{Path B}: Unknown success rate (high exploration).
    \end{enumerate}
    \begin{block}{Insight}
        If the agent always takes Path A, it may miss exploring Path B, which could lead to a 90\% success rate if discovered!
    \end{block}    
\end{frame}

\begin{frame}[fragile]
    \frametitle{Strategies to Address the Dilemma}
    \begin{enumerate}
        \item \textbf{Epsilon-Greedy Algorithm}:
        \begin{itemize}
            \item Choose the best-known action (exploitation) with probability \( 1 - \epsilon \).
            \item Explore randomly with probability \( \epsilon \).
            \item Example: If \( \epsilon = 0.1 \), there's a 10\% chance the agent will explore an alternative action.
        \end{itemize}
        
        \item \textbf{UCB (Upper Confidence Bound) Approach}:
        \begin{itemize}
            \item Select actions based on both the average reward and uncertainty of that action to effectively balance exploration and exploitation.
        \end{itemize}
        
        \item \textbf{Boltzmann Exploration}:
        \begin{itemize}
            \item Use a temperature parameter to control exploration vs. exploitation. Lower temperatures favor exploitation, while higher temperatures encourage exploration.
        \end{itemize}
    \end{enumerate}
\end{frame}

\begin{frame}[fragile]
    \frametitle{Key Takeaways}
    \begin{itemize}
        \item Finding the right balance between exploration and exploitation is essential for effective learning in RL.
        \item The selection strategy can significantly impact the agent's performance and its ability to discover optimal actions.
        \item Various algorithms can help navigate this trade-off, each with its advantages and disadvantages.
    \end{itemize}
\end{frame}

\begin{frame}[fragile]
    \frametitle{Common Algorithms in Reinforcement Learning - Overview}
    \begin{block}{Q-Learning}
        \begin{itemize}
            \item \textbf{Definition}: A model-free algorithm that learns the value of actions in states to maximize cumulative rewards.
            \item \textbf{Key Concepts}:
                \begin{itemize}
                    \item Q-Value: Represents expected future rewards for an action in a given state.
                    \item Bellman Equation: Updates Q-values based on rewards and estimated future rewards.
                \end{itemize}
        \end{itemize}
    \end{block}
\end{frame}

\begin{frame}[fragile]
    \frametitle{Q-Learning - Update Formula}
    \begin{block}{Q-Learning Update Formula}
        \begin{equation}
            Q(s, a) \leftarrow Q(s, a) + \alpha \left( r + \gamma \max_{a'} Q(s', a') - Q(s, a) \right)
        \end{equation}
        Where:
        \begin{itemize}
            \item \( s \) = current state
            \item \( a \) = action taken
            \item \( r \) = reward received
            \item \( s' \) = next state
            \item \( \alpha \) = learning rate (0 < \(\alpha\) ≤ 1)
            \item \( \gamma \) = discount factor (0 ≤ \(\gamma\) < 1)
        \end{itemize}
    \end{block}
    
    \begin{block}{Example}
        Imagine training a robot to navigate a maze. The robot updates its Q-values at each junction based on rewards received for actions taken.
    \end{block}
\end{frame}

\begin{frame}[fragile]
    \frametitle{Deep Q-Networks (DQN)}
    \begin{block}{DQN Overview}
        \begin{itemize}
            \item \textbf{Definition}: Combines deep learning with Q-learning to learn from high-dimensional sensory input.
            \item \textbf{Key Concepts}:
                \begin{itemize}
                    \item Neural Network: Approximates Q-values for all actions in a state.
                    \item Experience Replay: Samples from past experiences to improve learning stability.
                    \item Target Network: Separate network for calculating target Q-values stable during training.
                \end{itemize}
        \end{itemize}
    \end{block}
    
    \begin{block}{DQN Algorithm Steps}
        \begin{enumerate}
            \item Store the current state and action in replay memory.
            \item Sample a random batch of experiences from memory.
            \item Use the neural network to predict current and target Q-values.
            \item Update the weights of the neural network based on loss.
        \end{enumerate}
    \end{block}
\end{frame}

\begin{frame}[fragile]
  \frametitle{Temporal Difference Learning}
  \begin{block}{Understanding Temporal Difference (TD) Learning}
    Temporal Difference (TD) Learning is a reinforcement learning technique that combines ideas from Monte Carlo methods and dynamic programming. It allows agents to learn directly from raw experience without needing a model of the environment.
  \end{block}
\end{frame}

\begin{frame}[fragile]
  \frametitle{Key Concepts}
  \begin{itemize}
    \item \textbf{State Value Function (V)}:
      \begin{itemize}
        \item Represents the expected return (future rewards) for being in a given state.
        \item The goal is to learn an accurate value of V for all states in an environment.
      \end{itemize}
    
    \item \textbf{Q-Value Function (Q)}:
      \begin{itemize}
        \item Represents the expected return of taking a specific action in a given state.
        \item Relationship:
        \[
        Q(s, a) = R + \gamma V(s')
        \]
      \end{itemize}
    
    \item \textbf{Bootstrapping}:
      \begin{itemize}
        \item TD methods update their estimates based on other estimates, improving current state value.
      \end{itemize}
  \end{itemize}
\end{frame}

\begin{frame}[fragile]
  \frametitle{TD Learning Algorithm}
  \begin{block}{TD(0) Update Rule}
    \[
    V(s) \leftarrow V(s) + \alpha \left[ R + \gamma V(s') - V(s) \right]
    \]
    where:
    \begin{itemize}
      \item \( \alpha \) = learning rate (controls adjustment of value estimates)
      \item \( R \) = reward received after taking action from state \( s \)
      \item \( V(s) \) = current estimate of value for state \( s \)
      \item \( V(s') \) = estimate of value for the next state
    \end{itemize}
  \end{block}
\end{frame}

\begin{frame}[fragile]
  \frametitle{Example of TD Learning}
  \begin{block}{Scenario}
    Imagine a simple grid-world where an agent moves in a 5x5 grid and receives rewards based on its location:
    \begin{itemize}
      \item Reward for reaching a goal state: +10
      \item Penalty for stepping into a pit: -10
    \end{itemize}
  \end{block}
  
  \begin{block}{Application of TD Learning}
    \begin{enumerate}
      \item Randomly initialize the value function \( V(s) \) for each state.
      \item Start the agent at a state and choose an action.
      \item After executing the action, the agent receives a reward and visits a new state.
      \item Update the value function using the TD(0) rule based on received reward and estimated value of the next state.
    \end{enumerate}
  \end{block}
\end{frame}

\begin{frame}[fragile]
  \frametitle{Key Points and Conclusion}
  \begin{itemize}
    \item TD Learning balances exploration and exploitation, allowing for dynamic learning.
    \item Suitable for stochastic environments or when dynamics are unknown.
    \item Foundational for many reinforcement learning algorithms, like Q-Learning.
  \end{itemize}
  
  \begin{block}{Conclusion}
    Temporal Difference Learning is essential for enabling agents to learn from experiences and predict future rewards, bridging basic learning algorithms and complex architectures.
  \end{block}
    
  \begin{block}{Next Steps}
    Transition to the next slide, "Applications of Reinforcement Learning," examining real-world use cases leveraging these concepts.
  \end{block}
\end{frame}

\begin{frame}[fragile]
    \frametitle{Applications of Reinforcement Learning}
    
    \begin{block}{Overview}
        Reinforcement Learning (RL) is transforming various fields by enabling systems to learn strategies through interactions with their environments. Below, we explore four prominent domains where RL is applied effectively.
    \end{block}
\end{frame}

\begin{frame}[fragile]
    \frametitle{Applications of Reinforcement Learning - Gaming}

    \begin{itemize}
        \item \textbf{Concept}: 
        RL has been a crucial element in training AI to play games, where agents learn through trial and error.
        
        \item \textbf{Example}: 
        \textit{AlphaGo}, developed by DeepMind, utilized RL techniques to master the game of Go, beating a world champion by predicting optimal moves through millions of simulated games.
        
        \item \textbf{Key Point}: 
        RL allows systems to discover and optimize strategies in complex environments where traditional programming would fail.
    \end{itemize}
\end{frame}

\begin{frame}[fragile]
    \frametitle{Applications of Reinforcement Learning - Robotics}

    \begin{itemize}
        \item \textbf{Concept}: 
        In robotics, RL is used to train robots to perform tasks through interactions with their physical environments.
        
        \item \textbf{Example}: 
        Robot arms harness RL to learn how to manipulate objects effectively, as seen in robotic assembly lines where the robot learns to pick and place items accurately.
        
        \item \textbf{Key Point}: 
        RL facilitates adaptive learning, enabling robots to improve their performance in real-time.
    \end{itemize}
\end{frame}

\begin{frame}[fragile]
    \frametitle{Applications of Reinforcement Learning - Finance}

    \begin{itemize}
        \item \textbf{Concept}: 
        In finance, RL is applied for algorithmic trading, optimizing portfolios, and risk management.
        
        \item \textbf{Example}: 
        An RL algorithm could learn the best trading strategies by evaluating past market conditions and making decisions that maximize profit while minimizing risk.
        
        \item \textbf{Key Point}: 
        RL can mimic the trial-and-error learning process of investors, adapting strategies based on real-time data.
    \end{itemize}
\end{frame}

\begin{frame}[fragile]
    \frametitle{Applications of Reinforcement Learning - Healthcare}

    \begin{itemize}
        \item \textbf{Concept}: 
        RL is utilized for personalized treatment plans and optimizing healthcare delivery.
        
        \item \textbf{Example}: 
        These algorithms might help in recommendation systems that suggest the best treatment options for patients based on their unique medical histories and demographics.
        
        \item \textbf{Key Point}: 
        Increased efficiency in healthcare delivery through RL can enhance patient outcomes and optimize resource use.
    \end{itemize}
\end{frame}

\begin{frame}[fragile]
    \frametitle{Applications of Reinforcement Learning - Summary Points}

    \begin{itemize}
        \item RL is a versatile technology applicable across many domains.
        \item It relies on the foundational principle of learning through feedback from interactions with the environment.
        \item The scope of RL continues to grow, promising advancements in automation, decision-making, and efficiency.
    \end{itemize}
\end{frame}

\begin{frame}[fragile]
    \frametitle{Formulas and Code Snippets in Reinforcement Learning}

    \begin{block}{Key Concepts}
        \begin{itemize}
            \item \textbf{Reward Function} \( R(s, a) \): Represents the immediate reward received after taking action \( a \) in state \( s \).
            \item \textbf{Value Function} \( V(s) \): Estimates the expected return from state \( s \).
        \end{itemize}
    \end{block}

    \begin{block}{Code Snippet}
        \begin{lstlisting}[language=Python]
import numpy as np

def update_q_value(q, state, action, reward, next_state, alpha, gamma):
    best_next_action = np.argmax(q[next_state])
    td_target = reward + gamma * q[next_state][best_next_action]
    td_delta = td_target - q[state][action]
    q[state][action] += alpha * td_delta
    return q
        \end{lstlisting}
    \end{block}
\end{frame}

\begin{frame}[fragile]
  \frametitle{Challenges in Reinforcement Learning - Overview}
  Reinforcement Learning (RL) has become a powerful tool in various domains, yet its application presents several challenges. Understanding these limitations is essential for effectively leveraging RL in complex environments.
\end{frame}

\begin{frame}[fragile]
  \frametitle{Challenges in Reinforcement Learning - Key Challenges}
  \begin{enumerate}
    \item \textbf{Exploration vs. Exploitation Trade-off}
    \begin{itemize}
      \item \textbf{Concept}: RL agents must balance exploring new actions to discover their effects versus exploiting known actions that yield high rewards.
      \item \textbf{Example}: In a game, an agent might always choose the best move it knows (exploitation) but may miss out on discovering a better move if it doesn't explore.
    \end{itemize}
    
    \item \textbf{Sample Efficiency}
    \begin{itemize}
      \item \textbf{Concept}: RL often requires a large number of interactions with the environment to learn effective policies, which can be time-consuming.
      \item \textbf{Example}: Training a robot to navigate a maze might require thousands of trials, which isn't feasible in the real world.
    \end{itemize}
  \end{enumerate}
\end{frame}

\begin{frame}[fragile]
  \frametitle{Challenges in Reinforcement Learning - Remaining Challenges}
  \begin{enumerate}
    \setcounter{enumi}{2}
    \item \textbf{Sparse and Delayed Rewards}
    \begin{itemize}
      \item \textbf{Concept}: In many environments, rewards may come infrequently or after delays, complicating the learning process.
      \item \textbf{Example}: A player might perform several actions before scoring points in a game.
    \end{itemize}

    \item \textbf{Non-stationarity of the Environment}
    \begin{itemize}
      \item \textbf{Concept}: Environments can change over time, affecting learned dynamics and leading to performance degradation.
      \item \textbf{Example}: Market trends in stock trading can shift suddenly, rendering strategies ineffective.
    \end{itemize}

    \item \textbf{High Dimensionality}
    \begin{itemize}
      \item \textbf{Concept}: The state or action space can be vast, complicating the learning problem.
      \item \textbf{Example}: A driving agent must consider thousands of possible states based on various factors.
    \end{itemize}

    \item \textbf{Safety and Constraints}
    \begin{itemize}
      \item \textbf{Concept}: It's crucial to ensure agents operate safely and adhere to constraints, especially in sensitive applications.
      \item \textbf{Example}: A healthcare bot must avoid dangerous medication doses while learning effective strategies.
    \end{itemize}
  \end{enumerate}
\end{frame}

\begin{frame}[fragile]
  \frametitle{Conclusion and Further Reading}
  \begin{block}{Key Points to Emphasize}
    \begin{itemize}
      \item Balancing exploration and exploitation is crucial for effective learning.
      \item Sample efficiency should be prioritized to reduce training time and resource usage.
      \item The reward structure needs careful design to facilitate learning, especially in sparse environments.
      \item Be aware of environmental changes and adapt strategies accordingly.
      \item Address dimensionality issues using techniques like function approximation or deep learning.
      \item Design systems that adhere to safety constraints, particularly in critical applications.
    \end{itemize}
  \end{block}
  
  \textbf{Suggested Further Reading:}
  \begin{itemize}
    \item "Reinforcement Learning: An Introduction" by Sutton and Barto
    \item "Playing Atari with Deep Reinforcement Learning" by Mnih et al.
  \end{itemize}
  
  Understanding these challenges will help students prepare for real-world applications and the complexities in Reinforcement Learning.
\end{frame}

\begin{frame}[fragile]
    \frametitle{Ethics in Reinforcement Learning - Introduction}
    Reinforcement Learning (RL) involves training algorithms to make decisions based on rewards and penalties. 
    While this technology has immense potential, its application—particularly in sensitive areas—raises significant ethical concerns.

    \begin{block}{Key Areas of Ethical Consideration}
        \begin{itemize}
            \item \textbf{Bias and Fairness:} RL systems can perpetuate or amplify existing biases present in training data or reward structures.
            \item \textbf{Transparency:} The decision-making process of RL algorithms is often opaque, making it difficult to understand or contest their actions.
            \item \textbf{Autonomy and Control:} As RL systems are applied to critical areas like healthcare or criminal justice, the question of human oversight arises.
            \item \textbf{Consequentialism:} Actions taken by an RL agent may lead to unforeseen negative consequences, raising the question of accountability.
        \end{itemize}
    \end{block}
\end{frame}

\begin{frame}[fragile]
    \frametitle{Ethics in Reinforcement Learning - Implications}
    \begin{enumerate}
        \item \textbf{Healthcare:}
            \begin{itemize}
                \item RL is used in treatment recommendations. If a system learns from biased data, it might suggest less effective treatments for certain demographics.
                \item \textit{Key Point:} Ensuring equitable outcomes is crucial; monitoring and adjusting for biases is necessary.
            \end{itemize}

        \item \textbf{Autonomous Vehicles:}
            \begin{itemize}
                \item An RL system might choose to minimize collisions at the expense of passenger safety in certain scenarios.
                \item \textit{Key Point:} Ethical dilemmas arise, highlighting the need for ethical frameworks governing decision-making processes.
            \end{itemize}

        \item \textbf{Criminal Justice:}
            \begin{itemize}
                \item Predictive policing tools utilize RL. If these systems are biased based on historical data, they may disproportionately target certain communities.
                \item \textit{Key Point:} Safeguards must be established to prevent discrimination and ensure fair policing practices.
            \end{itemize}
    \end{enumerate}
\end{frame}

\begin{frame}[fragile]
    \frametitle{Ethics in Reinforcement Learning - Guidelines and Conclusion}
    \begin{block}{Ethical Guidelines and Strategies}
        \begin{itemize}
            \item \textbf{Establish Clear Ethical Standards:} Formulate guidelines that prioritize fairness, accountability, and transparency.
            \item \textbf{Diverse Training Data:} Utilize a dataset that encompasses a wide range of scenarios to minimize bias.
            \item \textbf{Human Oversight:} Always involve human operators in critical decisions made by RL systems, especially in sensitive areas.
            \item \textbf{Regular Audits:} Conduct ongoing assessments of RL systems to identify and rectify potential ethical concerns.
        \end{itemize}
    \end{block}

    \textbf{Conclusion:} Addressing the ethical considerations in Reinforcement Learning is paramount to building trust and ensuring the responsible use of technology. By prioritizing fairness, accountability, and transparency, we can harness the power of RL while mitigating its risks.

    \textbf{Summary:}
    \begin{itemize}
        \item RL can introduce ethical risks, particularly in sensitive fields.
        \item Key areas to focus on include bias, transparency, autonomy, and unintended consequences.
        \item Effective guidelines are essential for navigating these ethical challenges responsibly.
    \end{itemize}
\end{frame}

\begin{frame}[fragile]
  \frametitle{Case Study: Reinforcement Learning in Gaming}
  \begin{block}{Introduction to Reinforcement Learning (RL) in Gaming}
    \begin{itemize}
      \item \textbf{Reinforcement Learning} is a type of machine learning where an agent learns to make decisions by taking actions in an environment to maximize cumulative rewards.
      \item In gaming, RL enables agents to learn optimal strategies through interaction with dynamic environments.
    \end{itemize}
  \end{block}
\end{frame}

\begin{frame}[fragile]
  \frametitle{Notable Examples of RL in Video Games}
  \begin{enumerate}
    \item \textbf{AlphaGo (DeepMind)}
      \begin{itemize}
        \item \textbf{Game}: Go
        \item \textbf{Achievement}: In 2016, AlphaGo became the first AI to defeat a world champion Go player.
        \item \textbf{Breakthrough}: Combined RL with deep learning, utilizing self-play to improve strategies.
      \end{itemize}
      % Include figure or illustration here
      % \includegraphics[width=0.5\textwidth]{alphago_diagram.png}
      
    \item \textbf{OpenAI Five}
      \begin{itemize}
        \item \textbf{Game}: Dota 2
        \item \textbf{Achievement}: In 2019, OpenAI Five competed against professional Dota 2 teams, showcasing strategic depth.
        \item \textbf{Breakthrough}: Used Proximal Policy Optimization (PPO) for real-time adaptations.
      \end{itemize}
      % Include figure or illustration here
      % \includegraphics[width=0.5\textwidth]{openai_five_diagram.png}
      
    \item \textbf{Atari Games (Deep Q-Networks)}
      \begin{itemize}
        \item \textbf{Game}: Multiple Atari titles (e.g., Breakout, Pong)
        \item \textbf{Achievement}: In 2015, DQNs demonstrated human-level performance in various retro games.
        \item \textbf{Breakthrough}: Combined Q-learning with neural networks, learning from raw pixel inputs.
      \end{itemize}
      % Include figure or chart here
      % \includegraphics[width=0.5\textwidth]{dqn_performance_chart.png}
  \end{enumerate}
\end{frame}

\begin{frame}[fragile]
  \frametitle{Key Concepts Highlighted by Reinforcement Learning Examples}
  \begin{itemize}
    \item \textbf{Self-Play}: Agents learn by playing against themselves, leading to rapid skill improvement.
    \item \textbf{Generalization}: RL methods like DQNs enable agents to adapt their learning to different game contexts.
    \item \textbf{Dynamic Strategies}: RL agents can develop complex strategies that outmaneuver human opponents, showcasing adaptability.
  \end{itemize}
  
  \begin{block}{Summary of Achievements}
    \begin{itemize}
      \item RL has transformed agent-environment interactions in gaming.
      \item Advancements in RL algorithms have led to breakthroughs in performance and strategy development.
      \item \textbf{Impact on Gaming}: Enhances AI capabilities, providing insights for broader applications in robotics and automation.
    \end{itemize}
  \end{block}
\end{frame}

\begin{frame}[fragile]
  \frametitle{Conclusion and Further Exploration}
  \begin{block}{Conclusion}
    \begin{itemize}
      \item The intersection of RL and gaming demonstrates AI's potential to solve complex problems.
      \item Continued advancements pave the way for innovative applications beyond gaming.
    \end{itemize}
  \end{block}
  
  \begin{block}{Further Reading and Exploration}
    \begin{itemize}
      \item Explore literature on Deep Reinforcement Learning algorithms.
      \item Investigate ethical ramifications of AI in competitive contexts.
    \end{itemize}
  \end{block}
\end{frame}

\begin{frame}[fragile]
    \frametitle{Future Directions in Reinforcement Learning}
    \begin{block}{Introduction}
        Reinforcement Learning (RL) has made remarkable strides in recent years, particularly in gaming, robotics, and healthcare. This slide explores emerging trends and future research directions in RL that will likely reshape technology and industries.
    \end{block}
\end{frame}

\begin{frame}[fragile]
    \frametitle{Key Areas for Future Research in RL - Part 1}
    \begin{enumerate}
        \item \textbf{Generalization and Transfer Learning}
        \begin{itemize}
            \item \textbf{Explanation}: Future RL models should efficiently generalize learnings across different tasks, handling varying training and real-world environments.
            \item \textbf{Example}: An RL agent trained to play a specific video game should quickly adapt to similar games.
            \item \textbf{Key Point}: Enhanced transfer learning can significantly reduce training times.
        \end{itemize}

        \item \textbf{Multi-Agent Systems}
        \begin{itemize}
            \item \textbf{Explanation}: Research in multi-agent reinforcement learning (MARL) focuses on cooperation, competition, and strategy development among agents.
            \item \textbf{Example}: Swarm robotics coordinating multiple agents for tasks like search and rescue.
            \item \textbf{Key Point}: Innovations in MARL can lead to complex solutions requiring negotiation and collaboration.
        \end{itemize}
    \end{enumerate}
\end{frame}

\begin{frame}[fragile]
    \frametitle{Key Areas for Future Research in RL - Part 2}
    \begin{enumerate}
        \setcounter{enumi}{2} % Continue enumeration
        \item \textbf{Safety and Ethical Considerations}
        \begin{itemize}
            \item \textbf{Explanation}: Creating safe and ethical RL systems is crucial for applications in healthcare, transportation, and finance.
            \item \textbf{Example}: Autonomous vehicles programmed to make safe decisions in unforeseen circumstances.
            \item \textbf{Key Point}: Incorporating ethical frameworks is essential for responsible deployment.
        \end{itemize}

        \item \textbf{Incorporating Human Feedback}
        \begin{itemize}
            \item \textbf{Explanation}: Human feedback can guide RL agents effectively beyond mere reward signals.
            \item \textbf{Example}: In training language models, human feedback refines responses for better contextual appropriateness.
            \item \textbf{Key Point}: Aligning RL systems with human values enhances performance and trust.
        \end{itemize}
    \end{enumerate}
\end{frame}

\begin{frame}[fragile]
    \frametitle{Key Areas for Future Research in RL - Part 3}
    \begin{enumerate}
        \setcounter{enumi}{4} % Continue enumeration
        \item \textbf{Exploration vs. Exploitation in Complex Environments}
        \begin{itemize}
            \item \textbf{Explanation}: Balancing exploration (trying new actions) and exploitation (maximizing known rewards) is crucial in dynamic environments.
            \item \textbf{Example}: A recommendation system that needs to explore while optimizing for user satisfaction.
            \item \textbf{Key Point}: Advanced exploration strategies can improve decision-making under uncertainty.
        \end{itemize}
    \end{enumerate}
\end{frame}

\begin{frame}[fragile]
    \frametitle{Conclusion and Visual Aids}
    \begin{block}{Conclusion}
        The landscape of Reinforcement Learning is evolving with promising research directions. Focusing on these areas allows for developing intelligent, adaptable, and safe systems.
    \end{block}

    \begin{block}{Visual Aids}
        - Diagrams: Flowchart illustrating exploration vs. exploitation.
        - Graphs: Comparing traditional RL performance to multi-agent systems in collaborative tasks.
    \end{block}
\end{frame}

\begin{frame}[fragile]
    \frametitle{Additional Information}
    \begin{block}{Ethics in Reinforcement Learning}
        Incorporating ethical considerations and human feedback into RL is vital for deployment and societal acceptance, emphasizing a responsible approach to AI development.
    \end{block}
\end{frame}

\begin{frame}[fragile]
    \frametitle{Summary and Conclusion - Overview}
    \begin{block}{Overview of Reinforcement Learning}
        Reinforcement Learning (RL) is a dynamic area of machine learning where an agent learns to make decisions through interaction with an environment.
    \end{block}
    \begin{itemize}
        \item \textbf{Agent:} Learns and makes decisions.
        \item \textbf{Environment:} Everything the agent interacts with.
        \item \textbf{State (s):} The current situation of the agent.
        \item \textbf{Action (a):} Any operation the agent can perform.
        \item \textbf{Reward (r):} Feedback from the environment based on actions taken.
    \end{itemize}
\end{frame}

\begin{frame}[fragile]
    \frametitle{Summary and Conclusion - Key Points Recap}
    \begin{enumerate}
        \item \textbf{Learning Mechanism:}
            \begin{itemize}
                \item RL utilizes a trial-and-error approach through \textbf{reward feedback}.
                \item The agent aims to maximize cumulative rewards.
                \item \textit{Example:} In a game, a player (agent) receives points (rewards) for winning rounds.
            \end{itemize}
        
        \item \textbf{Exploration vs. Exploitation:}
            \begin{itemize}
                \item \textbf{Exploration:} Trying new actions to discover their potential rewards.
                \item \textbf{Exploitation:} Utilizing known actions that provide the highest reward.
            \end{itemize}
        
        \item \textbf{Markov Decision Process (MDP):}
            \begin{itemize}
                \item Framework used to describe an RL problem, defined by:
                \begin{itemize}
                    \item A set of states (S)
                    \item A set of actions (A)
                    \item Transition probabilities (P)
                    \item Reward function (R)
                \end{itemize}
                \item \textbf{Formula:}
                \begin{equation}
                    V(s) = \max_a \sum_{s'} P(s'|s, a) \left[ R(s, a, s') + \gamma V(s') \right]
                \end{equation}
            \end{itemize}
    \end{enumerate}
\end{frame}

\begin{frame}[fragile]
    \frametitle{Summary and Conclusion - Importance and Conclusion}
    \begin{block}{Importance of Reinforcement Learning}
        \begin{itemize}
            \item \textbf{Innovation Catalyst:} RL has driven advances in AI by enabling autonomous learning.
            \item \textbf{Adaptive Systems:} Facilitates the creation of systems that adapt to changing environments.
            \item \textbf{Deep Learning Integration:} The combination of RL and deep learning leads to more robust models for complex tasks.
        \end{itemize}
    \end{block}
    
    \begin{block}{Conclusion}
        Reinforcement Learning is at the forefront of AI research, showcasing unprecedented capabilities across various domains. Understanding its principles empowers researchers and practitioners to push the boundaries of what machines can learn and accomplish.
    \end{block}
    
    \textbf{Next Steps:} Prepare questions for the upcoming Q\&A session on Reinforcement Learning principles!
\end{frame}

\begin{frame}[fragile]
    \frametitle{Q\&A Session}
    % Open floor for audience questions and discussions about Reinforcement Learning principles.
    This slide opens the floor to questions and discussions about Reinforcement Learning (RL) principles. 
    It’s an opportunity to clarify concepts, share insights, and engage in deeper conversations about applications and challenges in RL.
\end{frame}

\begin{frame}[fragile]
    \frametitle{Key Concepts in Reinforcement Learning}
    \begin{enumerate}
        \item \textbf{Reinforcement Learning Basics}:
        \begin{itemize}
            \item \textbf{Definition}: A type of machine learning where an agent learns to make decisions by taking actions in an environment to maximize a cumulative reward.
            \item \textbf{Components}: Agent, Environment, Action, State, Reward.
            \item \textbf{Example}: A robot learning to navigate a maze for rewards and penalties.
        \end{itemize}
        
        \item \textbf{Learning Paradigm}:
        \begin{itemize}
            \item \textbf{Agent-Environment Interaction}: Explore vs. exploit.
            \item \textbf{Markov Decision Processes (MDPs)}: Defined by states, actions, transition probabilities, and rewards.
        \end{itemize}
    \end{enumerate}
\end{frame}

\begin{frame}[fragile]
    \frametitle{Key Algorithms and Concepts}
    \begin{enumerate}
        \setcounter{enumi}{2} % Continue numbering from previous frame
        
        \item \textbf{Key Algorithms}:
        \begin{itemize}
            \item \textbf{Q-Learning}:
            \begin{equation}
                Q(s, a) \gets Q(s, a) + \alpha \left[ r + \gamma \max_{a'} Q(s', a') - Q(s, a) \right]
            \end{equation}
            where:
            \begin{itemize}
                \item \( Q(s, a) \): Current value
                \item \( \alpha \): Learning rate
                \item \( r \): Reward received
                \item \( \gamma \): Discount factor
                \item \( s' \): New state after action \( a \)
            \end{itemize}

            \item \textbf{Deep Q-Networks (DQN)}: Combines Q-Learning with deep learning to handle high-dimensional input spaces.
        
            \item \textbf{Exploration vs. Exploitation}: Discuss strategies and the importance of balancing them.
        \end{itemize}
    \end{enumerate}
\end{frame}

\begin{frame}[fragile]
    \frametitle{Further Reading on Reinforcement Learning}
    \begin{block}{Introduction}
        Reinforcement Learning (RL) is a dynamic field in machine learning that simulates how agents learn to make decisions by interacting with their environment. 
        To deepen your understanding of RL concepts, methodologies, and applications, we recommend the following literature and resources.
    \end{block}
\end{frame}

\begin{frame}[fragile]
    \frametitle{Key Texts}
    \begin{enumerate}
        \item \textbf{"Reinforcement Learning: An Introduction" by Richard S. Sutton and Andrew G. Barto}
        \begin{itemize}
            \item \textbf{Overview}: This foundational text covers core principles of RL, including the exploration-exploitation dilemma and various algorithms.
            \item \textbf{Key Concepts}: Temporal-Difference Learning, Markov Decision Processes (MDPs).
            \item \textbf{Example}: Illustrates the bandit problem to show the trade-offs between trying new actions and leveraging known rewards.
        \end{itemize}

        \item \textbf{"Deep Reinforcement Learning Hands-On" by Maxim Lapan}
        \begin{itemize}
            \item \textbf{Overview}: Focuses on implementing RL algorithms using Python and PyTorch, making theory actionable.
            \item \textbf{Key Concepts}: Deep Q-Networks, Policy Gradients, and Advanced Techniques.
            \item \textbf{Example}: Implementation of DQN to play classic Atari games.
        \end{itemize}

        \item \textbf{"Algorithms for Reinforcement Learning" by Dimitri Bertsekas and John N. Tsitsiklis}
        \begin{itemize}
            \item \textbf{Overview}: Offers a comprehensive look at optimization methods used in RL.
            \item \textbf{Key Concepts}: Value Function Approximation, Policy Search Algorithms.
            \item \textbf{Illustration}: Provides mathematical proofs that illuminate the conditions for convergence in RL algorithms.
        \end{itemize}
    \end{enumerate}
\end{frame}

\begin{frame}[fragile]
    \frametitle{Online Resources}
    \begin{enumerate}
        \item \textbf{OpenAI Spinning Up in Deep RL}
        \begin{itemize}
            \item \textbf{Link}: \url{https://spinningup.openai.com/}
            \item \textbf{Description}: An educational resource introducing fundamental concepts and practical applications of deep reinforcement learning with clear tutorials.
        \end{itemize}
        
        \item \textbf{Coursera: Reinforcement Learning Specialization}
        \begin{itemize}
            \item \textbf{Provider}: University of Alberta
            \item \textbf{Key Features}: A series of courses covering classical and deep learning methods in reinforcement learning, including assignments.
        \end{itemize}
    \end{enumerate}
\end{frame}

\begin{frame}[fragile]
    \frametitle{Key Points and Conclusion}
    \begin{block}{Key Points to Emphasize}
        \begin{itemize}
            \item \textbf{Exploration vs. Exploitation}: Understanding how agents balance trying new actions (exploration) with utilizing known rewarding actions (exploitation).
            \item \textbf{Value Functions}: Familiarize with the computation of value functions (\(V(s)\) and \(Q(s, a)\)) to assess potential future rewards.
        \end{itemize}
    \end{block}

    \begin{block}{Example Formula}
        The Q-learning update rule is central to many RL algorithms:
        \begin{equation}
            Q(s, a) \leftarrow Q(s, a) + \alpha [r + \gamma \max_{a'} Q(s', a') - Q(s, a)]
        \end{equation}
        Where:
        \begin{itemize}
            \item \(s\) = current state
            \item \(a\) = current action
            \item \(r\) = reward received
            \item \(\alpha\) = learning rate
            \item \(\gamma\) = discount factor
        \end{itemize}
    \end{block}
    
    \begin{block}{Conclusion}
        Reinforcement Learning is a multifaceted field with ongoing research and applications. Engaging with recommended texts and resources will solidify your understanding and inspire further exploration. Happy learning!
    \end{block}
\end{frame}


\end{document}