\documentclass[aspectratio=169]{beamer}

% Theme and Color Setup
\usetheme{Madrid}
\usecolortheme{whale}
\useinnertheme{rectangles}
\useoutertheme{miniframes}

% Additional Packages
\usepackage[utf8]{inputenc}
\usepackage[T1]{fontenc}
\usepackage{graphicx}
\usepackage{booktabs}
\usepackage{listings}
\usepackage{amsmath}
\usepackage{amssymb}
\usepackage{xcolor}
\usepackage{tikz}
\usepackage{pgfplots}
\pgfplotsset{compat=1.18}
\usetikzlibrary{positioning}
\usepackage{hyperref}

% Custom Colors
\definecolor{myblue}{RGB}{31, 73, 125}
\definecolor{mygray}{RGB}{100, 100, 100}
\definecolor{mygreen}{RGB}{0, 128, 0}
\definecolor{myorange}{RGB}{230, 126, 34}
\definecolor{mycodebackground}{RGB}{245, 245, 245}

% Set Theme Colors
\setbeamercolor{structure}{fg=myblue}
\setbeamercolor{frametitle}{fg=white, bg=myblue}
\setbeamercolor{title}{fg=myblue}
\setbeamercolor{section in toc}{fg=myblue}
\setbeamercolor{item projected}{fg=white, bg=myblue}
\setbeamercolor{block title}{bg=myblue!20, fg=myblue}
\setbeamercolor{block body}{bg=myblue!10}
\setbeamercolor{alerted text}{fg=myorange}

% Set Fonts
\setbeamerfont{title}{size=\Large, series=\bfseries}
\setbeamerfont{frametitle}{size=\large, series=\bfseries}
\setbeamerfont{caption}{size=\small}
\setbeamerfont{footnote}{size=\tiny}

% Document Start
\begin{document}

\frame{\titlepage}

\begin{frame}[fragile]
    \maketitle
\end{frame}

\begin{frame}[fragile]
    \frametitle{Introduction to Data Pipelines}
    \begin{block}{Overview of Data Pipelines}
        A data pipeline is a series of processing steps that move data from one system to another, enabling data aggregation, transformation, and storage in an automated process.
    \end{block}
    
    \begin{itemize}
        \item \textbf{Data Sources:} Locations where data originates (e.g., databases, APIs, IoT devices).
        \item \textbf{Processing Units:} Tools/frameworks (e.g., Apache Spark, Apache Beam) for data manipulation.
        \item \textbf{Data Sinks:} Destinations for processed data (e.g., data warehouses like Amazon Redshift, dashboards like Tableau).
    \end{itemize}
\end{frame}

\begin{frame}[fragile]
    \frametitle{Importance of Workflow Management}
    \begin{itemize}
        \item \textbf{Automation \& Efficiency:} Reduces human errors and increases efficiency.
        \item \textbf{Scalability:} Handles growing data volumes without significant reengineering.
        \item \textbf{Consistency:} Delivers repeatable processes for reliable data results.
        \item \textbf{Monitoring \& Maintenance:} Provides insights into data flow to troubleshoot performance proactively.
    \end{itemize}
\end{frame}

\begin{frame}[fragile]
    \frametitle{Key Points and Example}
    \begin{block}{Key Points to Emphasize}
        \begin{enumerate}
            \item \textbf{Agility:} Enables quick responses to changing data needs and market conditions.
            \item \textbf{Interconnectivity:} Integrates various data sources for comprehensive insights.
            \item \textbf{Data Quality Management:} Ensures only accurate and valuable data is processed.
            \item \textbf{User-Friendly Interfaces:} Intuitive dashboards for better usability in workflow management tools.
        \end{enumerate}
    \end{block}

    \begin{block}{Example}
        Consider an e-commerce platform that collects user behavioral data from its website (Data Source), passes it through a real-time analytics processing unit (Processing Unit), and stores it in a data warehouse (Data Sink) for reporting.
    \end{block}
\end{frame}

\begin{frame}[fragile]
    \frametitle{Conclusion}
    Data pipelines are critical in modern data management, enabling structured, efficient, and reliable data processing workflows. Effective implementation can significantly enhance organizational data capabilities.
\end{frame}

\begin{frame}[fragile]
    \frametitle{What is a Data Pipeline? - Definition}
    \begin{block}{Definition}
        A \textbf{Data Pipeline} is a series of data processing steps that involve the movement of data from one system to another through various stages of transformation, ultimately delivering processed data to a target location for further analysis or storage.
    \end{block}
\end{frame}

\begin{frame}[fragile]
    \frametitle{What is a Data Pipeline? - Key Components}
    \begin{itemize}
        \item \textbf{Data Sources:}
            \begin{itemize}
                \item Sources where raw data is generated or collected.
                \item Examples:
                    \begin{itemize}
                        \item \textbf{Databases} (SQL, NoSQL)
                        \item \textbf{APIs} (RESTful services, web services)
                        \item \textbf{Files} (CSV, JSON, XML)
                        \item \textbf{Real-Time Data Streams} (IoT devices, web servers)
                    \end{itemize}
            \end{itemize}
        
        \item \textbf{Processing Units:}
            \begin{itemize}
                \item Intermediate processes that transform the data.
                \item Key functions:
                    \begin{itemize}
                        \item Data Cleaning
                        \item Data Transformation
                        \item Data Enrichment
                    \end{itemize}
                \item Examples of Processing Technologies:
                    \begin{itemize}
                        \item Apache Spark
                        \item Apache Flink
                        \item AWS Lambda
                    \end{itemize}
            \end{itemize}
        
        \item \textbf{Data Sinks:}
            \begin{itemize}
                \item The final destination where processed data is stored or presented.
                \item Examples:
                    \begin{itemize}
                        \item \textbf{Data Warehouses} (Amazon Redshift, Google BigQuery)
                        \item \textbf{Dashboards} (Tableau, Power BI)
                        \item \textbf{Data Lakes} (Amazon S3, Azure Data Lake)
                        \item \textbf{Machine Learning Models}
                    \end{itemize}
            \end{itemize}
    \end{itemize}
\end{frame}

\begin{frame}[fragile]
    \frametitle{What is a Data Pipeline? - Illustration and Key Points}
    \begin{block}{Basic Data Pipeline Flow}
        \begin{center}
            [Data Source] $\rightarrow$ [Data Processing Unit] $\rightarrow$ [Data Sink]
        \end{center}
    \end{block}

    \textbf{Example:}
    \begin{itemize}
        \item Data Source: Customer transaction logs from an e-commerce website.
        \item Processing Unit: A data transformation step that aggregates sales data by product category.
        \item Data Sink: The aggregated data is stored in a data warehouse for reporting.
    \end{itemize}

    \begin{block}{Key Points to Emphasize}
        \begin{itemize}
            \item \textbf{Modularity:} Each stage of the pipeline is independent.
            \item \textbf{Scalability:} Pipelines can scale to growing data volumes.
            \item \textbf{Automation:} Data pipelines can be automated.
            \item \textbf{Real-time vs. Batch Processing:} Pipelines can be designed based on business requirements.
        \end{itemize}
    \end{block}
\end{frame}

\begin{frame}[fragile]
    \frametitle{Importance of Workflow Management}
    \begin{block}{What is Workflow Management in Data Processing?}
        Workflow management refers to the orchestration of data processing tasks to ensure that data moves seamlessly through a pipeline from source to destination. It encompasses:
        \begin{itemize}
            \item Planning
            \item Executing
            \item Monitoring
            \item Optimizing the sequence of processes that transform raw data into valuable insights.
        \end{itemize}
    \end{block}
\end{frame}

\begin{frame}[fragile]
    \frametitle{Significance of Effective Workflow Management - Part 1}
    \begin{enumerate}
        \item \textbf{Efficiency and Automation}
        \begin{itemize}
            \item Automating repetitive tasks minimizes manual intervention, leading to faster operations and fewer errors.
            \item \textit{Example:} Automated processing of daily reports allows real-time market response.
        \end{itemize}

        \item \textbf{Data Quality Assurance}
        \begin{itemize}
            \item Facilitates data validation and cleaning, ensuring integrity in analytics.
            \item \textit{Example:} Rules for null checks ensure only quality data is analyzed.
        \end{itemize}

        \item \textbf{Scalability}
        \begin{itemize}
            \item A structured workflow accommodates increased data volume, allowing easy scaling.
            \item \textit{Example:} E-commerce sites expand data pipelines to include new sources.
        \end{itemize}
    \end{enumerate}
\end{frame}

\begin{frame}[fragile]
    \frametitle{Significance of Effective Workflow Management - Part 2}
    \begin{enumerate}
        \setcounter{enumi}{3} % Continue numbering
        \item \textbf{Collaboration \& Visibility}
        \begin{itemize}
            \item Provides a collaborative environment for simultaneous user work while maintaining oversight.
            \item \textit{Example:} Apache Airflow dashboards illustrate task statuses, aiding in bottleneck identification.
        \end{itemize}

        \item \textbf{Error Detection and Recovery}
        \begin{itemize}
            \item Identifies errors in processing and allows automatic retries or personnel alerts.
            \item \textit{Example:} Automatic reconnect attempts or alerts when data sources fail.
        \end{itemize}
    \end{enumerate}
    
    \begin{block}{Key Points to Emphasize}
        \begin{itemize}
            \item Integration of data sources enhances analytical capabilities.
            \item Adaptability to changing business needs is crucial.
            \item Choose tools like Apache NiFi, Airflow, or Luigi based on requirements.
        \end{itemize}
    \end{block}
\end{frame}

\begin{frame}[fragile]
    \frametitle{Key Terminology - ETL (Extract, Transform, Load)}
    \begin{itemize}
        \item \textbf{Definition}: 
        ETL is a data processing framework that involves three key steps:
        \begin{itemize}
            \item \textbf{Extract}: Fetching data from various sources such as databases, APIs, and files.
            \item \textbf{Transform}: Cleaning and converting the extracted data into a suitable format.
            \item \textbf{Load}: Loading the transformed data into a target system for analytics and reporting.
        \end{itemize}
        \item \textbf{Example}: 
        A retail company extracts sales data, consolidates it by product, and loads it into a data warehouse.
    \end{itemize}
\end{frame}

\begin{frame}[fragile]
    \frametitle{Key Terminology - Orchestration}
    \begin{itemize}
        \item \textbf{Definition}: 
        Orchestration refers to automated coordination and management of data workflows.
        \item \textbf{Key Points}:
        \begin{itemize}
            \item Manages complex workflows involving multiple data sources and tasks.
            \item Ensures proper sequencing of tasks, handling dependencies and failures.
        \end{itemize}
        \item \textbf{Example}: 
        A user utilizes Apache Airflow to orchestrate a pipeline that extracts from an API, transforms the data, and loads it into a database.
    \end{itemize}
\end{frame}

\begin{frame}[fragile]
    \frametitle{Key Terminology - Batch vs. Stream Processing}
    \begin{itemize}
        \item \textbf{Batch Processing}:
        \begin{itemize}
            \item Processes data in chunks collected over time.
            \item \textbf{Example}: A nightly job that processes daily sales data for reporting.
        \end{itemize}
        \item \textbf{Stream Processing}:
        \begin{itemize}
            \item Processes data in real-time as it flows into the system.
            \item \textbf{Example}: Real-time updates in a social media analytics tool.
        \end{itemize}
        \item \textbf{Key Differences}:
        \begin{itemize}
            \item \textbf{Latency}: Batch has higher latency; stream has lower latency.
            \item \textbf{Complexity}: Stream processing requires managing continuous data streams.
        \end{itemize}
    \end{itemize}
\end{frame}

\begin{frame}[fragile]
    \frametitle{Overview of Apache Airflow}
    \begin{block}{Introduction}
        Apache Airflow is a popular open-source platform that allows users to programmatically schedule and monitor workflows, enabling effective management of data pipelines.
    \end{block}
\end{frame}

\begin{frame}[fragile]
    \frametitle{What is Apache Airflow?}
    Apache Airflow is an open-source platform designed to programmatically schedule and monitor workflows. It allows users to define workflows with a series of dependent tasks, enabling the efficient management of data pipelines.
\end{frame}

\begin{frame}[fragile]
    \frametitle{Key Features of Apache Airflow}
    \begin{itemize}
        \item \textbf{Dynamic Pipeline Generation:} Pipelines are defined in Python, allowing for flexible, dynamic generation of complex workflows.
        \item \textbf{Extensibility:} Users can create reusable components (Operators, Sensors, Hooks) for incorporation into multiple workflows.
        \item \textbf{Rich User Interface:} A convenient UI to visualize task flow, track progress, and troubleshoot workflow issues.
        \item \textbf{Scalability:} Airflow accommodates increasing workflow loads by enabling deployments across clustered machines for parallel processing.
    \end{itemize}
\end{frame}

\begin{frame}[fragile]
    \frametitle{Workflow Management with Airflow}
    Airflow manages workflows through Directed Acyclic Graphs (DAGs), which represent a collection of tasks organized by their dependencies and execution order.
    
    \begin{block}{Example of a Simple DAG}
    \begin{lstlisting}[language=Python]
from airflow import DAG
from airflow.operators.dummy_operator import DummyOperator
from airflow.operators.python_operator import PythonOperator
from datetime import datetime

def my_function():
    print("Task Completed")

default_args = {
    'owner': 'user',
    'start_date': datetime(2023, 10, 1),
}

dag = DAG('simple_dag', default_args=default_args, schedule_interval='@daily')

start = DummyOperator(task_id='start', dag=dag)

task1 = PythonOperator(task_id='task1', python_callable=my_function, dag=dag)
task2 = PythonOperator(task_id='task2', python_callable=my_function, dag=dag)

start >> [task1, task2]  # This defines the execution order
    \end{lstlisting}
    \end{block}
\end{frame}

\begin{frame}[fragile]
    \frametitle{Key Points to Emphasize}
    \begin{itemize}
        \item \textbf{Task Dependencies:} Clearly organize tasks that depend on the completion of others within the DAG.
        \item \textbf{Execution Flow:} Tasks are executed based on defined dependencies; circular dependencies are not permitted.
        \item \textbf{Scheduling:} Offers various scheduling options, including cron-like scheduling and event-triggered workflows.
        \item \textbf{Monitoring:} Built-in UI for monitoring execution status, observing logs, and managing retries.
    \end{itemize}
\end{frame}

\begin{frame}[fragile]
    \frametitle{Use Cases}
    \begin{itemize}
        \item \textbf{ETL Processes:} Automating data extraction, transformation, and loading tasks.
        \item \textbf{Machine Learning Pipelines:} Orchestrating tasks for data preparation, model training, and evaluation.
        \item \textbf{Data Integration Workflows:} Managing batch or stream data ingestion from various sources into databases or data lakes.
    \end{itemize}
\end{frame}

\begin{frame}[fragile]
    \frametitle{Summary}
    Apache Airflow is a powerful tool that simplifies workflow management for data pipelines. Its dynamic DAG definitions, extensive features, and scalability make it a preferred choice for data engineering teams. Understanding how to leverage Apache Airflow effectively is a critical skill for managing complex data workflows.
\end{frame}

\begin{frame}[fragile]
    \frametitle{Airflow Architecture - Overview}
    \begin{itemize}
        \item Apache Airflow enables users to define, schedule, and monitor workflows as Directed Acyclic Graphs (DAGs).
        \item Understanding its architecture is essential for effective data pipeline management.
    \end{itemize}
\end{frame}

\begin{frame}[fragile]
    \frametitle{Airflow Architecture - Key Components}
    \begin{enumerate}
        \item \textbf{Scheduler}
        \begin{itemize}
            \item \textbf{Function}: Executes tasks based on scheduling parameters and DAG dependencies.
            \item \textbf{How it Works}:
                \begin{itemize}
                    \item Scans for new DAGs and task instances needing execution.
                    \item Checks task states and handles dependencies.
                \end{itemize}
            \item \textbf{Example}: Ensures tasks like \texttt{extract}, \texttt{transform}, and \texttt{load} run in the correct sequence.
        \end{itemize}
        
        \item \textbf{Web Server}
        \begin{itemize}
            \item \textbf{Function}: Provides a user interface to interact with Airflow.
            \item \textbf{How it Works}:
                \begin{itemize}
                    \item Runs a web application accessible through a web browser.
                    \item Enables manual task execution, DAG management, and log viewing for debugging.
                \end{itemize}
            \item \textbf{Key Feature}: User Interface (UI) helps monitor long-running workflows.
        \end{itemize}

        \item \textbf{Workers}
        \begin{itemize}
            \item \textbf{Function}: Execute the actual tasks defined in DAGs, working in parallel.
            \item \textbf{How it Works}:
                \begin{itemize}
                    \item Workers poll the message queue for tasks, execute them, and update the Scheduler.
                    \item Multiple workers can be run to scale processing.
                \end{itemize}
            \item \textbf{Example}: Three independent tasks can run concurrently with three workers.
        \end{itemize}
    \end{enumerate}
\end{frame}

\begin{frame}[fragile]
    \frametitle{Airflow Architecture - Workflow and Key Points}
    \textbf{Workflow Steps:}
    \begin{enumerate}
        \item The Scheduler detects DAGs and initiates runnable tasks.
        \item Workers execute tasks based on Scheduler's instructions.
        \item Workers log outputs and statuses.
        \item The Web Server presents a real-time monitoring dashboard.
    \end{enumerate}
    
    \textbf{Key Points:}
    \begin{itemize}
        \item \textbf{Separation of Concerns}: Scheduling, execution, and UI are distinct, enhancing scalability.
        \item \textbf{Extensibility}: Custom plugins and operators can be integrated.
        \item \textbf{Scalability}: Adding more workers allows for handling larger workloads effectively.
    \end{itemize}
\end{frame}

\begin{frame}[fragile]
    \frametitle{Airflow Architecture - Example Code}
    \begin{block}{DAG Definition Example}
    \begin{lstlisting}[language=Python]
from airflow import DAG
from airflow.operators.dummy_operator import DummyOperator
from airflow.utils.dates import days_ago

default_args = {
    'owner': 'airflow',
    'start_date': days_ago(1),
}

dag = DAG('example_dag', default_args=default_args, schedule_interval='@daily')

start_task = DummyOperator(task_id='start', dag=dag)
end_task = DummyOperator(task_id='end', dag=dag)

start_task >> end_task
    \end{lstlisting}
    \end{block}
\end{frame}

\begin{frame}[fragile]
    \frametitle{Airflow Architecture - Conclusion}
    \begin{itemize}
        \item Understanding Apache Airflow's architecture is crucial for leveraging its full potential.
        \item The roles of Scheduler, Web Server, and Workers streamline task execution and monitoring.
        \item Airflow is a powerful solution for managing complex data pipelines.
    \end{itemize}
\end{frame}

\begin{frame}[fragile]
    \frametitle{Creating a Directed Acyclic Graph (DAG)}
    \begin{block}{Understanding Directed Acyclic Graphs (DAGs) in Apache Airflow}
        A Directed Acyclic Graph (DAG) is a fundamental concept used in workflow orchestration with Apache Airflow. 
        It represents a series of tasks and their dependencies, ensuring that tasks are executed in the correct order without circular dependencies.
    \end{block}
\end{frame}

\begin{frame}[fragile]
    \frametitle{Key Components of a DAG}
    \begin{itemize}
        \item \textbf{Directed:} Indicates a direction of dependency (e.g., Task A must complete before Task B starts).
        \item \textbf{Acyclic:} Ensures that the graph does not loop back on itself (no cyclic paths).
        \item \textbf{Graph:} Consists of tasks (nodes) and dependencies (edges).
    \end{itemize}
\end{frame}

\begin{frame}[fragile]
    \frametitle{Why are DAGs Important?}
    \begin{itemize}
        \item Defines the workflow and the order of execution of tasks.
        \item Ensures a clear structure for dependent tasks to be executed systematically.
        \item Facilitates better error handling and task monitoring.
    \end{itemize}
\end{frame}

\begin{frame}[fragile]
    \frametitle{Steps to Create a DAG in Apache Airflow}
    \begin{enumerate}
        \item \textbf{Install Apache Airflow:} 
        Make sure Airflow is installed in your environment. It can be installed via pip:
        \begin{lstlisting}[language=bash]
pip install apache-airflow
        \end{lstlisting}
        
        \item \textbf{Import Required Libraries:} 
        Start your Python script with the necessary imports:
        \begin{lstlisting}[language=python]
from airflow import DAG
from airflow.operators.dummy_operator import DummyOperator
from datetime import datetime
        \end{lstlisting}
        
        \item \textbf{Define the Default Arguments:} 
        These arguments will be applied to the tasks within the DAG:
        \begin{lstlisting}[language=python]
default_args = {
    'owner': 'your_name',
    'depends_on_past': False,
    'start_date': datetime(2023, 1, 1),
    'retries': 1,
}
        \end{lstlisting}
    \end{enumerate}
\end{frame}

\begin{frame}[fragile]
    \frametitle{Steps to Create a DAG in Apache Airflow (Cont.)}
    \begin{enumerate}[start=4]
        \item \textbf{Instantiate the DAG:} 
        Create a DAG instance:
        \begin{lstlisting}[language=python]
dag = DAG(
    'example_dag',
    default_args=default_args,
    description='An example DAG to demonstrate basic features',
    schedule_interval='@daily',
)
        \end{lstlisting}

        \item \textbf{Define Tasks:} 
        Use operators to define tasks:
        \begin{lstlisting}[language=python]
start = DummyOperator(task_id='start', dag=dag)
end = DummyOperator(task_id='end', dag=dag)
        \end{lstlisting}

        \item \textbf{Define Task Dependencies:} 
        Specify the order in which tasks should execute:
        \begin{lstlisting}[language=python]
start >> end  # start task finishes before end task begins
        \end{lstlisting}
    \end{enumerate}
\end{frame}

\begin{frame}[fragile]
    \frametitle{Key Points to Remember}
    \begin{itemize}
        \item Always ensure that your tasks are well-defined and that their dependencies reflect the actual order of execution required for your workflow.
        \item Utilize the Airflow UI to monitor DAG execution and troubleshoot any issues with task dependencies.
    \end{itemize}
\end{frame}

\begin{frame}[fragile]
    \frametitle{Operators and Tasks in Airflow}
    \begin{block}{Introduction to Operators}
        In Apache Airflow, \textbf{Operators} are a core component that define the actual work performed by an Airflow task. Each \textbf{Operator} corresponds to a specific type of task, managing interactions with various systems such as databases, cloud services, or file systems.
    \end{block}
\end{frame}

\begin{frame}[fragile]
    \frametitle{Types of Operators - Overview}
    \begin{enumerate}
        \item BashOperator
        \item PythonOperator
        \item BranchPythonOperator
        \item EmailOperator
        \item DummyOperator
    \end{enumerate}
    
    \begin{block}{Key Points}
        \begin{itemize}
            \item Operators describe what tasks are supposed to do within the DAG.
            \item Each operator has specific attributes for particular operations.
            \item Understanding operators helps choose the right tool for data workflows.
        \end{itemize}
    \end{block}
\end{frame}

\begin{frame}[fragile]
    \frametitle{Operators in Detail}
    \begin{itemize}
        \item \textbf{BashOperator:} Executes a single command in the Bash shell.
        \begin{lstlisting}[language=Python]
from airflow.operators.bash import BashOperator

bash_task = BashOperator(
    task_id='run_bash_command',
    bash_command='echo "Hello, World!"'
)
        \end{lstlisting}

        \item \textbf{PythonOperator:} Executes a Python callable as a task.
        \begin{lstlisting}[language=Python]
from airflow.operators.python import PythonOperator

def my_python_function():
    print("Hello from Python!")

python_task = PythonOperator(
    task_id='run_python_function',
    python_callable=my_python_function
)
        \end{lstlisting}

        \item \textbf{BranchPythonOperator:} Allows branching based on conditions.
        \begin{lstlisting}[language=Python]
from airflow.operators.python import BranchPythonOperator

def branch_logic():
    return 'task_a' if some_condition else 'task_b'

branch_task = BranchPythonOperator(
    task_id='branching_task',
    python_callable=branch_logic
)
        \end{lstlisting}
    \end{itemize}
\end{frame}

\begin{frame}[fragile]
    \frametitle{More Operators}
    \begin{itemize}
        \item \textbf{EmailOperator:} Sends email notifications. Effective for alerting users of task status.
        \begin{lstlisting}[language=Python]
from airflow.operators.email import EmailOperator

email_task = EmailOperator(
    task_id='send_email',
    to='user@example.com',
    subject='Airflow Task Notification',
    html_content='Your task has completed!'
)
        \end{lstlisting}

        \item \textbf{DummyOperator:} A placeholder operator that doesn’t perform any action.
        \begin{lstlisting}[language=Python]
from airflow.operators.dummy import DummyOperator

dummy_task = DummyOperator(
    task_id='dummy_task'
)
        \end{lstlisting}
    \end{itemize}
\end{frame}

\begin{frame}[fragile]
    \frametitle{Conclusion}
    Operators form the backbone of Airflow tasks. Effective task automation and management begins with the proper selection and implementation of these operators. 
    In subsequent slides, we will explore how these tasks can be monitored and managed through Airflow's user interface.
\end{frame}

\begin{frame}[fragile]
    \frametitle{Monitoring and Managing Workflows}
    \begin{block}{Overview}
        Monitoring and managing workflows is crucial for ensuring the smooth operation of data pipelines. Apache Airflow provides a robust user interface for tracking task status, understanding dependencies, and handling failures effectively.
    \end{block}
\end{frame}

\begin{frame}[fragile]
    \frametitle{Key Concepts}
    \begin{enumerate}
        \item \textbf{User Interface Overview}
            \begin{itemize}
                \item The Airflow UI visualizes workflows for easy interaction with data pipelines.
                \item Access the UI at \texttt{http://localhost:8080}.
            \end{itemize}
        \item \textbf{DAGs and Task Monitoring}
            \begin{itemize}
                \item \textbf{DAG (Directed Acyclic Graph)}: A collection of tasks with dependencies, outlining execution order.
                \item Color coding in the UI: 
                \begin{itemize}
                    \item Green: Success
                    \item Red: Failure
                    \item Yellow: Running
                    \item Gray: Skipped or Pending
                \end{itemize}
            \end{itemize}
    \end{enumerate}
\end{frame}

\begin{frame}[fragile]
    \frametitle{Monitoring Workflows}
    \begin{enumerate}
        \item \textbf{DAG Runs}
            \begin{itemize}
                \item Click on a specific DAG to view past and current runs.
                \item Use "Graph View" for visual identification of bottlenecks and dependencies.
            \end{itemize}
        \item \textbf{Task Instances}
            \begin{itemize}
                \item Click a task to see details:
                \begin{itemize}
                    \item Logs for debugging
                    \item Duration for performance assessment
                    \item Execution Date of the task
                \end{itemize}
            \end{itemize}
        \item \textbf{Monitoring Metrics}
            \begin{itemize}
                \item Analyze Task Duration and Retries for insights on performance.
                \item Utilize "Gantt View" for detailed task timing breakdown.
            \end{itemize}
    \end{enumerate}
\end{frame}

\begin{frame}[fragile]
    \frametitle{Managing Workflows}
    \begin{enumerate}
        \item \textbf{Triggering Manual Runs}
            \begin{itemize}
                \item DAGs can be manually triggered from the UI for testing or re-running a sequence.
            \end{itemize}
        \item \textbf{Pause and Unpause DAGs}
            \begin{itemize}
                \item Pausing halts all running and future executions until resumed—useful for maintenance.
            \end{itemize}
    \end{enumerate}
\end{frame}

\begin{frame}[fragile]
    \frametitle{Example: Basic Workflow Management}
    Consider a DAG named \texttt{etl\_process} with tasks \texttt{extract}, \texttt{transform}, and \texttt{load}:
    \begin{itemize}
        \item \textbf{Pause DAG}: Stop all tasks:
        \begin{lstlisting}
        airflow dags pause etl_process
        \end{lstlisting}
        \item \textbf{Manual Task Retry}: If \texttt{transform} fails, retry the task.
        \item \textbf{View Logs}: Select "View Logs" in the UI to troubleshoot.
    \end{itemize}
\end{frame}

\begin{frame}[fragile]
    \frametitle{Conclusion}
    \begin{itemize}
        \item The Airflow UI is essential for monitoring task status, dependencies, and performance metrics.
        \item Use visual tools (Graph and Gantt views) for effective workflow oversight.
        \item Managing DAGs through pausing, triggering, and retries is crucial for operational control.
        \item Mastering the UI ensures reliability and efficiency of data pipelines.
    \end{itemize}
\end{frame}

\begin{frame}[fragile]
    \frametitle{Code Snippet Example}
    A simple DAG definition in Python:
    \begin{lstlisting}[language=Python]
    from airflow import DAG
    from airflow.operators.bash_operator import BashOperator
    from datetime import datetime

    default_args = {
        'owner': 'airflow',
        'start_date': datetime(2023, 1, 1),
        'retries': 1
    }

    dag = DAG('etl_process', default_args=default_args, schedule_interval='@daily')

    extract_task = BashOperator(task_id='extract', bash_command='python extract.py', dag=dag)
    transform_task = BashOperator(task_id='transform', bash_command='python transform.py', dag=dag)
    load_task = BashOperator(task_id='load', bash_command='python load.py', dag=dag)

    extract_task >> transform_task >> load_task # Setting dependencies
    \end{lstlisting}
\end{frame}

\begin{frame}[fragile]
    \frametitle{Error Handling and Retries}
    \begin{block}{Understanding Error Handling in Data Pipelines}
        \begin{itemize}
            \item \textbf{Definition}: Anticipating, detecting, and responding to data processing errors.
            \item \textbf{Importance}: Reduces downtime and maintains data integrity; prevents data loss and incorrect insights.
        \end{itemize}
    \end{block}
\end{frame}

\begin{frame}[fragile]
    \frametitle{Common Types of Errors in Data Pipelines}
    \begin{enumerate}
        \item \textbf{Transient Errors}: Temporary issues (e.g., network timeouts).
        \item \textbf{Permanent Errors}: Persistent issues (e.g., data format mismatches).
        \item \textbf{User Errors}: Errors due to incorrect configurations or manual inputs.
    \end{enumerate}
\end{frame}

\begin{frame}[fragile]
    \frametitle{Strategies for Effective Error Handling}
    \begin{enumerate}
        \item \textbf{Error Logging}:
            \begin{itemize}
                \item Capture error details (time, type, context) for troubleshooting.
                \item Example:
                \begin{lstlisting}[language=Python]
import logging

logging.basicConfig(filename='pipeline_errors.log', level=logging.ERROR)
logging.error('Data processing failed at step X', exc_info=True)
                \end{lstlisting}
            \end{itemize}
        \item \textbf{Graceful Degradation}:
            \begin{itemize}
                \item Continue processing despite errors in parts of the workflow. 
            \end{itemize}
        \item \textbf{Alerts and Notifications}:
            \begin{itemize}
                \item Notify developers when critical errors occur (e.g., via email or Slack).
            \end{itemize}
    \end{enumerate}
\end{frame}

\begin{frame}[fragile]
    \frametitle{Implementing Retry Policies}
    \begin{block}{Definition}
        A retry policy defines how and when to reprocess failed tasks, especially useful for transient errors.
    \end{block}
    \begin{block}{Key Elements of a Retry Policy}
        \begin{itemize}
            \item \textbf{Max Retries}: Limits to avoid endless loops.
            \item \textbf{Exponential Backoff}: Gradually increasing wait time between retries.
            \item \textbf{Timeouts}: Limits on response wait time before retrying.
        \end{itemize}
    \end{block}
    \begin{block}{Example in Airflow}
        \begin{lstlisting}[language=Python]
from airflow import DAG
from airflow.operators.dummy_operator import DummyOperator
from datetime import datetime, timedelta

default_args = {
    'retries': 3,
    'retry_delay': timedelta(seconds=5),
}

dag = DAG('example_dag', default_args=default_args, start_date=datetime(2023, 10, 20))

task = DummyOperator(
    task_id='dummy_task',
    dag=dag,
    retries=3,
    retry_delay=timedelta(seconds=5)
)
        \end{lstlisting}
    \end{block}
\end{frame}

\begin{frame}[fragile]
    \frametitle{Key Takeaways}
    \begin{itemize}
        \item Robust error handling and retry strategies ensure pipeline reliability.
        \item Effective logging, graceful degradation, and alerts improve manageability.
        \item A sound retry strategy mitigates the impact of transient errors.
    \end{itemize}
\end{frame}

\begin{frame}
    \frametitle{Integrating with Other Tools}
    \begin{block}{Overview of Integration in Apache Airflow}
        Apache Airflow is designed to facilitate complex workflows, enabling seamless integration with various databases and data processing tools.
    \end{block}
\end{frame}

\begin{frame}
    \frametitle{Key Integration Points}
    \begin{enumerate}
        \item \textbf{Databases}:
            \begin{itemize}
                \item Examples: MySQL, PostgreSQL, SQLite, MongoDB
                \item Airflow connects to databases using built-in hooks and operators.
            \end{itemize}
        
        \item \textbf{Data Processing Frameworks}:
            \begin{itemize}
                \item Examples: Apache Spark, Apache Hive, Dask
                \item Specific operators are available to run jobs in distinct cluster environments.
            \end{itemize}
        
        \item \textbf{Cloud Services}:
            \begin{itemize}
                \item Examples: AWS, Google Cloud Platform, Azure
                \item Interact through various providers like S3 and GCS hooks.
            \end{itemize}
        
        \item \textbf{Message Brokers}:
            \begin{itemize}
                \item Examples: Apache Kafka, RabbitMQ
                \item Enables real-time data processing and responsiveness with operators like KafkaProducerOperator.
            \end{itemize}
    \end{enumerate}
\end{frame}

\begin{frame}[fragile]
    \frametitle{Examples of Code Integration}
    \begin{block}{Database Example}
        \begin{lstlisting}[language=Python]
from airflow.providers.postgres.operators.postgres import PostgresOperator

update_data = PostgresOperator(
    task_id='update_data',
    sql='UPDATE sales SET amount = amount * 1.1 WHERE year = 2023',
    postgres_conn_id='my_postgres_db',
    dag=dag,
)
        \end{lstlisting}
    \end{block}

    \begin{block}{Data Processing Framework Example}
        \begin{lstlisting}[language=Python]
from airflow.providers.apache.spark.operators.spark_submit import SparkSubmitOperator

spark_task = SparkSubmitOperator(
    task_id='run_spark_job',
    application='s3://my-bucket/my_spark_app.py',
    name='spark_app',
    conf={'key': 'value'},
    dag=dag,
)
        \end{lstlisting}
    \end{block}
\end{frame}

\begin{frame}[fragile]
    \frametitle{Key Points to Emphasize}
    \begin{itemize}
        \item \textbf{Modular Architecture}:
            Airflow’s plugin architecture enables the addition of new integration tools easily.
        
        \item \textbf{Extensibility}:
            Custom operators and hooks can be developed for specific needs.
        
        \item \textbf{DAG Coordination}:
            Airflow's DAG structure allows users to orchestrate interdependent tasks, including interactions with external systems.
    \end{itemize}
\end{frame}

\begin{frame}
    \frametitle{Conclusion}
    Integration with other tools is one of Apache Airflow's strengths, providing users with the versatility to connect various components of their data ecosystem efficiently.
\end{frame}

\begin{frame}
    \frametitle{Next Steps}
    For hands-on experience, refer to the upcoming slide: "Case Study: Building a Sample Data Pipeline," where we will see these integrations in action.
\end{frame}

\begin{frame}
    \frametitle{Case Study: Building a Sample Data Pipeline}
    \begin{block}{Introduction to Data Pipelines}
        A \textbf{data pipeline} is a series of data processing steps that involve collecting, processing, and moving data from one system to another. 
        Apache Airflow is a powerful tool for orchestrating these pipelines by allowing you to define, schedule, and monitor workflows.
    \end{block}
\end{frame}

\begin{frame}
    \frametitle{Case Study Overview}
    In this case study, we will walk through building a simple data pipeline using \textbf{Apache Airflow}, demonstrating the process of:
    \begin{enumerate}
        \item Extracting data from a source
        \item Transforming the data
        \item Loading it into a destination for analysis
    \end{enumerate}
\end{frame}

\begin{frame}
    \frametitle{Components of the Data Pipeline}
    \begin{enumerate}
        \item \textbf{Extract}:
            \begin{itemize}
                \item \textbf{Source}: Data pulled from a CSV file located in an S3 bucket.
                \item \textbf{Example}: Monthly sales data for various products.
            \end{itemize}
        \item \textbf{Transform}:
            \begin{itemize}
                \item \textbf{Processing}: Clean and aggregate data, handling missing values and calculating totals.
                \item \textbf{Example Transformation}: Convert sales prices from USD to EUR and calculate total sales per product.
            \end{itemize}
        \item \textbf{Load}:
            \begin{itemize}
                \item \textbf{Destination}: Load transformed data into a PostgreSQL database for reporting.
                \item \textbf{Example Table}: A table named \texttt{monthly\_sales\_summary}.
            \end{itemize}
    \end{enumerate}
\end{frame}

\begin{frame}[fragile]
    \frametitle{Airflow Implementation Steps}
    \begin{enumerate}
        \item \textbf{Define Your DAG (Directed Acyclic Graph)}:
            A DAG in Airflow represents your workflow and contains tasks and their dependencies.
            \begin{lstlisting}[language=Python]
from airflow import DAG
from airflow.operators.python_operator import PythonOperator
from datetime import datetime

def extract():
    # Code to extract data from S3

def transform():
    # Code to transform the data

def load():
    # Code to load data into PostgreSQL

default_args = {
    'owner': 'airflow',
    'start_date': datetime(2023, 1, 1),
}

dag = DAG('sample_data_pipeline', default_args=default_args, schedule_interval='@monthly')

extract_task = PythonOperator(task_id='extract', python_callable=extract, dag=dag)
transform_task = PythonOperator(task_id='transform', python_callable=transform, dag=dag)
load_task = PythonOperator(task_id='load', python_callable=load, dag=dag)

extract_task >> transform_task >> load_task
            \end{lstlisting}
        \item \textbf{Schedule the Workflow}:
            Define how often the pipeline should run (e.g., daily, weekly). In this case, it runs monthly.
    \end{enumerate}
\end{frame}

\begin{frame}
    \frametitle{Airflow Implementation Steps (cont'd)}
    \begin{itemize}
        \item \textbf{Monitor and Log}:
            Utilize Airflow's UI to monitor the progress of your pipeline and check logs for errors during execution.
    \end{itemize}
\end{frame}

\begin{frame}
    \frametitle{Key Points to Emphasize}
    \begin{itemize}
        \item \textbf{Modular Design}: Each step (Extract, Transform, Load) is handled as a separate task.
        \item \textbf{Scalability}: Airflow allows scaling of tasks and executions as data grows.
        \item \textbf{Integration}: Airflow integrates with various data sources and destinations (e.g., databases, cloud storage).
    \end{itemize}
\end{frame}

\begin{frame}
    \frametitle{Conclusion}
    By leveraging Apache Airflow, you can efficiently build a robust data pipeline that automates the ETL process, enabling timely and insights-driven decision-making.
\end{frame}

\begin{frame}
    \frametitle{Next Steps}
    In the following slides, we will explore best practices for designing and managing data pipelines to ensure efficiency, reliability, and scalability.
\end{frame}

\begin{frame}[fragile]
    \frametitle{Best Practices for Data Pipelines - Overview}
    \begin{itemize}
        \item Define Clear Objectives
        \item Design for Scalability and Flexibility
        \item Implement Strong Monitoring and Logging
        \item Ensure Data Quality
        \item Maintain Documentation
        \item Test Thoroughly
        \item Optimize Performance
        \item Enforce Security Measures
    \end{itemize}
\end{frame}

\begin{frame}[fragile]
    \frametitle{Best Practices for Data Pipelines - 1. Define Clear Objectives}
    \begin{itemize}
        \item Establish what the data pipeline intends to achieve.
        \item Define goals such as:
        \begin{itemize}
            \item Data ingestion
            \item Transformation
            \item Storage
            \item Analysis
        \end{itemize}
        \item \textbf{Example:} Process sales transaction data daily to generate real-time sales reports for analytics.
    \end{itemize}
\end{frame}

\begin{frame}[fragile]
    \frametitle{Best Practices for Data Pipelines - 2. Design for Scalability and Flexibility}
    \begin{itemize}
        \item Build pipelines that easily scale with data volume and complexity.
        \item \textbf{Illustration:} Use microservices architecture to allow independent scaling based on load.
    \end{itemize}
\end{frame}

\begin{frame}[fragile]
    \frametitle{Best Practices for Data Pipelines - 3. Implement Strong Monitoring and Logging}
    \begin{itemize}
        \item Effective monitoring identifies issues before impacting users.
        \item Implement logging to track data flow and errors.
        \item \textbf{Key Points:}
        \begin{itemize}
            \item Use tools like Apache Airflow's monitoring features.
            \item Set up alerts for failure conditions or anomalies.
        \end{itemize}
    \end{itemize}
\end{frame}

\begin{frame}[fragile]
    \frametitle{Best Practices for Data Pipelines - 4. Ensure Data Quality}
    \begin{itemize}
        \item Incorporate data validation and cleansing steps.
        \item \textbf{Example:} Use checksums or data profiling to validate data integrity.
        \item \textbf{Code Snippet:}
        \begin{lstlisting}[language=Python]
def validate_data(data):
    if data.isnull().values.any():
        raise ValueError("Data contains null values.")
    return True
        \end{lstlisting}
    \end{itemize}
\end{frame}

\begin{frame}[fragile]
    \frametitle{Best Practices for Data Pipelines - 5. Maintain Documentation}
    \begin{itemize}
        \item Create thorough documentation for each component.
        \item \textbf{Key Points:}
        \begin{itemize}
            \item Document schema changes in databases.
            \item Maintain an inventory of data sources and transformation processes.
        \end{itemize}
    \end{itemize}
\end{frame}

\begin{frame}[fragile]
    \frametitle{Best Practices for Data Pipelines - 6. Test Thoroughly}
    \begin{itemize}
        \item Implement automated testing for pipeline stages.
        \item \textbf{Illustration:} Use unit tests for transformation functions and integration tests for workflows.
    \end{itemize}
\end{frame}

\begin{frame}[fragile]
    \frametitle{Best Practices for Data Pipelines - 7. Optimize Performance}
    \begin{itemize}
        \item Optimize data workflows by minimizing data movement.
        \item \textbf{Key Points:}
        \begin{itemize}
            \item Use efficient storage formats (e.g., Parquet, Avro).
            \item Batch processing can be more efficient than real-time processing for large datasets.
            \item Use partitioning in databases to improve query performance.
        \end{itemize}
    \end{itemize}
\end{frame}

\begin{frame}[fragile]
    \frametitle{Best Practices for Data Pipelines - 8. Enforce Security Measures}
    \begin{itemize}
        \item Protect sensitive data via encryption and secure access controls.
        \item \textbf{Example:} Use role-based access control (RBAC) to restrict access to sensitive data.
    \end{itemize}
\end{frame}

\begin{frame}[fragile]
    \frametitle{Best Practices for Data Pipelines - Conclusion}
    \begin{itemize}
        \item Employing these best practices enables building robust, efficient, and scalable data pipelines.
        \item Ensures smooth workflows and a high level of confidence in the data delivered to stakeholders.
    \end{itemize}
\end{frame}

\begin{frame}[fragile]
    \frametitle{Challenges in Data Pipeline Management}
    \begin{block}{Understanding Data Pipelines}
        Data pipelines are sequences of data processing steps involving:
        \begin{itemize}
            \item Extracting data from sources
            \item Transforming it into a suitable format
            \item Loading it into a destination database or data store
        \end{itemize}
        Effective management is crucial for ensuring:
        \begin{itemize}
            \item Data integrity
            \item Timely processing
            \item Accurate insights
        \end{itemize}
    \end{block}
\end{frame}

\begin{frame}[fragile]
    \frametitle{Common Challenges in Data Pipeline Management}
    \begin{enumerate}
        \item \textbf{Data Quality Issues}
            \begin{itemize}
                \item Incomplete, inaccurate, or inconsistent data affects analytics outcomes.
                \item \textit{Example:} Duplicate entries in customer data can lead to erroneous analysis.
                \item \textbf{Key Point:} Implement validation checks at each stage.
            \end{itemize}
        
        \item \textbf{Scalability Concerns}
            \begin{itemize}
                \item Pipelines that perform well with small datasets may struggle with larger volumes.
                \item \textit{Example:} Real-time data processing pipelines must scale during data bursts.
                \item \textbf{Key Point:} Design with scalability in mind, potentially using cloud solutions.
            \end{itemize}
    \end{enumerate}
\end{frame}

\begin{frame}[fragile]
    \frametitle{Additional Challenges in Data Pipeline Management}
    \begin{enumerate}
        \setcounter{enumi}{2}
        \item \textbf{Inadequate Monitoring and Logging}
            \begin{itemize}
                \item Lack of visibility into pipeline performance complicates issue resolution.
                \item \textit{Example:} Unnoticed failures can lead to incorrect data weeks later.
                \item \textbf{Key Point:} Implement comprehensive monitoring solutions.
            \end{itemize}
        
        \item \textbf{Complex Dependencies}
            \begin{itemize}
                \item Interconnected processes can make maintenance challenging.
                \item \textit{Example:} Changes upstream may lead to downstream failures.
                \item \textbf{Key Point:} Use workflow management tools to visualize dependencies.
            \end{itemize}
        
        \item \textbf{Frequent Changes in Source Data Formats}
            \begin{itemize}
                \item Changes in formats can break downstream pipelines.
                \item \textit{Example:} Unprepared pipelines can fail with new data schemas.
                \item \textbf{Key Point:} Design flexible parsing logic.
            \end{itemize}
        
        \item \textbf{Latency Issues}
            \begin{itemize}
                \item Unoptimized pipelines can lead to delays.
                \item \textit{Example:} Slow ETL processes hinder decision-making.
                \item \textbf{Key Point:} Regularly optimize the workflow to reduce latency.
            \end{itemize}
        
        \item \textbf{Security Risks}
            \begin{itemize}
                \item Sensitive data is at risk without proper security measures.
                \item \textit{Example:} Lack of encryption could expose data.
                \item \textbf{Key Point:} Implement encryption and access controls.
            \end{itemize}
    \end{enumerate}
\end{frame}

\begin{frame}[fragile]
    \frametitle{Conclusion and Call to Action}
    \begin{block}{Conclusion}
        Managing data pipelines requires proactive approaches:
        \begin{itemize}
            \item Focus on data quality and scalability
            \item Ensure adequate monitoring and manage dependencies
            \item Adapt to format changes and address latency
            \item Implement robust security measures
        \end{itemize}
    \end{block}
    \begin{block}{Call to Action}
        Consider these challenges when designing your data pipelines
        and leverage best practices for successful data management!
    \end{block}
\end{frame}

\begin{frame}[fragile]
    \frametitle{Future Trends in Data Workflow Management}
    \begin{block}{Introduction}
        The landscape of data workflow management is rapidly evolving, influenced by technology advancements and increasing data demands. This presentation explores future trends shaping data pipeline technologies and workflow management systems.
    \end{block}
\end{frame}

\begin{frame}[fragile]
    \frametitle{Key Trends in Data Workflow Management - Part 1}
    \begin{enumerate}
        \item \textbf{Automation and Orchestration}
        \begin{itemize}
            \item \textbf{Explanation}: Automation involves using software tools to manage data workflows without human intervention. Orchestration coordinates these automated processes.
            \item \textbf{Example}: Tools like Apache Airflow schedule complex data workflows, triggering tasks automatically based on completion of previous steps.
        \end{itemize}

        \item \textbf{Serverless Architecture}
        \begin{itemize}
            \item \textbf{Explanation}: Serverless computing allows developers to build applications without managing servers. This model reduces infrastructure management overhead.
            \item \textbf{Example}: AWS Lambda executes code in response to events, enhancing the efficiency of data processing tasks.
        \end{itemize}
        
        \item \textbf{Real-Time Data Processing}
        \begin{itemize}
            \item \textbf{Explanation}: The growing demand for real-time insights necessitates workflows capable of instant data processing. 
            \item \textbf{Example}: Apache Kafka processes streaming data in real time, enabling timely decision-making.
        \end{itemize}
    \end{enumerate}
\end{frame}

\begin{frame}[fragile]
    \frametitle{Key Trends in Data Workflow Management - Part 2}
    \begin{enumerate}
        \setcounter{enumi}{3} % Continue enumeration
        \item \textbf{Integration of Machine Learning and AI}
        \begin{itemize}
            \item \textbf{Explanation}: Incorporating machine learning models enhances decision-making and automates complex data handling.
            \item \textbf{Example}: Automated Machine Learning (AutoML) tools streamline model selection and hyperparameter tuning.
        \end{itemize}
        
        \item \textbf{Data Governance and Compliance}
        \begin{itemize}
            \item \textbf{Explanation}: Tightening global data regulations necessitate compliance within data workflows. 
            \item \textbf{Example}: Tools like Collibra track data usage and ensure adherence to compliance policies.
        \end{itemize}
        
        \item \textbf{Low-Code/No-Code Platforms}
        \begin{itemize}
            \item \textbf{Explanation}: These platforms allow users to create workflows with minimal coding knowledge, promoting accessibility.
            \item \textbf{Example}: Zapier enables users to create automated workflows (Zaps) connecting various applications.
        \end{itemize}
        
        \item \textbf{Collaborative Features}
        \begin{itemize}
            \item \textbf{Explanation}: Integrated collaboration tools in workflow management systems are essential as teams become more distributed.
            \item \textbf{Example}: Tools like Notion and Trello enhance teamwork on data projects.
        \end{itemize}
    \end{enumerate}
\end{frame}

\begin{frame}[fragile]
    \frametitle{Conclusion and Key Points}
    \begin{block}{Conclusion}
        The future of data workflow management is shaped by automation, real-time processing, and enhanced collaboration. Staying informed on these trends enables organizations to harness the full potential of their data and remain competitive.
    \end{block}

    \begin{block}{Key Points to Emphasize}
        \begin{itemize}
            \item Automation and orchestration enhance efficiency.
            \item Serverless architecture impacts cost and management.
            \item Real-time data processing is crucial for agile decision-making.
            \item Integrating machine learning boosts workflow intelligence.
            \item Ensuring data governance amidst evolving compliance requirements.
        \end{itemize}
    \end{block}
\end{frame}

\begin{frame}[fragile]
  \frametitle{Conclusion and Summary - Key Concepts Recap}
  \begin{itemize}
    \item \textbf{Data Pipelines}: 
    A series of data-processing steps for ingesting, processing, and storing data to ensure data harmonization and accessibility.
    
    \item \textbf{Workflow Management}: 
    Coordination of complex processes involving multiple tasks and data sources, ensuring tasks are executed in the correct order.
  \end{itemize}
\end{frame}

\begin{frame}[fragile]
  \frametitle{Conclusion and Summary - Apache Airflow}
  \begin{block}{What is Apache Airflow?}
    An open-source platform designed for programmatically authoring, scheduling, and monitoring workflows.
  \end{block}
  
  \begin{itemize}
    \item \textbf{Key Features of Airflow}:
      \begin{itemize}
        \item \textbf{Directed Acyclic Graphs (DAGs)}: Visual representation of task flows and dependencies.
        \item \textbf{Task Dependencies}: Allows for defining relationships among tasks to ensure correct execution order.
        \item \textbf{Extensibility}: Can be extended with custom plugins and integrates seamlessly with various services.
      \end{itemize}
  \end{itemize}
\end{frame}

\begin{frame}[fragile]
  \frametitle{Conclusion and Summary - Example and Key Points}
  \begin{block}{Example of a Simple DAG in Airflow}
  \begin{lstlisting}[language=Python]
from airflow import DAG
from airflow.operators.dummy_operator import DummyOperator
from airflow.operators.python_operator import PythonOperator
from datetime import datetime

def sample_task():
    print("Executing Task")

with DAG('my_sample_dag', start_date=datetime(2023, 10, 1), schedule_interval='@daily') as dag:
    start = DummyOperator(task_id='start')
    task1 = PythonOperator(task_id='task1', python_callable=sample_task)
    end = DummyOperator(task_id='end')

    start >> task1 >> end  # Set task dependencies
  \end{lstlisting}
  \end{block}

  \begin{itemize}
    \item \textbf{Key Points to Emphasize}:
      \begin{enumerate}
        \item Automation and Efficiency in ETL processes.
        \item Monitoring and logging capabilities through Airflow's UI.
        \item Scalability to handle complex datasets and workflows.
        \item Versatility in integrating with various tools and data sources.
      \end{enumerate}
  \end{itemize}
\end{frame}

\begin{frame}[fragile]
  \frametitle{Conclusion and Summary - Final Thoughts}
  \begin{block}{Conclusion}
    Understanding the role of data pipelines and systems like Apache Airflow is crucial in modern data engineering. 
  \end{block}
  
  \begin{itemize}
    \item Future trends indicate a shift towards intelligent, automated systems and AI integration for predictive data processing.
    \item Equipped with these tools, you can build robust frameworks that enhance decision-making and operational efficiency.
  \end{itemize}
\end{frame}


\end{document}