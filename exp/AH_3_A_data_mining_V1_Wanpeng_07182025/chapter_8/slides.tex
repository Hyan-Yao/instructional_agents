\documentclass[aspectratio=169]{beamer}

% Theme and Color Setup
\usetheme{Madrid}
\usecolortheme{whale}
\useinnertheme{rectangles}
\useoutertheme{miniframes}

% Additional Packages
\usepackage[utf8]{inputenc}
\usepackage[T1]{fontenc}
\usepackage{graphicx}
\usepackage{booktabs}
\usepackage{listings}
\usepackage{amsmath}
\usepackage{amssymb}
\usepackage{xcolor}
\usepackage{tikz}
\usepackage{pgfplots}
\pgfplotsset{compat=1.18}
\usetikzlibrary{positioning}
\usepackage{hyperref}

% Custom Colors
\definecolor{myblue}{RGB}{31, 73, 125}
\definecolor{mygray}{RGB}{100, 100, 100}
\definecolor{mygreen}{RGB}{0, 128, 0}
\definecolor{myorange}{RGB}{230, 126, 34}
\definecolor{mycodebackground}{RGB}{245, 245, 245}

% Set Theme Colors
\setbeamercolor{structure}{fg=myblue}
\setbeamercolor{frametitle}{fg=white, bg=myblue}
\setbeamercolor{title}{fg=myblue}
\setbeamercolor{section in toc}{fg=myblue}
\setbeamercolor{item projected}{fg=white, bg=myblue}
\setbeamercolor{block title}{bg=myblue!20, fg=myblue}
\setbeamercolor{block body}{bg=myblue!10}
\setbeamercolor{alerted text}{fg=myorange}

% Set Fonts
\setbeamerfont{title}{size=\Large, series=\bfseries}
\setbeamerfont{frametitle}{size=\large, series=\bfseries}
\setbeamerfont{caption}{size=\small}
\setbeamerfont{footnote}{size=\tiny}

% Document Start
\begin{document}

\frame{\titlepage}

\begin{frame}[fragile]
    \frametitle{Introduction to Clustering Techniques}
    \begin{block}{What is Clustering?}
        Clustering is a data mining technique that groups similar items based on specific characteristics. 
        Items in the same cluster are more similar to each other than those in other clusters.
    \end{block}
\end{frame}

\begin{frame}[fragile]
    \frametitle{Importance of Clustering}
    \begin{itemize}
        \item \textbf{Analyzing Complex Datasets:} 
            Clustering helps discover patterns in complex datasets where traditional analysis may fail.
            It enhances understanding and insights by organizing data.
        \item \textbf{Applications:} 
            \begin{itemize}
                \item \textbf{Marketing:} Identifying customer segments for targeted advertising.
                \item \textbf{Biology:} Classifying species based on genetic data.
                \item \textbf{Image Processing:} Grouping pixels with similar colors.
                \item \textbf{Social Network Analysis:} Detecting communities within large networks.
            \end{itemize}
    \end{itemize}
\end{frame}

\begin{frame}[fragile]
    \frametitle{Key Features and Types of Clustering Techniques}
    \textbf{Key Features of Clustering:}
    \begin{itemize}
        \item \textit{Unsupervised Learning:} 
            Operates on unlabeled data, making it versatile.
        \item \textit{Distance Metrics:} 
            Clustering algorithms use metrics such as Euclidean, Manhattan, or Cosine similarity for measuring distance between data points.
    \end{itemize}

    \textbf{Types of Clustering Techniques:}
    \begin{enumerate}
        \item \textbf{Partitioning Methods:} 
            E.g., K-Means and K-Medoids.
        \item \textbf{Hierarchical Methods:} 
            Agglomerative (bottom-up) and Divisive (top-down).
        \item \textbf{Density-Based Methods:} 
            E.g., DBSCAN, identifying clusters based on the density of data points.
    \end{enumerate}
\end{frame}

\begin{frame}[fragile]
    \frametitle{K-Means Algorithm}
    \textbf{K-Means Algorithm Steps:}
    \begin{enumerate}
        \item \textbf{Assign clusters:}
        \begin{equation}
            C_i = \{ x_j : \text{argmin}_{k} || x_j - \mu_k ||^2 \}
        \end{equation}
        \item \textbf{Update centroids:}
        \begin{equation}
            \mu_k = \frac{1}{|C_k|} \sum_{x_j \in C_k} x_j
        \end{equation}
    \end{enumerate}
\end{frame}

\begin{frame}[fragile]
    \frametitle{Conclusion}
    Clustering is an essential tool for data scientists to extract meaningful insights from complex datasets. 
    Mastering various clustering techniques and their context leads to better analytical results and informed decision-making.
\end{frame}

\begin{frame}[fragile]
    \frametitle{What is Clustering?}
    \begin{block}{Definition of Clustering}
        Clustering is a fundamental technique in data mining that groups a set of objects such that:
        \begin{itemize}
            \item Objects within the same group (or cluster) are more similar to each other 
            \item Objects in different groups are more dissimilar
        \end{itemize}
        The essence is to identify a structure in data based on attributes.
    \end{block}
    \begin{block}{Key Concept}
        Clusters are formed based on distance metrics, measuring similarity or dissimilarity.
    \end{block}
\end{frame}

\begin{frame}[fragile]
    \frametitle{Role and Goals of Clustering}
    \begin{block}{Role in Data Mining}
        Clustering helps analysts uncover patterns and insights from complex datasets. It's useful for:
        \begin{itemize}
            \item Exploratory Data Analysis
            \item Data Segmentation
            \item Anomaly Detection
        \end{itemize}
    \end{block}

    \begin{block}{Goals of Clustering}
        \begin{itemize}
            \item Maximize intra-cluster similarity
            \item Minimize inter-cluster similarity
        \end{itemize}
    \end{block}
\end{frame}

\begin{frame}[fragile]
    \frametitle{Applications of Clustering Techniques}
    \begin{enumerate}
        \item \textbf{Marketing:} Customer segmentation for targeted advertising.
        \item \textbf{Image Processing:} Classifying and compressing image datasets.
        \item \textbf{Biology:} Classifying species based on genetic information.
        \item \textbf{Social Network Analysis:} Identifying communities within social networks.
    \end{enumerate}
    
    \begin{block}{Illustrative Example}
        Consider a dataset of animals with features like weight, height, and species. Using K-means, we can group animals into clusters such as 'mammals', 'birds', and 'reptiles'.
    \end{block}
    
    \begin{block}{Key Takeaways}
        \begin{itemize}
            \item Clustering uncovers patterns in data.
            \item Enhances understanding of large datasets.
            \item The choice of algorithm and parameters impacts results significantly.
        \end{itemize}
    \end{block}
\end{frame}

\begin{frame}[fragile]
    \frametitle{K-means Clustering Formula}
    The K-means clustering algorithm iteratively assigns data points to the nearest centroid based on the objective function:
    \begin{equation}
        J(C, \mu) = \sum_{j=1}^{k}\sum_{x_i \in C_j} \| x_i - \mu_j \|^2
    \end{equation}
    Where:
    \begin{itemize}
        \item $ J $ is the objective function (sum of squared distances).
        \item $ C $ represents clusters.
        \item $ \mu $ are the centroids of clusters.
        \item $ x_i $ are the data points.
    \end{itemize}
\end{frame}

\begin{frame}[fragile]{Types of Clustering Techniques - Overview}
    \begin{block}{Clustering Overview}
        Clustering is a key concept in data analysis that involves grouping similar data points together. 
        Different approaches are designed for varying types of data and analytical goals. 
    \end{block}
\end{frame}

\begin{frame}[fragile]{Types of Clustering Techniques - Partitioning Methods}
    \frametitle{Types of Clustering Techniques - Partitioning Methods}
    \begin{itemize}
        \item \textbf{Description:} Divides the dataset into a predefined number of clusters (k). 
        Each data point belongs to the cluster with the nearest mean.
        \item \textbf{Example:} \textbf{K-means clustering}
            \begin{enumerate}
                \item Initialize k centroids randomly.
                \item Assign each point to the nearest centroid.
                \item Recalculate the centroids based on the assignments.
                \item Repeat steps 2 and 3 until centroids do not change significantly.
            \end{enumerate}
        \item \textbf{Key Point:} The quality depends on the initial choice of centroids and the value of k.
    \end{itemize}
\end{frame}

\begin{frame}[fragile]{Types of Clustering Techniques - Hierarchical & Density-Based Methods}
    \frametitle{Types of Clustering Techniques - Hierarchical & Density-Based}
    \begin{block}{Hierarchical Methods}
        \begin{itemize}
            \item \textbf{Description:} Creates a hierarchy of clusters, shown in a dendrogram.
            \item \textbf{Types:}
                \begin{itemize}
                    \item \textbf{Agglomerative:} Merges data points into clusters.
                    \item \textbf{Divisive:} Divides one cluster into smaller ones.
                \end{itemize}
            \item \textbf{Key Point:} Does not require a predefined number of clusters; provides insights into data structure.
        \end{itemize}
    \end{block}

    \begin{block}{Density-Based Methods}
        \begin{itemize}
            \item \textbf{Description:} Identifies clusters based on the density of data points.
            \item \textbf{Example:} \textbf{DBSCAN} (Density-Based Spatial Clustering of Applications with Noise).
            \item \textbf{Key Features:}
                \begin{itemize}
                    \item Finds arbitrarily shaped clusters.
                    \item Robust to noise and outliers.
                \end{itemize}
            \item \textbf{Key Point:} Requires parameters: $\epsilon$ (max distance for clustering) and \texttt{minPts} (minimum points for dense region).
        \end{itemize}
    \end{block}
\end{frame}

\begin{frame}[fragile]{Summary of Key Points and Mathematical Insight}
    \frametitle{Summary of Key Points and Mathematical Insight}
    \begin{block}{Summary of Key Points}
        \begin{itemize}
            \item \textbf{Partitioning Techniques (e.g., K-means):} Good for flat clusters; sensitive to initial conditions.
            \item \textbf{Hierarchical Techniques:} Flexibility in data exploration; produces a dendrogram.
            \item \textbf{Density-Based Techniques (e.g., DBSCAN):} Effective for irregular clusters; handles noise well.
        \end{itemize}
    \end{block}
    
    \begin{block}{Mathematical Insight for K-means}
        The distance between point $x$ and centroid $c$:
        \begin{equation}
        d(x, c) = \sqrt{\sum_{i=1}^{n}(x_i - c_i)^2}
        \end{equation}
    \end{block}
\end{frame}

\begin{frame}[fragile]
    \frametitle{K-means Clustering - Introduction}
    \begin{block}{Overview}
        K-means clustering is a widely-used partitioning technique that divides a dataset into \textbf{K distinct, non-overlapping subsets (clusters)}. 
        \begin{itemize}
            \item Aims to group data points such that points in the same cluster are more similar.
            \item Useful for exploratory data analysis, pattern recognition, and data compression.
        \end{itemize}
    \end{block}
\end{frame}

\begin{frame}[fragile]
    \frametitle{K-means Clustering - Algorithm Steps}
    \begin{enumerate}
        \item \textbf{Initialization}:
        \begin{itemize}
            \item Select K initial centroids randomly from the dataset.
        \end{itemize}
        \item \textbf{Assignment}:
        \begin{itemize}
            \item For each data point, calculate the distance to each centroid (using Euclidean distance) and assign to the nearest cluster:
            \[
            j = \underset{1 \leq k \leq K}{\arg \min} \, ||x_i - c_k||^2
            \]
        \end{itemize}
        \item \textbf{Update}:
        \begin{itemize}
            \item Recalculate centroids as the mean of assigned data points:
            \[
            c_k = \frac{1}{N_k} \sum_{x_i \in C_k} x_i
            \]
        \end{itemize}
        \item \textbf{Convergence}:
        \begin{itemize}
            \item Repeat assignment and update until centroids stabilize.
        \end{itemize}
    \end{enumerate}
\end{frame}

\begin{frame}[fragile]
    \frametitle{K-means Clustering - Key Points}
    \begin{block}{Key Considerations}
        \begin{itemize}
            \item \textbf{Number of Clusters (K)}: Predefined; optimal K may be determined using elbow method or silhouette analysis.
            \item \textbf{Distance Metric}: Generally uses Euclidean distance, but can vary.
            \item \textbf{Complexity}: Efficient with complexity of $O(nKdi)$.
        \end{itemize}
    \end{block}
    
    \begin{block}{Advantages}
        \begin{itemize}
            \item \textbf{Simplicity}: Easy to understand and implement.
            \item \textbf{Speed}: Converges quickly, suitable for large datasets.
            \item \textbf{Efficiency}: Works well with globular and equally sized clusters.
        \end{itemize}
    \end{block}
    
    \begin{block}{Limitations}
        \begin{itemize}
            \item \textbf{Choosing K}: Determining the optimal number of clusters can be challenging.
            \item \textbf{Sensitivity to Initialization}: Different initial centroids influence results.
            \item \textbf{Spherical Clusters Assumption}: Assumes globular clusters, may not perform well on non-globular datasets.
            \item \textbf{Outliers}: Sensitive to outliers, which can skew results significantly.
        \end{itemize}
    \end{block}
\end{frame}

\begin{frame}[fragile]
    \frametitle{K-means Clustering - Example}
    \begin{block}{Example Dataset}
        Consider the points in a 2D space:
        \[
        (2, 3), (3, 3), (6, 8), (8, 9)
        \]
        If we set \( K = 2 \):
        \begin{itemize}
            \item Randomly select two initial centroids, say \( (2, 3) \) and \( (6, 8) \).
            \item Assign points to clusters based on proximity.
            \item Update centroids and repeat until convergence.
        \end{itemize}
    \end{block}
    
    \begin{block}{Conclusion}
        K-means clustering is powerful for data clustering, with clear advantages and some challenges.
    \end{block}
\end{frame}

\begin{frame}[fragile]
    \frametitle{K-means Algorithm Steps - Overview}
    The K-means algorithm is a widely used clustering technique that segments data into distinctly defined groups based on their characteristics. Understanding the iterative steps involved in K-means is essential for effectively applying this algorithm.
    
    \begin{block}{Overview of Steps}
        \begin{enumerate}
            \item Initialization
            \item Assignment Phase
            \item Update Phase
        \end{enumerate}
    \end{block}
\end{frame}

\begin{frame}[fragile]
    \frametitle{K-means Algorithm Steps - Initialization and Assignment}
    
    \textbf{1. Initialization}
    \begin{itemize}
        \item \textbf{Select Number of Clusters (K):} Decide how many clusters (K) to create.
        \item \textbf{Randomly Initialize Centroids:} Choose K initial centroids from the dataset.
    \end{itemize}
    
    \textbf{2. Assignment Phase}
    \begin{itemize}
        \item \textbf{Assign Data Points to Closest Centroid:} Calculate the distance to each centroid:
        \begin{equation}
            \text{Distance} = \sqrt{\sum_{i=1}^{n}(x_i - c_j)^2}
        \end{equation}
        where \(x_i\) is a data point and \(c_j\) is the centroid of cluster \(j\).
        
        \item \textbf{Cluster Formation:} Each data point is assigned to the cluster of the nearest centroid.
    \end{itemize}
\end{frame}

\begin{frame}[fragile]
    \frametitle{K-means Algorithm Steps - Update Phase and Key Points}
    
    \textbf{3. Update Phase}
    \begin{itemize}
        \item \textbf{Recalculate Centroids:} Compute new centroid:
        \begin{equation}
            c_j = \frac{1}{N_j} \sum_{x_i \in C_j} x_i
        \end{equation}
        where \(N_j\) is the number of points in cluster \(C_j\).
        
        \item \textbf{Check for Convergence:} If centroids do not change significantly, the algorithm stops; otherwise, return to the assignment phase.
    \end{itemize}
    
    \textbf{Key Points to Emphasize}
    \begin{itemize}
        \item Sensitivity to initial centroid placement.
        \item Importance of selecting the right K, to be covered next.
        \item Assumes spherical clusters of similar size.
    \end{itemize}
\end{frame}

\begin{frame}[fragile]
    \frametitle{Choosing the Right K - Introduction}
    \begin{block}{Overview}
        In k-means clustering, selecting the appropriate number of clusters (k) is crucial to achieving meaningful results. This slide outlines two prominent techniques for determining the optimal k: the \textbf{Elbow Method} and the \textbf{Silhouette Score}.
    \end{block}
\end{frame}

\begin{frame}[fragile]
    \frametitle{Choosing the Right K - The Elbow Method}
    \begin{block}{Concept}
        The Elbow Method involves plotting the explained variance (or inertia) against the number of clusters (k) and identifying the point where the rate of decrease sharply changes, resembling an "elbow."
    \end{block}
    
    \begin{block}{Steps}
        \begin{enumerate}
            \item Run the k-means algorithm with different values of k (e.g., 1 to 10).
            \item Calculate the Within-Cluster Sum of Squares (WCSS) for each k:
            \begin{equation}
                \text{WCSS} = \sum_{i=1}^{k} \sum_{j=1}^{n_i} ||x_j - \mu_i||^2
            \end{equation}
            \item Plot WCSS against k.
        \end{enumerate}
    \end{block}
    
    \begin{block}{Interpretation}
        Look for the "elbow" point where the WCSS diminishes at a slower rate, indicating the optimal k.
    \end{block}
\end{frame}

\begin{frame}[fragile]
    \frametitle{Choosing the Right K - Silhouette Score}
    \begin{block}{Concept}
        The Silhouette Score assesses how similar an object is to its own cluster compared to other clusters, quantifying data point density and separation.
    \end{block}
    
    \begin{block}{Formula}
        \begin{equation}
            s = \frac{b - a}{\max(a, b)}
        \end{equation}
        Where:
        \begin{itemize}
            \item \( a \): average distance to points in the same cluster.
            \item \( b \): average distance to points in the nearest cluster.
        \end{itemize}
    \end{block}
    
    \begin{block}{Interpretation}
        A silhouette score between -1 and 1 indicates clustering quality:
        \begin{itemize}
            \item **Close to 1**: Well clustered.
            \item **Close to 0**: Near the decision boundary.
            \item **Negative values**: Likely misassigned to the wrong cluster.
        \end{itemize}
    \end{block}
\end{frame}

\begin{frame}[fragile]
    \frametitle{Choosing the Right K - Key Points and Code Snippet}
    \begin{block}{Key Points}
        \begin{itemize}
            \item Both methods help in objectively selecting the value of k based on data characteristics.
            \item Using multiple techniques (Elbow and Silhouette) can provide a more robust estimation of optimal clusters.
        \end{itemize}
    \end{block}
    
    \begin{block}{Python Code Snippet}
    \begin{lstlisting}[language=Python]
from sklearn.cluster import KMeans
from sklearn.metrics import silhouette_score
import matplotlib.pyplot as plt

# Sample data
data = ...

# Elbow Method
wcss = []
for k in range(1, 11):
    kmeans = KMeans(n_clusters=k)
    kmeans.fit(data)
    wcss.append(kmeans.inertia_)

plt.plot(range(1, 11), wcss)
plt.title('Elbow Method')
plt.xlabel('Number of clusters (k)')
plt.ylabel('WCSS')
plt.show()

# Silhouette Score
scores = []
for k in range(2, 11):
    kmeans = KMeans(n_clusters=k)
    kmeans.fit(data)
    score = silhouette_score(data, kmeans.labels_)
    scores.append(score)

plt.plot(range(2, 11), scores)
plt.title('Silhouette Score Method')
plt.xlabel('Number of clusters (k)')
plt.ylabel('Silhouette Score')
plt.show()
    \end{lstlisting}
    \end{block}
\end{frame}

\begin{frame}[fragile]
    \frametitle{Applications of K-means Clustering - Part 1}
    \begin{block}{Understanding K-means Clustering}
        K-means clustering is an unsupervised machine learning technique that partitions a dataset into K distinct, non-overlapping subsets (clusters). Each data point is assigned to the cluster with the nearest mean, which serves as the cluster's prototype.
    \end{block}
\end{frame}

\begin{frame}[fragile]
    \frametitle{Applications of K-means Clustering - Part 2}
    \begin{block}{Real-World Applications}
        \begin{itemize}
            \item \textbf{Market Segmentation}
                \begin{itemize}
                    \item \textbf{Description:} Used to segment consumers based on characteristics like purchase behavior and demographics.
                    \item \textbf{Example:} Retail companies identify distinct customer segments to tailor marketing campaigns.
                    \item \textbf{Benefit:} Improved targeting increases conversion rates and customer satisfaction.
                \end{itemize}
                
            \item \textbf{Image Compression}
                \begin{itemize}
                    \item \textbf{Description:} Reduces the number of colors in an image, aiding in file size reduction.
                    \item \textbf{Example:} Grouping pixels into K clusters lets each pixel be represented by a centroid color, thus reducing data size.
                    \item \textbf{Benefit:} Faster load times and reduced bandwidth usage, particularly in web development.
                \end{itemize}
                
            \item \textbf{Anomaly Detection}
                \begin{itemize}
                    \item \textbf{Description:} Detects outliers by identifying clusters and recognizing data points that do not fit.
                    \item \textbf{Example:} Flags suspicious transactions by clustering normal patterns in fraud detection.
                    \item \textbf{Benefit:} Enhances security and risk management.
                \end{itemize}
        \end{itemize}
    \end{block}
\end{frame}

\begin{frame}[fragile]
    \frametitle{Applications of K-means Clustering - Part 3}
    \begin{block}{Key Points to Emphasize}
        \begin{itemize}
            \item \textbf{Scalability:} K-means is suitable for large datasets across various domains.
            \item \textbf{Iterative Approach:} Refines clusters based on distance to centroids through iteration.
        \end{itemize}
    \end{block}
    
    \begin{block}{Notable Formula}
        K-means Objective Function: 
        \begin{equation}
        J = \sum_{i=1}^{K}\sum_{x_j \in C_i} ||x_j - \mu_i||^2
        \end{equation}
        \begin{itemize}
            \item where \(J\) is total within-cluster variance.
            \item \(C_i\) is the set of points in cluster \(i\).
            \item \(x_j\) is a point in cluster \(C_i\).
            \item \(\mu_i\) is the centroid of cluster \(i\).
        \end{itemize}
    \end{block}
    
    \begin{block}{Additional Notes}
        \begin{itemize}
            \item \textbf{Choosing K:} Sensitivity to the choice of K; use methods like the Elbow method or Silhouette Score.
            \item \textbf{Limitations:} Assumes spherical clusters and equal sizes, which might not apply universally.
        \end{itemize}
    \end{block}
\end{frame}

\begin{frame}
    \frametitle{Hierarchical Clustering - Introduction}
    \begin{block}{Overview}
        Hierarchical clustering is a cluster analysis method that builds a hierarchy of clusters.
        \begin{itemize}
            \item No need to specify the number of clusters in advance
            \item Useful in exploratory data analysis
        \end{itemize}
    \end{block}
\end{frame}

\begin{frame}
    \frametitle{Hierarchical Clustering - Types}
    \begin{block}{Two Main Types}
        \begin{enumerate}
            \item \textbf{Agglomerative Clustering}
            \begin{itemize}
                \item Bottom-up approach starting with individual points
                \item Merges clusters based on proximity
                \item Continues until one cluster remains
                \item Distance metrics: Euclidean, Manhattan, Cosine
            \end{itemize}

            \item \textbf{Divisive Clustering}
            \begin{itemize}
                \item Top-down approach starting with one cluster
                \item Recursively splits into smaller clusters
                \item Continues until single data points are left
            \end{itemize}
        \end{enumerate}
    \end{block}
\end{frame}

\begin{frame}[fragile]
    \frametitle{Hierarchical Clustering - Formulas and Example}
    \begin{block}{Distance Calculation}
        \begin{equation}
            d(A, B) = 
            \begin{cases}
                \min{d(a, b)} & \text{Single Linkage} \\
                \max{d(a, b)} & \text{Complete Linkage} \\
                \frac{1}{|A||B|} \sum_{a \in A} \sum_{b \in B} d(a,b) & \text{Average Linkage}
            \end{cases}
        \end{equation}
    \end{block}

    \begin{block}{Example Python Code}
        \begin{lstlisting}[language=python]
from scipy.cluster.hierarchy import dendrogram, linkage
import matplotlib.pyplot as plt

# Sample data
data = [[1, 2], [2, 3], [3, 3], [5, 8], [8, 8]]

# Perform hierarchical clustering
Z = linkage(data, 'ward')  # 'ward' minimizes variance
    
# Plot dendrogram
plt.figure(figsize=(10, 7))
dendrogram(Z)
plt.title('Hierarchical Clustering Dendrogram')
plt.xlabel('Data Points')
plt.ylabel('Distance')
plt.show()
        \end{lstlisting}
    \end{block}
\end{frame}

\begin{frame}[fragile]
    \frametitle{Agglomerative vs Divisive Clustering}
    \begin{block}{Overview of Hierarchical Clustering Techniques}
        Hierarchical clustering techniques categorize data into nested clusters. The two primary approaches are:
        \begin{enumerate}
            \item \textbf{Agglomerative Clustering}: A bottom-up approach.
            \item \textbf{Divisive Clustering}: A top-down approach.
        \end{enumerate}
    \end{block}
\end{frame}

\begin{frame}[fragile]
    \frametitle{Agglomerative Clustering}
    \begin{block}{Algorithm Steps}
        \begin{enumerate}
            \item \textbf{Initialization}: Start with each data point as an individual cluster.
            \item \textbf{Distance Calculation}: Compute the distance between all pairs of clusters.
            \item \textbf{Cluster Merging}: Identify the two closest clusters and merge them into a single cluster.
            \item \textbf{Iteration}: Repeat until only one cluster remains (or until a desired number of clusters is achieved).
        \end{enumerate}
    \end{block}

    \begin{block}{Distance Metrics}
        \begin{itemize}
            \item Euclidean distance
            \item Manhattan distance
            \item Cosine similarity
        \end{itemize}
    \end{block}

    \begin{block}{Use Cases}
        \begin{itemize}
            \item Market segmentation
            \item Image segmentation
            \item Social network analysis
        \end{itemize}
    \end{block}
\end{frame}

\begin{frame}[fragile]
    \frametitle{Example: Agglomerative Clustering}
    \begin{block}{Example}
        Imagine a dataset of customer spending behaviors. Agglomerative clustering can group together customers with similar spending patterns, helping businesses to target marketing efforts.
    \end{block}

    \begin{block}{Divisive Clustering}
        \begin{itemize}
            \item \textbf{Initialization}: Start with all data points in a single cluster.
            \item \textbf{Splitting}: Identify the 'most dissimilar' sub-cluster and divide it into two.
            \item \textbf{Iteration}: Continue splitting until every point is in its own cluster (or a desired number of clusters is achieved).
        \end{itemize}
    \end{block}
\end{frame}

\begin{frame}[fragile]
    \frametitle{Divisive Clustering}
    \begin{block}{Key Strategies for Splitting}
        \begin{itemize}
            \item Use K-means or another method to partition the identified sub-cluster.
            \item Choose splits based on variance or density within the sub-cluster.
        \end{itemize}
    \end{block}

    \begin{block}{Use Cases}
        \begin{itemize}
            \item Document clustering in text mining
            \item Gene expression analysis
            \item Customer behavior partitioning
        \end{itemize}
    \end{block}

    \begin{block}{Example}
        In a dataset of articles, divisive clustering could separate topics into broad categories before further subdividing them into finer groups.
    \end{block}
\end{frame}

\begin{frame}[fragile]
    \frametitle{Key Points and Visualization}
    \begin{block}{Key Points}
        \begin{itemize}
            \item \textbf{Agglomerative}: Bottom-up approach; effective for large datasets with complex structures.
            \item \textbf{Divisive}: Top-down approach; useful for datasets with a clear hierarchy.
        \end{itemize}
    \end{block}

    \begin{block}{Visualization}
        A dendrogram visualizes clusters formed from both methods. The upward formation represents agglomerative clustering, while the downward formation represents divisive clustering.
    \end{block}
\end{frame}

\begin{frame}[fragile]
    \frametitle{Conclusion}
    \begin{block}{Conclusion}
        Both agglomerative and divisive clustering offer unique advantages for data analysis. Their applications range from market analysis to biotechnology, serving as foundational techniques within hierarchical clustering frameworks.
    \end{block}

    \begin{block}{Formulas}
        For distance:
        \[
        d(x, y) = \sqrt{\sum_{i=1}^n (x_i - y_i)^2} \quad (\text{Euclidean distance})
        \]
    \end{block}
\end{frame}

\begin{frame}[fragile]
    \frametitle{Dendrogram Representation}
    \begin{block}{What is a Dendrogram?}
        A \textbf{dendrogram} is a tree-like diagram that reflects the arrangement of clusters produced by a hierarchical clustering algorithm. It visually represents the relationships between clusters and how they are grouped based on their similarities.
    \end{block}
    \begin{itemize}
        \item Helps interpret hierarchical clustering results
        \item Assists in decision-making about the number of clusters
    \end{itemize}
\end{frame}

\begin{frame}[fragile]
    \frametitle{Key Components of a Dendrogram}
    \begin{itemize}
        \item \textbf{Leaves}: Represents individual data points or observations in the dataset.
        \item \textbf{Branches}: Lines connecting leaves indicate how data points are clustered together.
        \item \textbf{Height}: Indicates distance or dissimilarity between clusters.
    \end{itemize}
    \begin{block}{Understanding Height}
        When two clusters are combined, the height at which they merge shows how similar they are.
    \end{block}
\end{frame}

\begin{frame}[fragile]
    \frametitle{Interpreting a Dendrogram}
    \begin{enumerate}
        \item \textbf{Visualizing Relationships}
            \begin{itemize}
                \item Clusters closer together are more similar.
                \item Join height indicates similarity: lower is more similar.
            \end{itemize}
        \item \textbf{Choosing the Number of Clusters}
            \begin{itemize}
                \item "Cut" the dendrogram at a certain height to define clusters.
                \item Groupings below the cut represent defined clusters.
            \end{itemize}
    \end{enumerate}
\end{frame}

\begin{frame}[fragile]
    \frametitle{Example of Dendrogram Interpretation}
    Consider a simplified dendrogram with five data points (A, B, C, D, E):
    
    \begin{center}
    \includegraphics[width=0.6\textwidth]{dendrogram_example.png} % Placeholder for dendrogram image
    \end{center}

    \begin{itemize}
        \item Points A and B are more closely related than C and D, indicated by their lower merging height.
        \item Cutting the dendrogram at a certain height results in clusters: {A, B} and {C, D, E}.
    \end{itemize}
\end{frame}

\begin{frame}[fragile]
    \frametitle{Mathematical Representation}
    Hierarchical clustering uses different linkage criteria:

    \begin{equation}
        d(A, B) = \min \{d(a_i, b_j)\;|\; a_i \in A,\, b_j \in B\} \quad \text{(Single Linkage)}
    \end{equation}
    
    \begin{equation}
        d(A, B) = \max \{d(a_i, b_j)\;|\; a_i \in A,\, b_j \in B\} \quad \text{(Complete Linkage)}
    \end{equation}
    
    \begin{equation}
        d(A, B) = \frac{1}{|A|\cdot|B|} \sum_{a_i \in A} \sum_{b_j \in B} d(a_i, b_j) \quad \text{(Average Linkage)}
    \end{equation}
\end{frame}

\begin{frame}[fragile]
    \frametitle{Key Points to Remember}
    \begin{itemize}
        \item Dendrograms are essential for visualizing hierarchical clustering results.
        \item Facilitate understanding of data relationships.
        \item Interpretation depends on linkage method.
        \item Cutting the dendrogram is critical for meaningful clusters.
    \end{itemize}
    \begin{block}{Next Steps}
        In the following slide, we will discuss techniques to determine the optimal number of clusters from a dendrogram.
    \end{block}
\end{frame}

\begin{frame}[fragile]
    \frametitle{Choosing the Number of Clusters in Hierarchical Clustering}
    \begin{block}{Introduction to Dendrograms}
        \begin{itemize}
            \item \textbf{Dendrograms}: A tree-like diagram representing clusters in hierarchical clustering.
            \item \textbf{Objective}: Determine the optimal number of clusters (k) for a dataset.
        \end{itemize}
    \end{block}
\end{frame}

\begin{frame}[fragile]
    \frametitle{Techniques to Determine the Number of Clusters - Part 1}
    \begin{enumerate}
        \item \textbf{Cutting the Dendrogram}
            \begin{itemize}
                \item A dendrogram can be cut at different levels to form distinct clusters.
                \item Each cut defines a threshold distance; clusters below this are separate groups.
                \item Height increases indicate merging of clusters. A downward cut defines the cluster number.
            \end{itemize}

        \item \textbf{Identifying Height Thresholds}
            \begin{itemize}
                \item The cut height impacts the number of clusters.
                \item Look for large vertical spaces between clusters for natural cuts representing well-separated clusters.
            \end{itemize}
    \end{enumerate}
\end{frame}

\begin{frame}[fragile]
    \frametitle{Techniques to Determine the Number of Clusters - Part 2}
    \begin{enumerate}[resume]
        \item \textbf{Statistical Methods}
            \begin{itemize}
                \item \textbf{Silhouette Method}: Measures similarity of an object to its own cluster versus others.
                \begin{equation}
                    s(i) = \frac{b(i) - a(i)}{\max(a(i), b(i))}
                \end{equation}
                \item Where:
                    \begin{itemize}
                        \item \( a(i) \): Average distance to points in the same cluster.
                        \item \( b(i) \): Average distance to points in the nearest cluster.
                    \end{itemize}
                \item A higher silhouette score indicates a better-defined cluster.
            \end{itemize}

        \item \textbf{The Gap Statistic}
            \begin{itemize}
                \item Compares total intracluster variation for different values of k with expected values under a null reference distribution.
            \end{itemize}
        
        \item \textbf{Elbow Method}
            \begin{itemize}
                \item Plot sum of squared errors (SSE) versus different k values.
                \item The "elbow" point indicates the optimal k value.
            \end{itemize}
    \end{enumerate}
\end{frame}

\begin{frame}[fragile]
    \frametitle{Key Points & Conclusion}
    \begin{block}{Key Points to Emphasize}
        \begin{itemize}
            \item Dendrograms provide a visual approach to identify clusters.
            \item Selection of cut height is critical and should be informed by visual and statistical insights.
            \item Confirming the chosen number of clusters through multiple methods is essential.
        \end{itemize}
    \end{block}

    \begin{block}{Conclusion}
        Choosing the right number of clusters involves combining visual analysis and statistical methods like silhouette coefficients or the elbow method for enhanced clustering effectiveness.
    \end{block}
\end{frame}

\begin{frame}[fragile]
    \frametitle{Applications of Hierarchical Clustering - Overview}
    \begin{block}{Brief Summary}
        Hierarchical clustering is a versatile technique that generates hierarchies of clusters. Key applications include:
        \begin{itemize}
            \item Biological Taxonomy
            \item Document Classification
            \item Customer Segmentation
        \end{itemize}
    \end{block}
\end{frame}

\begin{frame}[fragile]
    \frametitle{Applications of Hierarchical Clustering - Biological Taxonomy}
    \begin{block}{Concept}
        Hierarchical clustering is used in biology to classify and organize living organisms based on characteristics and DNA sequences.
    \end{block}
    
    \begin{block}{Example}
        In phylogenetics, species are clustered based on genetic similarities, leading to dendrogram constructions that illustrate evolutionary relationships. 
    \end{block}

    \begin{itemize}
        \item Helps visualize relationships among species.
        \item Aids in understanding evolutionary patterns.
    \end{itemize}
\end{frame}

\begin{frame}[fragile]
    \frametitle{Applications of Hierarchical Clustering - Document Classification}
    \begin{block}{Concept}
        In natural language processing, hierarchical clustering groups similar documents into categories based on content.
    \end{block}
    
    \begin{block}{Example}
        News articles can be clustered by topics using Hierarchical Agglomerative Clustering (HAC). Articles on environmental issues may cluster together regardless of the source.
    \end{block}
    
    \begin{itemize}
        \item Supports efficient organization of large datasets.
        \item Enhances search and retrieval processes in information systems.
    \end{itemize}
\end{frame}

\begin{frame}[fragile]
    \frametitle{Applications of Hierarchical Clustering - Customer Segmentation}
    \begin{block}{Concept}
        Businesses leverage hierarchical clustering to segment customers based on purchasing behavior, demographics, or preferences.
    \end{block}
    
    \begin{block}{Example}
        An e-commerce platform analyzes customer data to identify distinct groups (e.g., bargain hunters vs. luxury buyers), facilitating tailored marketing strategies.
    \end{block}
    
    \begin{itemize}
        \item Improves targeted marketing efforts.
        \item Helps personalize customer experiences.
    \end{itemize}
\end{frame}

\begin{frame}[fragile]
    \frametitle{Applications of Hierarchical Clustering - Summary and Further Considerations}
    \begin{block}{Summary}
        Hierarchical clustering is powerful for organizing data into meaningful structures across various domains, enhancing decision-making.
    \end{block}
    
    \begin{block}{Further Considerations}
        The choice of distance metric (e.g., Euclidean) and linkage criterion (e.g., single, complete) is crucial for effective clustering.
    \end{block}
    
    \begin{itemize}
        \item Integrating these concepts enriches students' understanding of hierarchical clustering in real-world scenarios.
    \end{itemize}
\end{frame}

\begin{frame}[fragile]
    \frametitle{Comparison of K-means and Hierarchical Clustering - Overview}
    \begin{block}{Summary}
        - K-means and Hierarchical Clustering are clustering techniques used to group data points based on similarity. 
        - They differ in approach, strengths, weaknesses, and applications. 
    \end{block}
\end{frame}

\begin{frame}[fragile]
    \frametitle{Basic Concepts}
    \begin{columns}
        \begin{column}{0.5\textwidth}
            \textbf{K-means Clustering:}
            \begin{itemize}
                \item \textbf{Definition:} Iterative algorithm that partitions data into K distinct clusters.
                \item \textbf{Process:}
                \begin{enumerate}
                    \item Initialize K centroids randomly.
                    \item Assign points to the nearest centroid.
                    \item Recalculate centroids.
                    \item Repeat until convergence.
                \end{enumerate}
            \end{itemize}
        \end{column}
        \begin{column}{0.5\textwidth}
            \textbf{Hierarchical Clustering:}
            \begin{itemize}
                \item \textbf{Definition:} Builds a hierarchy of clusters via agglomerative or divisive methods.
                \item \textbf{Types:}
                \begin{itemize}
                    \item \textit{Agglomerative:} Merging points step by step.
                    \item \textit{Divisive:} Splitting one cluster recursively.
                \end{itemize}
            \end{itemize}
        \end{column}
    \end{columns}
\end{frame}

\begin{frame}[fragile]
    \frametitle{Similarities and Differences}
    \begin{block}{Comparison Table}
        \begin{tabular}{|l|l|l|}
        \hline
        \textbf{Feature} & \textbf{K-means Clustering} & \textbf{Hierarchical Clustering} \\
        \hline
        Number of Clusters & Predefined (K) & Not predefined, uses dendrogram \\
        Scalability & Scalable (O(n*K)) & Less scalable (O(n²)) \\
        Distance Metric & Euclidean distance & Flexible, multiple metrics \\
        Cluster Shape & Spherical clusters & Complex shapes \\
        Final Output & Fixed clusters & Hierarchical structure \\
        \hline
        \end{tabular}
    \end{block}
\end{frame}

\begin{frame}[fragile]
    \frametitle{Strengths and Weaknesses}
    \textbf{K-means Clustering:}
    \begin{itemize}
        \item \textbf{Strengths:}
        \begin{itemize}
            \item Efficient for large datasets.
            \item Easy to implement.
            \item Works well with spherical clusters.
        \end{itemize}
        \item \textbf{Weaknesses:}
        \begin{itemize}
            \item Requires predefined K.
            \item Sensitive to outliers.
            \item Prone to local minima.
        \end{itemize}
    \end{itemize}

    \textbf{Hierarchical Clustering:}
    \begin{itemize}
        \item \textbf{Strengths:}
        \begin{itemize}
            \item No need to specify clusters.
            \item Produces a comprehensive dendrogram.
            \item Robust against noise.
        \end{itemize}
        \item \textbf{Weaknesses:}
        \begin{itemize}
            \item Computationally intensive (O(n²)).
            \item Cutting decisions can be subjective.
            \item Sensitive to scale; normalization often necessary.
        \end{itemize}
    \end{itemize}
\end{frame}

\begin{frame}[fragile]
    \frametitle{Example Applications}
    \begin{itemize}
        \item \textbf{K-means:} Customer segmentation for identifying purchasing behavior.
        \item \textbf{Hierarchical Clustering:} Biological taxonomy for classifying species with tree representation.
    \end{itemize}
\end{frame}

\begin{frame}[fragile]
    \frametitle{Key Takeaways}
    \begin{itemize}
        \item Choose K-means for scalability and efficiency with known K.
        \item Opt for Hierarchical Clustering for complex relationships without predefined clusters.
    \end{itemize}

    \begin{block}{Mathematical Formula for K-means}
        To assign a data point \( x_i \) to the nearest centroid \( C_k \):
        \begin{equation}
            \text{Assign}(x_i) = \arg \min_{k} \|x_i - C_k\|^2
        \end{equation}
    \end{block}

    \begin{block}{Conclusion}
        Both K-means and Hierarchical Clustering have niches in data analytics, with distinct advantages and drawbacks.
    \end{block}
\end{frame}

\begin{frame}[fragile]
    \frametitle{Challenges in Clustering}
    Clustering is a powerful technique in data analysis, but it can be challenging due to various factors. This presentation explores three primary challenges:
    \begin{itemize}
        \item Noise in Data
        \item High Dimensionality
        \item Determining Appropriate Distance Measures
    \end{itemize}
\end{frame}

\begin{frame}[fragile]
    \frametitle{Challenge 1: Noise in Data}
    \begin{block}{Definition}
        Noise refers to random errors or variances in measured data that do not reflect the true underlying patterns.
    \end{block}
    
    \begin{itemize}
        \item **Impact**: Noise can obscure the true structure within the data, leading to inaccurate cluster assignments.
        \item **Example**: An erroneous entry of spending \$0 instead of \$50 could mislead clustering algorithms about customer behavior.
    \end{itemize}

    \begin{block}{Key Points}
        \begin{itemize}
            \item Noise can introduce outliers that skew cluster formation.
            \item Preprocessing steps such as noise filtering using statistical methods are vital to improve clustering outcomes.
        \end{itemize}
    \end{block}
\end{frame}

\begin{frame}[fragile]
    \frametitle{Challenge 2: High Dimensionality}
    \begin{block}{Definition}
        High dimensionality occurs when data has a large number of features or variables, complicating the clustering process.
    \end{block}
    
    \begin{itemize}
        \item **Curse of Dimensionality**: As dimensions increase, the volume of space increases exponentially, making data points sparse.
        \item **Example**: In datasets with hundreds of features, points that seem similar in lower dimensions may actually be far apart in high dimensions.
    \end{itemize}
    
    \begin{block}{Key Points}
        \begin{itemize}
            \item Dimensionality reduction techniques, like PCA or t-SNE, can minimize dimensions while retaining data structure.
            \item Feature selection methods can help determine the most relevant variables for clustering.
        \end{itemize}
    \end{block}
\end{frame}

\begin{frame}[fragile]
    \frametitle{Challenge 3: Determining Appropriate Distance Measures}
    \begin{block}{Definition}
        Distance measures quantify how similar or dissimilar data points are. Common metrics include Euclidean, Manhattan, and cosine similarity.
    \end{block}
    
    \begin{itemize}
        \item **Considerations**: Different algorithms may require different distance measures.
        \item **Example**: For textual data, cosine similarity captures the angle between vectors better than Euclidean distance.
    \end{itemize}
    
    \begin{block}{Key Points}
        \begin{itemize}
            \item Selecting the right metric is crucial for acquiring meaningful clusters.
            \item Experimentation with multiple distance measures can help determine the most suitable option for clustering tasks.
        \end{itemize}
    \end{block}
\end{frame}

\begin{frame}[fragile]
    \frametitle{Conclusion}
    Clustering challenges such as noise, high dimensionality, and distance measures can greatly affect the outcome of analysis. Understanding these challenges and applying proper strategies can improve the quality and interpretability of clustering results.
\end{frame}

\begin{frame}[fragile]
    \frametitle{Did You Know?}
    Common distance calculations include:
    \begin{itemize}
        \item **Euclidean Distance**: 
            \begin{equation}
                d(x, y) = \sqrt{\sum_{i=1}^{n} (x_i - y_i)^2}
            \end{equation}
        \item **Manhattan Distance**: 
            \begin{equation}
                d(x, y) = \sum_{i=1}^{n} |x_i - y_i|
            \end{equation}
        \item **Cosine Similarity**: 
            \begin{equation}
                \text{cosine}(x, y) = \frac{x \cdot y}{||x|| \times ||y||}
            \end{equation}
    \end{itemize}
    Understanding how to address these challenges can significantly enhance the effectiveness of clustering techniques used in data science and machine learning.
\end{frame}

\begin{frame}[fragile]
    \frametitle{Future Directions in Clustering Techniques}
    \begin{block}{Overview}
        Clustering is a key technique in data mining that has evolved significantly over the years. As the volume and complexity of data continue to grow, innovative approaches to clustering are gaining traction. 
        This presentation highlights three emerging trends: 
        \begin{itemize}
            \item Fuzzy Clustering
            \item Automated Clustering
            \item Clustering in Large Datasets
        \end{itemize}
    \end{block}
\end{frame}

\begin{frame}[fragile]
    \frametitle{Fuzzy Clustering}
    \begin{block}{Concept}
        Unlike traditional hard clustering methods, fuzzy clustering allows for partial membership. Each data point can belong to multiple clusters with varying degrees.
    \end{block}
    
    \begin{block}{Example}
        In a dataset of animals, a "panther" might have fuzzy membership values such as:
        \begin{itemize}
            \item "Big Cats": 0.7
            \item "Endangered Species": 0.4
        \end{itemize}
    \end{block}

    \begin{equation}
        J = \sum_{i=1}^{c} \sum_{j=1}^{n} u_{ij}^m d_{ij}^2
    \end{equation}
    \begin{itemize}
        \item $c$: number of clusters
        \item $n$: number of data points
        \item $u_{ij}$: degree of membership of data point $j$ in cluster $i$
        \item $d_{ij}$: distance between data point $j$ and cluster center $i$
        \item $m$: fuzziness parameter ($m > 1$)
    \end{itemize}
\end{frame}

\begin{frame}[fragile]
    \frametitle{Automated Clustering}
    \begin{block}{Concept}
        Automating the clustering process allows algorithms to determine the number of clusters and optimize parameters without user intervention.
    \end{block}

    \begin{block}{Example}
        Autoencoders employ deep learning architectures to learn data representations, enabling direct clustering without prior feature engineering.
    \end{block}

    \begin{block}{Tools and Methods}
        Adaptive clustering solutions include:
        \begin{itemize}
            \item DBSCAN (Density-Based Spatial Clustering of Applications with Noise)
            \item OPTICS (Ordering Points to Identify the Clustering Structure)
        \end{itemize}
    \end{block}
\end{frame}

\begin{frame}[fragile]
    \frametitle{Clustering in Large Datasets}
    \begin{block}{Concept}
        Traditional clustering methods often become computationally infeasible with large datasets. New scalable techniques are developed for efficiency.
    \end{block}

    \begin{block}{Example}
        MiniBatch K-Means optimizes standard K-Means by processing small random batches, reducing memory usage and computation time.
    \end{block}

    \begin{block}{Key Approach}
        Utilizing parallel and distributed computing (e.g., Apache Spark) enables efficient execution of clustering algorithms across multiple nodes.
    \end{block}
\end{frame}

\begin{frame}[fragile]
    \frametitle{Conclusion}
    \begin{block}{Key Points to Emphasize}
        \begin{itemize}
            \item Fuzzy Clustering captures uncertainty in data with degrees of membership.
            \item Automated Clustering reduces the need for manual parameter tuning.
            \item Clustering in Large Datasets focuses on scalability and efficiency through innovative algorithms.
        \end{itemize}
    \end{block}
    Understanding these trends is essential for effectively leveraging modern clustering techniques in evolving data environments.
\end{frame}

\begin{frame}[fragile]
    \frametitle{Summary and Conclusion - Key Points Recap}
    \begin{block}{Clustering Techniques}
        Clustering techniques are powerful tools in data mining, allowing us to group similar data points together based on their attributes.
    \end{block}
    \begin{enumerate}
        \item \textbf{Definition and Purpose}:
        \begin{itemize}
            \item Clustering aims to partition a dataset into distinct groups.
            \item Objects in the same group are more similar to each other than to those in other groups.
        \end{itemize}
    \end{enumerate}
\end{frame}

\begin{frame}[fragile]
    \frametitle{Summary and Conclusion - Types of Clustering Algorithms}
    \begin{enumerate}
        \setcounter{enumi}{1}
        \item \textbf{Types of Clustering Algorithms}:
        \begin{itemize}
            \item \textbf{Partitioning Methods (e.g., k-means)}:
            \begin{itemize}
                \item Divides data into \( k \) clusters based on nearest mean.
                \item Example: Customer segmentation based on purchasing behaviors.
                \item Formula:
                \begin{equation}
                J = \sum_{i=1}^{k} \sum_{x \in C_i} \|x - \mu_i\|^2
                \end{equation}
                where \( \mu_i \) is the centroid of cluster \( C_i \).
            \end{itemize}
            \item \textbf{Hierarchical Methods}:
            \begin{itemize}
                \item Builds a hierarchy of clusters.
                \item Example: Organizing topics by relevant keywords.
            \end{itemize}
            \item \textbf{Density-Based Methods (e.g., DBSCAN)}:
            \begin{itemize}
                \item Groups together dense areas of data points.
                \item Example: Identifying geographic clusters based on event density.
            \end{itemize}
        \end{itemize}
    \end{enumerate}
\end{frame}

\begin{frame}[fragile]
    \frametitle{Summary and Conclusion - Applications and Trends}
    \begin{enumerate}
        \setcounter{enumi}{3}
        \item \textbf{Applications of Clustering}:
        \begin{itemize}
            \item Market segmentation.
            \item Image compression.
            \item Anomaly detection.
        \end{itemize}
        \item \textbf{Evaluation Metrics}:
        \begin{itemize}
            \item Silhouette Score.
            \item Davies-Bouldin Index.
        \end{itemize}
        \item \textbf{Emerging Trends}:
        \begin{itemize}
            \item Fuzzy clustering for partial membership.
            \item Automated clustering techniques.
            \item Solutions for big data challenges.
        \end{itemize}
    \end{enumerate}
\end{frame}

\begin{frame}[fragile]
    \frametitle{Summary and Conclusion - Final Thoughts}
    \begin{block}{Conclusion}
        Clustering represents a fundamental technique in data mining, applicable in various fields. Understanding each method's nuances allows practitioners to tackle real-world datasets and gain valuable insights.
    \end{block}
    \begin{block}{Engagement Tip}
        Encourage students to explore open datasets (e.g., UCI Machine Learning Repository) and apply clustering algorithms to practice and assess their results.
    \end{block}
\end{frame}


\end{document}