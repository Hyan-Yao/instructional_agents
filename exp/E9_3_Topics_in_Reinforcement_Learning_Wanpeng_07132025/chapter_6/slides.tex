\documentclass[aspectratio=169]{beamer}

% Theme and Color Setup
\usetheme{Madrid}
\usecolortheme{whale}
\useinnertheme{rectangles}
\useoutertheme{miniframes}

% Additional Packages
\usepackage[utf8]{inputenc}
\usepackage[T1]{fontenc}
\usepackage{graphicx}
\usepackage{booktabs}
\usepackage{listings}
\usepackage{amsmath}
\usepackage{amssymb}
\usepackage{xcolor}
\usepackage{tikz}
\usepackage{pgfplots}
\pgfplotsset{compat=1.18}
\usetikzlibrary{positioning}
\usepackage{hyperref}

% Custom Colors
\definecolor{myblue}{RGB}{31, 73, 125}
\definecolor{mygray}{RGB}{100, 100, 100}
\definecolor{mygreen}{RGB}{0, 128, 0}
\definecolor{myorange}{RGB}{230, 126, 34}
\definecolor{mycodebackground}{RGB}{245, 245, 245}

% Set Theme Colors
\setbeamercolor{structure}{fg=myblue}
\setbeamercolor{frametitle}{fg=white, bg=myblue}
\setbeamercolor{title}{fg=myblue}
\setbeamercolor{section in toc}{fg=myblue}
\setbeamercolor{item projected}{fg=white, bg=myblue}
\setbeamercolor{block title}{bg=myblue!20, fg=myblue}
\setbeamercolor{block body}{bg=myblue!10}
\setbeamercolor{alerted text}{fg=myorange}

% Set Fonts
\setbeamerfont{title}{size=\Large, series=\bfseries}
\setbeamerfont{frametitle}{size=\large, series=\bfseries}
\setbeamerfont{caption}{size=\small}
\setbeamerfont{footnote}{size=\tiny}

% Custom Commands
\newcommand{\hilight}[1]{\colorbox{myorange!30}{#1}}
\newcommand{\concept}[1]{\textcolor{myblue}{\textbf{#1}}}
\newcommand{\separator}{\begin{center}\rule{0.5\linewidth}{0.5pt}\end{center}}

% Title Page Information
\title[Actor-Critic Algorithms]{Chapter 6: Actor-Critic Algorithms}
\author[J. Smith]{John Smith, Ph.D.}
\institute[University Name]{
  Department of Computer Science\\
  University Name\\
  \vspace{0.3cm}
  Email: email@university.edu\\
  Website: www.university.edu
}
\date{\today}

% Document Start
\begin{document}

\frame{\titlepage}

\begin{frame}[fragile]
    \frametitle{Introduction to Actor-Critic Algorithms}
    \begin{block}{Overview of Actor-Critic Architecture}
        Actor-Critic algorithms in reinforcement learning (RL) combine value-based and policy-based methods, consisting of:
        \begin{itemize}
            \item \textbf{Actor}: Selects actions based on the current policy, mapping states to action probabilities and improving the policy during training.
            \item \textbf{Critic}: Evaluates actions taken by the actor by estimating the value function, producing a numerical value (advantage or value estimate) that reflects the quality of actions.
        \end{itemize}
    \end{block}
\end{frame}

\begin{frame}[fragile]
    \frametitle{Significance of Actor-Critic Algorithms}
    \begin{itemize}
        \item \textbf{Combining Strengths}: Balances exploratory and exploitative strategies, enabling faster convergence and better performance than pure methods.
        \item \textbf{Policy Improvement}: Actor adjusts probabilities using feedback from the critic, leading to continuous improvement of the policy.
        \item \textbf{Stability}: Separating value estimation and action selection creates more stable learning dynamics, beneficial for complex environments.
    \end{itemize}
    
    \begin{block}{Applications}
        Actor-Critic algorithms are utilized in:
        \begin{itemize}
            \item \textbf{Robotics}: Training robots for tasks like manipulation and navigation.
            \item \textbf{Game Playing}: Achieving high performance in games such as Go and various video games.
            \item \textbf{Finance}: Portfolio management and optimizing strategies in dynamic environments.
        \end{itemize}
    \end{block}
\end{frame}

\begin{frame}[fragile]
    \frametitle{Key Concepts in Actor-Critic}
    \begin{itemize}
        \item \textbf{Learning Process}: Actor and critic learn simultaneously and can update at different rates, adapting learning dynamics to specific problems.
        \item \textbf{Exploration vs. Exploitation}: The stochastic nature of the actor encourages exploration, while the critic supports exploitation of learned actions.
    \end{itemize}
    
    \begin{block}{Example}
        In a grid world, the agent moves toward the goal while avoiding obstacles:
        \begin{itemize}
            \item \textbf{Actor}: Chooses among actions (up, down, left, right) based on the current policy.
            \item \textbf{Critic}: Provides a value estimate reflecting the expected reward of reaching the goal.
        \end{itemize}
    \end{block}
\end{frame}

\begin{frame}[fragile]
    \frametitle{Mathematical Foundations}
    \begin{block}{Value Estimation}
        The value of state \(s\) is given by:
        \begin{equation}
            V(s) = E[R_t | s_t = s]
        \end{equation}
        where \(V(s)\) is the value function and \(R_t\) is the reward at time \(t\).
    \end{block}
    
    \begin{block}{Policy Update}
        The policy parameters are updated using:
        \begin{equation}
            \theta_{new} = \theta + \alpha \nabla_\theta J(\theta)
        \end{equation}
        where \(\theta\) are the policy parameters, \(\alpha\) is the learning rate, and \(J(\theta)\) is the performance objective.
    \end{block}
\end{frame}

\begin{frame}[fragile]
    \frametitle{Reinforcement Learning Fundamentals - Overview}
    \begin{itemize}
        \item A review of foundational concepts in reinforcement learning (RL).
        \item Key topics include:
        \begin{itemize}
            \item Agents and environments
            \item States and actions
            \item Rewards and policies
        \end{itemize}
    \end{itemize}
\end{frame}

\begin{frame}[fragile]
    \frametitle{Understanding Key Concepts in Reinforcement Learning}
    \begin{itemize}
        \item \textbf{Agent}:
            \begin{itemize}
                \item Definition: The learner or decision maker.
                \item Role: Choose actions to maximize cumulative rewards.
                \item Example: A robot navigating a maze.
            \end{itemize}

        \item \textbf{Environment}:
            \begin{itemize}
                \item Definition: Everything the agent interacts with.
                \item Role: Responds to the agent's actions.
                \item Example: The maze including walls and paths.
            \end{itemize}
        
        \item \textbf{State (s)}:
            \begin{itemize}
                \item Definition: Description of the current situation.
                \item Role: Influences agent behavior.
                \item Example: The robot's location in the maze.
            \end{itemize}
    \end{itemize}
\end{frame}

\begin{frame}[fragile]
    \frametitle{Continuing Key Concepts in RL}
    \begin{itemize}
        \item \textbf{Action (a)}:
            \begin{itemize}
                \item Definition: A choice made by the agent.
                \item Role: Represents moves the agent can make.
                \item Example: Move forward, turn left/right.
            \end{itemize}

        \item \textbf{Reward (r)}:
            \begin{itemize}
                \item Definition: Feedback signal after an action.
                \item Role: Indicates success or failure, guiding learning.
                \item Example: +1 for reaching exit, -1 for hitting wall.
            \end{itemize}
        
        \item \textbf{Policy ($\pi$)}:
            \begin{itemize}
                \item Definition: A strategy that defines agent behavior.
                \item Role: Maps states to actions.
                \item Example: "If at (3, 5), turn right."
            \end{itemize}
    \end{itemize}
\end{frame}

\begin{frame}[fragile]
    \frametitle{Reinforcement Learning Process}
    \begin{enumerate}
        \item \textbf{Perceive State}: Observe the environment's current state.
        \item \textbf{Select Action}: Choose an action based on policy.
        \item \textbf{Receive Reward}: Environment provides a reward and state transition.
        \item \textbf{Update Policy}: Adjust policy based on received reward.
    \end{enumerate}

    \begin{block}{Cumulative Reward (Return)}
        \begin{equation}
            G_t = r_t + \gamma r_{t+1} + \gamma^2 r_{t+2} + \ldots
        \end{equation}
        Total reward accumulated over time, essential for evaluating performance.
    \end{block}
\end{frame}

\begin{frame}[fragile]
    \frametitle{Key Points and Conclusion}
    \begin{itemize}
        \item \textbf{Feedback Loop}: Agent learns through trial and error, optimizing policy based on experiences.
        \item \textbf{Exploration vs Exploitation}: Balancing new actions vs known rewarding actions.
    \end{itemize}

    \begin{block}{Conclusion}
        Understanding these concepts is crucial for diving into actor-critic methods—where the roles of the agent split into distinct functions.
    \end{block}
\end{frame}

\begin{frame}[fragile]
    \frametitle{The Actor-Critic Framework - Overview}
    \begin{itemize}
        \item The Actor-Critic framework merges value-based and policy-based methods in reinforcement learning.
        \item Key components:
        \begin{itemize}
            \item **Actor**: Learns the policy mapping states to actions.
            \item **Critic**: Evaluates the Actor's actions and estimates the value function.
        \end{itemize}
        \item This interaction optimizes learning and policy performance.
    \end{itemize}
\end{frame}

\begin{frame}[fragile]
    \frametitle{The Actor-Critic Framework - Roles}
    \begin{block}{1. The Actor}
        \begin{itemize}
            \item **Purpose**: Learns the policy to decide actions based on the current state.
            \item **Feedback Mechanism**: Updates its parameters through feedback from the Critic.
            \item **Example**: Navigating a maze by proposing moves that lead to rewards.
        \end{itemize}
    \end{block}

    \begin{block}{2. The Critic}
        \begin{itemize}
            \item **Purpose**: Evaluates the actions chosen by the Actor by computing the value function.
            \item **Feedback Mechanism**: Utilizes the Temporal-Difference error for updates.
            \item **Example**: Assessing the effectiveness of the Actor's moves during maze navigation.
        \end{itemize}
    \end{block}
\end{frame}

\begin{frame}[fragile]
    \frametitle{The Actor-Critic Framework - Interaction and Formulas}
    \begin{itemize}
        \item **Interaction**:
        \begin{itemize}
            \item The Actor adjusts its policy based on the Critic's evaluations.
            \item Forms a feedback loop improving policy and action choices over time.
        \end{itemize}
        
        \item **Key Formulas**:
        \begin{equation}
            \delta_t = r_t + \gamma V(s_{t+1}) - V(s_t)
        \end{equation}
        \begin{equation}
            \theta \leftarrow \theta + \alpha \delta_t \nabla \log \pi(a_t | s_t; \theta)
        \end{equation}
        \begin{itemize}
            \item \(\delta_t\): Temporal-Difference error
            \item \( \theta \): Policy parameters
            \item \( \alpha \): Learning rate
        \end{itemize}
    \end{itemize}
\end{frame}

\begin{frame}[fragile]
    \frametitle{Actor vs. Critic: Key Differences - Function}
    
    \begin{itemize}
        \item \textbf{Actor:}
        \begin{itemize}
            \item \textbf{Definition:} The actor selects actions based on the current policy.
            \item \textbf{Role:} Maps states to actions (policy function).
            \item \textbf{Output:} Produces action probabilities (e.g., \( \pi(a|s) \)).
        \end{itemize}
        
        \item \textbf{Critic:}
        \begin{itemize}
            \item \textbf{Definition:} Evaluates the actions taken by the actor, providing feedback.
            \item \textbf{Role:} Estimates the value function (either \( V(s) \) or \( Q(s,a) \)).
            \item \textbf{Output:} Provides a scalar value (e.g., \( V(s) \)).
        \end{itemize}
    \end{itemize}
\end{frame}

\begin{frame}[fragile]
    \frametitle{Actor vs. Critic: Key Differences - Learning Objectives}
    
    \begin{itemize}
        \item \textbf{Actor:}
        \begin{itemize}
            \item \textbf{Objective:} Optimize the policy directly by improving action selection over time.
            \item \textbf{Goal:} Maximize expected return from each state by adjusting \( \pi \) based on feedback from the critic.
        \end{itemize}
        
        \item \textbf{Critic:}
        \begin{itemize}
            \item \textbf{Objective:} Evaluate the performance of the actor by estimating value of states and actions.
            \item \textbf{Goal:} Minimize the difference between estimated values and observed returns (temporal-difference learning).
        \end{itemize}
    \end{itemize}
\end{frame}

\begin{frame}[fragile]
    \frametitle{Actor vs. Critic: Key Differences - Algorithmic Implementation}
    
    \begin{itemize}
        \item \textbf{Actor:}
        \begin{itemize}
            \item \textbf{Method:} Uses policy gradient methods to adjust policy parameters \( \theta \).
            \item \textbf{Update Rule:}
            \begin{equation}
                \theta \leftarrow \theta + \alpha \nabla \log \pi(a|s; \theta) \cdot (G_t - V(s; w))
            \end{equation}
            \end{itemize}
        
        \item \textbf{Critic:}
        \begin{itemize}
            \item \textbf{Method:} Utilizes value-based techniques, such as Temporal Difference (TD) learning.
            \item \textbf{Update Rule:}
            \begin{equation}
                w \leftarrow w + \beta (G_t - V(s; w)) \nabla V(s; w)
            \end{equation}
        \end{itemize}
        
        \item \textbf{Key Points:}
        \begin{itemize}
            \item Complementary roles create a feedback loop for policy and value refinement.
            \item Separation of policy improvement and value estimation enhances stability and efficiency.
        \end{itemize}
    \end{itemize}
\end{frame}

\begin{frame}[fragile]
    \frametitle{Actor-Critic Architecture}
    \begin{block}{Concept Overview}
        Actor-Critic methods combine two key components in reinforcement learning:
        \begin{itemize}
            \item \textbf{Actor:} Proposes actions based on a policy.
            \item \textbf{Critic:} Evaluates the action taken by assessing its value.
        \end{itemize}
        This dual architecture enables efficient exploration of the action space while stabilizing the learning process.
    \end{block}
\end{frame}

\begin{frame}[fragile]
    \frametitle{Advantages of Actor-Critic Methods - Stability and Sample Efficiency}
    \begin{enumerate}
        \item \textbf{Stability}
            \begin{itemize}
                \item Actor-Critic methods stabilize training by addressing variance in policy evaluation.
                \item \textbf{Example:} In variable environments like video games, the Critic reduces reward signal fluctuations, allowing reliable learning.
            \end{itemize}
        \item \textbf{High Sample Efficiency}
            \begin{itemize}
                \item Fewer interactions with the environment are required to learn effective policies.
                \item \textbf{Example:} In robotic tasks, Actor-Critic methods achieve optimal movements using fewer trials than traditional methods.
            \end{itemize}
    \end{enumerate}
\end{frame}

\begin{frame}[fragile]
    \frametitle{Advantages of Actor-Critic Methods - Direct Optimization and Key Points}
    \begin{enumerate}
        \setcounter{enumi}{2}
        \item \textbf{Directly Optimizing Policy}
            \begin{itemize}
                \item These methods update policy parameters directly, leading to smoother updates.
                \item \textbf{Example:} In navigation tasks, direct action evaluation by the Critic facilitates optimal decision-making.
            \end{itemize}
    \end{enumerate}
    
    \begin{block}{Key Points to Emphasize}
        \begin{itemize}
            \item Efficient blending of policy gradient and value function methods enhances performance.
            \item Robust against fluctuations in less stable reward environments.
            \item Adaptable architectures using modern enhancements like DDPG and A3C for complex problems.
        \end{itemize}
    \end{block}
\end{frame}

\begin{frame}[fragile]
    \frametitle{Formulas and Code Snippets}
    \begin{block}{Policy Gradient Update}
        The policy for the Actor can be updated using the gradient of the expected return:
        \begin{equation}
            \nabla J(\theta) \approx \mathbb{E}_t \left[ \nabla \log \pi_{\theta}(a_t | s_t) A_t \right]
        \end{equation}
        Where:
        \begin{itemize}
            \item $J(\theta)$ is the objective function.
            \item $\pi_{\theta}(a_t | s_t)$ is the policy.
            \item $A_t$ is the advantage function.
        \end{itemize}
    \end{block}

    \begin{block}{Value Update for Critic}
        The Critic refines its value estimate through temporal difference learning:
        \begin{equation}
            V(s_t) \gets V(s_t) + \alpha (R_t + \gamma V(s_{t+1}) - V(s_t))
        \end{equation}
        Where:
        \begin{itemize}
            \item $R_t$ is the immediate reward.
            \item $\alpha$ is the learning rate.
            \item $\gamma$ is the discount factor.
        \end{itemize}
    \end{block}
\end{frame}

\begin{frame}[fragile]
    \frametitle{Types of Actor-Critic Algorithms}
    Actor-Critic algorithms combine the benefits of policy-based and value-based reinforcement learning. They consist of two main components:
    \begin{itemize}
        \item \textbf{Actor}: Updates the policy based on feedback from the critic.
        \item \textbf{Critic}: Evaluates the action taken by the actor and provides feedback for improvement.
    \end{itemize}
\end{frame}

\begin{frame}[fragile]
    \frametitle{Key Variants of Actor-Critic Algorithms}
    \begin{enumerate}
        \item \textbf{A3C (Asynchronous Actor-Critic Agents)}
        \begin{itemize}
            \item \textit{Overview}: Uses multiple agents exploring in parallel and updates a shared model.
            \item \textit{Key Points}:
            \begin{itemize}
                \item Asynchronous Updates: Faster and more stable convergence.
                \item Entropy Regularization: Encourages exploration.
            \end{itemize}
            \item \textit{Example Use Case}: Video games with simultaneous agent learning.
        \end{itemize}

        \item \textbf{DDPG (Deep Deterministic Policy Gradient)}
        \begin{itemize}
            \item \textit{Overview}: Designed for continuous action spaces with off-policy learning.
            \item \textit{Key Points}:
            \begin{itemize}
                \item Actor-Critic Architecture: Actor selects actions, critic evaluates them.
                \item Replay Buffer: Stores transitions to improve learning efficiency.
            \end{itemize}
            \item \textit{Example Use Case}: Robotics control tasks requiring precise actions.
        \end{itemize}
    \end{enumerate}
\end{frame}

\begin{frame}[fragile]
    \frametitle{More Actor-Critic Variants}
    \begin{enumerate}
        \setcounter{enumi}{2}
        \item \textbf{PPO (Proximal Policy Optimization)}
        \begin{itemize}
            \item \textit{Overview}: Balances performance with implementation simplicity.
            \item \textit{Key Points}:
            \begin{itemize}
                \item Clipped Objective Function: Prevents destabilizing updates.
                \item Trust Region Updates: Ensures stable learning by limiting policy deviation.
            \end{itemize}
            \item \textit{Example Use Case}: Robust training in continuous or discrete environments.
        \end{itemize}

        \item \textbf{TRPO (Trust Region Policy Optimization)}
        \begin{itemize}
            \item \textit{Overview}: Optimizes policy with constraints for stable updates.
            \item \textit{Key Points}:
            \begin{itemize}
                \item Constrained Optimization: Balances exploration and exploitation.
                \item Natural Gradients: Enhances stability in updates.
            \end{itemize}
            \item \textit{Example Use Case}: Complex continuous control tasks like drone flight.
        \end{itemize}
    \end{enumerate}
\end{frame}

\begin{frame}[fragile]
    \frametitle{Key Considerations}
    \begin{itemize}
        \item \textbf{Sample Efficiency}: Higher sample efficiency compared to pure actor or critic methods.
        \item \textbf{Stability}: Combined actor and critic structure allows for stable convergence.
        \item \textbf{Exploration vs. Exploitation}: Critical for success; balancing new and known actions is essential.
    \end{itemize}
    
    \textbf{Conclusion:} Each variant of Actor-Critic algorithms has unique advantages tailored for diverse reinforcement learning problems.
\end{frame}

\begin{frame}[fragile]
    \frametitle{Implementation Steps for Actor-Critic Algorithms}
    \begin{block}{Overview}
        Actor-Critic algorithms combine policy-based and value-based methods. The \textbf{actor} updates the policy, guided by the \textbf{critic}, which evaluates actions and provides value feedback.
    \end{block}
\end{frame}

\begin{frame}[fragile]
    \frametitle{Implementation Steps - Initialization}
    \begin{enumerate}
        \item \textbf{Define the Environment}
            \begin{itemize}
                \item Specify states and actions (e.g., grid locations and movements).
            \end{itemize}
        \item \textbf{Initialize Actor and Critic Networks}
            \begin{itemize}
                \item Choose neural network architectures and initialize weights.
            \end{itemize}
        \item \textbf{Set Hyperparameters}
            \begin{itemize}
                \item Learning rates for actor and critic.
                \item Discount factor ($\gamma$).
                \item Exploration strategy (e.g., $\epsilon$-greedy).
            \end{itemize}
        \item \textbf{Example}
            \begin{itemize}
                \item For DDPG: Set up a deep neural network with separate heads for actor and critic.
            \end{itemize}
    \end{enumerate}
\end{frame}

\begin{frame}[fragile]
    \frametitle{Implementation Steps - Training Updates}
    \begin{enumerate}
        \setcounter{enumi}{1}
        \item \textbf{Collect Experience}
            \begin{itemize}
                \item Generate actions and store states, actions, rewards, and next states.
            \end{itemize}
        \item \textbf{Compute Returns}
            \begin{equation}
            R_t = r_t + \gamma r_{t+1} + \gamma^2 r_{t+2} + \ldots
            \end{equation}
        \item \textbf{Critic Update}
            \begin{equation}
            L(\theta^{critic}) = \frac{1}{N} \sum_{t=0}^{N} (R_t - V(s_t; \theta^{critic}))^2
            \end{equation}
        \item \textbf{Actor Update}
            \begin{equation}
            \nabla J(\theta^{actor}) \approx \frac{1}{N} \sum_{t=0}^{N} \nabla \log \pi(a_t | s_t; \theta^{actor}) (R_t - V(s_t; \theta^{critic}))
            \end{equation}
        \item \textbf{Example}
            \begin{itemize}
                \item Use the critic's feedback to refine the actor's policy based on past actions.
            \end{itemize}
    \end{enumerate}
\end{frame}

\begin{frame}[fragile]
    \frametitle{Implementation Steps - Performance Evaluation}
    \begin{enumerate}
        \setcounter{enumi}{2}
        \item \textbf{Testing the Policy}
            \begin{itemize}
                \item Evaluate learning performance using total rewards over episodes.
            \end{itemize}
        \item \textbf{Monitor Learning Metrics}
            \begin{itemize}
                \item Plot average rewards, successful episodes, and episodic returns.
            \end{itemize}
        \item \textbf{Fine-Tuning}
            \begin{itemize}
                \item Tune hyperparameters and architectures if performance plateaus.
            \end{itemize}
        \item \textbf{Key Points to Monitor}
            \begin{itemize}
                \item Convergence speed and stability of updates.
                \item Variance of returns.
                \item Risks of overfitting or underfitting.
            \end{itemize}
    \end{enumerate}
\end{frame}

\begin{frame}[fragile]
    \frametitle{Summary}
    Implementing an Actor-Critic algorithm involves:
    \begin{itemize}
        \item Initializing models and parameters.
        \item Collecting experiences and optimizing through updates.
        \item Continuously evaluating agent performance for effective learning.
    \end{itemize}
\end{frame}

\begin{frame}[fragile]
    \frametitle{Additional Notes}
    \begin{itemize}
        \item Use frameworks like TensorFlow or PyTorch for efficient implementation.
        \item Balance exploration and exploitation for effective learning.
    \end{itemize}
\end{frame}

\begin{frame}[fragile]
    \frametitle{Case Study: Actor-Critic in Practice}
    \begin{block}{Overview}
        Actor-Critic algorithms integrate strengths of policy-based and value-based methods in reinforcement learning.
        We explore how an actor-critic algorithm optimized the energy consumption of heating systems in smart buildings.
    \end{block}
\end{frame}

\begin{frame}[fragile]
    \frametitle{Actor-Critic Architecture}
    \begin{itemize}
        \item \textbf{Actor}: Learns and updates the policy function, deciding actions based on the current state.
        \item \textbf{Critic}: Evaluates the action taken by the actor with a value function, providing feedback to enhance policy.
    \end{itemize}
\end{frame}

\begin{frame}[fragile]
    \frametitle{Heating System Optimization Case Study}
    \begin{block}{Context}
        Smart buildings require efficient energy management for cost reduction and environmental sustainability. 
        An actor-critic algorithm was implemented to optimize heating in a commercial building.
    \end{block}
    
    \begin{block}{Implementation Steps}
        \begin{enumerate}
            \item \textbf{Initialization}
                \begin{itemize}
                    \item Define state representation: temperature, outside conditions, occupancy.
                    \item Initialize actor (policy) and critic (value) networks.
                \end{itemize}
            \item \textbf{Training}
                \begin{itemize}
                    \item State: constructed using sensory data.
                    \item Action Selection: actor uses neural network to choose actions (e.g., adjust heating).
                    \item Feedback: critic evaluates action based on fluctuations and consumption using:
                    \begin{equation}
                        \delta_t = r_t + \gamma V(s_{t+1}) - V(s_t)
                    \end{equation}
                \end{itemize}
            \item \textbf{Policy Update}
                \begin{itemize}
                    \item Actor: adjusts policy parameters with critic feedback.
                    \item Critic: refines value function estimates.
                \end{itemize}
        \end{enumerate}
    \end{block}
\end{frame}

\begin{frame}[fragile]
    \frametitle{Results and Key Points}
    \begin{block}{Results}
        \begin{itemize}
            \item \textbf{Improved Energy Efficiency}: Reduced consumption by up to 20\%.
            \item \textbf{User Comfort}: Maintained optimal temperatures for occupant satisfaction.
        \end{itemize}
    \end{block}
    
    \begin{block}{Key Points}
        \begin{itemize}
            \item \textbf{Adaptability}: Actor-Critic model adjusts to occupancy and weather changes.
            \item \textbf{Real-time Learning}: Demonstrated ability to learn and refine actions continuously.
        \end{itemize}
    \end{block}
    
    \begin{block}{Conclusion}
        Actor-Critic algorithms show promise in practical applications, enhancing energy efficiency in smart buildings for sustainable urban development.
    \end{block}
\end{frame}

\begin{frame}[fragile]
    \frametitle{Challenges and Limitations of Actor-Critic Algorithms - Introduction}
    
    \begin{itemize}
        \item Actor-Critic algorithms blend value-based and policy-based reinforcement learning methods.
        \item Despite their successes, they face several challenges.
        \item This slide will explore the key issues impacting their effectiveness.
    \end{itemize}
    
\end{frame}

\begin{frame}[fragile]
    \frametitle{Key Challenges and Limitations - Part 1}

    \begin{enumerate}
        \item \textbf{Stability and Convergence Issues}
        \begin{itemize}
            \item Divergence between Actor and Critic hinders stable learning.
            \item Can slow down learning or prevent convergence to an optimal policy.
            \item \textit{Example:} Overestimation by Critic leads to erratic policy updates.
        \end{itemize}

        \item \textbf{Variance in Gradient Estimates}
        \begin{itemize}
            \item High variance in policy gradient estimates affects learning consistency.
            \item \textit{Example:} Noisy value estimations from the Critic yield unpredictable actions.
        \end{itemize}
    \end{enumerate}
    
\end{frame}

\begin{frame}[fragile]
    \frametitle{Key Challenges and Limitations - Part 2}

    \begin{enumerate}
        \setcounter{enumiii}{2} % Resume the enumeration from the previous frame
        \item \textbf{Complexity of Hyperparameter Tuning}
        \begin{itemize}
            \item Requires careful tuning of learning rates for Actor and Critic.
            \item Suboptimal hyperparameters can degrade performance.
            \item \textit{Example:} High learning rates may overshoot optimal policies; low rates can delay convergence.
        \end{itemize}

        \item \textbf{Sample Efficiency}
        \begin{itemize}
            \item Many methods require substantial data for effective learning.
            \item \textit{Example:} In costly interaction scenarios (e.g., robotics), prolonged training is impractical.
        \end{itemize}
        
        \item \textbf{Function Approximation Limitations}
        \begin{itemize}
            \item Use of function approximators can lead to generalization issues.
            \item Poor generalization may cause overfitting to specific states.
            \item \textit{Example:} Insufficient exploration of the action space leads to missing optimal long-term strategies.
        \end{itemize}
    \end{enumerate}

\end{frame}

\begin{frame}[fragile]
    \frametitle{Future Directions in Actor-Critic Research - Overview}
    \begin{block}{Overview}
        Actor-Critic algorithms are pivotal in reinforcement learning (RL) for their unique ability to merge value-based and policy-based strategies. As research progresses, future directions focus on enhancing algorithm performance and applicability.
    \end{block}
\end{frame}

\begin{frame}[fragile]
    \frametitle{Future Directions in Actor-Critic Research - Key Trends}
    \begin{enumerate}
        \item \textbf{Integration with Deep Learning}
            \begin{itemize}
                \item Utilization of deep neural networks enhances scalability and flexibility.
                \item Example: Recurrent neural networks (RNNs) are integrated to manage sequential data effectively.
            \end{itemize}
        \item \textbf{Multi-Agent Architectures}
            \begin{itemize}
                \item Focus on multiple actors learning simultaneously in competitive or cooperative settings.
                \item Example: AI agents in gaming environments learn strategies against each other.
            \end{itemize}
    \end{enumerate}
\end{frame}

\begin{frame}[fragile]
    \frametitle{Future Directions in Actor-Critic Research - Continued Trends}
    \begin{enumerate}
        \setcounter{enumi}{2} % Start enumerating from 3
        \item \textbf{Sample Efficiency and Offline RL}
            \begin{itemize}
                \item Aim to improve learning from limited data using offline datasets.
                \item Example: Dataset Aggregation (D4PG) utilizes offline experiences to enhance learning efficiency.
            \end{itemize}
        \item \textbf{Exploration Strategies}
            \begin{itemize}
                \item Developing adaptive techniques for improved exploration, crucial for enhancing performance.
                \item Example: Actors incentivized to explore uncertain states discover rewarding regions effectively.
            \end{itemize}
    \end{enumerate}
\end{frame}

\begin{frame}[fragile]
    \frametitle{Future Directions in Actor-Critic Research - Final Trends and Conclusion}
    \begin{enumerate}
        \setcounter{enumi}{4} % Continue from where the last frame ended
        \item \textbf{Hierarchical RL}
            \begin{itemize}
                \item Decomposes tasks into subtasks with distinct actor-critic pairs, enhancing learning efficiency.
                \item Example: A robot learns to navigate and then pick up objects as separate episodes with dedicated models.
            \end{itemize}
        \item \textbf{Improving Stability and Convergence}
            \begin{itemize}
                \item Ongoing methods aim to enhance stability and convergence rates of algorithms.
                \item Example: Techniques like Trust Region Policy Optimization (TRPO) modify updates for stable learning.
            \end{itemize}
    \end{enumerate}
    
    \begin{block}{Conclusion}
        The future of actor-critic research is filled with promise, driven by AI advancements and methodological innovations that will bolster robustness and efficiency in reinforcement learning systems.
    \end{block}
\end{frame}


\end{document}