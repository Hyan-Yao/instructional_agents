\documentclass{beamer}

% Theme choice
\usetheme{Madrid} % You can change to e.g., Warsaw, Berlin, CambridgeUS, etc.

% Encoding and font
\usepackage[utf8]{inputenc}
\usepackage[T1]{fontenc}

% Graphics and tables
\usepackage{graphicx}
\usepackage{booktabs}

% Code listings
\usepackage{listings}
\lstset{
basicstyle=\ttfamily\small,
keywordstyle=\color{blue},
commentstyle=\color{gray},
stringstyle=\color{red},
breaklines=true,
frame=single
}

% Math packages
\usepackage{amsmath}
\usepackage{amssymb}

% Colors
\usepackage{xcolor}

% TikZ and PGFPlots
\usepackage{tikz}
\usepackage{pgfplots}
\pgfplotsset{compat=1.18}
\usetikzlibrary{positioning}

% Hyperlinks
\usepackage{hyperref}

% Title information
\title{Chapter 6: Dimensionality Reduction Techniques}
\author{Your Name}
\institute{Your Institution}
\date{\today}

\begin{document}

\frame{\titlepage}

\begin{frame}[fragile]
    \frametitle{Introduction to Dimensionality Reduction}
    \begin{block}{Overview}
        Dimensionality Reduction (DR) is the process of reducing the number of random variables or features under consideration in a dataset. It can be achieved by selecting a subset of the original variables or by transforming them into a lower-dimensional space.
    \end{block}
\end{frame}

\begin{frame}[fragile]
    \frametitle{Significance in Data Visualization and Analysis}
    \begin{itemize}
        \item \textbf{Data Complexity:} High-dimensional data can be challenging to visualize and interpret, making it difficult to draw meaningful insights.
        \item \textbf{Computational Efficiency:} Reducing dimensions leads to lower computational costs for algorithms, speeding up processes such as training and prediction.
        \item \textbf{Overfitting Prevention:} Helps adopt a more robust model by removing irrelevant features and reducing noise in the data.
    \end{itemize}
\end{frame}

\begin{frame}[fragile]
    \frametitle{Key Concepts and Techniques}
    \begin{itemize}
        \item \textbf{Curse of Dimensionality:} 
        \begin{itemize}
            \item As dimensions increase, the volume of the space increases, leading to data sparsity, which is problematic for statistical and machine learning methods.
        \end{itemize}
        
        \item \textbf{Common Techniques of Dimensionality Reduction:}
        \begin{itemize}
            \item \textbf{Principal Component Analysis (PCA):} Identifies directions (principal components) that maximize variance in the data.
            \item \textbf{t-Distributed Stochastic Neighbor Embedding (t-SNE):} Useful for visualizing high-dimensional datasets in 2D or 3D by preserving local structures.
            \item \textbf{Linear Discriminant Analysis (LDA):} Finds a linear combination of features that best separates multiple classes.
        \end{itemize}
    \end{itemize}
\end{frame}

\begin{frame}[fragile]
    \frametitle{Example of Dimensionality Reduction with PCA}
    \begin{enumerate}
        \item \textbf{Original Dataset:} A dataset with features: height, weight, age, and income (4 dimensions).
        \item \textbf{Applying PCA:} Transforms these features into 2 new features (principal components) that capture most variance (e.g., PC1 = 70\% variance, PC2 = 20\% variance).
        \item \textbf{Visualization:} The transformed data can be visualized in a 2D plot, making it easier to identify patterns or clusters.
    \end{enumerate}
\end{frame}

\begin{frame}[fragile]
    \frametitle{Key Points to Emphasize}
    \begin{itemize}
        \item Dimensionality Reduction is crucial for better data visualization and analysis.
        \item Addresses challenges associated with high-dimensional data, such as computational inefficiency and overfitting.
        \item Familiarity with tools and techniques like PCA, t-SNE, and LDA is beneficial for practical applications.
    \end{itemize}
\end{frame}

\begin{frame}[fragile]
    \frametitle{Conclusion}
    Dimensionality Reduction simplifies data analysis and is essential for making sense of high-dimensional datasets, ultimately leading to more effective data-driven decisions.
\end{frame}

\begin{frame}[fragile]
    \frametitle{Next Steps}
    \begin{block}{}
        Prepare to explore the need for Dimensionality Reduction in the next slide, focusing on the challenges imposed by high-dimensional data.
    \end{block}
\end{frame}

\begin{frame}[fragile]
    \frametitle{Need for Dimensionality Reduction}
    % Overview of the need for dimensionality reduction 
    % and its significance in high-dimensional data.
    High-dimensional data presents challenges in analysis and interpretation. Key issues include:
    \begin{itemize}
        \item Curse of Dimensionality
        \item Overfitting
    \end{itemize}
\end{frame}

\begin{frame}[fragile]
    \frametitle{Introduction to High-Dimensional Data}
    % Explanation of high-dimensional data
    High-dimensional data refers to datasets with:
    \begin{itemize}
        \item A vast number of features (or dimensions)
        \item Fewer observations compared to features
    \end{itemize}
    This is common in fields such as:
    \begin{itemize}
        \item Genomics
        \item Image Processing
        \item Finance
    \end{itemize}
\end{frame}

\begin{frame}[fragile]
    \frametitle{Challenges Posed by High-Dimensional Data}
    % Detailed discussion of the curse of dimensionality and overfitting.
    \begin{block}{1. Curse of Dimensionality}
        \begin{itemize}
            \item \textbf{Definition:} Phenomena arising in high-dimensional spaces making data sparse.
            \item \textbf{Impact on Analysis:}
            \begin{itemize}
                \item Increased Complexity in algorithms
                \item Distance metrics lose meaning
            \end{itemize}
            \item \textbf{Example:} As dimensions increase, visualizing clustering becomes infeasible.
        \end{itemize}
    \end{block}
\end{frame}

\begin{frame}[fragile]
    \frametitle{Challenges Continued: Overfitting}
    % Explain the concept of overfitting in high-dimensional data.
    \begin{block}{2. Overfitting}
        \begin{itemize}
            \item \textbf{Definition:} When a model learns noise along with patterns, typically in complex models.
            \item \textbf{Impact on Performance:}
            \begin{itemize}
                \item Poor generalization to unseen data
            \end{itemize}
            \item \textbf{Illustration:} High-degree polynomial fits perfectly to few points but misses the general trend.
        \end{itemize}
    \end{block}
\end{frame}

\begin{frame}[fragile]
    \frametitle{Key Points and Conclusion}
    % Summarizing the key points and conclusion.
    \begin{itemize}
        \item Dimensionality reduction techniques help address:
        \begin{itemize}
            \item The curse of dimensionality
            \item Overfitting
        \end{itemize}
        \item Benefits include:
        \begin{itemize}
            \item Simplified models
            \item Easier visualization
            \item Improved performance and interpretability
        \end{itemize}
        \item Understanding these challenges leads to better modeling choices.
    \end{itemize}
\end{frame}

\begin{frame}[fragile]
    \frametitle{Formula Reference}
    % Reference for metrics related to dimensionality reduction.
    Common metrics to consider:
    \begin{equation}
        d(x, y) = \sqrt{\sum (x_i - y_i)^2}
    \end{equation}
    where \(x\) and \(y\) are points in n-dimensional space.
\end{frame}

\begin{frame}[fragile]
    \frametitle{Code Snippet for Dimensionality Reduction}
    % A simple example code for PCA in Python.
    \begin{lstlisting}[language=Python]
from sklearn.decomposition import PCA
import pandas as pd

# Assuming df is your DataFrame containing high-dimensional data
pca = PCA(n_components=2)  # Reducing to 2 dimensions
reduced_data = pca.fit_transform(df)
    \end{lstlisting}
\end{frame}

\begin{frame}[fragile]
    \frametitle{Principal Component Analysis (PCA) - Introduction}
    \begin{block}{What is PCA?}
        Principal Component Analysis (PCA) is a statistical technique used for:
        \begin{itemize}
            \item Dimensionality reduction while preserving variance.
            \item Transforming data into a new coordinate system.
        \end{itemize}
    \end{block}
    PCA identifies the directions (principal components) that maximize variance:
    \begin{itemize}
        \item First principal component: greatest variance.
        \item Second principal component: second greatest variance.
    \end{itemize}
\end{frame}

\begin{frame}[fragile]
    \frametitle{Principal Component Analysis (PCA) - Purpose}
    The main purposes of PCA include:
    \begin{itemize}
        \item \textbf{Reducing Dimensionality}: Simplifies datasets while retaining essential information.
        \item \textbf{Data Visualization}: Allows visualization of high-dimensional data in 2D or 3D plots.
        \item \textbf{Noise Reduction}: Eliminates less important features contributing mainly to noise.
    \end{itemize}
\end{frame}

\begin{frame}[fragile]
    \frametitle{Principal Component Analysis (PCA) - Mathematics}
    \begin{enumerate}
        \item \textbf{Standardize the Data}:
        \[
        Z = \frac{X - \mu}{\sigma}
        \]

        \item \textbf{Covariance Matrix}:
        \[
        Cov(X) = \frac{1}{n-1} (Z^T Z)
        \]

        \item \textbf{Eigenvalues and Eigenvectors}:
        \[
        Cov(X) \cdot v = \lambda v
        \]

        \item \textbf{Choosing Principal Components}: Sort eigenvalues to identify significant eigenvectors.

        \item \textbf{Projection}:
        \[
        Y = X \cdot V_k
        \]
    \end{enumerate}
\end{frame}

\begin{frame}[fragile]
    \frametitle{Principal Component Analysis (PCA) - Example and Conclusion}
    \textbf{Example:}
    Consider a dataset with features like height, weight, and age. By applying PCA:
    \begin{itemize}
        \item The first principal component may represent a combination of height and weight.
        \item Age may contribute less to overall variance.
        \item Dataset can be effectively reduced from three dimensions to two.
    \end{itemize}

    \textbf{Conclusion:} PCA simplifies data, enhances visualization, and improves machine learning performance while minimizing complexity.
\end{frame}

\begin{frame}[fragile]
    \frametitle{PCA - Steps Involved - Overview}
    \begin{block}{Overview of PCA}
        Principal Component Analysis (PCA) is a powerful statistical tool for:
        \begin{itemize}
            \item Reducing the dimensionality of data
            \item Preserving as much variability as possible
            \item Facilitating visualization of high-dimensional data
            \item Improving the efficiency of predictive models
        \end{itemize}
    \end{block}
\end{frame}

\begin{frame}[fragile]
    \frametitle{PCA - Steps Involved - Standardization and Covariance Matrix}
    \begin{block}{1. Standardization}
        \begin{itemize}
            \item \textbf{Definition}: Scaling the dataset for equal feature contribution.
            \item \textbf{Process}:
            \begin{equation}
            z_i = \frac{x_i - \mu}{\sigma}
            \end{equation}
            where:
            \begin{itemize}
                \item \(z_i\) = standardized value
                \item \(x_i\) = original value
                \item \(\mu\) = mean of the feature
                \item \(\sigma\) = standard deviation of the feature
            \end{itemize}
            \item \textbf{Example}: Standardizing height (cm) and weight (kg) to mean 0 and variance 1.
        \end{itemize}
    \end{block}

    \begin{block}{2. Covariance Matrix Computation}
        \begin{itemize}
            \item \textbf{Definition}: Captures how dimensions vary with respect to each other.
            \item \textbf{Process}:
            \begin{equation}
            C = \frac{1}{n-1} (X^T X)
            \end{equation}
            where \(X\) is the standardized data matrix.
            \item \textbf{Key Point}: High covariance indicates strong relationships between features.
        \end{itemize}
    \end{block}
\end{frame}

\begin{frame}[fragile]
    \frametitle{PCA - Steps Involved - Eigenvalue Decomposition and Projection}
    \begin{block}{3. Eigenvalue Decomposition}
        \begin{itemize}
            \item \textbf{Purpose}: Decomposes the covariance matrix to find principal components.
            \item \textbf{Definitions}:
            \begin{itemize}
                \item **Eigenvalues** (\(\lambda\)): Amount of variance explained by each component.
                \item **Eigenvectors**: Directions of principal components in feature space.
            \end{itemize}
            \item \textbf{Process}:
            \begin{equation}
            C \mathbf{v} = \lambda \mathbf{v}
            \end{equation}
            \item \textbf{Key Insight}: Eigenvectors with largest eigenvalues account for most variance.
        \end{itemize}
    \end{block}

    \begin{block}{4. Projection}
        \begin{itemize}
            \item \textbf{Definition}: Projects data into lower-dimensional space using principal components.
            \item \textbf{Process}:
            \begin{equation}
            Y = XW
            \end{equation}
            where:
            \begin{itemize}
                \item \(Y\) = projected data
                \item \(X\) = standardized data
                \item \(W\) = matrix of selected eigenvectors
            \end{itemize}
            \item \textbf{Example}: Selecting top 2 components captures significant variance in high-dimensional data.
        \end{itemize}
    \end{block}
\end{frame}

\begin{frame}[fragile]
    \frametitle{PCA Visualization}
    Principal Component Analysis (PCA) is a powerful technique used for dimensionality reduction. It simplifies data while preserving variance and helps visualize high-dimensional data. 

    \begin{itemize}
        \item **Data Transformation**: PCA identifies principal components along which data varies the most.
        \item **Visualization**: Significant components can be plotted in 2D or 3D.
    \end{itemize}
\end{frame}

\begin{frame}[fragile]
    \frametitle{Steps Involved in PCA Visualization}
    \begin{enumerate}
        \item **Standardization**: Standardize the data.
            \begin{equation}
            Z_i = \frac{X_i - \mu}{\sigma}
            \end{equation}
            Where:
            \begin{itemize}
                \item \(X_i\) is the original data point,
                \item \(\mu\) is the mean,
                \item \(\sigma\) is the standard deviation.
            \end{itemize}
        
        \item **Covariance Matrix Computation**: Represents how variables co-vary.
        \item **Eigenvalue Decomposition**: Computes eigenvalues and eigenvectors.
        \item **Projection onto Principal Components**: Projects original data into lower-dimensional space.
    \end{enumerate}
\end{frame}

\begin{frame}[fragile]
    \frametitle{Example and Key Points}
    \textbf{Example: Visualizing a 2D Dataset Using PCA}
    \begin{itemize}
        \item Dataset: Height and weight of individuals.
        \item Steps: Standardize, compute covariance, extract eigenvectors, project data onto PC1 and PC2.
    \end{itemize}
    
    \textbf{Key Points to Emphasize}
    \begin{itemize}
        \item **Dimensionality Reduction**: Maintain variance while reducing dimensions.
        \item **Variance Explained**: First component captures most variance.
        \item **Data Visualization**: Essential for visualizing high-dimensional datasets.
    \end{itemize}

    \textbf{Conclusion}
    PCA facilitates exploring and visualizing complex datasets. It simplifies analysis and reveals underlying data structures.
\end{frame}

\begin{frame}[fragile]
    \frametitle{t-Distributed Stochastic Neighbor Embedding (t-SNE) - Introduction}
    \begin{itemize}
        \item \textbf{What is t-SNE?}
        \begin{itemize}
            \item t-Distributed Stochastic Neighbor Embedding (t-SNE) is a machine learning technique used for dimensionality reduction, primarily for the visualization of high-dimensional data in a two or three-dimensional space.
            \item It is effective for visualizing data points while preserving their local similarities.
        \end{itemize}
    \end{itemize}
\end{frame}

\begin{frame}[fragile]
    \frametitle{t-SNE - Purpose and Key Differences}
    \begin{itemize}
        \item \textbf{Why Use t-SNE?}
        \begin{itemize}
            \item Captures complex relationships in data, especially with non-linear structures.
            \item Widely used in fields like biology, image processing, and natural language processing.
        \end{itemize}
        
        \item \textbf{Key Differences Between t-SNE and PCA:}
        \begin{enumerate}
            \item \textbf{Handling Non-Linearity:}
            \begin{itemize}
                \item PCA focuses on linear relationships and may miss intricate manifold structures.
                \item t-SNE excels at revealing non-linear patterns by focusing on local data structure.
            \end{itemize}
            \item \textbf{Distance Metrics:}
            \begin{itemize}
                \item PCA uses Euclidean distance which may not represent clusters effectively.
                \item t-SNE uses probabilities with a Student's t-distribution to better maintain local structures.
            \end{itemize}
        \end{enumerate}
    \end{itemize}
\end{frame}

\begin{frame}[fragile]
    \frametitle{t-SNE - Example and Key Points}
    \begin{itemize}
        \item \textbf{Example: Visualizing Handwritten Digits}
        \begin{itemize}
            \item PCA can blend different digit classes, while t-SNE maintains separation and shows clusters of similar digits effectively.
        \end{itemize}
        
        \item \textbf{Key Points to Emphasize:}
        \begin{itemize}
            \item \textbf{Local vs. Global Structure:} t-SNE retains local structure; PCA retains global structure.
            \item \textbf{Parameter Sensitivity:} Hyperparameters like perplexity greatly influence results.
            \item \textbf{Scalability:} t-SNE can be computationally intensive and may not scale well to large datasets.
        \end{itemize}
        
        \item \textbf{Conclusion:} t-SNE is powerful for visualizing high-dimensional data with non-linear relationships, contrasting with linear techniques like PCA.
    \end{itemize}
\end{frame}

\begin{frame}[fragile]
    \frametitle{t-SNE - Probability Distribution}
    \begin{block}{Formula: Probability Distribution in t-SNE}
        The similarity between points \( x_i \) and \( x_j \) is modeled as follows:
        \begin{equation}
            P_{j|i} = \frac{exp(-||x_i - x_j||^2/2\sigma_i^2)}{\sum_{k \neq i} exp(-||x_i - x_k||^2/2\sigma_i^2)}
        \end{equation}
        Here, \( \sigma_i \) is the bandwidth of the Gaussian distribution centered at \( x_i \).
    \end{block}
\end{frame}

\begin{frame}[fragile]
    \frametitle{t-SNE Algorithm Steps - Overview}
    t-Distributed Stochastic Neighbor Embedding (t-SNE) is a powerful method for visualizing high-dimensional data by reducing it to two or three dimensions while preserving the structure of local neighborhoods. 
\end{frame}

\begin{frame}[fragile]
    \frametitle{t-SNE Algorithm Steps - Key Steps}
    \begin{enumerate}
        \item \textbf{Pairwise Similarity Calculations}:
            \begin{itemize}
                \item Calculates pairwise similarity of points in high-dimensional space.
                \item For each data point \( x_i \), the similarity with another point \( x_j \) is computed using a Gaussian distribution:
                \begin{equation}
                    p_{j|i} = \frac{\exp\left(-\frac{\lVert x_i - x_j \rVert^2}{2\sigma_i^2}\right)}{\sum_{k \neq i} \exp\left(-\frac{\lVert x_i - x_k \rVert^2}{2\sigma_i^2}\right)}
                \end{equation}
            \end{itemize}

        \item \textbf{Probability Distribution}:
            \begin{itemize}
                \item Transforms similarity scores into probabilities using symmetric normalization:
                \begin{equation}
                    P_{ij} = \frac{p_{j|i} + p_{i|j}}{2N}
                \end{equation}
            \end{itemize}

        \item \textbf{Cost Function Optimization}:
            \begin{itemize}
                \item Minimize divergence between high-dimensional distribution \( P \) and low-dimensional distribution \( Q \):
                \begin{equation}
                    C = KL(P || Q) = \sum_{i \neq j} P_{ij} \log\left(\frac{P_{ij}}{Q_{ij}}\right)
                \end{equation}
            \end{itemize}
    \end{enumerate}
\end{frame}

\begin{frame}[fragile]
    \frametitle{t-SNE Algorithm Steps - Key Points}
    \begin{itemize}
        \item \textbf{Non-Linearity}: t-SNE preserves local structure and captures complex, non-linear relationships, unlike PCA.
        \item \textbf{High Dimensionality}: Particularly effective for revealing clusters and patterns in high-dimensional datasets.
        \item \textbf{Computationally Intensive}: May be slow on large datasets—requires techniques for efficiency.
    \end{itemize}
\end{frame}

\begin{frame}[fragile]
    \frametitle{t-SNE Algorithm Steps - Example Illustration}
    Imagine having a dataset of handwritten digits represented as high-dimensional vectors. t-SNE can transform these into a 2D representation, allowing exploration of how similar digits cluster together.
    \begin{block}{Important Note}
        \textit{Understand the trade-offs between computational efficiency and the level of detail retained in visualizations using t-SNE.}
    \end{block}
\end{frame}

\begin{frame}[fragile]
    \frametitle{t-SNE Visualization - Overview}
    t-Distributed Stochastic Neighbor Embedding (t-SNE) is a popular technique for visualizing high-dimensional data in a lower-dimensional space. 
    \begin{itemize}
        \item Effective in revealing structures, clusters, and relationships in data
        \item Primarily used for 2D or 3D visualizations
    \end{itemize}
\end{frame}

\begin{frame}[fragile]
    \frametitle{t-SNE Visualization - Key Concepts}
    \begin{enumerate}
        \item \textbf{High-Dimensional Space}
            \begin{itemize}
                \item Data exists in spaces with many features (e.g., images, text)
            \end{itemize}
        \item \textbf{Dimensionality Reduction}
            \begin{itemize}
                \item Reduces the number of features while preserving significant information
                \item Aims to maintain local structure of the data
            \end{itemize}
        \item \textbf{Pairwise Similarities}
            \begin{itemize}
                \item Measures similarity by converting distances into probabilities
                \item Close points in high-dim space have higher probabilities of being neighbors
            \end{itemize}
        \item \textbf{Visualization}
            \begin{itemize}
                \item 2D or 3D scatter plots with clusters of similar data points
            \end{itemize}
    \end{enumerate}
\end{frame}

\begin{frame}[fragile]
    \frametitle{t-SNE Visualization - Example & Code}
    \textbf{Example Illustration:}
    Consider a dataset of different flowers characterized by petal length and sepal width.
    \begin{itemize}
        \item t-SNE might show distinct clusters, grouping similar species together.
    \end{itemize}

    \textbf{Code Snippet: t-SNE Implementation in Python}
    \begin{lstlisting}[language=Python]
from sklearn.manifold import TSNE
import matplotlib.pyplot as plt

# Example data
X = ... # high-dimensional data (e.g., feature matrix)

# Apply t-SNE
tsne = TSNE(n_components=2, random_state=42)
X_embedded = tsne.fit_transform(X)

# Visualization
plt.scatter(X_embedded[:, 0], X_embedded[:, 1])
plt.title('t-SNE Visualization')
plt.xlabel('t-SNE Component 1')
plt.ylabel('t-SNE Component 2')
plt.show()
    \end{lstlisting}
\end{frame}

\begin{frame}[fragile]
    \frametitle{Comparison of PCA and t-SNE - Introduction}
    
    \begin{block}{Introduction to Dimensionality Reduction Techniques}
        Dimensionality reduction techniques like PCA (Principal Component Analysis) and t-SNE (t-Distributed Stochastic Neighbor Embedding) are vital in data analysis, particularly when dealing with high-dimensional datasets. 
        Both methodologies simplify data while preserving essential characteristics but differ significantly in their approaches and applications.
    \end{block}
\end{frame}

\begin{frame}[fragile]
    \frametitle{Comparison of PCA and t-SNE - Key Differences}

    \begin{table}[ht]
        \centering
        \begin{tabular}{|l|l|l|}
            \hline
            \textbf{Feature} & \textbf{PCA} & \textbf{t-SNE} \\
            \hline
            Purpose & Linear transformation for variance maximization & Non-linear dimensionality reduction with focus on preserving local structure \\
            \hline
            Output & Preserves global structure and variance & Emphasizes local relationships and clusters \\ 
            \hline
            Speed & Computationally efficient for large datasets & Slower, especially with very large datasets \\
            \hline
            Interpretability & Easier to interpret due to linear components & Less interpretable since projections are non-linear \\
            \hline
            Parameters & No hyperparameters (except number of components) & Hyperparameters like perplexity affect output \\
            \hline
            Use Cases & Exploratory data analysis, noise reduction & Visualization of complex data distributions \\
            \hline
        \end{tabular}
    \end{table}
\end{frame}

\begin{frame}[fragile]
    \frametitle{PCA vs t-SNE - Strengths and Weaknesses}

    \begin{block}{PCA}
        \begin{itemize}
            \item \textbf{Strengths:}
                \begin{itemize}
                    \item Fast and scalable to large datasets
                    \item Provides a clear understanding of variance in data
                    \item Low computational complexity ($O(n^2)$ with $n$ features)
                \end{itemize}
            \item \textbf{Weaknesses:}
                \begin{itemize}
                    \item Assumes linearity, may miss complex relationships
                    \item Sensitive to outliers
                    \item Can oversimplify data, losing significant information
                \end{itemize}
        \end{itemize}
    \end{block}
    
    \begin{block}{t-SNE}
        \begin{itemize}
            \item \textbf{Strengths:}
                \begin{itemize}
                    \item Captures non-linear structure, ideal for clustering
                    \item Excellent for visualizing high-dimensional data in 2D or 3D
                    \item Preserves local neighborhood structures
                \end{itemize}
            \item \textbf{Weaknesses:}
                \begin{itemize}
                    \item Computationally intensive; may not scale well
                    \item Results can vary significantly with parameter changes
                    \item Can create misleading portrayals of data if misused (e.g., clusters may seem more separate than they really are)
                \end{itemize}
        \end{itemize}
    \end{block}
\end{frame}

\begin{frame}[fragile]
    \frametitle{Example Use Cases}

    \begin{itemize}
        \item \textbf{PCA:} Used in image compression, gene expression analysis, and exploratory data analysis to visualize the structure of the data across a few dimensions.
        
        \item \textbf{t-SNE:} Widely used in the visualization of high-dimensional systems like single-cell RNA sequencing data, where the goal is to reveal the underlying clusters of cells based on expression patterns.
    \end{itemize}
\end{frame}

\begin{frame}[fragile]
    \frametitle{Summary and Conclusion}

    \begin{block}{Summary}
        - PCA is suitable for analyzing data with a known structure, where linear relationships dominate. It effectively reduces dimensions while preserving variance, making it easier to visualize and interpret data flow.
        
        - t-SNE is preferred for visualizing complex high-dimensional datasets where local relationships and the formation of clusters are of primary interest. However, its computational cost and sensitivity to parameter tuning need to be carefully managed.
    \end{block}
    
    \begin{block}{Conclusion}
        Choosing between PCA and t-SNE hinges on the dataset's nature, analysis goals, and the need for interpretability. Understanding their strengths and weaknesses will empower data analysts to select the most suitable technique for their specific tasks.
    \end{block}
\end{frame}

\begin{frame}[fragile]
    \frametitle{Practical Applications}
    Dimensionality Reduction techniques such as PCA (Principal Component Analysis) and t-SNE (t-distributed Stochastic Neighbor Embedding) are crucial in simplifying high-dimensional datasets while preserving essential patterns. 
    We will explore their applications in healthcare, finance, and image processing.
\end{frame}

\begin{frame}[fragile]
    \frametitle{Applications in Healthcare}
    \begin{itemize}
        \item \textbf{PCA in Genomics:}
        \begin{itemize}
            \item \textbf{Application:} PCA analyzes gene expression data to identify disease predispositions.
            \item \textbf{Example:} Reducing thousands of gene expression measurements to reveal clusters related to diseases like cancer.
        \end{itemize}
        
        \item \textbf{t-SNE in Medical Imaging:}
        \begin{itemize}
            \item \textbf{Application:} Visualize complex imaging data (e.g., MRI scans) in 2/3D.
            \item \textbf{Example:} Categorizing brain tissue types and understanding conditions like Alzheimer’s.
        \end{itemize}
    \end{itemize}
\end{frame}

\begin{frame}[fragile]
    \frametitle{Applications in Finance and Image Processing}
    \begin{itemize}
        \item \textbf{Finance:}
        \begin{itemize}
            \item \textbf{PCA for Risk Management:}
            \begin{itemize}
                \item \textbf{Application:} Reduce complexity in portfolio data to identify risk factors.
                \item \textbf{Example:} Discovering correlations in assets to optimize investment strategies.
            \end{itemize}
            \item \textbf{t-SNE for Fraud Detection:}
            \begin{itemize}
                \item \textbf{Application:} Identify unusual transaction patterns indicative of fraud.
                \item \textbf{Example:} Visualizing transaction data to highlight clusters showing anomalous behavior.
            \end{itemize}
        \end{itemize}

        \item \textbf{Image Processing:}
        \begin{itemize}
            \item \textbf{PCA in Face Recognition:}
            \begin{itemize}
                \item \textbf{Application:} Compress facial image datasets by reducing dimensions.
                \item \textbf{Example:} Using principal components for efficient face recognition and classification.
            \end{itemize}
            \item \textbf{t-SNE for Image Classification:}
            \begin{itemize}
                \item \textbf{Application:} Visualize image embeddings in lower dimensions for clustering.
                \item \textbf{Example:} Classifying digit images (0-9) and identifying misclassifications.
            \end{itemize}
        \end{itemize}
    \end{itemize}
\end{frame}

\begin{frame}[fragile]
    \frametitle{Key Points and Conclusion}
    \begin{itemize}
        \item \textbf{Understanding Relationships:} PCA and t-SNE uncover relationships in complex datasets through dimensionality reduction.
        \item \textbf{Choice of Technique:} PCA is suited for linear interpretations, while t-SNE excels at visualizing non-linear structures.
        \item \textbf{Broader Implications:} These methods enhance decision-making across sectors by clarifying insights from large datasets.
    \end{itemize}
    
    \begin{block}{Conclusion}
        Dimensionality reduction techniques like PCA and t-SNE are invaluable tools across various domains, offering enriched analysis and visualization of complex data.
    \end{block}
\end{frame}


\end{document}