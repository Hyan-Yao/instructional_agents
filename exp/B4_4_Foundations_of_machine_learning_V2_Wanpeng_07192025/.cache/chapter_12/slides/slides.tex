\documentclass[aspectratio=169]{beamer}

% Theme and Color Setup
\usetheme{Madrid}
\usecolortheme{whale}
\useinnertheme{rectangles}
\useoutertheme{miniframes}

% Additional Packages
\usepackage[utf8]{inputenc}
\usepackage[T1]{fontenc}
\usepackage{graphicx}
\usepackage{booktabs}
\usepackage{amsmath}
\usepackage{amssymb}
\usepackage{hyperref}

% Custom Colors
\definecolor{myblue}{RGB}{31, 73, 125}
\definecolor{myorange}{RGB}{230, 126, 34}

% Set Theme Colors
\setbeamercolor{structure}{fg=myblue}
\setbeamercolor{frametitle}{fg=white, bg=myblue}
\setbeamercolor{title}{fg=myblue}

% Title Page Information
\title[Unsupervised Learning]{Chapter 12: Unsupervised Learning: Deep Learning}
\author[J. Smith]{John Smith, Ph.D.}
\institute[University Name]{
  Department of Computer Science\\
  University Name\\
  \vspace{0.3cm}
  Email: email@university.edu\\
  Website: www.university.edu
}
\date{\today}

% Document Start
\begin{document}

\frame{\titlepage}

\begin{frame}[fragile]
    \frametitle{Introduction to Unsupervised Learning}
    
    \begin{block}{What is Unsupervised Learning?}
        Unsupervised Learning analyzes unlabeled data to identify patterns or structures without pre-existing labels.
    \end{block}
    
    \begin{itemize}
        \item \textbf{No Labels:} Operates on datasets without target outputs.
        \item \textbf{Pattern Recognition:} Infers natural structures in data points.
        \item \textbf{Exploratory Analysis:} Discovers hidden patterns or structures.
    \end{itemize}
\end{frame}

\begin{frame}[fragile]
    \frametitle{Significance of Unsupervised Learning}
    
    Unsupervised learning is crucial in various applications:
    
    \begin{enumerate}
        \item \textbf{Data Preprocessing:} Assists in feature reduction (e.g., PCA).
        \item \textbf{Clustering:} Identifies natural groupings (e.g., customer segmentation).
        \item \textbf{Anomaly Detection:} Detects outliers indicating fraud or unusual patterns.
        \item \textbf{Dimensionality Reduction:} Reduces features using techniques like autoencoders.
    \end{enumerate}
\end{frame}

\begin{frame}[fragile]
    \frametitle{Examples of Unsupervised Learning Algorithms}

    \begin{itemize}
        \item \textbf{K-Means Clustering:}
        \begin{itemize}
            \item Iteratively assigns data points to $k$ clusters based on similarity.
            \item \textbf{Minimization Objective:}
            \begin{equation}
            J = \sum_{i=1}^{k} \sum_{j=1}^{m} || x^{(j)} - \mu_i ||^2
            \end{equation}
        \end{itemize}
        
        \item \textbf{Hierarchical Clustering:} Builds dendrograms for multiple clustering levels.
        
        \item \textbf{Autoencoders:} Neural networks for efficient data coding and reconstruction.
    \end{itemize}
\end{frame}

\begin{frame}[fragile]{Deep Learning Defined}
    \begin{block}{Overview}
        Introduction to deep learning and its role within the unsupervised learning framework.
    \end{block}
\end{frame}

\begin{frame}[fragile]{What is Deep Learning?}
    \begin{itemize}
        \item Deep Learning is a subset of machine learning focused on modeling high-level abstractions using multiple layers of neural networks.
        \item Unlike traditional machine learning, deep learning automatically discovers representations from data without relying on hand-crafted features.
    \end{itemize}
\end{frame}

\begin{frame}[fragile]{Characteristics of Deep Learning}
    \begin{itemize}
        \item \textbf{Layered Structure:} Comprises multiple layers (input, hidden, and output) to learn complex data representations.
        \item \textbf{Neural Networks:} Utilizes artificial neural networks that mimic human brain processing.
        \item \textbf{Data-Driven:} Models learn from raw data, effective where data is abundant and labels are scarce.
    \end{itemize}
\end{frame}

\begin{frame}[fragile]{Deep Learning in the Unsupervised Learning Framework}
    \begin{block}{Overview}
        Deep learning finds patterns and structures in unlabeled datasets, seamlessly fitting within the unsupervised learning framework.
    \end{block}

    \begin{itemize}
        \item \textbf{Feature Extraction:} Automatically learns features from data without human intervention. For example, in image processing, it identifies shapes and textures.
        \item \textbf{Representation Learning:} Uses architectures like autoencoders to capture compact data representations for clustering or anomaly detection.
        \item \textbf{Generative Models:} Frameworks like GANs can create new examples mimicking training data, useful for unsupervised tasks.
    \end{itemize}
\end{frame}

\begin{frame}[fragile]{Key Points to Emphasize}
    \begin{itemize}
        \item \textbf{Transformative Potential:} Deep learning has revolutionized fields such as image recognition and natural language processing by learning intricate patterns in large datasets.
        \item \textbf{Flexibility:} Not limited to complex datasets; simpler datasets can benefit from automated feature learning.
        \item \textbf{Growing Applications:} Industries using deep learning for unsupervised tasks include healthcare (imaging), finance (fraud detection), and marketing (customer segmentation).
    \end{itemize}
\end{frame}

\begin{frame}[fragile]{Example: Autoencoders}
    \begin{block}{Definition}
        An autoencoder is a type of neural network that learns efficient representations for dimensionality reduction or feature learning.
    \end{block}

    \begin{itemize}
        \item Compresses input data into a smaller representation, reconstructing it back to the original.
        \item The model identifies key features representing the data effectively using its encoder-decoder architecture.
    \end{itemize}
\end{frame}

\begin{frame}[fragile]{Mathematical Representation of Autoencoders}
    The loss function for autoencoders is typically the Mean Squared Error (MSE):

    \begin{equation}
        L = \frac{1}{n} \sum_{i=1}^{n} (x_i - \hat{x}_i)^2
    \end{equation}

    where $x_i$ is the original input, $\hat{x}_i$ is the reconstructed output, and $n$ is the number of data points.
\end{frame}

\begin{frame}[fragile]{Conclusion}
    In summary, deep learning is a powerful approach to unsupervised learning, leveraging neural networks to understand the underlying structure of data and offering innovative applications across various industries.
\end{frame}

\begin{frame}[fragile]
    \frametitle{Key Techniques in Deep Learning}
    \begin{block}{Overview of Key Techniques}
        Deep learning, a subset of machine learning, has revolutionized various fields through unsupervised learning techniques. This slide covers two prominent methods: Autoencoders and Generative Adversarial Networks (GANs).
    \end{block}
\end{frame}

\begin{frame}[fragile]
    \frametitle{Key Techniques in Deep Learning - Part 1: Autoencoders}
    \begin{block}{Definition}
        Autoencoders are neural networks designed to learn efficient representations of data, typically for dimensionality reduction or feature learning. They consist of two parts: the encoder and the decoder.
    \end{block}
    
    \begin{itemize}
        \item \textbf{Encoder:} Compresses the input data into a lower-dimensional representation (latent space).
        \item \textbf{Decoder:} Reconstructs the data from this compressed representation.
    \end{itemize}
    
    \begin{block}{Key Points}
        \begin{itemize}
            \item \textbf{Training Objective:} Minimize the difference between input and reconstructed output, often using Mean Squared Error (MSE):
            \begin{equation}
                \text{Loss} = \frac{1}{N} \sum_{i=1}^{N} (x_i - \hat{x}_i)^2
            \end{equation}
            \item \textbf{Use Cases:} Image denoising, anomaly detection, and data compression.
        \end{itemize}
    \end{block}
\end{frame}

\begin{frame}[fragile]
    \frametitle{Key Techniques in Deep Learning - Part 2: Generative Adversarial Networks (GANs)}
    \begin{block}{Definition}
        GANs comprise two competing neural networks: the generator and the discriminator. The generator creates fake data, while the discriminator evaluates whether data is real or fake.
    \end{block}

    \begin{itemize}
        \item \textbf{Generator:} Aims to produce data indistinguishable from the real dataset.
        \item \textbf{Discriminator:} Trains to differentiate between real and generated data.
    \end{itemize}

    \begin{block}{Key Points}
        \begin{itemize}
            \item \textbf{Training Process:} GANs use a game-theoretical approach where both networks are trained simultaneously:
            \begin{itemize}
                \item The generator tries to maximize the likelihood of the discriminator making a mistake.
                \item The discriminator aims to minimize its classification error.
            \end{itemize}
            \item \textbf{Loss Functions:}
            \begin{itemize}
                \item Generator Loss: 
                \begin{equation}
                    L_G = -\log(D(G(z)))
                \end{equation}
                \item Discriminator Loss: 
                \begin{equation}
                    L_D = -(\log(D(x)) + \log(1 - D(G(z))))
                \end{equation}
            \end{itemize}
            \item \textbf{Applications:} Image synthesis, video generation, and innovative design in art and music.
        \end{itemize}
    \end{block} 
\end{frame}

\begin{frame}[fragile]
    \frametitle{Key Techniques in Deep Learning - Summary}
    \begin{itemize}
        \item \textbf{Autoencoders} efficiently represent input data through encoding and decoding, aiding tasks like denoising and anomaly detection.
        \item \textbf{GANs} create new data by pitting a generator against a discriminator, enabling applications in art and media production.
    \end{itemize}

    \begin{block}{Invite for Questions}
        As we progress to explore applications in unsupervised learning on the next slide, consider the real-world impacts of these techniques!
    \end{block}
\end{frame}

\begin{frame}[fragile]
    \frametitle{Applications of Deep Learning in Unsupervised Learning}
    \begin{block}{Introduction to Unsupervised Learning}
        Unsupervised learning identifies patterns and structures in data without labeled responses.
    \end{block}
\end{frame}

\begin{frame}[fragile]
    \frametitle{Key Deep Learning Techniques for Unsupervised Learning}
    \begin{enumerate}
        \item \textbf{Autoencoders}
            \begin{itemize}
                \item Learn efficient codings: An encoder compresses input data, while a decoder reconstructs it.
                \item \textit{Use Case:} Image denoising by removing noise from images.
            \end{itemize}
            
        \item \textbf{Generative Adversarial Networks (GANs)}
            \begin{itemize}
                \item A generator and a discriminator compete in a zero-sum game.
                \item \textit{Use Case:} Image synthesis (e.g., generating realistic art or human faces).
            \end{itemize}
            
        \item \textbf{Clustering Algorithms}
            \begin{itemize}
                \item Enhance traditional clustering methods (e.g., K-means, DBSCAN) using neural features.
                \item \textit{Use Case:} Customer segmentation based on purchasing behavior in e-commerce.
            \end{itemize}
    \end{enumerate}
\end{frame}

\begin{frame}[fragile]
    \frametitle{Real-World Applications}
    \begin{enumerate}
        \item \textbf{Anomaly Detection}
            \begin{itemize}
                \item \textit{Example:} Credit card fraud detection using autoencoders.
            \end{itemize}
            
        \item \textbf{Dimensionality Reduction}
            \begin{itemize}
                \item \textit{Example:} t-SNE reduces high-dimensional data for visualization, such as genetic data.
            \end{itemize}
            
        \item \textbf{Natural Language Processing (NLP)}
            \begin{itemize}
                \item \textit{Example:} Topic modeling with LDA discovers hidden topics in large text corpuses.
            \end{itemize}
            
        \item \textbf{Image Recognition}
            \begin{itemize}
                \item \textit{Example:} CNNs cluster and categorize images from vast datasets.
            \end{itemize}
    \end{enumerate}
\end{frame}

\begin{frame}[fragile]
    \frametitle{Key Takeaways and Conclusion}
    \begin{itemize}
        \item \textbf{Flexibility:} Applicable in finance, healthcare, and entertainment.
        \item \textbf{Scalability:} Efficiently handles large datasets with advanced models.
        \item \textbf{Exploratory Data Analysis:} Provides insights and generates hypotheses before applying supervised techniques.
    \end{itemize}
    
    \begin{block}{Conclusion}
        Understanding unsupervised learning applications empowers researchers to tackle real-world problems effectively.
    \end{block}
\end{frame}

\begin{frame}[fragile]
    \frametitle{Autoencoders - Overview}
    \begin{itemize}
        \item Autoencoders are artificial neural networks for unsupervised learning tasks.
        \item They primarily serve for:
        \begin{itemize}
            \item Dimensionality reduction
            \item Feature extraction
        \end{itemize}
        \item The main goal is to learn a compact representation of input data for efficient analysis or reconstruction.
    \end{itemize}
\end{frame}

\begin{frame}[fragile]
    \frametitle{Autoencoders - Structure}
    An autoencoder consists of three main components:
    \begin{enumerate}
        \item \textbf{Encoder}: Compresses input data into lower-dimensional representation (latent space).
        \item \textbf{Latent Space (Bottleneck)}: Central part holding the compressed representation.
        \item \textbf{Decoder}: Reconstructs original data from the compressed representation.
    \end{enumerate}
\end{frame}

\begin{frame}[fragile]
    \frametitle{Key Concepts and Applications}
    \begin{block}{Key Concepts}
        \begin{itemize}
            \item \textbf{Dimensionality Reduction}: Simplifies models, reduces processing time.
            \item \textbf{Feature Extraction}: Captures essential characteristics for downstream tasks (e.g., classification).
        \end{itemize}
    \end{block}

    \begin{block}{Applications}
        \begin{itemize}
            \item \textbf{Image Compression}: Efficiently compressing image data.
            \item \textbf{Anomaly Detection}: Identifying deviations from normal patterns.
        \end{itemize}
    \end{block}
\end{frame}

\begin{frame}[fragile]
    \frametitle{Example of Autoencoders}
    Consider a dataset of grayscale images of faces:
    \begin{itemize}
        \item Input: A face image (e.g., 64x64 pixels, 4096 input variables)
        \item Compression: Encoder reduces to latent space (e.g., 128 features)
        \item Reconstruction: Decoder reconstructs the image from these 128 features.
    \end{itemize}
\end{frame}

\begin{frame}[fragile]
    \frametitle{Mathematical Representation}
    Consider an input vector $\mathbf{x}$:
    \begin{align*}
        & \text{Encoder function: } f: \mathbf{x} \rightarrow \mathbf{z} \\
        & \text{Decoder function: } g: \mathbf{z} \rightarrow \hat{\mathbf{x}}
    \end{align*}
    The objective is to minimize the reconstruction error given by:
    \begin{equation}
        L(\mathbf{x}, \hat{\mathbf{x}}) = ||\mathbf{x} - \hat{\mathbf{x}}||^2
    \end{equation}
    where \( L \) is the loss function and \( ||\cdot||^2 \) denotes the squared Euclidean norm.
\end{frame}

\begin{frame}[fragile]
    \frametitle{Summary and Key Points}
    \begin{itemize}
        \item Autoencoders are powerful tools for unsupervised learning.
        \item They aid in:
        \begin{itemize}
            \item Dimensionality reduction
            \item Feature extraction
        \end{itemize}
        \item Applications include image compression and anomaly detection.
    \end{itemize}
\end{frame}

\begin{frame}[fragile]
    \frametitle{Generative Adversarial Networks (GANs) - Overview}
    \begin{itemize}
        \item GANs are machine learning frameworks to generate new data that resembles an existing dataset.
        \item Introduced by Ian Goodfellow in 2014.
        \item Comprised of two main components:
        \begin{itemize}
            \item \textbf{Generator (G)}: Produces synthetic data from random noise.
            \item \textbf{Discriminator (D)}: Evaluates data authenticity by classifying it as real or fake.
        \end{itemize}
        \item Both networks are trained simultaneously through adversarial training.
    \end{itemize}
\end{frame}

\begin{frame}[fragile]
    \frametitle{Generative Adversarial Networks (GANs) - Architecture and Training Process}
    \begin{block}{Architecture of GANs}
        \begin{itemize}
            \item \textbf{Generator (G)}: Aims to fool the discriminator by generating realistic data.
            \item \textbf{Discriminator (D)}: Tries to accurately classify data as real or fake.
        \end{itemize}
    \end{block}
    
    \vspace{0.2cm}
    
    \begin{block}{The Training Process}
        \begin{enumerate}
            \item Random noise input to generator.
            \item Generator creates synthetic data.
            \item Discriminator evaluates real and generated data.
            \item Calculate losses for G and D.
            \item Backpropagation for both networks.
        \end{enumerate}
    \end{block}
\end{frame}

\begin{frame}[fragile]
    \frametitle{Generative Adversarial Networks (GANs) - Mathematical Representation and Applications}
    \begin{block}{Mathematical Representation}
        The objective function for GANs:
        \begin{equation}
            \min_G \max_D V(D, G) = E_{x \sim p_{data}(x)} [\log D(x)] + E_{z \sim p_z(z)} [\log(1 - D(G(z)))]
        \end{equation}
        Where:
        \begin{itemize}
            \item \( D(x) \): Discriminator output for real data \( x \).
            \item \( G(z) \): Generated data from noise \( z \).
        \end{itemize}
    \end{block}

    \vspace{0.2cm}
    
    \begin{block}{Key Applications of GANs}
        \begin{itemize}
            \item Image Generation (e.g. StyleGAN, BigGAN).
            \item Data Augmentation for training datasets.
            \item Enhancing image resolution (Super Resolution).
            \item Text-to-Image synthesis (e.g. DALL-E).
        \end{itemize}
    \end{block}
\end{frame}

\begin{frame}[fragile]
    \frametitle{Clustering Techniques - Overview}
    \begin{block}{What is Clustering?}
        Clustering is an unsupervised learning technique that groups similar data points together based on their features. The goal is to reduce the complexity of the data while uncovering inherent patterns. 
    \end{block}

    \begin{itemize}
        \item Used in market segmentation, image processing, and social network analysis.
    \end{itemize}
\end{frame}

\begin{frame}[fragile]
    \frametitle{Key Concepts in Clustering}
    \begin{itemize}
        \item \textbf{Unsupervised Learning:} No reliance on labeled data; algorithms identify patterns without predefined categories.
        
        \item \textbf{Distance Measures:} Similarity quantified using metrics like:
        \begin{itemize}
            \item Euclidean distance
            \item Manhattan distance
            \item Cosine similarity
        \end{itemize}
    \end{itemize}
\end{frame}

\begin{frame}[fragile]
    \frametitle{K-Means Clustering}
    K-means clustering is a popular algorithm that clusters data points based on their similarity. The steps include:
    \begin{enumerate}
        \item \textbf{Initialization:} Select \( K \) initial centroids randomly.
        \item \textbf{Assignment Step:} Assign each data point to the nearest centroid.
        \item \textbf{Update Step:} Calculate new centroids as the mean of assigned points.
        \item \textbf{Iteration:} Repeat the assignment and update steps until convergence.
    \end{enumerate}

    \begin{block}{Objective Function}
        \begin{equation}
            J = \sum_{i=1}^{K} \sum_{x \in C_i} \| x - \mu_i \|^2
        \end{equation}
        Where:
        \begin{itemize}
            \item \( J \) is the cost function,
            \item \( K \) is the number of clusters,
            \item \( \mu_i \) is the centroid of cluster \( C_i \),
            \item \( x \) are the data points in cluster \( C_i \).
        \end{itemize}
    \end{block}
\end{frame}

\begin{frame}[fragile]
    \frametitle{Example of K-Means Clustering}
    \begin{block}{Customer Segmentation}
        Example with customer data (features: age, income):
        \begin{itemize}
            \item **Cluster 1:** Young, low income
            \item **Cluster 2:** Middle-aged, middle income
            \item **Cluster 3:** Older, high income
        \end{itemize}
        This grouping assists in targeted marketing strategies.
    \end{block}
\end{frame}

\begin{frame}[fragile]
    \frametitle{Integration of K-Means with Deep Learning}
    \begin{itemize}
        \item \textbf{Feature Extraction:} Use neural networks (e.g., CNNs) for feature extraction from complex data before applying K-means.
        
        \item \textbf{Deep Clustering:} Combines clustering and deep learning into a single framework where the network learns representations while clustering.
    \end{itemize}
\end{frame}

\begin{frame}[fragile]
    \frametitle{Example in Code (Python)}
    \begin{lstlisting}[language=Python]
from sklearn.cluster import KMeans
from keras.models import Sequential
from keras.layers import Dense

# Sample neural network for feature extraction
model = Sequential()
model.add(Dense(64, activation='relu', input_shape=(input_dim,)))
model.add(Dense(32, activation='relu'))
model.add(Dense(num_clusters, activation='softmax'))

# Use model to extract features
features = model.predict(data)

# Apply K-means on the extracted features
kmeans = KMeans(n_clusters=3)
kmeans.fit(features)
labels = kmeans.labels_
    \end{lstlisting}
\end{frame}

\begin{frame}[fragile]
    \frametitle{Key Points to Emphasize}
    \begin{itemize}
        \item Clustering is essential for discovering patterns in unlabeled data.
        \item K-means is popular due to its simplicity and effectiveness.
        \item Integrating deep learning enhances performance in complex datasets.
    \end{itemize}
\end{frame}

\begin{frame}[fragile]
    \frametitle{Dimensionality Reduction Techniques - Introduction}
    % Introduction to dimensionality reduction techniques in unsupervised learning and deep learning.
    
    \begin{block}{Overview}
        In unsupervised learning and deep learning, managing high-dimensional data can be challenging due to the "curse of dimensionality." 
        Dimensionality reduction techniques simplify data for visualization and analysis while maintaining essential information.
    \end{block}

    \begin{itemize}
        \item **Principal Component Analysis (PCA)**
        \item **t-distributed Stochastic Neighbor Embedding (t-SNE)**
    \end{itemize}
\end{frame}

\begin{frame}[fragile]
    \frametitle{Principal Component Analysis (PCA)}
    % Detailed explanation of PCA, its properties, and applications.
    
    \begin{block}{Explanation}
        PCA transforms a dataset into orthogonal components, the linear combinations of the original features, to minimize redundancy and maximize variance.
    \end{block}
    
    \begin{itemize}
        \item **Dimensionality Reduction**: Reduces features but retains essential data relationships.
        \item **Linearity**: Best suited for data close to a linear subspace.
        \item **Eigenvalues and Eigenvectors**: Finds directions of maximum variance through covariance matrix decomposition.
    \end{itemize}

    \begin{block}{Example}
        Reducing four flower attributes (petal length, petal width, sepal length, sepal width) to two dimensions capturing most variance.
    \end{block}

    \begin{equation}
        Z = X \cdot W
    \end{equation}
    Where:
    \begin{itemize}
        \item $Z$: Transformed data
        \item $X$: Original data (centered)
        \item $W$: Eigenvectors of the covariance matrix
    \end{itemize}
\end{frame}

\begin{frame}[fragile]
    \frametitle{t-Distributed Stochastic Neighbor Embedding (t-SNE)}
    % Detailed explanation of t-SNE, its characteristics, and applications.
    
    \begin{block}{Explanation}
        t-SNE is a non-linear technique for visualizing high-dimensional datasets, converting distances into conditional probabilities reflecting similarity.
    \end{block}
    
    \begin{itemize}
        \item **Non-linearity**: Captures complex relationships and manifold structures.
        \item **Local Structure Preservation**: Retains local relationships, clustering similar points.
        \item **Computational Intensity**: Can be expensive; needs parameter tuning (e.g., perplexity).
    \end{itemize}

    \begin{block}{Example}
        Visualizing handwritten digits by projecting 784 dimensions (28x28 pixels) down to 2D, revealing clusters of similar digits.
    \end{block}

    \begin{equation}
        C = \sum_{i} \sum_{j} P_{ij} \log\left(\frac{P_{ij}}{Q_{ij}}\right)
    \end{equation}
    Where:
    \begin{itemize}
        \item $P_{ij}$: Conditional probability of similarity in high-dimensional space
        \item $Q_{ij}$: Conditional probability of similarity in low-dimensional space
    \end{itemize}
\end{frame}

\begin{frame}[fragile]
    \frametitle{Dimensionality Reduction Techniques - Conclusion}
    % Summary of the effectiveness of PCA and t-SNE in unsupervised learning.
    
    \begin{block}{Summary}
        Both PCA and t-SNE are essential techniques in unsupervised learning and deep learning:
        \begin{itemize}
            \item PCA: Ideal for linear dimensionality reduction.
            \item t-SNE: Excellent for preserving local data structure.
        \end{itemize}
    \end{block}

    \begin{block}{Further Learning}
        \begin{itemize}
            \item Experiment with PCA and t-SNE using Python libraries like Scikit-learn.
            \item Use scatter plots for visualizing results and gaining insights.
        \end{itemize}
    \end{block}

    Master these techniques to enhance your ability to interpret complex datasets and discover meaningful patterns in unsupervised learning tasks.
\end{frame}

\begin{frame}[fragile]
    \frametitle{Challenges in Deep Learning for Unsupervised Learning}
    
    \begin{block}{Overview of Unsupervised Learning}
        Unsupervised learning involves training models on unlabeled data, allowing the model to identify patterns and structures without human intervention. Deep learning has advanced the field but presents distinct challenges in unsupervised contexts.
    \end{block}
\end{frame}

\begin{frame}[fragile]
    \frametitle{Key Challenges in Unsupervised Learning}
    
    \begin{enumerate}
        \item \textbf{Lack of Clear Objectives}
        \begin{itemize}
            \item Models lack definitive targets, leading to potential overfitting to noise.
            \item Example: In clustering, undefined centers can hinder meaningful categorization.
        \end{itemize}
        
        \item \textbf{High Dimensionality}
        \begin{itemize}
            \item Increased dimensions complicate generalization due to sparsity in data.
            \item Illustration: Points become sparse in higher dimensions, impacting clustering.
        \end{itemize}
    \end{enumerate}
\end{frame}

\begin{frame}[fragile]
    \frametitle{More Challenges in Unsupervised Learning}
    
    \begin{enumerate}
        \setcounter{enumi}{2}
        \item \textbf{Noisy Data}
        \begin{itemize}
            \item Noise misleads feature representation during learning.
            \item Example: Irrelevant patterns in image segmentation can skew results.
        \end{itemize}
        
        \item \textbf{Evaluating Model Performance}
        \begin{itemize}
            \item Challenges arise from the absence of labeled outputs.
            \item Common metrics: silhouette score, Davies-Bouldin index.
        \end{itemize}
        
        \item \textbf{Model Complexity and Overfitting}
        \begin{itemize}
            \item Complex models risk overfitting, learning patterns specific to the training set.
            \item Example: A deep autoencoder without regularization may fail on unseen data.
        \end{itemize}
    \end{enumerate}
\end{frame}

\begin{frame}[fragile]
    \frametitle{Interpreting Results and Conclusion}
    
    \begin{enumerate}
        \setcounter{enumi}{5}
        \item \textbf{Interpreting Results}
        \begin{itemize}
            \item The black-box nature complicates understanding learned features.
            \item Needs tools like attention mechanisms for better interpretability.
        \end{itemize}
    \end{enumerate}
    
    \begin{block}{Key Points to Emphasize}
        \begin{itemize}
            \item Unsupervised learning holds potential but is fraught with challenges.
            \item Ongoing research aims to refine algorithms and evaluation methods.
        \end{itemize}
    \end{block}
    
    \begin{block}{Conclusion}
        Understanding the challenges helps in developing robust models. Awareness of issues like noise and dimensionality informs strategies in unsupervised tasks.
    \end{block}
\end{frame}

\begin{frame}[fragile]
    \frametitle{Ethical Considerations in Unsupervised Learning}
    
    \begin{block}{Introduction to Ethical Considerations}
        Unsupervised learning models, including deep learning techniques, work with unlabelled data to identify patterns and structures. 
        The ethical implications arise due to biases in datasets, lack of transparency in algorithmic decisions, and potential misuse of insights.
    \end{block}

\end{frame}

\begin{frame}[fragile]
    \frametitle{Key Ethical Considerations - Part 1}
    
    \begin{enumerate}
        \item \textbf{Bias and Fairness}
        \begin{itemize}
            \item Explanation: Models can inadvertently learn and perpetuate biases present in the training data.
            \item Example: A clustering algorithm may group individuals based on socioeconomic status or gender if those variables are correlated, potentially leading to discriminatory outcomes.
        \end{itemize}
        
        \item \textbf{Privacy Concerns}
        \begin{itemize}
            \item Explanation: Unsupervised learning often analyzes sensitive data without direct consent.
            \item Example: User data from social media can reveal personal information, such as preferences or habits, which can be exploited if not handled properly.
        \end{itemize}
    \end{enumerate}

\end{frame}

\begin{frame}[fragile]
    \frametitle{Key Ethical Considerations - Part 2}

    \begin{enumerate}
        \setcounter{enumi}{2}
        \item \textbf{Transparency and Explainability}
        \begin{itemize}
            \item Explanation: Many unsupervised algorithms, like deep autoencoders, operate as 'black boxes' where decision-making remains opaque.
            \item Example: Understanding why a model identifies a user segment as “high-risk” can be challenging, hindering accountability.
        \end{itemize}
        
        \item \textbf{Autonomy and Manipulation}
        \begin{itemize}
            \item Explanation: Insights from unsupervised learning can influence behaviors, impacting individual autonomy.
            \item Example: Marketing strategies may exploit clustering of consumer behavior, manipulating purchasing decisions.
        \end{itemize}

        \item \textbf{Social Consequences}
        \begin{itemize}
            \item Explanation: Deployment of unsupervised models can have broad societal impacts that may go unconsidered.
            \item Example: Surveillance tools may utilize clustering techniques to profile marginalized communities disproportionately.
        \end{itemize}
    \end{enumerate}

\end{frame}

\begin{frame}[fragile]
    \frametitle{Conclusion and Key Takeaways}
    
    \begin{block}{Conclusion}
        The rich potential of unsupervised learning must be balanced with a robust ethical framework. Responsible AI practices should integrate fairness, transparency, and privacy respect to foster trust in deep learning applications.
    \end{block}

    \begin{block}{Key Takeaways}
        \begin{itemize}
            \item Recognize the ethical implications of biases, privacy, transparency, autonomy, and social impact in unsupervised learning.
            \item Engage with stakeholders to develop frameworks ensuring fairness and accountability in AI systems.
        \end{itemize}
    \end{block}

\end{frame}

\begin{frame}[fragile]
    \frametitle{Additional References}
    
    \begin{itemize}
        \item Barocas, S., Hardt, M., \& Narayanan, A. (2019). Fairness and Machine Learning: Limitations and Opportunities. 
        \item Dastin, J. (2018). Amazon Scraps Secret AI Recruiting Tool That Showed Bias Against Women. Reuters.
    \end{itemize}

\end{frame}

\begin{frame}[fragile]
    \frametitle{Introduction to Future Trends}
    As deep learning continues to evolve, the integration with unsupervised learning methodologies will unlock new potentials across various domains. These advancements will pave the way for smarter systems capable of making inferences without extensive label data.
\end{frame}

\begin{frame}[fragile]
    \frametitle{Key Future Trends}
    \begin{itemize}
        \item \textbf{Improved Representation Learning}
            \begin{itemize}
                \item \textit{Concept}: Unsupervised learning excels at feature extraction, enhancing models' understanding of data.
                \item \textit{Example}: Techniques like autoencoders and Generative Adversarial Networks (GANs) will be refined to create richer embeddings for image or speech synthesis.
            \end{itemize}
        \item \textbf{Self-Supervised Learning}
            \begin{itemize}
                \item \textit{Concept}: Uses pretext tasks to learn representations from unlabeled data.
                \item \textit{Example}: Models predict missing parts of images or audio segments, reducing the need for manual annotation.
            \end{itemize}
        \item \textbf{Generative Models Evolution}
            \begin{itemize}
                \item \textit{Concept}: Focus on creating more realistic synthetic data with advanced generative models.
                \item \textit{Example}: GANs and Variational Autoencoders (VAEs) producing indistinguishable high-dimensional data for various applications.
            \end{itemize}
        \item \textbf{Multimodal Learning}
            \begin{itemize}
                \item \textit{Concept}: Simultaneous processing of different data modalities (e.g., text, image, audio).
                \item \textit{Example}: Models encoding both visual and textual information to generate insights from complex datasets.
            \end{itemize}
    \end{itemize}
\end{frame}

\begin{frame}[fragile]
    \frametitle{Real-World Applications}
    \begin{itemize}
        \item \textbf{Healthcare}: Discovering hidden patterns in patient data can lead to personalized treatments and early diagnosis.
        \item \textbf{Finance}: Unsupervised techniques can aid fraud detection by identifying anomalies and unusual patterns in transaction data.
    \end{itemize}
\end{frame}

\begin{frame}[fragile]
    \frametitle{Challenges Ahead}
    \begin{itemize}
        \item \textbf{Scalability}: Training large unsupervised models will require efficient algorithms and computational resources as data grows.
        \item \textbf{Interpretability}: Enhancing transparency to understand decisions made by unsupervised models is crucial for real-world applications.
    \end{itemize}
\end{frame}

\begin{frame}[fragile]
    \frametitle{Conclusion}
    The trajectory of deep learning within the unsupervised learning framework points towards a future where machines learn more independently, providing robust solutions in various fields. Researchers and practitioners must address the ethical and technical challenges that arise with these advancements.
\end{frame}

\begin{frame}[fragile]
    \frametitle{Key Points to Emphasize}
    \begin{itemize}
        \item Synergy between unsupervised and self-supervised learning techniques.
        \item Importance of generative and multimodal models in real-world applications.
        \item Scalability and interpretability as essential challenges for future developments.
    \end{itemize}
\end{frame}

\begin{frame}[fragile]
    \frametitle{Code Snippet: Simple Autoencoder}
    \begin{lstlisting}[language=Python]
import torch
import torch.nn as nn

# Simple autoencoder architecture
class Autoencoder(nn.Module):
    def __init__(self):
        super(Autoencoder, self).__init__()
        self.encoder = nn.Sequential(
            nn.Linear(784, 256),
            nn.ReLU(),
            nn.Linear(256, 64),
        )
        self.decoder = nn.Sequential(
            nn.Linear(64, 256),
            nn.ReLU(),
            nn.Linear(256, 784),
            nn.Sigmoid(),
        )

    def forward(self, x):
        x = self.encoder(x)
        x = self.decoder(x)
        return x
    \end{lstlisting}
    This code represents a simple autoencoder that can be trained on unlabeled data to learn a compact representation of images, demonstrating the concept of unsupervised learning in action.
\end{frame}

\begin{frame}[fragile]
    \frametitle{Collaborative Projects in Deep Learning}
    \begin{block}{Importance of Collaborative Projects}
        Collaborative projects play a critical role in advancing unsupervised deep learning by bringing together diverse expertise and resources to tackle complex problems.
    \end{block}
\end{frame}

\begin{frame}[fragile]
    \frametitle{Key Concepts}
    \begin{itemize}
        \item \textbf{Unsupervised Learning}: 
            A category of machine learning that finds patterns in datasets without labeled responses. Techniques include:
            \begin{itemize}
                \item Clustering
                \item Dimensionality reduction
                \item Generative models
            \end{itemize}
        
        \item \textbf{Deep Learning}: 
            A subset of machine learning that utilizes deep architectures of neural networks to learn from large datasets, excelling in fields such as:
            \begin{itemize}
                \item Image recognition
                \item Natural language processing
            \end{itemize}
    \end{itemize}
\end{frame}

\begin{frame}[fragile]
    \frametitle{Why Collaboration Matters}
    \begin{enumerate}
        \item \textbf{Diversity of Ideas}: 
            Innovative solutions from various fields enhance problem-solving capabilities.
        \item \textbf{Access to Resources}: 
            Joint access to datasets and computational power is crucial for training deep learning models effectively.
        \item \textbf{Scalability and Efficiency}: 
            Task division streamlines workflow and reduces timelines, allowing focused expertise on specific tasks.
        \item \textbf{Shared Learning}: 
            Knowledge transfer among team members fosters a culture of continual improvement.
    \end{enumerate}
\end{frame}

\begin{frame}[fragile]
    \frametitle{Illustrative Example}
    \begin{block}{Project Case Study: Wildfire Pattern Identification}
        \begin{itemize}
            \item \textbf{Team Composition}: 
                Data scientists, remote sensing specialists, and environmental scientists.
            \item \textbf{Collaboration Benefits}:
                \begin{itemize}
                    \item Data scientists apply clustering techniques (e.g., K-means) to segment satellite images.
                    \item Environmental scientists interpret data patterns for actionable wildfire prevention insights.
                \end{itemize}
        \end{itemize}
    \end{block}
\end{frame}

\begin{frame}[fragile]
    \frametitle{Key Points to Emphasize}
    \begin{itemize}
        \item \textbf{Real-World Applications}: Unsupervised learning improves systems in various sectors, including healthcare and fraud detection.
        \item \textbf{Collaborative Tools}: Platforms like GitHub and Kaggle enable effective collaboration and result sharing.
        \item \textbf{Case Studies}: Highlighting successful collaborations can inspire new projects and motivate engagement in teamwork.
    \end{itemize}
\end{frame}

\begin{frame}[fragile]
    \frametitle{Conclusion and Call to Action}
    \begin{block}{Conclusion}
        Engaging in collaborative projects is essential for innovation and addressing complex challenges in unsupervised deep learning.
    \end{block}
    \begin{block}{Call to Action}
        Participate in collaborative initiatives such as educational programs, hackathons, or interdisciplinary research to enhance your understanding and skills!
    \end{block}
\end{frame}

\begin{frame}[fragile]
  \frametitle{Case Studies}
  \begin{block}{Unsupervised Learning in Deep Learning: Key Case Studies}
    This presentation highlights significant case studies demonstrating successful applications of deep learning in the realm of unsupervised learning. We will explore how various organizations and researchers have harnessed these techniques to extract valuable insights from complex datasets.
  \end{block}
\end{frame}

\begin{frame}[fragile]
  \frametitle{Case Study 1: Image Clustering with Convolutional Neural Networks}
  \begin{itemize}
    \item \textbf{Context:} 
      A multinational retail company aimed to optimize product recommendations by analyzing customer photos on social media.
    \item \textbf{Approach:} 
      - Employed \textit{Convolutional Neural Networks (CNNs)} for feature extraction from images.
      - Utilized \textit{K-means clustering} to categorize images into thematic groups (e.g., fashion styles).
    \item \textbf{Results:} 
      \begin{itemize}
        \item 20\% increase in customer engagement.
        \item Enhanced inventory management by reducing excess stock.
      \end{itemize}
    \item \textbf{Key Point:} 
      Combining CNNs with clustering techniques effectively enhances customer experience by understanding visual data.
  \end{itemize}
\end{frame}

\begin{frame}[fragile]
  \frametitle{Case Study 2: Topic Modeling for Document Clustering}
  \begin{itemize}
    \item \textbf{Context:} 
      A large news organization sought to categorize vast amounts of articles for better user engagement.
    \item \textbf{Approach:} 
      - Used \textit{Latent Dirichlet Allocation (LDA)} and embeddings from \textit{Word2Vec} for topic clustering.
    \item \textbf{Results:} 
      \begin{itemize}
        \item Efficient grouping of articles by topic improved search features.
        \item Increased user engagement through personalized content delivery.
      \end{itemize}
    \item \textbf{Key Point:} 
      Unsupervised learning techniques like topic modeling are essential for effective content organization and dissemination.
  \end{itemize}
\end{frame}

\begin{frame}[fragile]
  \frametitle{Case Study 3: Anomaly Detection in Cybersecurity}
  \begin{itemize}
    \item \textbf{Context:} 
      A cybersecurity firm aimed to detect unusual patterns in network traffic to identify security threats.
    \item \textbf{Approach:} 
      - Implemented a \textit{Variational Autoencoder (VAE)} to learn normal traffic distributions.
      - Flagged deviations as suspicious activities.
    \item \textbf{Results:} 
      \begin{itemize}
        \item Improved detection rates with a 30\% reduction in false positives.
        \item Faster incident response times protecting clients better.
      \end{itemize}
    \item \textbf{Key Point:} 
      Utilizing deep learning for unsupervised anomaly detection enhances cybersecurity measures by identifying patterns without labeled data.
  \end{itemize}
\end{frame}

\begin{frame}[fragile]
  \frametitle{Conclusion}
  These case studies highlight the efficacy and versatility of deep learning in unsupervised learning applications. By leveraging powerful algorithms, organizations can derive meaningful insights, optimize processes, and improve their offerings.
  
  \begin{block}{Next Up:}
    In the following slide, we will explore the essential tools and libraries that facilitate these deep learning techniques in unsupervised learning contexts.
  \end{block}
\end{frame}

\begin{frame}[fragile]
    \frametitle{Tools and Libraries}
    \begin{block}{Overview}
        Overview of essential tools and libraries for implementing deep learning techniques in unsupervised learning.
    \end{block}

    Unsupervised learning allows models to discover patterns in unlabeled data using various tools and libraries.
\end{frame}

\begin{frame}[fragile]
    \frametitle{Key Libraries}
    
    \begin{enumerate}
        \item \textbf{TensorFlow}
        \begin{itemize}
            \item Developed by Google; flexible for machine learning.
            \item Implements clustering and autoencoders for unsupervised learning.
        \end{itemize}
        
        \item \textbf{Keras}
        \begin{itemize}
            \item High-level API running on top of TensorFlow; fast experimentation.
            \item Used for models like VAEs and GANs in unsupervised learning.
        \end{itemize}
        
        \item \textbf{Scikit-learn}
        \begin{itemize}
            \item Robust library for Python with various unsupervised algorithms.
            \item Implements clustering (K-Means, DBSCAN) and dimensionality reduction techniques (PCA, t-SNE).
        \end{itemize}
        
        \item \textbf{PyTorch}
        \begin{itemize}
            \item Open-source library by Facebook; known for dynamic computation graph.
            \item Supports deep generative models like GANs and VAEs.
        \end{itemize}
    \end{enumerate}
\end{frame}

\begin{frame}[fragile]
    \frametitle{Examples and Additional Tools}
    
    \begin{block}{Example: TensorFlow Autoencoder}
    \begin{lstlisting}[language=Python]
    import tensorflow as tf
    from tensorflow.keras import layers

    # Creating an Autoencoder
    input_layer = layers.Input(shape=(input_dim,))
    encoded = layers.Dense(encoding_dim, activation='relu')(input_layer)
    decoded = layers.Dense(input_dim, activation='sigmoid')(encoded)
    autoencoder = tf.keras.Model(input_layer, decoded)

    autoencoder.compile(optimizer='adam', loss='mean_squared_error')
    \end{lstlisting}
    \end{block}

    \begin{block}{Example: K-Means Clustering}
    \begin{lstlisting}[language=Python]
    from sklearn.cluster import KMeans
    
    kmeans = KMeans(n_clusters=3)
    kmeans.fit(data)
    labels = kmeans.labels_
    \end{lstlisting}
    \end{block}
    
    \begin{itemize}
        \item \textbf{Additional Tools:}
        \begin{itemize}
            \item \textbf{Matplotlib and Seaborn:} For data visualization.
            \item \textbf{Jupyter Notebooks:} For interactive computing and experimentation.
        \end{itemize}
    \end{itemize}
\end{frame}

\begin{frame}[fragile]
    \frametitle{Key Points and Conclusion}
    
    \begin{itemize}
        \item \textbf{Flexibility and Performance:} TensorFlow and Keras excel for deep learning; Scikit-learn for traditional models.
        \item \textbf{Community and Documentation:} Choose libraries with strong support for easier learning.
        \item \textbf{Integration:} Libraries can work together, leveraging their unique strengths.
    \end{itemize}

    \begin{block}{Conclusion}
        Utilizing the right tools is crucial for effectively implementing unsupervised learning algorithms in deep learning.
    \end{block}

    \begin{block}{Engagement Tip}
        Consider hands-on practice with Jupyter Notebooks for a practical learning experience!
    \end{block}
\end{frame}

\begin{frame}[fragile]
    \frametitle{Summary and Key Takeaways - Overview}
    \begin{block}{Unsupervised Learning}
        \begin{itemize}
            \item A type of machine learning without labeled responses.
            \item Aims to identify patterns and structures within data.
        \end{itemize}
        \begin{itemize}
            \item \textbf{Purpose:} Discover hidden patterns, group similar items, or reduce data dimensionality.
            \item \textbf{Common Techniques:} 
                \begin{itemize}
                    \item Clustering (e.g., K-means, Hierarchical Clustering)
                    \item Dimensionality Reduction (e.g., PCA, t-SNE)
                \end{itemize}
        \end{itemize}
    \end{block}
    \begin{block}{Deep Learning}
        \begin{itemize}
            \item A subset of machine learning using deep neural networks to analyze various data forms.
        \end{itemize}
        \begin{itemize}
            \item \textbf{Key Features:} 
                \begin{itemize}
                    \item Models complex, non-linear relationships.
                    \item Processes high-dimensional data.
                    \item Extracts features without manual input.
                \end{itemize}
        \end{itemize}
\end{frame}

\begin{frame}[fragile]
    \frametitle{Summary and Key Takeaways - Key Points}
    \begin{enumerate}
        \item \textbf{Integration of Unsupervised Learning with Deep Learning}
            \begin{itemize}
                \item Deep learning models like Autoencoders and GANs play a crucial role in unsupervised tasks.
                \item \textbf{Example:} An Autoencoder learns to compress and reconstruct data to reveal underlying structures.
            \end{itemize}
        \item \textbf{Applications}
            \begin{itemize}
                \item Anomaly Detection: Identify unusual patterns.
                \item Recommender Systems: Suggest content to users with similar interests.
                \item Image Compression: Learn efficient representations for storage.
            \end{itemize}
        \item \textbf{Key Tools and Libraries}
            \begin{itemize}
                \item TensorFlow and PyTorch for deep learning.
                \item Scikit-learn for preprocessing and clustering.
            \end{itemize}
        \item \textbf{Challenges}
            \begin{itemize}
                \item Lack of interpretability in unsupervised methods.
                \item Requires careful model selection and tuning.
            \end{itemize}
    \end{enumerate}
\end{frame}

\begin{frame}[fragile]
    \frametitle{Summary and Key Takeaways - Illustrative Example}
    \begin{block}{Clustering with Deep Learning}
        \begin{lstlisting}[language=Python]
# K-Means clustering using Keras
from keras.models import Model
from keras.layers import Input, Dense
from sklearn.cluster import KMeans
import numpy as np

# Autoencoder structure
input_data = Input(shape=(input_dim,))
encoded = Dense(encoding_dim, activation='relu')(input_data)
decoded = Dense(input_dim, activation='sigmoid')(encoded)
autoencoder = Model(input_data, decoded)

autoencoder.compile(optimizer='adam', loss='mean_squared_error')

# Fit model and predict clusters
autoencoder.fit(data, data, epochs=50)
encoded_data = autoencoder.predict(data)
kmeans = KMeans(n_clusters=number_of_clusters)
clusters = kmeans.fit_predict(encoded_data)
        \end{lstlisting}
    \end{block}
\end{frame}

\begin{frame}[fragile]
    \frametitle{Closing Thoughts and Transition}
    \begin{itemize}
        \item **Unsupervised Learning** in the realm of **Deep Learning** provides numerous opportunities for analyzing data without labels.
        \item Understanding these concepts is essential for practical applications.
    \end{itemize}
    \begin{block}{Transition to Q\&A}
        Moving forward, we will have an open discussion where you can ask questions regarding any topic covered in this chapter. Let's dive deeper into how these concepts apply to real-world problems!
    \end{block}
\end{frame}

\begin{frame}[fragile]
  \frametitle{Q\&A Session - Introduction to Unsupervised Learning}
  \begin{block}{Open Floor for Questions and Discussions}
    This session is dedicated to discussing deep learning techniques in the context of unsupervised learning.
  \end{block}
  
  \begin{itemize}
    \item \textbf{Unsupervised Learning:} Learning from input data without labeled outputs.
    \item \textbf{Goal:} Uncover hidden patterns and intrinsic structures in data.
  \end{itemize}
  
\end{frame}

\begin{frame}[fragile]
  \frametitle{Q\&A Session - Key Concepts}
  \begin{enumerate}
    \item \textbf{Techniques and Algorithms}
      \begin{itemize}
        \item \textbf{Clustering:} K-Means, Hierarchical Clustering, DBSCAN.
        \item \textbf{Dimensionality Reduction:} PCA, t-SNE, Autoencoders.
        \item \textbf{Anomaly Detection:} Identifying outliers in datasets.
      \end{itemize}
    
    \item \textbf{Deep Learning Methods in Unsupervised Learning}
      \begin{itemize}
        \item \textbf{Autoencoders:} Learn efficient data representations.
        \item \textbf{GANs:} Generative Adversarial Networks for data generation.
        \item \textbf{SOMs:} Self-Organizing Maps for clustering and visualization.
      \end{itemize}
  \end{enumerate}
  
\end{frame}

\begin{frame}[fragile]
  \frametitle{Q\&A Session - Key Points and Discussion Questions}
  \begin{block}{Key Points to Emphasize}
    \begin{itemize}
      \item \textbf{Feature Selection:} Critical for effective insights.
      \item \textbf{Evaluation Challenges:} No direct measure of correctness.
      \item \textbf{Applications:} Widely used in image recognition, market segmentation, social networks, and recommendation systems.
    \end{itemize}
  \end{block}

  \begin{block}{Potential Questions to Discuss}
    \begin{enumerate}
      \item How do you select the number of clusters in K-Means clustering?
      \item Can you explain how an Autoencoder differs from standard neural networks?
      \item What are common pitfalls in applying unsupervised techniques?
      \item In what scenarios might unsupervised learning outperform supervised learning?
    \end{enumerate}
  \end{block}
  
\end{frame}


\end{document}